\documentclass{article}

    \usepackage[T1]{fontenc}
    \usepackage{inconsolata}
    \usepackage{amsmath}
    
    \usepackage{color}
    \usepackage{comment}
    \usepackage{graphicx}
    \usepackage{tikz}
    \usepackage{import}
    \usepackage{enumitem}% http://ctan.org/pkg/enumitem
    \usepackage{amssymb}
    \newlist{todolist}{itemize}{2}
    \setlist[todolist]{label=$\square$}
    
    \usepackage{multicol}
    
    \usepackage{hyperref}
    \hypersetup{
        linktoc=all
    }
    
    \graphicspath{ {./images/} }
    
    \definecolor{pblue}{rgb}{0.13,0.13,1}
    \definecolor{pgreen}{rgb}{0,0.5,0}
    \definecolor{pred}{rgb}{0.9,0,0}
    \definecolor{pgrey}{rgb}{0.46,0.45,0.48}
    
    \usepackage[draft]{todonotes}
    \usepackage[margin=0.5in]{geometry}
    
    \usepackage{etoolbox,refcount}
    \usepackage{multicol}
    \usepackage{pgfplots}
    \newcounter{countitems}
    \newcounter{nextitemizecount}
    \newcommand{\setupcountitems}{%
      \stepcounter{nextitemizecount}%
      \setcounter{countitems}{0}%
      \preto\item{\stepcounter{countitems}}%
    }
    \makeatletter
    \newcommand{\computecountitems}{%
      \edef\@currentlabel{\number\c@countitems}%
      \label{countitems@\number\numexpr\value{nextitemizecount}-1\relax}%
    }
    \newcommand{\nextitemizecount}{%
      \getrefnumber{countitems@\number\c@nextitemizecount}%
    }
    \newcommand{\previtemizecount}{%
      \getrefnumber{countitems@\number\numexpr\value{nextitemizecount}-1\relax}%
    }
    \makeatother
    \newenvironment{AutoMultiColItemize}{%
    \ifnumcomp{\nextitemizecount}{>}{3}{\begin{multicols}{2}}{}%
    \setupcountitems\begin{itemize}}%
    {\end{itemize}%
    \unskip\computecountitems\ifnumcomp{\previtemizecount}{>}{3}{\end{multicols}}{}}
    \usepackage{listings}
    \lstset{language=Java,
      showspaces=false,
      showtabs=false,
      breaklines=true,
      showstringspaces=false,
      breakatwhitespace=true,
      commentstyle=\color{pgreen},
      keywordstyle=\color{pblue},
      stringstyle=\color{pred},
      basicstyle=\ttfamily,
      moredelim=[il][\textcolor{pgrey}]{$ $},
      moredelim=[is][\textcolor{pgrey}]{\%\%}{\%\%}
    }
    
    
    \title{\textbf{Too Big to Fail (2011) - Reading Notes}}
    \date{\textbf{August 27$^\text{th}$, 2018}}
    \author{\textbf{Finn Frankis}}
    \begin{document}
    \maketitle
    \mbox{} \\ \textbf{Content Notes: }
    \begin{itemize}
        \item Begins with news reports describing sale of Bear Stearns to JPMorgan Chase
        \item Next major bank with potential to collapse: Lehman Brothers
        \begin{itemize}
            \item Henry Paulson, treasury secretary, continually struggles with how to keep banks afloat
            \begin{itemize}
                \item Otherwise, has no choice but to buy the banks out to prevent economic collapse (like definition of "Too big to fail")
                \item After assistance with buyout of Bear Stearns, many banks (including Fuld) became increasingly reliant on support from government
            \end{itemize}
            \item Offer from Buffett to purchase represents significantly lower (yet realistic) amount from months previous, leading CEO Dick Fuld to reject
            \item New President/COO negotiates deal with Korean investors, warning Fuld to stay behind
            \begin{itemize}
                \item Fuld arrives unexpectedly, warning investors that their deal is undervalued
                \item Immediately leave due to poor attitude of Fuld, ending deal
            \end{itemize}
        \end{itemize}
        \item Paulson makes clear announcement that the government will subsidize no more acquisitions of other banks, turning to other banks for Lehman acquisition
        \begin{itemize}
            \item Calls CEOs of major banks together, requesting that all assist in BoA purchase of Lehman 
            \begin{itemize}
                \item BoA had agreed to only assist in purchase with federal support 
            \end{itemize}
            \item Merill Lynch, firm in a position nearly as dangerous as Lehman, privately requests acquisition from BoA 
            \begin{itemize}
                \item Agreed to, ending possibility for Lehman deal
            \end{itemize}
            \item Lehman only acquirable by Barclays (who had been a potential buyer from the start)
            \begin{itemize}
                \item British banking regulator prevents involvement of Barclays
                \item Leads to complete bankruptcy of Lehman
            \end{itemize}
        \end{itemize}
        \item Collapse of Lehman has other dramatic effects on the stock market, leading to the decline of international insurance firm AIG
        \begin{itemize}
            \item GE unable to do business with freefalling stock market
            \item AIG's decline is responded to with \$85 billion loan 
        \end{itemize}
        \item Paulson's solution to continual drying up is purchase of toxic assets
        \begin{itemize}
            \item Receives support from Congress on second try
            \begin{itemize}
                \item Earned with emphasis on possibility of second Great Depression
            \end{itemize}
            \item Not supported heavily by president
        \end{itemize}
        \item Time to purchase toxic assets revealed to be too great, leading to final option: capital injections 
        \begin{itemize}
            \item Banks given large amounts of money
            \item Paulson's staff's suggests that banks' use of money be regulated
            \begin{itemize}
                \item Inability of banks to agree forces Paulson to give money without any strings attached
            \end{itemize}
        \end{itemize}
    \end{itemize}

    \mbox{} \\ \textbf{Thematic Notes:} \\~\\
    In the greater context of Chapter 32, this movie takes place near the end of the Bush presidency (Obama's reforms after taking over as president are not mentioned) and is the representation of the adjustable-rate mortgage, the housing bubble, and the steps taken by Henry Paulson to recover from these terrible events. These events impacted the presidential election in a significant way: they allowed Obama to take an edge over McCain due to McCain's association with Bush purely due to his party (and some shared policies). Obama's several policies to recover from the recession also put him in a relatively favorable light for the 2012 election.

    \end{document} 