%%%%%%%%%%%%%%%%%%%%%%%%%%%%%%%%%%%%%%%%%
% Dictionary
% LaTeX Template
% Version 1.1 (6/8/17)
%
% This template was downloaded from:
% http://www.LaTeXTemplates.com
%
% Original author:
% Vel (vel@latextemplates.com) inspired by a template by Marc Lavaud
%
% License:
% CC BY-NC-SA 3.0 (http://creativecommons.org/licenses/by-nc-sa/3.0/)
%
%%%%%%%%%%%%%%%%%%%%%%%%%%%%%%%%%%%%%%%%%

%----------------------------------------------------------------------------------------
%	PACKAGES AND OTHER DOCUMENT CONFIGURATIONS
%----------------------------------------------------------------------------------------

\documentclass[10pt,a4paper,twoside]{article} % 10pt font size, A4 paper and two-sided margins

    \usepackage[top=3.5cm,bottom=3.5cm,left=3.7cm,right=4.7cm,columnsep=30pt]{geometry} % Document margins and spacings
    
    \usepackage[utf8]{inputenc} % Required for inputting international characters
    \usepackage[T1]{fontenc} % Output font encoding for international characters
    
    \usepackage{palatino} % Use the Palatino font
    
    \usepackage{microtype} % Improves spacing
    
    \usepackage{multicol} % Required for splitting text into multiple columns
    
    \usepackage[bf,sf,center]{titlesec} % Required for modifying section titles - bold, sans-serif, centered
    
    \usepackage{fancyhdr} % Required for modifying headers and footers
    \fancyhead[L]{\textsf{\rightmark}} % Top left header
    \fancyhead[R]{\textsf{\leftmark}} % Top right header
    \renewcommand{\headrulewidth}{1.4pt} % Rule under the header
    \fancyfoot[C]{\textbf{\textsf{\thepage}}} % Bottom center footer
    \renewcommand{\footrulewidth}{1.4pt} % Rule under the footer
    \pagestyle{fancy} % Use the custom headers and footers throughout the document
    
    \newcommand{\entry}[2]{\textbf{#1}\markboth{#1}{#1}\  $\bullet$\ {#2}} % Defines the command to print each word on the page, \markboth{}{} prints the first word on the page in the top left header and the last word in the top right
    
    %----------------------------------------------------------------------------------------
    
    \begin{document}
    
    \section*{Chapter 13}
    
    \begin{multicols}{2}
    
    \entry{Manifest Destiny}{Coined by John L. O'Sullivan, the idea that America was destined by God to expand. Often supported with racial justification and publicized by cheap newspapers.}

    \entry{Stephen Austin}{One of the most successful early intermediaries to recruit American immigrants to Texas.}

    \entry{General Santa Anna}{Military dictator who took over Mexico, denounced the federal government, and encouraged Texans to lead a successful revolt.}

    \entry{Sam Houston}{Remained powerful in revolt against Santa Anna, successfully winning Texan independence and becoming the first President of Texas. Congress rejected his plea for annexation and Mexico still recognized Texas as part of their land.}

    \entry{Tejanos}{Mexican residents of Texas who had supported the independence cause but were often treated as subordinates under American rule.}

    \entry{Joint occupation}{Initial terms of settlement allowing British and American citizens equal access to Oregon; soon questioned as American settlers greatly outnumbered natives and Brits.}

    \entry{Oregon Trail}{Along with the Santa Fe trail, one of the main migratory routes taken by Americans travelling west. Known for challenging terrain but also kind natives who supported the travellers. Life on the trail generally stuck to traditional stereotypes.}
    
    \entry{James K. Polk}{Democrat nominated instead of famed Van Buren and Calhoun who won the election of 1844 and pushed strongly for American annexation of Mexican territories.}

    \entry{Annexation of Texas}{Pushed by Tyler after Polk's win but sparked great anger in the Mexican government.}

    \entry{Mexican-American War}{With the first battle initiated by Mexicans, Polk issued several armies under Colonel Kearny and Zachary Taylor. The war took longer than expected but saw an American win after the seizure of Mexico City.}
    
    \entry{Treaty of Guadalupe Hidalgo}{Negotiated by Nicholas Trist, it settled the Mexican-American War by giving the U.S. New Mexico, California, and Texas with the Rio Grande border as well as outstanding debts of absorbed Mexican citizens.}

    \entry{Wilmot Proviso}{Added to Polk's bill to purchase peace with Mexico, it sought to ban slavery in all Mexican Territory but failed in the Senate.}

    \entry{California Gold Rush}{After gold was discovered in the Sierra Nevadas in early 1848, San Francisco was depopulated and migrants from both throughout America and China flocked to California, giving California a population spike and labor source.} 

    \entry{Compromise of 1850}{Initiated by Henry Clay, it aimed to combine multiple bills into one large compromise. Though initially fought over by older nationalists like Calhoun, Webster, and Clay, a younger, less ideological generation took over and broke it up into several bills, ultimately reaching an agreement.}

    \entry{Franklin Pierce}{Winning the election of 1852, Pierce struggled with the issue of slavery and the Fugitive Slave Act and ultimately allowed the Compromise of 1850 to be overturned.}

    \entry{"Young America"}{Pushing to expand American democracy throughout the world to divert American attention from slavery, it was ultimately a failure.}

    \entry{Transcontinental railroad}{A topic of heated sectionalist debate over whether the first transcontinental railroad were to pass through the North or the South.}

    \entry{Gadsen Purchase}{Made by secretary of war Jefferson Davis, it opened up a southern transcontinental railroad by purchasing a strip of land comprising modern Arizona and New Mexico.}

    \entry{Kansas-Nebraska Controversy}{To remove the tribes in the way of a northern railroad, Stephen A. Douglas began to organize a new territory: Nebraska. However, the South demanded that it be divided into free Nebraska and slave Kansas, directly violating the Missouri Compromise.}

    \entry{Bleeding Kansas}{Used to describe the state of great turmoil which Kansas suffered shortly after the Kansas-Nebraska Act.}

    \entry{John Brown}{A Kansas abolitionist who used terrorist tactics by murdering pro-slavers to instill fear in Southerners.}

    \entry{Free-Soil Ideology}{A Northern justification against slavery, it stressed that slavery went against the fabric of American democracy. Generally pushed not for abolition but instead for limited expansion.}

    \entry{\textit{The Pro-Slavery Argument}}{An anthology detailing those who supported slavery, stressing it was beneficial to society and essential to the peaceful Southern way of life. Used biological inferiority and religion as justifications.}

    \entry{James Buchanan}{Winning the election of 1856, he was known for his age and indecisiveness, making few crucial decisions.}

    \entry{\textit{Dred Scott v. Sandford}}{Supreme Court case in which Taney overturned a Missouri court ruling by claiming that slaves were not citizens and had no right to sue for their rights.}

    \entry{Lecompton Constitution}{Formed by pro-slavery forces in Kansas, it was rejected by the majority anti-slavery population and delayed Kansas' admission as a state.}

    \entry{Lincoln-Douglas Debates}{Took place between Abraham Lincoln and Stephen A. Douglas in Illinos Senate elections, known for Lincoln's strong stance on the moral issues behind slavery.}
    
    \entry{Abraham Lincoln}{Elected in 1860 without the popular vote against a disunited Democrat force, he pushed for the expansion of slavery to be ceased and felt that blacks deserved all basic rights given to whites.}
    \end{multicols} 
    
    \newpage
    \section*{Chapter 14}
    
    \begin{multicols}{2}
    \entry{Fort Sumter}{As federal property which secessionist forces attempted to take over, it was the location where the first shots were fired by Confederate forces.}

    \entry{Crittenden Compromise}{Posed by Kentucky senator John Crittenden, it hoped to prevent war with additional compromise over the Fugitive Slave Act.}

    \entry{National Bank Acts of 1863-1864}{Implemented by Republican senators in the North, they allowed for a national bank system issuing uniform currency.}

    \entry{Bonds}{The greatest source of revenue for the North, they were essentially loans from the American people as well as corporations.}

    \entry{Draft Riots}{Riots by Irish workers who opposed the forced draft issued by the North due to shrinking volunteers for its high cost to resist as well as the war's potential to bring African-American workers into the North.}

    \entry{Peace Democrats}{Fearing a reduced influence for the Northwest, they opposed Lincoln's policies and were often imprisoned under his suspension of \textit{habeas corpeus}, or the right to trial by jury.}

    \entry{Election of 1864}{Saw Lincoln winning as part of the Union Party (Republicans + War Democrats) against Democrat George B. McClellan.}

    \entry{Confiscation Acts}{Allowed \textit{any} slave associated with Confederate forces to be freed and invited to join the Union army.}

    \entry{Emancipation Proclamation}{Representing Lincoln's ultimate sympathies with the Radical Republicans, it freed all slaves in the Confederacy and gave slaves a new hope.}

    \entry{Fifty-Fourth Infrantry}{One of the few all-black fighting regiments (but still led by a white commander). Usually, however, blacks were given non-fighting tasks, making their mortality rate in the war higher than for whites.}
    
    \entry{U.S. Sanitary Commission}{Led by Dorothea Dix, it empowered women to become nurses and take a critical role in the war effort despite frequent male opposition.}

    \entry{Woman's Loyal League}{Founded by feminists in 1863, it pushed for the abolition of slavery and for suffrage.}

    \entry{Jefferson Davis}{The first and only president of the Confederacy, he was a skilled administrator and a powerful leader but rarely made decisions on a national level.}

    \entry{Southern currency}{The South relied primarily on paper currency (varied by bank) issued in mass amounts, which caused significant inflation.}

    \entry{Conscription Act}{Mandated three years of service for all Southern white males. Its exclusion of wealthy men owning more than 20 slaves angered many.}

    \entry{Food draft}{Implemented by the increasingly centralized Confederate government, it allowed Confederate soldiers to eat crops from farms which they passed.}

    \entry{Robert E. Lee}{Made Davis' military advisor in 1862.}
    \end{multicols}

    \newpage
    \section*{Chapter 15}
    
    \begin{multicols}{2}
    \entry{Lost Cause}{A growing ideology in the post-Civil War South lamenting their former culture and lifestyle in the midst of great mourning and loss.}

    \entry{Freedom}{To Southern whites, a state without Northern intervention and often white supremacy. To Southern blacks, equal rights to whites and no white control.}
    
    \entry{Lincoln Governments}{Formed under Lincoln, these Southern governments only required 10\% of voters to pledge their allegiance to the Union.}

    \entry{Wade-Davis Bill}{Formed in response to the Lincoln governments, it required a majority to pledge their allegiance and limited the power former Confederate supporters.}

    \entry{John Wilkes Booth}{A member of a Southern Confederate conspiracy seeking to assassinate Union powers, he assassinated Abraham Lincoln in mid-1865 at the Ford Theatre.}

    \entry{Johnson's Restoration}{Mostly based around the Wade-Davis Bill, requiring the direct appeals of Southern officials before pardoning and the abolition of slavery.}

    \entry{Black Codes}{Enacted by several Southern states, they allowed whites to gain increasing dominance over the freedmen.}

    \entry{14th Amendment}{Passed in response to the Black Codes, it gave citizenship to all people born in the U.S. or naturalized, giving blacks equal rights.}

    \entry{Tenure of Office Act}{Implemented by Radical Republicans, it required Senate approval before the President could remove cabinet officials. Johnson's vioaltion of its terms led to his impeachment.}

    \entry{Scalawags}{White Southern Republicans who were often poor formers seeking to attain their personal economic goals.}

    \entry{Carpetbaggers}{White Northern Republicans who were generally well-educated and middle-class but often derided by Southern critics.}

    \entry{Freedmen's Bureau}{Created to enforce freedom and allow freedmen to reach equal footing as whites, its greatest strides were in educational institutions. It failed to provide significant amounts of land to blacks.}

    \entry{Crop-Lien System}{Providing credit at high rates to poor farmers who could not afford to purchase goods in local shops, it often trapped them in a never-ending cycle of debt and forced them to give up their land.}

    \entry{Ulysses S. Grant}{Renowned as a war hero, Grant won the popular vote in 1872 but his presidency was marked by several scandals involving the bribery of his key officials.}

    \entry{Greenbackers}{Seeking paper money to be printed in large amounts to redeem their war bonds, they started the National Greenback Party.}

    \entry{Specie Resumption Act}{Passed by Congress in 1875, it held that, by 1879, all circulated money had to be backed by physical reserves of gold and silver.}

    \entry{Seward's Folly}{Secretary of State William Seward's purchase of Alaska from Russia, it was denounced by critics as a terrible mistake.}

    \entry{Treaty of Washington}{Arranged by Grant's Secretary of State Hamilton Fish, it was Britain's formal expression of regret for assisting the Confederacy economically during the war.}

    \entry{Ku Klux Klan}{Formed in retaliation to growing black power in government, it employed terrorist tactics and violence to influence the black vote.}

    \entry{Enforcement Acts}{Seeking to ban voter discrimination based on race, they marked the first time the federal government was to take a direct role in prosecuting citizens.}

    \entry{Social Darwinism}{A racist argument used to justify African American inferiority, it stressed that they were simply social misfits.}

    \entry{Compromise of 1877}{After the election of 1876 saw no majority of electoral votes but 20 outstanding votes, Congress formed a special committee to decide the winner of the election. All 20 votes were given to Republican Hayes in exchange for severely reduced Northern intervention in the South.}

    \entry{Bourbons}{The aristocratic leaders of the post-Reconstruction South who were often merchants and industrialists rather than planters.}

    \entry{Old South}{A powerful image stressed by authors and journalists, it limited industry by arguing the beauty of former Southern society.}

    \entry{Convict-lease system}{Leasing criminals as cheap labor - all wages went to the government - it was known for poor working conditions and reducing opportunities for the free labor force.}

    \entry{Maggie Lena}{The first female black bank president.}

    \entry{Booker T. Washington}{A former slave, he pushed all blacks to actively improve their living conditions but felt that segregation needn't be challenged if blacks were allowed economic opportunity.}

    \entry{Grandfather laws}{Allowing those whose ancestors had voted in the election of 1860 to bypass literacy tests and poll taxes, they limited the black vote severely while keeping the white vote powerful.}

    \entry{Jim Crow Laws}{Expanding segregation throughout Southern life, they stimulated widespread violence against blacks, notably lynching.}

    \entry{Ida B. Wells}{A black journalist who spoke up against the horrors of lynching, she sought federal control over the matter.}
    \end{multicols}
    \end{document}