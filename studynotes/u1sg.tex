\documentclass{article}

    \usepackage[T1]{fontenc}
    \usepackage{inconsolata}
    \usepackage{amsmath}
    
    \usepackage{color}
    \usepackage{comment}
    \usepackage{graphicx}
    \usepackage{tikz}
    \usepackage{chemfig}
    \usepackage{import}
    \usepackage{enumitem}% http://ctan.org/pkg/enumitem
    
    \usepackage{hyperref}
    \hypersetup{
        linktoc=all
    }
    
    \graphicspath{ {./images/} }
    
    \definecolor{pblue}{rgb}{0.13,0.13,1}
    \definecolor{pgreen}{rgb}{0,0.5,0}
    \definecolor{pred}{rgb}{0.9,0,0}
    \definecolor{pgrey}{rgb}{0.46,0.45,0.48}
    
    \usepackage[draft]{todonotes}
    \usepackage[margin=0.5in]{geometry}
    
    \usepackage{etoolbox,refcount}
    \usepackage{multicol}
    \usepackage{pgfplots} 
    \newcounter{countitems}
    \newcounter{nextitemizecount}
    \newcommand{\setupcountitems}{%
      \stepcounter{nextitemizecount}%
      \setcounter{countitems}{0}%
      \preto\item{\stepcounter{countitems}}%
    }
    \makeatletter
    \newcommand{\computecountitems}{%
      \edef\@currentlabel{\number\c@countitems}%
      \label{countitems@\number\numexpr\value{nextitemizecount}-1\relax}%
    }
    \newcommand{\nextitemizecount}{%
      \getrefnumber{countitems@\number\c@nextitemizecount}%
    }
    \newcommand{\previtemizecount}{%
      \getrefnumber{countitems@\number\numexpr\value{nextitemizecount}-1\relax}%
    }
    \makeatother    
    \newenvironment{AutoMultiColItemize}{%
    \ifnumcomp{\nextitemizecount}{>}{3}{\begin{multicols}{2}}{}%
    \setupcountitems\begin{itemize}}%
    {\end{itemize}%
    \unskip\computecountitems\ifnumcomp{\previtemizecount}{>}{3}{\end{multicols}}{}}
    
      
      
    
    \title{%
      \LARGE \textbf{APUSH} \\ \Large \textbf{Periods 1-3: Study Guide}}
    \author{\textbf{Finn Frankis}}
    \date{\textbf{September 15th, 2018}}
    
    \begin{document}
    \maketitle
    \setcounter{tocdepth}{2}
    \tableofcontents
    \section{Period 1: 1492-1607}
    \subsection{Early America}
    \textbf{Key Concept 1.1: As native populations migrated and settled across North America over time, they developed distinct, complex societies through adaptation to/transformation of their diverse environments.}
    \subsubsection{Native American Comparison Chart}
    \subsection{Effects of Cross-Cultural Diffusion}
        
    \textbf{Key Concept 1.2: Contact among Europeans, Native Americans, and Africans led to the Columbian Exchange and significant social, cultural, and political changes on both hemispheres.}
    \subsubsection{European Motives for Exploration}
    \begin{itemize}
        \item Growing mercantilist movement driven by \textit{commercial} and \textit{national} ideals as a result of more successful nation states (military competition)
        \item Desire to spread Christianity and fulfill divine mission
        \begin{itemize}
            \item Christopher Columbus: deeply religious, believed that God had laid out prophecy for him to fulfill
        \end{itemize}
        \item Major new technologies and business revolutions
        \begin{itemize}
            \item Astrolabe
            \item Joint-stock companies often government-sponsored, based on stock shared among shareholders
        \end{itemize}
    \end{itemize}
    \subsection{Demographic Changes in the Spanish Empire}
    \begin{itemize}
        \item Widespread epidemics of diseases like smallpox, typhus rapidly pared down native population
        \item New crops and livestock introduced, including cattle, pigs, sheep, horse and sugar and bananas
        \item Europeans received critical new goods and techniques
        \begin{itemize}
            \item Learned how to better transform American soil
            \item Received new crops (\textbf{maize})
        \end{itemize}
        \item \textit{Encomienda} system subjugated natives to Spanish dominance
        \begin{itemize}
            \item License-based system for natives to work for settlers in exchange for conversion to Christianity
        \end{itemize}
    \end{itemize}
    \subsection{Early European-Native Conflicts}

    \section{Period 2: 1607-1754}

    \section{Period 3: 1754-1800}
    
    \end{document} 