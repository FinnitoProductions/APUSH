\documentclass[a4paper]{article}
    \usepackage[T1]{fontenc}
    \usepackage{tcolorbox}
    \usepackage{amsmath}
    \tcbuselibrary{skins}
    
    \title{
    \vspace{-3em}
    \begin{tcolorbox}
    \Huge\sffamily \begin{center} AP US History  \\
    \LARGE Chapter 6 - The Constitution and the New Republic \\
    \Large Finn Frankis \end{center} 
    \end{tcolorbox}
    \vspace{-3em}
    }
    \date{}
    \author{}
    
    \usepackage{background}
    \SetBgScale{1}
    \SetBgAngle{0}
    \SetBgColor{red}
    \SetBgContents{\rule[0em]{4pt}{\textheight}}
    \SetBgHshift{-2.3cm}
    \SetBgVshift{0cm}
    \usepackage[margin=2cm]{geometry} 
    
    \makeatletter
    \def\cornell{\@ifnextchar[{\@with}{\@without}}
    \def\@with[#1]#2#3{
    \begin{tcolorbox}[enhanced,colback=gray,colframe=black,fonttitle=\large\bfseries\sffamily,sidebyside=true, nobeforeafter,before=\vfil,after=\vfil,colupper=blue,sidebyside align=top, lefthand width=.3\textwidth,
    opacityframe=0,opacityback=.3,opacitybacktitle=1, opacitytext=1,
    segmentation style={black!55,solid,opacity=0,line width=3pt},
    title=#1
    ]
    \begin{tcolorbox}[colback=red!05,colframe=red!25,sidebyside align=top,
    width=\textwidth,nobeforeafter]#2\end{tcolorbox}%
    \tcblower
    \sffamily
    \begin{tcolorbox}[colback=blue!05,colframe=blue!10,width=\textwidth,nobeforeafter]
    #3
    \end{tcolorbox}
    \end{tcolorbox}
    }
    \def\@without#1#2{
    \begin{tcolorbox}[enhanced,colback=white!15,colframe=white,fonttitle=\bfseries,sidebyside=true, nobeforeafter,before=\vfil,after=\vfil,colupper=blue,sidebyside align=top, lefthand width=.3\textwidth,
    opacityframe=0,opacityback=0,opacitybacktitle=0, opacitytext=1,
    segmentation style={black!55,solid,opacity=0,line width=3pt}
    ]
    
    \begin{tcolorbox}[colback=red!05,colframe=red!25,sidebyside align=top,
    width=\textwidth,nobeforeafter]#1\end{tcolorbox}%
    \tcblower
    \sffamily
    \begin{tcolorbox}[colback=blue!05,colframe=blue!10,width=\textwidth,nobeforeafter]
    #2
    \end{tcolorbox}
    \end{tcolorbox}
    }
    \makeatother

    \parindent=0pt
    
    \begin{document}
    \maketitle
    \SetBgContents{\rule[0em]{4pt}{\textheight}}
    \cornell[Key Concepts]{What are this chapter's key concepts?}{
        \begin{itemize}
            \item \textbf{3.2.II.C} - Delegates from each state negotiated a constitution allowing powers to be separated between three branches of central power
            \item \textbf{3.2.II.D} - Constitutional Convention compromised over slave state representation in Congress and federal representation; slave trade was prohibited after 1808
            \item \textbf{3.2.II.E} - Debate over constitutional ratification saw Anti-Federalists against Federalists; Federalists (including Alexander Hamilton) ensured ratification w/ Bill of Rights for individuals, restricting federal power
            \item \textbf{3.2.III.A} - During presidencies, of George Washington, John Adams, new precedents were set to put Constitution into practice
            \item \textbf{3.2.III.B} - 1790s political leaders took varied positions on relationship between national government/states, liberty/order, foreign policy; political parties emerged with Federalists (Alexander Hamilton) and Democratic-Republicans (Thomas Jefferson, James Madison)
            \item \textbf{3.3.I.A} - Various native groups repeatedly reevaluated alliances with Europeans and U.S.; sought to limit migration of whites and control tribal lands; British alliances -> U.S./British tension
            \item \textbf{3.3.II.A} - U.S. government formed diplomatic goals to handle British/Spanish presence in N.A. with settlers seeking free migration
            \item \textbf{3.3.II.B} - French-British war as a result of the French Revolution created challenges regarding free trade
            \item \textbf{3.3.II.C} - George Washington's Farewell Address encouraged national unity w/ cautioning against political factions, warning about danger of foreign relationships
        \end{itemize}
    }
    \cornell[Framing a New Government]{What were the key political decisions made in creating the new American government?}{\textbf{The new American government began with the Constitution, a long-awaited change to the government which replaced the Confederation with a more organized system answering questions of sovereignty, power limitation, and the true control of the federal government; however, this was limited in its representation of whites only. In the process of ratifying the Constitution, the Federalists emerged in opposition to the "Anti-Federalists," those who rejected the Constitution. The Constitution was finally ratified, however, with the first Congress making key amendments, including the Bill of Rights, the fundamental legal system, and executive departments.}}
    \cornell{What was the pretext to the framing of the new government?}{\begin{itemize}
        \item Confederation Congress -> extremely unpopular; members departed Philadelphia due to anger
        \item Temporarily in Princeton, NJ, then Annapolis, then NY (1785)
        \item Struggled greatly to make key decisions
        \begin{itemize}
            \item Extremely challeging to ratify treaty with Britain to end Revolutionary War
            \item Eighteen members voted on Northwest Ordinance
        \end{itemize}
        \item Significant public debate emerged over future of Confederation
    \end{itemize}
    \textbf{The Confederation had become extremely unpopular, having to change locations and greatly unable to make major decisions. Public debate soon began to emerge over its future.}}
    \cornell{Who were the main advocates for federal centralization?}{\begin{itemize}
        \item Confederation had supported many people, believing they had fought Revolutionary War to resist tyrannical authority; hoped to keep political power centered in states due to easy control
        \item Wealthiest/powerful groups sought centralized power able to deal with major problems
        \begin{itemize}
            \item Military men (many members of Society of the Cincinnati, based on those who fought in the war) unhappy about Congress' inability to fund pensions; some envisioned military dictatorship
            \begin{itemize}
                \item Direct rebellion began to brew; Washington put down
            \end{itemize}
            \item Manufacturers of cities/towns hoped to create consistent national duty, replacing state policies with one single tax
            \item Land speculators hoped to remove natives 
            \item Debtees sought to ban paper money (lowered value of debt)
            \item Investors in Confederation wanted enhanced value
            \item Property owners feared mobs (especially after Shays' Rebellion), seeking protection
            \begin{itemize}
                \item Root of fear was defense of individual rights, goal for safety/security
                \item Debate over liberty vs order became increasingly prominent
            \end{itemize}
        \end{itemize}
        \item Issues grew to extent of inevitability: even main defenders conceded need for strengthening
        \item Most resourceful reformer: Alexander Hamilton, NY lawyer, military aide to Washington
        \begin{itemize}
            \item Unhappy with weak Articles of Confederation
            \item Called for national convention, finding support in James Madison (VA), who convinced for interstate conference to discuss economic issues
            \begin{itemize}
                \item Five states sent delegates to attend in Annapolis, MD, but approved proposal by Hamilton (NY) asking for Philadelphia meeting among all delegates to ratify constitution
            \end{itemize}
            \item Centralizers believed that Washington was key ally in cause
            \begin{itemize}
                \item Jefferson (American minister in Paris) felt Shays' rebellion was natural; Washington deeply disturbed, giving support for Constitutional Congress
            \end{itemize}
        \end{itemize}
    \end{itemize}
        \textbf{Although the Confederation had supported many, the most powerful groups began to seek a greater centralized power, including Revolutionary veterans, manufacturers, speculators, and property owners. Hamilton led the reform and called for a national convention in Philadelphia, which eventually won Washington's support.}}
        \cornell{What was the initial division at Congress in Philadelphia and what were the critical solutions?}{\begin{itemize}
            \item 55 men attended (all states except for Rhode Island; May-September 1787) - soon known as "Founding Fathers"
            \begin{itemize}
                \item Most quite young but well-educated, representing landowners but fearing power too concentrated
            \end{itemize}
            \item Washington unanimously selected to lead; all business closed to public/press 
            \begin{itemize}
                \item Each state given single vote; major decisions required majority 
                \item VA (as most populous state under Madison) sent best delegation
            \end{itemize}
            \item Edmund Randolph (VA) began with proposal that national government needed supreme Legislative, Executive, and Judiciary branches
            \begin{itemize}
                \item Very different to Confederation (no executive branch)
                \item Approved due to great desire for reform
                \item Madison's plan (VA plan) sought legislature with two houses (lower -> state delegates proportional to population, upper -> elected by lower house without restrictions, meaning some states would not have members)
                \begin{itemize}
                    \item Smaller states opposed greatly, stating that Congress had no greater authority than to revise Articles of Confederation
                \end{itemize}
            \end{itemize}
            \item William Paterson created NJ Plan, preserving one-house legislature with equal representation; Congress would receive greater taxation power 
            \begin{itemize}
                \item Proposal was tabled
            \end{itemize}
            \item VA Plan remained topic of discussion -> smaller states realized need for concessions
            \begin{itemize}
                \item Smaller states permitted members of upper house to be elected by states rather than lower house
            \end{itemize}
        \end{itemize}
        \textbf{The Congress, 55 educated (mostly young) men led by Washington, was greatly divided over key issues. Randolph of Virginia created a proposal with three governmental branches and a legislature with upper/lower houses where population influenced representation. The smaller states, after their plan for equal representation failed, began to compromise, including on how members of the upper house were elected.}}
        \cornell{What debates remained unresolved even after many concessions were made by both sides?}{\begin{itemize}
            \item Debate remained as to how many members each state would receive in the upper house, whether slaves would be considered in population numbers for representation 
        \begin{itemize}
            \item Slave-owning states hoped to have slaves counted for representation (allowing them to possibly have more power) while considering them property if population influenced taxes
            \item States without slavery argued \textit{opposite}
        \end{itemize}
    \end{itemize}
    \textbf{Large questions remained: would members of the upper house be represented based on population and how would slaves be counted?}}
        \cornell{What were the major compromises which had to be made?}{\begin{itemize}
            \item By end of June, tensions began to grow, with risk of collapsing; Franklin remained a calm voice amidst chaos
            \begin{itemize}
                \item Franklin warned of importance of meeting; delegates refused to give up as a result
            \end{itemize}
            \item July 2nd, 1787: agreed to create "grand committee" with one delegate per state, finally coming to "Grand Compromise"
            \begin{itemize}
                \item Lower house based on population (slave counted for 3/5 in representation and taxation due to assumptions about productivity)
                \item Upper house would allow for equal representation with two members each
                \item Slavery compromise emerged as southern states feared power to regulate trade would damage agricultural economy
                \begin{itemize}
                    \item Congress prevented from taxing exports, imposing duty > \$10 on imported slaves, unable to stop slave trade
                    \item Very challenging concession 
                \end{itemize}
            \end{itemize}
            \item Many critical disagreements were ignored -> questions sparked up in following years
            \begin{itemize}
                \item No definition for citizenship
                \item List of individual rights never produced; Madison believed rights would be reserved to people of authority while others feared abuse
            \end{itemize}
        \end{itemize}
        \textbf{The "Grand Compromise" at the onset of July agreed on a population-based lower house and a upper house with two representatives each. However, a major blow to the power of the national government emerged with a compromise over slavery preventing taxation on exports, large slave duties, or the ability to stop the slave trade. Because the compromise ignored many key questions, debate continued in the following years.}}
        \cornell{What were the tenets of the Constitution of 1787?}{
            James Madison was the key contributor to the Constitution, resolving questions of sovereignty and power limitation.
            \begin{itemize}
                \item Sovereignty had been struggle w/ GB; Madison resolved that power was based in people regardless of governmental level: no body was truly sovereign apart from collective one of people
                \begin{itemize}
                    \item Distribution of power between state/nation saw broad powers, including power to tax, control currency, pass necessary laws; no state could defy
                    \item State still recognized as body with some major powers
                \end{itemize}
                \item Problem of concentrated authority solved with fear of tyrannical government
                \begin{itemize}
                    \item Initial belief that public had to be confined to level of the people in small area -> states had most of power
                    \item Madison argued that numerous factions of larger power would allow for checks and balances without tyrannical rule
                    \begin{itemize}
                        \item Separated powers among legislative, executive, judicial branches
                        \item Governmental powers constantly competing, with two chambers of Congress (Senate/House of Representatives) with members elected differently and checking each other
                        \item President given power to veto Congress
                        \item Federal courts given ultimate protection with president-appointed judges approved by Senate given role for life
                    \end{itemize}
                    \item Federal structure aimed to protect from despotism in both directions, with elimination of "mob" (democratic excess, like in Shays' Rebellion) and no single tyrannical ruler
                    \begin{itemize}
                        \item Only House of Representatives elected directly by the people; remainder given various levels of insulation 
                    \end{itemize}
                \end{itemize}
                \item Constitution signed in September 1787 by 39 delegates
        \end{itemize}
        \textbf{James Madison corrected the key questions of sovereignty and power with the Constitution: for sovereignty, he recognized only the people as a sovereign group and all others as having their power derived from the people; for authority, he argued that the national government's multiple factions would provide checks to prevent tyrannical rule while insulating majority of houses from direct voting to prevent a democratic excess.}}
        \cornell{What were the major limitations of the Constitution?}{
            The Constitution was designed mostly for white people, ignoring the rights/needs of natives and African Americans.{\begin{itemize}
            \item Natives had some treaties which promised guaranteed land, but they were rarely followed (constantly driven west)
            \begin{itemize}
                \item Many leaders of the U.S. felt that natives could be "civilized" (including Jefferson), but popular support could not be earned as most remained drawn to traditional culture
            \end{itemize}
            \item African Americans even more removed, with essentially no rights given 
            \begin{itemize}
                \item French-Canadian writer Crèvecoeur settling in the U.S. post-revolution wrote \textit{Letters from an American Farmer}, discussing common citizenry
                \item Ideas reflected in Naturalization Act of 1790, legalizing immigrants with possibility for citizenship
                \begin{itemize}
                    \item All African-Americans barred
                \end{itemize}
                \item Jefferson had no aspiration to give complete rights due to failure for Native Americans
                \begin{itemize}
                    \item Hesitantly defended slavery, hoping not to ignore entire race in fundamental rights
                    \item Unable to come to terms with possibility of blacks coming to same level of power/intelligence as whites, despite close relationship with one of his slaves
                    \item Citizenship only legalized a century later after Civil War
                \end{itemize}
            \end{itemize}
        \end{itemize}
        \textbf{The Constitution, tailored to whites, gave little hope of representation for either natives or African Americans, despite legalizing white immigrants and Jefferson's attempts for Native Americans.}}
    }
    \cornell{What were the first signs of partisan divisions within the United States?}{\begin{itemize}
        \item Philadelphia delegates had surpassed instructions from Congress/states, creating completely different form of government
        \begin{itemize}
            \item Feared Constitution would never be ratified (required unanimous approval), prompting change with 9/13 states required
            \item Called for state convention (not legislatures) to consider document, voting "yes" or "no"
            \begin{itemize}
                \item Confederation Congress passively accepted work, submitting to states; all but Rhode Island began to consider
            \end{itemize}
        \end{itemize}
        \item Great national debate had emerged over Constitution, with frequent fights between factions (death in Albany)
        \begin{itemize}
            \item Supporters were far more organized, had support of two key men of Washington/Franklin
            \begin{itemize}
                \item Created label of "Federalists"
                \item Supported by Hamilton, Madison, John Jay, who wrote essays justifying Constitution, becoming known as \textit{The Federalist Papers}
            \end{itemize}
            \item Federalists deemed Critics "Antifederalists," who also had intelligent arguments
            \begin{itemize}
                \item Presented themselves as Revolutionary defenders against tyrannical government and aristocrachy 
                \item Lack of bill of rights led to fundamental mistrust, with no guarantees that government would protect liberties of people
            \end{itemize}
            \item Debate fundamentally between two fears: for Federalists, anarchy, chaos, and unchecked power of masses; for Antifederalists, concentrated power (\textit{not} anarchy)
        \end{itemize}
        \item Ratification quickly proceeded through winter
        \begin{itemize}
            \item Delaware first, followed by NJ/GA, ratifying unanimously
            \item PA and MA saw greater challenge but loss of Antifederalists in final vote; NH, ninth state to do so, did in June 1788
            \item Divided VA and NY meant that constitution could not truly go into effect; finally narrowly consented by end of June (NY feared commercial effects of being left out), demanding bill of rights
            \begin{itemize}
                \item NC adjourned without action (waiting for amendments) while RI did not call convention to consider
            \end{itemize}
        \end{itemize}
    \end{itemize}
    \textbf{The Philadelphia delegates, fearing the Constitution would not be ratified, implemented a unique system for ratification requiring only 9 states to approve. Partisan divisions began, with tensions between Federalists and Antifederalists. However, ratification eventually proceeded with multiple key conditions.}}
    \cornell{What were the results of the first key Congress meetings?}{\begin{itemize}
        \item Most new members had supported ratification
        \item Washington became first president with John Adams as VP
        \begin{itemize}
            \item Inaugaurated w/ NYC as capital in 1789
        \end{itemize}
        \item First Congress tasked with filling gaps in Constitution, most notably Bill of Rights
        \begin{itemize}
            \item Even Madison had agreed on significance in eyes of opponents
            \item Agreed on 12 amendments; 10 had been ratified by end of 1791, becoming parts of modern Bill of Rights (freedom of religion, speech, press; trial by jury; no arbitrary arrest)
        \end{itemize}
        \item Constitution had created Supreme Court; Congress left to decide number of judges
            \begin{itemize}
                \item Created six members in Supreme Court (chief justice, five associate justices); thirteen district courts with one judge each; three circuit courts of appeal with one district judges w/ two Supreme Court ones
                \item Supreme Court given final say in state laws
            \end{itemize}
        \item Exact executive departments left unclear
        \begin{itemize}
            \item First Congress created state, treasury, war; offices of attorney general, postmaster general
            \item Hamilton appointed as treasury secretary, Henry Knox as war secretary, Edmund Randolph as attorney general, 
            Jefferson as secretary of state 
        \end{itemize}
    \end{itemize}
    \textbf{Congress was tasked with filling gaps in the Constitution, including the Bill of Rights with ten key amendments, the Supreme Court with six members and two other courts beneath them, and executive departments of state, treasury, war, attorney general, and postmaster general.}}
    \end{document}