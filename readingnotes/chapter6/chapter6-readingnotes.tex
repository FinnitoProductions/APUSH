\documentclass[a4paper]{article}
    \usepackage[T1]{fontenc}
    \usepackage{tcolorbox}
    \usepackage{amsmath}
    \tcbuselibrary{skins}
    
    \title{
    \vspace{-3em}
    \begin{tcolorbox}
    \Huge\sffamily \begin{center} AP US History  \\
    \LARGE Chapter 6 - The Constitution and the New Republic \\
    \Large Finn Frankis \end{center} 
    \end{tcolorbox}
    \vspace{-3em}
    }
    \date{}
    \author{}
    
    \usepackage{background}
    \SetBgScale{1}
    \SetBgAngle{0}
    \SetBgColor{red}
    \SetBgContents{\rule[0em]{4pt}{\textheight}}
    \SetBgHshift{-2.3cm}
    \SetBgVshift{0cm}
    \usepackage[margin=2cm]{geometry} 
    
    \makeatletter
    \def\cornell{\@ifnextchar[{\@with}{\@without}}
    \def\@with[#1]#2#3{
    \begin{tcolorbox}[enhanced,colback=gray,colframe=black,fonttitle=\large\bfseries\sffamily,sidebyside=true, nobeforeafter,before=\vfil,after=\vfil,colupper=blue,sidebyside align=top, lefthand width=.3\textwidth,
    opacityframe=0,opacityback=.3,opacitybacktitle=1, opacitytext=1,
    segmentation style={black!55,solid,opacity=0,line width=3pt},
    title=#1
    ]
    \begin{tcolorbox}[colback=red!05,colframe=red!25,sidebyside align=top,
    width=\textwidth,nobeforeafter]#2\end{tcolorbox}%
    \tcblower
    \sffamily
    \begin{tcolorbox}[colback=blue!05,colframe=blue!10,width=\textwidth,nobeforeafter]
    #3
    \end{tcolorbox}
    \end{tcolorbox}
    }
    \def\@without#1#2{
    \begin{tcolorbox}[enhanced,colback=white!15,colframe=white,fonttitle=\bfseries,sidebyside=true, nobeforeafter,before=\vfil,after=\vfil,colupper=blue,sidebyside align=top, lefthand width=.3\textwidth,
    opacityframe=0,opacityback=0,opacitybacktitle=0, opacitytext=1,
    segmentation style={black!55,solid,opacity=0,line width=3pt}
    ]
    
    \begin{tcolorbox}[colback=red!05,colframe=red!25,sidebyside align=top,
    width=\textwidth,nobeforeafter]#1\end{tcolorbox}%
    \tcblower
    \sffamily
    \begin{tcolorbox}[colback=blue!05,colframe=blue!10,width=\textwidth,nobeforeafter]
    #2
    \end{tcolorbox}
    \end{tcolorbox}
    }
    \makeatother

    \parindent=0pt
    
    \begin{document}
    \maketitle
    \SetBgContents{\rule[0em]{4pt}{\textheight}}
    \cornell[Key Concepts]{What are this chapter's key concepts?}{
        \begin{itemize}
            \item \textbf{3.2.II.C} - Delegates from each state negotiated a constitution allowing powers to be separated between three branches of central power
            \item \textbf{3.2.II.D} - Constitutional Convention compromised over slave state representation in Congress and federal representation; slave trade was prohibited after 1808
            \item \textbf{3.2.II.E} - Debate over constitutional ratification saw Anti-Federalists against Federalists; Federalists (including Alexander Hamilton) ensured ratification w/ Bill of Rights for individuals, restricting federal power
            \item \textbf{3.2.III.A} - During presidencies, of George Washington, John Adams, new precedents were set to put Constitution into practice
            \item \textbf{3.2.III.B} - 1790s political leaders took varied positions on relationship between national government/states, liberty/order, foreign policy; political parties emerged with Federalists (Alexander Hamilton) and Democratic-Republicans (Thomas Jefferson, James Madison)
            \item \textbf{3.3.I.A} - Various native groups repeatedly reevaluated alliances with Europeans and U.S.; sought to limit migration of whites and control tribal lands; British alliances -> U.S./British tension
            \item \textbf{3.3.II.A} - U.S. government formed diplomatic goals to handle British/Spanish presence in N.A. with settlers seeking free migration
            \item \textbf{3.3.II.B} - French-British war as a result of the French Revolution created challenges regarding free trade
            \item \textbf{3.3.II.C} - George Washington's Farewell Address encouraged national unity w/ cautioning against political factions, warning about danger of foreign relationships
        \end{itemize}
    }
    \cornell[Framing a New Government]{What were the key political decisions made in creating the new American government?}{\textbf{The new American government began with the Constitution, a long-awaited change to the government which replaced the Confederation with a more organized system answering questions of sovereignty, power limitation, and the true control of the federal government; however, this was limited in its representation of whites only. In the process of ratifying the Constitution, the Federalists emerged in opposition to the "Anti-Federalists," those who rejected the Constitution. The Constitution was finally ratified, however, with the first Congress making key amendments, including the Bill of Rights, the fundamental legal system, and executive departments.}}
    \cornell{What was the pretext to the framing of the new government?}{\begin{itemize}
        \item Confederation Congress -> extremely unpopular; members departed Philadelphia due to anger
        \item Temporarily in Princeton, NJ, then Annapolis, then NY (1785)
        \item Struggled greatly to make key decisions
        \begin{itemize}
            \item Extremely challeging to ratify treaty with Britain to end Revolutionary War
            \item Eighteen members voted on Northwest Ordinance
        \end{itemize}
        \item Significant public debate emerged over future of Confederation
    \end{itemize}
    \textbf{The Confederation had become extremely unpopular, having to change locations and greatly unable to make major decisions. Public debate soon began to emerge over its future.}}
    \cornell{Who were the main advocates for federal centralization?}{\begin{itemize}
        \item Confederation had supported many people, believing they had fought Revolutionary War to resist tyrannical authority; hoped to keep political power centered in states due to easy control
        \item Wealthiest/powerful groups sought centralized power able to deal with major problems
        \begin{itemize}
            \item Military men (many members of Society of the Cincinnati, based on those who fought in the war) unhappy about Congress' inability to fund pensions; some envisioned military dictatorship
            \begin{itemize}
                \item Direct rebellion began to brew; Washington put down
            \end{itemize}
            \item Manufacturers of cities/towns hoped to create consistent national duty, replacing state policies with one single tax
            \item Land speculators hoped to remove natives 
            \item Debtees sought to ban paper money (lowered value of debt)
            \item Investors in Confederation wanted enhanced value
            \item Property owners feared mobs (especially after Shays' Rebellion), seeking protection
            \begin{itemize}
                \item Root of fear was defense of individual rights, goal for safety/security
                \item Debate over liberty vs order became increasingly prominent
            \end{itemize}
        \end{itemize}
        \item Issues grew to extent of inevitability: even main defenders conceded need for strengthening
        \item Most resourceful reformer: Alexander Hamilton, NY lawyer, military aide to Washington
        \begin{itemize}
            \item Unhappy with weak Articles of Confederation
            \item Called for national convention, finding support in James Madison (VA), who convinced for interstate conference to discuss economic issues
            \begin{itemize}
                \item Five states sent delegates to attend in Annapolis, MD, but approved proposal by Hamilton (NY) asking for Philadelphia meeting among all delegates to ratify constitution
            \end{itemize}
            \item Centralizers believed that Washington was key ally in cause
            \begin{itemize}
                \item Jefferson (American minister in Paris) felt Shays' rebellion was natural; Washington deeply disturbed, giving support for Constitutional Congress
            \end{itemize}
        \end{itemize}
    \end{itemize}
        \textbf{Although the Confederation had supported many, the most powerful groups began to seek a greater centralized power, including Revolutionary veterans, manufacturers, speculators, and property owners. Hamilton led the reform and called for a national convention in Philadelphia, which eventually won Washington's support.}}
        \cornell{What was the initial division at Congress in Philadelphia and what were the critical solutions?}{\begin{itemize}
            \item 55 men attended (all states except for Rhode Island; May-September 1787) - soon known as "Founding Fathers"
            \begin{itemize}
                \item Most quite young but well-educated, representing landowners but fearing power too concentrated
            \end{itemize}
            \item Washington unanimously selected to lead; all business closed to public/press 
            \begin{itemize}
                \item Each state given single vote; major decisions required majority 
                \item VA (as most populous state under Madison) sent best delegation
            \end{itemize}
            \item Edmund Randolph (VA) began with proposal that national government needed supreme Legislative, Executive, and Judiciary branches
            \begin{itemize}
                \item Very different to Confederation (no executive branch)
                \item Approved due to great desire for reform
                \item Madison's plan (VA plan) sought legislature with two houses (lower -> state delegates proportional to population, upper -> elected by lower house without restrictions, meaning some states would not have members)
                \begin{itemize}
                    \item Smaller states opposed greatly, stating that Congress had no greater authority than to revise Articles of Confederation
                \end{itemize}
            \end{itemize}
            \item William Paterson created NJ Plan, preserving one-house legislature with equal representation; Congress would receive greater taxation power 
            \begin{itemize}
                \item Proposal was tabled
            \end{itemize}
            \item VA Plan remained topic of discussion -> larger states realized need for concessions to smaller states
            \begin{itemize}
                \item Permitted members of upper house to be elected by states rather than lower house
            \end{itemize}
        \end{itemize}
        \textbf{The Congress, 55 educated (mostly young) men led by Washington, was greatly divided over key issues. Randolph of Virginia created a proposal with three governmental branches and a legislature with upper/lower houses where population influenced representation. The smaller states, after their plan for equal representation failed, began to compromise, including on how members of the upper house were elected.}}
        \cornell{What debates remained unresolved even after many concessions were made by both sides?}{\begin{itemize}
            \item Debate remained as to how many members each state would receive in the upper house, whether slaves would be considered in population numbers for representation 
        \begin{itemize}
            \item Slave-owning states hoped to have slaves counted for representation (allowing them to possibly have more power) while considering them property if population influenced taxes
            \item States without slavery argued \textit{opposite}
        \end{itemize}
    \end{itemize}
    \textbf{Large questions remained: would members of the upper house be represented based on population and how would slaves be counted?}}
        \cornell{What were the major compromises which had to be made?}{\begin{itemize}
            \item By end of June, tensions began to grow, with risk of collapsing; Franklin remained a calm voice amidst chaos
            \begin{itemize}
                \item Franklin warned of importance of meeting; delegates refused to give up as a result
            \end{itemize}
            \item July 2nd, 1787: agreed to create "grand committee" with one delegate per state, finally coming to "Grand Compromise"
            \begin{itemize}
                \item Lower house based on population (slave counted for 3/5 in representation and taxation due to assumptions about productivity)
                \item Upper house would allow for equal representation with two members each
                \item Slavery compromise emerged as southern states feared power to regulate trade would damage agricultural economy
                \begin{itemize}
                    \item Congress prevented from taxing exports, imposing duty > \$10 on imported slaves, unable to stop slave trade
                    \item Very challenging concession 
                \end{itemize}
            \end{itemize}
            \item Many critical disagreements were ignored -> questions sparked up in following years
            \begin{itemize}
                \item No definition for citizenship
                \item List of individual rights never produced; Madison believed rights would be reserved to people of authority while others feared abuse
            \end{itemize}
        \end{itemize}
        \textbf{The "Grand Compromise" at the onset of July agreed on a population-based lower house and a upper house with two representatives each. However, a major blow to the power of the national government emerged with a compromise over slavery preventing taxation on exports, large slave duties, or the ability to stop the slave trade. Because the compromise ignored many key questions, debate continued in the following years.}}
        \cornell{What were the tenets of the Constitution of 1787?}{
            James Madison was the key contributor to the Constitution, resolving questions of sovereignty and power limitation.
            \begin{itemize}
                \item Sovereignty had been struggle w/ GB; Madison resolved that power was based in people regardless of governmental level: no body was truly sovereign apart from collective one of people
                \begin{itemize}
                    \item Distribution of power between state/nation saw broad powers, including power to tax, control currency, pass necessary laws; no state could defy
                    \item State still recognized as body with some major powers
                \end{itemize}
                \item Problem of concentrated authority solved with fear of tyrannical government
                \begin{itemize}
                    \item Initial belief that public had to be confined to level of the people in small area -> states had most of power
                    \item Madison argued that numerous factions of larger power would allow for checks and balances without tyrannical rule
                    \begin{itemize}
                        \item Separated powers among legislative, executive, judicial branches
                        \item Governmental powers constantly competing, with two chambers of Congress (Senate/House of Representatives) with members elected differently and checking each other
                        \item President given power to veto Congress
                        \item Federal courts given ultimate protection with president-appointed judges approved by Senate given role for life
                    \end{itemize}
                    \item Federal structure aimed to protect from despotism in both directions, with elimination of "mob" (democratic excess, like in Shays' Rebellion) and no single tyrannical ruler
                    \begin{itemize}
                        \item Only House of Representatives elected directly by the people; remainder given various levels of insulation 
                    \end{itemize}
                \end{itemize}
                \item Constitution signed in September 1787 by 39 delegates
        \end{itemize}
        \textbf{James Madison corrected the key questions of sovereignty and power with the Constitution: for sovereignty, he recognized only the people as a sovereign group and all others as having their power derived from the people; for authority, he argued that the national government's multiple factions would provide checks to prevent tyrannical rule while insulating majority of houses from direct voting to prevent a democratic excess.}}
        \cornell{What were the major limitations of the Constitution?}{
            The Constitution was designed mostly for white people, ignoring the rights/needs of natives and African Americans.{\begin{itemize}
            \item Natives had some treaties which promised guaranteed land, but they were rarely followed (constantly driven west)
            \begin{itemize}
                \item Many leaders of the U.S. felt that natives could be "civilized" (including Jefferson), but popular support could not be earned as most remained drawn to traditional culture
            \end{itemize}
            \item African Americans even more removed, with essentially no rights given 
            \begin{itemize}
                \item French-Canadian writer Crèvecoeur settling in the U.S. post-revolution wrote \textit{Letters from an American Farmer}, discussing common citizenry
                \item Ideas reflected in Naturalization Act of 1790, legalizing immigrants with possibility for citizenship
                \begin{itemize}
                    \item All African-Americans barred
                \end{itemize}
                \item Jefferson had no aspiration to give complete rights due to failure for Native Americans
                \begin{itemize}
                    \item Hesitantly defended slavery, hoping not to ignore entire race in fundamental rights
                    \item Unable to come to terms with possibility of blacks coming to same level of power/intelligence as whites, despite close relationship with one of his slaves
                    \item Citizenship only legalized a century later after Civil War
                \end{itemize}
            \end{itemize}
        \end{itemize}
        \textbf{The Constitution, tailored to whites, gave little hope of representation for either natives or African Americans, despite legalizing white immigrants and Jefferson's attempts for Native Americans.}}
    }
    \cornell{What were the first signs of partisan divisions within the United States?}{\begin{itemize}
        \item Philadelphia delegates had surpassed instructions from Congress/states, creating completely different form of government
        \begin{itemize}
            \item Feared Constitution would never be ratified (required unanimous approval), prompting change with 9/13 states required
            \item Called for state convention (not legislatures) to consider document, voting "yes" or "no"
            \begin{itemize}
                \item Confederation Congress passively accepted work, submitting to states; all but Rhode Island began to consider
            \end{itemize}
        \end{itemize}
        \item Great national debate had emerged over Constitution, with frequent fights between factions (death in Albany)
        \begin{itemize}
            \item Supporters were far more organized, had support of two key men of Washington/Franklin
            \begin{itemize}
                \item Created label of "Federalists"
                \item Supported by Hamilton, Madison, John Jay, who wrote essays justifying Constitution, becoming known as \textit{The Federalist Papers}
            \end{itemize}
            \item Federalists deemed Critics "Antifederalists," who also had intelligent arguments
            \begin{itemize}
                \item Presented themselves as Revolutionary defenders against tyrannical government and aristocrachy 
                \item Lack of bill of rights led to fundamental mistrust, with no guarantees that government would protect liberties of people
            \end{itemize}
            \item Debate fundamentally between two fears: for Federalists, anarchy, chaos, and unchecked power of masses; for Antifederalists, concentrated power (\textit{not} anarchy)
        \end{itemize}
        \item Ratification quickly proceeded through winter
        \begin{itemize}
            \item Delaware first, followed by NJ/GA, ratifying unanimously
            \item PA and MA saw greater challenge but loss of Antifederalists in final vote; NH, ninth state to do so, did in June 1788
            \item Divided VA and NY meant that constitution could not truly go into effect; finally narrowly consented by end of June (NY feared commercial effects of being left out), demanding bill of rights
            \begin{itemize}
                \item NC adjourned without action (waiting for amendments) while RI did not call convention to consider
            \end{itemize}
        \end{itemize}
    \end{itemize}
    \textbf{The Philadelphia delegates, fearing the Constitution would not be ratified, implemented a unique system for ratification requiring only 9 states to approve. Partisan divisions began, with tensions between Federalists and Antifederalists. However, ratification eventually proceeded with multiple key conditions.}}
    \cornell{What were the results of the first key Congress meetings?}{\begin{itemize}
        \item Most new members had supported ratification
        \item Washington became first president with John Adams as VP
        \begin{itemize}
            \item Inaugaurated w/ NYC as capital in 1789
        \end{itemize}
        \item First Congress tasked with filling gaps in Constitution, most notably Bill of Rights
        \begin{itemize}
            \item Even Madison had agreed on significance in eyes of opponents
            \item Agreed on 12 amendments; 10 had been ratified by end of 1791, becoming parts of modern Bill of Rights (freedom of religion, speech, press; trial by jury; no arbitrary arrest)
        \end{itemize}
        \item Constitution had created Supreme Court; Congress left to decide number of judges
            \begin{itemize}
                \item Created six members in Supreme Court (chief justice, five associate justices); thirteen district courts with one judge each; three circuit courts of appeal with one district judges w/ two Supreme Court ones
                \item Supreme Court given final say in state laws
            \end{itemize}
        \item Exact executive departments left unclear
        \begin{itemize}
            \item First Congress created state, treasury, war; offices of attorney general, postmaster general
            \item Hamilton appointed as treasury secretary, Henry Knox as war secretary, Edmund Randolph as attorney general, 
            Jefferson as secretary of state 
        \end{itemize}
    \end{itemize}
    \textbf{Congress was tasked with filling gaps in the Constitution, including the Bill of Rights with ten key amendments, the Supreme Court with six members and two other courts beneath them, and executive departments of state, treasury, war, attorney general, and postmaster general.}}
    \cornell[Federalists and Republicans]{What characterized the major divisions in the American government?}{\textbf{The two key parties which emerged due to the vagueness of the Constitution in many key questions were the Federalists, led by Washington's powerful treasury secretary, Hamilton, an aristocrat who founded a national bank and created new taxes, and the Republicans, led by Jefferson, a Virginia planter who believed in a powerful agrarian republic.}}
    \cornell{What was the context behind the deeply divided partisan system?}{\begin{itemize}
        \item First twelve years known for level of extreme bitterness due to vague responses to key question by framers of constitution
        \item Basic debate very similar to that over Constitution: one side with nationalist group seeking centralized authority and powerful standing in world; other side (initially minority) believed that American society should remain rural/agrarian with modest centralized government
        \begin{itemize}
            \item Centralizers -> Federalists under Hamilton
            \item Opponents -> Republicans under Madison/Jefferson
        \end{itemize}
    \end{itemize}
    \textbf{Because the framers of the Constitution had left many vague questions open to the interpretation of governments, a level of extreme bitterness emerged which eventually led into the creation of two distinct parties: the Federalists, who sought a strong national government, and the Republicans, who sought a weaker national government and greater authority centered in the states.}}
    \cornell{What was Hamilton's stance as a Federalist?}{\begin{itemize}
        \item First twelve years saw government dominance by Federalists, mainly due to Washington's prestige, but Washington believed that the presidency was above controversies -> kept personal matters away from congressional behavior
        \item Dominant figure in administration was Hamilton, who exterted a significant amount of power in domestic/foreign policy: more than any others during his term and after it
        \begin{itemize}
            \item Aristocratic in philosophy/tastes despite Caribbean origins; believed in stable government with support of wealthy/powerful
            \begin{itemize}
                \item Argued that elites were key to success of government, public debt should be taken, in part, by government
                \item Many depreciated certificates to signify debt were called back, exchanged for bonds which could eventually be cashed in
                \item Sought to create \textit{national debt} to allow creditors to lend money to the government
            \end{itemize}
            \item Sought to create national bank to give place for federal funds, collected taxes
            \begin{itemize}
                \item Would be chartered by government with a monopoly over federal banking
                \item Represented stable base to replace existing system of dispersed banks in large cities
            \end{itemize}
            \item For greater sources of revenue beyond sale of western lands, proposed taxes
            \begin{itemize}
                \item Tax paid by distillers of alcoholic liquors mostly for backcountry PA, VA, NC 
                \item Import tariffs to raise revenue, protect American manufacturing
            \end{itemize}
        \end{itemize}
        \item Federalists offered viewpoint for future of America 
    \end{itemize}
    \textbf{Hamilton, the dominant Federalist due to Washington's attempt at an unbiased interference in Congress, believed in the power of elites in government, proposing a national debt for bonds, sought to create a national bank for funds and taxes to replace the antiquated, decentralized system, and new taxes to raise revenue.}}
    \cornell{How did Hamilton enact the Federalist program?}{\begin{itemize}
        \item Many Congress members opposed proposal to accept debt at face value
        \begin{itemize}
            \item Initial debt certificates had been issued to farmers/merchants for Revolutionary War (supplies, military service)
            \begin{itemize}
                \item Many had been sold to speculators in 1780s (but at fraction of face value)
            \end{itemize}
            \item Some believed that original purchasers should be returned key bonds, but Hamilton felt that the true return belonged to the bondholders (not the ones who chose to sell), eventually accepted 
        \end{itemize}
        \item Proposal that federal governments should take on state debt was objected to due to fear that states with few debts would have to pay taxes for states with larger debts; struck a deal
        \begin{itemize}
            \item Deal involved movement of national capital: had moved back to PA in 1790, but VA wanted capital nearer to them in South 
            \item When Jefferson and Hamilton met, Hamilton promised to move capital between VA and MD in exchange for votes on state tax bill (Potomac River)
        \end{itemize}
        \item National bank initially difficult to pass due to vague alignment with Constitution (Republicans), but Congress finally signed with Washington's hesitant approval
        \begin{itemize}
            \item Bank of United States began in 1791 with charter to continue for 20 years 
            \item Won support of key parts of population, restoring public credit, raising bond prices to above face value, speculators earned large profits, manufacturers received tariff revenue, merchants benefited 
        \end{itemize}
        \item Small farmers complained about uneven taxation due to distillery tax generally applying to them -> great political opposition rose
    \end{itemize}
    \textbf{Hamilton encountered many roadblocks on his path to enacting his programs, but nearly all of them succeeded. Although many Congress members feared that his face-value debt proposal would risk the well-being of the original purchases, he convinced them that the bondholders were more important; the process of taking on state debt was approved hesitantly in exchange for the national capital moving from PA to between VA/MD, and the national bank was successful. However, an opposition arose due to many smaller farmers.}
    }
    \cornell{What were the key traits of the Republican opposition?}{\begin{itemize}
        \item Constitution did not reference political parties because most framers felt the prospect was too dangerous, most agreeing that disagreements needn't leed to fractioning
        \item Despite Madison's strong initial conviction, changed his mind due to powerful majority of Hamilton's followers with partisan system already having emerged
        \begin{itemize}
            \item Hamilton had awarded franchises to key supporters in hopes of winning allies, formed local associations to strengthen standing in communities
            \begin{itemize}
                \item Many felt these were on par with the behavior of the British government
            \end{itemize}
            \item Opponents believed vigorous opposition was critical: the Republican Party (\textit{not} related to modern Republicans)
        \end{itemize}
        \item Republicans went to great lengths to establish partisan influence
        \begin{itemize}
            \item Corresponded between states, banded to influence elections, fought to defend against Federalists
            \item Despite this, neither side admitted that they were behaving as a party
        \end{itemize}
        \item Prominent Republicans from the beginning were Jefferson/Madison, close collaborators
        \begin{itemize}
            \item Charismatic Jefferson, a farmer-planter believed in agrarian republic; quickly became most prominent spokesperson
            \item Jefferson believed in commercial activity: farmers should market their crops internationally and the U.S. should develop manufacturing capacity
            \begin{itemize}
                \item Feared large cities due to urban mobs; development of advanced economy would increase \# of propertyless
                \item Sought decentralized society with smaller property owners
            \end{itemize} 
        \end{itemize}
    \end{itemize}
        \textbf{Republicans, led by Thomas Jefferson, grew in numbers as many began to resent the growing partisanism of Hamilton and his allies. Jefferson, despite believing in commercial and industrial activity, feared the growth of large cities and an advanced economy due to the potential of urban mobs.}}
        \cornell{What were the key differences between Republicans and Federalists?}{\begin{itemize}
            \item Federalist-Republican differences most visible in French Revolution
        \begin{itemize}
            \item Federalists horrified by radical execution of king/queen, attacks on organized religion
            \item Republicans felt antiaristocratic spirit was key, often imitating Jacobins
        \end{itemize}
        \item Federalists more common in NE while Republicans generally in rural South/West
        \item By election of 1792, Jefferson/Hamilton urged Washington to run for a second term with his reluctant agreement; although most Americans believed Washington did not partake in partisan battle, he had more sympathy for Federalists
    \end{itemize}
    \textbf{The differences between the two parties became evident during and after the French Revolution, with the parties supporting opposite sides. Federalists were more common in the Northeast while Republicans were more common in the rural South and West.}}
    \cornell[Establishing National Sovereignty]{How did the Federalists ensure the emergence of national sovereignty?}{\textbf{The Federalists methodically approached each growing issue with varying degrees of success: for instance, they secured the frontier through intimidation and promises of statehood, they worked to maintain neutrality amidst the French Revolution, and created key treaties with various nations to reduce tensions, but they were unable to truly appease or satisfy the natives with true promises of citizenship.}}
    \cornell{How did the Federalists secure the frontier?}{\begin{itemize}
        \item Confederation unable toe tie western areas into government, with western MA farmers revolting, settlers in VT, KY, TN considering separation
        \item 1794 PA farmer revolt due to whiskey excise tax -> federal government took on rebellion rather than leaving to states (unlike Shays' rebellion)
        \begin{itemize}
            \item Washington sent army of 15,000 to Pittsburgh, with rebellion quickly collapsing
        \end{itemize}
        \item Allegiance to other groups won by accepting statehood
        \begin{itemize}
            \item NC and RI joined after Bill of Rights was developed
            \item VT, with independent state government since Revolution joined, followed by KY (from VA's western lands) and TN (from NC's western lands)
        \end{itemize}
    \end{itemize}
    \textbf{The Federalists ensured the allegiance of distant regions either by direct intimidation (like how they put down the farmer revolt) or by ensuring the statehood of developed regions.}}
    \cornell{How were natives treated in the Constitutional U.S.?}{\begin{itemize}
        \item Greater challenge to government was native resistance to U.S. claiming of their lands
        \begin{itemize}
            \item U.S. had successfully put down most border revolts which had emerged due to ordinances
            \item Question remained as to who would own the western lands
        \end{itemize}
        \item Reflected issue of racial representation in Constitution 
        \begin{itemize}
            \item Natives barely mentioned, only in that those who were not taxed were excluded from the population and that Congress had the power to regulate trade with natives, and that land treaties should be followed
            \item Recognized tribes as legal entities but not as foreign nations nor citizens, with no governmental representation; but natives believed in sovereignty over their own lands, but Constitution offered no answer
        \end{itemize}
    \end{itemize}
    \textbf{Tensions remained between natives despite the U.S. ability to put down nearly all of their rebellions - the roots were in the vagueness of the Constitution.}}
    \cornell{How did the U.S. maintain neutrality between disparate foreign nations?}{\begin{itemize}
        \item GB only sent foreign minister eight years after conclusion of Revolution due to threat of Republicans for trade restrictions on their ships
        \item Greater tensions with Britain emerged in new French government (from Revolution) waged war against British: Americans attempted neutrality
        \begin{itemize}
            \item Edmond Genet, French diplomatic representative to America, arrived at Charleston rather than meeting president, making plans for American ports to store and create French warships, encouraging Americans to ally
            \begin{itemize}
                \item Conduct angered Washington, who gave him icy reception in Philadelphia
                \item Genet embarrassed supporters of French Revolution, but his faction soon lost power in France, forcing him to receive asylum in U.S. 
            \end{itemize}
            \item Greater challenge from GB when Royal Navy began to capture ships trading with French West Indies
            \begin{itemize}
                \item Angered public
            \end{itemize}
            \item Governor general of Canada even delivered speech to natives on border, encouraging them to target American dominance 
        \end{itemize}
    \end{itemize}
    \textbf{Although the Neutrality Act attempted to maintain American neutrality amidst the French Revolution, the Americans began to face extreme challenges to this tenet from both sides. Genet, a French representative, tried to garner the public to ally against the British; on the other side, Britain began to seize American ships which were engaged in trade benefitting the French.}}
    \cornell{What were Jay's and Pinckney's Treaties?}{\begin{itemize}
        \item Randolph, Jefferson's successor, and Hamilton, saw severity of relationship with British and hoped to mend -> Washington created special commissioner to England (John Jay)
        \begin{itemize}
            \item Jay's position intended to receive compensation for assaults, demand withdrawal of frontier forces with new commercial treaty
            \item Jay's treaty did not accomplish primary goals, but did ensure reduced conflict and prevented major war; inability to fully accomplish aims -> public opinion was poor
            \begin{itemize}
                \item Opponents (most Republicans, some Federalists) desperately tried to prevent passthrough at Senate
                \item Even Randolph attempted to prevent ratification, but ultimately, Jay's Treaty was successful
            \end{itemize}
        \end{itemize}
        \item Treaty guaranteed settlement of conflcit with Spanish: Spanish began to fear that British would join w/ Americans against them
        \begin{itemize}
            \item Thomas Pickney assigned to be special negotiator with Spain, easily able to gain all of key U.S. desires
            \item In Pickney's Treaty, Spain recognized American right to navigate Mississippi River, renegotiated northern border w/ Florida (31st parallel)
        \end{itemize}
    \end{itemize}
    \textbf{Jay's and Pinckney's Treaties both aimed to correct American foreign relations. John Jay, appointed as the special commissioner to England, did loosen tensions and prevented a war; however, his inability to solve many poignant issues caused many to dislike the treaty greatly. However, Jay's treaty allowed for the passage of Pickney's Treaty with Spain, which ceded American rights to parts of Spanish territory.}}
    \cornell[The Downfall of the Federalists]{What caused the federalists to become reduced in power and control over time?}{\textbf{Although the Federalists won the election of 1796 after a close race, their confidence after their success in negotiating a treaty with the French led them to create two dangerous laws to suppress their opposition. This caused the Republicans to mobilize their opposition to these laws and split the nation gravely. By the election of 1800, the Republicans had significant enough a following to win with Thomas Jefferson and Aaron Burr; however, there was still a long road ahead until complete dominance.}}
    \cornell{What was the context behind the Federalist downfall?}{\textbf{Most Americans disagreed with the nature of partisanship: they believed that a single party should exist without an organized opposition. Federalists, too, feared the threat to stability posed by Republicans; however, they began to mobilize against the Republicans, prioritizing stability over individual rights. Federalists eventually disappeared as a powerful political force.}}
    \cornell{What was the result of the election of 1796?}{\begin{itemize}
        \item Washington retired from office in 1797 despite pressure to run for a third term, delivering Farewell Address (in fact long paper)
        \begin{itemize}
            \item Reacted harshly against Republicans, especially those who conspired with French to curb Federalist control
        \end{itemize}
        \item Washington's retirement meant that partisan rivalries were unchecked, with Jefferson immediately emerging as the leader of the Republicans, but John Adams (VP) being selected over Hamilton (too many enemies!) for Federalists
        \item Federalists remained dominant, appearing likely to win election; however, factional rivalries which had been mediated by Washington began to run freely
        \begin{itemize}
            \item Many Federalists (especially Hamilton) disagreed with Adams' role as forerunner, voting instead for running mate, Pinckney -> Jefferson nearly won nomination
            \item Adams was successful but Jefferson became vice president (second most votes) <- system changed in 1804 w/ Twelfth Amendment
        \end{itemize}
        \item Adams ruled over greatly divided party without even being the true forerunner (Hamilton remained most dominant)
        \begin{itemize}
            \item Despite diplomatic ability, Adams was an austere and rigid politician, unable to solicit support or inspire enthusiasm 
            \item Assumed that own virtue would inevitably lead to greatness
        \end{itemize}
    \end{itemize}
    \textbf{The election of 1796 had the major undertone of Washington's retirement, which let partisan divisions run free. It saw the close election of John Adams despite the large Federalist numbers due to factional divisions, with Jefferson as his vice president. Adams was not a skilled politician, unable to earn support from others or inspire enthusiasm among the people.}}
    \cornell{What was the quasi-war with France?}{\begin{itemize}
        \item Strengthened relations w/ GB and Spain saw deterioration of French relations
        \begin{itemize}
            \item Vessels began to capture American ships
            \item Refused Pinckney's brother (Charles) from entering as an official diplomat
        \end{itemize}
        \item Many of Adams' advisors believed in war (notably Secretary of State Thomas Pickering of New England)
        \begin{itemize}
            \item Hamilton sought reconciliation with France with Adams' agreement -> bipartisan commission with various Americans to negotiate with France 
            \item Prince Talleyrand (French foreign minister) demanded loan before negotiations; met with denial 
            \item Adams sent message to Congress demanding war over Talleyrand's behavior, calling messengers "X," "Y," and "Z" -> known as "XYZ affair"
            \begin{itemize}
                \item Persuaded to end all trade with France, authorize capturing of French ships
                \item Began to win duel with French vessels, cooperating with Britain
            \end{itemize}
        \end{itemize}
        \item France ultimately chose to appease the Americans, with Adams' second commission in 1800 agreeing with Bonaparte to treaty with new commercial agreements
    \end{itemize}
    \textbf{The quasi-war with France emerged as Britain and Spain began to forge closer alliances with the Americans, causing the French to greatly fear invasion and begin to attack American vessels. Adams' advisors believed in war, but Hamilton encouraged Adams to create a bipartisan commission to negotiate with France, which was demanded a loan from France before they could form any agreements. A quasi-war began to emerge with mutual naval battles as ships were captured; the French finally decided to appease the Americans, approving their next envoys and forming a treaty.}}
    \cornell{What protests emerged following the war with France?}{\begin{itemize}
        \item Federalists able to consolidate power due to conflict with France, beginning to search for ways to silence Republicans
        \begin{itemize}
            \item Passed Alien Act, restricting foreigners from becoming citizens, strengthening the president's control
            \begin{itemize}
                \item Discouraged immigration
            \end{itemize}
            \item Sedition Act allowed government to prosecute any suspected of "sedition," which was often abused to stifle any form of opposition 
            \begin{itemize}
                \item Led to conviction of ten men (mostly Republican newspaper editors)
            \end{itemize}
            \item Adams was cautious about laws, ensuring no illegal immigrants were deported and preventing and major crusade against Republicans 
        \end{itemize}
        \item Republicans hoped to reverse acts using state legislatures
        \begin{itemize}
            \item Emerged due to misconception that states had power to nullify congressional legislation, not only Supreme Court
            \item Virginia (by Madison) and Kentucky (by Jefferson) Resolutions adopted in state legislatures, arguing that states had power to remove abusive laws
            \item Wide support not earned, but Republicans able to move dispute to national level, causing great crisis
            \begin{itemize}
                \item Nation divided politically: state legislatures extremely tense and Congress frequently known for violent disagreements
            \end{itemize}
        \end{itemize}
    \end{itemize}
    \textbf{The Federalists' attempt to consolidate power through the Alien and Sedition Acts led to great anger among the Republicans, who hoped to reverse these harmful acts on the state level. When these attempts were not met with widespread success, they elevated the dispute to a national leve, causing a crisis.}}
    \cornell{What was the result of the election of 1800?}{\begin{itemize}
        \item Adams and Jefferson were opposed again, but campaign was very ugly
        \begin{itemize}
            \item Although two main candidates mainly composed w/ dignity, supporters showed no restraint
            \item Federalists called Jefferson a radical leading wild men who would bring reign of terror; Republicans called Adams tyrant who sought to be king who would impose slavery on the people
            \begin{itemize}
                \item Federalists accused Jefferson of relationship with slave woman
            \end{itemize}
        \end{itemize}
        \item Election was very close; deciding vote in NY, where Aaron Burr had created organization of war veterans to support Republicans
        \begin{itemize}
            \item Burr's group carried entire city by large majority and entire state; Jefferson believed to have won
            \item Complication emerged: practice was for each elector to cast two votes: one for presidential candidate and another for vice presidential candidate
            \begin{itemize}
                \item Plan was for one elector not to vote for Burr but to vote for Jefferson; plan failed leaving tie between Jefferson and Burr 
                \item House of Representatives would have to decide between the two by casting a single vote each 
            \end{itemize}
            \item New Congress (Republican majority) could not convene until president had been selected and inaugaurated -> Federalist Congress forced to choose
            \begin{itemize}
                \item Some hoped to strike deal with Burr, but ultimately selected Jefferson due to fear of unreliability
            \end{itemize}
        \end{itemize}
        \item Only remaining Federalist control was judiciary branch
        \begin{itemize}
            \item Adams spent final months keeping hold secure: reduced Supreme Court justiceships while increasing number as a whole in Judiciary Act of 1801
            \item Federalists rapidly appointed to fill newly created positions, supposedly working until last day in office to fill places
        \end{itemize}
        \item Republicans viewed their victory as complete (false), believing tyranny had ended and a new era would be ushered in 
        \begin{itemize}
            \item Called election "revolution" of 1800, but true extent of revolution remained unclear
        \end{itemize}
    \end{itemize}
    \textbf{The Republicans won the election, with Jefferson earning the presidency after a close race with running mate Aaron Burr. The Federalists saw a massive blow to their dominance, having control only over the judiciary branch. Although the Republicans viewed their success as a complete victory for the future of the nation, there were many more obstacles to come.}}
    \end{document}