\documentclass[a4paper]{article}
    \usepackage{tcolorbox}
    \usepackage{amsmath}
    \tcbuselibrary{skins}
    
    \title{
    \vspace{-3em}
    \begin{tcolorbox}
    \Huge\sffamily \begin{center} Chapter 1  \mbox{} \\ \huge The Collision of Cultures\mbox{} \\
    \LARGE Finn Frankis \mbox{} \\
    \Large AP US History - August 22{$^\text{nd}$}, 2018 \end{center} 
    \end{tcolorbox}
    \vspace{-3em}
    }
    \date{}
    \author{}
    
    \usepackage{background}
    \SetBgScale{1}
    \SetBgAngle{0}
    \SetBgColor{red}
    \SetBgContents{\rule[0em]{4pt}{\textheight}}
    \SetBgHshift{-2.3cm}
    \SetBgVshift{0cm}
    \usepackage[margin=2cm]{geometry} 
    
    \makeatletter
    \def\cornell{\@ifnextchar[{\@with}{\@without}}
    \def\@with[#1]#2#3{
    \begin{tcolorbox}[enhanced,colback=gray,colframe=black,fonttitle=\large\bfseries\sffamily,sidebyside=true, nobeforeafter,before=\vfil,after=\vfil,colupper=blue,sidebyside align=top, lefthand width=.3\textwidth,
    opacityframe=0,opacityback=.3,opacitybacktitle=1, opacitytext=1,
    segmentation style={black!55,solid,opacity=0,line width=3pt},
    title=#1
    ]
    \begin{tcolorbox}[colback=red!05,colframe=red!25,sidebyside align=top,
    width=\textwidth,nobeforeafter]#2\end{tcolorbox}%
    \tcblower
    \sffamily
    \begin{tcolorbox}[colback=blue!05,colframe=blue!10,width=\textwidth,nobeforeafter]
    #3
    \end{tcolorbox}
    \end{tcolorbox}
    }
    \def\@without#1#2{
    \begin{tcolorbox}[enhanced,colback=white!15,colframe=white,fonttitle=\bfseries,sidebyside=true, nobeforeafter,before=\vfil,after=\vfil,colupper=blue,sidebyside align=top, lefthand width=.3\textwidth,
    opacityframe=0,opacityback=0,opacitybacktitle=0, opacitytext=1,
    segmentation style={black!55,solid,opacity=0,line width=3pt}
    ]
    
    \begin{tcolorbox}[colback=red!05,colframe=red!25,sidebyside align=top,
    width=\textwidth,nobeforeafter]#1\end{tcolorbox}%
    \tcblower
    \sffamily
    \begin{tcolorbox}[colback=blue!05,colframe=blue!10,width=\textwidth,nobeforeafter]
    #2
    \end{tcolorbox}
    \end{tcolorbox}
    }
    \makeatother

    \parindent=0pt
    
    \begin{document}
    \maketitle
    \SetBgContents{\rule[0em]{4pt}{\textheight}}
    
    \cornell[America Before Columbus]{What were the major characteristics of pre-Columbian America?}{\textbf{Society began in the Americas likely after the arrival of boat-riding Japanese and Australian natives. Initially, almost all were characterized by hunting-and-gathering lifestyles, but agriculture slowly emerged, subsequently changing society and tribal culture significantly. Notable Native American hotspots include Mesoamerica, Peru, and the Southwest U.S.}}
    \cornell{What were the origins and lifestyles of the peoples of precontact Americas?}{
        \begin{itemize}
            \item Initial belief that all early migrations to Americas from humans crossing Bering Land bridge as result of newly developed tools
            \begin{itemize}
                \item Referred to as "Clovis" people, originated from Mongolian stock connected to Siberia
                \item Clovis established first American civilizations, eating animals and creating tools
                \item Gradually expanded to warmer region
            \end{itemize}
            \item Recent evidence has revealed that earlier Asian migrants had already arrived in South America long before Clovis
            \begin{itemize}
                \item Likely arrived by boat (backed up by early population of Japan, Australia, Pacific)
            \end{itemize}
            \item Early population far more diverse/ scattered than scholars originally believed 
            \begin{itemize}
                \item Some likely arrived from Polynesia, Japan; possible from Europe/Africa
                \item Most American Indians most similar to Siberians: Mongolians likely dominated 
            \end{itemize}
            \item Period referred to as "Archaic Period"  ($\approx$ 8000 - 3000 BCE)
            \begin{itemize}
                \item Larger animals driven to extinction with spear-hunting, stone tools
                \item Bows and arrows unknown until 400-500 CE 
                \item New tools emerged, including fishing nets, traps, gathering baskets
                \item Most important crop waas corn, followed by beans/squash in many communities
                \item Sedentary settlements slowly began to form
            \end{itemize}
        \end{itemize}
        \textbf{The earliest settlers in America, opposite to the long-held belief that they had arrived over the Bering Land Bridge, likely travelled by boat from Polynesia, Japan, and Mongolia. They arrived with stone tools but gradually developed more advanced technology, including baskets, nets, and fishing traps. Their most significant crops were corn, beans, and squash.}
    }
    \cornell{What defined the growth of early South American civilizations?}{
        The most prosperous early civilizations were in fact south of the United States, mostly in Central/South America and Mexico. \\~\\
        The Incas built a 2000-mile empire in the Peruvian region.
            \begin{itemize}
                \item Persuaded most leaders to join forces w/ them 
                \item Connected by paved roads, which encouraged significant trade 
            \end{itemize}
        \textbf{The expansive Incan Empire was the dominant force in South America for multiple decades, known for their well-kept paved roads.}
    }
    \cornell{What defined the growth of pre-Aztec civilizations in Central America?}{
        In Mesoamerica, civilization emerged as early as 10000 BCE.
        \begin{itemize}
            \item Olmec society emerged around 1000 BCE 
            \item Most sophisticated: Mayans around 800 CE
            \begin{itemize}
                \item Written language, numerical system, calendar, agriculture, trade routes
            \end{itemize}
        \end{itemize}
        \textbf{The civilizations in Central America emerged far earlier than those in South America. Significant prosperity began with the Olmecs around 1000 BCE, followed by the Mayans, both of whom were known for a written language and calendar.}
    }
    \cornell{What were the key characteristics of the Aztec Empire?}{
        The Mayans were gradually suceeded by the Aztecs, who called themselves the Mexica.
        \begin{itemize}
            \item Established city of Tenochtitlan in modern Mexico City 
            \begin{itemize}
                \item Became largest American city \textit{by far}
                \item Characterized by many aqueducts, public buildings, organized military 
            \end{itemize}
            \item Incorporated other neighboring groups into society
            \begin{itemize}
                \item Rather than directly conquering, drew into elaborate tribute network enforced by military
                \item Most groups maintained significant autonomy
            \end{itemize}
            \item Religion based on human sacrifice
            \begin{itemize}
                \item Kept historical tradition of blood-letting
                \item Introduced new concept of living hearts (generally captives)
            \end{itemize}
        \end{itemize}
        \textbf{The Aztecs were the last successful native civilization in Central America. Despite having been crushed by European disease, the Aztecs were the dominant force in southern North/Central America for centuries due to their extensive tribute system and large capital city.}
    }
    \cornell{What characterized the early North American civilizations?}{
        \begin{itemize}
            \item Northern civilizations did not develop elaborate political systems, primarily basing society on hunting/gathering/fishing
            \item Some civilizations primarily based around hunting/fishing
            \begin{itemize} 
                \item In Arctic Circle, Eskimos fished/hunted seals/caribou/moose, travelled by dogsled
                \item In PNW, focused on salmon fishing with coastal settlements
                \item Far West contained numerous successful fishing communities, with some gathering
            \end{itemize}
            \item Most elaborate agricultural society in Southwest 
            \begin{itemize}
                \item Built irrigation for dryland farming, towns as trade centers 
                \item Architecture primarily stone/adobe, known as pueblos
            \end{itemize}
            \item Other farming communities in Great Plains, though some hunted buffalo
            \item Eastern U.S., covered with forests, inhabited by Woodland Indians
            \begin{itemize}
                \item Enjoyed most food on continent; farmed, fished, hunted, gathered
                \item Major city of trade: \textbf{Cahokia}, larged mound society
                \item Northeastern societies far more nomadic due to less fertile terrain
                \begin{itemize}
                    \item Techniques designed for quick exploitation, immediately killing land 
                    \item Moved after a few years, dispersing in winter
                \end{itemize}
                \item Tribes east of Mississippi unified by linguistic roots
                \begin{itemize}
                    \item Largest language group: Algonquian (Atlantic from Canada to Virginia)
                    \item Iroquoian centered in upstate NY 
                    \begin{itemize}
                        \item Comprised of Seneca, Cayuga, Onondoga, Oneida, Mohawk
                        \item Linked to Carolinas, Georgia, Cherokees
                    \end{itemize}
                    \item Muskogean included southernmost region
                    \begin{itemize}
                        \item Included Seminoles, Creeks, Choctaws
                        \item Fragile connections (did not view as single group)
                    \end{itemize}
                \end{itemize}
            \end{itemize}
        \end{itemize}
        \textbf{Early North American civilizations were characterized by vastly different traits: many fished and hunted (especially in the northwest), others focused on agriculture (mostly iun Southwest w/ adobe style), and others were primarily nomadic (like in the Northeast). Finally, most tribes east of the Mississippi were unified by linguistic roots, but they rarely made connections because of this.}
    }
    \cornell{What were the key cultural elements of Native American tribes?}{
        \begin{itemize}
            \item All tribes could be characterized by growing sedentary behavior as well as growing agriculture, population
            \item Defined social customs began to emerge, with central point often religion based on natural environment 
            \begin{itemize}
                \item Generally gods associated with crops, game, forests, rivers, etc. 
                \item Many tribes used colored totems for rituals, large festivals 
            \end{itemize}
            \item Social stratification continued with division of tasks by gender
            \begin{itemize}
                \item Women \textit{always} cared for children, prepared meals
                \item Farming often restricted to men (Pueblos), but others relied on men for hunting, battles
                \item Women respected for ability to keep family together while men battled/hunted 
                \begin{itemize}
                    \item Generally earned powerful role within family
                \end{itemize}
            \end{itemize}
        \end{itemize}
        \textbf{By the arrival of Europeans, almost all tribes had begun to adopt a sedentary, agricultural lifestyle. Settling often led to defined religion based around their natural environment; furthermore, agriculture increased social division between men and women.}
    }
    \cornell[Europe Looks Westward]{What influenced Europe to conquer the Americas and what were the effects of this?}{}
    \cornell{What characterized Europe's initial attempts at reaching the Americas?}{\begin{itemize}
        \item Early Norse seamen (like Leif Eriksson) had gotten glimpse at New World, but discoveries were not well-known
        \item Middle Ages far from adventurous w/ provincial outlook due to small duchies/kingdoms
        \begin{itemize}
            \item Lack of centralized rule limited wealth for large expeditions
            \item Focus on subsistence agriculture, local trade encouraged few to explore
            \item Church, w/ limited territorial control, unable to launch large venture
        \end{itemize}
    \end{itemize}
    \textbf{Except for some slight interest by some Norse seamen to explore the New World, few others cared to explore beyond their province due to the divided political state of the Middle Ages.}}
    \cornell{What incentivized Europeans to look for new lands?}{
        \begin{itemize}
            \item Population rebound from Black Death led to rising land value, greater merchant class seeking goods to trade, and advanced navigational technology
            \item New united governments seen as necessary due to weak authority of church had sufficient wealth to expand beyond Europe
            \item Many merchants hoped to find direct trade route with East (w/o Muslim middlemen)
            \begin{itemize}
                \item New monarchs agreed, prepared to finance voyages
                \item Portuguese initially most active as dominant maritime power
                \begin{itemize}
                    \item Sent Henry the Navigator to explore western African coast for powerful Christian empire to fight against northern Moors
                    \begin{itemize}
                        \item Reached Cape Verde
                    \end{itemize}
                    \item Others included Bartholomeu Dias, rounding cape of Africa; Vasco de Gama, reaching India
                \end{itemize}
            \end{itemize}
        \end{itemize}
        \textbf{The emergence of larger, more centralized empires in Europe provided rulers with wealth and a desire to expand their territory to find raw materials and to trade with far lands. This encouraged numerous voyages, most notably from Portugal, whose greatest accomplishment was Vasco de Gama's quick sea route to India.} 
    }
    \cornell{What were Christopher Columbus' major accomplishments?}{\begin{itemize}
        \item Columbus, a Genoese seafarer, hoped to reach Asia by westerly route (assumed smaller world, longer Asian continent, and no continents between Europe/Asia)
        \item Earned support from ambitious Spanish monarchy (not Portuguese), seeking to display great power created by marriage of Isabella/Ferdinand despite lesser maritime ability
        \item Columbus set off for Japan in 1492, reaching Bahamas after 10 weeks
        \begin{itemize}
            \item Believed to have reached China, returning w/ many "Indians"
            \item Returned after a year with goal to find wealth of Indies, Chinese \textit{khan}; created small colony in Hispaniola
            \item On third voyage, finally realized he had discovered new, potentially large continent
            \begin{itemize}
                \item Continued to believe he was close to Asia (unable to check due to Panamanian isthmus)
            \end{itemize}
        \end{itemize}
    \end{itemize}}
    \end{document} 