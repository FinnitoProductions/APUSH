\documentclass[a4paper]{article}
    \usepackage{tcolorbox}
    \usepackage{amsmath}
    \tcbuselibrary{skins}
    
    \title{
    \vspace{-3em}
    \begin{tcolorbox}
    \Huge\sffamily \begin{center} Chapter 1  \mbox{} \\ \huge The Collision of Cultures\mbox{} \\
    \LARGE Finn Frankis \mbox{} \\
    \Large AP US History - August 22{$^\text{nd}$}, 2018 \end{center} 
    \end{tcolorbox}
    \vspace{-3em}
    }
    \date{}
    \author{}
    
    \usepackage{background}
    \SetBgScale{1}
    \SetBgAngle{0}
    \SetBgColor{red}
    \SetBgContents{\rule[0em]{4pt}{\textheight}}
    \SetBgHshift{-2.3cm}
    \SetBgVshift{0cm}
    \usepackage[margin=2cm]{geometry} 
    
    \makeatletter
    \def\cornell{\@ifnextchar[{\@with}{\@without}}
    \def\@with[#1]#2#3{
    \begin{tcolorbox}[enhanced,colback=gray,colframe=black,fonttitle=\large\bfseries\sffamily,sidebyside=true, nobeforeafter,before=\vfil,after=\vfil,colupper=blue,sidebyside align=top, lefthand width=.3\textwidth,
    opacityframe=0,opacityback=.3,opacitybacktitle=1, opacitytext=1,
    segmentation style={black!55,solid,opacity=0,line width=3pt},
    title=#1
    ]
    \begin{tcolorbox}[colback=red!05,colframe=red!25,sidebyside align=top,
    width=\textwidth,nobeforeafter]#2\end{tcolorbox}%
    \tcblower
    \sffamily
    \begin{tcolorbox}[colback=blue!05,colframe=blue!10,width=\textwidth,nobeforeafter]
    #3
    \end{tcolorbox}
    \end{tcolorbox}
    }
    \def\@without#1#2{
    \begin{tcolorbox}[enhanced,colback=white!15,colframe=white,fonttitle=\bfseries,sidebyside=true, nobeforeafter,before=\vfil,after=\vfil,colupper=blue,sidebyside align=top, lefthand width=.3\textwidth,
    opacityframe=0,opacityback=0,opacitybacktitle=0, opacitytext=1,
    segmentation style={black!55,solid,opacity=0,line width=3pt}
    ]
    
    \begin{tcolorbox}[colback=red!05,colframe=red!25,sidebyside align=top,
    width=\textwidth,nobeforeafter]#1\end{tcolorbox}%
    \tcblower
    \sffamily
    \begin{tcolorbox}[colback=blue!05,colframe=blue!10,width=\textwidth,nobeforeafter]
    #2
    \end{tcolorbox}
    \end{tcolorbox}
    }
    \makeatother

    \parindent=0pt
    
    \begin{document}
    \maketitle
    \SetBgContents{\rule[0em]{4pt}{\textheight}}
    
    \cornell[America Before Columbus]{What were the major characteristics of pre-Columbian America?}{\textbf{Society began in the Americas likely after the arrival of boat-riding Japanese and Australian natives. Initially, almost all were characterized by hunting-and-gathering lifestyles, but agriculture slowly emerged, subsequently changing society and tribal culture significantly. Notable Native American hotspots include Mesoamerica, Peru, and the Southwest U.S.}}
    \cornell{What were the origins and lifestyles of the peoples of precontact Americas?}{
        \begin{itemize}
            \item Initial belief that all early migrations to Americas from humans crossing Bering Land bridge as result of newly developed tools
            \begin{itemize}
                \item Referred to as "Clovis" people, originated from Mongolian stock connected to Siberia
                \item Clovis established first American civilizations, eating animals and creating tools
                \item Gradually expanded to warmer region
            \end{itemize}
            \item Recent evidence has revealed that earlier Asian migrants had already arrived in South America long before Clovis
            \begin{itemize}
                \item Likely arrived by boat (backed up by early population of Japan, Australia, Pacific)
            \end{itemize}
            \item Early population far more diverse/ scattered than scholars originally believed 
            \begin{itemize}
                \item Some likely arrived from Polynesia, Japan; possible from Europe/Africa
                \item Most American Indians most similar to Siberians: Mongolians likely dominated 
            \end{itemize}
            \item Period referred to as "Archaic Period"  ($\approx$ 8000 - 3000 BCE)
            \begin{itemize}
                \item Larger animals driven to extinction with spear-hunting, stone tools
                \item Bows and arrows unknown until 400-500 CE 
                \item New tools emerged, including fishing nets, traps, gathering baskets
                \item Most important crop waas corn, followed by beans/squash in many communities
                \item Sedentary settlements slowly began to form
            \end{itemize}
        \end{itemize}
        \textbf{The earliest settlers in America, opposite to the long-held belief that they had arrived over the Bering Land Bridge, likely travelled by boat from Polynesia, Japan, and Mongolia. They arrived with stone tools but gradually developed more advanced technology, including baskets, nets, and fishing traps. Their most significant crops were corn, beans, and squash.}
    }
    \cornell{What defined the growth of early South American civilizations?}{
        The most prosperous early civilizations were in fact south of the United States, mostly in Central/South America and Mexico. \\~\\
        The Incas built a 2000-mile empire in the Peruvian region.
            \begin{itemize}
                \item Persuaded most leaders to join forces w/ them 
                \item Connected by paved roads, which encouraged significant trade 
            \end{itemize}
        \textbf{The expansive Incan Empire was the dominant force in South America for multiple decades, known for their well-kept paved roads.}
    }
    \cornell{What defined the growth of pre-Aztec civilizations in Central America?}{
        In Mesoamerica, civilization emerged as early as 10000 BCE.
        \begin{itemize}
            \item Olmec society emerged around 1000 BCE 
            \item Most sophisticated: Mayans around 800 CE
            \begin{itemize}
                \item Written language, numerical system, calendar, agriculture, trade routes
            \end{itemize}
        \end{itemize}
        \textbf{The civilizations in Central America emerged far earlier than those in South America. Significant prosperity began with the Olmecs around 1000 BCE, followed by the Mayans, both of whom were known for a written language and calendar.}
    }
    \cornell{What were the key characteristics of the Aztec Empire?}{
        The Mayans were gradually suceeded by the Aztecs, who called themselves the Mexica.
        \begin{itemize}
            \item Established city of Tenochtitlan in modern Mexico City 
            \begin{itemize}
                \item Became largest American city \textit{by far}
                \item Characterized by many aqueducts, public buildings, organized military 
            \end{itemize}
            \item Incorporated other neighboring groups into society
            \begin{itemize}
                \item Rather than directly conquering, drew into elaborate tribute network enforced by military
                \item Most groups maintained significant autonomy
            \end{itemize}
            \item Religion based on human sacrifice
            \begin{itemize}
                \item Kept historical tradition of blood-letting
                \item Introduced new concept of living hearts (generally captives)
            \end{itemize}
        \end{itemize}
        \textbf{The Aztecs were the last successful native civilization in Central America. Despite having been crushed by European disease, the Aztecs were the dominant force in southern North/Central America for centuries due to their extensive tribute system and large capital city.}
    }
    \cornell{What characterized the early North American civilizations?}{
        \begin{itemize}
            \item Northern civilizations did not develop elaborate political systems, primarily basing society on hunting/gathering/fishing
            \item Some civilizations primarily based around hunting/fishing
            \begin{itemize} 
                \item In Arctic Circle, Eskimos fished/hunted seals/caribou/moose, travelled by dogsled
                \item In PNW, focused on salmon fishing with coastal settlements
                \item Far West contained numerous successful fishing communities, with some gathering
            \end{itemize}
            \item Most elaborate agricultural society in Southwest 
            \begin{itemize}
                \item Built irrigation for dryland farming, towns as trade centers 
                \item Architecture primarily stone/adobe, known as pueblos
            \end{itemize}
            \item Other farming communities in Great Plains, though some hunted buffalo
            \item Eastern U.S., covered with forests, inhabited by Woodland Indians
            \begin{itemize}
                \item Enjoyed most food on continent; farmed, fished, hunted, gathered
                \item Major city of trade: \textbf{Cahokia}, larged mound society
                \item Northeastern societies far more nomadic due to less fertile terrain
                \begin{itemize}
                    \item Techniques designed for quick exploitation, immediately killing land 
                    \item Moved after a few years, dispersing in winter
                \end{itemize}
                \item Tribes east of Mississippi unified by linguistic roots
                \begin{itemize}
                    \item Largest language group: Algonquian (Atlantic from Canada to Virginia)
                    \item Iroquoian centered in upstate NY 
                    \begin{itemize}
                        \item Comprised of Seneca, Cayuga, Onondoga, Oneida, Mohawk
                        \item Linked to Carolinas, Georgia, Cherokees
                    \end{itemize}
                    \item Muskogean included southernmost region
                    \begin{itemize}
                        \item Included Seminoles, Creeks, Choctaws
                        \item Fragile connections (did not view as single group)
                    \end{itemize}
                \end{itemize}
            \end{itemize}
        \end{itemize}
        \textbf{Early North American civilizations were characterized by vastly different traits: many fished and hunted (especially in the northwest), others focused on agriculture (mostly iun Southwest w/ adobe style), and others were primarily nomadic (like in the Northeast). Finally, most tribes east of the Mississippi were unified by linguistic roots, but they rarely made connections because of this.}
    }
    \cornell{What were the key cultural elements of Native American tribes?}{
        \begin{itemize}
            \item All tribes could be characterized by growing sedentary behavior as well as growing agriculture, population
            \item Defined social customs began to emerge, with central point often religion based on natural environment 
            \begin{itemize}
                \item Generally gods associated with crops, game, forests, rivers, etc. 
                \item Many tribes used colored totems for rituals, large festivals 
            \end{itemize}
            \item Social stratification continued with division of tasks by gender
            \begin{itemize}
                \item Women \textit{always} cared for children, prepared meals
                \item Farming often restricted to men (Pueblos), but others relied on men for hunting, battles
                \item Women respected for ability to keep family together while men battled/hunted 
                \begin{itemize}
                    \item Generally earned powerful role within family
                \end{itemize}
            \end{itemize}
        \end{itemize}
        \textbf{By the arrival of Europeans, almost all tribes had begun to adopt a sedentary, agricultural lifestyle. Settling often led to defined religion based around their natural environment; furthermore, agriculture increased social division between men and women.}
    }
    \cornell[Europe Looks Westward]{What influenced Europe to conquer the Americas and what were the effects of this?}{\textbf{Europe's push to conquer the Americas was a need for more territory and a desire for a direct trade route with Asia. The Spanish were the first to conquer Europe, starting with Columbus, and they were known for their harsh native treatment, strict commercial policies, and the beginning of the Columbian Exchange.}}
    \cornell{What characterized Europe's initial attempts at reaching the Americas?}{\begin{itemize}
        \item Early Norse seamen (like Leif Eriksson) had gotten glimpse at New World, but discoveries were not well-known
        \item Middle Ages far from adventurous w/ provincial outlook due to small duchies/kingdoms
        \begin{itemize}
            \item Lack of centralized rule limited wealth for large expeditions
            \item Focus on subsistence agriculture, local trade encouraged few to explore
            \item Church, w/ limited territorial control, unable to launch large venture
        \end{itemize}
    \end{itemize}
    \textbf{Except for some slight interest by some Norse seamen to explore the New World, few others cared to explore beyond their province due to the divided political state of the Middle Ages.}}
    \cornell{What incentivized Europeans to look for new lands?}{
        \begin{itemize}
            \item Population rebound from Black Death led to rising land value, greater merchant class seeking goods to trade, and advanced navigational technology
            \item New united governments seen as necessary due to weak authority of church had sufficient wealth to expand beyond Europe
            \item Many merchants hoped to find direct trade route with East (w/o Muslim middlemen)
            \begin{itemize}
                \item New monarchs agreed, prepared to finance voyages
                \item Portuguese initially most active as dominant maritime power
                \begin{itemize}
                    \item Sent Henry the Navigator to explore western African coast for powerful Christian empire to fight against northern Moors
                    \begin{itemize}
                        \item Reached Cape Verde
                    \end{itemize}
                    \item Others included Bartholomeu Dias, rounding cape of Africa; Vasco de Gama, reaching India
                \end{itemize}
            \end{itemize}
        \end{itemize}
        \textbf{The emergence of larger, more centralized empires in Europe provided rulers with wealth and a desire to expand their territory to find raw materials and to trade with far lands. This encouraged numerous voyages, most notably from Portugal, whose greatest accomplishment was Vasco de Gama's quick sea route to India.} 
    }
    \cornell{What were Christopher Columbus' major accomplishments?}{\begin{itemize}
        \item Columbus, a Genoese seafarer, hoped to reach Asia by westerly route (assumed smaller world, longer Asian continent, and no continents between Europe/Asia)
        \item Earned support from ambitious Spanish monarchy (not Portuguese), seeking to display great power created by marriage of Isabella/Ferdinand despite lesser maritime ability
        \item Columbus set off for Japan in 1492, reaching Bahamas after 10 weeks
        \begin{itemize}
            \item Believed to have reached China, returning w/ many "Indians"
            \item Returned after a year with goal to find wealth of Indies, Chinese \textit{khan}; created small colony in Hispaniola
            \item On third voyage, finally realized he had discovered new, potentially large continent
            \begin{itemize}
                \item Continued to believe he was close to Asia (unable to check due to Panamanian isthmus)
            \end{itemize}
        \end{itemize}
        \item Columbus' strong religious sentiment influenced voyages
        \begin{itemize}
            \item Believed to be fulfilling divine mission, biblical prophecy
            \item Saw himself as messenger of new heaven/earth, appointed by God
        \end{itemize}
    \end{itemize}
    \textbf{Columbus, a deeply religious man, inadvertently found the Americas in a westerly journey hoping to reach Asia. He made three key voyages, only discovering that he was not truly in Asia by his third voyage. In summary, Columbus was the first European to reach the New World.}}
    \cornell{What was Columbus' impact on future exploration?}{\begin{itemize}
        \item Spain devoted greater resources/energy to maritime exploration
        \begin{itemize}
            \item Eventually replaced Portugal
        \end{itemize}
        \item Vasco de Balboa (Spain) crossed Isthmus of Panama
        \item Magellan found strait at southern end of South America, eventually reaching Philippines 
        \begin{itemize}
            \item Despite death in native conflict, completed first circumnavigation of globe
        \end{itemize}
        \item Spain had also explored coasts of North America (far north as Oregon)
    \end{itemize}
    \textbf{Columbus' voyages inspired Spain to invest a significant amount of wealth into the development of maritime exploration. Because of this, Spain was the first to reach Asia via a westerly route.}}
    \cornell{What characterized the shift from bypass to conquering in North America?}{
        \begin{itemize}
            \item Evenutally, explorers viewed as potential source of great wealth
            \begin{itemize}
                \item Spanish claimed entire New World, apart from Brazil (papally declared to belong to Portugal)
            \end{itemize}
            \item Columbus' colonists settled on Caribbean islands, hoping to find gold (no luck)
            \item \textbf{Hernan Cort\'es} led 600 men to Mexico (after 14 years as Cuban official)
            \begin{itemize}
                \item Followed stories of great treasure
                \item Met resistance from Aztecs under emperor Montezuma
                \begin{itemize}
                    \item Resistance quickly collapsed due to inadvertent exposure to smallpox
                    \item Saw easy victory as vindication from God
                \end{itemize}
                \item Cort\'es earned reputation as most brutal of \textit{conquistadores}
            \end{itemize}
        \end{itemize}
        \textbf{Spanish explorers slowly began to see the potential in the New World as a source of great wealth. They initially limited themselves to the Caribbean area,but Hernan Cort\'es led 600 men to Mexico, where their spread of smallpox easily defeated the Aztec Empire.}}
        \cornell{To where did other Spanish explorers travel after news of Cort\'es' voyage surfaced?}{\begin{itemize} 
            \item Pizarro conquered Peru, wealth of Incas
            \item Pizarro's deputy, Hernando de Soto, led search for gold through Florida, travelling west
            \begin{itemize}
                \item First white man to cross Mississippi River
                \item Opened Southwest to Spanish conquering
            \end{itemize}
            \item Spanish warrior story described by both military achievement and brutality/greed
            \begin{itemize}
                \item Continual trend throughout remainder of European conquering of Americas
            \end{itemize}
        \end{itemize}
        \textbf{After Cort\'es reached and exploited Mexico, Spanish explorers conquered Peru and the southern U.S. Many of their voyages were characterized by immense greed and native brutality.}}
        \cornell{What were the major characteristics of America under Spanish rule?}{
            The history of Spanish rule over America can be broken up into three distinct periods. 
            \begin{itemize}
                \item First, age of discovery/exploration following Columbus' voyages
                \begin{itemize}
                    \item First to arrive primarily sought riches, extremely successful
                    \item Large, unexploited gold mines yielded unparalleled quantities 
                \end{itemize}
                \item Second, age of conquest with powerful military forces extending their realms
                \begin{itemize}
                    \item Conquistadores left little more than destruction
                    \item Settlers established civilization
                \end{itemize}
                \item Third, once new laws prevented brutal conquests, colonialism emerged motivated by Catholic Church, pushed by Ferdinand/Isabella
                \begin{itemize}
                    \item Required sole religion to be Catholicism
                    \item Catholic mission extremely powerful with goal to convert natives
                    \begin{itemize}
                        \item Military garrisons prevented attack on missionaries, encourage conversion
                    \end{itemize}
                    \item Priests/friars accompanied all ventures
                    \item Ultimately extended throughout continents
                \end{itemize}
            \end{itemize}
            \textbf{Spanish America, broken into the three periods of discovery, conquest, and colonialism, was characterized heavily by religious missions.}
        }
        \cornell{What were the major characteristics of Spain's first northern outpost?}{
            \textbf{Spain's first northern outpost, established in St. Augustine, Florida in 1565, served as a military outpost, administrative center for Franciscan missionaries, and headquarters for the failed campaigns against the natives.}} 
        \cornell{What were the major characteristics of Spain's Southwest outpost?}{
            In 1598, Don Juan de O\~nate departed Mexico with 500 men and claimed the Pueblo lands in modern New Mexico.
            \begin{itemize}
                \item Settlement defined by traits similar traits to those already existing
                \item Assigned \textit{encomiendas} to settlers for native labor exploitation
                \begin{itemize}
                    \item Modeled on system exacted on Spanish Moors
                \end{itemize}
                \item Treated natives extremely harshly, including demanding tribute (despite natives outnumbering them greatly)
                \begin{itemize}
                    \item Resulted in removal as governor (1606), leading to stronger relations w/ Pueblo, including conversion and trade
                \end{itemize}
                \item Settlement grew despite constant threats to Spanish and Pueblo from Apache and Navajo raiders
                \item Little gold to be discovered, instead raising cattle and sheep on \textit{ranchos}
                \item Colony nearly collapsed after Pueblo revolts in 1680
                \begin{itemize}
                    \item Sparked by priest/government suppression of tribal rituals
                    \item Heavily influenced by major drought, Apache raids leading to instability within natives
                    \item Religious leader named Pope led uprising killing hundreds of Europeans, capturing Santa Fe
                    \begin{itemize}
                        \item Drove Spanish out temporarily
                        \item Soon returned (1696), crushing final revolts 
                    \end{itemize}
                \end{itemize}
                \item Exploitation of Pueblos continued, but Spanish understood severity and allowed natives to own land, focused on greater integration into society
                \begin{itemize}
                    \item Led to significant intermarriage
                    \item Pueblos began to view Spanish as allies against Apache/Navajo 
                \end{itemize}
                \item By 1750, Pueblo population had declined (disease/war/migration) while Spanish had grown somewhat
            \end{itemize}
            \textbf{Spain's outpost in the Southwest, despite being one of their most successful northern outposts, was known for harsh treatment of the natives through forced labor, tribute, and suppression of rituals. Damaging rebellions forced the Spanish to successfully reduce their exploitation through land ownership and social integration.}}
            \cornell{What was the peak of Spanish America?}{\begin{itemize}
                \item By the 16th century, Spanish Empire had become one of largest in world history
                \begin{itemize}
                    \item Caribbean, Mexico, coastal/inland South America (to Brazil), southern North America
                \end{itemize}
                \item Different from eventual English colonies in many ways
                \begin{itemize}
                    \item Despite initial independence from monarchy, gradual exercise of control with impossibility for independent political institutions
                    \item Spanish more successful in extracting wealth from surface but placed little emphasis on building agriculture/commerce
                    \begin{itemize}
                        \item Strict Spanish commercial policies (never imposed by British) w/ only one Spanish port, a few colonial ports allowed for trade, two voyages per year
                    \end{itemize}
                    \item Spanish placed considerably less effort than British, French, Dutch on peopling new society
                    \begin{itemize}
                        \item Some from mainland, others from Spanish outposts
                        \item Always outnumbered by natives, attempting to impose rule over large group
                    \end{itemize}
                \end{itemize}
            \end{itemize}
            \textbf{Spanish America's peak realm included a large portion of the American continent group. However, their success was limited due to strict monarchical control over the economy and politics and their inability to people society significantly.}}
            \cornell{What were the key biological and cultural exchanges in Spanish America?}{
            Unlike in British colonies, racial lines began to blur, but culture remained relatively distinct. Despite this, numerous exchanges occured between natives and settlers.
            \begin{itemize}
                \item Mostly unbeneficial for natives
                    \begin{itemize}
                    \item Inadvertent spread of diseases like flu, measles, chicken pox, typhus, \textbf{smallpox}
                    \begin{itemize}
                        \item Led to practical extinction of many native groups in Mexico, Caribbean islands (like Hispaniola, Mayan areas)
                        \item Some more successful with less intimate European contact 
                        \item Far worse than Black Death
                    \end{itemize}
                    \item More deliberate decimation of populations through subjugation, extermination
                    \begin{itemize}
                        \item Reflected ruthfulness of European war
                        \item Result of conviction that natives were "savages"
                    \end{itemize}
                    \item Relied on for slave-like labor system 
                    \begin{itemize}
                        \item Forced to work for fixed period, unable to depart without employer consent
                        \item Europeans began to value native villages over gold due to potential for great labor
                    \end{itemize}
                \end{itemize}
                \item Some forms of exchange less disastrous
                \begin{itemize}
                    \item New crops (sugar/bananas), livestock (cattle/pigs/sheep), and \textbf{horse} (vanished during ice age, transforming native lifestyle)
                    \begin{itemize}
                        \item Imported solely for Europeans, but natives slowly learned how to cultivate and harness power
                    \end{itemize}
                    \item Europeans learned about new agricultural techniques, crops (corn, squash, potatoes, tomatoes, peppers) even reaching Europe
                    \item Many societies involved natives in close contact with Europeans, encouraging natives to adopt European practices
                    \begin{itemize}
                        \item Led to creation of new mixed dialects
                        \item Adoption of Christianity led to blend with native traditions
                    \end{itemize}
                    \item European men outnumbered women by 10:1, leading to sexual contact with native women 
                    \begin{itemize}
                        \item Society became dominated by mixed-race \textit{mestizos}
                        \item Hierarchy placed Spanish at top, mixed race in middle, natives at bottom, though far more fluid than expected with "Spanish" assigned to anyone powerful
                    \end{itemize}
                \end{itemize}
            \end{itemize}
            \textbf{Exchanges between the natives and settlers were mixed in character: negative exchanges included the spread of disease and the subjugation of native populations into a slave-like labor system while postive included the mutual spread of agricultural techniques, crops, and culture.}
            }
        \cornell{What were the characteristics of African society before European intervention?}{
            The decline of native population in Spanish America forced the settlers to turn to a new region for slave labor: Africa. Most African men/women came from Guinea, home to a \textit{wide} variety of cultures.
                \begin{itemize}
                    \item Made up more than half of new arrivals to New World, giving them a great cultural impact
                    \item Despite European portrayal as uncivilized, many had advanced civilizations w/ powerful economies
                    \item Upper Guinea frequently made contact w/ Mediterranean world for ivory/gold/slave trade, making them early converts to Islam
                    \begin{itemize}
                        \item Larger kingdoms emerged, including Ghana and Mali
                        \item Fishing and rice center of economy, along with trade 
                    \end{itemize}
                    \item Farther south, more isolated from Europe/Mediterranean, less politically cohesive
                    \begin{itemize}
                        \item Village was major social unit w/ occasional kingdom (Benin, Congo, Songhay)
                        \item Known for development of fabrics, ceramics, wood, crops, livestock
                        \item Many nomadic tribes in interior with even less developed social systems
                    \end{itemize}
                    \item Culturally, mostly matrilineal (husband joins family of wife)
                    \begin{itemize}
                        \item Work often divided by gender, but women played major role in trade, farming, child care, food preparation
                        \item Political power often divided by gender: women chose female leaders and men chose male leaders, but chiefs generally men
                        \begin{itemize}
                            \item Role of chief often passed on to son of chief's eldest sister
                        \end{itemize}
                        \item Most lands south of Mali preserved indigenous religion
                        \begin{itemize}
                            \item Worshiped natural gods, ancestors
                        \end{itemize}
                        \item Social hierarchies generally began with priests/nobles, followed by farmers/traders/craftworkers/etc.,with slaves at bottom of society 
                        \begin{itemize}
                            \item Slavery not generally permanent nor inherited from parents
                        \end{itemize}
                    \end{itemize}
                \end{itemize}
                \textbf{African society, primarily centered in Guinea, was known for prosperous economies (especially in the north, with the gold/ivory/slave trade), mixed political cohesion, and matrilineal society.}
        }
        \cornell{What were the key characteristics of the African slave trade?}{
            \begin{itemize}
                    \item African slave trade had begun es early as eighth century CE - west Africans sold slaves to Mediterranean traders
                    \begin{itemize}
                        \item Response to demand for domestic servants, labor shortages
                        \item Portuguese sailors exploring African coast often brought back African slaves
                    \end{itemize}
                    \item High demand for sugar cane led to dramatic rise of slave market
                    \begin{itemize}
                        \item Portuguese islands inadequate, moving to Brazil/Caribbean islands 
                        \item Labor-intensive crop, required numerous workers
                        \item Led to wars between kingdoms for slave capturing
                        \begin{itemize}
                            \item Though initially mostly Portugal, Dutch later entered and won control
                        \end{itemize}
                    \end{itemize}
                \end{itemize}
                \textbf{The African slave trade began long before the arrival of western Europeans, instead motivated by Mediterranean demands. However, it was aggravated due to the high labor requirements of the sugar cane market.}
        }
        \cornell[The Arrival of the English]{What characterized English expansion in the New World?}{Although the English made contact with the New World shortly after Spain did, their internal society was not prepared for long-term settling of new lands.}
        \cornell{What was the commercial incentive which pushed the British to conquer the New World?}{
            \begin{itemize}
                \item Part of attraction was newness, ability to start fresh without flaws
                \begin{itemize}
                    \item Seen in More's \textit{Utopia}
                    \item Many viewed Tudor England as place of social and economic ills, partly due to frequent, costly wars and religious strife; mostly due to economic transformation
                    \begin{itemize}
                        \item Demand for wool grew rapidly, leading to rapid conversion of farmland to pastures
                        \item Required numerous serfs; land often given to tenants who were evicted for various reasons (led to gangs roaming countryside)
                        \item Government attempted to correct with legal changes with little effect
                        \item Enclosure movement removed land from cultivation, reducing food for growing population
                    \end{itemize}
                    \item England also known for significant overpopulation
                \end{itemize}
                \item More significant motivation was rising class of merchant capitalists seeking to expand foreign trade
                \begin{itemize}
                    \item Merchants had transformed England's export economy, increasing the importance of goods other than wool (like cloth)
                    \item Chartered companies emerged (like EIC) with major interests in foreign trade
                    \item Central ideal: \textbf{mercantilism}
                    \begin{itemize}
                        \item Believed that nation as a whole was central to economy, not singular individual
                        \item To boost everyone, nation's total wealth should grow at expense of others
                        \item Boosted position of merchant capitalists (believed to have been helping the nation, requiriing government assistance)
                        \item Led to greater competition between nations 
                    \end{itemize}
                    \item When cloth market collapsed, many began to seek overseas trade, with many scholars arguing for colonization (specifically clergyman Richard Hakluyt)
                    \begin{itemize}
                        \item Would help to alleviate poverty/unemployment
                        \item Commerce would allow England to acquire new, unique products
                    \end{itemize}
                \end{itemize} 
            \end{itemize}
            \textbf{One of the greatest incentives for colonization was to escape the ills of the Old World, known for the collapse of the countryside. However, the growing mercantilist movement was another major stimuli, encouraging colonization to achieve a competitive edge against other nations.}
        }
        \cornell{What was the religious incentive for British colonization?}{\begin{itemize}
            \item Rooted in major reformations throughout Europe
            \begin{itemize}
                \item Protestant Reformation: Martin Luther challenged basic practices of Catholicism due to increased wealth of church
                \begin{itemize}
                    \item Won support of ordinary people, initially insisting connection with church until excommunication by pope
                \end{itemize}
                \item Catholic Reformation: Augustinian monk denied belief that salvation colud be achieved through payments to church
                \item French John Calvin emphasized \textbf{predestination}
                \begin{itemize}
                    \item Destiny of all decided before birth, fate predetermined
                    \item Goal of life to strive to learn destiny in order to determine chances at salvation
                    \item Served as incentive to serve virtuous lives 
                \end{itemize}
            \end{itemize}
            \item English Reformation very different from Protestant Reformation
            \begin{itemize}
                \item Originated from dispute between pope/king: pope refused to grant divorce to King Henry VIII 
                \begin{itemize}
                    \item Henry proceeded to cut ties with Catholic Church, make himself head of Christianity in England 
                    \item Few other changes made, Protestants remained safe
                \end{itemize}
                \item Catholic daughter Mary, after ascending throne, restored allegiance to Rome
                \begin{itemize}
                    \item Executed Protestants ("Bloody Mary")
                \end{itemize}
                \item Half-sister, Elizabeth, soon ascended throne, again severing ties
                \begin{itemize}
                    \item Many English Christians remained unhappy, continuing allegiance to Rome
                    \item Protestants unhappy with "reformation" as insufficient
                \end{itemize}
                \item Most radical Protestants became "Puritans" for desire to purify church
                \begin{itemize}
                    \item Some were Separatists, going against English 
                    \begin{itemize}
                        \item Separatists most radical in allowance for women as preachers
                    \end{itemize}
                    \item Most Puritans sought to simplify worship, reduce power of bishops (seen as corrupt), local clergy
                    \item After James I succeeded Elizabeth, Puritan discontent skyrocketed
                    \begin{itemize}
                        \item Believed in divine right of kings, antagonizing Puritans with increased taxation (many were businessmen)
                        \item Many sought refuge outside kingdom
                    \end{itemize}
                \end{itemize}
            \end{itemize}
        \end{itemize}
        \textbf{The major religious incentive for colonization was driven by a growing discontent for the Anglican Church, which manifested itself in the Puritan movement. Puritan persecution led many to seek refuge in overseas lands.}}
        \cornell{What was England's first experience with colonization?}{England began colonization in Ireland. \begin{itemize}
            \item English had claimed island for many years; only maintained small settlement around Dublin 
            \item During 1560s/1570s, began to colonize, subduing native population
            \begin{itemize}
                \item Led to many important assumptions which influenced American colonization, including that natives were untameable savages who must be isolated from society 
                \begin{itemize}
                    \item Later encouraged complete separation from natives, building independent "plantations" with no connection to local culture
                \end{itemize}
                \item Began with Humphrey Gilbert (an educated, supposedly civilized man), governor of an Irish district
                \begin{itemize}
                    \item Suppressed rebellions with viciousness
                    \item Often beheaded soldiers after losing a battle
                \end{itemize}
            \end{itemize}
        \end{itemize}
        \textbf{England's colonization of Ireland was signifcant in that it encouraged the English to make numerous assumptions about colonization, most significantly that "natives" were generally inferior and uncultured.}}
        \cornell{What was significant about French and Dutch colonization in America?}{
            \begin{itemize}
                \item Unlike in Ireland, British faced other Europeans driven by mercantilist ideals (like Spanish)
                \item Most significant rivals: French in Quebec
                \begin{itemize}
                    \item Population grew relatively slowly due to lack of motivation within France, banning of Protestants
                    \item Exercised significant influence due to convenient location on coastline, trade with natives
                    \item Focused on forging deep ties with natives through Jesuit religion
                    \item Most significant were \textit{coureurs de bois}, fur traders and trappers with extensive trade networks in wilderness
                    \begin{itemize}
                        \item Fur trade heavily driven by natives: fur traders indirectly supported Algonquins/Hurons
                        \item Partnerships emerged as French integrated into Indian society, oftn by marrying Indian women
                    \end{itemize}
                    \item Fur trade opened opportunities for colonization elsewhere (; alliance with Algonquins allowing for competitive edge)
                    \begin{itemize}
                        \item \textit{Seigneuries}, or agricultural estates, emerged along St. Lawrence
                        \item Formed alliance with Algonquins for edge against British
                        \begin{itemize}
                            \item Brought into conflict with Iroquois, major players in British fur trade 
                        \end{itemize}
                    \end{itemize}
                \end{itemize}
                \item After Dutch won independence from Spain, began to establish North American presence
                \begin{itemize}
                    \item Known as one of leading trading nations in the world
                    \item Traders active in Africa, Asia, Europe, America
                    \begin{itemize}
                        \item Henry Hudson sailed up wide river in New York (Hudson River), encouraging Dutch to lay claim on lands
                    \end{itemize}
                    \item Active fur trade around New York in conjunction w/ Dutch West India Company's trading posts on rivers
                    \begin{itemize}
                        \item Encouraged settlement of region throughout Europe, transporting whole families in search of wealth 
                    \end{itemize}
                    \item Population remained small, leadership weak
                \end{itemize}
            \end{itemize}
            \textbf{Both the French and Dutch participated havily in the fur trade. The French, despite their small population, had an advantage due to their close relationship with th natives and their coastal position. The Dutch, international traders, lay claim around the New York area, leading New Amsterdam for decades.}
        }
        \cornell{What were the first English settlements in the New World?}{\begin{itemize}
            \item First settlment at Jamestown, in Virginia (1607); succeeded numerous failed efforts to colonize throughout America
            \begin{itemize}
                \item England known for mixed feelings about New World, fearing dominance of powerful Spain but growing national sentiment under Elizabeth I 
                \item Sentiment changed whn Sir Francis Drake began to raid Spanish fleets
                \begin{itemize}
                    \item After Philip II of spain united with Portugal, invaded England for supremacy w/ "Spanish Armada"
                    \item Significant 
                \end{itemize}
            \end{itemize}
        \end{itemize}}
    \end{document} 