\documentclass[a4paper]{article}
    \usepackage[T1]{fontenc}
    \usepackage{tcolorbox}
    \usepackage{amsmath}
    \tcbuselibrary{skins}
    
    \usepackage{background}
    \SetBgScale{1}
    \SetBgAngle{0}
    \SetBgColor{red}
    \SetBgContents{\rule[0em]{4pt}{\textheight}}
    \SetBgHshift{-2.3cm}
    \SetBgVshift{0cm}
    \usepackage[margin=2cm]{geometry} 
    
    \makeatletter
    \def\cornell{\@ifnextchar[{\@with}{\@without}}
    \def\@with[#1]#2#3{
    \begin{tcolorbox}[enhanced,colback=gray,colframe=black,fonttitle=\large\bfseries\sffamily,sidebyside=true, nobeforeafter,before=\vfil,after=\vfil,colupper=blue,sidebyside align=top, lefthand width=.3\textwidth,
    opacityframe=0,opacityback=.3,opacitybacktitle=1, opacitytext=1,
    segmentation style={black!55,solid,opacity=0,line width=3pt},
    title=#1
    ]
    \begin{tcolorbox}[colback=red!05,colframe=red!25,sidebyside align=top,
    width=\textwidth,nobeforeafter]#2\end{tcolorbox}%
    \tcblower
    \sffamily
    \begin{tcolorbox}[colback=blue!05,colframe=blue!10,width=\textwidth,nobeforeafter]
    #3
    \end{tcolorbox}
    \end{tcolorbox}
    }
    \def\@without#1#2{
    \begin{tcolorbox}[enhanced,colback=white!15,colframe=white,fonttitle=\bfseries,sidebyside=true, nobeforeafter,before=\vfil,after=\vfil,colupper=blue,sidebyside align=top, lefthand width=.3\textwidth,
    opacityframe=0,opacityback=0,opacitybacktitle=0, opacitytext=1,
    segmentation style={black!55,solid,opacity=0,line width=3pt}
    ]
    
    \begin{tcolorbox}[colback=red!05,colframe=red!25,sidebyside align=top,
    width=\textwidth,nobeforeafter]#1\end{tcolorbox}%
    \tcblower
    \sffamily
    \begin{tcolorbox}[colback=blue!05,colframe=blue!10,width=\textwidth,nobeforeafter]
    #2
    \end{tcolorbox}
    \end{tcolorbox}
    }
    \makeatother

    \parindent=0pt
    \usepackage[normalem]{ulem}

    \newcommand{\chapternumber}{27}
    \newcommand{\chaptertitle}{The Cold War}
    \title{\vspace{-3em}
    \begin{tcolorbox}
    \Huge\sffamily \begin{center} AP US History  \\
    \LARGE Chapter \chapternumber \, - \chaptertitle \\
    \Large Finn Frankis \end{center} 
    \end{tcolorbox}
    \vspace{-3em}
    }
    \date{}
    \author{}
    
    \begin{document}
        \maketitle
        \SetBgContents{\rule[0em]{4pt}{\textheight}}
        \cornell[Key Concepts]{What are this chapter's key concepts?}{\begin{itemize}
            \item \textbf{8.1.I.A} - Postwar tensions $\to$ collapse of Allied alliance betw. U.S. democracies/USSR $\to$ U.S. foreign policy aimed to promote non-Communist nations
            \item \textbf{8.1.I.B} - U.S. feared Communist expansion $\to$ sought to contain Communism, particularly w/ wars in Korea/Vietnam
            \item \textbf{8.1.II.A} - U.S. debated policies over how to expose communists w/in U.S. despite both parties supporting containment of communism
            \item \textbf{8.1.II.C} - U.S. debated benefits of nuclear arsenal, large military, strength of exec. branch  
        \end{itemize}}
        \cornell[Origins of the Cold War]{What factors stimulated the beginning of the Cold War?}{\textbf{The Cold War began with differing post-war visions: the U.S. sought a world ruled by self-determination and no military alliances, while the USSR sought to preserve their colonial holdings. The late 1943 Tehran conference saw some successes, with the US and the USSR both agreeing to further support each other in the war. The Yalta conference, designed to find peace in post-war Europe, saw the successful creation of the United Nations but vague agreements over the future of Poland and Germany.}}
        \cornell{What were the primary sources of tensions between the U.S. and the USSR?}|{\begin{itemize}
            \item U.S., w/ Atlantic Charter, believed in world w/o military alliances or spheres of influence; all ruled through democracy and self-determination (like Wilson's post-WWI ideals)
            \item USSR (and somewhat GB) sought to preserve empires w/ spheres of influence in foreign territories 
            \begin{itemize}
                \item Greatly resembled pre-war makeup of Europe
            \end{itemize}
        \end{itemize}
        \textbf{One of the greatest sources of tension was the fundamental disparity between the post-war vision of the U.S. and the USSR. The U.S., aligning with Wilson's idealistic beliefs, sought a world without military alliance or spheres of influence, with all ruling through self-determination. The USSR (and Britain, to an extent) sought to preserve their overseas empire.}}
        \cornell{What were the primary foreign policy tensions during the war?}{\begin{itemize}
            \item Jan. 1943: Roosevelt/Churchill met in Casablanca w/ Stalin having denied invitation
            \begin{itemize}
                \item Would not accept Stalin's demand to immediately open second front in Europe
                \item Promised to continue fighting until unconditional surrender, guaranteeing that the Soviets would not be left fighting alone 
            \end{itemize}
            \item Nov. 1943: Big Three met together for the first time in Tehran 
            \begin{itemize}
                \item Roosevelt had lost ability to bargain: USSR now on offensive, not needing help of US
                \item Conference largely successful w/ personal relationship betw. Stalin/Roosevelt, agreement that USSR would enter war after tensions subsided in Europe, Anglo-American second front in Europe w/in 6 months
                \item Unable to agree on Poland: Roosevelt/Churchill willing to allow some annexation of Polish territory but did not want installment of communist government 
            \end{itemize}
        \end{itemize}
        \textbf{The January 1943 Casablanca conference furthered tensions in that Roosevelt and Churchill rejected Stalin's demand that they immediately open a second front in Europe. The November 1943 Tehran Conference was relatively successful, with the USSR agreeing to soon enter the war in the Pacific and Roosevelt and Churchill agreeing to open a second front soon. The question of the new Polish government, revealed the beginning of significant tensions over communist ideologies.}}
        \cornell{What was the result of the peace conference at Yalta?}{\begin{itemize}
            \item Feb. 1945: Big Three met at Yalta on Black Sea; Roosevelt promised USSR territory in Japan in exchange for entering Pacific war
            \item Finalized plan for new world organization: \textbf{United Nations} w/ \textbf{General Assembly} representing all members and \textbf{Security Council} representing major powers (US/Britain/France/USSR/China)
            \begin{itemize}
                \item April 25, 1945: UN charter created in conference of 50 nations; US Senate easily ratified in July 
            \end{itemize}
            \item Disagreement remained over Poland: USSR had already installed pro-communist "Lublin" poles; Roosevelt/Churchill demanded place for pro-Western Poles in London 
            \begin{itemize}
                \item Roosevelt sought democratic govt. (which would clearly favor the pro-Western Poles); Stalin agreed to find place in govt., eventually hold elections (not for 50 yrs.)
            \end{itemize}
            \item Future of Germany unclear: Roosevelt wanted reunited Germany while Stalin wanted major reparations $\to$ very unstable agreement
            \begin{itemize}
                \item France, USSR, US, Britain would control \textbf{zone of occupation} based around troop position at the end of the war 
                \item Berlin, despite being w/in Soviet territory, would be divided into four due to importance 
                \item Reunion agreed upon w/o any date set; governmental structure unclear
            \end{itemize}
            \item Yalta accords were very loose and unclear, w/ each power interpreting them to their own liking $\to$ Roosevelt shocked that Stalin failed to follow the principles as he had interpreted them 
            \item Roosevelt died hopeful for change; stroke in April 12, 1945; succeeded by \textbf{Harry S. Truman}
        \end{itemize}
        \textbf{The Yalta accords were successful in their creation of the United Nations; however, disagreement remained over the governmental structure in Poland (democratic vs. communist) and the future of Germany, which was not reunited but instead divided into four parts. In all, they were relatively unclear and offered room for great personal interpretation.}}
    \end{document}