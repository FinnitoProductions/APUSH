\documentclass[a4paper]{article}
    \input{../notesheader.tex}
    \usepackage[normalem]{ulem}

    \newcommand{\chapternumber}{27}
    \newcommand{\chaptertitle}{The Cold War}
    \title{\vspace{-3em}
    \begin{tcolorbox}
    \Huge\sffamily \begin{center} AP US History  \\
    \LARGE Chapter \chapternumber \, - \chaptertitle \\
    \Large Finn Frankis \end{center} 
    \end{tcolorbox}
    \vspace{-3em}
    }
    \date{}
    \author{}
    
    \begin{document}
        \maketitle
        \SetBgContents{\rule[0em]{4pt}{\textheight}}
        \cornell[Key Concepts]{What are this chapter's key concepts?}{\begin{itemize}
            \item \textbf{8.1.I.A} - Postwar tensions $\to$ collapse of Allied alliance betw. U.S. democracies/USSR $\to$ U.S. foreign policy aimed to promote non-Communist nations
            \item \textbf{8.1.I.B} - U.S. feared Communist expansion $\to$ sought to contain Communism, particularly w/ wars in Korea/Vietnam
            \item \textbf{8.1.II.A} - U.S. debated policies over how to expose communists w/in U.S. despite both parties supporting containment of communism
            \item \textbf{8.1.II.C} - U.S. debated benefits of nuclear arsenal, large military, strength of exec. branch  
        \end{itemize}}
        \cornell[Origins of the Cold War]{What factors stimulated the beginning of the Cold War?}{\textbf{The Cold War began with differing post-war visions: the U.S. sought a world ruled by self-determination and no military alliances, while the USSR sought to preserve their colonial holdings. The late 1943 Tehran conference saw some successes, with the US and the USSR both agreeing to further support each other in the war. The Yalta conference, designed to find peace in post-war Europe, saw the successful creation of the United Nations but vague agreements over the future of Poland and Germany.}}
        \cornell{What were the primary sources of tensions between the U.S. and the USSR?}|{\begin{itemize}
            \item U.S., w/ Atlantic Charter, believed in world w/o military alliances or spheres of influence; all ruled through democracy and self-determination (like Wilson's post-WWI ideals)
            \item USSR (and somewhat GB) sought to preserve empires w/ spheres of influence in foreign territories 
            \begin{itemize}
                \item Greatly resembled pre-war makeup of Europe
            \end{itemize}
        \end{itemize}
        \textbf{One of the greatest sources of tension was the fundamental disparity between the post-war vision of the U.S. and the USSR. The U.S., aligning with Wilson's idealistic beliefs, sought a world without military alliance or spheres of influence, with all ruling through self-determination. The USSR (and Britain, to an extent) sought to preserve their overseas empire.}}
        \cornell{What were the primary foreign policy tensions during the war?}{\begin{itemize}
            \item Jan. 1943: Roosevelt/Churchill met in Casablanca w/ Stalin having denied invitation
            \begin{itemize}
                \item Would not accept Stalin's demand to immediately open second front in Europe
                \item Promised to continue fighting until unconditional surrender, guaranteeing that the Soviets would not be left fighting alone 
            \end{itemize}
            \item Nov. 1943: Big Three met together for the first time in Tehran 
            \begin{itemize}
                \item Roosevelt had lost ability to bargain: USSR now on offensive, not needing help of US
                \item Conference largely successful w/ personal relationship betw. Stalin/Roosevelt, agreement that USSR would enter war after tensions subsided in Europe, Anglo-American second front in Europe w/in 6 months
                \item Unable to agree on Poland: Roosevelt/Churchill willing to allow some annexation of Polish territory but did not want installment of communist government 
            \end{itemize}
        \end{itemize}
        \textbf{The January 1943 Casablanca conference furthered tensions in that Roosevelt and Churchill rejected Stalin's demand that they immediately open a second front in Europe. The November 1943 Tehran Conference was relatively successful, with the USSR agreeing to soon enter the war in the Pacific and Roosevelt and Churchill agreeing to open a second front soon. The question of the new Polish government, revealed the beginning of significant tensions over communist ideologies.}}
        \cornell{What was the result of the peace conference at Yalta?}{\begin{itemize}
            \item Feb. 1945: Big Three met at Yalta on Black Sea; Roosevelt promised USSR territory in Japan in exchange for entering Pacific war
            \item Finalized plan for new world organization: \textbf{United Nations} w/ \textbf{General Assembly} representing all members and \textbf{Security Council} representing major powers (US/Britain/France/USSR/China)
            \begin{itemize}
                \item April 25, 1945: UN charter created in conference of 50 nations; US Senate easily ratified in July 
            \end{itemize}
            \item Disagreement remained over Poland: USSR had already installed pro-communist "Lublin" poles; Roosevelt/Churchill demanded place for pro-Western Poles in London 
            \begin{itemize}
                \item Roosevelt sought democratic govt. (which would clearly favor the pro-Western Poles); Stalin agreed to find place in govt., eventually hold elections (not for 50 yrs.)
            \end{itemize}
            \item Future of Germany unclear: Roosevelt wanted reunited Germany while Stalin wanted major reparations $\to$ very unstable agreement
            \begin{itemize}
                \item France, USSR, US, Britain would control \textbf{zone of occupation} based around troop position at the end of the war 
                \item Berlin, despite being w/in Soviet territory, would be divided into four due to importance 
                \item Reunion agreed upon w/o any date set; governmental structure unclear
            \end{itemize}
            \item Yalta accords were very loose and unclear, w/ each power interpreting them to their own liking $\to$ Roosevelt shocked that Stalin failed to follow the principles as he had interpreted them 
            \item Roosevelt died hopeful for change; stroke in April 12, 1945; succeeded by \textbf{Harry S. Truman}
        \end{itemize}
        \textbf{The Yalta accords were successful in their creation of the United Nations; however, disagreement remained over the governmental structure in Poland (democratic vs. communist) and the future of Germany, which was not reunited but instead divided into four parts. In all, they were relatively unclear and offered room for great personal interpretation.}}
        \cornell[The Collapse of the Peace]{How did post-war peace ultimately collapse?}{\textbf{Truman took a far less hopeful policy with the Soviets, believing there was no hope for peace. After China's democratic government became increasingly powerless against the growing communist forces from within, Truman, following an extensive study, resolved that all-out war was the only solution; he thus turned to Japan as a potential powerful ally. He strongly advocated for a doctrine of containment - to ensure that communism would not be further expanded - as well as economic support for non-communist nations as seen in the Marshall Plan. The Cold War also entailed direct preparation for war: like atomic mobilization and as well as new wartime organizations. Another major step was the union of the United States with several other western nations behind NATO, paired with the reunion of western Germany. However, the USSR continued to make strides, with a successful atomic test; China, too, rapidly fell to communuist control, so Truman passed NSC-68 to guarantee that the U.S. would take a leadership role. Some conservatives pushed for a "rollback" policy, feeling containment was far too weak.}}
        \cornell{How did Truman's early attempts to chastise the Soviet Union fail?}{\begin{itemize}
            \item Truman believed USSR inflexible $\to$ immediately criticized violation of Yalta; very little leverage but still sought rigid following of accords
            \item Stalin made some concessions to pro-Westerners in Poland $\to$ U.S. recognized Warsaw govt., hoping noncommunists would replace (did not happen until 1980s)
            \item Germany question remained $\to$ Truman met in Potsdam, Germany w/ Churchill (soon replaced by Attlee) and Stalin
            \begin{itemize}
                \item Truman accepted new Polish-German border
                \item Refused any German reparations from U.S./British/French zones of Germany $\to$ Germany remained divided
            \end{itemize}
        \end{itemize}
        \textbf{Truman did not share Roosevelt's view about the flexibility of the Soviet Union; he thus immediately chastised Stalin for breaking the Yalta accord. Truman reluctantly recognized the new Polish government and accepted the border; however, Potsdam guaranteed that Germany remained divided as Truman refused to allow any German reparations from non-Soviet-held territories.}}
        \cornell{How did China fit into the post-WWII world?}{\begin{itemize}
            \item America believed in independent post-war China; communist struggle w/in China posed great threat
            \begin{itemize}
                \item \textbf{Chiang Kai-shek}, favoring U.S., weak and disliked $\to$ communist armies of \textbf{Mao Zedong} gained significant support
            \end{itemize}
            \item Many Americans wanted third option to take over China; others sought agreement w/ Mao; many, including Truman, continued to support Chiang 
            \begin{itemize}
                \item Civil war betw. Mao/Chiang saw U.S. mainly support Chiang's forces
                \item Truman eventually sent \textbf{George Marshall} to study China, determine best U.S. policy; many Americans sought increased mil. presence 
                \item Marshall believed all-out war was only way to stop communist expansion; both Marshall and Truman refused war $\to$ many Americans greatly angered
            \end{itemize}
            \item Recognizing that China had been effectively taken by communist powers, U.S. turned to reviving Japan as industrial ally
            \begin{itemize}
                \item Lifted any restrictions on development, encouraged econ. growth; unrestricted world replaced by one w/ pro-American sphere of influence 
            \end{itemize}
        \end{itemize}
        \textbf{The U.S. believed strongly in an independent China; however, the slow communist takeover of Chiang's feeble government which George Marshall determined could be mended only by all-out war meant that the U.S. turned to the growing Japan.}}
        \cornell{How did Truman begin a doctrine of containment?}{\begin{itemize}
            \item By end of 1945, Atlantic Charter essentially impossible $\to$ Truman created policy of \textbf{containment} to limit Soviet expansion
            \begin{itemize}
                \item Responded to events in 1946: Stalin attempted to win control over Medit. sea lanes, communist govt. threatened western govt. in Greece
                \item British announced unable to provide any further assistance
            \end{itemize}
            \item Truman created firm policy drawing from American diplomat \textbf{George F. Kennan}, consistently advocating for containment of communist govts.
            \item \textbf{Truman Doctrine} pushed for U.S. to help all free peoples not desiring communist govt. $\to$ helped Turkey, Greek govt. while creating formal foreign policy
        \end{itemize}
        \textbf{Realizing that the goals of the Atlantic Charter to live in complete harmony, Truman advocated a policy of containment aiming to limit the amount of influence Stalin's communist government could have.}}
        \cornell{What was the Marshall Plan?}{\begin{itemize}
            \item Containment policy hoped to reconstruct Western Europe for humanitarian sake, prevention of Europe becoming drain on U.S., desire for Euro. market; believed critical to strengthen pro-western govts. of Western Europe
            \item June 1947: \textbf{George C. Marshall} (sec. of state) planned to economically assist all European nations (including USSR)
            \begin{itemize}
                \item USSR/Eastern allies immediately rejected; 16 other European nations partook 
                \item Some domestic opposition but vanished after Czechoslovakia experienced coup leading to communist govt.
                \item Congress created \textbf{Economic Cooperation Administration} to administer Marshall Plan; sent \$12b in aid to Europe, creating revival of industrial production and limiting communist control
            \end{itemize}
        \end{itemize}
        \textbf{The Marshall Plan sought to provide economic assistance to European nations, pouring 12 billion dollars into improving the governments of Western European nations at the expense of communist ideals.}}
        \cornell{How did the U.S. mobilize for war on the home front?}{\begin{itemize}
            \item Clear commitment to containment in 1947/1948 w/ Truman strengthening mil. power to near-wartime levels; new draft created based around \textbf{Selective Service System}
            \item US continued atomic research, emphasizing nuclear weapons; \textbf{Atomic Energy Commission} created in 1946 to supervise mil. research
            \begin{itemize}
                \item Truman approved \textbf{hydrogen bomb} in 1950
            \end{itemize}
            \item \textbf{National Security Act of 1947} created several organizations
            \begin{itemize}
                \item \textbf{Department of Defense} combined War and Navy Departments
                \item \textbf{National Security Council} (NSC) out of White House to oversee foreign/mil. policy
                \item \textbf{Central Intelligence Agency} (CIA) to replace Office of Strategic Services, gathering info. through open/covert methods
                \item Ultimately expanded executive powers
            \end{itemize}
        \end{itemize}
        \textbf{To preapre for war, Truman strengthened military power with a peacetime draft, continued atomic research with a hydrogen bomb, and passed the National Security Act of 1947 to promote knowledge-gathering organizations and wartime organizations.}}
        \cornell{How did several Western European nations unite behind NATO?}{\begin{itemize}
            \item U.S. pushing to strengthen Western European capabilities
            \item Truman agreed w/ England/France to merge zones of Germany into West German republic in 1948 (including their portions of Berlin)
            \begin{itemize}
                \item Stalin blockaded western parts of Berlin in retaliation; Truman would not back down
                \item Western portions of Berlin supplied with food/fuel for over 10 months, sustaining pop. of 2m; represented commitment to resist communism 
                \item Stalin lifted blockade in 1949 $\to$ Germany officially divided into two nations 
            \end{itemize}
            \item Western European nations united behind \textbf{North Atlantic Treaty Organization} (NATO) w/ military agreement to retaliate against other members
            \begin{itemize}
                \item Imposed permanent standing military force to defend against Soviet invasion
                \item Encouraged USSR to create \textbf{Warsaw Pact} w/ Eastern Europe
            \end{itemize}
        \end{itemize}
        \textbf{After Truman facilitated the reunion of western Germany against Stalin's demands, he fought for this belief and guaranteed that Germany would be divided into two distinct nations. The alliance between Western European nations was formalized in NATO, imposing a permanent military force in Western Europe and guaranteeing that an attack on one member would be seen as an attack on all.}}
        \cornell{How did the US reevaluate their Cold War policy?}{\begin{itemize}
            \item 1949: Cold War sent in new direction after USSR announced first explosion of atomic weapon earlier than expected, Chiang's govt. collapsed in China (moving to Taiwan) to be replaced w/ communist control 
            \begin{itemize}
                \item Prioritized revival of Japan by spurring industry, eventually removing forces
            \end{itemize}
            \item Truman called for review of foreign policy w/ \textbf{NSC-68} report released, shifting overall focus of U.S. policy
            \begin{itemize}
                \item Previous containment policy had called upon U.S. to help other nations, emphasizing certain areas of greater importance 
                \item NSC-68's stressed that U.S had to take a direct role in leading the noncommunist world, stopping communist expansion \underline{anywhere} regardless of relevance to U.S.
            \end{itemize}
        \end{itemize}
        \textbf{As communism continued to expand with the USSR exploding an atomic weapon and Chiang Kai-shek's government collapsing in China, Truman published the NSC-68 report, stressing that the U.S. had to take a role of leadership in all parts of the world to ensure that communist would never spread.}}
        \cornell{How did some U.S. conservatives oppose containment?}{\begin{itemize}
            \item Relatively widespread support to containment from both parties; some Americans on left felt too aggressive: peace would have been possible; many others felt too weak, seeing as appeasement
            \item Anticommunist \textbf{John Birch Society}, led by Robert Welch, who believed the U.S. govt. was engaging in direct collaboration with the Soviets 
            \begin{itemize}
                \item Stressed that govt. filled with \underline{treason}, communist Americans who were directly attacking nation 
                \item Drew upon similarities betw. both govts.: led by corrupt/wealthy bankers, politicians
                \item Felt UN and other international groups were origins of treason 
                \item Many Americans saw as radical; most shared great fear
            \end{itemize}
            \item \textbf{John Foster Dulles}, future secretary of state, felt containment was weak and U.S. had left several Eastern European nations behind
            \begin{itemize}
                \item Advocated for "rollback," or a push against the borders of communism despite the risk of war 
            \end{itemize}
        \end{itemize}
        \textbf{Although containment was widely supported, several important Americans strongly opposed: the John Birch Society, led by Robert Welch, stressed that the treasonous U.S. government was collaborating with the Soviets through organizations like the UN and major banks. John Foster Dulles, too, advocated a stronger "rollback" policy to push back against the borders of communism.}}
        \cornell[American Society and Politics After the War]{How did American society shift following the war?}{\textbf{Following the war, the U.S. economy remained prosperous, even seeing some inflation due to an extreme increase in spending and very high consumer demand. As a result, some labor unrest emerged, with many strikes and suffering for minorities and women. Truman sought a "Fair Deal" following the war, mimicking the New Deal by seeking expanded Social Security, raised minimum wage, and better employment practices and infrastructure. The 1946 midterms, making both houses of Congress led by Republicans, however, generated great opposition to the New Deal and the Fair Deal. Truman won the election of 1948, much to the surprise of many Americans, primarily for his passionate speeches and support of farmers and African Americans; although the Democratic Congress would only support some aspects of the Fair Deal, he worked himself to propel the civil rights movement. Finally, Americans faced a great cultural conflict between appreciating nuclear weapons for their technological potential as energy sources or fearing the potential of nuclear war.}}
        \cornell{How did the US recover from the war?}{\begin{itemize}
            \item War ended far earlier than predicted w/ atomic bombs $\to$ rapid process of reconversion 
            \item Peace expected to bring Depression unemployment levels w/ returning soldiers $\to$ market flooded; econ. in fact fine for several reasons
            \begin{itemize}
                \item Govt. spending dropped
                \item \$35b in war contracts ended but compenstaed by increased consumer demand for new goods (workers spent savings)
                \item \$6b tax cut added money into circulation
                \item \textbf{Servicemen's Readjustment Act of 1944} (GI Bill of Rights) gave econ./educational assistance to veterans
                \item Overall increase in spending, money entering/circulating throughout econ.
            \end{itemize}
            \item Consumer demand $\to$ two yrs. of significant inflation
            \begin{itemize}
                \item Aggravated after Truman vetoed OPA $\to$ no more price controls for goods, w/ inflation soaring to 25\% before he signed previous bill back into law
            \end{itemize}
            \item Significant labor unrest due to inflation: major strikes in automobile, electrical, steel industries; major mine strike in April 1946
            \begin{itemize}
                \item Growing fear that econ. would end w/o coal supplies $\to$ Truman seized mines and pushed mine owners to grant most of union demands $\to$ major railroad shutdown w/ two unions walking out on strike
                \item Truman used military pressure to push workers back into work
            \end{itemize}
            \item Women and minorities (who had enjoyed expanded rights) lost jobs to white males
            \begin{itemize}
                \item Some women left voluntarily to return to past lives 
                \item Up to 80\% wanted to continue working, particularly women w/ postwar inflation, growing divorce rate $\to$ women responsible for futures but excluded from industry $\to$ moved to service sector
            \end{itemize}
        \end{itemize}
        \textbf{With peace coming far earlier than expected, a rapid process of reconversion began. The economy never faced significant issues of unemployment: several factors encouraged an increase in spending and thus more money circulated through the economy. Significant inflation emerged due to high consumer demand and Truman's temporary veto of the OPA. This inflation stimulated labor unrest, with several strikes emerging in the automobile, electrical, and steel industries as well as the mining industry. Finally, several minorities and women who had received expanded rights due to the war lost their jobs after the war concluded.}}
        \cornell{What was Truman's "Fair Deal"?}{\begin{itemize}
            \item Shortly after war, Truman called for \textbf{Fair Deal}
            \begin{itemize}
                \item Sought to expand Social Security benefits, raise minimum wage (to 65 cents), promote aggressive federal spending
                \item Hoped to improve employment practices through permanent \textbf{Fair Employment Practices Act} 
                \item Aimed to improve infrastructure through public works planning, scientific research
            \end{itemize}
            \item Fair Deal suffered similar conservative opposition to New Deal, w/ Republicans winning both houses of Congress in Nov. 1946
            \begin{itemize}
                \item New Congress attacked New Deal, lifting price controls, deregulating economy $\to$ prices rose, inflation grew w/ \textbf{Robert Taft} attacking equality
                \item Congress vetoed most Fair Deal plans, barring $\uparrow$ education/Social Security/public works projects; cut tax rates for high-income families
                \item Wagner Act of 1935, giving new powers to unions, overriden with \textbf{Labor-Management Relations Act of 1947} (\textbf{Taft-Hartley Act}) making closed shop - workplace requiring to be a union member before hiring - illegal 
                \begin{itemize}
                    \item States even allowed to pass individual laws banning union shops (requiring workers to join a union after being hired)
                    \item Workers saw as "slave labor bill"; Truman vetoed, but House/Senate overruled rapidly
                    \item Most directly attacked minorities/women/Southern workers, relying on unions for econ. success
                \end{itemize}
            \end{itemize}
        \end{itemize}
        \textbf{Truman's progressive Fair Deal sought to expand Social Security benefits, raise the minimum wage, promote spending and fair employment practicies, and advance the nation's infrastructure and science. Truman's goals were stifled by a Republican House and Senate following the 1946 midterms; they attacked the New Deal as well as most of the Fair Deal plans, most notably by overriding the Wagner Act by barring a closed shop, dealing a significant blow to unions.}} 
        \cornell{What was the result of the election of 1948?}{\begin{itemize}
            \item Truman felt American public had not abandoned New Deal $\to$ policies based around traditional Democratic reform (although overturned by Congress, helped to build campaign)
            \item Significant personal unpopularity, seen as inept $\to$ several abandoned party
            \begin{itemize}
                \item Southern conservatives opposing civil rights bill formed \textbf{States' Rights Party} led by Strom Thurmond
                \item Leftmost wing formed \textbf{Progressive Party} led by Wallace, feeling Truman's policies were ineffectual and stance against USSR too confrontational
                \item After Truman nominated, some liberals formed \textbf{Americans for Democratic Action}, attempting to push Eisenhower (war hero) to contest Truman's nomination; Eisenhower refused
            \end{itemize}
            \item Republicans nominated \textbf{Thomas E. Dewey} of NY, known for austere/dignified campaign aiming never to antagonize anyone
            \item Truman remained confident, attacking Dewey and the Republican Congress; attempted to prove point by calling special session of Congress in July to enact his policies (had no effect)
            \begin{itemize}
                \item Travelled throughout nation, giving impassioned speeches supporting Taft-Hartley repeal, farmer support, black protection
                \item Succeeded, winning election w/ popular vote as well as signif. electoral vote; Dems. regained both houses
            \end{itemize}
        \end{itemize}
        \textbf{Truman remained confident in himself despite the Democratic loss of two major factions, with southern conservatives forming the States' Rights Party and the leftmost Democrats forming the Progressive Party. Facing Thomas E. Dewey from New York, known for a careful, unantagonistic campaign, Truman ultimately won the election through his impassioned speeches against Dewey and supporting black/farmer rights.}}
        \cornell{How did Truman's victory stimulate the revival of the Fair Deal?}{\begin{itemize}
            \item New Congress remained relatively opposed to Fair Deal, but some victories like minimum wage, Social Security expansion, \textbf{National Housing Act of 1949} to support low-income housing (relatively unsucessful)
            \item National health insurance/educational aid/civil rights saw \underline{no progress}: unable to make lynching federal crime, expand black voting rights, abolish poll tax, create new FERA to prevent workplace hiring discrimination
            \item Personally battled racial discrim. by banning it when hiring govt. employees, limiting segregation in army, giving Justice Department greater involvement in court battles against discrim.
            \item \textit{Shelley v. Kraemer} case in 1948 saw courts banned from enforcing private covenants to ban Afr. Americans from neighborhoods
        \end{itemize}
        \textbf{Truman's victory saw some Fair Deal policies apss, like minimum wage, Social Security expansion, and low-income housing. However, civil rights remained continually stifled, with lynching and black voting and workplace discrimination continuing. Regardless, Truman personally attacked discrimination within the government, the army, and the courts. The Supreme Court, too, banned courts from barring African Americans from neighborhoods in \textit{Shelley v. Kraemer}.}}
        \cornell{How did the USSR and the U.S. push the world into a time of great nuclear expansion?}{\begin{itemize}
            \item Americans feared nuclear potential while treating w/ expectation to drive forward U.S. technology (significant conflict betw. fear and potential)
            \item \textbf{\textit{Film noir}} filmmaking originating in France saw dark lighting, individuals in lonely world with potential for vast destruction
            \item \textit{The Twilight Zone} described world after nuclear war; comics showed superheroes saving world from nuclear destruction
            \item Constant preparation for nuclear attack in schools/office buildings; radios tested emergency broadcasting; fallout shelters began to emerge
            \item Nuclear fear paired w/ significant U.S. prosperity through tech. development (notably in nuclear power, showing that this scary technology also had potential)
            \begin{itemize}
                \item Most believed that nuclear weaponry would be successful over long-term
            \end{itemize}
        \end{itemize}
        \textbf{Americans were faced with great conflict: most were extremely scared of the potential for nuclear war but also recognized the technological potential of nuclear advancements, notably in nuclear power. The fear of nuclear war was manifested in the post-apocalyptic world of \textit{film noir} films as well as in shows and comic books; it became a more pressing issue in the lives of many Americans with frequent drills and the construction of fallout shelters.}}
    \end{document}