\documentclass[a4paper]{article}
    \usepackage[T1]{fontenc}
    \usepackage{tcolorbox}
    \usepackage{amsmath}
    \tcbuselibrary{skins}
    
    \usepackage{background}
    \SetBgScale{1}
    \SetBgAngle{0}
    \SetBgColor{red}
    \SetBgContents{\rule[0em]{4pt}{\textheight}}
    \SetBgHshift{-2.3cm}
    \SetBgVshift{0cm}
    \usepackage[margin=2cm]{geometry} 
    
    \makeatletter
    \def\cornell{\@ifnextchar[{\@with}{\@without}}
    \def\@with[#1]#2#3{
    \begin{tcolorbox}[enhanced,colback=gray,colframe=black,fonttitle=\large\bfseries\sffamily,sidebyside=true, nobeforeafter,before=\vfil,after=\vfil,colupper=blue,sidebyside align=top, lefthand width=.3\textwidth,
    opacityframe=0,opacityback=.3,opacitybacktitle=1, opacitytext=1,
    segmentation style={black!55,solid,opacity=0,line width=3pt},
    title=#1
    ]
    \begin{tcolorbox}[colback=red!05,colframe=red!25,sidebyside align=top,
    width=\textwidth,nobeforeafter]#2\end{tcolorbox}%
    \tcblower
    \sffamily
    \begin{tcolorbox}[colback=blue!05,colframe=blue!10,width=\textwidth,nobeforeafter]
    #3
    \end{tcolorbox}
    \end{tcolorbox}
    }
    \def\@without#1#2{
    \begin{tcolorbox}[enhanced,colback=white!15,colframe=white,fonttitle=\bfseries,sidebyside=true, nobeforeafter,before=\vfil,after=\vfil,colupper=blue,sidebyside align=top, lefthand width=.3\textwidth,
    opacityframe=0,opacityback=0,opacitybacktitle=0, opacitytext=1,
    segmentation style={black!55,solid,opacity=0,line width=3pt}
    ]
    
    \begin{tcolorbox}[colback=red!05,colframe=red!25,sidebyside align=top,
    width=\textwidth,nobeforeafter]#1\end{tcolorbox}%
    \tcblower
    \sffamily
    \begin{tcolorbox}[colback=blue!05,colframe=blue!10,width=\textwidth,nobeforeafter]
    #2
    \end{tcolorbox}
    \end{tcolorbox}
    }
    \makeatother

    \parindent=0pt
    \usepackage[normalem]{ulem}

    \newcommand{\chapternumber}{28}
    \newcommand{\chaptertitle}{The Affluent Society}
    \title{\vspace{-3em}
    \begin{tcolorbox}
    \Huge\sffamily \begin{center} AP US History  \\
    \LARGE Chapter \chapternumber \, - \chaptertitle \\
    \Large Finn Frankis \end{center} 
    \end{tcolorbox}
    \vspace{-3em}
    }
    \date{}
    \author{}
    
    \begin{document}
        \maketitle
        \SetBgContents{\rule[0em]{4pt}{\textheight}}
        \cornell[Key Concepts]{What are this chapter's key concepts?}{\begin{itemize}
            \item \textbf{8.1.I.D} - Decolonization after the war $\to$ nationalist movements in Asia/Africa/Middle East $\to$ new alliances formed on both sides of Cold War; some non-aligned
            \item \textbf{8.1.II.C} - Debates continued over practicality of nuclear arsenal, power of executive branch, military-industrial complex
            \item \textbf{8.2.I.A} - Post-WWII, $\uparrow$ civil rights movement combatting racial discrimination w/ law, direct action, non-violent protest
            \item \textbf{8.2.I.B} - Three branches of fed. govt. enforced army desegration, \textit{Brown v. Board of Education}, Civil Rights Act of 1964 to promote racial equality
            \item \textbf{8.2.I.C} - Resistance $\to$ $\downarrow$ desegregation w/ social and political unrest; civil rights activists debated effectiveness of nonviolence
            \item \textbf{8.2.II.C} - Poverty remained prominent despite overall affluence
            \item \textbf{8.3.I.A} - Private sector, fed. spending, baby boom, tech. dev. $\to$ $\uparrow$ econ.
            \item \textbf{8.3.I.B} - Higher education oppos. new tech. $\to$ social mobility, migration of middle class to suburbs, many others to South/West; Sun Belt became more prominent
            \item \textbf{8.3.II.A} - Mass culture more homogeneous $\to$ challenges to conformity
        \end{itemize}}
        \cornell[The "Economic Miracle"]{How did the U.S. economy expand rapidly?}{\textbf{In the 1950s and the early 1960s, government spending, an increased birth rate, and suburban growth for the middle class were all sources of economic growth. As a result of government spending and migration, the West grew rapidly as an economic force. The U.S. economic structure changed, too, favoring Keynesian economics, or the belief that the government could regulate the economy by changing the flow of money through spending and taxes rather than by regulating private industry. Labor became more consolidated, too; these new corporations resented strikes, making many concessions to unions to prevent them from occurring. The merger of the AFL and the CIO led to an overall resurgence of unions as a force, though membership remained relatively stagnant and corruption within unions became prominent.}}
        \cornell{What were the primary sources of economic expansion?}{\begin{itemize}
            \item 1945-1960: $\uparrow$ GDP, $\downarrow$ unemployment, stable inflation
            \item Govt. spending continued throughout period after Depression w/ public school funding, veterans' benefits, welfare, road development, mil. spending
            \begin{itemize}
                \item Korean War greatly stimulated growth; period after saw relative decline in growth due to reduced armaments
            \end{itemize}
            \item Birth rate increased significantly in \textbf{baby boom} w/ population $\uparrow$ by 20\% 
            \begin{itemize}
                \item Led to $\uparrow$ consumer demand, econ. growth
            \end{itemize}
            \item Suburbs expanded $\to$ growth in private cars, housing industry, road construction
            \item Econ. grew 10x as fast as pop. w/ avg. American having signif. more purchasing power; highest standard of living in world
        \end{itemize}
        \textbf{Government spending in schools, veterans, welfare, roads, and especially military paired with an increased birth rate leading to growing consumer demand and the expansion of suburbs leading to booms in road development and the housing industry meant that the GDP increased, unemployment decreased, and the economy grew rapidly, contributing to an improved standard of living.}}
        \cornell{How did the West expand as an economic force?}{\begin{itemize}
            \item Pre-war, West had assisted Eastern industry, supplying agricultural goods / raw materials 
            \item By 1960s, some parts had become prosperous in their own right 
            \begin{itemize}
                \item Govt. spending during war $\to$ expanded infrastructure (dams/highways/power), military growth
                \item Automobile expansion $\to$ need for oil $\to$ TX/CO oil fields grew w/ cities in Dallas/Houston/Denver
                \item State govts. focused on education w/ UT and UC systems 
                \item Warm, dry climates attracted many
            \end{itemize}
        \end{itemize}
        \textbf{Primarily as a result of WWII, the West transformed from a force reliant on the East for demand into an independently prosperous region. Government spending during the war, automobile expansion, university deveopment, as well as climate, allowed the region to develop and the population to surge.}}
        \cornell{How did the economic structure of the U.S. change?}{\begin{itemize}
            \item Discovery of power of U.S. economy $\to$ increased confidence in capitalism
            \item Belief that Keynesian econ. allowed govt. to regulate without disrupting private sector became more prominent
            \begin{itemize}
                \item \textbf{John Maynard Keynes} had emphasized in 1920s that govt. could vary flow of spending, taxation, and overall currency output to prevent recession and inflation 
                \item Confirmed by successful govt. policies during Great Depression
                \item "New economics" officially accepted in 1963 w/ Kennedy's proposed tax cut; Lyndon B. Johnson finally passed, w/ result confirming theory
            \end{itemize}
            \item Poverty reformers believed new way forward not redistribution but instead continued growth for all of society to raise standard of living uniformly
        \end{itemize}
        \textbf{Keynesian economic beliefs - the belief that the government could regulate the economy without intervention in the private sector but instead by varying the flow of spending and taxation - took a greater foothold in U.S. policy in the 1960s with Kennedy and Johnson's support. Furthermore, the idea of redistribution became superceded by the belief that all of society should expand.}}
        \cornell{How did labor structure change during the 1950s?}{\begin{itemize}
            \item More and more corporate mergers $\to$ smaller number of corporations dominated than ever before
            \begin{itemize}
                \item Promoted by defense spending: govt. mainly provided armament contracts to largest population
                \item Mechanized agriculture $\to$ agri. workforce, family farms declined w/ corporations owning more and more farms
            \end{itemize}
            \item Corporations reluctant to support strikes $\to$ concessions to unions
            \begin{itemize}
                \item \textit{ex}: 1948 saw Walter Reuther of U.S. Automobile Workers obtain contract from GM to allow standard of living to match consumer price index; promised wages evem during layoffs
                \item Labor unions formed \textbf{postwar contract} w/ steel, auto., other unionized industries seeing increase in wages/benefits; unions agreed to ignore other issues, limiting strikes
            \end{itemize}
            \item Econ. success of 1950s $\to$ reunification of labor movement w/ AFL and CIO (\textbf{Congress of Industrial Organization}) merging to form \textbf{AFL-CIO}
            \begin{itemize}
                \item Led by \textbf{George Meany}; some tensions w/ CIO correctly fearing dominance of AFL w/in merger and AFL fearing radical past of CIO
            \end{itemize}
            \item Corruption accompanied econ. growth
            \begin{itemize}
                \item \textbf{Teamsters Union} under congressional investigation due to misappropriation of funds under David Beck; successor, Jimmy Hoffa, eventually convicted of tax evasion
                \item United Mine Workers under John Lewis faced many scandals; successor convicted of complicity in murder of rival
            \end{itemize}
            \item Unorganized labor made little progress w/ union membership stable due to transition from blue-collar to white-collar, obstacles to organization like Taft-Hartley Act  
        \end{itemize}
        \textbf{During the 1950s, corporations grew further consolidated; comcomitant to this was increased concessions to unions to prevent strikes from occurring. The merger of the AFL and the CIO further developed the labor union; however, corruption grew simultaneously. Furthermore, unorganized labor made relatively little progress with stable union membership.}}
    \end{document}