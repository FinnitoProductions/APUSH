\documentclass[a4paper]{article}
    \usepackage[T1]{fontenc}
    \usepackage{tcolorbox}
    \usepackage{amsmath}
    \tcbuselibrary{skins}
    
    \usepackage{background}
    \SetBgScale{1}
    \SetBgAngle{0}
    \SetBgColor{red}
    \SetBgContents{\rule[0em]{4pt}{\textheight}}
    \SetBgHshift{-2.3cm}
    \SetBgVshift{0cm}
    \usepackage[margin=2cm]{geometry} 
    
    \makeatletter
    \def\cornell{\@ifnextchar[{\@with}{\@without}}
    \def\@with[#1]#2#3{
    \begin{tcolorbox}[enhanced,colback=gray,colframe=black,fonttitle=\large\bfseries\sffamily,sidebyside=true, nobeforeafter,before=\vfil,after=\vfil,colupper=blue,sidebyside align=top, lefthand width=.3\textwidth,
    opacityframe=0,opacityback=.3,opacitybacktitle=1, opacitytext=1,
    segmentation style={black!55,solid,opacity=0,line width=3pt},
    title=#1
    ]
    \begin{tcolorbox}[colback=red!05,colframe=red!25,sidebyside align=top,
    width=\textwidth,nobeforeafter]#2\end{tcolorbox}%
    \tcblower
    \sffamily
    \begin{tcolorbox}[colback=blue!05,colframe=blue!10,width=\textwidth,nobeforeafter]
    #3
    \end{tcolorbox}
    \end{tcolorbox}
    }
    \def\@without#1#2{
    \begin{tcolorbox}[enhanced,colback=white!15,colframe=white,fonttitle=\bfseries,sidebyside=true, nobeforeafter,before=\vfil,after=\vfil,colupper=blue,sidebyside align=top, lefthand width=.3\textwidth,
    opacityframe=0,opacityback=0,opacitybacktitle=0, opacitytext=1,
    segmentation style={black!55,solid,opacity=0,line width=3pt}
    ]
    
    \begin{tcolorbox}[colback=red!05,colframe=red!25,sidebyside align=top,
    width=\textwidth,nobeforeafter]#1\end{tcolorbox}%
    \tcblower
    \sffamily
    \begin{tcolorbox}[colback=blue!05,colframe=blue!10,width=\textwidth,nobeforeafter]
    #2
    \end{tcolorbox}
    \end{tcolorbox}
    }
    \makeatother

    \parindent=0pt
    \usepackage[normalem]{ulem}

    \newcommand{\chapternumber}{28}
    \newcommand{\chaptertitle}{The Affluent Society}
    \title{\vspace{-3em}
    \begin{tcolorbox}
    \Huge\sffamily \begin{center} AP US History  \\
    \LARGE Chapter \chapternumber \, - \chaptertitle \\
    \Large Finn Frankis \end{center} 
    \end{tcolorbox}
    \vspace{-3em}
    }
    \date{}
    \author{}
    
    \begin{document}
        \maketitle
        \SetBgContents{\rule[0em]{4pt}{\textheight}}
        \cornell[Key Concepts]{What are this chapter's key concepts?}{\begin{itemize}
            \item \textbf{8.1.I.D} - Decolonization after the war $\to$ nationalist movements in Asia/Africa/Middle East $\to$ new alliances formed on both sides of Cold War; some non-aligned
            \item \textbf{8.1.II.C} - Debates continued over practicality of nuclear arsenal, power of executive branch, military-industrial complex
            \item \textbf{8.2.I.A} - Post-WWII, $\uparrow$ civil rights movement combatting racial discrimination w/ law, direct action, non-violent protest
            \item \textbf{8.2.I.B} - Three branches of fed. govt. enforced army desegration, \textit{Brown v. Board of Education}, Civil Rights Act of 1964 to promote racial equality
            \item \textbf{8.2.I.C} - Resistance $\to$ $\downarrow$ desegregation w/ social and political unrest; civil rights activists debated effectiveness of nonviolence
            \item \textbf{8.2.II.C} - Poverty remained prominent despite overall affluence
            \item \textbf{8.3.I.A} - Private sector, fed. spending, baby boom, tech. dev. $\to$ $\uparrow$ econ.
            \item \textbf{8.3.I.B} - Higher education oppos. new tech. $\to$ social mobility, migration of middle class to suburbs, many others to South/West; Sun Belt became more prominent
            \item \textbf{8.3.II.A} - Mass culture more homogeneous $\to$ challenges to conformity
        \end{itemize}}
    \end{document}