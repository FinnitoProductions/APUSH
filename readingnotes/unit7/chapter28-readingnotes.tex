\documentclass[a4paper]{article}
    \usepackage[T1]{fontenc}
    \usepackage{tcolorbox}
    \usepackage{amsmath}
    \tcbuselibrary{skins}
    
    \usepackage{background}
    \SetBgScale{1}
    \SetBgAngle{0}
    \SetBgColor{red}
    \SetBgContents{\rule[0em]{4pt}{\textheight}}
    \SetBgHshift{-2.3cm}
    \SetBgVshift{0cm}
    \usepackage[margin=2cm]{geometry} 
    
    \makeatletter
    \def\cornell{\@ifnextchar[{\@with}{\@without}}
    \def\@with[#1]#2#3{
    \begin{tcolorbox}[enhanced,colback=gray,colframe=black,fonttitle=\large\bfseries\sffamily,sidebyside=true, nobeforeafter,before=\vfil,after=\vfil,colupper=blue,sidebyside align=top, lefthand width=.3\textwidth,
    opacityframe=0,opacityback=.3,opacitybacktitle=1, opacitytext=1,
    segmentation style={black!55,solid,opacity=0,line width=3pt},
    title=#1
    ]
    \begin{tcolorbox}[colback=red!05,colframe=red!25,sidebyside align=top,
    width=\textwidth,nobeforeafter]#2\end{tcolorbox}%
    \tcblower
    \sffamily
    \begin{tcolorbox}[colback=blue!05,colframe=blue!10,width=\textwidth,nobeforeafter]
    #3
    \end{tcolorbox}
    \end{tcolorbox}
    }
    \def\@without#1#2{
    \begin{tcolorbox}[enhanced,colback=white!15,colframe=white,fonttitle=\bfseries,sidebyside=true, nobeforeafter,before=\vfil,after=\vfil,colupper=blue,sidebyside align=top, lefthand width=.3\textwidth,
    opacityframe=0,opacityback=0,opacitybacktitle=0, opacitytext=1,
    segmentation style={black!55,solid,opacity=0,line width=3pt}
    ]
    
    \begin{tcolorbox}[colback=red!05,colframe=red!25,sidebyside align=top,
    width=\textwidth,nobeforeafter]#1\end{tcolorbox}%
    \tcblower
    \sffamily
    \begin{tcolorbox}[colback=blue!05,colframe=blue!10,width=\textwidth,nobeforeafter]
    #2
    \end{tcolorbox}
    \end{tcolorbox}
    }
    \makeatother

    \parindent=0pt
    \usepackage[normalem]{ulem}

    \newcommand{\chapternumber}{28}
    \newcommand{\chaptertitle}{The Affluent Society}
    \title{\vspace{-3em}
    \begin{tcolorbox}
    \Huge\sffamily \begin{center} AP US History  \\
    \LARGE Chapter \chapternumber \, - \chaptertitle \\
    \Large Finn Frankis \end{center} 
    \end{tcolorbox}
    \vspace{-3em}
    }
    \date{}
    \author{}
    
    \begin{document}
        \maketitle
        \SetBgContents{\rule[0em]{4pt}{\textheight}}
        \cornell[Key Concepts]{What are this chapter's key concepts?}{\begin{itemize}
            \item \textbf{8.1.I.D} - Decolonization after the war $\to$ nationalist movements in Asia/Africa/Middle East $\to$ new alliances formed on both sides of Cold War; some non-aligned
            \item \textbf{8.1.II.C} - Debates continued over practicality of nuclear arsenal, power of executive branch, military-industrial complex
            \item \textbf{8.2.I.A} - Post-WWII, $\uparrow$ civil rights movement combatting racial discrimination w/ law, direct action, non-violent protest
            \item \textbf{8.2.I.B} - Three branches of fed. govt. enforced army desegration, \textit{Brown v. Board of Education}, Civil Rights Act of 1964 to promote racial equality
            \item \textbf{8.2.I.C} - Resistance $\to$ $\downarrow$ desegregation w/ social and political unrest; civil rights activists debated effectiveness of nonviolence
            \item \textbf{8.2.II.C} - Poverty remained prominent despite overall affluence
            \item \textbf{8.3.I.A} - Private sector, fed. spending, baby boom, tech. dev. $\to$ $\uparrow$ econ.
            \item \textbf{8.3.I.B} - Higher education oppos. new tech. $\to$ social mobility, migration of middle class to suburbs, many others to South/West; Sun Belt became more prominent
            \item \textbf{8.3.II.A} - Mass culture more homogeneous $\to$ challenges to conformity
        \end{itemize}}
        \cornell[The "Economic Miracle"]{How did the U.S. economy expand rapidly?}{\textbf{In the 1950s and the early 1960s, government spending, an increased birth rate, and suburban growth for the middle class were all sources of economic growth. As a result of government spending and migration, the West grew rapidly as an economic force. The U.S. economic structure changed, too, favoring Keynesian economics, or the belief that the government could regulate the economy by changing the flow of money through spending and taxes rather than by regulating private industry. Labor became more consolidated, too; these new corporations resented strikes, making many concessions to unions to prevent them from occurring. The merger of the AFL and the CIO led to an overall resurgence of unions as a force, though membership remained relatively stagnant and corruption within unions became prominent.}}
        \cornell{What were the primary sources of economic expansion?}{\begin{itemize}
            \item 1945-1960: $\uparrow$ GDP, $\downarrow$ unemployment, stable inflation
            \item Govt. spending continued throughout period after Depression w/ public school funding, veterans' benefits, welfare, road development, mil. spending
            \begin{itemize}
                \item Korean War greatly stimulated growth; period after saw relative decline in growth due to reduced armaments
            \end{itemize}
            \item Birth rate increased significantly in \textbf{baby boom} w/ population $\uparrow$ by 20\% 
            \begin{itemize}
                \item Led to $\uparrow$ consumer demand, econ. growth
            \end{itemize}
            \item Suburbs expanded $\to$ growth in private cars, housing industry, road construction
            \item Econ. grew 10x as fast as pop. w/ avg. American having signif. more purchasing power; highest standard of living in world
        \end{itemize}
        \textbf{Government spending in schools, veterans, welfare, roads, and especially military paired with an increased birth rate leading to growing consumer demand and the expansion of suburbs leading to booms in road development and the housing industry meant that the GDP increased, unemployment decreased, and the economy grew rapidly, contributing to an improved standard of living.}}
        \cornell{How did the West expand as an economic force?}{\begin{itemize}
            \item Pre-war, West had assisted Eastern industry, supplying agricultural goods / raw materials 
            \item By 1960s, some parts had become prosperous in their own right 
            \begin{itemize}
                \item Govt. spending during war $\to$ expanded infrastructure (dams/highways/power), military growth
                \item Automobile expansion $\to$ need for oil $\to$ TX/CO oil fields grew w/ cities in Dallas/Houston/Denver
                \item State govts. focused on education w/ UT and UC systems 
                \item Warm, dry climates attracted many
            \end{itemize}
        \end{itemize}
        \textbf{Primarily as a result of WWII, the West transformed from a force reliant on the East for demand into an independently prosperous region. Government spending during the war, automobile expansion, university deveopment, as well as climate, allowed the region to develop and the population to surge.}}
        \cornell{How did the economic structure of the U.S. change?}{\begin{itemize}
            \item Discovery of power of U.S. economy $\to$ increased confidence in capitalism
            \item Belief that Keynesian econ. allowed govt. to regulate without disrupting private sector became more prominent
            \begin{itemize}
                \item \textbf{John Maynard Keynes} had emphasized in 1920s that govt. could vary flow of spending, taxation, and overall currency output to prevent recession and inflation 
                \item Confirmed by successful govt. policies during Great Depression
                \item "New economics" officially accepted in 1963 w/ Kennedy's proposed tax cut; Lyndon B. Johnson finally passed, w/ result confirming theory
            \end{itemize}
            \item Poverty reformers believed new way forward not redistribution but instead continued growth for all of society to raise standard of living uniformly
        \end{itemize}
        \textbf{Keynesian economic beliefs - the belief that the government could regulate the economy without intervention in the private sector but instead by varying the flow of spending and taxation - took a greater foothold in U.S. policy in the 1960s with Kennedy and Johnson's support. Furthermore, the idea of redistribution became superceded by the belief that all of society should expand.}}
        \cornell{How did labor structure change during the 1950s?}{\begin{itemize}
            \item More and more corporate mergers $\to$ smaller number of corporations dominated than ever before
            \begin{itemize}
                \item Promoted by defense spending: govt. mainly provided armament contracts to largest population
                \item Mechanized agriculture $\to$ agri. workforce, family farms declined w/ corporations owning more and more farms
            \end{itemize}
            \item Corporations reluctant to support strikes $\to$ concessions to unions
            \begin{itemize}
                \item \textit{ex}: 1948 saw Walter Reuther of U.S. Automobile Workers obtain contract from GM to allow standard of living to match consumer price index; promised wages evem during layoffs
                \item Labor unions formed \textbf{postwar contract} w/ steel, auto., other unionized industries seeing increase in wages/benefits; unions agreed to ignore other issues, limiting strikes
            \end{itemize}
            \item Econ. success of 1950s $\to$ reunification of labor movement w/ AFL and CIO (\textbf{Congress of Industrial Organization}) merging to form \textbf{AFL-CIO}
            \begin{itemize}
                \item Led by \textbf{George Meany}; some tensions w/ CIO correctly fearing dominance of AFL w/in merger and AFL fearing radical past of CIO
            \end{itemize}
            \item Corruption accompanied econ. growth
            \begin{itemize}
                \item \textbf{Teamsters Union} under congressional investigation due to misappropriation of funds under David Beck; successor, Jimmy Hoffa, eventually convicted of tax evasion
                \item United Mine Workers under John Lewis faced many scandals; successor convicted of complicity in murder of rival
            \end{itemize}
            \item Unorganized labor made little progress w/ union membership stable due to transition from blue-collar to white-collar, obstacles to organization like Taft-Hartley Act  
        \end{itemize}
        \textbf{During the 1950s, corporations grew further consolidated; comcomitant to this was increased concessions to unions to prevent strikes from occurring. The merger of the AFL and the CIO further developed the labor union; however, corruption grew simultaneously. Furthermore, unorganized labor made relatively little progress with stable union membership.}}
        \cornell[The Explosion of Science and Technology]{How did science and technology develop rapidly during the 1950s?}{\textbf{In the 1950s, several medical breakthroughs were made, with antibiotics like sulfa drugs and penicillin and vaccines for smallpox, typhoid fever, tetanus, and, most notably, polio (created by Jonas Salk).  Pesticides greatly assisted agriculture, notably DDT. The electronic development industry skyrocketed after the creation of the transistor, ultimately allowing for the creation of an advanced computer capable of doing more than simple math. As part of the Cold War, weapons technology developed, too, with the first dropping of the H-Bomb encouraging long-range missile tests; furthermore, the space race against the USSR led to major strides in space travel, with NASA putting a person on the moon and developing the space shuttle.}}
        \cornell{What were the most significant medical breakthroughs?}{\begin{itemize}
            \item Medical science saw rapid development w/ antibacterial drugs to fight infections 
            \item Antibiotics originated w/ \textbf{Louis Pasteur} in 1870s; revealed viruses could be defeated by regular bacteria
            \begin{itemize}
                \item Using antibacterial devices to combat disease began in 1930s w/ \textbf{sulfa drugs} from \textbf{sulfanilamide} for streptococcal blood infections developed at rapid rate; treated major cause of death
            \end{itemize}
            \item 1928: \textbf{Alexander Fleming} discovered properties penicillin; discovered to target bacterial disease at Oxford in 1941
            \begin{itemize}
                \item U.S. labs developed penicillin for mass production; soon widely available to doctors/hospitals
                \item New antibiotics developed $\to$  bacterial illnesses among most successfully treated
            \end{itemize}
            \item Immunization w/ smallpox vaccine developed by \textbf{Edward Jenner}; typhoid vaccine by \textbf{Almorth Wright}; tetanus vaccine widespread during/before WWI 
            \begin{itemize}
                \item Viruses difficult to prevent $\to$ slow progress against vaccines; studies began only in 1930s w/ discovery that viruses could be grown in cultures
                \item Gradually created virus unable to stimulate disease but capable of triggering antibodies
                \item Yellow fever, influenza viruses appeared in first half of 20th century
                \item Polio vaccine by \textbf{Jonas Salk} extremely effective 
                \begin{itemize}
                    \item Polio killed/crippled thousands of children/adults
                    \item Became free to public after 1955; oral vaccine by Albert Sabin after 1960
                \end{itemize}
            \end{itemize}
            \item Death rate overall decreased, particularly among younger children
        \end{itemize}
        \textbf{Antibacterial drugs became particularly effective in medical science, starting with sulfa drugs to combat blood infection; penicillin was another major, but slow-developing, breakthrough. Treating viral diseases through immunizations developed far more slowly; it started with the smallpox vaccine, but the polio vaccine was a major breakthrough, which was offered free to the public by the government and saved thousands of children and adults from death.}}
        \cornell{How did pesticides impact agriculture in the 1950s?}{\begin{itemize}
            \item Scientists developed \textbf{pesticides} to prevent crops from destruction by insects while protecting humans from crop-carrying disease
            \item 1939: Paul Muller discovered \textbf{DDT}, seemingly harmless to humans but killer to insects $\to$ sent to Africa during WWII to protect soldiers from malaria
            \begin{itemize}
                \item Used on large scale during/following the war primarily to prevent typhus/malaria outbreaks
                \item Only later realized to be toxic
            \end{itemize}
        \end{itemize}
        \textbf{Pesticides developed for targeting insects; among the most effective was DDT, developed shortly before WWII. It became critical to preventing malaria and typhus outbreaks; only later was it realized to be toxic.}}
        \cornell{How did electronic technology expand after the war?}{\begin{itemize}
            \item 1940s researchers created televisions on commercial level functional over large ranges; color television created in late 1950s
            \item 1948: Bell Labs created first \textbf{transistor} to amplify electrical signals; effective for small-scale. making smaller radios, TVs, weapons, satellites
            \item \textbf{Integrated circuits} used transistors, resistors, diodes, etc. all in one device; allowed for complex electronic devices
        \end{itemize}
        \textbf{After the war, television technology developed rapidly, converting to color by the late 1950s. The transistor, released by Bell Labs in 1948, allowed devices to be miniaturized and integrated circuits to be created, paving the way for advanced devices like the computer.}}
        \cornell{What was the earliest computer technology to develop?}{\begin{itemize}
            \item Pre-1950s: computers for mathematical tasks like breaking military codes; first commercial functions in 1950s
            \item \textbf{UNIVAC} created for U.S. Bureau of the Census to store alphabetical/numerical info., processing this data more rapid than predecessor (\textbf{ENIAC}) 
            \begin{itemize}
                \item Developers, Remington Rand Company, used to predict election results in 1952 for publicity $\to$ many people learned about computer for first time
            \end{itemize}
            \item UNIVAC had little marketing success, but \textbf{IBM} found wide market in mid-1950s; soon became leader
        \end{itemize}
        \textbf{Computers first became used for commercial purposes in the 1950s. The first commercial computer, UNIVAC, was intended for census bureau data processing; however, it failed to gain significant commercial traction. IBM began to market computers more effectively, soon dominating the market.}}
        \cornell{How did weapons technology develop in the 1950s?}{\begin{itemize}
            \item 1952: U.S. detonated first \textbf{hydrogen bomb}, notable for using fusion explosions (far more effective)
            \item H-Bomb development pushed for rocket/missile technology for both U.S. and USSR; U.S. relied on emigrating German scientists
            \begin{itemize}
                \item Early research by Air Force w/ some early successes in \underline{short-range missiles}
                \item More experiments in early 1950s: struggled w/ long range due to difficulty of finding stable fuel 
                \item 1958: solid fuel paired w/ guidance system $\to$ creation of \textbf{Minuteman} type of missiles, w/ several thousand miles of travel
                \item Some development in underwater missiles w/ \textbf{Polaris} fired from underwater
            \end{itemize}
        \end{itemize}
        \textbf{After the U.S. dropped the first powerful hydrogen bomb, missile and bomb research expanded; the Air Force struggled to develop a long-range missile for several years. However, the creation of the Minuteman type of missile in 1958 using solid fuel was an effective long-range device.}}
        \cornell{What characterized the growth of the space program in the U.S.?}{\begin{itemize}
            \item Space program began after USSR revealed having launched \textbf{\textit{Sputnik}}, satellite, into space in 1957
            \begin{itemize}
                \item U.S. took as personal indication of failure $\to$ significant development of scientific education in school, research labs 
                \item By Jan. 1958, U.S. launched first satellite 
            \end{itemize}
            \item Creation of manned space program by new agency, \textbf{National Aeronautics and Space Administration} (NASA) w/ selection of first astronauts
            \begin{itemize}
                \item Alan Shepard first American launched into space in \textbf{Mercury Project}; unable to fully orbit the Earth and preceded by Soviet Yuri Gagarin 
                \item John Glenn first to orbit the Earth in 1962
                \item \textbf{Apollo program} intended to send men to moon; several setbacks w/ many deaths, but \textbf{Neil Armstrong}, Buzz Aldrin, Michael Collins travelling in orbit around moon in 1969
                \begin{itemize}
                    \item Armstrong/Aldrin sent off smaller capsule to land and walk on moon; missions followed until 1972 but government funding soon cut w/ $\downarrow$ public enthusiasm
                \end{itemize}
            \end{itemize}
            \item Goal no longer to reach distant planets; instead shifted to develop \textbf{space shuttle} for easily launching into space and arriving w/ similar ease to aircraft
            \begin{itemize}
                \item First launched in 1982; Deadly explosion of \textit{Challenger} put things on hold after 1986
                \item Shuttle launched \textbf{Hubble Space Telescope}, deposited
            \end{itemize}
            \item Space program impacted several other industries 
        \end{itemize}
        \textbf{After the USSR launched the first satellite into space in 1957, the space race began. The U.S. quickly launched their own satellite, then created NASA, culminating in the Apollo program, which made American Neil Armstrong the first person to set foot on the moon. After the moon landing, however, the goal shifted more toward developing a versatile space shuttle allowing for easy takeoff and landing; several successful launches have been made to this day.}}
        \cornell[People of Plenty]{How did Americans become accustomed to a life of consumerism and wealth?}{\textbf{A growing consumer culture paired with road development allowed middle-class Americans to move further out to the suburbs in the 1950s; the suburbs focused primarily on the family while reinforcing traditional gender roles, but the amount of married women in the workplace in fact increased. Television developed significantly during this period; travel to national parks, too, stimulated the preservationist movement. Youth culture became increasingly rebellious, with "beats" criticizing the middle-class; rock 'n' roll became an extremely popular genre, led by Elvis Presley.}}
        \cornell{How did the middle class embrace a consumer culture?}{\begin{itemize}
            \item Middle class became increasingly centered around consumer goods like dishwashers, TVs, stereos due to increased prosperity, credit cards $\to$ purchasing power
            \item $\uparrow$ marketing of goods $\to$ several consumer crazes of goods
            \begin{itemize}
                \item Late 1950s saw great popularity of hula hoop
                \item Walt Disney's \textit{The Mickey Mouse Club} immensely successful, sparking popularity of Disneyland
            \end{itemize}
        \end{itemize}
        \textbf{The middle class became increasingly centered on consumer culture based around crazes for several major goods due to an overall increase in purchasing power.}}
        \cornell{How did road development occur in the 1950s?}{\begin{itemize}
            \item Disneyland great indicator of popularity of automobile due to surrounding highways, large parking lot
            \item \textbf{Federal Highway Act of 1956}, by setting off \$25b for highway development, created links betw. every major city, reducing travel time and encouraging trucking
            \begin{itemize}
                \item Railroad industry declined
            \end{itemize}
            \item Economic activities moved more into rural/suburban regions due to cheap land $\to$ downtowns declined somewhat, w/ "edge cities"
            \item Families able to move into homes farther than work $\to$ enjoyed larger houses in suburbs w/ amenities like garages, pools, swing sets in suburban society 
            \item Motels became popular, w/ \textbf{Holiday Inn} opening on highway betw. Memphis and Nashville and soon spreading, along with drive-in theaters
        \end{itemize}
        \textbf{Road development was extremely rapid in the 1950s, with several highways created to link major cities. This pushed the railroad industry to decline and pushed economic and residential activities into the suburbs due to cheaper land. Motels, too, became very popular.}}
        \cornell{What was the overall impact of the growing suburban popularity?}{\begin{itemize}
            \item Suburbanization encouraged home-building, w/ \textbf{William Levitt}, most popular postwar suburban deeloper, using mass-production to create NY suburb
            \begin{itemize}
                \item Created \textbf{Levittown} w/ many two-bedroom identical houses selling for very cheap prices; young couples purchased, often w/ help of GI Bill
            \end{itemize}
            \item Suburban growth largely due to greater emphasis on family w/ birth rate increasing; friendships easier to form in suburbs, especially for mothers to meet other nonworking mothers; most suburbs primarily white 
            \item Suburbs never uniform: some, like Levittown, for lower-middle class; others far more affluent
        \end{itemize}
        \textbf{Suburbanization led to a growth in the real estate industry, particularly with the large Levittown, a New York suburb. The growth of suburbs was primarily due to a prioritization of families and making new friendships; several other whites flocked to the suburbs to escape racial tensions.}}
        \cornell{How did the suburbs change family structure?}{\begin{itemize}
            \item For men, suburbs $\to$ sharp division betw. work/personal; for women, suburbs $\to$ isolation from workplace w/ 1950s seeing increased prejudice against women working at paid jobs
            \item Benjamin Spock's \textit{Baby and Child Care} offered child-centered approach to parenting, focusing on parenting as process of allowing child to learn/grow
            \begin{itemize}
                \item Initially presented fathers as having very little influence
            \end{itemize}
            \item $\uparrow$ consumer demand $\to$ many women needed to work to sustain lifestyle $\to$ number of married women outside the home increased despite pressure
        \end{itemize}
        \textbf{The suburbs meant that women were increasingly pressured to remain at home and focus on child-rearing; however, consumer demand meant that many women had to find a sustainable job to ensure that the middle class lifestyle could be sustained.}}
        \cornell{How did television develop in the 1950s?}{\begin{itemize}
            \item Broadcast experiments in 1920s; commercial television by WWII w/ rapid growth: 40m sets by 1957 
            \item Emerged as offshoot of radio industry w/ three major networks having started as radio companies 
            \begin{itemize}
                \item Industry driven by advertising w/ most content driven by needs of advertisers: shows often named after advertisers (like \textbf{Chrysler Playhouse})
            \end{itemize}
            \item Television had replaced newspapers, magazines, radios by late 1950s as most important spreader of information
            \begin{itemize}
                \item Televised athletics $\to$ college/professional sports critical to entertainment in U.S. 
                \item Entertainment programs replaced movies/radio
            \end{itemize}
            \item Programming created image of U.S. life as white, middle-class, suburban; reinforced gender roles emphasized the father as dominant in the household
            \begin{itemize}
                \item Some portrayed urban working-class families; some were childless and explorative; some showed Afr. Americans
                \item Stress on suburban lifestyle alienated lower classes unable to indulge in such pleasures
            \end{itemize}
        \end{itemize}
        \textbf{Television became widespread in the 1950s, with the programs driven by advertisements rapidly replacing newspapers, magazines, and radios. Sports as well as entertainment programs portraying white, middle-class, suburban life became the most popular programs.}}
        \cornell{How was travel a significant activity for Americans?}{\begin{itemize}
            \item Paid vacation began in 1920s; vacation travel common only in postwar years w/ highway system, new wealth $\to$ cars 
            \item National parks saw greatest surge in hiking, camping, fishing, hunting; some simply sought to explore the wilderness 
            \item Preservationist movement seen in \textbf{Echo Park}
            \begin{itemize}
                \item Echo Park large valley betw. UT/CO; govt. decided to build dam in early 1950s
                \item Environmental movement dormant after failure in Hetch Hetchy Valley dam; soon spoke up
                \item \textbf{Bernard DeVoto}, champion of West, created sesnsational piece describing dam as ruining parks; Sierra Club returned w/ aggressive leader \textbf{David Brower}
                \begin{itemize}
                    \item Succeeded: Congress cancelled dam plans in 1956
                \end{itemize}
            \end{itemize}
        \end{itemize}
        \textbf{Travel to national parks became extremely common in the years after WWII. The American environmental movement resurged after the U.S. planned to build a dam at Echo Park; the Sierra Club returned and several writers spoke against the building of the dam, with Congress finally reverting.}}
        \cornell{What was the overall labor structure of the 1950s?}{\begin{itemize}
            \item White-collar workers outnumbered laborers in 1950s w/ many more in rigid corporate industries; industrial workers faced large bureaucracies 
            \item Americans believed more skills required to work in large orgs. $\to$ educational philosophy changed
            \begin{itemize}
                \item More focus on science, math, languages; universities expanded curriculum for specialized skills
            \end{itemize}
            \item Social debate about bureaucratic life
            \begin{itemize}
                \item William H. Whyte Jr. created book describing mentality of corporate lifestyle - self-reliance was to be replaced by cooperation
                \item David Riesman stressed that man reliant on his own values was replaced by man seeking to be approved by community
            \end{itemize}
            \item Novelists like Saul Bellow, and notably J.D. Salinger w /\textit{The Catcher in the Rye} believed in impersonality of society
        \end{itemize}
        \textbf{In the 1950s, corporate workers began to outnumber laborers; as a result, the educational system shifted to focus more on science, math, and languages, while universities focused on specialized skills. Many writers lamented the loss of individuality and self-reliance being replaced by a mentality driven by community approval.}}
        \cornell{What was the state of youth culture in the 1950s?}{\begin{itemize}
            \item Critics of middle-class society were young poets/writers/authors known as \textbf{beats}; criticized conformity of U.S. life, useless politics
            \begin{itemize}
                \item \textbf{Allen Ginsberg}, \textbf{Jack Kerouac} were two notable \textbf{Beat Generation} authors
            \end{itemize}
            \item Beats driven by restlessness motivated by prosperity w/ limitless opportunities
            \item Fear of \textbf{juvenile delinquency} w/ increased criminality of youth; though crime did not increase dramatically, significant fear reflected in works 
            \item Ordinary middle-class society saw teen rebelliousness, obsession with fast cars, sex; popularity of moody teen James Dean 
        \end{itemize}
        \textbf{The "beats" were harsh critics of middle-class society; they were driven by constant restlessness. Although many feared increased youth crime, it did not increase significantly. This rebelliousness extended even into middle-class teans, with a focus on fast cars and teen sex.}}
        \cornell{How did rock 'n' roll grow as a music genre?}{\begin{itemize}
            \item Popularity of \textbf{rock 'n' roll} transformed U.S. society, driven by \textbf{Elvis Presley}
            \begin{itemize}
                \item Presley symbolized goal to push borders of conventional w/ good looks, rebellious style, sexuality of music/performances $\to$ young Americans loved him
                \item Drew from black rhythm/blues; began to appeal to white youth 
                \item Some roots in country music, gospel, jazz
            \end{itemize}
            \item White audiences remained hesitant to accept black musicians; still $\uparrow$ popularity of bands and singers but never rivaling Presley
            \item Radio/television very significant for popularity of rock w/ radio stations beginning to play recorded music; radio announcers known as "disk jockeys" created programs for rock fans; some music showcased on TV
            \begin{itemize}
                \item Encouraged sale of records, jukeboxes
                \item Payments (known as \textbf{payola}) made to radio stations/TV programs to showcase unknown artists' songs
            \end{itemize}
        \end{itemize}
        \textbf{Rock 'n' roll, drawing heavily from black rhythm and blues, was popularized for white audiences by Elvis Presley. It was also encouraged by the radio and television industry, with DJs creating song mixes and encouraging the purchase of records and jukeboxes.}}
        \cornell[The "Other America"]{What was the state of Americans outside of the middle-class circle of abundance?}{\textbf{Although economic expansion reduced poverty, several Americans, notably the elderly, African Americans, Hispanics, and Native Americans, remained in extreme poverty. Farmers, too, became increasingly poor: as farm prices collapsed, many entered a state of desperate poverty. Finally, the inner cities saw great poverty particularly in their African American and Hispanic inhabitants; they struggled to find jobs and suffered frmo racial discrimination.}}
        \cornell{Which Americans were extremely impoverished during the 1950s?}{\begin{itemize}
            \item Economic expansion $\to$ $\downarrow$ poverty, but never eliminated; in 1960, more than a fifth of U.S. families below the poverty line
            \begin{itemize}
                \item Most experienced intermittent/temporary poverty while trying to find a job
                \item 20\% of poor saw constant, irreparable poverty, including elderly, Afr. Americans, Hispanics
                \item Native Americans were poorest group in country, becoming even more impoverished than on reservations
            \end{itemize}
            \item Worst poverty seemed unaffected by econ. growth
        \end{itemize}
        \textbf{Although poverty decreased in the 1950s, "hard-core" poverty, with 20\% of the poor seeing constant, irreparable poverty, remained widespread, notably for African Americans, Hispanics, the elderly, and especially Native Americans.}}
        \cornell{What was the state of rural poverty in the 1950s?}{\begin{itemize}
            \item Rural Americans on margins of affluent society w/ farm pop. continually shrinking
            \begin{itemize}
                \item Surpluses $\to$ farm prices collapsing; even those able to survive lost income due to rising consumer goods 
            \end{itemize}
            \item Not all farmers poor (some wealthy landowners); most extremely impoverished, like black sharecroppers/tenant farmers 
            \begin{itemize}
                \item Migrant farmworkers in West/Southwest lived in terrible circumstances
                \item Rural regions w/o commercial agriculture saw dangerous conditions, notably in regions where coal industry began to falter
            \end{itemize}
        \end{itemize}
        \textbf{In the 1950s, the farm population continually shrank as farm income became reduced due to surpluses. In all, the farm industry generated a large impoverished population, notably black sharecroppers and tenant farmers, migrant farmworkers in the West and Southwest, as well as farmers isolated from commercial agriculture in more remote regions.}}
        \cornell{What were the inner cities?}{\begin{itemize}
            \item White families moved to suburbs $\to$ downtown region sbecame increasingly impoverished, often inhabited by Afr. Americans migrating into cities in large numbers, as well as Mexicans/Puerto Ricans 
            \begin{itemize}
                \item Southern U.S. cities like San Antonio, Houston, LA saw surge of Mex.-Amer. pop. 
            \end{itemize}
            \item Most inner-city communities remained extremely poor
            \begin{itemize}
                \item Some argued because migrants were victims of past: work habits/values/family not adapted to industrial city
                \item Others felt crime of inner city created "culture of poverty"
                \item Many feel that $\downarrow$ blue-collar jobs, discrimination against minorities in schooling, overall racism $\to$ struggled to sustain lifestyle
                \begin{itemize}
                    \item Employers began to move factories away from inner city into suburbs/small cities due to lower labor costs
                    \item Remaining factories saw increased automation
                \end{itemize}
            \end{itemize}
            \item Poverty of urban cities responded to w/ \textbf{urban renewal}, or goal to tear down worst buildings and restore to create new housing; oftentimes just as bad as ones before teardown
        \end{itemize}
        \textbf{As more and more white families moved into the suburbs, the inner cities became populated primarily by impoverished minority communities. Inner city inhabitants typically struggled to accrue wealth due to reduced blue-collar jobs reduced by automation, discrimination in schooling and hiring, as well as simply a "culture of poverty." Efforts at urban renewal were typically unsuccessful at creating long-term impact.}}
        \cornell[The Rise of the Civil Rights Movement]{How did a fight for civil rights begin in the U.S.?}{\textbf{Starting with the \textit{Brown} decision ending the enforcement of "separate but equal," several tangible changes to U.S. civil rights began to appear: despite resistance, the federal government was generally on board for gradual desegregation within the armed forces and federal employment. The Montgomery bus boycott led to desegregation in all U.S. public transit. The civil rights movement was rapidly pushed by WWII, an empowered black middle class, and television.}}
        \cornell{What was the \textit{Brown} decision?}{\begin{itemize}
            \item \textit{Plessy v. Ferguson} allowing "separate but equal" rejected w/ 1954 \textit{Brown v. Board of Education of Topeka}
            \begin{itemize}
                \item Culmination of several decades of work by NAACP lawyers, who had been continually filing justified legal challenges, slowly working into heart of system
                \item \textit{Brown} decision described Afr. American girl forced to travel several miles to segregated school despite living next door to white elementary school
                \item Rather than solely analyzing law, court looked at psychological effects and ultimately ruled against segregated schools, w/ \textbf{Earl Warren} speaking out against them
                \item Issued \textit{Brown} II in the following year to create rooles to end segregation in schools; no clear timetable 
            \end{itemize}
            \item Particularly in South, "massive resistance" emerged w/ some school districts ignoring; many Congress members came together w/ "manifesto" to denounce \textit{Brown} decision
            \begin{itemize}
                \item Some schools created "placement laws" to place students based on ability/social behavior (blatant segregation) but Court would not declare unconstitutional
                \item By fall of 1957, less than a quarter of Southern schools had even started desegregation
                \begin{itemize}
                    \item Some white parents placed children in white "segregation academies"; state governments often funded new schools
                \end{itemize}
                \item Eisenhower unwilling to commit to enforcement, but after girl in Little Rock, AR banned from entering desegregated school by white mob, Eisenhower sent AR Nat. Guard to ensure order
            \end{itemize}
        \end{itemize}
        \textbf{The \textit{Brown} decision, ruling against "separate but equal" for schools, marked the culmination of years of efforts by African American reformers. In the South, significant resistance emerged toward the decision, causing the process of desegregation to take far longer than necessary; Eisenhower, though hesitant, finally enforced the decision in Little Rock, Arkansas, ensuring that an African American girl would not be mobbed on her way to school by sending the National Guard.}}
        \cornell{How did the movement expand with the \textit{Brown} decision?}{\begin{itemize}
            \item Rosa Parks launched spontaneous resistance in Montgomery, AL, refusing to give up seat to white man; quickly arrested
            \begin{itemize}
                \item Black leaders responded by boycotting bus system
                \item Black citizens' groups had been planning boycott for long time, w/ carpools arranged betw. black workers; many others forced to walk
                \item Boycott placed econ. pressure on bus system as well as downtown merchants less easily accessible
                \item 1956 Supreme Court decision banned segregation in public transportation 
            \end{itemize}
            \item Bus boycott allowed prominent leader \textbf{Martin Luther King Jr.} to rise to prominence
            \begin{itemize}
                \item Doctrine based on \textbf{nonviolence} by finding moral highground; urged constant peaceful demonstrations even in the face of attack
                \item Led \textbf{Southern Christian Leadership Conference}, interracial group, believing in fighting hate with love 
            \end{itemize}
            \item Change in other areas: \textbf{Jackie Robinson} first Afr. American to play in MLB; Eisenhower successfully desegregated armed forces and signed civil rights act in 1957 to protect Afr. Americans seeking to vote
            \begin{itemize}
                \item Represented federal committment to "Second Reconstruction" w/ federal offices desegregated
            \end{itemize}
        \end{itemize}
        \textbf{The Montgomery, AL bus boycott, stimulated after African American Rosa Parks was arrested for refusing to give up her seat to a white man, created economic pressure leading to a Supreme Court decision in 1956 banning segregation on public transit. MLK was able to rise to power, too, with his doctrine of nonviolence becoming increasingly popular; progressions occurred in other areas, too, with Jackie Robinson being the first African American admitted to the MLB and Eisenhower making changes in the armed forces and federal workforce.}}
        \cornell{What caused the civil rights movement to grow so significantly in the 1950s?}{\begin{itemize}
            \item WWII $\to$ black men/women gained broader worldview by working in factories
            \item Urban black middle class after the war generated by educated leaders of black communities with a far greater stake in society felt empowered to advance society by coming together
            \item Television stimulated reform by reminding African Americans of the differences between their lives and that of surburban whites
            \item Cold War pushed white Americans to make changes: hoped to be seen as model nation to world; political mobilization of northern blacks in the Democratic Party meant politicians had to consider; labor unions w/ black memberships supported movement
        \end{itemize}
        \textbf{The civil rights movement was inspired by WWII, an increasingly empowered black middle class, television giving African Americans a glimpse into a life they currently could not have, as well as the Cold War and political and labor shifts.}}
        \cornell[Eisenhower Republicanism]{What characterized Eisenhower's brand of Republicanism?}{\textbf{Eisenhower surrounded himself with business leaders supporting Keynesian economics; he sought to limit federal acitvities in favor of private enterprise. However, he refused to dismantle the New Deal, extending Social Security and expanding public infrastructure, most notably highways. Finally, his administration saw the demise of McCarthy, finally being portrayed as a bully and receiving public condemnation from the Senate.}}
        \cornell{What factors motivated Eisenhower's staffing decisions?}{\begin{itemize}
            \item Admin. staffed w/ men from 1920s Republican staffers: business community; most corporate leaders, however, began to see Keynesian welfare state as most beneficial option
            \item Eisenhower's cabinet consisisted of wealthy corporate lawyers/business executives
            \begin{itemize}
                \item Charles Wilson, president of GM, stressed "what was good for our country was good for General Motors"
            \end{itemize}
            \item Goal to limit federal impact, encourage private enterprise, encouraging private development of resources; eliminated Truman price controls, social service, reduced fed. expenditure, ending with budget surplus
        \end{itemize}
        \textbf{Eisenhower surrounded himself with members from the business community who supported Keynesian economics; most of his staffers stressed that their decisions would be good for both the country and their companies. Eisenhower sought to reduce federal activities in favor of private enterprise and development; as a result, he ended with a budget surplus.}}
        \cornell{How did Eisenhower's policies support a welfare state?}{\begin{itemize}
            \item Refused to dismantle New Deal, extending Social Security system to 10m more people; unemployment comp. to 4m more; minimum wage increased from 75c to \$1 
            \item \textbf{Federal Highway Act of 1956} created ten-year project for interstate highways; funded by highway "trust fund," where revenues would come from fuel taxes
            \item Eisenhower won by landslide again in 1956 against Adlai Stevenson despite major heart
        \end{itemize}
        \textbf{Eisenhower refused to dismantle the New Deal, extending Social Security, unemployment compensation, and the minimum wage; he also created a project to expand interstate highways. He won the 1956 election by a landslide.}}
        \cornell{How did McCarthyism decline?}{\begin{itemize}
            \item Early years of administration refused to discourage anticommunism; popular opposition against McCarthy $\to$ Eisenhower made changes
            \begin{itemize}
                \item During first year of Eisenhower admin., McCarthy operated without punishment
                \item In January 1954, after attacking Sec. of Army Robert Stevens, Army-McCarthy hearings began
                \begin{itemize}
                    \item Portrayed him as outright bully, villain, and buffoon to public w/ Senate voting to condemn actions
                \end{itemize}
            \end{itemize}
        \end{itemize}
        \textbf{Although the first year of the Eisenhower administration saw no condemnation to anticommunism, the tone shifted as public opinion began to turn against McCarthy. After he attacked the renowned Secretary of the Army Robert Stevens, hearings began which portrayed him as a bully. The Senate agreed to publicly condemn him.}}
        \cornell[Eisenhower; Dulles; The Cold War]{How did Eisenhower and his secretary of state address the Cold War?}{\textbf{Dulles, Eisenhower's secretary of state, believed in massive retaliation to enforce U.S. desires through economic power. After a conflict emerged in Vietnam between French colonial rule and communist rebels, the U.S. initially refused to intervene. Several other crises emerged in the Middle East, notably Israel and Egypt, as well as in Latin America, shutting down a communist government in Guatemala and severing ties with a communist government in Cuba. Eisenhower ultimately failed to reduce tensions with the USSR: the Hungarian Revolution further aggravated the situation and the USSR's discovery of a U.S. spy plane flying over their territory ruined any chance at reconciliation.}}
        \cornell{How did Eisenhower's secretary of state play a critical role in Cold War politics?}{\begin{itemize}
            \item \textbf{John Foster Dulles}, Eisenhower's sec. of state, detested communism, denouncing Truman's containment as too passive, desiring "rollback" of communism (president far more moderate)
            \item Instituted \textbf{massive retaliation} policy where U.S. would use massive power in enforcing their desires
            \begin{itemize}
                \item Aimed to prevent long stalemates like the Korean War
                \item Greatest power was economic: believed less money into military and more into atomic weapons
            \end{itemize}
        \end{itemize}
        \textbf{John Foster Dulles believed in a rollback of communism through massive retaliation, or the direct application of U.S. power, primarily their economic resources, to enforce their desires.}}
        \cornell{How did the U.S. become close to conflict in Vietnam?}{\begin{itemize}
            \item Korean War ended under Eisenhower's admin. w/ troops withdrawing beyond parallel to preserve peace; no true resolution with ceasefire line left as permanent border
            \item U.S. became drawn into conflict in Vietnam w/ \textbf{Ho Chi Minh} and his communist nationalists opposing French colonial rule
            \begin{itemize}
                \item After French troops became dominated in conflict, U.S intervention seemed necessary
                \item Dulles ready for action, but Eisenhower refused permission
            \end{itemize}
        \end{itemize}
        \textbf{The Korean War ended shortly after Eisenhower entered office. However, soon after, when communists in Vietnam began to challenge French colonial rule and becoe increasingly more powerful, several Americans sought to send troops into action.}}
        \cornell{What were the most significant crises of the Cold War?}{\begin{itemize}
            \item Containment policy remained dominant in U.S. foreign policy; crises continued throughout world, notably in Middle East 
            \item 1948: Zionist forces created independent Israel immediately recognized by Truman; Palestinian Arabs felt they were displaced from their own country $\to$ several wars emerged betw. Israel and Arab neighbors
            \begin{itemize}
                \item U.S. govt. committed to Israel but also relied on oil of Middle East: shocked that Iranian PM began to cut down Western corporations in 1950s
                \item CIA organized coup to take down PM, replace with conservative military leader as young shah
            \end{itemize}
            \item 1956: Egypt under Nasser formed friendly trade relationship w/ USSR $\to$ U.S. punished by retracting offered funds for dam-building; Egypt retaliated by seizing British Suez Canal
            \begin{itemize}
                \item Israeli forces attacked Egypt; soon assisted by arriving British/French troops
                \item Eisenhower/Dulles feared that U.S. assistance would drive Egypt to help USSR, creating world war $\to$ U.S. refused to join, pressuring GB/Fr. to withdraw, persuading Israeli-Egyptian truce
            \end{itemize}
            \item Significant Cold War relations in Latin America
            \begin{itemize}
                \item CIA toppled leftist government of Arbenz in Guatemala due to fear of communism
                \item Cuba had been tied to U.S. w/ mil. dictator Fulgencio Batista supported by U.S., w/ U.S. corporations valuing Cuba's prosperous resources (notably sugar)
                \begin{itemize}
                    \item 1957: resistance movement under \textbf{Fidel Castro} gained traction, ultimately driving out Batista in 1949
                    \item Castro's radical land reform policies as well as those against foreign businesses paired with support from Soviet Union $\to$ U.S. cut down export ties to Cuba, eventually completely severing ties; Castro began alliance w/ USSR
                \end{itemize}
            \end{itemize}
        \end{itemize}
        \textbf{The Cold War saw conflicts in Israel between Arabs and Jews, in Egypt between Nasser, beginning to form a cordial relationship with the USSR and the French, British, and the Israeli, as well as in Latin America in Guatemala and Cuba.}}
        \cornell{How did tensions continue with the USSR?}{\begin{itemize}
            \item Eisenhower remained focused on interactions w/ USSR
            \item 1955: Eisenhower and NATO leaders met w/ Soviet premier \textbf{Bulganin} in Geneva; foreign ministers unable to resolve issues
            \item 1956: relations further soured after \textbf{Hungarian Revolution} where dissidents launched uprising to demand democracy but USSR crushed regime
        \end{itemize}
        \textbf{Relations between the USSR and the West remained tense: the U.S. sought to reist communist expansion in Europe. In 1955, NATO leaders were unable to reach an agreement over specific issues; relations became worse after the Hungarian Revolution, where the USSR suppressed an effort by a group of Hungarians to return democracy to the region.}}
        \cornell{What was the U-2 crisis?}{\begin{itemize}
            \item 1958: \textbf{Nikita Khrushchev} continually pushed NATO powers to abandon West Berlin
            \begin{itemize}
                \item U.S./Allies refused $\to$ Khrushchev sought personal discussion
                \item Khrushchev visited U.S. in 1959, w/ relatively polite but cool public response
            \end{itemize}
            \item Eisenhower/Khrushchev made plans for Paris summit, Moscow visit
            \begin{itemize}
                \item After USSR shot down U.S. U-2 spy plane over Russian territory, Khrushchev responded angrily, ending all plans to meet
            \end{itemize}
            \item Eisenhower had increased tensions w/ USSR overall: realized limits of U.S. power by limiting Vietnam intervention, placing restraint on U.S. military; cautioned in Farewell Address against military too great
        \end{itemize}
        \textbf{Relations seemed positive between Eisenhower and the new Soviet premier Khrushchev; however, after the USSR shot down a U.S. spy plane above their territory, any plans to reconcile were immediately stopped. In summary, Eisenhower had increased tensions with the USSR overall but also realized the limits of the U.S. power: he was far more careful about blindly sending the U.S. military into conflict.}}
    \end{document}