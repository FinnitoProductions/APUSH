\documentclass[a4paper]{article}
    \input{../notesheader.tex}
    \usepackage[normalem]{ulem}

    \newcommand{\chapternumber}{25}
    \newcommand{\chaptertitle}{The Global Crisis, 1921-1941}

    \title{\vspace{-3em}
\begin{tcolorbox}
\Huge\sffamily \begin{center} AP US History  \\
\LARGE Chapter \chapternumber \, - \chaptertitle \\
\Large Finn Frankis \end{center} 
\end{tcolorbox}
\vspace{-3em}
}
\date{}
\author{}
    \begin{document}
        \maketitle
        \SetBgContents{\rule[0em]{4pt}{\textheight}}
        \cornell[Key Concepts]{What are this chapter's key concepts?}{\begin{itemize}
            \item \textbf{7.3.II.D} - Following WWI, US pursued unilateral foreign policy using international investment, peace treaties, mil. intervention to maintain U.S. isolationism while ensuring international order
            \item \textbf{7.3.II.E} - Most Americans opposed military action in WWII despite opposition to fascism/totalitarianism until Pearl Harbor
        \end{itemize}}
        \cornell[The Diplomacy of the New Era]{How did the New Era U.S. handle foreign policy?}{\textbf{In the 1920s, foreign policy started off devoted entirely to avoiding another war; agreements were made between the US, Britain, and Japan to prevent a potential naval arms race. Furthermore, the U.S. began to take a greater economic presence in Latin America and Europe, with several industries taking a major foothold in the weak European economy unable to recover due to high debts and strong American tariff barriers. When the Depression hit, Hoover struggled to make significant strides: he cut out nearly all intervention in Latin America regardless of the gravity of the situation, refused to compromise with Europe on the issue of debt, and refused cooperation with the League of Nations or military action against Japan after their invasion of Manchuria.}}
        \cornell{What alternatives emerged for the League of Nations?}{\begin{itemize}
            \item 1921: Charles Evans Hughes (sec. of state) secured legislation to declare war at end, negotiating peace treaties w/ each of Central Powers
            \begin{itemize}
                \item Sought to replace League of Nations through less involved program
            \end{itemize}
            \item 1921: \textbf{Washington Naval Conference} intended to prevent costly, war-inducing naval armaments race betw. Japan, U.S., GB
            \begin{itemize}
                \item Hughes proposed reductions in naval production; no large warships for 10 yrs.
                \item \textbf{Five-Power Pact} set limit for tonnage, fixed ratio w/ for each 5 tons of US/GB warships $\to$ Japan could have 3 tons, France/Italy 1.75 
                \item Most terms agreed to: Japanese able to secure power in East Asia by focusing all naval output toward East Asian development
                \item \textbf{Nine-Power Pact} ensured Open Door in China; \textbf{Four-Power Pact} promised no aggression betw. US/GB/Japan/France in Pacific territories
            \end{itemize}
            \item 1928: \textbf{Kellogg-Briand Pact} after France asked U.S. to join alliance against Germany $\to$ Sec. of State Kellogg proposed treaty to ban war entirely: 14 nations signed
            \begin{itemize}
                \item No official ways to enforce but represented moral beacon
            \end{itemize}
            \end{itemize}
            \textbf{After Hughes negotiated individual treaties with each of the Central Powers, ending the war, efforts were taken against it ever repeating. The 1921 Washington Naval Conference agreed to limit naval production among Britain, the US, and Japan to prevent an arms race while securing the open door in China and preventing war in the Pacific. The Kellogg-Briand Pact, signed by 14 nations, outlawed war entirely.}}
        \cornell{How did the U.S. engage in financial diplomacy?}{\begin{itemize}
            \item Several debts owed by Europe to U.S. w/ Allies unable to pay debts to U.S. and Germany unable to pay back Allies, Republican U.S. unwilling to forgive $\to$ Charles G Dawes, American banker, intervened
            \begin{itemize}
                \item \textbf{Dawes Plan} loaned large sums to Germany from U.S. to pay back Allies (who would reduce the amount of required payments); Allies would then use money to pay back U.S. $\to$ circular pattern 
                \item American loans $\to$ U.S. became powerful economic force in Europe
                \begin{itemize}
                    \item Automobile industry opened factories in Europe
                    \item Many other industries exploited weak European econ. to establish foothold $\to$ many feared excessive overseas dependence
                \end{itemize}
                \item Tariff barriers $\to$ Europe unable to export to U.S. $\to$ loans went unpaid; few responded to tariff objections
            \end{itemize}
            \item Strong military presence in Latin America during 1920s
            \begin{itemize}
                \item Argued necessary to quell revolution; truly to increase access to natural resources 
                \item Latin America unable to repay debts (tariffs) $\to$ continued dependence of other nations on U.S. ("\textbf{yankee imperialism}")
            \end{itemize}
        \end{itemize}
        \textbf{With Europe, the United States produced the Dawes Plan, entailing loans to Germany to help them repay debts to the Allies, which the Allies used to pay off debts to the U.S.; this circular flow of money was not a long-term solution, but it allowed several U.S. industries to establish a foothold in the weak European economy. The U.S. maintained a military presence in Latin America. High tariffs for Latin America and Europe to export goods to the U.S. meant that both regions struggled to pay off their debts to American banks.}}
        \cornell{How did Hoover approach foreign policy during the Depression?}{\begin{itemize}
            \item Depression $\to$ destruction of already weak international agreements of 1920s; existing political leaders forced out of office by Depression w/ more radical ones $\to$ risk of war
            \item Latin America: Hoover made 10-week tour for goodwill, attempted to prevent major intervention w/ military withdrawal 
            \begin{itemize}
                \item Repudiated Roosevelt corollary, stressing that economically collapsing Latin American nations would receive no assistance; chose to recognize all governments regardless of how they came to power
            \end{itemize}
            \item Little success in Europe w/ moratorium on debts unsuccessful; refused to cancel debts $\to$ several nations defaulted
            \item Ineffective diplomacy especially concerning w/ new govts. beginning to emerge
            \begin{itemize}
                \item Italy: \textbf{Benito Mussolini}'s \textbf{Fascists} grew more powerful, nationalistic, militaristic w/ imperial expansion
                \item Germany: \textbf{National Socialist (Nazi) Party} grew rapidly in support w/ leader, \textbf{Adolf Hitler} rising to power despite anti-Semitism, militarism
            \end{itemize}
            \item Crisis in Asia after Japan, suffering from econ. depression, saw military govt. takeover, invading northern Manchuria
            \begin{itemize}
                \item U.S., under \textbf{Henry Stimson}, hoped moderate govt. would regain control but never happened $\to$ Stimson simply sent warnings (Hoover banned from working w/ League of Nations)
                \item U.S. did not officially recognize new territories; Japan didn't care, pushing further into China (reaching Shanghai)
            \end{itemize}
            \item By end of Hoover's presidency, clear failure w/ \textbf{voluntary cooperation} for foreign policy fundamentally unsuccessful; U.S. had to choose betw. isolationism and internationalism 
        \end{itemize}
        \textbf{The Depression forced Hoover to take immediate action regarding foreign policy. In Latin America, he ended the Roosevelt corollary by refusing to intervene economically in any collapsing nations. In Europe, too, his unwillingness to cancel any debts caused strained relations; in the midst of these tensions, Mussolini's Fascist and Hitler's Nazi governments began to take power in Italy and Germany, respectively. In Asia, after Japan invaded Manchuria, the U.S. was powerless to intervene as Hoover barred any cooperation with the League of Nations.}}
        \cornell[Isolationism and Internationalism]{How did the U.S. face great conflict between isolationism and internationalism under FDR?}{\textbf{Roosevelt reduced American diplomatic presence in currency and war debt negotiations; however, he attempted to improve the nation's standing in foreign trade through treaties to reduce tariffs. He also attempted (and failed) to improve relations with Russia; he cut off all military intervention in Latin America, replacing it with economic pressure. With Roosevelt unable to make any major strides toward world peace, isolationist belief began to return to American culture due to fear of war; these were particularly manifested in the Neutrality Acts to respond to Italy's invasion of Ethiopia. Growing fear developed among the American people as Japan pushed further into China. World War II began as Hitler pushed into Czechoslovakia, ignoring the Munich Conference with France and Britain, and ultimately invaded Poland, stimulating a response from Britain and France.}}
        \cornell{How did Roosevelt approach diplomatic affairs during the Depression?}{\begin{itemize}
            \item Hoover believed in gold standard $\to$ agreed for U.S. participation in \textbf{World Economic Conference} to find currency stability; Roosevelt had already ended gold standard $\to$ refused to take part
            \begin{itemize}
                \item Roosevelt opposed conference $\to$ unable to make any true progress
            \end{itemize}
            \item Abandoned goal to settle war debts through international discussion
            \begin{itemize}
                \item 1934: Ended Dawes Plan by preventing banks from loaning to nations having defaulted on debts; let issue die w/ war-debt payments simply stopping
            \end{itemize}
            \item Active interest in improving U.S. role in international trade w/ \textbf{Reciprocal Trade Agreement of 1934}
            \begin{itemize}
                \item Authorized negotiation of treaties to lower tariffs up to 50\%
                \item 1939: Sec. of State Cordell Hull had negotiated 21 treaties $\to$ U.S. exports increased 40\%, but imports remained behind bc. agreements only accepted goods not competing w/ American industries 
                \begin{itemize}
                    \item Lagging imports $\to$ not enough American currency circulating to other nations
                \end{itemize}
            \end{itemize}
        \end{itemize}
        \textbf{Roosevelt, having repudiated the gold standards, refused to partake in international agreement to stabilize the currency; he als had little concern for settling the issue of war debts. However, he did seek to improve the U.S. role in world trade, passing the Reciprocal Trade Agreement to allow lowered tariffs; exports increasd dramatically.}}
        \cornell{How did Roosevelt attempt to forge a relationship with the Soviet Union?}{\begin{itemize}
            \item Hostility betw. U.S/Russia since \textbf{Bolshevik Revolution of 1917} w/ U.S. only recognizing Soviets after 1933 
            \item U.S. sought new relationship for trade, Russians sought U.S. help in preventing Japanese dominance
            \begin{itemize}
                \item Soviet foreign minister reached agreement w/ FDR: Soviets would end propaganda if U.S. recognized Soviet regime
                \item U.S. trade unsucessful in Russia; U.S. showed no indication of wanting to help against Japan $\to$ mistrust resumed on both sides
            \end{itemize}
        \end{itemize}
        \textbf{FDR attempted to form a new relationship with the Soviet Union; though he reached a promising agreement with the foreign minister, both sides fell short of their goals and thus mistrust once again emerged.}}
        \cornell{How did the U.S. implement the Good Neighbor policy in Latin America?}{\begin{itemize}
            \item Latin America key target for trade reciprocity w/ exports and imports increased
            \item Hoover admin. had ended military intervention to push Latin American governments; Roosevelt admin. banned any direct intervention in Latin America
            \begin{itemize}
                \item Successful in limiting tensions but created growing economic pressure
            \end{itemize}
        \end{itemize}
        \textbf{Roosevelt's Good Neighbor Policy ended military intervention in Latin America in favor of economic pressure; as a result, exports and imports both increased dramatically and tensions were reduced.}}
        \cornell{How did isolationism return to U.S. foreign policy in its relations with Europe?}{\begin{itemize}
            \item First years of Roosevelt admin. made clear that world peace would not be reached through official treaties
            \begin{itemize}
                \item Arms control conference in Geneva met frequently w/ little success; Roosevelt submitted new proposal after over a year but negotiations failed
                \item After Hitler/Mussolini withdrew from Geneva, Japan from \textbf{London Naval Conference} (extension of Washington Conference), held little weight 
            \end{itemize}
            \item Most Americans began to turn to isolationism
            \begin{itemize}
                \item Even Wilsonians saw unsuccessful League of Nations (notably in Manchuria)
                \item Other Americans convinced that bankers had tricked U.S. into WWI participation to protect loans after investigation by Senator Nye revealed tax evasion during war; Roosevelt was impressed by findings
            \end{itemize}
            \item Roosevelt still sought some role: asked Senate to ratify treaty to allow U.S. to join \textbf{World Court}, international organization with little true significance
            \begin{itemize}
                \item Charles Coughlin, newspaper opposition $\to$ Senate defeated in major blow to Roosevelt 
            \end{itemize}
            \item After great fear emerged that Italy would invade Ethiopia, American legislators produced \textbf{Neutrality Acts}
            \begin{itemize}
                \item Designed to prevent causes of U.S. entry into WWI: forced arms embargo against victim/aggressor, required president to warn U.S. citizens before travelling on potentially dangerous ships
                \item 1936 Neutrality Act renewed; 1937 Neutrality Act created cash-and-carry policy where all belligerents could only buy nonmilitary goods, would need to pay cash 
                \item Mussolini launched attack on Ethiopia, resigning from League; formed \textbf{Axis} alliance w/ Nazis $\to$ isolationist sentiment grew in U.S. w/ govt. offering no assistance to either side during Spanish Civil War
            \end{itemize}
        \end{itemize}
        \textbf{Roosevelt failed to get anywhere through international agreements and treaties; the failed League of Nations and fear that U.S. war entry had been rigged by bankers caused growing isolationist sentiment against Roosevelt. The Neutrality Acts were designed to prevent the causes of U.S. entry into WWI from repeating; they prevented the U.S. from supplying military goods to any nations involved in war; they were successful in preventing immediate U.S. involvement in Italy's invasion of Ethiopia.}}
        \cornell{How did isolationism return to U.S. foreign policy in its relations with Japan?}{\begin{itemize}
            \item Relations w/ Japan weakened as they began to move further into China in 1937 $\to$ Roosevelt issued vague "quarantine" in public speech, but public response hostile $\to$ Roosevelt stepped back
                \begin{itemize}
                    \item Dec. 1937: Japanese plane bombed U.S. gunboat as it sailed in China; obviously deliberate $\to$ outrage but isolationists hoped Roosevelt would accept Japanese apology
                \end{itemize}
            \end{itemize}
        \textbf{As Japan further moved into China, FDR issued a "quarantine" (without specifics). Tensions ran even higher after they bombed a U.S. gunbaot passing through China.}}
        \cornell{How did the Munich Conference fail?}{\begin{itemize}
            \item Hitler annexed the Rhineland, a neutral area protected by the Versailles treaty and long-controlled by France; created \textit{\textbf{Anschluss}} w/ Austria $\to$ intentions clear but rest of Europe/U.S. initially had little concern
            \item Hitler sought to annex Sudetenland region of Czechoslovakia to give more space for Germany; Czechoslovakia willing to fight but no other Euro. nation would come to aid, fearing war
            \item \textbf{Munich Conference} on September 29, 1938 saw Hitler agree to only annex Czechoslovakia and nowhere else $\to$ formal agreement
            \begin{itemize}
                \item Policy known as "appeasement" 
                \item Hitler occupied remainder of Czechoslovakia, began to threaten Poland (GB/France promised intervention if Germany threatened); formed non-agression agreement w/ Russia (had not been invited to Munich Conference)
                \item Germany staged attack on Polish border as justification for attack $\to$ GB/France came to defense, beginning WWII on Sept. 1st, 1939
            \end{itemize}
        \end{itemize}
        \textbf{The Munich Conference, attempting to appease Hitler's demands by allowing him to annex part of Czechoslovakia but nowhere else; Hitler immediately broke this by annexing the entire nation. After he invaded Poland, Britain and France came to his aid, beginning World War Two on September 1st, 1939.}}
        \cornell[From Neutrality to Intervention]{What factors ultimately led the United States to partake in WWII?}{\textbf{U.S. neutrality became strained as Britain became in more and more dire straits, requiring the lending of armaments and the overall violation of the Neutrality Acts; isolationist tendencies became severely reduced as Germany became a serious threat to democracy in Europe. Germany began to attack American ships due to the U.S.' assisting both the Soviet Union and Britain, two enemies of Germany; at the same time, Japan somewhat unexpectedly targeted Pearl Harbor, killing American sailors and destroying several fleets. Pearl Harbor was the ultimate stimulus for the United States to break its neutrality and enter the war.}}
        \cornell{How did U.S. neutrality become increasingly strained?}{\begin{itemize}
            \item FDR never advocated neutral thought (unlike Wilson): clearly favored Allies; believed U.S. should supply arms
            \begin{itemize}
                \item Convinced Congress to pass revision to Neutrality Acts to remove arms embargo by extending cash-and-carry to arms; isolationists limited effect w/ American ships still barred from entering war zones
            \end{itemize}
            \item After Germany annexed Poland, war in Europe saw no real western fighting; main war in east w/ Russia annexing Baltic Republics, Finland 
            \begin{itemize}
                \item Congress only created "moral embargo"
            \end{itemize}
            \item War returned in western Europe after Germany invaded, attacking Denmark/Norway, then $\to$ Netherlands/Belgium; Mussolini joined war against France $\to$ France ultimately fell to German control w/ Britain the only remaining power
            \item Roosevelt immediately began to mobilize w/ demands for defense funds, new warplanes, agreement w/ \textbf{Churchill} to make war materials available despite isolationist opposition
            \begin{itemize}
                \item Violated cash-and-carry provisions by trading U.S. destroyers for right to build U.S. bases on western GB territory
                \item Able to make decisions so freely due to shift in public opinion w/ collapse of Germany $\to$ direct threat
                \item Congress approved \textbf{Burke-Wadsworth Act}, approving \underline{peacetime military draft}
            \end{itemize}
            \item Major debate emerged over isolation w/ some activists advocating more involvement (\textbf{Committee to Defend America by Aiding the Allies}), often desiring immediate decl. of war; others, forming \textbf{America First Committee} w/ several prominent Americans, support of Republican party
        \end{itemize}
        \textbf{FDR, though initially not partaking in the war, always supported the Allies: he circumvented the Neutrality Acts by providing ships to Britain and started an early peacetime military draft. He was able to make these decisions so freely due to a shift in public opinion toward fear after Germany successfully invaded France. Regardless, significant debate emerged between Americans in the summer of 1940.}}
        \cornell{How did Roosevelt win his third term?}{\begin{itemize}
            \item Roosevelt did not make decision for 1940 election clear $\to$ no rival Democrat could truly rise to captivate party while he remained prominent; ultimately agreed to accept "draft" from Dems. 
            \item Roosevelt's position on the center of the war debate (not extreme) $\to$ Republicans had no strong alternative: selected \textbf{Wendell Willkie} w/ very similar positions to Roosevelt
            \begin{itemize}
                \item Though one of most successful Repub. candidates in decades, unable to topple FDR
            \end{itemize}
        \end{itemize}
        \textbf{Roosevelt easily won his third term against Republican Wendell Willkie (who put up a strong effort).}}
        \cornell{How did the U.S. ultimately abandon neutrality?}{\begin{itemize}
            \item Britain had very limited wealth in Dec. 1940 $\to$ Roosevelt implemented \textbf{lend-lease} program to replace cash-and-carry 
            \begin{itemize}
                \item Allowed govt. to lend/lease all arms deemed important to safety of U.S.
                \item Congress enacted easily
            \end{itemize}
            \item German submarines destroyed large amounts of shipped goods $\to$ U.S. implemented \textbf{hemispheric defense}, defending all ships in western Atlantic
            \begin{itemize}
                \item Radioed information to GB abt. location of Nazi submarines
            \end{itemize}
            \item Germany invaded Soviet Union in 1941, breaking pact: drove rapidly into Soviet territory $\to$ Roosevelt extended lend-lease to Soviets in hopes of forming wartime relationship
            \begin{itemize}
                \item U.S. providing assistance on both fronts $\to$ Germany took more direct notice of U.S.
                \item Nazi submarines directly attacked American vessels, attacking American destroyer \textit{Greer}, killing many American sailors in \textit{Reuben James}
                \item After Americans were killed, Congress effectively began naval war
            \end{itemize}
            \item Roosevelt met w/ Churchill several times in private to determine future of nations: essentially wartime plan, publishing \textbf{Atlantic Charter} based around "common principles"
            \item Roosevelt had to wait for direct attack before declaring war
        \end{itemize}
        \textbf{With Britain extremely limited in its sources of wealth, Roosevelt implemented the lend-lease program allowing him to lend American armaments to Britain for no cost; he soon extended this to the Soviet Union after Germany began to invade them. As America became more involved in the war, Germany began to take more notice and directly attack American vessels; as a result, Congress began a naval war. In the meantime, Roosevelt and Churchill made several covert plans in the event of America's entering the war.}}
        \cornell{How did the U.S. ultimately enter the war after Pearl Harbor?}{\begin{itemize}
            \item Japan signed \textbf{Tripartite Pact} to ally w/ Germany and Italy (but conflict remained largely separate throughout war) 
            \item Roosevelt publicly unhappy w/ Japanese policies w/ termination of economic agreements; after imperial troops moved into Vietnam and the U.S. (having broken Japanese secret codes) saw they were planning to invade the Dutch East Indies, Roosevelt frose assets, initiated trade embargo 
            \item Japan either had to repair relations w/ U.S. or find new supplies by conquering more ports
            \begin{itemize}
                \item Moderate prime minister \textbf{Prince Konoye} seemed willing to compromise; militants soon routed out of office, replacing w/ \textbf{Hideki Tojo} 
                \item Tojo pretended to want to compromise, bringing agreement (which favored Japan) to Washington; already settled in that motivations of two parties were fundamentally opposing
            \end{itemize}
            \item U.S. had decoded Japanese messages $\to$ knew attack would be soon, but expected on British/Dutch islands; U.S. focused more on large convoy south through China Sea, overlooking routine warning of navy approaching Pearl Harbor
            \item Dec. 7, 1941: Japanese bombers attacked naval base at \textbf{Pearl Harbor} w/ \underline{no precautions} $\to$ all ships bunched up w/o defenses
            \begin{itemize}
                \item Sailors were killed, ships lost; luckily none of prized aircraft carriers were there
                \item Roosevelt addressed Congress w/ declaration of war against Japan $\to$ Germany and Italy declared war against U.S. $\to$ Congress reciprocated
            \end{itemize}
        \end{itemize}
        \textbf{After Japan allied with Germany and Italy and refused to compromise with the U.S. for their imperialist acquisitions despite a U.S.-initiated trade embargo, tensions rose. Although the U.S. had decoded Japanese signals and knew an attack was coming, they were unable to determine where and were unprepared for the devastating attack at Pearl Harbor; the devastation meant that America almost immediately entered World War II with little dissent.}}
    \end{document}