\documentclass[a4paper]{article}
    \input{../notesheader.tex}
    \usepackage[normalem]{ulem}

    \newcommand{\chapternumber}{25}
    \newcommand{\chaptertitle}{The Global Crisis, 1921-1941}
    \title{\vspace{-3em}
    \begin{tcolorbox}
    \Huge\sffamily \begin{center} AP US History  \\
    \LARGE Chapter \chapternumber \, - \chaptertitle \\
    \Large Finn Frankis \end{center} 
    \end{tcolorbox}
    \vspace{-3em}
    }
    \date{}
    \author{}
    
    \begin{document}
        \maketitle
        \SetBgContents{\rule[0em]{4pt}{\textheight}}
        \cornell[Key Concepts]{What are this chapter's key concepts?}{\begin{itemize}
            \item \textbf{7.3.II.D} - Following WWI, US pursued unilateral foreign policy using international investment, peace treaties, mil. intervention to maintain U.S. isolationism while ensuring international order
            \item \textbf{7.3.II.E} - Most Americans opposed military action in WWII despite opposition to fascism/totalitarianism until Pearl Harbor
        \end{itemize}}
        \cornell[The Diplomacy of the New Era]{How did the New Era U.S. handle foreign policy?}{\textbf{In the 1920s, foreign policy started off devoted entirely to avoiding another war; agreements were made between the US, Britain, and Japan to prevent a potential naval arms race. Furthermore, the U.S. began to take a greater economic presence in Latin America and Europe, with several industries taking a major foothold in the weak European economy unable to recover due to high debts and strong American tariff barriers. When the Depression hit, Hoover struggled to make significant strides: he cut out nearly all intervention in Latin America regardless of the gravity of the situation, refused to compromise with Europe on the issue of debt, and refused cooperation with the League of Nations or military action against Japan after their invasion of Manchuria.}}
        \cornell{What alternatives emerged for the League of Nations?}{\begin{itemize}
            \item 1921: Charles Evans Hughes (sec. of state) secured legislation to declare war at end, negotiating peace treaties w/ each of Central Powers
            \begin{itemize}
                \item Sought to replace League of Nations through less involved program
            \end{itemize}
            \item 1921: \textbf{Washington Naval Conference} intended to prevent costly, war-inducing naval armaments race betw. Japan, U.S., GB
            \begin{itemize}
                \item Hughes proposed reductions in naval production; no large warships for 10 yrs.
                \item \textbf{Five-Power Pact} set limit for tonnage, fixed ratio w/ for each 5 tons of US/GB warships $\to$ Japan could have 3 tons, France/Italy 1.75 
                \item Most terms agreed to: Japanese able to secure power in East Asia by focusing all naval output toward East Asian development
                \item \textbf{Nine-Power Pact} ensured Open Door in China; \textbf{Four-Power Pact} promised no aggression betw. US/GB/Japan/France in Pacific territories
            \end{itemize}
            \item 1928: \textbf{Kellogg-Briand Pact} after France asked U.S. to join alliance against Germany $\to$ Sec. of State Kellogg proposed treaty to ban war entirely: 14 nations signed
            \begin{itemize}
                \item No official ways to enforce but represented moral beacon
            \end{itemize}
            \end{itemize}
            \textbf{After Hughes negotiated individual treaties with each of the Central Powers, ending the war, efforts were taken against it ever repeating. The 1921 Washington Naval Conference agreed to limit naval production among Britain, the US, and Japan to prevent an arms race while securing the open door in China and preventing war in the Pacific. The Kellogg-Briand Pact, signed by 14 nations, outlawed war entirely.}}
        \cornell{How did the U.S. engage in financial diplomacy?}{\begin{itemize}
            \item Several debts owed by Europe to U.S. w/ Allies unable to pay debts to U.S. and Germany unable to pay back Allies, Republican U.S. unwilling to forgive $\to$ Charles G Dawes, American banker, intervened
            \begin{itemize}
                \item \textbf{Dawes Plan} loaned large sums to Germany from U.S. to pay back Allies (who would reduce the amount of required payments); Allies would then use money to pay back U.S. $\to$ circular pattern 
                \item American loans $\to$ U.S. became powerful economic force in Europe
                \begin{itemize}
                    \item Automobile industry opened factories in Europe
                    \item Many other industries exploited weak European econ. to establish foothold $\to$ many feared excessive overseas dependence
                \end{itemize}
                \item Tariff barriers $\to$ Europe unable to export to U.S. $\to$ loans went repaid; few responded to tariff objections
            \end{itemize}
            \item Strong military presence in Latin America during 1920s
            \begin{itemize}
                \item Argued necessary to quell revolution; truly to increase access to natural resources 
                \item Latin America unable to repay debts (tariffs) $\to$ continued dependence of other nations on U.S. ("\textbf{yankee imperialism}")
            \end{itemize}
        \end{itemize}
        \textbf{With Europe, the United States produced the Dawes Plan, entailing loans to Germany to help them repay debts to the Allies, which the Allies used to pay off debts to the U.S.; this circular flow of money was not a long-term solution, but it allowed several U.S. industries to establish a foothold in the weak European economy. The U.S. maintained a military presence in Latin America. High tariffs for Latin America or Europe to export goods to the U.S. meant that both regions struggled to pay off their debts to American banks.}}
        \cornell{How did Hoover approach foreign policy during the Depression?}{\begin{itemize}
            \item Depression $\to$ destruction of already weak international agreements of 1920s; existing political leaders forced out of office by Depression w/ more radical ones $\to$ risk of war
            \item Latin America: Hoover made 10-week tour for goodwill, attempted to prevent major intervention w/ military withdrawal 
            \begin{itemize}
                \item Repudiated Roosevelt corollary, stressing that economically collapsing Latin American nations would receive no assistance; chose to recognize all governments regardless of how they came to power
            \end{itemize}
            \item Little success in Europe w/ moratorium on debts unsuccessful; refused to cancel debts $\to$ several nations defaulted
            \item Ineffective diplomacy especially concerning w/ new govts. beginning to emerge
            \begin{itemize}
                \item Italy: \textbf{Benito Mussolini}'s \textbf{Fascists} grew more powerful, nationalistic, militaristic w/ imperial expansion
                \item Germany: \textbf{National Socialist (Nazi) Party} grew rapidly in support w/ leader, \textbf{Adolf Hitler} rising to power despite anti-Semitism, militarism
            \end{itemize}
            \item Crisis in Asia after Japan, suffering from econ. depression, saw military govt. takeover, invading northern Manchuria
            \begin{itemize}
                \item U.S., under \textbf{Henry Stimson}, hoped moderate govt. would regain control but never happened $\to$ Stimson simply sent warnings (Hoover banned from working w/ League of Nations)
                \item U.S. did not officially recognize new territories; Japan didn't care, pushing further into China (reaching Shanghai)
            \end{itemize}
            \item By end of Hoover's presidency, clear failure w/ \textbf{voluntary cooperation} for foreign policy fundamentally unsuccessful; U.S. had to choose betw. isolationism and internationalism 
        \end{itemize}
        \textbf{The Depression forced Hoover to take immediate action regarding foreign policy. In Latin America, he ended the Roosevelt corollary by refusing to intervene economically in any collapsing nations. In Europe, too, his unwillingness to cancel any debts caused strained relations; in the midst of these tensions, Mussolini's Fascist and Hitler's Nazi governments began to take power in Italy and Germany, respectively. In Asia, after Japan invaded Manchuria, the U.S. was powerless to intervene as Hoover barred any cooperation with the League of Nations.}}
    \end{document}