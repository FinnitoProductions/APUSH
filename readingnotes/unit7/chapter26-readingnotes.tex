\documentclass[a4paper]{article}
    \usepackage[T1]{fontenc}
    \usepackage{tcolorbox}
    \usepackage{amsmath}
    \tcbuselibrary{skins}
    
    \usepackage{background}
    \SetBgScale{1}
    \SetBgAngle{0}
    \SetBgColor{red}
    \SetBgContents{\rule[0em]{4pt}{\textheight}}
    \SetBgHshift{-2.3cm}
    \SetBgVshift{0cm}
    \usepackage[margin=2cm]{geometry} 
    
    \makeatletter
    \def\cornell{\@ifnextchar[{\@with}{\@without}}
    \def\@with[#1]#2#3{
    \begin{tcolorbox}[enhanced,colback=gray,colframe=black,fonttitle=\large\bfseries\sffamily,sidebyside=true, nobeforeafter,before=\vfil,after=\vfil,colupper=blue,sidebyside align=top, lefthand width=.3\textwidth,
    opacityframe=0,opacityback=.3,opacitybacktitle=1, opacitytext=1,
    segmentation style={black!55,solid,opacity=0,line width=3pt},
    title=#1
    ]
    \begin{tcolorbox}[colback=red!05,colframe=red!25,sidebyside align=top,
    width=\textwidth,nobeforeafter]#2\end{tcolorbox}%
    \tcblower
    \sffamily
    \begin{tcolorbox}[colback=blue!05,colframe=blue!10,width=\textwidth,nobeforeafter]
    #3
    \end{tcolorbox}
    \end{tcolorbox}
    }
    \def\@without#1#2{
    \begin{tcolorbox}[enhanced,colback=white!15,colframe=white,fonttitle=\bfseries,sidebyside=true, nobeforeafter,before=\vfil,after=\vfil,colupper=blue,sidebyside align=top, lefthand width=.3\textwidth,
    opacityframe=0,opacityback=0,opacitybacktitle=0, opacitytext=1,
    segmentation style={black!55,solid,opacity=0,line width=3pt}
    ]
    
    \begin{tcolorbox}[colback=red!05,colframe=red!25,sidebyside align=top,
    width=\textwidth,nobeforeafter]#1\end{tcolorbox}%
    \tcblower
    \sffamily
    \begin{tcolorbox}[colback=blue!05,colframe=blue!10,width=\textwidth,nobeforeafter]
    #2
    \end{tcolorbox}
    \end{tcolorbox}
    }
    \makeatother

    \parindent=0pt
    \usepackage[normalem]{ulem}

    \newcommand{\chapternumber}{25}
    \newcommand{\chaptertitle}{The Global Crisis, 1921-1941}
    \title{\vspace{-3em}
    \begin{tcolorbox}
    \Huge\sffamily \begin{center} AP US History  \\
    \LARGE Chapter \chapternumber \, - \chaptertitle \\
    \Large Finn Frankis \end{center} 
    \end{tcolorbox}
    \vspace{-3em}
    }
    \date{}
    \author{}
    
    \begin{document}
        \maketitle
        \SetBgContents{\rule[0em]{4pt}{\textheight}}
        \cornell[Key Concepts]{What are this chapter's key concepts?}{\begin{itemize}
            \item 
        \end{itemize}}
        \cornell[War on Two Fronts]{How did America handle the simultaneous battles in Europe and the Pacific?}{\textbf{In the Pacific, the U.S. initially experienced rapid losses in armaments due to successive Japanese bombings; however, they approached the Japanese from both sides and began to hone in on Japan and the Philippines by 1943. In Europe, the Allies started around the edge of Nazi territory, notably in North Africa and tehn to Italy; the USSR, facing the brunt of German forces, wanted an invasion of France as soon as possible to divert some forces, but the battles on the empire's fringe meant this was consistently postponed; the Soviet victory in Stalingrad was further justification for the Allies not to assist. Regarding the Holocaust in Europe, despite widespread public discontent, U.S. forces chose neither to accept many Jewish refugees nor to attack concentration camps; they believed winning the war was the most important focus to save the Jews.}}
        \cornell{How did the U.S. handle the war with the Japanese?}{\begin{itemize}
            \item Shortly after Pearl Harbor, Japan attacked \textbf{Manila} in Philippines, destroying U.S. air power in Pacific; shortly after, took Guam from U.S., 3-month battle w/ Philippines $\to$ defeat
            \item U.S. strategists had two strategies: \textbf{Douglas MacArthur} moved north from Australia, through New Guinea to Philippines; \textbf{Chester Nimitz} would move west from Hawaii
            \begin{itemize}
                \item Two paths would ultimately converge to invade Japan itself
            \end{itemize}
            \item First major Allied victory: \textbf{Battle of Coral Sea} (near Australia) after Japanese fleet turned back
            \item Turning point NW of Hawaii on \textbf{Midway Island}: though U.S. lost many lives, ultimately emerged successful
            \begin{itemize}
                \item Regained control of central Pacific
            \end{itemize}
            \item Took offensive position soon after in \textbf{Solomon Islands}, attacking three islands w/ terrible struggle (particularly at Guadalcanal); Japanese eventually forced to abandon
            \item Power balance returned to U.S by mid-1943. in southern and central Pacific, particularly w/ assistance from Australia/NZ 
        \end{itemize}
        \textbf{Although Japan inflicted great losses on American forces immediately after Pearl Harbor, a strategy to approach Japan from two sides was very successful, with the U.S. experiencing major victories, notably at Midway Island. Power had returned to the U.S. in the southern and central Pacific; now they had to turn to the Philippines and Japan.}}
        \cornell{How did the U.S. handle the war in Europe?}{\begin{itemize}
            \item U.S. forces fought w/ Britain, "Free French" forces in west; attempts to connect w/ new ally, USSR
            \item \textbf{General George C. Marshall} supported Allied invasion of France across the English Chanmel 
            \begin{itemize}
                \item USSR supported immediate attack on France because their eastern forces were experiencing majority of German troops
                \item British wanted Allied attacks around edge of Nazi empire to weaken first 
            \end{itemize}
            \item Roosevelt realized mainland invasion would take long to prepare for $\to$ supported British plan (despite opposition from advisers)
            \begin{itemize}
                \item Oct. 1942: GB began counteroffensive against Nazis in North Africa around Suez Canal
                \item Germans poured effort into fight against U.S. $\to$ major U.S. losses at \textbf{Kasserine Pass} in Tunisia in late 1942
                \item Germans finally driven from Africa in May 1943 w/ Allied air/naval power 
            \end{itemize}
            \item Battle in North Africa had used large sums of resources $\to$ invasion of France delayed $\to$ Soviets unhappy
            \item Soviets in far less dangerous position: successfully drove away large numbers of Hitler's forces at \textbf{Stalingrad} during winter of 1942-43
            \begin{itemize}
                \item German siege had decimated Stalingrad; Soviet Union lost far more people than any other nation $\to$ angry
            \end{itemize}
            \item Soviet success in beating Germans $\to$ Roosevelt agreed w/ Churchill to begin Allied invasion of Sicily (further pushing back invasion of France bc. felt Soviets did not need signif. assistance)
            \begin{itemize}
                \item GB/U.S. armies arrived in Sicily in July 1943; easily pushed out and moved onto Italian mainland
                \item Mussolini's govt. collapsed, successor quickly joined Allies; Germany moved forces into country, created defense south of Rome
                \item Allied faced setback w/ winter, resuming in May 1944 and finally capturing Rome in June 1944 
                \item Continual postponement angered Soviets, believing delays were deliberate; Soviets simultaneously began to invade eastern Europe
            \end{itemize}
        \end{itemize}
        \textbf{The USSR supported an immediate U.S. invasion of France to divert some German troops away from them; Britain, however, wanted a more gradual weakening starting in North Africa and passing through Italy. Because the Soviets had significant success in Stalingrad, the Allies chose the latter, with slow successes but ultimately causing Mussolini's government to collapse; the Germans took over, but the Allies eventually pushed into Rome.}}
        \cornell{How did America address the Holocaust?}{\begin{itemize}
            \item U.S. knew about Holocaust as early as 1942: aware that Hitler rounded up Jews and other foreigners, sending to concentration camps, murdering in gas chambers
            \begin{itemize}
                \item Public pressure demanded rescue of some Jews
            \end{itemize}
            \item U.S. govt. generally rejected demands to save Jews w/ sending bombers to destroy camps or railroads leading to camps safely seen as unfeasible
            \item Refused to accept large numbers of Jewish refugees escaping Europe 
            \begin{itemize}
                \item \textit{ex}: \textit{St. Louis} boat carrying 1k Jews turned away in Cuba and then Miami, returning to Europe
                \item Did not use close to quota of visas: largely due to anti-Semitic state officials (like \textbf{Breckenridge Long}) 
            \end{itemize}
            \item Policymakers argued that best use of time to successfully save Jews was to concentrate on winning war
        \end{itemize}
        \textbf{The U.S. knew about the Holocaust as early as 1942, aware of the atrocities committed against Jews and other foreigners. The U.S. government refused to bomb concentration camps and railroads to concentration camps (mainly due to unfeasability) and would accept very few Jewish refugees due to anti-Semitic state officials.}}
    \end{document}