\documentclass[a4paper]{article}
    \usepackage[T1]{fontenc}
    \usepackage{tcolorbox}
    \usepackage{amsmath}
    \tcbuselibrary{skins}
    
    \usepackage{background}
    \SetBgScale{1}
    \SetBgAngle{0}
    \SetBgColor{red}
    \SetBgContents{\rule[0em]{4pt}{\textheight}}
    \SetBgHshift{-2.3cm}
    \SetBgVshift{0cm}
    \usepackage[margin=2cm]{geometry} 
    
    \makeatletter
    \def\cornell{\@ifnextchar[{\@with}{\@without}}
    \def\@with[#1]#2#3{
    \begin{tcolorbox}[enhanced,colback=gray,colframe=black,fonttitle=\large\bfseries\sffamily,sidebyside=true, nobeforeafter,before=\vfil,after=\vfil,colupper=blue,sidebyside align=top, lefthand width=.3\textwidth,
    opacityframe=0,opacityback=.3,opacitybacktitle=1, opacitytext=1,
    segmentation style={black!55,solid,opacity=0,line width=3pt},
    title=#1
    ]
    \begin{tcolorbox}[colback=red!05,colframe=red!25,sidebyside align=top,
    width=\textwidth,nobeforeafter]#2\end{tcolorbox}%
    \tcblower
    \sffamily
    \begin{tcolorbox}[colback=blue!05,colframe=blue!10,width=\textwidth,nobeforeafter]
    #3
    \end{tcolorbox}
    \end{tcolorbox}
    }
    \def\@without#1#2{
    \begin{tcolorbox}[enhanced,colback=white!15,colframe=white,fonttitle=\bfseries,sidebyside=true, nobeforeafter,before=\vfil,after=\vfil,colupper=blue,sidebyside align=top, lefthand width=.3\textwidth,
    opacityframe=0,opacityback=0,opacitybacktitle=0, opacitytext=1,
    segmentation style={black!55,solid,opacity=0,line width=3pt}
    ]
    
    \begin{tcolorbox}[colback=red!05,colframe=red!25,sidebyside align=top,
    width=\textwidth,nobeforeafter]#1\end{tcolorbox}%
    \tcblower
    \sffamily
    \begin{tcolorbox}[colback=blue!05,colframe=blue!10,width=\textwidth,nobeforeafter]
    #2
    \end{tcolorbox}
    \end{tcolorbox}
    }
    \makeatother

    \parindent=0pt
    \usepackage[normalem]{ulem}

    \newcommand{\chapternumber}{26}
    \newcommand{\chaptertitle}{America in a World at War}

    \title{\vspace{-3em}
\begin{tcolorbox}
\Huge\sffamily \begin{center} AP US History  \\
\LARGE Chapter \chapternumber \, - \chaptertitle \\
\Large Finn Frankis \end{center} 
\end{tcolorbox}
\vspace{-3em}
}
\date{}
\author{}
    \begin{document}
        \maketitle
        \SetBgContents{\rule[0em]{4pt}{\textheight}}
        \cornell[Key Concepts]{What are this chapter's key concepts?}{\begin{itemize}
            \item \textbf{7.2.II.B} - $\uparrow$ war production/labor demand during WWI/WWII $\to$ migration to urban centers
            \item \textbf{7.2.II.D} - $\uparrow$ Mexico $\to$ U.S. migration despite oppressive policies
            \item \textbf{7.3.III.A} - Americans viewed war as freedom/democracy v. fascist/militarist ideologies; reinforced by Japanese atrocities/Holocaust
            \item \textbf{7.3.III.B} - Mobilization of U.S. society $\to$ end of Great Depression; industrial base helped to win war by equipping millions of troops
            \item \textbf{7.3.III.C} - Women/minorities able to improve positions due to mil. service / econ. mobilization; debates over segregation emerged; war saw some groups particularly oppressed, like Japanese-Americans
            \item \textbf{7.3.III.D} - Allies won war through cooperation, technology, servicemen/women, Pacific "island-hopping campaigns, D-Day; atomic bombs did end war war more rapidly but led to greqt questions abt. morality
            \item \textbf{7.3.III.E} - U.S. emerged from war as the greatest world superpower due to ravaged state of Asia/Europe, essential role in finding peace post-war
        \end{itemize}}
        \cornell[War on Two Fronts]{How did America handle the simultaneous battles in Europe and the Pacific?}{\textbf{In the Pacific, the U.S. initially experienced rapid losses in armaments due to successive Japanese bombings; however, they approached the Japanese from both sides and began to hone in on Japan and the Philippines by 1943. In Europe, the Allies started around the edge of Nazi territory, notably in North Africa and tehn to Italy; the USSR, facing the brunt of German forces, wanted an invasion of France as soon as possible to divert some forces, but the battles on the empire's fringe meant this was consistently postponed; the Soviet victory in Stalingrad was further justification for the Allies not to assist. Regarding the Holocaust in Europe, despite widespread public discontent, U.S. forces chose neither to accept many Jewish refugees nor to attack concentration camps; they believed winning the war was the most important focus to save the Jews.}}
        \cornell{How did the U.S. handle the war with the Japanese?}{\begin{itemize}
            \item Shortly after Pearl Harbor, Japan attacked \textbf{Manila} in Philippines, destroying U.S. air power in Pacific; shortly after, took Guam from U.S., 3-month battle w/ Philippines $\to$ defeat
            \item U.S. strategists had two strategies: \textbf{Douglas MacArthur} moved north from Australia, through New Guinea to Philippines; \textbf{Chester Nimitz} would move west from Hawaii
            \begin{itemize}
                \item Two paths would ultimately converge to invade Japan itself
            \end{itemize}
            \item First major Allied victory: \textbf{Battle of Coral Sea} (near Australia) after Japanese fleet turned back
            \item Turning point NW of Hawaii on \textbf{Midway Island}: though U.S. lost many lives, ultimately emerged successful
            \begin{itemize}
                \item Regained control of central Pacific
            \end{itemize}
            \item Took offensive position soon after in \textbf{Solomon Islands}, attacking three islands w/ terrible struggle (particularly at Guadalcanal); Japanese eventually forced to abandon
            \item Power balance returned to U.S by mid-1943. in southern and central Pacific, particularly w/ assistance from Australia/NZ 
        \end{itemize}
        \textbf{Although Japan inflicted great losses on American forces immediately after Pearl Harbor, a strategy to approach Japan from two sides was very successful, with the U.S. experiencing major victories, notably at Midway Island. Power had returned to the U.S. in the southern and central Pacific; now they had to turn to the Philippines and Japan.}}
        \cornell{How did the U.S. handle the war in Europe?}{\begin{itemize}
            \item U.S. forces fought w/ Britain, "Free French" forces in west; attempts to connect w/ new ally, USSR
            \item \textbf{General George C. Marshall} supported Allied invasion of France across the English Chanmel 
            \begin{itemize}
                \item USSR supported immediate attack on France because their eastern forces were experiencing majority of German troops
                \item British wanted Allied attacks around edge of Nazi empire to weaken first 
            \end{itemize}
            \item Roosevelt realized mainland invasion would take long to prepare for $\to$ supported British plan (despite opposition from advisers)
            \begin{itemize}
                \item Oct. 1942: GB began counteroffensive against Nazis in North Africa around Suez Canal
                \item Germans poured effort into fight against U.S. $\to$ major U.S. losses at \textbf{Kasserine Pass} in Tunisia in late 1942
                \item Germans finally driven from Africa in May 1943 w/ Allied air/naval power 
            \end{itemize}
            \item Battle in North Africa had used large sums of resources $\to$ invasion of France delayed $\to$ Soviets unhappy
            \item Soviets in far less dangerous position: successfully drove away large numbers of Hitler's forces at \textbf{Stalingrad} during winter of 1942-43
            \begin{itemize}
                \item German siege had decimated Stalingrad; Soviet Union lost far more people than any other nation $\to$ angry
            \end{itemize}
            \item Soviet success in beating Germans $\to$ Roosevelt agreed w/ Churchill to begin Allied invasion of Sicily (further pushing back invasion of France bc. felt Soviets did not need signif. assistance)
            \begin{itemize}
                \item GB/U.S. armies arrived in Sicily in July 1943; easily pushed out and moved onto Italian mainland
                \item Mussolini's govt. collapsed, successor quickly joined Allies; Germany moved forces into country, created defense south of Rome
                \item Allied faced setback w/ winter, resuming in May 1944 and finally capturing Rome in June 1944 
                \item Continual postponement angered Soviets, believing delays were deliberate; Soviets simultaneously began to invade eastern Europe
            \end{itemize}
        \end{itemize}
        \textbf{The USSR supported an immediate U.S. invasion of France to divert some German troops away from them; Britain, however, wanted a more gradual weakening starting in North Africa and passing through Italy. Because the Soviets had significant success in Stalingrad, the Allies chose the latter, with slow successes but ultimately causing Mussolini's government to collapse; the Germans took over, but the Allies eventually pushed into Rome.}}
        \cornell{How did America address the Holocaust?}{\begin{itemize}
            \item U.S. knew about Holocaust as early as 1942: aware that Hitler rounded up Jews and other foreigners, sending to concentration camps, murdering in gas chambers
            \begin{itemize}
                \item Public pressure demanded rescue of some Jews
            \end{itemize}
            \item U.S. govt. generally rejected demands to save Jews w/ sending bombers to destroy camps or railroads leading to camps safely seen as unfeasible
            \item Refused to accept large numbers of Jewish refugees escaping Europe 
            \begin{itemize}
                \item \textit{ex}: \textit{St. Louis} boat carrying 1k Jews turned away in Cuba and then Miami, returning to Europe
                \item Did not use close to quota of visas: largely due to anti-Semitic state officials (like \textbf{Breckenridge Long}) 
            \end{itemize}
            \item Policymakers argued that best use of time to successfully save Jews was to concentrate on winning war
        \end{itemize}
        \textbf{The U.S. knew about the Holocaust as early as 1942, aware of the atrocities committed against Jews and other foreigners. The U.S. government refused to bomb concentration camps and railroads to concentration camps (mainly due to unfeasability) and would accept very few Jewish refugees due to anti-Semitic state officials.}}
        \cornell[The American People in Wartime]{How did the Second World War affect American society?}{\textbf{The Second World War led to great economic prosperity, particularly in Western development. Civilian employment and union membership (though restricted) boomed; the government carefully prevented inflation by rationing and setting caps on prices and wages and funded the war with bonds and income tax. They encouraged wartime production as well as the development of technology, with the Allied creation of radar and sonar, antiaircraft technology, new bombers, and advanced code-breaking computers like the Enigma machine. African Americans won new labor rights during the war effort; over time, the military became notably less segregated and more accepting of black participation due to public opinion and the need for manpower. The war encouraged Native Americans to assimilate into capitalist society as code-breakers or factory workers; after being deported in large numbers during the Depression, countless Mexican Americans returned to find agricultural labor and new factory work. Women were able to enter previously male roles, often focusing on heavy work; most, however, continued to work in the service industry, finding clerical jobs; they often struggled to raise children while their husbands were at war and they worked. The war led to a great buoyancy in American culture, motivated by the return of material comfort. Finally, though most racial groups were far less persecuted than in WWI, Japanese Americans were sent to prison-like internment camps for many years, often used as forced labor; Chinese Americans, however, were permitted to immigrate to the U.S. and saw greater social acceptance. Roosevelt won the election of 1944, marking his fourth term, by focusing on domestic issues and for his willingness to abandon the New Deal in favor of wartime policies.}}
        \cornell{How did the war affect U.S. economic prosperity?}{\begin{itemize}
            \item War ended Great Depression by pushing industrial expansion
            \item \underline{Federal spending was most significant} w/ govt. pumping more money into econ. than all New Deal relief agencies had combined
            \begin{itemize}
                \item 1939: federal budget was \$9b (highest ever in peacetime); reached \$100b in 1945 $\to$ GDP grew rapidly w/ personal incomes doubling
                \item Wartime demands $\to$ shortage of consumer goods $\to$ many ppl. put majority of money into savings
            \end{itemize}
        \end{itemize}
        \textbf{The war ultimately ended the Great Depression, primarily due to the very high federal budget during the war, greatly increasing the GDP and personal incomes.}}
        \cornell{How did the war affect the West?}{\begin{itemize}
            \item Govt. spending most significant in West (consistently relied on federal support)
            \begin{itemize}
                \item West Coast, as launching point for battles w/ Japan, saw new govt. manufacturing facilities (notably CA)
                \item Govt. invested \$40b during war, w/ 10\% of all federal money $\to$ CA during war
            \end{itemize}
            \item Economy of Pacific transformed by conclusion of war as center of American aircraft industry; shipbuilding industry in CA, WA; LA became a major industrial center beyond just Hollywood
        \end{itemize}
        \textbf{The majority of government spending was poured into the West due to its strategic location relative to Japan. Factories, highways, power plants, and military facilities were created, allowing states like California and Washington to boom and cities like Los Angeles to grow rapidly.}}
        \cornell{How did the war affect labor distribution in the U.S.?}{\begin{itemize}
            \item War created \underline{labor shortage} due to armed forces
            \item Civilian workforce increased by 20\% due to 7m unemployed pre-war and those previously considered inappropriate (women, children, the elderly) being employed
            \item Union membership expanded rapidly w/ 13m by 1945
            \begin{itemize}
                \item Govt. feared inflation, wanted to ensure production proceeded $\to$ created \textbf{Little Steel formula} (capping wartime wage increases at 15\%), created "no-strike" pledge w/ unions 
                \item In return, govt. created "\textbf{maintenance-of-membership}" agreement allowing all new workers to join unions
                \item Many workers resented govt. control $\to$ strikes regardless (\textbf{wildcat strikes} unauthorized by unions themselves)
                \item United Mine Workers striked in May 1943 $\to$ Congress passed (despite FDR's veto) \textbf{Smith-Connally Act}, requiring union to wait 30 days before striking, allowed president to seize struck war plant
            \end{itemize}
            \item Union strikes during wartime $\to$ significant discontent toward labor emerged
        \end{itemize}
        \textbf{The war in fact created a labor shortage, with the civilian workforce expanding rapidly as the previously unemployed as well as women began to enter. Unions grew significantly in prominence; however, their power was limited by the government, capping wartime wage increases and banning strikes in return for a guarantee that all new workers could join unions. Several workers defied the government, striking without union permission, requiring the Smith-Connally Act to limit strikes.}}
        \cornell{How did the U.S. begin to stabilize the wartime boom?}{\begin{itemize}
            \item Great fear of inflation w/ boom $\to$ Congress passed \textbf{Anti-Inflation Act}, allowing govt. to freeze agri. prices, wages, salaries, rents 
            \begin{itemize}
                \item \textbf{Office of Price Administration} (OPA) intended to enforce provisions of Anti-Inflation Act: guaranteed that inflation was not as major a problem during WWII as it had been during WWI 
                \item OPA relatively resented due to strict control over wages, rationing of important consumer goods like coffee, sugar, meat, gasoline, etc.; encouraged growth of blackmarkets
            \end{itemize}
            \item 1941-1945: U.S. govt. spent double \underline{all the money it had spent in its existence}, ten times WWI; obtained needed revenue in several ways 
            \begin{itemize}
                \item Sold \underline{bonds} to citizens; majority to financial institutions
                \item Created \textbf{Revenue Act of 1942}, increasing \underline{income tax} dramatically for highest brackets; introduced tax even for lowest-income families
                \begin{itemize}
                    \item 1943: Congress created withhoding system for constant payroll deductions
                \end{itemize}
            \end{itemize}
        \end{itemize}
        \textbf{As the U.S. economy began to return to prosperity during the war, the government focused on preventing the return of inflation. The Office of Price Administration enforced the Anti-Inflation Act, rationing important goods and strictly controlling prices and wages. In all, the government spent an amount unparalleled in all of its history; it earned this money by selling bonds and raising income tax for all people.}}
        \cornell{How did the U.S. seek to mobilize wartime production?}{\begin{itemize}
            \item 1939: started attempt to mobilize economy for war, but four years of continual failures
            \item 1942: president created \textbf{War Production Board} led by Donald Nelson w/ broad powers over econ.
            \begin{itemize}
                \item Never as strong/authoritative as WWI counterpart led by powerful Bernard Baruch
                \item Could not control military purchases w/ army/navy often bypassing board to form contracts w/ producers directly 
                \item Unable to satisfy small business' complaints that all of contracts to largest corporations
            \end{itemize}
            \item President soon transferred authority to new office within White House: \textbf{Office of War Mobilization}, led by former Supreme Court justice; not much more successful than WPB
            \item War economy met nation's needs w/ new factories funded by govt.'s \textbf{Defense Plants Corporation}, rubber industry emerging to make up for loss of natural access
            \begin{itemize}
                \item Soon produced more than the govt. needed w/ twice that of all Axis nations combined
                \item Many feared that military production too large, w/ some hoping to resume civilian production; mil. refused
            \end{itemize}
        \end{itemize}
        \textbf{After several failed attempts to mobilize the economy for the war, FDR created the War Production Board intended to have sweeping powers, but it was never as strong as expected due to the military's frequent circumvention of their policies; its successor, the Office of War Mobilization was not much more successful. Regardless, industrial growth with new factories and an emerging rubber industry meant that the economy was able to sustain the wartime demand.}}
        \cornell{How did science and technology develop during wartime?}{\begin{itemize}
            \item WWII essential for innovation due to govt. support w/ \textbf{National Defense Research Committee} led by \textbf{Vannevar Bush} of MIT; spent more than \$100m on research (4x that of prev. years)
            \item Initially, Germans/Japanese had tech. advantages w/ mechanized armor, tank development, advanced U boats allowing for destruction of Allied forces; Japan had naval-air tech., fighter planes
            \item U.S. mass production techniques w/ assembly lines allowed for far more efficient production
            \begin{itemize}
                \item Able to produce goods in greater numbers 
                \item Scientists worked actively to improve armaments, focusing on \underline{submarines and tanks}
            \end{itemize}
            \item U.S./British created \textbf{radar and sonar} using expanding radio technology, allowing Allies to destroy U-boats
            \begin{itemize}
                \item \textbf{Centimetric radar} used narrow beams more effective than ever before, allowing detection of submarines ten miles away at night 
                \item Could be made smaller for use on planes/submarines w/ small aerial 
            \end{itemize}
            \item Allied determined how to detect naval mines; Germans soon retaliated to create acoustic mines, blowing up on proximity, not contact; Allies created countermeasures
            \item Antiaircraft tech. developed far; not enough to stop bombing raids 
            \item German rocket-propelled bombs did inflict significant psychological damage on British but never able to truly tip balance in favor
            \item 1942: Allied produced four-engine bombers like \textbf{British Lancaster B1} to fly bomb load of 6k lbs. for 1.3k miles; flew higher and longer than German equiv. $\to$ extensive missions
            \begin{itemize}
                \item \textbf{Gee navigation system} meant expert navigators no longer needed; successful in Ruhr Valley
                \item \textbf{Oboe System} sent sonic nations to airplanes after within 20 yards of targets
            \end{itemize}
        \end{itemize}
        \textbf{The government poured significant funds into the development of technology; although the Germans and Japanese were initially dominated, U.S. mass production allowed for far more efficient creation of goods. Furthermore, radar and sonar allowed for submarine detection; the Allies could find German naval mines; they created antiaircraft devices to prevent bombing raids. Though the Germans made powerful rocket-propelled bombs, the British developed new bombers powered by new navigational systems.}}
        \cornell{How did the Allied forces make progress in intelligence?}{\begin{itemize}
            \item \textbf{Ultra project} in Britain driven by cryptologists w/ some stealing of German, Japanese intelligence devices, applying decoding technology to decipher 
            \item \textbf{Enigma machine} critical to decoding German signals
            \begin{itemize}
                \item Polish intelligence started w/ electro-mechanical computer called \textbf{Bombe}
                \item \textbf{Alan Turing} expanded Bombe to work far more rapidly, ultimately able to decode 1k messages each day to allow for constant flow of info. 
            \end{itemize}
            \item \textbf{Colossus II}, first programmable digital computer, created just before Normandy invasion, able to decipher German messages nearly instantly
            \item U.S. had key breakthroughs w/ \textbf{American Magic} to break Japanese codes w/ system known as \textbf{Purple} (1941)
            \begin{itemize}
                \item Could have alerted to Pearl Harbor but too inconceivable 
            \end{itemize}
        \end{itemize}
        \textbf{The key advances in intelligence were made in the British Ultra project, where the Enigma machine rapidly decoded German signals (eventually replaced by the powerful Colossus II), and the American Magic project, breaking Japanese codes with the Purple machine.}}
        \cornell{How did African Americans fit into the war effort?}{\begin{itemize}
            \item WWI saw most Afr. Americans hoping to fight in battles to increase prestige, win new position, but disappointed; tried again in WWII by \underline{making demands}
            \begin{itemize}
                \item A. Philip Randolph, president of black union, insisted that govt. would require companies to receive defense contracts to integrate workforce 
                \item Planned march in Washington w/ 100k demonstrators $\to$ Roosevelt created \textbf{Fair Employment Practices Commission} to limit workplace discrimination in exchange for ending march
                \begin{itemize}
                    \item Limited enforcement but important symbolic victory
                \end{itemize}
            \end{itemize}
            \item War plants needed labor $\to$ significant migration of blacks from South into cities, continuing even after war like the Great Migration; created urban tension (\textit{ex}: racial violence in Detroit, June 1943)
            \item Leading black organizations continued to challenge segregation: \textbf{Congress of Racial Equality} (CORE) mobilized popular resistance led by young black leaders (resistance to past conservative movements)
            \begin{itemize}
                \item Forced D.C. restaurant to serve Afr. Americans; spirit continued through to 1950s, encouraging civil rights
            \end{itemize}
            \item Pressure within military to change black assignments
            \begin{itemize}
                \item Initially relegated to menial jobs in segregated camps; barred from Marine Corps/Army Air Forces
                \item Public/political pressure paired with practical need of as much manpower as possible forced military leaders to put African Americans in battle directly w/ whites (paired w/ white sailors)
                \item All-black units also increased dramatically
                \begin{itemize}
                    \item Segregated divisions $\to$ riots, like in Fort Dix, NJ; segregation continued but traditional ways still shifted
                \end{itemize}
            \end{itemize}
        \end{itemize}
        \textbf{A. Philip Randolph used the war to demand greater rights in the labor force, leading to Roosevelt's Fair Employment Practices Commission. Furthermore, the need for labor encouraged black migration to cities; this also led to the direct challenging of segregation and a new civil rights movement with the Congress of Racial Equality. Finally, due to public pressure and the need for manpower, the military structure slowly began to shift to give blacks more assignments.}}
        \cornell{How were Native Americans integrated into the war?}{\begin{itemize}
            \item $\approx$ 25k Native Ameriacns partook in war; many in combat (\textit{ex}: \textbf{Ira Hayes}), others worked as "code-talkers" to speak code in native languages to prevent enemy forces from deciphering
            \item Native Americans remaining civilians saw little work coming to tribes; government subsidies severely reduced
            \begin{itemize}
                \item Many talented tribe members turned to war plants $\to$ direct contact w/ capitalist society, many choosing to assimilate into new world even after the war; others lost employment opportunities, returning home after war
            \end{itemize}
            \item Support for tribal autonomy encouraged by Indian Reorganization Act of 1934 limited; new pressures for assimilation $\to$ John Collier, previous fighter for native rights, resigned
        \end{itemize}
        \textbf{Many Native Americans partook in the war, some in combat but others to speak messages in their native languages to prevent the enemies from deciphering them. Many Native Americans, seeing support for tribal autonomy and subsidies undermined, were forced to assimilate into capitalist societies, working at war plants.}}
        \cornell{How did Mexican American workers partake in the war?}{\begin{itemize}
            \item Labor shortages on Pacific Coast $\to$ large numbers of Mexican workers entered
            \begin{itemize}
                \item U.S./Mexican govts. agreed on program to allow \textit{braceros} (contract labors) to enter for short period of time for specific jobs
                \item Many Mexican American farmers deported during Depression were rehired; many others found jobs in factories for the first time ever
            \end{itemize}
            \item Expansion of Mexican American populations $\to$ racial conflicts
            \begin{itemize}
                \item White residents of LA feared Mexican American teens joining street gangs - \textit{pachucos} - and wearing \textbf{zoot suits}, long, baggy jackets
                \item Four-day LA riot w/ white sailors invading Mexican American communities, attacking members; police sided with white sailors and arrested those who fought back $\to$ LA banned zoot suits
            \end{itemize}
        \end{itemize}
        \textbf{Labor shortages on the Pacific Coast encouraged large numbers of Mexican workers to return to America and work in farms, and, unlike before, factories. The rapid influx of Mexican American immigrants led to significant racial conflicts, culminating in a four day riot where white sailors attacked Mexican Americans and were supported by the police.}}
        \cornell{How were women and children affected by the war?}{\begin{itemize}
            \item Women drawn into workforce roles from which they had been previously banned
            \begin{itemize}
                \item Workforce women increased 60\%, becoming $\frac{1}{3}$ of all paid workers in 1945
                \item More likely to be married/older
            \end{itemize}
            \item Many entered factory work to replace male workers; still obstacles w/ factory owners classifying work by both gender and race, determining wages as such; employers reduced need for heavy labor w/ automated assembly
            \item Women recruited w/ analogies to domestic life: airplane wings connected to making a dress, chemical mixing to making a cake; many took on jobs seen as "men's work"
            \begin{itemize}
                \item \textbf{"Rosie the Riveter"} symbolized female industrial workforce; eroded some prejudice, particularly against working mothers
            \end{itemize}
            \item \underline{Most women in service-sector jobs}, working for govt. to fulfill bureaucratic needs
            \begin{itemize}
                \item Many became typists, living in cramped quarters 
                \item Public/private clerical employment 
                \item Some women as \textbf{WACs} (in army) and \textbf{WAVEs} (in navy)
            \end{itemize}
            \item New opportunities $\to$ problems w/ women w/ husbands in mil. simultaneously working and caring for kids
            \begin{itemize}
                \item Often forced to leave children at home or in locked cars while at work (\underline{limited childcare facilities})
                \item Family tensions $\to$ juvenile crime w/ young boys often theft/burglary, teenage girls prostitution and transmitting STDs 
                \item Many more children betw. 14-18 forced to work 
            \end{itemize}
            \item Wartime prosperity $\to$ $\uparrow$ marriage rate, $\downarrow$ marriage age, but also $\uparrow$ divorce rate due to wartime pressures; $\uparrow$ birth rate $\to$ \textbf{baby boom}
        \end{itemize}
        \textbf{Women were frequently drawn into roles which they had always been barred from, often partaking in standard factory work but often more heavy jobs (like in the advertisement for Rosie the Riveter). Most, however, remained in service jobs, with clerical employment as typists for the government. Many women struggled to care for their children while working, leading to family tensions and encouraging juvenile crime. Finally, the war saw a baby boom with increased marriage rates also causing a baby boom.}}
        \cornell{How did WWII affect American wartime life and culture?}{\begin{itemize}
            \item War $\to$ great anxiety w/ families fearing for loved ones, losing family members from home; businesses/communities struggled w/ shortages of goods
            \item Great buoyancy w/ Depression over $\to$ money to spend again, some things to spend it on; magazines peaked in popularity, movies very popular, resort hotels/casinos/racetracks/dance halls attractive to both civilians and soldiers
            \item Advertisements tied war effort into need for material comfort, arguing that U.S. soldiers were fighting for return of consumer culture and employment; many troops shared/motivated by this sentiment
            \item Many fighting men dreamed of material comforts of home, women (wives/girlfriends, but also movie stars and \textbf{pinups} to motivate fighters)
            \item Servicemen remaining in U.S. had desire for women $\to$ \textbf{USOs} recruited women to host clubs and entertain men; others joined "dance brigades," travelling to mil. bases to interact w/ men
            \begin{itemize}
                \item Instructed to preserve no further relations w/ the men they met (often broken: military crushed homosexuality but tolerated some heterosexuality)
            \end{itemize}
        \end{itemize}
        \textbf{WWII, despite leading to widespread anxiety, also created great happiness with the return of money to spend; material comfort became a significant motivating factor for soldiers and defined what they were fighting for. Furthermore, several servicemen dreamed of women; those who remained in the U.S. often had these needs met by USOs, providing temporary women for the soldiers' enjoyment.}}
        \cornell{What were the main characteristics of wartime racial prejudice?}{\begin{itemize}
            \item WWI had seen anti-socialist, anti-Semitic, anti-German movements; WWII banned only most radical papers (like Coughlin's anti-Semitic, pro-fascist \textit{Social Justice})
            \begin{itemize}
                \item Most communists/socialists left alone, but some German spies executed
                \item Attempt to punish fascists ended in mistrial
            \end{itemize}
            \item Not much cultural animosity; though some violence like w/ "zoot-suit" riots and war-time restrictions on some Italians, most of popular culture blamed oppressive German/Italian govts. and \underline{not the people themselves}
            \begin{itemize}
                \item Heroism of many American soldiers of diff. backgrounds $\to$ difficult to persecute a given ethnicity
            \end{itemize}
        \end{itemize}
        \textbf{The second world war was characterized by far less general racial prejudice than WWI: Germans and Italians were rarely persecuted in large numbers due to their heroism in battle as well as the belief that their oppressive governments, not their ethnicities, were to blame.}}
        \cornell{What were the manifestations of animosity toward Japanese Americans?}{\begin{itemize}
            \item One exception: notable racial animosity toward Japanese due to govt., propaganda; Pearl Harbor further confirmed
            \item Soon extended to Japanese Americans (mostly in CA); $\frac{1}{3}$ unnaturalized first-generation immigrants, $\frac{2}{3}$ native-born or naturalized U.S. citizens
            \item Japanese Americans had long-been heavily persecuted as foreign; unable to eliminate prejudice regardless of assimilation $\to$ lived in tight-knit communities often separated from white society
            \item Pearl Harbor $\to$ great fear regarding Japanese-Americans, w/ conspiracies emerging that they were planning on assisting the Japanese fighters; many viewed passivity as evidence of danger (\textit{ex}: Earl Warren)
            \item Popular culture far more accepting of Japanese than official sentiment: mil. officials blamed Japanese-Americans in Hawaii, revealed no confidence in loyalty 
            \begin{itemize}
                \item 1942: FDR created \textbf{War Relocation Authority} to intern Japanese Americans, rounding up 100k Japanese Americans and taking property; put in prison-like camps which were said to help assimilate them; really used for forced agricultural labor
                \item 1943: conditions improved w/ many young Japanese-Americans leaving for East (still mostly barred from West) to take factory jobs, join American military 
                \item 1944: Supreme Court, in \textbf{\textit{Korematsu v. U.S.}}, allowed Japanese internment; later in the same year barred internment of "loyal" citizens (subjective)
            \end{itemize}
            \item Returned to West Coast by early 1945, facing persecution, having lost businesses/property; some reparations in 1988 w/ Congress giving reparations to descendants
            \end{itemize}
            \textbf{The Japanese faced great animosity by the U.S. government: many conspiracies emerged after Pearl Harbor that they had assisted in the battle from Hawaii. FDR's War Relocation Authority permitted their internment in prison-like camps, losing their property; though some were able to depart for the East to join the military, conditions continued and the Supreme Court upheld their internment in early 1944. By early 1945, they were permitted to return to the West Coast, but for many, their former lives had been forever lost.}}
            \cornell{How did the war affect the social standing of Chinese Americans?}{\begin{itemize}
                \item War $\to$ $\uparrow$ alliance w/ China $\to$ 1943: Chinese Exclusion Act repealed, w/ quote set at 105 immigrants per year; many Chinese woman able to entrer as brides, native Chinese became citizens
                \item Racial animosity toward Chinese $\downarrow$ w/ presentation in positive contrast to Japanese as well as their assistance by taking war jobs, filling up demand for labor; many were drafted
            \end{itemize}
            \textbf{The war led to a strengthened alliance with China, ending the Chinese Exclusion Act and allowing some restricted immigration; racial animosity was reduced due to their great assistance in the war.}}
            \cornell{How did Roosevelt's policy shift away from reform?}{\begin{itemize}
                \item Roosevelt aware that victory more important than reform; realized that conservatives were attacking New Deal policies $\to$ had to back down
                \item Liberals in govt. often displaced by conservative business leaders; Congress continued to assault New Deal w/ war and end of mass unemployment as justification 
                \begin{itemize}
                    \item Ended CCC/WPA due to strengthening numbers from midterm election of 1942
                    \item Roosevelt continued to publicly fight for New Deal but accepted collapse in favor of passing his wartime policies; sought 1944 reelection
                \end{itemize}
                \item 1944 election saw Republicans hoping to exploit wartime discontent w/ Democratic policies
                \begin{itemize}
                    \item Roosevelt pressured by party to abandon controversial vice president $\to$ randomly selected Harry Truman, known for success in Senate War Investigating Committee to investigate wartime production 
                    \item Election revolved around domestic issues and president's health but campaign temporarily revived him $\to$ won reelection w/ losses in Senate but gains in House
                \end{itemize}
            \end{itemize}
            \textbf{Roosevelt was aware of significant conservative opposition toward New Deal programs; because he prioritized passing his wartime and peace-seeking policies, he quietly allowed it to collapse. Roosevelt won the 1944 election, one centered around issues of domestic policy.}}
            \cornell[The Defeat of the Axis]{How did Allied forces ultimately win the war?}{\textbf{On June 6th, 1944, D-Day began, where Allied troops attacked Normandy and took it over from the Germans; they pushed into the heart of France with little powerful resistance over the following months while the Soviets pushed from the east, ultimately winning German surrender in May 1945. In the Pacific, Allied forces pushed closer and closer to Japan; each battle, however, was terribly brutal and led to great casualties on both sides; many Americans feared this brutality would only continue in Japan out of desperation despite great loss of resources. As such, they developed the atomic bomb and, after the Japanese failed to respond to an ultimatum, they dropped two bombs within the span of a week, leading to Japanese surrender in early September 1945. The war was finally over; the U.S. emerged a dominant world power.}}
            \cornell{What characterized the liberation of France?}{\begin{itemize}
                \item Early 1944: Allied bombers attacked German industry, bombing factories in major German cities (often extended to residential areas, like the raid on Dresden)
                \item Invasion on France assisted by major bombings, air battles over Germany, breaking the Enigma code
                \item Force gathering in England for 2 yrs., made up of 3m troops, culminated in D-Day
                \begin{itemize}
                    \item June 6, 1944: \textbf{General Eisenhower} sent powerful armada across English Channel, entering (unexpected) at Normandy
                    \item Planes/shipped attacked Nazi defenses while vessels brought supplies to beaches
                    \item Paratroopers had arrived previous night \underline{behind German forces} to block off key roads
                    \item Within a week, Allied forces prevailed and German forces removed from Normandy
                \end{itemize}
                \item Progress slow for following month; late July saw Omar Bradley smashing German forces; George S. Patton moved further into France; August 25th: Paris liberated for first time; Axis driven out of France/Belgium by mid-September
                \item Rhine River saw momentary pause of Allied forces w/ Germans pushing back 55 miles through Ardennes Forest; finally stopped at Bastogne
                \item Soviet forces entered central Europe w/ offensive in Germany; ready to assault Berlin by spring 1945, Omar Bradley having reached Cologne by early March 
                \item Bradley found unguarded bridge across Rhine $\to$ rapid transfer of forces across Rhine; American forces moving rapidly eastward but paused to wait for Russians to get eastern Germany/Czechoslovakia
                \item April 30th: Adolf Hitler killed himself; May 8, 1945 saw surrender of Germans with \textbf{V-E Day}
            \end{itemize}
            \textbf{The Allies prepared for the invasion of France with bombings of German industry and carefully studying the Enigma messages. D-Day in June 1944 saw a surprise invasion of Normandy, with the Allied forces prevailing within a week; Paris was liberated by late August; the Allied forces pushed through Germany with an unexpected speed. The Soviets were making significant strides in eastern Germany; the Americans timed their attack on Berlin to wait for Soviet arrival. On May 8th, 1945, the Germans surrendered, marking the end of the war in Europe.}}
            \cornell{What characterized the ongoing battles in the Pacific?}{\begin{itemize}
                \item Feb. 1944: Chester Nimitz won victories in Marshall Islands, breaking outer portion of Japanese Empire; attacked Japanese ships w/ submarines $\to$ domestic econ. limited
                \item America found ally in China $\to$ sent \textbf{Joseph W. Stilwell} to provide supplies to China through Indian land route; Japanese attempted to obstruct but \textbf{Burma Road} finally creted by fall 1944
                \item Japanese forces threatened Burma Road, Chinese wartime capital of Chungking; \textbf{Chiang Kai-shek}, PM, unwilling to use troops against Japanese, focused more on eliminating communists (who were against the Japanese)
                \begin{itemize}
                    \item Created rift betw. Stilwell/Chiang, ultimately leading to Stilwell's departure; successors had little more success
                \end{itemize}
                \item Greatest battles were at sea: armada won at Mariana Islands; got closer to Tokyo by taking Guam, Tinian, Saipan; entered Philippines by Oct. 1944 
                \begin{itemize}
                    \item As Americans rapidly neared, Japanese poured naval resources into battle, culminating in \textbf{Battle of Leyte Gulf}, which the Americans decisively won
                    \item Japan, despite being greatly weakened, desperately increased resistance, sending \textbf{kamikaze} planes to attack Allied ships in battle for Okinawa; held off Allies for some time but ultimately captured Okinawa in June 1945
                \end{itemize}
                \item Brutal battles seemed inevitable in Japan itself; however, resources (ships and planes) greatly reduced with Americans attacking factories, will weakened with bombing of Tokyo killing civilians in firestorm; moderate leaders had long sought to surrender (but military leaders prevailed for some time)
            \end{itemize}
            \textbf{In the Pacific, the U.S. won several naval battles, seizing many islands and nearing closer to Japan; though Japanese resistance was continually strong despite their limited resources, the Americans continued to win decisively (with great losses). Many Americans feared that such brutal battles would play out in Japan, too; however, the limited Japanese resources and the crushed will of most moderate leaders gave many hope for Japanese surrender.}}
            \cornell{What characterized the Manhattan Project?}{\begin{itemize}
                \item 1939: U.S. received reports that Germany had begun to develop atomic bomb $\to$ entered race to create before Germans
                \item Motivation for atomic bomb came from Einstein, with relativity arguing potential for built up matter to create tremendous energy; Einstein reported to FDR that Germans were working on bomb, pushing for uranium-based bomb to split uranium, creating nuclear chain reaction
                \item \textbf{Enrico Fermi} discovered uranium radioactivity in 1930s; Niels Bohr sent details of German radioactivity experiments to U.S. in 1939
                \item Army soon took control of research, appointing Leslie Groves to reorganize \textbf{Manhattan Project}; poured \$2b w/ hidden laboratories in Oak Ridge, TN; Los Alamos, NM, WA
                \begin{itemize}
                    \item Oak Ridge researchers discovered plutonium; Los Alamos charged with creating bomb
                    \item Pushed ahead rapidly, with Los Alamos researchers testing first atomic bomb in mid-July 1945, testing \textbf{Trinity Bomb}
                \end{itemize}
            \end{itemize}
            \textbf{The Manhattan Project emerged out of fear that the Germans were developing a dangerous atomic bomb; sponsored by the government, it made rapid progress and ultimately developed a plutonium bomb tested in mid-1945.}}
            \cornell{What characterized atomic warfare?}{\begin{itemize}
                \item Truman learned of bomb $\to$ issued ultimatum to Japanese, forcing surrender by August 3rd or bomb would be dropped
                \begin{itemize}
                    \item Premier wanted to accept Allied request; could not convince mil. leaders 
                    \item Some word of potential surrender, but U.S. wanted it \underline{unconditional}
                \end{itemize}
                \item Reached deadline w/o surrender $\to$ Truman ordered dropping of bomb
                \begin{itemize}
                    \item Some argue bomb was unnecessary with surrender close; others felt mil. leaders woudl never hadve surrenders
                    \begin{itemize}
                        \item Many felt regardless of intentions, weapon far too unethical to use
                    \end{itemize}
                    \item Truman saw as simple military decision to give edge in war 
                    \item Others believe that Soviet poised to enter Pacific War $\to$ Truman feared communist presence in Asia $\to$ wanted to end war early (little direct evidence to support or refute)
                    \item August 6, 1945, American B-29 dropped on \textbf{Hiroshima}, destroying four-square-mile area, killing 80k civilians and many more w/ radioactive fallout
                    \item Japanese govt. unable to reach consensus $\to$ second bomb dropped on \textbf{Nagasaki} on August 14 $\to$ govt. surrendered on September 2, 1945
                \end{itemize}
                \item War had concluded w/ bombing of Japan $\to$ U.S. had become major world power; world still faced uncertainty with USSR-US tensions; few could forget dramatic war casualties 
            \end{itemize}
            \textbf{After the bomb was developed, Truman issued an ultimatum to Japan, requiring surrender by August 3rd; when no surrender agreements were made, he controversially dropped a bomb on Hiroshima and then on Nagasaki; hundreds of thousands were killed and the areas were devastated by radiocative fallout. Japan finally surendered on September 2nd, 1945.}}
    \end{document}