\documentclass[a4paper]{article}
    \usepackage{tcolorbox}
    \usepackage{amsmath}
    \tcbuselibrary{skins}
    
    \title{
    \vspace{-3em}
    \begin{tcolorbox}
    \Huge\sffamily \begin{center} Chapter 2  \mbox{} \\ \huge Transplantations and Borderlands \mbox{} \\
    \LARGE Finn Frankis \mbox{} \\
    \Large AP US History - September 5{$^\text{nd}$}, 2018 \end{center} 
    \end{tcolorbox}
    \vspace{-3em}
    }
    \date{}
    \author{}
    
    \usepackage{background}
    \SetBgScale{1}
    \SetBgAngle{0}
    \SetBgColor{red}
    \SetBgContents{\rule[0em]{4pt}{\textheight}}
    \SetBgHshift{-2.3cm}
    \SetBgVshift{0cm}
    \usepackage[margin=2cm]{geometry} 
    
    \makeatletter
    \def\cornell{\@ifnextchar[{\@with}{\@without}}
    \def\@with[#1]#2#3{
    \begin{tcolorbox}[enhanced,colback=gray,colframe=black,fonttitle=\large\bfseries\sffamily,sidebyside=true, nobeforeafter,before=\vfil,after=\vfil,colupper=blue,sidebyside align=top, lefthand width=.3\textwidth,
    opacityframe=0,opacityback=.3,opacitybacktitle=1, opacitytext=1,
    segmentation style={black!55,solid,opacity=0,line width=3pt},
    title=#1
    ]
    \begin{tcolorbox}[colback=red!05,colframe=red!25,sidebyside align=top,
    width=\textwidth,nobeforeafter]#2\end{tcolorbox}%
    \tcblower
    \sffamily
    \begin{tcolorbox}[colback=blue!05,colframe=blue!10,width=\textwidth,nobeforeafter]
    #3
    \end{tcolorbox}
    \end{tcolorbox}
    }
    \def\@without#1#2{
    \begin{tcolorbox}[enhanced,colback=white!15,colframe=white,fonttitle=\bfseries,sidebyside=true, nobeforeafter,before=\vfil,after=\vfil,colupper=blue,sidebyside align=top, lefthand width=.3\textwidth,
    opacityframe=0,opacityback=0,opacitybacktitle=0, opacitytext=1,
    segmentation style={black!55,solid,opacity=0,line width=3pt}
    ]
    
    \begin{tcolorbox}[colback=red!05,colframe=red!25,sidebyside align=top,
    width=\textwidth,nobeforeafter]#1\end{tcolorbox}%
    \tcblower
    \sffamily
    \begin{tcolorbox}[colback=blue!05,colframe=blue!10,width=\textwidth,nobeforeafter]
    #2
    \end{tcolorbox}
    \end{tcolorbox}
    }
    \makeatother

    \parindent=0pt
    
    \begin{document}
    \maketitle
    \SetBgContents{\rule[0em]{4pt}{\textheight}}
    \cornell[Key Concepts]{What are this chapter's key concepts?}{\begin{itemize}
        \item \textbf{2.1.I.C:} English colonization attracted a large number of Europeans (mostly British) seeking social mobility, economic prosperity, religious freedom, improved living conditions; focused on agriculture on land taken from isolated natives
        \item \textbf{2.1.III.B:} Natives continually traded with European settlements, leading to cultural/economic changes and the rapid spread of disease
        \item \textbf{2.1.III.E:} Conflict with natives over land/resources/boundaries sparked military confrontations
        \item \textbf{2.2.I.A:} The unification of mixed ethnic groups through colonization increased intellectual exchange, pluralism enhanced by Enlightenment
        \item \textbf{2.2.II.A:} All colonies participated in slave trade (land abundance); New England/port cities used small numbers of slaves while Chesapeake, southern Atlantic coast had large numbers; greatest numbers sent to West Indies
    \end{itemize}}
    \cornell[The Early Chesapeake]{What were the characteristics of the early Chesapeake settlement?}{\textbf{The initial Chesapeake settlement, was centered in Virginia (led by the Virginia Company and later the crown) and Maryland (led by the Calverts and suffered major religious tensions between Catholics and protestants). Tobacco proved to be an extremely important crop, and, in Virginia especially, tensions with the natives led to great unrest in society, including the Bacon Rebellion.}}
    \cornell{What were the initial relations between the colonists and the natives in Jamestown?}{\begin{itemize}
        \item Journey of 144 men dwindled to 104 by arrival; named colony Jamestown in honor of King James
        \item Initial setup had multiple problems
        \begin{itemize}
            \item Selected swampy peninsula for security from natives, but extremely difficult to harvest; disease rampant
            \item Brought no women, reducing focus on community-building and more on hunt for gold
        \end{itemize}
        \item Colonist survival due to natives' showing agricultural techniques, important crops (\textbf{maize}), technology like canoes for river navigation
        \begin{itemize}
            \item Despite this, English insisted on inferiority of natives, calling them "savages"
        \end{itemize}
        \item Jamestown remained small for over a decade w/ natives far more powerful than colonists, united in large confederacy
        \begin{itemize}
            \item Within a few months, only 38 men remained alive
            \item Colony owed survival to Captain John Smith, age 27; known for powerful leadership (often at native expense)
        \end{itemize}
    \end{itemize}
    \textbf{Jamestown's survival was almost entirely due to the kindness of the natives, showing the colonists agricultural techniques, new crops, and technology. However, the colonists continually referred to the natives as "savages," despite being overwhelmed in population. The colony's population began to dwindle, but it was kept afloat by Captain John Smith.}}
    \cornell{How was Jamestown revived?}{\begin{itemize}
        \item London Company did not give up, obtaining new charter and sending nine new vessels to Jamestown
        \begin{itemize}
            \item Hit by disaster: one vessel ran aground in Bermuda Islands, other lost in hurricane 
            \item New settlers succumbed to fever by wintertime; after natives had realized threat posed by colonists, blocked off from further expansion, food 
        \end{itemize}
        \item Settlers endured terrible winter between 1609 and 1610
        \begin{itemize}
            \item When ship in Bermuda arrived in May, found only 60 people alive, all on the brink of death
            \item Departed for England but ran into another ship on James River led by e La Warr, new governor
            \begin{itemize}
                \item Convinced to return to colony, relief expeditions allowed for beginning of thriving time
            \end{itemize}
        \end{itemize}
        \item Settlers focused on cultivating tobacco, new crop to English
        \begin{itemize}
            \item First profitable crop; encouraged planters to move inland
        \end{itemize}
        \item Working conditions initially extremely harsh for settlers under first few governors, with little clear incentive
        \begin{itemize}
            \item As many began to deliberately dodge work despite punishment of potential death, Governor Thomas Dale allowed private ownership to provide personal incentive
            \item Landowners repaid company with part-time work, grain contributions 
        \end{itemize}
        \item Despite initial rocky leadership, Virginia began to expand, with settlements create beyond Jamestown
        \begin{itemize}
            \item Order imposed by governors essential to success along with profitable tobacco
        \end{itemize}
    \end{itemize}
    \textbf{Attempts by the London Company to continue sending vessels eventually failed, with native blockades reducing the settlers' food supply. Things were turned around when De La Warr, the first governor, arrived, encouraging all settlers to stay and imposing a harsh regime requiring all to work diligently. Although his successors eventually reduced his harsh policies, his rule, along with the importance of the tobacco crop, were critical to Jamestown's success.}}
    \cornell{What was the importance of tobacco farming in Virginia?}{\begin{itemize}
        \item Tobacco was initially discovered by Columbus w/ Cuban natives; had developed large stigma in England
        \begin{itemize}
            \item James I led campaign against it, touting connection to natives, cause of diseases, and profitability for Spanish
        \end{itemize}
        \item John Rolfe of Jamestown experimented w/ native-grown form of tobacco
        \begin{itemize}
            \item Especially harsh, leading to high quality tobacco
            \item Found many English buyers, spreading tobacco farming throughout Chesapeake area
        \end{itemize}
        \item Growth of tobacco farming required territorial expansion, further into native-owned territory
    \end{itemize}
    \textbf{Although tobacco was initially extremely stigmatic in England, even disapproved of by James I, John Rolfe's successful strain (from the natives) led to high demand for English-grown tobacco, turning it into a lucrative colonial industry.}}
    \cornell{What characterized the Virginia settlement after the emergence of tobacco farming?}{\begin{itemize}
        \item Virginia Company, despite emergence of tobacco farming, continued to grow in debt, launching program promoting migration
        \begin{itemize}
            \item Focused on "headright" system, based on land grants to new migrants based on size of family, number of sponsored immigrants
            \begin{itemize}
                \item Allowed many to establish large plantations with servants
            \end{itemize}
            \item Brought over ironworkers, craftsmen, women (generally between free and indentured) to diversify economy
            \begin{itemize}
                \item Promised all new settlers full rights of English citizen, end to strict rule
            \end{itemize}
        \end{itemize}
        \item Meeting in Jamestown church where delegates from each community met as House of Burgesses marked first meeting of elected legislature in U.S. (July 1619)
        \item In August 1619, first Africans came by boat
        \begin{itemize}
            \item Initially not likely treated with full harshness of slaves
            \item Use of black labor limited until 1670s: white indentured servants preferred (eventually scarce, expensive)
        \end{itemize}
        \item Relationships with natives became increasingly strained
        \begin{itemize}
            \item Governor Dale led continues assults against Powhatan group, kidnapping Pocohontas (daughter of chief) and converting her to Christianity; married John Rolfe
            \item Although Pocohontas' seizure led to reduced attacks temporarily; after brother took over, attacks continued
            \begin{itemize}
                \item Called settlers for trading, massacring 347 whites of all ages, sexes
                \item Failed uprising later pushed Powhatans to finally give up
            \end{itemize}
        \end{itemize}
        \item Virginia Company, amidst all these events, had become defunct
        \begin{itemize}
            \item Colony transferred to ownership of Crown
        \end{itemize}
    \end{itemize}
    \textbf{The Virginia Company continued to grow in debt despite the emergence of tobacco farming, forcing them to seek out new settlers and modify their ruling policies to be more democractic. With the settlers' relations with the natives becoming strained, climaxing in the massacre of 347 whites, the Virginia Company went defunct, leading to the colony's transfer into the hands of the crown.}}
    \cornell{What were the important agricultural techniques which the British learned from the natives?}{\begin{itemize}
        \item Despite continual hostility to natives for backwards technology (even blamed for inability to find gold), relied on techniques to farm New World soil 
        \begin{itemize}
            \item Natives had extremely successful farms
        \end{itemize}
        \item Despite not adopting all (like field clearing), learned a great deal, especially importance of maize for high yield, sugar source
    \end{itemize}
    \textbf{Although the British refused to learn a significant amount from the natives due to their insisted backwards technology, they most significantly learned about the importance of maize in the North American soil.}}
    \cornell{What marked the development of society in Maryland?}{\begin{itemize}
        \item Founded by son of Catholic George Calvert as retreat for English Catholics seeking refuge from Anglican-dominated England
        \begin{itemize}
            \item Lord Baltimore (Cecilius, son of Calvert) received grant carrying remarkable power, sent brother Leonard with 200-300 passengers
            \item First village: St. Mary's (named after Queen); native focus on rival tribes allowed Maryland settlers to experience no assaults/plagues/starving
        \end{itemize}
        \item To attract settlers, Calverts understood need to abandon Catholic emphasis, adopting policy of toleration for all
        \begin{itemize}
            \item Population quickly dominated by Protestants, soon appointed as governor 
            \item Political relations remained tense: Protestant majority banned Catholics from voting, repealed Toleration Act
        \end{itemize}
        \item Labor shortage encouraged "headright" system like in Virginia
        \begin{itemize}
            \item Major land grants to settlers; focus on tobacco cultivation, eventually driven by African slaves
        \end{itemize}
    \end{itemize}
    \textbf{Maryland was founded by the Calverts, a Catholic family hoping to create a place away from the Anglican-dominated England. However, when the migrants became majority Protestant, they adopted a policy of religious toleration. Politics quickly became turbulent, with the Protestant majority often discriminating against Catholics.}}
    \cornell{What were the major political tensions in mid-17th century Virginia?}{\begin{itemize}
        \item Having survived key initial hardships and expansion, Virginia began to take on political issues
        \item In 1642, King Charles I appointed William Berkeley as governor, who remained in power to 1670
        \begin{itemize}
            \item Initially popular for expansion rounds, defeating of natives in battle
            \item Part of native defeat involved large territory boost for settlers, but also promise not to expand beyond a certain line
            \begin{itemize}
                \item Challenge created by rapidly growing population (due to success of Oliver Cromwell leading his opponents to flee to colonies)
                \item Established three counties in territory designated for natives only
            \end{itemize}
            \item Berkeley soon ecame autocrat, reducing power to vote to landowners, keeping burgesses in power from year-to-year 
            \begin{itemize}
                \item Led to underrepresentation of those in "backcountry"
            \end{itemize}
        \end{itemize}
        \item By 1670s, many indentured servants had completed terms; left without home/money
                \begin{itemize}
                    \item Led to stealing, begging, working throughout colony
                \end{itemize}
    \end{itemize}
    \textbf{Although William Berkeley was initially a popular governor due to his territorial expansion, his growing autocratic regime began to lead to great discontent due to the underrepresentation of those not connected to him. Furthermore, expired indentured servants began to roam the colony without money or a home.}
    }
    \cornell{What was Bacon's rebellion?}{\begin{itemize}
        \item Nathaniel Bacon, wealthy young university graduate, arrived in Virginia as member of backcountry gentry (in west)
        \item Bacon disagreed with eastern leaders most significantly on native policy
        \begin{itemize}
            \item More directly threatened by native presence, leading him (and other western landowners) to push line of settlement further
            \item Unhappy with Berkeley's choice to exclude him from inner circle governor's council, fur trade
        \end{itemize}
        \item When angry natives struck against western plantation, local groups retaliated, leading to heavy response from natives
        \begin{itemize}
            \item Bacon became natural leader, defying Berkeley to attack Indians: declared as group of rebels 
            \item Transitioned to attack against colonial government
            \begin{itemize}
                \item Most powerful insurrection against authority in colonial history
            \end{itemize}
        \end{itemize}
        \item Bacon led army east, first winning temporary pardon but eventually (after pardon was not honored), poised to take over Jamestown; abruptly died of dysentery
        \begin{itemize}
            \item Troops defeated by arrival of British backup
            \item Did lead to Indian signing of new treaty allowing additional lands for settlement (aware of military power of settler forces)
        \end{itemize}
        \item Rebellion significant for symbolizing settlers' inability to abide by agreements with natives, natives' inability to tolerate additional expansion
        \begin{itemize}
            \item Revealed bitter competition between eastern and western landowners, potential for great instability in colony pushed by landless men
            \item Risk of social unrest from former indentured servants was one of many reasons for promotion of African slave trade (removal of indentured servants)
        \end{itemize}
    \end{itemize}
    \textbf{Bacon, an eastern landowner in Virginia, rebelled against Berkeley by attacking the hostile natives on the western border. He then led a revolution in Jamestown which, although eventually crushed by British troops, represented, in all, the tensions between natives and settlers (especially the inability to abide by agreements), competition between eastern and western landowners, and the risks posed by indentured servants.}}
    \cornell[The Growth of New England]{What characterized the growth of New England?}{\textbf{The New England settlement began with a group of discontent Puritan Separatists fleeing persecution at the hands of the government, for freedom of worship. In summary, New England contained the key colonies of Massachusetts, Connecticut, and Rhode Island, and had a very powerful Puritan undertone, though some areas (like Rhode Island), promoted religious tolerance.}}
    \cornell{What were the key traits of the Plymouth plantation?}{\begin{itemize}
        \item Separatists began to depart England quietly for Holland (known for toleration), but poor job opportunities led many to consider travel to New World
        \item Scrooby group (group of Separatists) received permission from Virginia Company, slight support of king to settle in British America
        \begin{itemize}
            \item Funded by merchants, hoping for funds at end of seven years
            \item Puritans fviewed themselves as pilgrims; travelled from Plymouth on \textit{Mayflower}
            \item Long trip forced early settlement on Cape Cod, near Plymouth (established by John Smith years earlier)
            \begin{itemize}
                \item Outside of London Company's territory; despite no legal basis, established civil government and proclaimed allegiance to king 
            \end{itemize}
        \end{itemize}
        \item Puritans settled on deserted native village (devastated by plague likely brought by Europeans), faced challenging first winter 
        \begin{itemize}
            \item William Bradford, leader in England, became colony governor; faced numerous personal hardships but remained strong
            \item Dramatically changed landscape, including decrease in wild animal population (demand for furs, skins, meats); nearly entirely eliminated native population through smallpox
            \begin{itemize}
                \item Farmed mix of native crops (corn, potatoes peas) and English crops (wheat, barley, oats)
            \end{itemize}
        \end{itemize}
        \item Experience with natives very different due to devastated population
        \begin{itemize}
            \item Natives understood importance of cooperation, assisted settlers in gathering seafood, cultivating corn, hunting local animals
            \begin{itemize}
                \item Key helpers: Squanto and Samoset (Squanto spoke English due to previous capture by English, time in Europe)
                \item Marked allegiance to natives through invitation to first Thanksgiving in 1621
            \end{itemize}
            \item Good relationship did not last long: second smallpox epidemic wiped out most remaining
        \end{itemize}
        \item Organization poor and profits low until arrival of military officer Miles Standish, leading to trading surplus
        \begin{itemize}
            \item Fur trade emerged with natives of Maine, population grew
        \end{itemize}
        \item "Plymouth Plantation" selected Bradford as governor once again, who persuaded Council for New England (successor to Plymouth Company) to allow legal permission to inhabit lands
        \begin{itemize}
            \item Ended harsh regime, communal labor plan of Standish
            \item Distributed land among families, eventually able to pay off financiers with wealth from fur trade
        \end{itemize}
        \item Pilgrims continually poor, but clung to God's word
        \begin{itemize}
            \item Cared less about how they were viewed by others than Puritans settling futher north
        \end{itemize}
    \end{itemize}
    \textbf{The Plymouth plantation, founded by pilgrim Separatists, was known for a stronger relationship with the natives and gradual productivity through the fur trade and corn farming. Despite being continually somewhat poor, they remained strong in their conviction that God had called for a New World settlement.}}
    \cornell{What was the Puritan Experiment?}{\begin{itemize}
        \item Tension created by King James and later Charles I (by disbanding Parliament, destroying nonconformity)
        \begin{itemize}
            \item Puritans established Massachusetts Bay Company with charter from Charles to establish colony in New World (did not reveal that they were Puritans)
            \begin{itemize}
                \item Supplies provided by defunct fishing/trading company 
            \end{itemize}
        \end{itemize}
        \item Many Puritans viewed enterprise in new colony more than business venture: as haven for religious freedom
        \begin{itemize}
            \item Members began to move en masse to America
            \item John Winthrop chosen to be initial governor for affluence, education, piety, power
            \begin{itemize}
                \item Led first initial migration, bringing mostly family groups 
                \item Carried charter to Massachusetts Bay Company: colonists responsible not to anyone in England, but to themselves
            \end{itemize}
            \item Numerous settlements emerged: Boston (HQ, port), other towns throughout area
        \end{itemize}
        \item Established colonial government initially with eight stockholders, but later all male citizens
        \begin{itemize}
            \item Winthrop eventually made to force election each year for role of governor 
        \end{itemize}
        \item Massachusetts founders had no intention from breaking away from English Church
        \begin{itemize}
            \item Had no remaining allegiance, but wanted to remain covert, with liberty to stand alone 
            \item Formed Congregational Church where each church has complete power
        \end{itemize}
        \item Puritans worshippers not of traditional faith of Anglican Church but instead emphasis on personal knowledge/belief of ministers, direct reading of Bible, John Calvin
        \begin{itemize}
            \item Exercised religious authority: dissidents in Massachusetts had no more freedom than the Puritans in England 
            \item Authority stemmed from individual communities, leading to pious society which many hoped would be beacon for New World
        \end{itemize}
        \item Political structure somewhat theocratic: church members were the only people who could vote/hold office, influenced heavily by ministers
        \begin{itemize}
            \item Government taxed all members, protected ministers
        \end{itemize}
        \item Colony had initial difficulties of winter, but greater number of families led to stronger community, more prosperous in long term
        \begin{itemize}
            \item Relied on natives for food/advice as well as Pilgrims
        \end{itemize}
    \end{itemize}
    \textbf{The Massachusetts colony, founded by covert Puritans in England, strongly emphasized the Puritan faith with a pious and serious society. They had an advanced theocratical political system, including elections, positions of office, and ministers. Their society grew rapidly due to the emphasis on family migrations and the advice from the natives/Pilgrims.}}
    \cornell{What characterized the expansion of New England?}{\begin{itemize}
        \item Because Massachusetts required all voters to be Puritan (otherwise they were forced to leave), many began to depart and expand outward into the New England area
        \item Connecticut Valley attracted numerous English families
        \begin{itemize}
            \item Thomas Hooker defined Massachusetts to created Hartford with colonial government, constitution
            \item Puritan minister, wealthy English merchant created New Haven
            \begin{itemize}
                \item Focused on combatting religious laxity in Massachusetts: extremely strict religious government
                \item Later combined with Hartford by royal decree 
            \end{itemize}
        \end{itemize}
        \item Rhode Island originated in Roger Williams, an amicable Separatist
        \begin{itemize}
            \item Demanded that Massachusetts abandon all allegiance to Anglicans, separation between church and state; led to banishment
            \item After spending time with Narragansett tribesmen, brought followers to Providence, later receiving charter allowing government
            \item Rhode Island was only colony for some time which emphasized complete religious tolerance
        \end{itemize}
        \item Anne Hutchinson posed greatest challenge to Massachusetts order
        \begin{itemize}
            \item Argued that many members of Massachusetts clergy not part of "elect" (or true conversion experience), no right to hold office
            \begin{itemize}
                \item Charged that all ministers were not among the elect
            \end{itemize}
            \item Focused on proper role of women in Puritan society
            \begin{itemize}
                \item Powerful religious figure
                \item Developed large following among women, merchants, young men, numerous dissidents 
            \end{itemize}
            \item Massachusetts began to observe threat: Hutchinson's followers prevented Winthrop's reelection as governor
            \begin{itemize}
                \item Next year, returned to office; banished Hutchinson for heresy (despite remarkable theological knowledge)
                \item Moved to Rhode Island, New Netherland with followers; later died during native uprising
            \end{itemize}
            \item In Massachusetts, clergy began to further restrict role of women in response to Hutchinson, leading many of her followers to depart, mostly to New Hampshire and Maine
            \begin{itemize}
                \item Colonies had been already established but had failed due to few settlers
                \item Boomed after Hutchinson's followers departed en masse, began to populate region
            \end{itemize}
        \end{itemize}
    \end{itemize}
    \textbf{As Massachusetts began to hone in on their religious strictness, many began to depart (often forced). Examples include in Connecticut, with Hartford and strictly religious New Haven, in Rhode Island, with Providence (separated church and state), and Maine and New Hampshire, where the followers of the influential yet controversial Puritan woman Anne Hutchinson eventually travelled to.}}
    \cornell{What were the key relations between settlers and natives in New England?}{\begin{itemize}
        \item Population initially very small due to epidemics; surviving natives had sold much of their land, converted to Christianity 
        \item Native advice and presence crucial to early success of nearly all colonies
        \begin{itemize}
            \item Taught about local food crops, techniques (annual burning for fertilization, beans to replenish soil)
            \item Served as trading partners, particularly in fur trade, manufactured goods (iron pots, arrows, guns, alcohol)
            \begin{itemize}
                \item Commerce w/ natives created wealthiest families
            \end{itemize}
        \end{itemize}
        \item Tensions began to develop as settlers continually expanded land due to agragian economy (domesticated animals as wild ones disappeared)
        \begin{itemize}
            \item Brutality of conflicts encouraged Puritans to view natives as "savages"
            \begin{itemize}
                \item Some sought to "civilize" through conversion (translation of Bible by John Eliot)
                \item Others believed in extermination or displacement
            \end{itemize}
            \item Natives felt English were land-hungry; they frequently let their livestock run wild, destroying crops
            \begin{itemize}
                \item Led to numerous land/food shortages
                \item Decline led to great despair, which often promoted alcoholism
            \end{itemize}
        \end{itemize}
    \end{itemize}
    \textbf{In New England, although the natives were initially essential to the development of society through commerce and knowledge (about crops, agricultural techniques), tensions rose as the settlers continually expanded. While the settlers wanted to "civilize" the natives through Christianity, the natives simply felt the English were land-hungry and their food shortages led to great social despair.}
    }
    \cornell{How did technology play a part in the Pequot War and King Philip's War?}{\begin{itemize}
        \item Pequot War emerged as competition over trade with Dutch, friction over land
        \begin{itemize}
            \item English allied with Mohegan, Narragansett
            \item Most violent act: English setting fire to Pequot stronghold, killing hundreds of natives
            \begin{itemize}
                \item Marked end of war: tribe nearly completely wiped out
            \end{itemize}
        \end{itemize}
        \item Most prolonged encounter: King Philip's War
        \begin{itemize}
            \item Wampanoags, under leader known as King Philip to English (Metacomet) began to resist English, fearing incursion of English law, taking of lands
            \begin{itemize}
                \item Terrorized Massachusetts towns for three years, using organizational skills and guns
            \end{itemize}
            \item Massachusetts society heavily weakened; received aid from Mohawks, rivals of Wampanoags, and converted spies
            \begin{itemize}
                \item White militiamen attacked Indian villages, native food supplies; Mohawks killed and beheaded Metacomet, delivering head to leaders
                \begin{itemize}
                    \item Metacomet's alliance collapsed, Wampanoag tribe had ended, with leaders sold to slavery or executed
                \end{itemize}
            \end{itemize}
        \end{itemize}
        \item Natives continued to attack English colonies; New England settlers also began to face competition from Dutch/French
        \begin{itemize}
            \item French posed threat by allying with Algonquins, later attacking English
        \end{itemize}
        \item Wars with natives heavily characterized by undertone of technology exchange
        \begin{itemize}
            \item While English settlers took time to adapt to new flintlock rifle (far more efficient than matchlock), natives adopted immediately and quickly taught themselves how to handle them
            \begin{itemize}
                \item Built forge for handling, repairing rifles
            \end{itemize}
            \item Natives also relied on traditional technologies
            \begin{itemize}
                \item Narragansetts built enormous fort in Great Swamp of Rhode Island, side of bloody battle (burned down); later built stone fort
            \end{itemize}
            \item Technology ultimately proved no match for numbers, firepower of English settlers
        \end{itemize}
    \end{itemize}
    \textbf{The Pequot War was sparked by a trade competition with the Dutch and friction over land. King Philip's War, with the Wampanoags, was initiated by the natives, but the English quickly retaliated by allying with the Mohawks, wiping out the tribe. Technology played a major part in both wars: the natives had received rifles in trade and taught themselves how to use them, often using more advanced rifles than the English soldiers themselves.}}
    \end{document} 