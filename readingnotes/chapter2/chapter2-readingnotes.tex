\documentclass[a4paper]{article}
    \usepackage{tcolorbox}
    \usepackage{amsmath}
    \tcbuselibrary{skins}
    
    \title{
    \vspace{-3em}
    \begin{tcolorbox}
    \Huge\sffamily \begin{center} Chapter 2  \mbox{} \\ \huge Transplantations and Borderlands \mbox{} \\
    \LARGE Finn Frankis \mbox{} \\
    \Large AP US History - September 5{$^\text{nd}$}, 2018 \end{center} 
    \end{tcolorbox}
    \vspace{-3em}
    }
    \date{}
    \author{}
    
    \usepackage{background}
    \SetBgScale{1}
    \SetBgAngle{0}
    \SetBgColor{red}
    \SetBgContents{\rule[0em]{4pt}{\textheight}}
    \SetBgHshift{-2.3cm}
    \SetBgVshift{0cm}
    \usepackage[margin=2cm]{geometry} 
    
    \makeatletter
    \def\cornell{\@ifnextchar[{\@with}{\@without}}
    \def\@with[#1]#2#3{
    \begin{tcolorbox}[enhanced,colback=gray,colframe=black,fonttitle=\large\bfseries\sffamily,sidebyside=true, nobeforeafter,before=\vfil,after=\vfil,colupper=blue,sidebyside align=top, lefthand width=.3\textwidth,
    opacityframe=0,opacityback=.3,opacitybacktitle=1, opacitytext=1,
    segmentation style={black!55,solid,opacity=0,line width=3pt},
    title=#1
    ]
    \begin{tcolorbox}[colback=red!05,colframe=red!25,sidebyside align=top,
    width=\textwidth,nobeforeafter]#2\end{tcolorbox}%
    \tcblower
    \sffamily
    \begin{tcolorbox}[colback=blue!05,colframe=blue!10,width=\textwidth,nobeforeafter]
    #3
    \end{tcolorbox}
    \end{tcolorbox}
    }
    \def\@without#1#2{
    \begin{tcolorbox}[enhanced,colback=white!15,colframe=white,fonttitle=\bfseries,sidebyside=true, nobeforeafter,before=\vfil,after=\vfil,colupper=blue,sidebyside align=top, lefthand width=.3\textwidth,
    opacityframe=0,opacityback=0,opacitybacktitle=0, opacitytext=1,
    segmentation style={black!55,solid,opacity=0,line width=3pt}
    ]
    
    \begin{tcolorbox}[colback=red!05,colframe=red!25,sidebyside align=top,
    width=\textwidth,nobeforeafter]#1\end{tcolorbox}%
    \tcblower
    \sffamily
    \begin{tcolorbox}[colback=blue!05,colframe=blue!10,width=\textwidth,nobeforeafter]
    #2
    \end{tcolorbox}
    \end{tcolorbox}
    }
    \makeatother

    \parindent=0pt
    
    \begin{document}
    \maketitle
    \SetBgContents{\rule[0em]{4pt}{\textheight}}
    \cornell[Key Concepts]{What are this chapter's key concepts?}{\begin{itemize}
        \item \textbf{2.1.I.C:} English colonization attracted a large number of Europeans (mostly British) seeking social mobility, economic prosperity, religious freedom, improved living conditions; focused on agriculture on land taken from isolated natives
        \item \textbf{2.1.III.B:} Natives continually traded with European settlements, leading to cultural/economic changes and the rapid spread of disease
        \item \textbf{2.1.III.E:} Conflict with natives over land/resources/boundaries sparked military confrontations
        \item \textbf{2.2.I.A:} The unification of mixed ethnic groups through colonization increased intellectual exchange, pluralism enhanced by Enlightenment
        \item \textbf{2.2.II.A:} All colonies participated in slave trade (land abundance); New England/port cities used small numbers of slaves while Chesapeake, southern Atlantic coast had large numbers; greatest numbers sent to West Indies
    \end{itemize}}
    \cornell[The Early Chesapeake]{What were the characteristics of the early Chesapeake settlement?}{}
    \cornell{What were the initial relations between the colonists and the natives in Jamestown?}{\begin{itemize}
        \item Journey of 144 men dwindled to 104 by arrival; named colony Jamestown in honor of King James
        \item Initial setup had multiple problems
        \begin{itemize}
            \item Selected swampy peninsula for security from natives, but extremely difficult to harvest; disease rampant
            \item Brought no women, reducing focus on community-building and more on hunt for gold
        \end{itemize}
        \item Colonist survival due to natives' showing agricultural techniques, important crops (\textbf{maize}), technology like canoes for river navigation
        \begin{itemize}
            \item Despite this, English insisted on inferiority of natives, calling them "savages"
        \end{itemize}
        \item Jamestown remained small for over a decade w/ natives far more powerful than colonists, united in large confederacy
        \begin{itemize}
            \item Within a few months, only 38 men remained alive
            \item Colony owed survival to Captain John Smith, age 27; known for powerful leadership (often at native expense)
        \end{itemize}
    \end{itemize}
    \textbf{Jamestown's survival was almost entirely due to the kindness of the natives, showing the colonists agricultural techniques, new crops, and technology. However, the colonists continually referred to the natives as "savages," despite being overwhelmed in population. The colony's population began to dwindle, but it was kept afloat by Captain John Smith.}}
    \end{document} 