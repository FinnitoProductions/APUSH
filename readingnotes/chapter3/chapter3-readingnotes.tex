\documentclass[a4paper]{article}
    \usepackage[T1]{fontenc}
    \usepackage{tcolorbox}
    \usepackage{amsmath}
    \tcbuselibrary{skins}
    
    \title{
    \vspace{-3em}
    \begin{tcolorbox}
    \Huge\sffamily \begin{center} Chapter 3  \mbox{} \\ \huge Society and Culture in Provincial America \mbox{} \\
    \LARGE Finn Frankis \mbox{} \\
    \Large AP US History - September 8{$^\text{th}$}, 2018 \end{center} 
    \end{tcolorbox}
    \vspace{-3em}
    }
    \date{}
    \author{}
    
    \usepackage{background}
    \SetBgScale{1}
    \SetBgAngle{0}
    \SetBgColor{red}
    \SetBgContents{\rule[0em]{4pt}{\textheight}}
    \SetBgHshift{-2.3cm}
    \SetBgVshift{0cm}
    \usepackage[margin=2cm]{geometry} 
    
    \makeatletter
    \def\cornell{\@ifnextchar[{\@with}{\@without}}
    \def\@with[#1]#2#3{
    \begin{tcolorbox}[enhanced,colback=gray,colframe=black,fonttitle=\large\bfseries\sffamily,sidebyside=true, nobeforeafter,before=\vfil,after=\vfil,colupper=blue,sidebyside align=top, lefthand width=.3\textwidth,
    opacityframe=0,opacityback=.3,opacitybacktitle=1, opacitytext=1,
    segmentation style={black!55,solid,opacity=0,line width=3pt},
    title=#1
    ]
    \begin{tcolorbox}[colback=red!05,colframe=red!25,sidebyside align=top,
    width=\textwidth,nobeforeafter]#2\end{tcolorbox}%
    \tcblower
    \sffamily
    \begin{tcolorbox}[colback=blue!05,colframe=blue!10,width=\textwidth,nobeforeafter]
    #3
    \end{tcolorbox}
    \end{tcolorbox}
    }
    \def\@without#1#2{
    \begin{tcolorbox}[enhanced,colback=white!15,colframe=white,fonttitle=\bfseries,sidebyside=true, nobeforeafter,before=\vfil,after=\vfil,colupper=blue,sidebyside align=top, lefthand width=.3\textwidth,
    opacityframe=0,opacityback=0,opacitybacktitle=0, opacitytext=1,
    segmentation style={black!55,solid,opacity=0,line width=3pt}
    ]
    
    \begin{tcolorbox}[colback=red!05,colframe=red!25,sidebyside align=top,
    width=\textwidth,nobeforeafter]#1\end{tcolorbox}%
    \tcblower
    \sffamily
    \begin{tcolorbox}[colback=blue!05,colframe=blue!10,width=\textwidth,nobeforeafter]
    #2
    \end{tcolorbox}
    \end{tcolorbox}
    }
    \makeatother

    \parindent=0pt
    
    \begin{document}
    \maketitle
    \SetBgContents{\rule[0em]{4pt}{\textheight}}
    \cornell[Key Concepts]{What are this chapter's key concepts?}{\begin{itemize}
        \item \textbf{Tobacco} was a central crop to the Chesapeake/North Carolina colonies; originally farmed by white male indentured servants and later by Africans
        \item New England colonies (initially Puritan) developed around small towns, family farms; thriving economy blending \textbf{agriculture} and \textbf{commerce}
        \item Middle colonies based around \textbf{export economy} of cereal crops; known for \textbf{diversity} of migrants (which promoted \textbf{tolerance})
        \item Southernmost colonies/West Indies relied on long growing seasons for \textbf{plantation economies} depending on enslaved Africans (majority of population -> cultural autonomy)
        \item Britain's lax attention led to democratic, \textbf{self-governing} institutions centered around town meetings which led to election to legislatures; dominated by plantation owners in south
        \item Atlantic trade developed with goods, Africans, natives sent between Europe, Africa, Americas; significant source of labor in Europe and Americas
        \item British colonies slowly \textbf{Anglicized}, leading to communities based on English models, spread of Protestant evangelicalism
        \item Chattel slavery led to numerous laws prohibiting \textbf{intermarriage}, making slavery \textbf{hereditary}
        \item Africans developed \textbf{overt} and \textbf{covert} methods of resisting slavery
    \end{itemize}}
    \newpage
    \cornell[The Colonial Population]{What factors affected the population of colonial society?}{\textbf{The colonial population was initially most heavily influenced by indentured servitude; however, an increasing birth rate led to a more stable sex ratio, primitive medical knowledge led to significant decreases especially during childbirth, African slaves came in large numbers after the 1690s, and changing sources of European immigration influenced the cultural diversity. Culturally, women were given numerous freedoms in the south due to the high death rate's undermining of male authority, while New England's stability and Puritanism saw a more stable society.}}
    \cornell{What were the social standings of the initial migrants?}{\begin{itemize}
        \item Took significant time for Europeans/Africans to outnumber natives despite significant growth
        \item Earliest settlers members of upper classes (younger members of lesser gentry); generally unaristocratic
        \begin{itemize}
            \item Businessmen like Winthrop migrating for commercial reasons; others for religious
            \item Laborers were dominant element, some of whom came independently (generally as religious dissenters)
            \begin{itemize}
                \item In Chesapeake, at least $\frac{3}{4}$ of immigrants were indentured servants
            \end{itemize}
        \end{itemize}
    \end{itemize}
    \textbf{Although many of the first migrants were upper class, most were unaristocratic, with businessmen, missionaries, and laborers (the majority).}}
    \cornell{What were the influences of indentured servitude?}{\begin{itemize}
        \item System of indentrued servitude developed out of existing passages; based on fixed term of service with free passage to America
        \begin{itemize}
            \item Although masters often promised clothing, tools, land to servants after freedom, rarely successful
            \item Women had more promising prospects ($\frac{1}{4}$) of servants due to easy marriage to plentiful men
        \end{itemize}
        \item Mostly voluntary, but government often sent convicts, POWs (Scots/Irish), orphans, vagrants; sometimes based around kidnapping
        \item Key incentives for landowners were reduced labor force in New World, headright promising land grants for more servants
        \item Servants often hoped to escape troubles, establish themselves
        \begin{itemize}
            \item Some became successful farmers/tradespeople/artisans
            \item Most left w/o prospects, travelling restlessly, causing social unrest
            \item Many free laborers began trend of moving to other places when times were difficult
        \end{itemize}
        \item Indentured servitude reduced in significance by 1670s
        \begin{itemize}
            \item Significant prosperity in England led to reduced pressure; landowners found less attractive due to social unrest
            \item All other servants avoided arduous labor of south, relying on North for greater prospects
        \end{itemize}
    \end{itemize}
    \textbf{Indentured servitude, which brought the majority of migrants to the New World based around a fixed term of work,  left most servants with few prospects, leading to great social unrest. Ultimately, it reduced in significance due to prosperity in England and the changing perspectives of the landowners.}}
    \cornell{What were the key rates of birth and death in the colonies?}{\begin{itemize}
        \item Great hardship led to slow initial growth; conditions eventually approved to point of expansion
        \item Natural reproduction slowly overtook immigration as greatest source of increase
        \begin{itemize}
            \item In New England, population quadrupled through reproduction due to unusual longevity created by cool climate, lack of diseases/population centers
            \item South $\approx$ 20 years lower than New England; England itself $\approx$ 10 years lower
        \end{itemize}
        \item South took much longer to improve
        \begin{itemize}
            \item Mortality rates high for whites (and higher for Africans), leading to many children dying at birth or losing both parents before maturity; white population of Chesapeake known for widows, widowers, orphans
            \item Salt-contaminated water, diseases remained damaging until slow immunity emerged
        \end{itemize}
        \item In all, natural increases led to gradual improvement in sex ratio, slowly approaching England's (matching in 18th century)
        \begin{itemize}
            \item Some change in migration patterns w/ more women arriving
        \end{itemize}
    \end{itemize}
    \textbf{New England was known for a rapid population increase due to exceptional longevity, contrasted with the South, known for its high mortality rate due to arduous working conditions, diseases, and contaminated water. In all, the increase in natural birth rate led the sex ratio to gradually equalize.}}
    \cornell{What was the extent of the medical knowledge in the colonies?}{\begin{itemize}
        \item Initial colonial medical knowledge very primitive, evidenced by large proportion of female deaths during childbirth 
        \begin{itemize}
            \item Easy for anyone to enter medical field: often, women became midwives to assist in childbirth, use herbs/other natural remedies
            \begin{itemize}
                \item Preferred due to social connection
                \item Theratened male physicians
            \end{itemize}
        \end{itemize}
        \item Prevailing theory of medicine: "humoralism," introduced by 2nd century Roman physician Galen
        \begin{itemize}
            \item Focused on four "humors" (fluids) in body: yellow bile, black bile, blood, phlegm
            \item Fluids must be in balance; illnesses caused by fluid inbalance -> treatment involved purging, explusion, bleeding
            \begin{itemize}
                \item Bleeding often practiced by male physicians; lack of evidence for success reflects lack of scientific method in pre-Enlightenment society
                \item Midwives generally prescribed more homeopathic treatments
            \end{itemize}
        \end{itemize}
    \end{itemize}
    \textbf{Colonial medicine was extremely primitive, seen in the numerous deaths of women during childbirth. Medicine was based on Galen's "humoralism," which required the balance of four essential bodily fluids. Despite little proof of its validity, it remained the accepted theory for centuries.}}
    \cornell{What were the key traits of women and families in the Chesapeake?}{\begin{itemize}
        \item High sex ratio -> few women remained unmarried for long (at much younger ages, too)
        \item High mortality rate, splintering of families -> male-centered family structure challenging to maintain
        \item Sexual behavior more flexible: indentured servants forbidden to marry 
        \begin{itemize}
            \item Female servants who became pregnant before term expiration expected harsh treatment (fines, whipping, removal of children)
            \begin{itemize}
                \item Children born out of marriage often became indentured themselves 
            \end{itemize}
            \item Women giving birth after expiry of term often married quickly (pregnant marriages common)
        \end{itemize}
        \item Most women devoted entire lives to childbearing (as many as 8 children, if having survived)
        \item Initial female scarcity led to higher freedom; able to choose partners, responsible for keeping up plantation after death of partners
        \begin{itemize}
            \item Need for male assistance often led to remarriage (often to widowers, leading to complex family structure; peacemaker role led to another path of great power)
        \end{itemize}
        \item Nature of typical family beginning to change as life expectancy increased for whites, leading to growth of patriarchy
    \end{itemize}
    \textbf{Sexual behavior was significantly more flexible in the Chesapeake, especially for indentured servants who were unable to marry. Initially, the highly skewed sex ratio gave women positions of great power in society; however, as life expectancy begin to increase and with it natural birth, society returned to its formerly patriarchal state.}}
    \cornell{What were the key traits of women and families in New England?}{\begin{itemize}
        \item Family structure far more stable, traditional due to balanced sex ratio 
        \item Women still married somewhat young
        \begin{itemize}
            \item Parents continued to control children far longer; almost always influenced decision of spouse
            \begin{itemize}
                \item Women required dowries for desirable husbands
            \end{itemize}
        \end{itemize}
        \item Northern children more likely to survive, fewer widows
        \begin{itemize}
            \item White parents generally lived to see children, often grandchildren grow to maturity
            \item New England 
        \end{itemize}
        \item Puritanism placed emphasis on family, male authority
        \begin{itemize}
            \item Women, with names like Prudence, Patience, Chastity, Comfort, expected to be modest/submissive
        \end{itemize}
    \end{itemize}
    \textbf{In contrast to the Chesapeake, the more stable birth rate, leading to a more even sex ratio, gave women significantly less freedom in the New England colonies, with an expectation of modesty and submission. Furthermore, because parents lived to much older ages, they exercised greater control over their children.}}
    \cornell{What marked the beginnings of slavery in British America?}{\begin{itemize}
        \item Demand for African slaves began with tobacco cultivation, though slave trade initially primarily served Caribbean
        \item Slave trade involved chieftains capturing whole groups to provide to British, who forced them through terrible middle passage to America
        \begin{itemize}
            \item Although some captains ensured health of slaves, many others crammed as many as possible onto boat
            \item Frequent sexual abuse, death; minimal food/water
            \item At arrival, auctioned to landowners
        \end{itemize}
        \item Until 1670s, most slaves to British America passed through Caribbean (due to labor-intensive sugar)
        \begin{itemize}
            \item Royal African Company maintained monopoly over slaves directly to North America during 1670s, deliberately keeping supply low
        \end{itemize}
        \item Monopoly broken by 1690s, leading to rapid arrival of slaves in British colonies
        \begin{itemize}
            \item Primarily to southern regions, beginning to outnumber Europeans in Chesapeake (due to reasonable conditions) but remain unsustainable in SC (rice fields)
            \item Few slaves in New England, middle colonies; vast majority remained in south
        \end{itemize}
        \item Slaves initially received equal treatment to white servants, often receiving freedom after certain period and owning their own land (often w/ slaves)
        \item Masters realized lack of necessity to free black workers, making terms indefinite for slaves and offspring
        \begin{itemize}
            \item Based on natural assumption of inferiority
            \item Furthered by "slave codes" of early eighteenth century, subjecting \textit{all} of African descent to have few rights
        \end{itemize}
    \end{itemize}
    \textbf{Although the slave trade to British America remained limited due to the initial Caribbean-only passthrough of slaves and later the monopoly of the Royal African Company, it began to boom by the 1690s, with most slaves centered in the south and beginning to slowly lose their rights.}}
    \cornell{How did sources of European immigration change over time?}{\begin{itemize}
        \item $\uparrow$ economy, government restrictions due to $\downarrow$ population led to reduced English immigration
        \item Immigration boomed from other European regions
        \begin{itemize}
            \item Earliest were French Calvinists/Huguenots after having lost statehood in south of France
            \item German Protestants suffered from religious discrimination; all Germans from wars with Louis XIV
            \begin{itemize}
                \item Ousted from NY, Mohawk Valley
                \item Most received warm welcome in Pennsylvania due to religious similarity 
            \end{itemize}
            \item Most numerous: Scots-Irish from Ulster, in Northern Ireland
            \begin{itemize}
                \item After Parliament prohibited exports of wools, other key economic products to England, banned Presbyterian religion, rents boomed, many left for America
                \item Often received coldly $\to$ travelled to westernmost borderlands without regard for natives
            \end{itemize}
            \item Scottish Highlanders $\to$ NC, Presbyterian Lowlanders $\to$ NJ/PA before American Revolution
        \end{itemize}
        \item Immigration led to rapid population growth, reflected in doubling of non-Indian population every 25 years
    \end{itemize}
    \textbf{As the English economy stabilized and laws changed, the English source of immigration began to diminish. Taking its place were numerous other European groups, including the French Calvinists, Germans, Scots-Irish, Scottish Highlanders, Presbyterian Lowlanders. This immigration boom led to a rapid population growth.}}
    \end{document} 