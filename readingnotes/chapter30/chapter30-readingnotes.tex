\documentclass[a4paper]{article}
\usepackage{tcolorbox}
\usepackage{amsmath}
\tcbuselibrary{skins}

\title{
\vspace{-3em}
\begin{tcolorbox}
\Huge\sffamily \begin{center} Chapter 30  \mbox{} \\ \huge From the "Age of Limits" to the Age of Reagan \mbox{} \\
\LARGE Finn Frankis \mbox{} \\
\Large AP US History - July 25{$^\text{th}$}, 2018 \end{center}
\end{tcolorbox}
\vspace{-3em}
}
\date{}
\author{}

\usepackage{background}
\SetBgScale{1}
\SetBgAngle{0}
\SetBgColor{red}
\SetBgContents{\rule[0em]{4pt}{\textheight}}
\SetBgHshift{-2.3cm}
\SetBgVshift{0cm}
\usepackage[margin=2cm]{geometry}

\makeatletter
\def\cornell{\@ifnextchar[{\@with}{\@without}}
\def\@with[#1]#2#3{
\begin{tcolorbox}[enhanced,colback=gray,colframe=black,fonttitle=\large\bfseries\sffamily,sidebyside=true, nobeforeafter,before=\vfil,after=\vfil,colupper=blue,sidebyside align=top, lefthand width=.3\textwidth,
opacityframe=0,opacityback=.3,opacitybacktitle=1, opacitytext=1,
segmentation style={black!55,solid,opacity=0,line width=3pt},
title=#1
]
\begin{tcolorbox}[colback=red!05,colframe=red!25,sidebyside align=top,
width=\textwidth,nobeforeafter]#2\end{tcolorbox}%
\tcblower
\sffamily
\begin{tcolorbox}[colback=blue!05,colframe=blue!10,width=\textwidth,nobeforeafter]
#3
\end{tcolorbox}
\end{tcolorbox}
}
\def\@without#1#2{
\begin{tcolorbox}[enhanced,colback=white!15,colframe=white,fonttitle=\bfseries,sidebyside=true, nobeforeafter,before=\vfil,after=\vfil,colupper=blue,sidebyside align=top, lefthand width=.3\textwidth,
opacityframe=0,opacityback=0,opacitybacktitle=0, opacitytext=1,
segmentation style={black!55,solid,opacity=0,line width=3pt}
]

\begin{tcolorbox}[colback=red!05,colframe=red!25,sidebyside align=top,
width=\textwidth,nobeforeafter]#1\end{tcolorbox}%
\tcblower
\sffamily
\begin{tcolorbox}[colback=blue!05,colframe=blue!10,width=\textwidth,nobeforeafter]
#2
\end{tcolorbox}
\end{tcolorbox}
}
\makeatother

\parindent=0pt

\begin{document}
\maketitle
\SetBgContents{\rule[0em]{4pt}{\textheight}}

\cornell[Politics and Diplomacy after Watergate (pgs. 838 - 841)]{How did American presidents following Nixon attempt to recover from the disaster of Watergate?}{Ford and Carter, the two presidents who served after Nixon, took different approaches in recovering from US economic and political turmoil: Ford relied more on the policies of his predecessor; Carter, on the other hand, seems to have been more strong-minded and  fought for his bold ideals.}
\cornell{How did Gerald Ford attempt to recover the nation's prosperity in the aftermath of Watergate?}{
    \begin{itemize}
        \item Major setback emerged immediately due to poor decision to completely pardon Nixon
        \begin{itemize}
            \item Led many to suspect collusion between Nixon and Ford
            \item Immediate decline in popularity
        \end{itemize}
        \item Economic policies relatively unsuccessful
        \begin{itemize}
            \item Attempted to curb inflation by calling for voluntary efforts of people, rejecting idea of price/wage controls
            \item Struggled with recession intensified by energy crisis
            \begin{itemize}
                \item \textbf{Arab oil embargo} of 1973 led to extreme increase in price of oil
            \end{itemize}
        \end{itemize}
        \item Political policies often simply continuations of Nixon administration
        \begin{itemize}
            \item Signed \textbf{SALT II}, arms control accord desired by Nixon
            \item Secretary of state \textbf{Henry Kissinger} required Israel to return parts of Sinai to Egypt
            \item Heavily questioned by both right and left: faced challenge from conservative \textbf{Ronald Reagan} for party nomination
            \begin{itemize}
                \item Democrats united before \textbf{Jimmy Carter}, praised for candor, piety; beat Ford in narrow victory
            \end{itemize}
        \end{itemize}
    \end{itemize}
    \textbf{Gerald Ford attempted to recover from the damage done by the Nixon administration by making new economic and political strides but was ultimately unsuccessful due to his close ties with and similar strides to Nixon.}
}
\cornell{What changes in policy did Carter make following the Ford presidency and were they more successful?}{
    \begin{itemize}
        \item Known for extreme intelligence and bold promises; Congress passed few promised reforms
        \item Devoted to improvement of economy amidst recession through modified energy use
        \begin{itemize}
            \item Oil prices rose during final years of presidency; interest rates rose to highest in American history
            \item Gave \textbf{"malaise"} speech after 10 days at Camp David (presidential retreat) describing American \textbf{energy crisis} and potential solutions
            \begin{itemize}
                \item Criticized for blaming of American people for state of nation
            \end{itemize}
        \end{itemize}
        \item Focused on \textbf{human rights}, criticizing many other nations (including the Soviet Union) for violations
        \item Frequently dealt with more traditional concerns
        \begin{itemize}
            \item Returned \textbf{Panama Canal} to Panamanian government
            \item Greatest achievement was \textbf{peace treaty} between Egypt and Israel
            \begin{itemize}
                \item Encouraged dialogue between Egyptian president and Israeli PM at Camp David
                \begin{itemize}
                    \item Helped to mediate disputes
                \end{itemize}
                \item Leaders later returned to sign \textbf{Camp David accords}
            \end{itemize}
            \item Tried to improve relationship with China, promoting Deng Xiaoping's overtures
            \item Completed SALT II w/ USSR started by Ford, limiting missiles and nuclear warheads for both nations
        \end{itemize}
        \item When Iranian people rebelled against US-promoted government and the shah fled to the US for health care, 53 hostages taken at American embassy
        \item Carter retaliated against USSR invasion of Afghanistan with Olympic withdrawal and cancellation of SALT II
        \item Carter finally fell out of popularity due to domestic economic troubles, international crises
    \end{itemize}
    \textbf{Democrat Jimmy Carter focused on energy use, human rights, and peace between disparate nations; he strongly stood by US traditional ideals and rebuked nations seeking to disrupt those ideals. In all, Carter seemed to have been more popular than Ford among the people:  he voted in for president rather than promoted by virtue of rank.}
}
\cornell{Another research question very long for one line.}{}
\cornell{Do you want fill much more than a page}{No problem.}
\cornell{More?}{No problem.}

\cornell[Last question]{This is the end?}{Yes.}
\end{document}