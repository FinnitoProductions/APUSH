\documentclass[a4paper]{article}
    \input{../notesheader.tex}
    \usepackage[normalem]{ulem}

    \newcommand{\chapternumber}{21}
    \newcommand{\chaptertitle}{America and the Great War}
    \title{\vspace{-3em}
    \begin{tcolorbox}
    \Huge\sffamily \begin{center} AP US History  \\
    \LARGE Chapter \chapternumber \, - \chaptertitle \\
    \Large Finn Frankis \end{center} 
    \end{tcolorbox}
    \vspace{-3em}
    }
    \date{}
    \author{}
    
    \begin{document}
        \maketitle
        \SetBgContents{\rule[0em]{4pt}{\textheight}}
        \cornell[Key Concepts]{What are this chapter's key concepts?}{\begin{itemize}
            \item, \textbf{7.1.I.A} - New technology, manufacturing methods $\to$ U.S. econ. focused on production of consumer goods $\to$ $\uparrow$ std. of living, social mobility, communication systems
            \item \textbf{7.1.I.B} - By 1920, majority of pop. lived in cities $\to$ new econ. oppos. for women, international migrants
            \item \textbf{7.2.I.A} - New mass media $\to$ national culture w/ distinctive regional ones
            \item \textbf{7.2.I.B} - Migration $\to$ new art/literature (\textit{ex}: Harlem Renaissance)
            \item \textbf{7.2.I.D} - 1920s debates over science, gender roles, modernism, religion $\to$ new forms of art/lit.
            \item \textbf{7.2.II.A} - Immigration from Europe peaked before WWI; nativist movements $\to$ reduced migration from southern/eastern Europe, $\uparrow$ barriers to Asian migration
        \end{itemize}}
        
        \end{document}