\documentclass[a4paper]{article}
    \input{../notesheader.tex}
    \usepackage[normalem]{ulem}

    \newcommand{\chapternumber}{21}
    \newcommand{\chaptertitle}{America and the Great War}
    \title{\vspace{-3em}
    \begin{tcolorbox}
    \Huge\sffamily \begin{center} AP US History  \\
    \LARGE Chapter \chapternumber \, - \chaptertitle \\
    \Large Finn Frankis \end{center} 
    \end{tcolorbox}
    \vspace{-3em}
    }
    \date{}
    \author{}
    
    \begin{document}
        \maketitle
        \SetBgContents{\rule[0em]{4pt}{\textheight}}
        \cornell[Key Concepts]{What are this chapter's key concepts?}{\begin{itemize}
            \item, \textbf{7.1.I.A} - New technology, manufacturing methods $\to$ U.S. econ. focused on production of consumer goods $\to$ $\uparrow$ std. of living, social mobility, communication systems
            \item \textbf{7.1.I.B} - By 1920, majority of pop. lived in cities $\to$ new econ. oppos. for women, international migrants
            \item \textbf{7.2.I.A} - New mass media $\to$ national culture w/ distinctive regional ones
            \item \textbf{7.2.I.B} - Migration $\to$ new art/literature (\textit{ex}: Harlem Renaissance)
            \item \textbf{7.2.I.D} - 1920s debates over science, gender roles, modernism, religion $\to$ new forms of art/lit.
            \item \textbf{7.2.II.A} - Immigration from Europe peaked before WWI; nativist movements $\to$ reduced migration from southern/eastern Europe, $\uparrow$ barriers to Asian migration
        \end{itemize}}
        \cornell[The New Economy]{How did the U.S. economy develop after WWI?}{\textbf{The U.S. saw massive economic expansion in the 1920s as a result of expanding technology, notably in radio and automobiles, as well as corporate consolidation, cooperation, and new techniques. Though some workers partook in this boom through "welfare capitalism," most remained impoverished and underpaid, including women and minorities like Filipinos, Mexicans, Chinese, and Japanese throughout the nation. The "American Plan" to restrict unionization did little to help workers. The agricultural industry, despite seeing expanding technology, experienced a crash with massive overproduction.}}
        \cornell{How did technology develop in the 1920s?}{\begin{itemize}
            \item Manufacturing output $\uparrow$ by 60\%; slight recession in 1923 but quickly subsided 
            \item Boom caused by $\downarrow$ Euro. industry; \underline{primary factor was expanding technology}
            \item Automobile industry paired w/ assembly line $\to$ among most important of nation; led to growth of steel/rubber/glass/tool/gasoline/road industries and mobility $\to$ construction boom
            \item Radio grew w/ commercial broadcasting starting in 1920; initially pulses only $\to$ Morse Code but Canadian Fessenden's modulation $\to$ speech/music transmission 
            \begin{itemize}
                \item Families flocked to buy vacuum tube-based radios transmitting strong signals over short distances
            \end{itemize}
            \item Commercial aviation developed slowly, still seen as novel entertainment; pressurized cabin/radial engine 
            \item Rail travel developed rapidly w/ more efficient diesel-electric engine 
            \item Electronics pushed econ. growth 
            \item Telephones spread rapidly 
            \item MIT researchers, led by Vannevar Bush, created first analog computer; Howard Aiken w/ help from Harvard/MIT created complex memory-based computer able to multiply eleven-digit numbers in 3 seconds
            \item Genetic development started w/ Catholic monk Gregor Mendel; Thomas Hunt Morgan discovered linked genes, chromosomal structure
        \end{itemize}
        \textbf{The manufacturing output of the 1920s developed rapidly as a result of proliferation of technologies like the automobile, the radio, commercial aviation (more prevalent from 1930s onward), rail travel, electronics, telephones, very early computers, as well as an expanding genetic understanding.}}
        \cornell{How did American business continue to consolidate and organize?}{\begin{itemize}
            \item Industries like steel w/ large-scale mass production naturally consolidated w/ U.S. Steel \textit{far} ahead of competitors (known collectively as "Little Steel"); many other industries far more resistant
            \item Consolidating industries saw corporate organization
            \begin{itemize}
                \item General Motors, founded by William Durant, largest automobile manufacturer; initially consistently based around personal, informal management; recession $\to$ based around efficient divisions to simplify process of handling subsidiaries
            \end{itemize}
            \item Other industries consolidated w/ cooperation like trade association to coordinate marketing strategies, production strategies; best for mass-production industries 
            \item Greatest fear of industrialists: overproduction due to recessions of 1893, 1907, 1920; \underline{stability} was ultimate goal
        \end{itemize}
        \textbf{Some forms of Aerican business developed through consolidation, primarily for the largest industries like steel and automobiles; General Motors, for example, expanded by eliminating their informal management for formal corporate organization. Other industries focused on cooperation in techniques. All of these efforts were intended to combat overproduction.}}
        \cornell{How did labor develop in the New Era?}{\begin{itemize}
            \item Econ. growth $\to$ continued unequal distribution of wealth w/ majority of Americans in poverty
            \item Most workers saw $\uparrow$ std. of living in 1920s, some saw better working conditions w/ employers adopting "welfare capitalism" to prevent unrest by raising wages, shortening workweek, giving paid vacations
            \begin{itemize}
                \item U.S. Steel made obvious efforts to improve safety in factories
                \item Companies began to offer pensions to retiring workers 
                \item Created company unions for workers to complain 
            \end{itemize}
            \item Welfare capitalism only sustainable during prosperous industrial time; company unions continually weak
            \item Few companies partook in welfare capitalism out of desire to save money; workers' improved productivity far greater than resultant wage increases $\to$ remained poor and w/ limited job security 
            \item Many sought independent unions but most unions unable to adapt to changing climate (AFL still focused only on craft unions, not unskilled workers; frowned on strikes)
        \end{itemize}
        \textbf{Economic growth brought with it a growing unequal distribution of wealth. Although some workers saw an improved standard of living and better conditions with "welfare capitalism," most remained relatively powerless and impoverished. Few truly effective options were available regarding unions.}}
        \cornell{How did women and minorities fit into the workplace?}{\begin{itemize}
            \item Women remained limited to "pink-collar" jobs like secretaries, salesclerks, telephone operators; all underpaid and rarely accepted by labor unions bc. not industerial
            \item Afr. Americans part of Great Migration during WWI had few opportunities w/ active exclusion from skilled crafts by AFL; generally worked as dishwashers, garbage collectors, laundry attendants 
            \begin{itemize}
                \item Welcomed by Brotherhood of Sleeping Car Porters under A. Philip Randolph, Afr. American himself; won significant rights for Afr. Americans as part of civil rights movement
            \end{itemize}
            \item In West/Southwest, Asians/Hispanics generally excluded from white-dom. unions
            \begin{itemize}
                \item Chinese Exclusion Acts $\to$ Japanese began to take place of Chinese in medial railroad/construction jobs 
                \begin{itemize}
                    \item \textit{Issei} (Japanese immigrants) and \textit{Nisei} (children) enjoyed success by creating small businesses, creating farms $\to$ California passed laws to restrict purchasing power
                    \item Filipinos began to make up unskilled workforce $\to$ Anti-Filipino riots $\to$ immigration eliminated
                \end{itemize}
                \item Mexican immigrants part of unskilled workforce in CA, TX, AZ, NM cities; Mexican barrio communities grew in cities w/ very poor conditions; rarely excluded due to need for unskilled, low-paid workforce
            \end{itemize}
        \end{itemize}
        \textbf{Throughout the 1920s, women consistently performed underpaid "pink-collar" jobs like secretaries and salesclerks. African Americans were rarely welcomed by white-dominated unions; A. Philip Randolph pushed for improved rights with his welcoming union. In the West and Southwest, Japanese immigrants experienced economic success, generating resentment; Filipinos were heavily restricted as the unskilled workforce. Mexicans, too, joined the unskilled workforce in the Southwest.}}
        \cornell{What was the "American Plan"?}{\begin{itemize}
            \item Corporate strength limited effective labor organization w/ leaders spreading that unionism subversive, crushing democratic capitalist freedoms of union-free "open shop"; pushed in American Plan $\to$ union busting
            \item Govt. assistance restricted strikes w/ 1921 Supreme Court ruling declaring picketing illegal; 1922, railroad strike quelled by Justice Deparment; 1924, courts would not protect mine workers facing violent oppression from owners
        \end{itemize}
        \textbf{The "American Plan" sought to restrict unionization by protecting the "open shop." It was generally supported and upheld by the government.}}
        \cornell{How did technology develop in the agricultural industry?}{\begin{itemize}
            \item Tractors powered by internal combustion engine expanded rapidly $\to$ new, diverse lands open to cultivation w/ fewer workers 
            \item Agricultural researchers looked into engineering like w/ hybrid corn, chemical fertilizers
            \item Demand for agri. goods slower than production $\to$ surpluses $\to$ decline in farmer income w/ massive decline
            \begin{itemize}
                \item Some farmers sought govt. price support w/ idea of "parity" to set minimum price for goods regardless of fluctuation; govt. would buy extra crops in times of need
                \item McNasry-Haugen Bill allowing parity for grain, cotton, tobacco, rice passed Congress; vetoed by Coolidge
            \end{itemize}
        \end{itemize}
        \textbf{Expanding agricultural technologies like the tracter and combine harvester  as well as new reserach increased agricultural productivity; this surplus was unable to meet demand and thus food prices declined, calling for government support through "parity," or the setting of a minimum price for goods.}}
        \end{document}