\documentclass[a4paper]{article}
    \usepackage[T1]{fontenc}
    \usepackage{tcolorbox}
    \usepackage{amsmath}
    \tcbuselibrary{skins}
    
    \usepackage{background}
    \SetBgScale{1}
    \SetBgAngle{0}
    \SetBgColor{red}
    \SetBgContents{\rule[0em]{4pt}{\textheight}}
    \SetBgHshift{-2.3cm}
    \SetBgVshift{0cm}
    \usepackage[margin=2cm]{geometry} 
    
    \makeatletter
    \def\cornell{\@ifnextchar[{\@with}{\@without}}
    \def\@with[#1]#2#3{
    \begin{tcolorbox}[enhanced,colback=gray,colframe=black,fonttitle=\large\bfseries\sffamily,sidebyside=true, nobeforeafter,before=\vfil,after=\vfil,colupper=blue,sidebyside align=top, lefthand width=.3\textwidth,
    opacityframe=0,opacityback=.3,opacitybacktitle=1, opacitytext=1,
    segmentation style={black!55,solid,opacity=0,line width=3pt},
    title=#1
    ]
    \begin{tcolorbox}[colback=red!05,colframe=red!25,sidebyside align=top,
    width=\textwidth,nobeforeafter]#2\end{tcolorbox}%
    \tcblower
    \sffamily
    \begin{tcolorbox}[colback=blue!05,colframe=blue!10,width=\textwidth,nobeforeafter]
    #3
    \end{tcolorbox}
    \end{tcolorbox}
    }
    \def\@without#1#2{
    \begin{tcolorbox}[enhanced,colback=white!15,colframe=white,fonttitle=\bfseries,sidebyside=true, nobeforeafter,before=\vfil,after=\vfil,colupper=blue,sidebyside align=top, lefthand width=.3\textwidth,
    opacityframe=0,opacityback=0,opacitybacktitle=0, opacitytext=1,
    segmentation style={black!55,solid,opacity=0,line width=3pt}
    ]
    
    \begin{tcolorbox}[colback=red!05,colframe=red!25,sidebyside align=top,
    width=\textwidth,nobeforeafter]#1\end{tcolorbox}%
    \tcblower
    \sffamily
    \begin{tcolorbox}[colback=blue!05,colframe=blue!10,width=\textwidth,nobeforeafter]
    #2
    \end{tcolorbox}
    \end{tcolorbox}
    }
    \makeatother

    \parindent=0pt
    \usepackage[normalem]{ulem}

    \newcommand{\chapternumber}{21}
    \newcommand{\chaptertitle}{America and the Great War}
    \title{\vspace{-3em}
    \begin{tcolorbox}
    \Huge\sffamily \begin{center} AP US History  \\
    \LARGE Chapter \chapternumber \, - \chaptertitle \\
    \Large Finn Frankis \end{center} 
    \end{tcolorbox}
    \vspace{-3em}
    }
    \date{}
    \author{}
    
    \begin{document}
        \maketitle
        \SetBgContents{\rule[0em]{4pt}{\textheight}}
        \cornell[Key Concepts]{What are this chapter's key concepts?}{\begin{itemize}
            \item, \textbf{7.1.I.A} - New technology, manufacturing methods $\to$ U.S. econ. focused on production of consumer goods $\to$ $\uparrow$ std. of living, social mobility, communication systems
            \item \textbf{7.1.I.B} - By 1920, majority of pop. lived in cities $\to$ new econ. oppos. for women, international migrants
            \item \textbf{7.2.I.A} - New mass media $\to$ national culture w/ distinctive regional ones
            \item \textbf{7.2.I.B} - Migration $\to$ new art/literature (\textit{ex}: Harlem Renaissance)
            \item \textbf{7.2.I.D} - 1920s debates over science, gender roles, modernism, religion $\to$ new forms of art/lit.
            \item \textbf{7.2.II.A} - Immigration from Europe peaked before WWI; nativist movements $\to$ reduced migration from southern/eastern Europe, $\uparrow$ barriers to Asian migration
        \end{itemize}}
        \cornell[The New Economy]{How did the U.S. economy develop after WWI?}{\textbf{The U.S. saw massive economic expansion in the 1920s as a result of expanding technology, notably in radio and automobiles, as well as corporate consolidation, cooperation, and new techniques. Though some workers partook in this boom through "welfare capitalism," most remained impoverished and underpaid, including women and minorities like Filipinos, Mexicans, Chinese, and Japanese throughout the nation. The "American Plan" to restrict unionization did little to help workers. The agricultural industry, despite seeing expanding technology, experienced a crash with massive overproduction.}}
        \cornell{How did technology develop in the 1920s?}{\begin{itemize}
            \item Manufacturing output $\uparrow$ by 60\%; slight recession in 1923 but quickly subsided 
            \item Boom caused by $\downarrow$ Euro. industry; \underline{primary factor was expanding technology}
            \item Automobile industry paired w/ assembly line $\to$ among most important of nation; led to growth of steel/rubber/glass/tool/gasoline/road industries and mobility $\to$ construction boom
            \item Radio grew w/ commercial broadcasting starting in 1920; initially pulses only $\to$ Morse Code but Canadian Fessenden's modulation $\to$ speech/music transmission 
            \begin{itemize}
                \item Families flocked to buy vacuum tube-based radios transmitting strong signals over short distances
            \end{itemize}
            \item Commercial aviation developed slowly, still seen as novel entertainment; pressurized cabin/radial engine 
            \item Rail travel developed rapidly w/ more efficient diesel-electric engine 
            \item Electronics pushed econ. growth 
            \item Telephones spread rapidly 
            \item MIT researchers, led by Vannevar Bush, created first analog computer; Howard Aiken w/ help from Harvard/MIT created complex memory-based computer able to multiply eleven-digit numbers in 3 seconds
            \item Genetic development started w/ Catholic monk Gregor Mendel; Thomas Hunt Morgan discovered linked genes, chromosomal structure
        \end{itemize}
        \textbf{The manufacturing output of the 1920s developed rapidly as a result of proliferation of technologies like the automobile, the radio, commercial aviation (more prevalent from 1930s onward), rail travel, electronics, telephones, very early computers, as well as an expanding genetic understanding.}}
        \cornell{How did American business continue to consolidate and organize?}{\begin{itemize}
            \item Industries like steel w/ large-scale mass production naturally consolidated w/ U.S. Steel \textit{far} ahead of competitors (known collectively as "Little Steel"); many other industries far more resistant
            \item Consolidating industries saw corporate organization
            \begin{itemize}
                \item General Motors, founded by William Durant, largest automobile manufacturer; initially consistently based around personal, informal management; recession $\to$ based around efficient divisions to simplify process of handling subsidiaries
            \end{itemize}
            \item Other industries consolidated w/ cooperation like trade association to coordinate marketing strategies, production strategies; best for mass-production industries 
            \item Greatest fear of industrialists: overproduction due to recessions of 1893, 1907, 1920; \underline{stability} was ultimate goal
        \end{itemize}
        \textbf{Some forms of Aerican business developed through consolidation, primarily for the largest industries like steel and automobiles; General Motors, for example, expanded by eliminating their informal management for formal corporate organization. Other industries focused on cooperation in techniques. All of these efforts were intended to combat overproduction.}}
        \cornell{How did labor develop in the New Era?}{\begin{itemize}
            \item Econ. growth $\to$ continued unequal distribution of wealth w/ majority of Americans in poverty
            \item Most workers saw $\uparrow$ std. of living in 1920s, some saw better working conditions w/ employers adopting "welfare capitalism" to prevent unrest by raising wages, shortening workweek, giving paid vacations
            \begin{itemize}
                \item U.S. Steel made obvious efforts to improve safety in factories
                \item Companies began to offer pensions to retiring workers 
                \item Created company unions for workers to complain 
            \end{itemize}
            \item Welfare capitalism only sustainable during prosperous industrial time; company unions continually weak
            \item Few companies partook in welfare capitalism out of desire to save money; workers' improved productivity far greater than resultant wage increases $\to$ remained poor and w/ limited job security 
            \item Many sought independent unions but most unions unable to adapt to changing climate (AFL still focused only on craft unions, not unskilled workers; frowned on strikes)
        \end{itemize}
        \textbf{Economic growth brought with it a growing unequal distribution of wealth. Although some workers saw an improved standard of living and better conditions with "welfare capitalism," most remained relatively powerless and impoverished. Few truly effective options were available regarding unions.}}
        \cornell{How did women and minorities fit into the workplace?}{\begin{itemize}
            \item Women remained limited to "pink-collar" jobs like secretaries, salesclerks, telephone operators; all underpaid and rarely accepted by labor unions bc. not industerial
            \item Afr. Americans part of Great Migration during WWI had few opportunities w/ active exclusion from skilled crafts by AFL; generally worked as dishwashers, garbage collectors, laundry attendants 
            \begin{itemize}
                \item Welcomed by Brotherhood of Sleeping Car Porters under A. Philip Randolph, Afr. American himself; won significant rights for Afr. Americans as part of civil rights movement
            \end{itemize}
            \item In West/Southwest, Asians/Hispanics generally excluded from white-dom. unions
            \begin{itemize}
                \item Chinese Exclusion Acts $\to$ Japanese began to take place of Chinese in medial railroad/construction jobs 
                \begin{itemize}
                    \item \textit{Issei} (Japanese immigrants) and \textit{Nisei} (children) enjoyed success by creating small businesses, creating farms $\to$ California passed laws to restrict purchasing power
                    \item Filipinos began to make up unskilled workforce $\to$ Anti-Filipino riots $\to$ immigration eliminated
                \end{itemize}
                \item Mexican immigrants part of unskilled workforce in CA, TX, AZ, NM cities; Mexican barrio communities grew in cities w/ very poor conditions; rarely excluded due to need for unskilled, low-paid workforce
            \end{itemize}
        \end{itemize}
        \textbf{Throughout the 1920s, women consistently performed underpaid "pink-collar" jobs like secretaries and salesclerks. African Americans were rarely welcomed by white-dominated unions; A. Philip Randolph pushed for improved rights with his welcoming union. In the West and Southwest, Japanese immigrants experienced economic success, generating resentment; Filipinos were heavily restricted as the unskilled workforce. Mexicans, too, joined the unskilled workforce in the Southwest.}}
        \cornell{What was the "American Plan"?}{\begin{itemize}
            \item Corporate strength limited effective labor organization w/ leaders spreading that unionism subversive, crushing democratic capitalist freedoms of union-free "open shop"; pushed in American Plan $\to$ union busting
            \item Govt. assistance restricted strikes w/ 1921 Supreme Court ruling declaring picketing illegal; 1922, railroad strike quelled by Justice Deparment; 1924, courts would not protect mine workers facing violent oppression from owners
        \end{itemize}
        \textbf{The "American Plan" sought to restrict unionization by protecting the "open shop." It was generally supported and upheld by the government.}}
        \cornell{How did technology develop in the agricultural industry?}{\begin{itemize}
            \item Tractors powered by internal combustion engine expanded rapidly $\to$ new, diverse lands open to cultivation w/ fewer workers 
            \item Agricultural researchers looked into engineering like w/ hybrid corn, chemical fertilizers
            \item Demand for agri. goods slower than production $\to$ surpluses $\to$ decline in farmer income w/ massive decline
            \begin{itemize}
                \item Some farmers sought govt. price support w/ idea of "parity" to set minimum price for goods regardless of fluctuation; govt. would buy extra crops in times of need
                \item McNasry-Haugen Bill allowing parity for grain, cotton, tobacco, rice passed Congress; vetoed by Coolidge
            \end{itemize}
        \end{itemize}
        \textbf{Expanding agricultural technologies like the tracter and combine harvester  as well as new reserach increased agricultural productivity; this surplus was unable to meet demand and thus food prices declined, calling for government support through "parity," or the setting of a minimum price for goods.}}
        \cornell[The New Culture]{How did the economic boom promote a new American culture?}{\textbf{The 1920s saw an American culture dominated by consumerism, lifestyle advertising, and restricted films as well as radio programs. Religion became less central to the American middle class and some of the burden of motherhood was distributed to experts, allowing women to live far more liberated, freer lives; several continued to focus on expanding their rights. Education of the youth was far more prioritized, creating a distinctive youth culture. In terms of literature, the Great War created a distinctive genre of disenchanted literature; African American culture, expanded, too, creating poetry and art focusing on rich African heritage.}}
        \cornell{How did consumerism come to influence American culture?}{\begin{itemize}
            \item 1920s America saw middle-class men/women able to afford more than just subsistence, buying additional pleasure goods like refrigerators, washing machines, vacuum cleaners, watches, cosmetics
            \item Widespread purchase of cars transformed economic life, expanding geographic horizons for rural men/women to escape farm life and urban men/women to escape city life w/ weekend drives through countryside
            \begin{itemize}
                \item Ease of access $\to$ urban inhabitants able to move to suburbs
                \item Traveling for pleasure no longer restricted to wealthy w/ businesses offering paid vacations
                \item Young ppl. able to escape families at earlier age, move away from parents/family to develop new social life, youth culture
            \end{itemize}
        \end{itemize}
        \textbf{1920s America saw the rise of consumerism as a force with the purchase of pleasure goods. Most important, however, was the automobile, allowing Americans to escape their normal lives for enjoyable, restful vacations.}}
        \cornell{How did the advertising industry partake in consumerism?}{\begin{itemize}
            \item First advertising firms of \textbf{N.W. Ayer} and \textbf{J. Walter Thompson} emerged before WWI; 1920s saw massive expansion due to propaganda
            \begin{itemize}
                \item Advertising focused less on information and more on lifestyle to deliberately convince
            \end{itemize}
            \item Effective publicists/salesmen valued in society w/ \textit{The Man Nobody Knows} portraying Jesus Christ as skilled salesman w/ religious parables advertising beliefs to world; popularity and amicability critical
            \item Advertising pushed by new forms of communication, notably newspapers to reach national audience
            \begin{itemize}
                \item \textit{The Saturday Evening Post} appealed to rural audiences by focusing on traditionalism of earlier, lost time
                \item \textit{Reader's Digest} focused on urban life to continually update readers on modern world
                \item \textit{Time} magazine kept busy readers constantly updated w/ brief work
            \end{itemize}
        \end{itemize}
        \textbf{The advertising industry expanded with the propaganda of WWI; effective sales techniques based around appealing to a lifestyle became far more valued. Communication forms like newspapers and magazines were critical.}}
        \cornell{How did films became an increasingly popular broadcasting technique?}{\begin{itemize}
            \item Rapid expansion of films over time; sound added in 1927 w/ \textit{The Jazz Singer}
            \begin{itemize}
                \item 1921 scandal w/ comedian Fatty Arbuckle $\to$ calls to "clean up" Hollywood w/ Motion Picture Association led by Will Hays, reviewing and banning any offensive films
            \end{itemize}
            \item Radio, starting w/ Pittsburgh KDKA in 1920, carefully monitored due to fear of govt. intervention, but far less so than filmmaking industry, allowing individual stations signif. autonomy 
        \end{itemize}
        \textbf{The film industry expanded rapidly as sound became incorporated; however, a 1921 scandal saw increased regulation under the Motion Picture Association. Radio also grew significantly; due to its endless hours of programming, it was regulated to a significantly lesser degree.}}
        \cornell{How did religion transform in the 1920s?}{\begin{itemize}
            \item Consumerism $\to$ ideological transformation of Christianity w/ certain elements of evangelical Christianity like faithful biblical interpretation, personification of deity, Trinity replaced w/ modernized beliefs
            \item \textbf{Harry Emerson Fosdick} of NYC believed Christianity was most important to foster strong personality 
            \item Most Americans remained traditional; some middle-class Americans began to place religion secondary to other activities; sociologists shocked by comparison betw. middle-aged adults and their parents in IN
        \end{itemize}
        \textbf{Religion was transformed by consumerism, placing a far greater emphasis on modern fulfillment. Though most remained devoted to a traditional Christian faith, some middle-class Americans focused on this modern Christianity while several others came close to abandoning religion entirely.}}
        \cornell{How did women enter the professional world?}{\begin{itemize}
            \item 1920s saw far more college-educated women entering professional fields
            \item Assumptions abt. reasonable fields for women despite success stories $\to$ most professional women stuck in "feminine" fields like fashion/education/nursing/low-level management 
            \item Middle-class women forced to choose betw. work and marriage; most married middle-class women unable to work outside home
        \end{itemize}
        \textbf{Though countless women began to graduate from college, prevailing assumptions about the effectivenss of women in certain fields meant that most were restricted to lower-level, stereotypical fields. Furthermore, most married middle-class women were unable to enter the professional field.}}
        \cornell{How did ideas about motherhood change?}{\begin{itemize}
            \item 1920s saw "behaviorist" psychologists arguing women were not instinctively perfect mothers; instead should still rely on experts like doctors, trained educators in nursery schools
            \item Changes, despite making motherhood more important, allowed women to become far less consumed/emotionally tied to children $\to$ greater focus on companionship of marriage w/ sex seen less for procreation, more for romantic love
            \item Birth control under Margaret Sanger influenced by Russian Emma Goldmen 
            \begin{itemize}
                \item Initially pushed to working-class families bc. large families seen as main downfall
                \item Eventually extended to middle-class women for sake of enjoying sex w/o reproducing
            \end{itemize}
        \end{itemize}
        \textbf{The 1920s saw the behaviorists taking a significant portion of the motherhood burden away from women, placing it onto experts. Women were able to become far less emotionally connected and consumed by the process of child rearing; they focused marriage companionship and enjoying sexual activity through birth control.}}
        \cornell{What was a "flapper"?}{\begin{itemize}
            \item Women felt 1920s offered new freedom to abandon Victorian "respectability" instead for emotional fulfullment by reaching peak of enjoyment; pushed by Freudian ideals
            \item "Flapper" was woman whose dress, speech, behavior reflected liberation; started among lower-middle/working classes and eventually expanded to upper classes
            \item Most women remained dependent on men as a result of social conditions
        \end{itemize}
        \textbf{A "flapper" was a woman who abandoned the Victorian concept of "respectability," instead focusing on emotional liberation and freedom. Most women, as a result of social conditions, remained dependent on the men around them.}}
        \cornell{How did women continue to push for their rights?}{\begin{itemize}
            \item Alice Paul, leading National Woman's Party, sought Equal Rights Amendment from 1923 onward as part of Constitution; little support in Congress due to radicalism
            \item Women's organizations grew regardless; League of Women Voters, Dem/Repub. auxiliaries for women expanded range of efforts
            \item \textbf{Sheppard-Towner Act} $\to$ "protective" legislation for women in workplace for child health care; great opposition w/ Paul arguing it made all women mothers, Sanger arguing it would discourage birth control, American Medical Association fearing untrained officials $\to$ quicklly terminated
        \end{itemize}
        \textbf{Alice Paul pushed for the Equal Rights Amendment in 1923 but never received widespread support in Congress. Women's organizations expanded as women became a more powerful, diverse political body. The Sheppard-Towner Act to support child health care for women, though supported by some, experienced great opposition for its controversial insinuations.}}
        \cornell{How did American culture place a greater focus on education of the youth?}{\begin{itemize}
            \item Secularism $\to$ education increasingly important w/ growing high school attendance, college/university enrollment, trade/vocational schools for economic need; schools adapted to modern need
            \item Education $\to$ unique youth culture w/ Freudian ideals stressing adolescence as unique period requiring special training/prep.; schools seen as places for social patterns, organized activities w/ peers
        \end{itemize}
        \textbf{A growingly secular society prioritized education, with high school, college, and vocational attendance all expanding. This focus on education created a distinctive youth culture, with adolescents surrounded by peers and partaking in unique extra-curricular activities.}}
        \cornell{How did the Great War stimulate a social disenchantment?}{\begin{itemize}
            \item Great War seen as useless endeavor making prosperous, consumer-driven era meaningless $\to$ writers isolated themselves from mainstream society w/ Gertrude Stein referring to young Americans as "Lost Generation"
            \item Personal alienation w/ rejection of Wilsonian idealism, restoration of consumer-driven society $\to$ everything seen as waste; Hemingway wrote \textit{A Farewell to Arms}, suggesting officer deserting useless war was admirable
            \item "Debunkers" heavily critiqued modern society
            \begin{itemize}
                \item \textbf{H. L. Mencken} felt civilization impossible under democracy due to power of common ppl
                \item Sinclair Lewis, first American to win Nobel Prize in Lit., ridiculed all aspects of American society
                \item F. Scott Fitzgerald ridiculed materialism in \textit{The Great Gatsby}, with materialistic world destroying him
            \end{itemize}
        \end{itemize}
        \textbf{The Great War stimulated great disenchantment, with several writers, including Hemingway, Mencken, Lewis, and Fitzgerald isolating themselves from American society and criticizing every aspect held dear to consumerist Americans.}}
        \cornell{How did the Harlem Renaissance represent the flourishing of African American culture?}{\begin{itemize}
            \item Spread of jazz musicians in nightclubs like \textbf{Duke Ellington}, \textbf{Jelly Roll Morton}, \textbf{Fletcher Henderson}; theaters w/ vaudeville, comedy (some whites enjoyed shows)
            \item Expansion of literature and poetry drawing from rich Afrrican and American roots
            \begin{itemize}
                \item Langston Hughes captured spirit in "I am a Negro - and beautiful": represented appreciation of African race
                \item Alain Locke, Hughes, Zora Neale Hurston, Countee Cullen, Claude McKay, James Weddon Johnson began to reach audience beyond black community
                \item Aaron Douglas created murals in public areas showing Afr. American experience
            \end{itemize}
        \end{itemize}
        \textbf{The Harlem Renaissance, though in part representing the growth in prominence of African American jazz musicians and vaudeville comedy, primarily focused on the creation of rich literature centering around African American heritage through poetry, literature, and visual art.}}
        \end{document}