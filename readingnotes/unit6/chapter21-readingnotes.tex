\documentclass[a4paper]{article}
    \usepackage[T1]{fontenc}
    \usepackage{tcolorbox}
    \usepackage{amsmath}
    \tcbuselibrary{skins}
    
    \usepackage{background}
    \SetBgScale{1}
    \SetBgAngle{0}
    \SetBgColor{red}
    \SetBgContents{\rule[0em]{4pt}{\textheight}}
    \SetBgHshift{-2.3cm}
    \SetBgVshift{0cm}
    \usepackage[margin=2cm]{geometry} 
    
    \makeatletter
    \def\cornell{\@ifnextchar[{\@with}{\@without}}
    \def\@with[#1]#2#3{
    \begin{tcolorbox}[enhanced,colback=gray,colframe=black,fonttitle=\large\bfseries\sffamily,sidebyside=true, nobeforeafter,before=\vfil,after=\vfil,colupper=blue,sidebyside align=top, lefthand width=.3\textwidth,
    opacityframe=0,opacityback=.3,opacitybacktitle=1, opacitytext=1,
    segmentation style={black!55,solid,opacity=0,line width=3pt},
    title=#1
    ]
    \begin{tcolorbox}[colback=red!05,colframe=red!25,sidebyside align=top,
    width=\textwidth,nobeforeafter]#2\end{tcolorbox}%
    \tcblower
    \sffamily
    \begin{tcolorbox}[colback=blue!05,colframe=blue!10,width=\textwidth,nobeforeafter]
    #3
    \end{tcolorbox}
    \end{tcolorbox}
    }
    \def\@without#1#2{
    \begin{tcolorbox}[enhanced,colback=white!15,colframe=white,fonttitle=\bfseries,sidebyside=true, nobeforeafter,before=\vfil,after=\vfil,colupper=blue,sidebyside align=top, lefthand width=.3\textwidth,
    opacityframe=0,opacityback=0,opacitybacktitle=0, opacitytext=1,
    segmentation style={black!55,solid,opacity=0,line width=3pt}
    ]
    
    \begin{tcolorbox}[colback=red!05,colframe=red!25,sidebyside align=top,
    width=\textwidth,nobeforeafter]#1\end{tcolorbox}%
    \tcblower
    \sffamily
    \begin{tcolorbox}[colback=blue!05,colframe=blue!10,width=\textwidth,nobeforeafter]
    #2
    \end{tcolorbox}
    \end{tcolorbox}
    }
    \makeatother

    \parindent=0pt
    \usepackage[normalem]{ulem}

    \newcommand{\chapternumber}{21}
    \newcommand{\chaptertitle}{America and the Great War}

    \title{\vspace{-3em}
\begin{tcolorbox}
\Huge\sffamily \begin{center} AP US History  \\
\LARGE Chapter \chapternumber \, - \chaptertitle \\
\Large Finn Frankis \end{center} 
\end{tcolorbox}
\vspace{-3em}
}
\date{}
\author{}
    \begin{document}
        \maketitle
        \SetBgContents{\rule[0em]{4pt}{\textheight}}
        \cornell[Key Concepts]{What are this chapter's key concepts?}{\begin{itemize}
            \item \textbf{6.1.I.E} - Businesses began to turn outside of the U.S. to control foreign markets/resources 
            \item \textbf{7.2.I.C} - WWI saw restrictions on freedom of speech w/ fears of radicalism $\to$ attacks on immigrants, unions
            \item \textbf{7.2.II.B} - $\uparrow$ demand for production/labor during WWI/WWII, Great Depression $\to$ $\uparrow$ urban migrations 
            \item \textbf{7.2.II.C} - Great Migration during/after WWI saw Afr. Americans departing South for North/West to escape discrimination; still far from perfect
            \item \textbf{7.3.II.A} - Initial U.S. WWI neutrality reversed by Wilson to defend democratic principles of nation 
            \item \textbf{7.3.II.B} - American Expeditionary Forces never played major combat role; arrival of U.S. $\to$ conflict began to favor Allies
            \item \textbf{7.3.II.C} - U.S. Senate would not ratify Treaty of Versailles or join League of Nations despite Wilson's signif. involvement in their creation
        \end{itemize}}
        \cornell[America and the World: 1901-1917]{How did America interact with the foreign world before WWI?}{\textbf{Roosevelt focused on the distinction between civilized and uncivilized nations as well as attempting to limit disputes between other nations; however, he was never afraid to show U.S. power as a threat. In Latin America, Roosevelt established the U.S. right to intervene in unstable Latin American governments; he also ensured the construction of the Panama Canal by encouraging a Panamanian Revolution. Taft focused on strengthening American investments in outside territories. Wilson hoped to take a more moral stance on diplomacy; however, he was never afraid to establish a military government or go to war with another nation over a conflict of ideology.}}
        \cornell{How did Roosevelt argue the importance of civilized nations?}{\begin{itemize}
            \item Roosevelt felt "civilized" nations defined by race (white-dominated) as well as economic proninence (\textit{ex}: Japan would be considered "civilized" for its great prosperity)
            \item Natural duty of civilized societies to assist/intervene in "backward" nations $\to$ inspired rapid growth of American navy
        \end{itemize}
        \textbf{Roosevelt used racial and economic justifications for the natural superiority of certain "civilized" nations. It was the fundamental duty of such nations to assist the more "backward" ones.}}
        \cornell{How did Roosevelt protect Asian free trade?}{\begin{itemize}
            \item Russo-Japanese War stimulated by surprise Japanese attack mediated by Roosevelt at request of Japanese
            \begin{itemize}
                \item Roosevelt formalized Japanese territorial gains as well as received Japanese agreement to cease aggression
                \item Secretly agreed to preserve free trade w/ Japanese
            \end{itemize}
            \item Roosevelt respected for work as mediator (Nobel Peace Prize); relations quickly deteriorated w/ Japan becoming dominant Pacific naval power $\to$ pushed out U.S. from trading w/ territories 
            \begin{itemize}
                \item U.S. asserted naval dominance w/ show of "Great White Fleet," demonstrating immense naval power
            \end{itemize}
        \end{itemize}
        \textbf{Roosevelt helped mediate the Russo-Japanese War, benefiting the U.S. by receiving a secret free trade agreement with the Japanese. As Japan grew more prosperous, they began to exclude the U.S. from several ports, forcing the U.S. to assert their naval dominance to Japan with a bold display of power.}}
        \cornell{How did the U.S. interact with Latin America?}{\begin{itemize}
            \item 1902: Venezuelan govt. unable to pay back debts to Euro. bankers $\to$ GB, Italy, Germany blockaded coast w/ Germany bombarding port $\to$ U.S. naval threat pushed away
            \item 1904: Roosevelt established \textbf{Roosevelt Corollary} to Monroe Doctrine stressing ability of U.S. to intervene in \underline{unstable governments} of Western Hemisphere
            \begin{itemize}
                \item Motivated by Dominican Republic crisis, unable to pay back debts to Europe after internal revolution $\to$ U.S. controlled customs, keeping 45\% of revenue internal, 55\% to Euro. creditors 
                \item Cuba's agreement to Platt Amendment (U.S. could intervene in foreign affairs) saw U.S. grant independence; U.S. troops quickly intervened in 1906 w/ domestic uprisings
            \end{itemize}
        \end{itemize}
        \textbf{Roosevelt passed a corollary to the Monroe Doctrine which allowed the U.S. to intervene in unstable governments of the Western Hemisphere. It justified American intervention in Dominican Republic customs as the government faced bankruptcy as well as a domestic Cuban uprising.}}
        \cornell{How did Roosevelt oversee the construction of the Panama Canal?}{\begin{itemize}
            \item Roosevelt turned to Isthmus of Panama in Colombia for canal linking Atlantic and Pacific due to short distance (despite not being at sea level $\to$ extra cost for locks), existing (but failed) construction by French company 
            \item Sent John Hay (sec. of state) to negotiate w/ Colombian govt., pressuring into agreement to give U.S. six-mile "canal zone"
            \begin{itemize}
                \item Colombian senate refused, demanding higher price
                \item Roosevelt encouraged revolution in Panama w/ new govt. independent of Colombia; resisted Colombian attempts at putting down 
                \item Panamanian govt. agreed to terms w/ work soon beginning
            \end{itemize}
            \item Panama Canal opened in 1914
        \end{itemize}
        \textbf{Roosevelt sought to link the Atlantic and Pacific with a sea route; he turned to the Colombian Isthmus of Panama. After the Colombian senate failed to agree to U.S. terms, Roosevelt encouraged a revolution in Panama, creating a new independent nation which quickly agreed to the desired terms.}}
        \cornell{How did Taft employ his "Dollar Diplomacy" policy?}{\begin{itemize}
            \item Taft had little interest in Roosevelt's broad goal for world stability; instead sought to bring US investments to other regions
            \item Targeted Caribbean: after 1909 rev. in Nicaragua, sided w/ rebels, sending troops to restore peace
            \begin{itemize}
                \item American bankers offered loans to new Nicaraguan govt. $\to$ financial power
                \item U.S. troops sent in and remained to protect govt.
            \end{itemize}
        \end{itemize}
        \textbf{Taft's "Dollar Diplomacy" policy sought to promote U.S. investment interests worldwide, particularly in the Caribbean, where he supported and offered loans to Nicaraguan insurgents.}}
        \cornell{How did Wilson embody the principle of "Moral Diplomacy"?}{\begin{itemize}
            \item Wilson faced large international challenges despite little experience in foreign affairs; mostly strengthened Roosevelt-Taft policies
            \item Wilson established mil. govt. in Dom. Repub.; sent marines to calm rev. in Haiti, remaining until 1934
            \item Bought Danish West Indies after fear that Germany would take over; signed treaty w/ Nicaragua agreeing to bar all other nations from building canals there
            \item In Mexico (w/ large American business presence), leader Díaz had been overthrown by popular Madero; U.S. govt. encouraged Huerta to depose him, w/ Taft admin. preparing to recognize Huerta govt.
            \begin{itemize}
                \item Huerta murdered Madero shortly before Wilson $\to$ refused to recognize as valid govt.
                \item Wilson hoped refusing recognition would end regime; American businessmen agreed w/ mil. regime $\to$ using pretense of Mexican army incorrectly arresting U.S. soldiers, giving insufficient apology to seize port of Veracruz 
                \item Wilson's attack on Veracruz $\to$ Carranza (Constitutionalist) took control, but Wilson remained unsatisfied due to limited support of U.S. guidelines
                \item Forced to recognize Carranza after orig. U.S. ally, Pancho Villa, greatly weakened $\to$ Villa felt betrayed, shooting U.S. miners in Mexico, killing several more past NM border 
                \item U.S. troops unsuccessfully pursued Carranza, engaging only in a few spars w/ Carranza's army $\to$ war betw. U.S. and MX seemed close until Wilson withdrew due to world war
            \end{itemize}
        \end{itemize}
        \textbf{Wilson took an active role in foreign affairs, never afraid to send the military to a potentially dangerous or economically beneficial region, including the Dominican Republic, Haiti, and the Danish West Indies. In Mexico, constant conflict unfolded as governments rapidly transitioned; after the U.S. abandoned an early Mexican ally and they pursued him for his retaliation, the U.S. and Mexico seemed once again close to war.}}
        \cornell[The Road to War]{How did America initially approach the Great War?}{\textbf{War broke out in 1914 in Europe due to the long-term rivarly between Britain and Germany. Wilson initially proclaimed American neutrality, but struggled greatly due to British economic ties, angering the Germans and promoting attacks on U.S. vessels; he even won the reelection by emphasizing his policy of neutrality. However, angered by continual German aggression, he ultimately chose to prepare for the war in 1915 and entered the war on April 2nd, 1917.}}
        \cornell{How did war break out in Europe?}{\begin{itemize}
            \item By 1914, two alliances: "Triple Entente" of GB, FR, RU and "Triple Alliance" of Germany, Austria-Hungary, Italy
            \begin{itemize}
                \item Main rivalry betw. GB/Germany as Germany began to approach Britain in colonial and naval supremacy 
            \end{itemize}
            \item Immediate conflict: Austro-Hungarian Archduke Franz Ferdinand assassinated in Sarajevo in June 1914 by Serbian nationalist
            \begin{itemize}
                \item Germany supported Austria-Hungary in attacking Serbia; Serbia requested Russian assistance
                \item Germany had declared war on Russia/France by August $\to$ invaded Belgium; GB quickly joined war to honor alliance w/ France, attack Germany 
                \item Italy initially neutral, later w/ Triple Entente; Ottoman Empire joined later $\to$ entire continent, part of Asia engaged in war 
            \end{itemize}
        \end{itemize}
        \textbf{The long-term cause of the war was a rivalry between Britain and Germany due to their competing colonial interests. The war was immediately stimulated by the assassination of Franz Fedinand; nations began to take sides and soon became embroiled in war.}}
        \cornell{How did Wilson initially proclaim neutrality in the war?}{\begin{itemize}
            \item Wilson initially declared neutrality, but supported British cause (along w/ many others) due to admiration of GB, skilled propaganda; some German/Irish Americans supported Germany
            \item Strong econ. ties to Britain, unable to trade w/ Germany due to British naval blockade; could afford to cut ties w/ Germany, other Central Powers but not w/ Britain and France (particularly due to $\uparrow$ wartime demands)
            \item Germans began to attack vessels shipping goods to GB using submarines; after sinking British passenger ship carrying 128 Americans, great anger; Wilson demanded German commitment to neutrality w/ agreement but very reluctant
            \item After Allies began to sink German submarines, Germany began attacking unarmed French boats $\to$ Americans injured
        \end{itemize}
        \textbf{Although the U.S. declared neutrality, their trading ties with Britain made it increasingly difficult. Seeking to cut off supplies, Germany often attacked American shipping vessels, once even sinking a British passenger vessel and killing 128 Americans. Tensions began to rise between the two nations.}}
        \cornell{How did Wilson partake in the debate between preparedness and pacifism?}{\begin{itemize}
            \item Facing reelection, Wilson was forced to address question of war; initially sided w/ antipreparedness forces but later changed mind w/ fall 1915 seeing increase in armed forces, working hard for Congressional approval
            \item Congressional peace side saw $\uparrow$ strength $\to$ Wilson began to argue that his work had kept nation out of war; opponent, Charles E. Hughes, far more likely
            \item Wilson won election narrowly
        \end{itemize}
        \textbf{Wilson eventually decided it was best for the nation to prepare for the potential of war, greatly expanding the size of the American navy despite Congressional opposition. In winning reelection, he stressed that his work had helped keep the nation safe and away from war.}}
        \cornell{How did the war ultimately begin as a battle for democracy?}{\begin{itemize}
            \item Wilson took idealistic approach: felt U.S. had no material benefit, rather fighting for new world order for peaceful league of nations after conclusion of war
            \item Several provocations essentially forced intervention
            \begin{itemize}
                \item Germans sought to sink all Allied defense ships before U.S. could intervene
                \item Foreign minister Zimmermann sent telegram to Mexico asking to join war if U.S. joined, in exchange for all "lost provinces" in Southwest
                \item Overthrow of Russian monarchy for republican govt. $\to$ no shame in intervening
            \end{itemize}
            \item Wilson gave powerful speech to Congress asking to join war; some opposition but eventually approved
        \end{itemize}
        \textbf{Wilson joined the war for the sake of promoting democratic ideals: American entry to war was promoted by the continual German attacking of U.S, and Allied ships, the German offer to Mexico in the Zimmermann Telegram for Mexico to join the war against the U.S. in exchange for their lost territories, and the democratization of Russia.}}
        \cornell["War Without Stint"]{What characterized American intervention in the war?}{\textbf{The U.S. initially entered the war hoping to end things with naval assistance; the Russian surrender meant that ground troops were required. Wilson recruited ground troops primarily through forced conscription; women and African Americans were permitted to join the army, too. The U.S. army slightly tipped the tide, allowing the war to end within eight monhts of their arrival. The technology of the war stimulated great devastation.}}
        \cornell{How did the U.S. enter the war?}{\begin{itemize}
            \item German submarines $\to$ GB lost $\approx \dfrac{1}{4}$ of supplies $\to$ U.S. helped attack German submarines (through direct attack, mines), escort merchant vessels w/ warships
            \begin{itemize}
                \item Dramatic reduction in loss of Allied goods
            \end{itemize}
            \item With Nov. 1917 Bolshevik Revolution in Russia $\to$ neutrality w/ Central Powers, Americans forced to send ground troops to assist weakened Allies
        \end{itemize}
        \textbf{Initially, the Americans helped primarily at sea, attacking German submarines and escorting merchant vessels with warships. However, after Russia negotiated neutrality with Germany, America was forced to send ground troops.}}
        \cornell{How did America prepare to mobilize ground troops?}{\begin{itemize}
            \item 1917 saw 120k soldiers in army, 80k in National Guard; neither had combat experience, very few had battle experience
            \item Several sought voluntary recruitment (including ailed Roosevelt); Wilson decided on forced conscription w/ Selective Service Act in May: brought 3m men into army and 2m men joining voluntarily
            \begin{itemize}
                \item Made up American Expeditionary Force (AEF)
            \end{itemize}
            \item Represented first time in U.S. history w/ signif. overseas battle; morale difficult to boost in polluted trenches
            \begin{itemize}
                \item Most soldiers went to bars/brothels in new towns $\to$ venereal disease widespread
            \end{itemize}
            \item Women permitted to enlist in AEF (not combat) in hospitals/offices
            \item Nearly 400k Afr. Americans enlisted/drafted in army/navy (barred from marines); 50k went to Europe in segregated, white-commanded troops
            \begin{itemize}
                \item Most remained at home, performing menial tasks $\to$ some revolts against oppressive whites
            \end{itemize}
            \item Widespread "IQ" testing for soldiers; measured education, not intelligence $\to$ half of whites, most of Afr. Americans scored very low
        \end{itemize}
        \textbf{Wilson dismissed voluntary conscription, instead encouraging a draft to obtain a reasonable number of fighters. The overseas battle saw great lowering in morale, only boosted by local bars and brothels. Both women (no combat) and African Americans enlisted in the army; African Americans were generally forced to perform menial tasks.}}
        \cornell{What characterized American fighting in the Great War?}{\begin{itemize}
            \item Signif. numbers of U.S. troops arrived by spring of 1918; war ended eight months later
            \item Army led by Pershing (commander in unsuccessful pursuit of Pancho Villa); command structure independent of Allies
            \item U.S. troops surrounded by others who had been fighting for nearly four years in violent, uncomfortable, boring, fearful trenches $\to$ limited morale 
            \item U.S. troops had very brief time in trenches: turned the table to allow allies to break out of trenches, advance 
            \begin{itemize}
                \item June 1918: U.S. forces at Château-Thierry assisted French, again at Rheims; offensive began by July 18th
                \item Sept. 1918: joined assault against Germans in Argonne Forest, pushing back nearly to border after seven weeks
            \end{itemize}
            \item Germany risked invasion $\to$ sought armistice (cease-fire) to stimulate negotiations; U.S. wanted to enter Germany but exhausted Allied leaders agreed $\to$ war ended in Nov. 1918
        \end{itemize}
        \textbf{When the U.S. ground troops arrived, morale was already greatly weakened among the Allied troops due to the crushing devastation characterized by trench warfare. However, the U.S. quickly turned the tide, advancing with the Allies on the Germans; as the troops approached the German border, Germany sought an armistice and the war ended.}}
        \cornell{How was the Great War characterized by new, devastating technology?}{\begin{itemize}
            \item Trench war necessary due to devastating modern warfare like machine guns $\to$ open warfare too deadly
            \begin{itemize}
                \item Led to creation of chemical weapons to attack trenches $\to$ troops constantly carried gas masks
                \item Technological warfare required far more maintenance w/ more ammunition for machine guns, fuel for vehicles $\to$ often difficult for forces to arrange rapid delivery of equipment 
            \end{itemize}
            \item WWI first airplane-based conflict: simple planes still effective due to limited anti-aircraft tech. $\to$ mainly for bombing, fighting, reconnaissance
            \item Navy most modernized: GB \textit{Dreadnought} characterized by hydraulic guns, electric power, wireless telegraph, navigation; submarines driven by powerful, long-range diesel engine (notably German U-boats) very significant
            \item Technology caused high number of casualties, with several million lost from each major nation; U.S. only lost 112k but half due to flu; battles w/ critical U.S. troops saw high casualties  
        \end{itemize}
        \textbf{The devastating technology of the Great War required trench warfare rather than the devastating open warfare; though airplanes became significant for the first time in a war, the navy was the most modernized, with powerful submarines and vessels. Technology was the cause of the high casualties.}}
        \cornell[The War and American Society]{What were the social and economic ramifications of the Great War in America?}{\textbf{To prepare for the war, significant economic mobilization had to occur, with bonds sold to the public and raised taxes as well as war boards to coordinate government decisions and purchases on specific wartime needs. The economic boom of the war saw increased labor opportunities, with the National War Labor Board fighting for workers' rights and strikes increasing; African Americans and women saw great opportunities available in the North.}}
        \cornell{How did the U.S. prepare its economy for war?}{\begin{itemize}
            \item By end of war, govt. had spent \$32 billion on directly related expenses; financed through "Liberty Bonds" sold to public (\$23 billion) as well as new taxes (\$10 billion) on corporations + income tax
            \item 1916: Wilson established Council of National Defense designed to disperse power to local communities w/ defense councils in each locality 
            \begin{itemize}
                \item Soon proved impossible: many members of Council of National Defense believed in "scientific management" ideology $\to$ sought to divide economy into planning bodies to focus on specific sector 
                \item "War boards" emerged to oversee railroads, fuel supplies, food (led by Herbert Hoover); able to meet wartime needs
            \end{itemize}
            \item War Industries Board (WIB) in 1917 designed to coordinate govt. purchases; early failure but restructured, given to Wall Street financier Bernard Baruch $\to$ power greater than any previous govt. agency
            \begin{itemize}
                \item Baruch decided which factories would produce which war materials; oversaw distribution of materials during times of need; selected specific corporations when competition emerged
                \item Appeared to provide centralized regulation desired by progressives
                \item Efficiency ultimately untrue; generally suffered from inefficiency but success simply due to American resources, power; WIB generally supported big business w/ salaries for assisting businessmen 
            \end{itemize}
            \item War saw many successes but also great failures w/ slow organization
            \begin{itemize}
                \item Most felt of utmost importance was encouraging relationship betw. public/private sectors 
            \end{itemize}
        \end{itemize}
        \textbf{The U.S. financed the war primarily through bonds sold to the public and secondarily through the raise of taxes. Wilson created "war boards" to divide the overseeing of specific wartime tasks like railroads and food between members. The famed War Industries Board led by a Wall Street financier, despite its apparent efficiency, really only functioned because of natural American prosperity. By the conclusion of the war, most agreed that the public and private sector both belonged in a more closely intertwined relationship.}}
        \cornell{How was wartime labor handled?}{\begin{itemize}
            \item National War Labor Board in Apr. 1918 resolved disputes, pressured industry for eight hour work day, living standards, equal pay for women, unionization; promised no worker strikes
            \item Able to provide with major (but usually temporary) gains; still several strikes in the mines of the West
            \begin{itemize}
                \item Ludlow, CO: workers walked out of mines, evicted from company housing $\to$ lived in tents; state militia protected employers by attacking workers $\to$ continued conflict
            \end{itemize}
        \end{itemize}
        \textbf{The National War Labor Board provided significant gains for workers, including a shortened workday and improved standards of living; however, strikes continued in the West, most notably with the strike in the Ludlow mines.}}
        \cornell{What were the economic and social ramifications of the war?}{\begin{itemize}
            \item $\uparrow$ European demand $\to$ economic boom w/ industrial production in shipbuilding on West Coast, greater employment for women, Afr. Americans, Mexicans, Asians
            \begin{itemize}
                \item Some workers experienced $\uparrow$ income; inflation often $\to$ net reduced purchasing power
                \item Agricultural economy profited from the war 
            \end{itemize}
            \item "Great Migration" w/ Afr. Americans pushed by racism. debt, violence in South, pulled by labor scarcity, factory jobs, freedom of urban North
            \begin{itemize}
                \item Several moved as a result of direct recruitment by Northerners
                \item Many blacks sent letters home abt. opportunities $\to$ migration continued particularly from AL, MS
                \item Black communities in industrial cities grew rapidly; several older Afr. American residents unsettled by rural ways/revivalistic religion 
                \begin{itemize}
                    \item New churches proliferated rapidly 
                    \item Low-paid workers congregated in small houses
                \end{itemize}
                \item Occasional violence betw. whites/blacks w/ white mob in East St. Louis, IL attacking black community 
            \end{itemize}
            \item Women saw $\uparrow$ employment, entering industrial jobs typically restricted to men; most changes temporary, apart from increased govt. focus on wartime jobs
        \end{itemize}
        \textbf{The war led to an economic boom due to wartime demand from Europe; new jobs emerged, particularly in northern factories, leading several African Americans to migrate from the South to the North to find new opportunities, adding to existing communities and proliferating their culture. Women, too, saw temporary employment opportunities in traditionally male jobs.}}
        \cornell[The Futile Search for Social Unity]{How did Americans seek social cohesion after the war?}{\textbf{A peace movement emerged before the war, led by immigrants, religious pacifists, and women's suffragists; most supporters of women's suffrage strategically changed sides after war broke out. Though the war sparked great patriotism in the majority, the government feared the influence of the minority, deciding to legally punish those who spoke against the war through the Espionage and Sedition Acts; immigrants were most often targeted by local groups.}}
        \cornell{How did a peace movement emerge resisting war?}{\begin{itemize}
            \item Several progressives justified war as path to inevitable social unity, peace
            \item Peace movement before 1917 made up of several German Americans, Irish Americans, religious pacifists, Socialists, IWW: saw war as meaningless battle betw. capitalist nations for industry
            \item Most active resistance \underline{from women's movement}
            \begin{itemize}
                \item 1915: Carrie Chapman Catt created Woman's Peace Party w/ active members hoping to prevent U.S. intervention
            \end{itemize}
            \item After entering war, women peace activists divided
            \begin{itemize}
                \item National American Woman Suffrage Association (largest women's organization) supporting war; even Catt felt war represented way for Americans to get involved in politics of nation
                \item Jane Addams refused to support $\to$ reviled
            \end{itemize}
            \item Women peace activists shared capitalist objections of Socialist Party; others felt that as women, should be devoted to pacifism 
        \end{itemize}
        \textbf{The peace movement before 1917, though made up somewhat of immigrants, religious pacifists, and Socialists, was dominated by the women's movement under Carrie Chapman Catt. After the nation entered the war, however, most women pivoted, seeing it as an opportunity for women to get involved in the functioning of the nation. The remaining women, notably Jane Addams, felt that their maternal nature went hand in hand with pacifism.}}
        \cornell{How did the world war spark bouts of patriotism in several communities?}{
            \begin{itemize}
                \item Enlistment, women $\to$ Red Cross, children raised money for war bonds, churches prayed for president/troops
                \item Religious support w/ growth of revivalist movement $\to$ promotion of mil. effort, notably Billy Sunday
            \end{itemize}
            \textbf{The war sparked great patriotism through enlistment, participation in the Red Cross, and church support through the revivalist movement.}}
        \cornell{How did the government push the minority group opposing the war to alter their perception?}{\begin{itemize}
                \item \textbf{Committee on Public Information} (CPI) led by George Creel distributed pro-war lit., encouraged newspapers to publicize govt. reason for war
                \begin{itemize}
                    \item Eventually promoted posters arguing savagery of Germans
                    \item CPI encouraged citizens to report dissenting citizens to Justice Department
                \end{itemize}
                \item 1917 Espionage Act allowed govt. to punish those appearing to resist war effort and eliminate any "sedition" against the war; 1918 Sabotage and Sedition Acts made any criticism \underline{illegal}
                \begin{itemize}
                    \item Generally targeted Socialists, IWW who opposed war; allowed legal oppression of already-disliked groups w/ Socialist Eugene V. Debs, imprisoned for 10 years in 1918; others from IWW oppressed $\to$ fled to Soviet Union 
                \end{itemize}
                \item Oppression on local level w/ vigilante mobs unexpectedly punishing anyone disagreeing w/ war; citizens' groups (notably American Protective League, National Security League, Boy Spies of America) emerged to prevent disloyalty through "agents"' spying on neighbors; supported by govt. 
                \item Immigrants most often targeted
                \begin{itemize}
                    \item Irish-Americans opposed Brits, Jews opposed anti-Semitic Russian govt.
                    \item Most German-Americans supported U.S. war effort; most public opinion turned against, dismissing culture, language; several fired from jobs even if unrelated to war 
                    \begin{itemize}
                        \item Lynching in IL; harassment/beating elsewhere
                        \item Most agreed upon some fundamental flaw
                    \end{itemize}
                    \item "Loyalist" citizens policed neighborhoods, monitored conversations w/ beliefs that immigrants represented Old World
                \end{itemize}
        \end{itemize}
        \textbf{The U.S. formed the Committee on Public Information to promote support for the war through propaganda; the Espionage and Sedition Acts allowed those who visibly opposed the war to be punished, notably Socialists and members of the IWW. Some citizen groups began to police the local level, spying on their neighbors and often attacking those who they feared resisted the effort. Immigrants were most often targeted: Irish-Americans and Jews for their understandable opposition and German-Americans typically simply for their ancestry.}}
        \cornell[The Search for a New World Order]{How did several seek a new path for the world following the war?}{\textbf{Wilson's ideological Fourteen Points received worldwide support for its progressive nature; however, Wilson faced opposition from British and French leaders, seeking far more punishment for Germany, as well as internally from those who sought to limit his authority. At the Paris Peace Conference, few of his sweeping ideological claims were successful with his three fellow leaders; the League of Nations was his greatest accomplishment. However, as his health began to deteriorate, he became more intransigent and refused any modifications to the League of Nations in the Senate; after suffering from a stroke, matters were even worse as he forced the rejection of even a slightly modified treaty. The Senate thus never approved the U.S. entry into the League of Nations.}}
        \cornell{What was Wilson's initial solution to peace following the war?}{\begin{itemize}
            \item Wilson presented 14 points encompassed by 3 broad categories
            \begin{itemize}
                \item 8 recommendations for postwar border shifts (belief in right of all to self-determination)
                \item 5 for controlling international affairs in future w/ free seas, open covenants, impartiality in mediation
                \item 1 for league of nations to implement, enforce principles
            \end{itemize}
            \item Proposal flawed: no plan to control subjugated ppl.; ignored economic rivalries which had caused war
            \item Supported by many for idealistic, progressive nature with end goal of peace for all
            \item Russian Bolshevik govt. used similar words to Wilson's to define war aims $\to$ Wilson publicly announced points to encourage Russians to remain in war, ensure words attributed to him
        \end{itemize}
        \textbf{Wilson's fourteen points focused on border adjustments, international affairs, and the creation of a league of nations. Despite lacking several key elements like how to handle economic rivalries, its idealistic, progressive nature allowed it to gain widespread support in the eyes of many.}}
        \cornell{What were some of Wilson's early obstacles?}{\begin{itemize}
            \item Most Allied forces disliked Wilson's moral superiority, decision in war to keep troop structure separate from Allied armies
            \begin{itemize}
                \item Massive losses for GB and FR $\to$ did not desire peaceful solution, instead seeking punishment for Germany
                \item David Lloyd George, GB prime minister, and Georges Clemenceau, president of France, sought gains against Germany
            \end{itemize}
            \item Wilson asked American voters to vote for Democrats so that Allied forces would respect leadership; midterms saw both houses earned by Republicans (mostly due to domestic issues; appeal didn't help)
            \item Republican leaders angered by Wilson's partisan appeal for midterms: many Republicans had supported Fourteen Points; Wilson refused to appoint any Republicans to negotiating team to represent U.S. at Paris Peace Conference
            \begin{itemize}
                \item Wilson felt his words were most important for negotiation
            \end{itemize}
        \end{itemize}
        \textbf{Wilson faced Allied opposition, disliking his tone of moral superiority and seeking far greater punishment for Germany. Furthermore, internal opposition arose after Wilson unwisely tied the issue of the Fouerteen Points into a partisan question; Republicans won both houses and were angered by Wilson's limited recognition of their support for the Fourteen Points.}}
        \cornell{How did Wilson participate in the Paris Peace Conference?}{\begin{itemize}
            \item Wilson greeted in Paris by large, hopeful crowd
            \item Great uneasiness over communist Russia w/ ongoing war, no representation
            \begin{itemize}
                \item Wilson heavily involved in issue of Soviet Union: had sent U.S. troops to Russia shortly before supposedly to rescue Czech soldiers; troops remained until 1920, Wilson refused to recognize Lenin's govt. 
            \end{itemize}
            \item Negotiations not promising for Wilson: he hoped to dominate other three leaders (GB, FR, Italy) and idealism conflicted with anger of other nations
            \begin{itemize}
                \item Wilson unable to convince of free trade, freedom of sea, open covenants, impartial mediation (all German colonies in Pacific $\to$ Japan); self-determination for all attacked
                \item Wilson did not seek compensation from Central Powers; all other leaders demanded $\to$ requested \$56 billion; Germany only able to pay \$9 billion, even a stretch for its economy (Germany weakened for some time)
            \end{itemize}
            \item Wilson experienced some victories
            \begin{itemize}
                \item Assigned boundaries in Pacific, placing Palestine and others under League of Nations, creating Yugoslavia and Czechoslovakia w/ some ethnic 
                \item League of Nations accepted in 1919 $\to$ Wilson felt work was complete: anything rejected could be later undone
                \begin{itemize}
                    \item Allowed regular meeting of nations to resolve issues; nine-member council w/ permanent members U.S., GB, FR, Italy, Japan 
                \end{itemize}
            \end{itemize}
        \end{itemize}
        \textbf{Wilson struggled to pass several of his Fourteen Points, notably free trade, freedom of sea, and impartial mediation; furthermore, the other three nations pushed for required Central Power compensation. However, Wilson experienced some colonial victories; most notable was his creation of the League of Nations, which he believed could undo any decision with which he disagreed.}}
        \cornell{How did Wilson face great struggle in ratifying the League of Nations?}{\begin{itemize}
            \item Many Americans opposed internationalism; Senate opposed covenant of League of Nations $\to$ Wilson returned, negotiating modifications to covenants
            \begin{itemize}
                \item Opponents would not budge; nor would Wilson
                \item Wilson presented Treaty of Versailles to Senate in July 1919; refused to make \underline{any} compromise; health also began to deteriorate
            \end{itemize}
            \item Senate had several western isolationists opposing ratification on principle; others simply sought to weaken Democrat president and create issue for debate in 1920 election
            \begin{itemize}
                \item \textbf{Henry Cabot Lodge}, connected w/ Roosevelt, hated Wilson, attempting to delay ratification by holding public hearings, carefully reading each word to stall time 
                \item Wilson would not budge on anything $\to$ Senate would not make final steps for ratification
            \end{itemize}
        \end{itemize}
        \textbf{Several Americans in the Senate opposed the Treaty of Versailles even after Wilson made a few concessions. Some Senators opposed on principle; most opponents were Republicans seeking to weaken a Democratic president, notably Henry Cabot Lodge. Wilson's intransigence forced him to turn to the public.}}
        \cornell{How did Wilson appeal to the public for support?}{\begin{itemize}
            \item Embarked on grueling speaking tour travelling 8000 miles by train, speaking as many as four times a day without rest; eventually collapsed in Colorado with headaches
            \item Returned to Washington, suffering a major stroke; close to death for two weeks, so ill for six weeks that unable to conduct business; wife and doctor protected from all public pressures, hiding condition from public 
            \item Wilson had left side entirely paralyzed; only partial control of mental/emotional state $\to$ unwilling to make any compromises, forcing Democrat supporters to reject treaty with even a few modifications; original treaty unsuccessful
            \item Peace effort ultimately collapsed due to internal issues
        \end{itemize}
        \textbf{Wilson attempted to appeal to the public for support for his peace treaty; his grueling journey by train caused him to eventually collapse and suffer from a major stroke. Paralyzed and left with limited emotional control, his intransigence grew: he refused to accept anything but his original treaty. As a result, nothing was passed and the peace effort faded.}}
        \cornell[A Society in Turmoil]{How did domestic issues in the U.S. begin to distract from post-war reforms?}{\textbf{Domestically, several issues unfolded in the U.S. after the war. Tensions ran high over labor as the economy collapsed, with several strikes; African Americans began to fight back after little change to their rights after the war brought on a time of great disillusionment; Americans began to fear radicalism more than ever before with the growth of communism in Russia, punishing anyone deemed revolutionary; ultimately, however, society turned toward a focus on concrete civil liberties. After the Nineteenth Amendment granted women the vote, Wilson's era of idealist "New Freedom" reforms had come to an end, replaced by Harding's more "normal" reforms.}}
        \cornell{How did continued tensions unfold over labor?}{\begin{itemize}
            \item After armistice announced, rapid reconversion of economy w/ war agencies cancelling govt. contracts 
            \item Postwar prosperity only an aftereffect of wartime boom w/ great deficit spending continuing, still demand from European nations
            \item $\uparrow$ inflation $\to$ collapse in 1920 w/ rampant bankruptcy, 5 million losing jobs, wartime wage gains immediately lost, job security threatened as veterans returned to workforce, working conditions resumed
            \item Large wave of strikes unfolded in 1919 
            \begin{itemize}
                \item Evolving shipyard strike in Seattle eventually $\to$ complete city standstill, w/ Marines intervening to put down 
                \item Sept. 1919: Boston police force went on strike $\to$ widespread violence unable to be put down by locals $\to$ governor called in National Guard, eventually hiring new police force
                \item 350k steelworkers in eastern/western steelworkers walked off, demanding eight-hour workday, union recognition; violence came from employers, hiring armed guards to attack, escort strikebreakers
                \begin{itemize}
                    \item Indiana riot saw 18 strikers killed
                    \item AFL initially endorsed but public opinion $\to$ repudiated; strike collapsed by January 
                \end{itemize}
            \end{itemize}
        \end{itemize}
        \textbf{The burst of the economic bubble temporarily created by the postwar boom led to great economic recession; millions of workers lost jobs or saw threatened job security. As a result, union membership increaed and several strikes unfolded. Most significant was a strike of 350,000 steel workers throughout the nation demanding a shorter workday; employers responded with violence.}}
        \cornell{What reforms did African Americans seek?}{\begin{itemize}
            \item Afr. Americans rejoiced in war victory surrounded by fellow troops; went again w/ jazz bands through black neighborhoods like Harlem 
            \begin{itemize}
                \item Inspired new hope in countless African Americans: hoped heroism would be welcomed by whites and new appreciation would emerge
                \item W.E.B. Du Bois anticipated new life for blacks
            \end{itemize}
            \item Black soldiers had little impact on whites; instead encouraged blacks, with soldiers having received little social reward and factory workers having filled wartime jobs receiving little economic gain $\to$ sought to take action 
            \begin{itemize}
                \item $\uparrow$ southern lynchings, firing of black workers in North, hostility of white communities $\to$ race riots characterized by \underline{blacks fighting back}; NAACP urged retaliation for rights
            \end{itemize}
            \item Jamaican Marcus Garvey encouraged pride in Afr. American heritage, rejection of white assimilation in favor of returning to Africa to breed superior black race
            \begin{itemize}
                \item Encouraged creation of black-only stores, businesses; enterprise expanded rapidly driven by his United Negro Improvement Association (UNIA) 
                \item Indicted in 1923, later deported for business fraud; black nationalism remained significant
            \end{itemize}
        \end{itemize}
        \textbf{African Americans hoped that their significant contributions to the war would bring them greater equality; it ultimately had little effect, with little social change and several blacks having ascended to take up white jobs in the workforce being fired with the return of veterans. This, paired with continued white oppression, encouraged African Americans to fight back against whites for their rights. Black nationalism became a significant movement, too, arguing that African Americans were part of a superior race not to be assimilated into white society.}}
        \cornell{What was the Red Scare?}{\begin{itemize}
            \item Russian Revolution indicated that communism was now a serious regime; fears expanded after Soviet govt. in 1919 announced Communist International (Comintern) to export communism throughout nation 
            \begin{itemize}
                \item Several Americans began to follow regime (American Communist Party and other radical groups)
                \begin{itemize}
                    \item Likely responsible for several 1919 bombings as well as mailing bombs to politicians, injuring public officials
                    \item Attorney General Palmer suffered bombing of home
                \end{itemize}
            \end{itemize}
            \item Communist scare pushed for growing antiradicalism within middle classes; created growing desire for being "100\% American" $\to$ Red Scare
            \begin{itemize}
                \item Newspapers began to depict race riots as work of revolutionaries; steel strike as tied to Bolsheviks $\to$ 30 states created sedition laws to punish those stimulating revolution w/ several jailed
                \item Off-duty soldiers in NYC attacked offices of socialist newspaper; mob in Washington dragged IWW member, castrated publicly
                \item Threatening books/radical employees terminated; even women's groups attacked as overly radical
                \item Govt. contributed to Red Scare after Attorney General Palmer raided radical centers throughout nations for weapons, arresting 6k but only finding 3 pistols; deported 500 non-American citizens
            \end{itemize}
            \item Red Scare soon abated; reinvigorated after two Italian immigrants Sacco and Vanzetti, charged w/ murder of paymaster in MA
            \begin{itemize}
                \item Both confessed to be anarchists $\to$ convicted by prejudiced judge in unfair trial; sentenced to death despite growing public support
                \item Died in electric chair despite protests $\to$ great public anger
            \end{itemize}
        \end{itemize}
        \textbf{The Red Scare emerged after countless antiradicalist Americans began to greatly fear the expansion of communism and general radicalism into the American nation. Society punished anyone deemed as radical; the government soon intervened, jailing anyone deemed revolutionary. Any books containing potentially radical thoughts were eliminated; any center deemed radical was raided.}}
        \cornell{How were the actions of the Red Scare ultimately reversed?}{\begin{itemize}
            \item Society shifted to place greater focus on civil liberties, crushing Attorney General Palmer and nearly destroying assistant J. Edgar Hoover, hurt Democratic Party
            \item Led to National Civil Liberties Bureau in 1917
            \item Supreme Court, notably Oliver Wendell Holmes and Louis Brandeis, shifted to defend seemingly unpopular, dissenting speech 
            \begin{itemize}
                \item Focus on free speech
            \end{itemize}
        \end{itemize}
        \textbf{A great reversal soon completely crushed the Red Scare, ruining the career of Attorney General Palmer and leading to a greater focus on civil liberties. The Supreme Court increasingly defended revolutionaries, focusing on the individual freedom to dissenting speech.}}
        \cornell{How did society begin to turn away from idealism?}{\begin{itemize}
            \item Women's suffrage ratified in 1920 seen as great progressive step initially
            \begin{itemize}
                \item Congress passed Sheppard-Towner Maternity and Infancy Act to fund women/infant health
                \item Women supported 1922 Cable Act to give women citizenship regardless of husband's rights
                \item Attempted to pass child labor constitutional amendment
            \end{itemize}
            \item Women's suffrage marked an ending of a time of great reform; intensity of antiradicalism, economic issues, feminist demands, racial issues $\to$ general disillusionment
            \begin{itemize}
                \item Democratic candidates in 1920 attempted to keep up idealistic League of Nations wishes of Wilson (James M. Cox, Franklin D. Roosevelt)
                \item Republican nominee, Warren G. Harding, focused little on ideals and more on return to "normalcy" $\to$ easily won, era of idealism under Wilson had come to an end
            \end{itemize}
        \end{itemize}
        \textbf{Although women's suffrage led to great reform particularly in women's rights, it marked the end of an era of long-term idealistic reform. Warren G. Harding, won the presidential election of 1920 against the idealistic Democrats; Wilson's era of idealism had come to an end.}}
        \end{document}