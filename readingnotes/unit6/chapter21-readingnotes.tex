\documentclass[a4paper]{article}
    \input{../notesheader.tex}
    \usepackage[normalem]{ulem}

    \newcommand{\chapternumber}{20}
    \newcommand{\chaptertitle}{The Progressives}
    \title{\vspace{-3em}
    \begin{tcolorbox}
    \Huge\sffamily \begin{center} AP US History  \\
    \LARGE Chapter \chapternumber - \chaptertitle \\
    \Large Finn Frankis \end{center} 
    \end{tcolorbox}
    \vspace{-3em}
    }
    \date{}
    \author{}
    
    \begin{document}
        \maketitle
        \SetBgContents{\rule[0em]{4pt}{\textheight}}
        \cornell[Key Concepts]{What are this chapter's key concepts?}{\begin{itemize}
            \item \textbf{6.1.I.E} - Businesses began to turn outside of the U.S. to control foreign markets/resources 
            \item \textbf{7.2.I.C} - WWI saw restrictions on freedom of speech w/ fears of radicalism $\to$ attacks on immigrants, unions
            \item \textbf{7.2.II.B} - $\uparrow$ demand for production/labor during WWI/WWII, Great Depression $\to$ $\uparrow$ urban migrations 
            \item \textbf{7.2.II.C} - Great Migration during/after WWI saw Afr. Americans departing South for North/West to escape discrimination; still far from perfect
            \item \textbf{7.3.II.A} - Initial U.S. WWI neutrality reversed by Wilson to defend democratic principles of nation 
            \item \textbf{7.3.II.B} - American Expeditionary Forces never played major combat role; arrival of U.S. $\to$ conflict began to favor Allies
            \item \textbf{7.3.II.C} - U.S. Senate would not ratify Treaty of Versailles or join League of Nations despite Wilson's signif. involvement in their creation
        \end{itemize}}
        \cornell[America and the World: 1901-1917]{How did America interact with the foreign world before WWI?}{\textbf{Roosevelt focused on the distinction between civilized and uncivilized nations as well as attempting to limit disputes between other nations; however, he was never afraid to show U.S. power as a threat. In Latin America, Roosevelt established the U.S. right to intervene in unstable Latin American governments; he also ensured the construction of the Panama Canal by encouraging a Panamanian Revolution. Taft focused on strengthening American investments in outside territories. Wilson hoped to take a more moral stance on diplomacy; however, he was never afraid to establish a military government or go to war with another nation over a conflict of ideology.}}
        \cornell{How did Roosevelt argue the importance of civilized nations?}{\begin{itemize}
            \item Roosevelt felt "civilized" nations defined by race (white-dominated) as well as economic proninence (\textit{ex}: Japan would be considered "civilized" for its great prosperity)
            \item Natural duty of civilized societies to assist/intervene in "backward" nations $\to$ inspired rapid growth of American navy
        \end{itemize}
        \textbf{Roosevelt used racial and economic justifications for the natural superiority of certain "civilized" nations. It was the fundamental duty of such nations to assist the more "backward" ones.}}
        \cornell{How did Roosevelt protect Asian free trade?}{\begin{itemize}
            \item Russo-Japanese War stimulated by surprise Japanese attack mediated by Roosevelt at request of Japanese
            \begin{itemize}
                \item Roosevelt formalized Japanese territorial gains as well as received Japanese agreement to cease aggression
                \item Secretly agreed to preserve free trade w/ Japanese
            \end{itemize}
            \item Roosevelt respected for work as mediator (Nobel Peace Prize); relations quickly deteriorated w/ Japan becoming dominant Pacific naval power $\to$ pushed out U.S. from trading w/ territories 
            \begin{itemize}
                \item U.S. asserted naval dominance w/ show of "Great White Fleet," demonstrating immense naval power
            \end{itemize}
        \end{itemize}
        \textbf{Roosevelt helped mediate the Russo-Japanese War, benefiting the U.S. by receiving a secret free trade agreement with the Japanese. As Japan grew more prosperous, they began to exclude the U.S. from several ports, forcing the U.S. to assert their naval dominance to Japan with a bold display of power.}}
        \cornell{How did the U.S. interact with Latin America?}{\begin{itemize}
            \item 1902: Venezuelan govt. unable to pay back debts to Euro. bankers $\to$ GB, Italy, Germany blockaded coast w/ Germany bombarding port $\to$ U.S. naval threat pushed away
            \item 1904: Roosevelt established \textbf{Roosevelt Corollary} to Monroe Doctrine stressing ability of U.S. to intervene in \underline{unstable governments} of Western Hemisphere
            \begin{itemize}
                \item Motivated by Dominican Republic crisis, unable to pay back debts to Europe after internal revolution $\to$ U.S. controlled customs, keeping 45\% of revenue internal, 55\% to Euro. creditors 
                \item Cuba's agreement to Platt Amendment (U.S. could intervene in foreign affairs) saw U.S. grant independence; U.S. troops quickly intervened in 1906 w/ domestic uprisings
            \end{itemize}
        \end{itemize}
        \textbf{Roosevelt passed a corollary to the Monroe Doctrine which allowed the U.S. to intervene in unstable governments of the Western Hemisphere. It justified American intervention in Dominican Republic customs as the government faced bankruptcy as well as a domestic Cuban uprising.}}
        \cornell{How did Roosevelt oversee the construction of the Panama Canal?}{\begin{itemize}
            \item Roosevelt turned to Isthmus of Panama in Colombia for canal linking Atlantic and Pacific due to short distance (despite not being at sea level $\to$ extra cost for locks), existing (but failed) construction by French company 
            \item Sent John Hay (sec. of state) to negotiate w/ Colombian govt., pressuring into agreement to give U.S. six-mile "canal zone"
            \begin{itemize}
                \item Colombian senate refused, demanding higher price
                \item Roosevelt encouraged revolution in Panama w/ new govt. independent of Colombia; resisted Colombian attempts at putting down 
                \item Panamanian govt. agreed to terms w/ work soon beginning
            \end{itemize}
            \item Panama Canal opened in 1914
        \end{itemize}
        \textbf{Roosevelt sought to link the Atlantic and Pacific with a sea route; he turned to the Colombian Isthmus of Panama. After the Colombian senate failed to agree to U.S. terms, Roosevelt encouraged a revolution in Panama, creating a new independent nation which quickly agreed to the desired terms.}}
        \cornell{How did Taft employ his "Dollar Diplomacy" policy?}{\begin{itemize}
            \item Taft had little interest in Roosevelt's broad goal for world stability; instead sought to bring US investments to other regions
            \item Targeted Caribbean: after 1909 rev. in Nicaragua, sided w/ rebels, sending troops to restore peace
            \begin{itemize}
                \item American bankers offered loans to new Nicaraguan govt. $\to$ financial power
                \item U.S. troops sent in and remained to protect govt.
            \end{itemize}
        \end{itemize}
        \textbf{Taft's "Dollar Diplomacy" policy sought to promote U.S. investment interests worldwide, particularly in the Caribbean, where he supported and offered loans to Nicaraguan insurgents.}}
        \cornell{How did Wilson embody the principle of "Moral Diplomacy"?}{\begin{itemize}
            \item Wilson faced large international challenges despite little experience in foreign affairs; mostly strengthened Roosevelt-Taft policies
            \item Wilson established mil. govt. in Dom. Repub.; sent marines to calm rev. in Haiti, remaining until 1934
            \item Bought Danish West Indies after fear that Germany would take over; signed treaty w/ Nicaragua agreeing to bar all other nations from building canals there
            \item In Mexico (w/ large American business presence), leader Díaz had been overthrown by popular Madero; U.S. govt. encouraged Huerta to depose him, w/ Taft admin. preparing to recognize Huerta govt.
            \begin{itemize}
                \item Huerta murdered Madero shortly before Wilson $\to$ refused to recognize as valid govt.
                \item Wilson hoped refusing recognition would end regime; American businessmen agreed w/ mil. regime $\to$ using pretense of Mexican army incorrectly arresting U.S. soldiers, giving insufficient apology to seize port of Veracruz 
                \item Wilson's attack on Veracruz $\to$ Carranza (Constitutionalist) took control, but Wilson remained unsatisfied due to limited support of U.S. guidelines
                \item Forced to recognize Carranza after orig. U.S. ally, Pancho Villa, greatly weakened $\to$ Villa felt betrayed, shooting U.S. miners in Mexico, killing several more past NM border 
                \item U.S. troops unsuccessfully pursued Carranza, engaging only in a few spars w/ Carranza's army $\to$ war betw. U.S. and MX seemed close until Wilson withdrew due to world war
            \end{itemize}
        \end{itemize}
        \textbf{Wilson took an active role in foreign affairs, never afraid to send the military to a potentially dangerous or economically beneficial region, including the Dominican Republic, Haiti, and the Danish West Indies. In Mexico, constant conflict unfolded as governments rapidly transitioned; after the U.S. abandoned an early Mexican ally and they pursued him for his retaliation, the U.S. and Mexico seemed once again close to war.}}
    \end{document}