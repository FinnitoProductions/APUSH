\documentclass[a4paper]{article}
    \usepackage[T1]{fontenc}
    \usepackage{tcolorbox}
    \usepackage{amsmath}
    \tcbuselibrary{skins}
    
    \usepackage{background}
    \SetBgScale{1}
    \SetBgAngle{0}
    \SetBgColor{red}
    \SetBgContents{\rule[0em]{4pt}{\textheight}}
    \SetBgHshift{-2.3cm}
    \SetBgVshift{0cm}
    \usepackage[margin=2cm]{geometry} 
    
    \makeatletter
    \def\cornell{\@ifnextchar[{\@with}{\@without}}
    \def\@with[#1]#2#3{
    \begin{tcolorbox}[enhanced,colback=gray,colframe=black,fonttitle=\large\bfseries\sffamily,sidebyside=true, nobeforeafter,before=\vfil,after=\vfil,colupper=blue,sidebyside align=top, lefthand width=.3\textwidth,
    opacityframe=0,opacityback=.3,opacitybacktitle=1, opacitytext=1,
    segmentation style={black!55,solid,opacity=0,line width=3pt},
    title=#1
    ]
    \begin{tcolorbox}[colback=red!05,colframe=red!25,sidebyside align=top,
    width=\textwidth,nobeforeafter]#2\end{tcolorbox}%
    \tcblower
    \sffamily
    \begin{tcolorbox}[colback=blue!05,colframe=blue!10,width=\textwidth,nobeforeafter]
    #3
    \end{tcolorbox}
    \end{tcolorbox}
    }
    \def\@without#1#2{
    \begin{tcolorbox}[enhanced,colback=white!15,colframe=white,fonttitle=\bfseries,sidebyside=true, nobeforeafter,before=\vfil,after=\vfil,colupper=blue,sidebyside align=top, lefthand width=.3\textwidth,
    opacityframe=0,opacityback=0,opacitybacktitle=0, opacitytext=1,
    segmentation style={black!55,solid,opacity=0,line width=3pt}
    ]
    
    \begin{tcolorbox}[colback=red!05,colframe=red!25,sidebyside align=top,
    width=\textwidth,nobeforeafter]#1\end{tcolorbox}%
    \tcblower
    \sffamily
    \begin{tcolorbox}[colback=blue!05,colframe=blue!10,width=\textwidth,nobeforeafter]
    #2
    \end{tcolorbox}
    \end{tcolorbox}
    }
    \makeatother

    \parindent=0pt
    \usepackage[normalem]{ulem}

    \newcommand{\chapternumber}{23}
    \newcommand{\chaptertitle}{The Great Depression}

    \title{\vspace{-3em}
\begin{tcolorbox}
\Huge\sffamily \begin{center} AP US History  \\
\LARGE Chapter \chapternumber \, - \chaptertitle \\
\Large Finn Frankis \end{center} 
\end{tcolorbox}
\vspace{-3em}
}
\date{}
\author{}
    \begin{document}
        \maketitle
        \SetBgContents{\rule[0em]{4pt}{\textheight}}
        \cornell[Key Concepts]{What are this chapter's key concepts?}{\begin{itemize}
            \item \textbf{7.1.I.C} - Credit/market instability $\to$ early 20th c. Great Depression $\to$ stronger finance regulation
            \item \textbf{7.2.II.B} - $\uparrow$ War production, labor demand during WWI/WWII $\to$ migration to cities for econ. opportunities
        \end{itemize}}
        \cornell[The Coming of the Great Depression]{What factors stimulated the Great Depression?}{\textbf{The Great Depression, triggered by the stock market crash on October 29th, 1929, was caused by excessive reliance on faltering industries, wealth maldistribution, growing debt, and reduced international trade. Over time, it caused the money supply to plummet, limiting purchasing power, income, and the GDP.}}
        \cornell{What immediate factors led up to the crash of the stock market?}{\begin{itemize}
            \item Feb. 1928 saw $\uparrow$ stock prices for continual year and a half; stocks doubled in value, trading grew $\to$ significant speculation w/ firms offering easy credit
            \item Autumn 1929 saw beginning of collapse; two Oct. crashes both met w/ recoveries (second engineered by leading bankers to ensure confidence); Oct. 29, 1929 ("Black Tuesday") saw ultimate crash, beginning of 4-year depression
        \end{itemize}
        \textbf{Starting in February 1928, stock prices began to skyrocket and continually increase for a year and a half; speculation spread rapidly. October 1929 saw several crashes; all efforts to recover ultimately failed on October 29th, where the ultimate crash occurred.}}
        \cornell{What were the long-term causes of the Great Depression?}{\begin{itemize}
            \item No consensus on exact cause; most agree that notable not for occurring (recessions very common) but for severity, length of time
            \item Excessive reliance on construction/automobile industries; decline in late 1920s $\to$ collapse w/ new growing markets (plastics, petroleum, chemicals) not developed enough to compensate
            \item Poor distribution of wealth $\to$ weak consumer demand \underline{among majority} $\to$ general market not wealthy enough to purchase consumer goods
            \item Economic credit structure saw farmers, small banks (particularly agricultural) in great debt w/ customers defaulting on loans; large banks making some very reckless investments
            \item $\downarrow$ Euro. demand for American goods due to high American tariffs, recovering Euro. industry/agri., impoverished Euro. nations like Germany due to WWI
            \item Euro. nations owed large amounts to U.S. banks, reparations from weak Germany/Austria too small to completely pay off
            \begin{itemize}
                \item U.S. govt. would not reduce debts, instead making loans to Euro. governments to pay off previous loans $\to$ increasing debt
                \item Weakening American economy, $\uparrow$ protective tariffs $\to$ Euro. nations struggled to borrow money, sell goods to U.S. $\to$ defaulted on loans w/ collapse of international credit 
            \end{itemize}
        \end{itemize}
        \textbf{Long-term causes of the Great Depression included an excessive reliance on the faltering construction and automobile industries, maldistribution of wealth, reckless investments by large banks and growing debts for farmers and small banks, reduced European demand for U.S. goods due to high tariffs and poverty, as well as the collapse of the international credit structure after World War I.}}
        \cornell{How did the depression progress over time?}{\begin{itemize}
            \item Crash of 1929 was stimulus for several events which revealed great faults of U.S. economy
            \item > 9k U.S. banks out of business betw. 1930-1933 $\to$ depositors lost \$2.5b in deposits, $\downarrow$ money supply $\to$ $\downarrow$ purchasing power, deflation 
            \begin{itemize}
                \item Forced manufacturers to reduce prices, production, workers
                \item Federal Reserve Board raised interest rates in 1931 to protect its own wealth $\to$ money supply furteher reduced
            \end{itemize}
            \item U.S. GDP plummeted by 25\%, investments more than decimated, farm income crushed, reduced consumer price index
        \end{itemize}
        \textbf{The Depression immediately caused several banks to go out of business; the overall money supply became severely reduced, causing layoffs and an overall reduction in purchasing power and the U.S. GDP.}}
        \cornell[The American People in Hard Times]{How did the Great Depression affect American society?}{\textbf{The Great Depression caused widespread unemployment and the collapse of relief programs on the local and state levels. Farmers saw their incomes slashed due to drought and overproduction. African Americans, Mexican Americans, and Asian Americans frequently lost their long-held jobs to whites in need; African Americans responded by migrating to the North (where conditions were little better), Mexican Americans were often deported without justification, Japanese Americans turned to working for family fruit stands, and Chinese Americans for laundries. Women saw great setbacks, with a decline in consumer culture, the returning belief that women belonged in the home, and the frequent desertion by humiliated men of their families.}}
        \cornell{How did the Depression cause widespread unemployment?}{\begin{itemize}
            \item Industrial cities saw unemployment rates as high as 80\% w/ widespread joblessness; adult men felt personally emasculated by no available jobs
            \item Families turned to state public relief systems but handled very few ppl. in early 1920s $\to$ most unprepared for demand $\to$ collapse of relief 
            \begin{itemize}
                \item Private charities unable to meet demand; $\downarrow$ tax revenues meant state govts. too weak to respond w/ tight budgets, feared reduction in morale if welfare system implemented
                \item Lines outside Red Cross/Salvation Army w/ thousands searching garbage for plate scrapings
                \item Young men became nomads travelling freight trains betw. cities
            \end{itemize}
            \item $\downarrow$ farm income $\to$ third of U.S. famrers lost land; Great Plains of South and West saw "Dust Bowl," untimely drought lasting for a decade bringing reduced rainfall, increased heat
            \begin{itemize}
                \item Soil in fertile lands completely lost moisture; bugs devoured all crops
                \item Dust storms suffocated livestock/people
                \item Overproduction continued despite conditions $\to$ many farmers left homes to find jobs; in South, farmers both white/Afr. American roamed lands seeking handouts
                \begin{itemize}
                    \item Known as "Okies" due to origins from Oklahoma
                    \item Rarely found conditions better than those they left 
                \end{itemize}
                \item Growing homelessness, deaths from starvation
            \end{itemize}
        \end{itemize}
        \textbf{Unemployment became rampant: families turned to state and localpublic relief systems but few were equipped to handle such large numbers; lines for basic relief became immensely long, forcing many to become nomads travelling between cities. Farm income reduced, too, due to the pairing of a great drought and continued overproduction. Homelessness expanded and deaths due to starvation very common.}}
        \cornell{How did African Americans suffer from the Depression?}{\begin{itemize}
            \item Beginning of Depression saw over half of Afr. Americans in South mostly as farmers; many left w/o income, land due to price collapses
            \begin{itemize}
                \item Afr. Americans migrated to southern cities but whites felt they had first claim to all work $\to$ displaced Afr. Americans previously holding lower-class jobs
                \item Many whites demanded firing of all Afr. Americans; almost all relief went directly to whites
            \end{itemize}
            \item Many black southerners $\to$ North w/ less direct discrimination but still widespread unemployment
            \item Traditional segregation had little major changes but \textbf{Scottsboro case} after in March 1931, nine black teenagers were arrested from a train in AL, later accused of rape by two white women on train (likely false)
            \begin{itemize}
                \item All-white jury sentenced eight of them to death
                \item Overturned in 1932 w/ new trials pushed by International Labor Defense coming to their sides, NAACP providing assistance; all gradually got freedom after 20 yrs.
            \end{itemize}
            \item Depression saw NAACP working for position for blacks within labor movement, encouraging welcoming of blacks within labor unions
            \begin{itemize}
                \item Walter White of NAACP encouraged Afr. Americans to work not just as strikebreakers $\to$ joined unions
            \end{itemize}
        \end{itemize}
        \textbf{Southern African Americans rapidly lost their jobs as they were replaced with whites, causing a great migration to the North; they were met with employment circumstances little better than in the South. Although institutionalized segregation saw little long-term changes, the Scottsboro case, where nine black teenagers were arrested and unjustifiably accused of rape, saw great support from several organizations. Blacks also took a greater role in the labor movement.}}
        \cornell{How did Mexican Americans face discrimination during the Great Depression?}{\begin{itemize}
            \item Some Mexicans held menial farming jobs as agri. migrants; most lived in cities as unskilled laborers in steel/automobiles/meatpacking 
            \item Whites demanded Hispanic jobs $\to$ Mexican unemployment rose far above white employment 
            \begin{itemize}
                \item Officials suddenly removed many from relief, forced across border: effectively forced to leave country
                \item Rarely had access to American schools, hospitals
            \end{itemize}
            \item Some organized resistance, with some CA Mexicans creating migrant farmworkers union; most migrated to cities and lived in poverty
        \end{itemize}
        \textbf{Most Mexican Americans lived in cities, working as unskilled laborers in various industries; most lost their jobs during the Depression (to be replaced with whites), many were forced to leave the country, and few had access to American relief programs. Most migrated to cities, like Los Angeles, and lived in poverty.}}
        \cornell{How did Asian Americans suffer during the Great Depression?}{\begin{itemize}
            \item In CA, w/ largest Japanese/Chinese American population, even educated Asians (college graduates) forced to work in mainstream professions like family fruit stands
            \begin{itemize}
                \item 20\% of Japanese Ameriacns worked at fruit stands by end of 1930s
                \item Those who found jobs in industrial economy mostly lost to arriving white migrants
            \end{itemize}
            \item CA saw Japanese American Democratic Clubs in cities to work against discrimination; Japanese Americans Citizens League encouraged assimilation 
            \item Chinese Americans generally worked in Chinese-owned laundries; mostly worked entry-level jobs
        \end{itemize}
        \textbf{In California, Asian Americans, regardless of education, turned to mainstream professions, with most Japanese working for family fruit stands and those in the industrial economy losing their jobs to whites. Many cities saw clubs to work against Japanese-American discrimination and encourage assimilation; Chinese Americans saw similar discrimination.}}
        \cornell{How did women suffer during the Great Depression?}{\begin{itemize}
            \item Strengthened belief that women belonged in home w/ most men feeling all jobs should go to men; unfair for women w/ employed husbands to find jobs
            \item Single/married women alike worked in 1930s despite stigma, w/ numbers growing rapidly; majority of those entering workforce were wives/mothers
            \item Professional roles began to be filled by men, female industrial workers more likely to be laid off than men; white women had advantage that nonprofessional stereotypical female jobs were unlikely to be challenged by men
            \item Black women saw massive unemployment w/ reduction in domestic service; greater percent were employed than white woman due to economic need
            \item Feminist movement experienced setbacks 
        \end{itemize}
        \textbf{During the Great Depression, many men began to feel that all jobs should go to men, and women should remain at home. Regardless, the number of employed single and married women increased despite this stigma; stereotypically female unprofessional jobs were the most stable. Black women, too, experienced major layoffs in the domestic service industry. The feminist movement suffered greatly.}}
        \cornell{How did families fare during the Great Depression?}{\begin{itemize}
            \item Econ. hardships w/ most families used to $\uparrow$ standard of living suddenly shocked
            \item Forced to depart from consumerism: women often began sewing clothes themselves, preserving their own food, starting home businesses selling baked goods, accepting boarders
            \begin{itemize}
                \item Households included distant relatives w/ parents moving in with children, grandparents moving in w/ grandchildren (or opposite)
            \end{itemize}
            \item Decline in divorce rates due to economic burden; more common were humiliated unemployed men deserting families
        \end{itemize}
        \textbf{Families were shocked by the suddenly reduced quality of life; households departed from their consumerist tendencies, returning to homemade goods. Humiliated men often deserted their families.}}
        \cornell[The Depression and American Culture]{How did the Great Depression affect American culture?}{\textbf{Social values of individualism and independent success remained widespread; the arts began to focus more on rural poverty than the former isolationism. Radio promoted communal values and a widespread national culture; movies, though heavily restricted, were heavily escapist and took people's minds off the Great Depression. Popular literature, though typically escapist, was able to explore far more radical topics. Furthermore, a leftist political movement supporting the American Communist Party saw a rapid rise in the Popular Front.}}
        \cornell{What were the critical social values during the depression?}{\begin{itemize}
            \item Many responded by affirming traditional values; sociologists published \textit{Middletown in Transition} to show that the culture of the Indiana town remained relatively unchanged; \underline{focus on individual}
            \item Economic crisis $\to$ idea of independent success undermined w/ many turning to govt., blaming corporations, bankers, others; not a long-term effect 
            \item Many others blamed themselves, confined to homes by shame of losing jobs; others found motivation through self-help manuals like \textit{How to Win Friends and Influence People} by Dale Carnegie to find their place
        \end{itemize}
        \textbf{Ideals of individualism and independent success, though undermined by the frequent blame of others for the depression, remained strong; many, in fact, blamed themselves for the recession, deeply ashamed. Others used their collapse to motivate themselves and got back up.}}
        \cornell{How did art develop during the Great Depression?}{\begin{itemize}
            \item Many Americans, like similar shock at urban poverty at turn of century, appalled by rural poverty $\to$ Farm Security Administration employed photographers to document agri. life
            \begin{itemize}
                \item Dorothea Lange, Walker Evans, Roy Stryker, Margaret Bourke-White, others, created in-depth studies to indicate struggle
            \end{itemize}
            \item Many writers turned away from isolationalism of past decade to focus on social injustice; \textit{Tobacco Road} indicated rural Southern Poverty; \textit{Native Son} portrayed urban ghetto; Steinbeck showed California migrant workers; others focused on capitalism and political radicalism
        \end{itemize}
        \textbf{Several American writers turned away from the isolationism of the 1920s and focused on social injustices, particularly by documenting rural poverty.}}
        \cornell{How did the radio transform the American family?}{\begin{itemize}
            \item Widespread by 1930s, even in rural families (powered by car batteries)
            \item Often communal, inviting friends to listen to radio and sit/talk/dance; drew ppl. together
            \item Focus on \textbf{escapist programming} to get away from Depression, like w/ comedies (\textit{Amos 'n Andy} abt. urban blacks), adventure (superheroes); brought new forms of comedy to wider audiences
            \begin{itemize}
                \item Others enjoyed soap operas, particularly women alone in house during the day (sponsored by soap companies due to connection to women working at home)
            \end{itemize}
            \item Radio live $\to$ public performances of music, comedies, theaters
            \item Direct access to news in public events: covered politics, sports, Academy Awards, broadcast of \textit{Hindenburg} crash; Orson Welles mistakenly created widespread panic for false documentary
        \end{itemize}
        \textbf{The radio transformed American society, with several coming together to listen to the programs at once in communal activities. The programs promoted a shared, national culture focused around escapist programming like comedies, adventure, and soap operas performed live. It also provided Americans with direct news access.}}
        \cornell{How did the habit of moviegoing transform during the Great Depression?}{\begin{itemize}
            \item Initially reduced due to limited wealth; resumed by mid-1930s as less expensive entertainment option than most w/ sound, color $\to$ very appealing
            \item Will Hays ensured movies remained convential w/o controversial messages; very strict control but unable to prevent social questions
            \begin{itemize}
                \item Adaptation of Steinbeck's \textit{The Grapes of Wrath} explored politics; Frank Capra provided subtle social message in comedies (\textit{Mr. Deeds Goes to Town}, \textit{Meet John Doe})
                \item Gangster movies (\textit{Little Caesar}, \textit{The Public Enemy}) depicted violent, terrible world which few Americans knew
            \end{itemize}
            \item Generally deliberately escapist w/ lavish musicals, screwball comedies from Capra, Marx Brothers designed to divert from troubles; Walt Disney debuted Mickey Mouse in late 1920s in \textit{Steamboat Willie}
            \item Adaptations of popular novels like \textit{The Wizard of Oz}, \textit{Gone With the Wind}
        \end{itemize}
        \textbf{Although movies experienced a temporary setback due to poverty from the depression, their status as a cheap form of entertainment with color and sound meant few could resist. The industry remained convential under Will Hays; some, however, explored some social and political questions. Most popular movies were escapist comedies or adaptations of novels.}}
        \cornell{How was popular literature transformed during the Great Depression?}{\begin{itemize}
            \item Literature/journalism able to handle directly widespread disillusionment, radicalism
            \item Most popular books were escapist/romantic (like \textit{Gone With the Wind}, \textit{Anthony Adverse})
            \item Photographic magazines like \textit{Life} (some focus on depression but mostly on other matters) focused on fashion/arts of nation
            \begin{itemize}
                \item \textit{Life} showed pictures of Americans having fun, political initiatives
            \end{itemize}
            \item Some Depression writing directly challenged American values; West's \textit{Miss Lonelyhearts} showed sadness in lives of many; \textit{The Disinherited} showed lives of coal miners; \textit{Studs Lonigan} showed working-class youth
        \end{itemize}
        \textbf{Literature, far freer than movies and radios, though generally remaining escapist through powerful photographs and romantic novels, often directly challenged the fabric of American cultural values.}}
        \cornell{How did the Popular Front rise in political prominence?}{\begin{itemize}
            \item Popular Front was coalition of "antifascist" groups (notably American Communist Party) against capitalism; began to support Franklin Roosevelt in 1935 bc. Stalin saw as ally in battle against Hitler
            \begin{itemize}
                \item Supported John L. Lewis, anticommunist labor leader to focus on "Americanism"; improved reputation of Communist Party, also encouraged criticism
                \item Allowed escape from isolationism; Spanish Civil War gave meaning to individuals w/ facists against republican govt. $\to$ Americans, under Abraham Lincoln Brigade, traveled to Spain and fought fascists, losing half
                \item Communist Party organized 1930s unemployed w/ D.C. hunger march; often took stand against racisl injustice
                \item Communist Party not truly patriotic; worked under USSR supervision, obediently following Moscow; subordination indicated at beginning of WWII, when USSR forced to return to criticism of liberals
                \item Socialist Party say depression as failure of capitalism $\to$ sought to further their own needs, particularly among rural poor w/ Southern Tenant Farmers' Union, biracial coalition; no true progress w/ declining membership
            \end{itemize}
            \item Widespread antiradicalism w/ direct govt. hostility toward Communist Party; direct imprisonment of perceived organizers; many attempted to drive out 
            \item More conventional to be part of left $\to$ rapid widening of mainstream art/politics, w/ New Deal art often challenging capitalist structures 
            \begin{itemize}
                \item Steinbeck, in \textit{Grapes of Wrath}, emphasized social conditions by describing voyage of Joad family from Dust Bowl to CA, continually facing failures
            \end{itemize}
        \end{itemize}
        \textbf{Several antifascist groups, collectively forming the Popular Front, rapidly grew in popularity, most strongly driven by the American Communist Party; partaking in the Spanish Civil War, organizing hunger marches, the Communist Party appeared openly patriotic but in fact took direct orders from Moscow. The Socialist Party, despite making several efforts, declined in influence. This influx in leftward thinking, despite spurring antiradicalism, also led to an expansion of art challenging capitalism.}}
        \cornell[The Unhappy Presidency of Herbert Hoover]{How did Hoover address the Great Depression?}{\textbf{Hoover's policies were immensely unsuccessful: he initially attempted voluntary cooperation between corporate leaders, but when straits became dire in 1931 as European nations began to default on their U.S. loans, he created the RFC, intended to use its large sums of money to support all economic aspects of society; it spent little of its money and only favored large corporations. Some groups protested during Hoover's presidency, most notably army veterans, who camped out at Washington for several months, demanding their WWI bonus payments in advance. Hoover was seen as aloof and disconnected from the people; New York governor Franklin Delano Roosevelt easily won the election with his limited controversies and broad promises for economic restoration.}}
        \cornell{What were Hoover's pre-Depression strides?}{\textbf{Hoover began his presidency believing U.S. would remain prosperous, hence he attempted to expand his former commerce policies. However, he was faced with the Depression before the end of the year; he relied on the policies which had forever governed his life of public service.}}
        \cornell{What was the Hoover Program?}{\begin{itemize}
            \item Hoover initially sought to guarantee public confidence; publicly stressed economy was still strong w/ voluntary cooperation betw. business leaders for recovery (no layoffs or production cuts but no rising of wages)
            \begin{itemize}
                \item Voluntary cooperation collapsed by 1931 due to drastic state of economy
            \end{itemize}
            \item Sought to combat Depression w/ $\uparrow$ govt. spending: proposed \$423m increase in funding on publics work progress but far from enough; unwilling to continue increasing spending as conditions got worse
            \begin{itemize}
                \item In 1932, proposed tax increase to prevent govt. deficit
            \end{itemize}
            \item April 1929: proposed Agricultural Marketing Act to maintain farmer prices
            \begin{itemize}
                \item Sought Farm Board to make loans to marketing cooperatives, push corporations to buy surplus crops
                \item Raised agri. tariffs in Tariff Act of 1930 to protect domestic farming; ultimately stifled food exports
            \end{itemize}
            \item Popularity rapidly declined by spring 1931 w/ Democrats winning house in midterms, coming close to Senate due to ppl. personally blaming president
            \begin{itemize}
                \item Hope of nearing end lost after Europeans unable to take loans from U.S. $\to$ unable to pay back debts $\to$ banks in Europe declined, American economy crushed
            \end{itemize}
            \item Desperate Hoover supported measures to protect banks/mortgages in late 1931; passed bill in early 1932 to create Reconstruction Finance Corporation
            \begin{itemize}
                \item RFC designed to provide loans to banks, railroads, local governments on \underline{large scale}
                \item Lent only to largest banks/corporations (collateral); only financed public works which would pay for themselves, like toll bridges
                \item Did not spend majority of its money
            \end{itemize}
        \end{itemize}
        \textbf{Hoover initially took an approach requiring voluntary cooperation between corporate leaders to increase public confidence; he sought to spend significant chunks of money, but even this was not enough to quell the Depression. He attempted to reform the agricultural sector by raising protective tariffs and supporting a fixed crop price; he ultimately stifled exports and had little overall effect. As his popularity began to decline and hope was lost as European nations defaulted on their loans, he created the RFC, which, despite receiving a large sum of money, lent only to large corporations and banks and did not spend most of its money. }}
        \cornell{How did Americans begin to protest the Great Depression?}{
            Most Americans were too stunned to protest at the beginning of Depression; mid-1932 saw the beginning of dissent.
            \begin{itemize}
                \item Summer 1932: farm owners met in Des Moines, established \textbf{Farmers' Holiday Association} to withhold products from market; some early traction w/ some markets blockaded but ultimately failed
                \item American veterans protested in 1932: all WWI veterans had been promised \$1k bonus starting in 1945; many began to demand immediately
                \begin{itemize}
                    \item 20k marched into Washington, camped out; some departed after Hoover rejected for second time but many stayed
                    \item Police ordered to remove veterans from unused federal buildings; veterans retaliated w/ rocks and guns $\to$ Hoover called in U.S. Army led by Douglas MacArthur w/ aide Eisenhower
                    \begin{itemize}
                        \item Third Cavalry, two infrantry regiments sent to PA $\to$ veterans fled, forming nearby tent city but burned down
                    \end{itemize}
                    \item Symbolized Hoover's failure to sympathize w/ ppl.
                \end{itemize}
            \end{itemize}
            \textbf{Two major revolts occurred: one by farm owners who attempted to blockade the market by not selling crops, which ended in failure, and one by veterans who sought to receive their \$1,000 bonus for fighting in WWI earlier than promised, camping out in Washington; the veterans were ultimately attacked by the U.S. Army at Hoover's demand, reflecting his overall lack of sympathy for the people.}}
            \cornell{What was the result of the election of 1932?}{
                Few believed Hoover could win the 1932 election. Democrats nominated the New York governor, \textbf{Franklin Delano Roosevelt}
                \begin{itemize}
                    \item Roosevelt was well-known Hudson Valley aristocrat, distant cousin of Teddy Roosevelt, married to his niece, Eleanor 
                    \item Rapidly rose ranks of party: started w/ seat in NY state legislature, then to assistant navy secretary during WWI, Dem. VP nomination in 1920 w/ James M. Cox
                    \item Polio in 1921 $\to$ lost use of legs (only able to walk w/ leg brace); returned to politics in 1928, succeeding Al Smith, Dem. presidential nominee as NY governor; won again in 1930 
                    \begin{itemize}
                        \item Work in NY not miraculous, but some positive programs; seen as energetic, hard-working leader 
                        \item Avoided controvesial issues like prohibition
                    \end{itemize}
                    \item Easily won party nomination, promised 'new deal' for American ppl. w/ little indication as to what it would be
                    \begin{itemize}
                        \item Won election regardless by a landslide (479 electoral votes to Hoover's 59); won both houses of Congress
                    \end{itemize}
                \end{itemize}
            \textbf{FDR, a well-known politician and aristocrat, rapidly rose the ranks of his party and became the governor of New York in 1928; he easily won the Democratic nomination in 1932. He promised a 'new deal' for the American people with few specifics; Hoover's destroyed reputation meant that FDR easily won the election.}}
            \cornell{What characterized the period between the election and FDR's inauguration?}{\textbf{Presidents-elect rarely get directly involved with government; however, Hoover and Roosevelt conflicted greatly during this period: Hoover hoped Roosevelt would assure that the would not make dramatic economic changes amidst a banking collapse, but he refused. On Roosevelt's inauguration, tensions were high between the bitter Hoover and the hopeful, confident Roosevelt.}}
        \end{document}