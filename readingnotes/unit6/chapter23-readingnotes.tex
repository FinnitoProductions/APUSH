\documentclass[a4paper]{article}
    \input{../notesheader.tex}
    \usepackage[normalem]{ulem}

    \newcommand{\chapternumber}{23}
    \newcommand{\chaptertitle}{The Great Depression}
    \title{\vspace{-3em}
    \begin{tcolorbox}
    \Huge\sffamily \begin{center} AP US History  \\
    \LARGE Chapter \chapternumber \, - \chaptertitle \\
    \Large Finn Frankis \end{center} 
    \end{tcolorbox}
    \vspace{-3em}
    }
    \date{}
    \author{}
    
    \begin{document}
        \maketitle
        \SetBgContents{\rule[0em]{4pt}{\textheight}}
        \cornell[Key Concepts]{What are this chapter's key concepts?}{\begin{itemize}
            \item \textbf{7.1.I.C} - Credit/market instability $\to$ early 20th c. Great Depression $\to$ stronger finance regulation
            \item \textbf{7.2.II.B} - $\uparrow$ War production, labor demand during WWI/WWII $\to$ migration to cities for econ. opportunities
        \end{itemize}}
        \cornell[The Coming of the Great Depression]{What factors stimulated the Great Depression?}{\textbf{The Great Depression, triggered by the stock market crash on October 29th, 1929, was caused by excessive reliance on faltering industries, wealth maldistribution, growing debt, and reduced international trade. Over time, it caused the money supply to plummet, limiting purchasing power, income, and the GDP.}}
        \cornell{What immediate factors led up to the crash of the stock market?}{\begin{itemize}
            \item Feb. 1928 saw $\uparrow$ stock prices for continual year and a half; stocks doubled in value, trading grew $\to$ significant speculation w/ firms offering easy credit
            \item Autumn 1929 saw beginning of collapse; two Oct. crashes both met w/ recoveries (second engineered by leading bankers to ensure confidence); Oct. 29, 1929 ("Black Tuesday") saw ultimate crash, beginning of 4-year depression
        \end{itemize}
        \textbf{Starting in February 1928, stock prices began to skyrocket and continually increase for a year and a half; speculation spread rapidly. October 1929 saw several crashes; all efforts to recover ultimately failed on October 29th, where the ultimate crash occurred.}}
        \cornell{What were the long-term causes of the Great Depression?}{\begin{itemize}
            \item No consensus on exact cause; most agree that notable not for occurring (recessions very common) but for severity, length of time
            \item Excessive reliance on construction/automobile industries; decline in late 1920s $\to$ collapse w/ new growing markets (plastics, petroleum, chemicals) not developed enough to compensate
            \item Poor distribution of wealth $\to$ weak consumer demand \underline{among majority} $\to$ general market not wealthy enough to purchase consumer goods
            \item Economic credit structure saw farmers, small banks (particularly agricultural) in great debt w/ customers defaulting on loans; large banks making some very reckless investments
            \item $\downarrow$ Euro. demand for American goods due to high American tariffs, recovering Euro. industry/agri., impoverished Euro. nations like Germany due to WWI
            \item Euro. nations owed large amounts to U.S. banks, reparations from weak Germany/Austria too small to completely pay off
            \begin{itemize}
                \item U.S. govt. would not reduce debts, instead making loans to Euro. governments to pay off previous loans $\to$ increasing debt
                \item Weakening American economy, $\uparrow$ protective tariffs $\to$ Euro. nations struggled to borrow money, sell goods to U.S. $\to$ defaulted on loans w/ collapse of international credit 
            \end{itemize}
        \end{itemize}
        \textbf{Long-term causes of the Great Depression included an excessive reliance on the faltering construction and automobile industries, maldistribution of wealth, reckless investments by large banks and growing debts for farmers and small banks, reduced European demand for U.S. goods due to high tariffs and poverty, as well as the collapse of the international credit structure after World War I.}}
        \cornell{How did the depression progress over time?}{\begin{itemize}
            \item Crash of 1929 was stimulus for several events which revealed great faults of U.S. economy
            \item > 9k U.S. banks out of business betw. 1930-1933 $\to$ depositors lost \$2.5b in deposits, $\downarrow$ money supply $\to$ $\downarrow$ purchasing power, deflation 
            \begin{itemize}
                \item Forced manufacturers to reduce prices, production, workers
                \item Federal Reserve Board raised interest rates in 1931 to protect its own wealth $\to$ money supply furteher reduced
            \end{itemize}
            \item U.S. GDP plummeted by 25\%, investments more than decimated, farm income crushed, reduced consumer price index
        \end{itemize}
        \textbf{The Depression immediately caused several banks to go out of business; the overall money supply became severely reduced, causing layoffs and an overall reduction in purchasing power and the U.S. GDP.}}
        \cornell[The American People in Hard Times]{How did the Great Depression affect American society?}{\textbf{The Great Depression caused widespread unemployment and the collapse of relief programs on the local and state levels. Farmers saw their incomes slashed due to drought and overproduction. African Americans, Mexican Americans, and Asian Americans frequently lost their long-held jobs to whites in need; African Americans responded by migrating to the North (where conditions were little better), Mexican Americans were often deported without justification, Japanese Americans turned to working for family fruit stands, and Chinese Americans for laundries. Women saw great setbacks, with a decline in consumer culture, the returning belief that women belonged in the home, and the frequent desertion by humiliated men of their families.}}
        \cornell{How did the Depression cause widespread unemployment?}{\begin{itemize}
            \item Industrial cities saw unemployment rates as high as 80\% w/ widespread joblessness; adult men felt personally emasculated by no available jobs
            \item Families turned to state public relief systems but handled very few ppl. in early 1920s $\to$ most unprepared for demand $\to$ collapse of relief 
            \begin{itemize}
                \item Private charities unable to meet demand; $\downarrow$ tax revenues meant state govts. too weak to respond w/ tight budgets, feared reduction in morale if welfare system implemented
                \item Lines outside Red Cross/Salvation Army w/ thousands searching garbage for plate scrapings
                \item Young men became nomads travelling freight trains betw. cities
            \end{itemize}
            \item $\downarrow$ farm income $\to$ third of U.S. famrers lost land; Great Plains of South and West saw "Dust Bowl," untimely drought lasting for a decade bringing reduced rainfall, increased heat
            \begin{itemize}
                \item Soil in fertile lands completely lost moisture; bugs devoured all crops
                \item Dust storms suffocated livestock/people
                \item Overproduction continued despite conditions $\to$ many farmers left homes to find jobs; in South, farmers both white/Afr. American roamed lands seeking handouts
                \begin{itemize}
                    \item Known as "Okies" due to origins from Oklahoma
                    \item Rarely found conditions better than those they left 
                \end{itemize}
                \item Growing homelessness, deaths from starvation
            \end{itemize}
        \end{itemize}
        \textbf{Unemployment became rampant: families turned to state and localpublic relief systems but few were equipped to handle such large numbers; lines for basic relief became immensely long, forcing many to become nomads travelling between cities. Farm income reduced, too, due to the pairing of a great drought and continued overproduction. Homelessness expanded and deaths due to starvation very common.}}
        \cornell{How did African Americans suffer from the Depression?}{\begin{itemize}
            \item Beginning of Depression saw over half of Afr. Americans in South mostly as farmers; many left w/o income, land due to price collapses
            \begin{itemize}
                \item Afr. Americans migrated to southern cities but whites felt they had first claim to all work $\to$ displaced Afr. Americans previously holding lower-class jobs
                \item Many whites demanded firing of all Afr. Americans; almost all relief went directly to whites
            \end{itemize}
            \item Many black southerners $\to$ North w/ less direct discrimination but still widespread unemployment
            \item Traditional segregation had little major changes but \textbf{Scottsboro case} after in March 1931, nine black teenagers were arrested from a train in AL, later accused of rape by two white women on train (likely false)
            \begin{itemize}
                \item All-white jury sentenced eight of them to death
                \item Overturned in 1932 w/ new trials pushed by International Labor Defense coming to their sides, NAACP providing assistance; all gradually got freedom after 20 yrs.
            \end{itemize}
            \item Depression saw NAACP working for position for blacks within labor movement, encouraging welcoming of blacks within labor unions
            \begin{itemize}
                \item Walter White of NAACP encouraged Afr. Americans to work not just as strikebreakers $\to$ joined unions
            \end{itemize}
        \end{itemize}
        \textbf{Southern African Americans rapidly lost their jobs as they were replaced with whites, causing a great migration to the North; they were met with employment circumstances little better than in the South. Although institutionalized segregation saw little long-term changes, the Scottsboro case, where nine black teenagers were arrested and unjustifiably accused of rape, saw great support from several organizations. Blacks also took a greater role in the labor movement.}}
        \cornell{How did Mexican Americans face discrimination during the Great Depression?}{\begin{itemize}
            \item Some Mexicans held menial farming jobs as agri. migrants; most lived in cities as unskilled laborers in steel/automobiles/meatpacking 
            \item Whites demanded Hispanic jobs $\to$ Mexican unemployment rose far above white employment 
            \begin{itemize}
                \item Officials suddenly removed many from relief, forced across border: effectively forced to leave country
                \item Rarely had access to American schools, hospitals
            \end{itemize}
            \item Some organized resistance, with some CA Mexicans creating migrant farmworkers union; most migrated to cities and lived in poverty
        \end{itemize}
        \textbf{Most Mexican Americans lived in cities, working as unskilled laborers in various industries; most lost their jobs during the Depression (to be replaced with whites), many were forced to leave the country, and few had access to American relief programs. Most migrated to cities, like Los Angeles, and lived in poverty.}}
        \cornell{How did Asian Americans suffer during the Great Depression?}{\begin{itemize}
            \item In CA, w/ largest Japanese/Chinese American population, even educated Asians (college graduates) forced to work in mainstream professions like family fruit stands
            \begin{itemize}
                \item 20\% of Japanese Ameriacns worked at fruit stands by end of 1930s
                \item Those who found jobs in industrial economy mostly lost to arriving white migrants
            \end{itemize}
            \item CA saw Japanese American Democratic Clubs in cities to work against discrimination; Japanese Americans Citizens League encouraged assimilation 
            \item Chinese Americans generally worked in Chinese-owned laundries; mostly worked entry-level jobs
        \end{itemize}
        \textbf{In California, Asian Americans, regardless of education, turned to mainstream professions, with most Japanese working for family fruit stands and those in the industrial economy losing their jobs to whites. Many cities saw clubs to work against Japanese-American discrimination and encourage assimilation; Chinese Americans saw similar discrimination.}}
        \cornell{How did women suffer during the Great Depression?}{\begin{itemize}
            \item Strengthened belief that women belonged in home w/ most men feeling all jobs should go to men; unfair for women w/ employed husbands to find jobs
            \item Single/married women alike worked in 1930s despite stigma, w/ numbers growing rapidly; majority of those entering workforce were wives/mothers
            \item Professional roles began to be filled by men, female industrial workers more likely to be laid off than men; white women had advantage that nonprofessional stereotypical female jobs were unlikely to be challenged by men
            \item Black women saw massive unemployment w/ reduction in domestic service; greater percent were employed than white woman due to economic need
            \item Feminist movement experienced setbacks 
        \end{itemize}
        \textbf{During the Great Depression, many men began to feel that all jobs should go to men, and women should remain at home. Regardless, the number of employed single and married women increased despite this stigma; stereotypically female unprofessional jobs were the most stable. Black women, too, experienced major layoffs in the domestic service industry. The feminist movement suffered greatly.}}
        \cornell{How did families fare during the Great Depression?}{\begin{itemize}
            \item Econ. hardships w/ most families used to $\uparrow$ standard of living suddenly shocked
            \item Forced to depart from consumerism: women often began sewing clothes themselves, preserving their own food, starting home businesses selling baked goods, accepting boarders
            \begin{itemize}
                \item Households included distant relatives w/ parents moving in with children, grandparents moving in w/ grandchildren (or opposite)
            \end{itemize}
            \item Decline in divorce rates due to economic burden; more common were humiliated unemployed men deserting families
        \end{itemize}
        \textbf{Families were shocked by the suddenly reduced quality of life; households departed from their consumerist tendencies, returning to homemade goods. Humiliated men often deserted their families.}}
        \end{document}