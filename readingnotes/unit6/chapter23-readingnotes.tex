\documentclass[a4paper]{article}
    \input{../notesheader.tex}
    \usepackage[normalem]{ulem}

    \newcommand{\chapternumber}{23}
    \newcommand{\chaptertitle}{The Great Depression}
    \title{\vspace{-3em}
    \begin{tcolorbox}
    \Huge\sffamily \begin{center} AP US History  \\
    \LARGE Chapter \chapternumber \, - \chaptertitle \\
    \Large Finn Frankis \end{center} 
    \end{tcolorbox}
    \vspace{-3em}
    }
    \date{}
    \author{}
    
    \begin{document}
        \maketitle
        \SetBgContents{\rule[0em]{4pt}{\textheight}}
        \cornell[Key Concepts]{What are this chapter's key concepts?}{\begin{itemize}
            \item \textbf{7.1.I.C} - Credit/market instability $\to$ early 20th c. Great Depression $\to$ stronger finance regulation
            \item \textbf{7.2.II.B} - $\uparrow$ War production, labor demand during WWI/WWII $\to$ migration to cities for econ. opportunities
        \end{itemize}}
        \cornell[The Coming of the Great Depression]{What factors stimulated the Great Depression?}{\textbf{The Great Depression, triggered by the stock market crash on October 29th, 1929, was caused by excessive reliance on faltering industries, wealth maldistribution, growing debt, and reduced international trade. Over time, it caused the money supply to plummet, limiting purchasing power, income, and the GDP.}}
        \cornell{What immediate factors led up to the crash of the stock market?}{\begin{itemize}
            \item Feb. 1928 saw $\uparrow$ stock prices for continual year and a half; stocks doubled in value, trading grew $\to$ significant speculation w/ firms offering easy credit
            \item Autumn 1929 saw beginning of collapse; two Oct. crashes both met w/ recoveries (second engineered by leading bankers to ensure confidence); Oct. 29, 1929 ("Black Tuesday") saw ultimate crash, beginning of 4-year depression
        \end{itemize}
        \textbf{Starting in February 1928, stock prices began to skyrocket and continually increase for a year and a half; speculation spread rapidly. October 1929 saw several crashes; all efforts to recover ultimately failed on October 29th, where the ultimate crash occurred.}}
        \cornell{What were the long-term causes of the Great Depression?}{\begin{itemize}
            \item No consensus on exact cause; most agree that notable not for occurring (recessions very common) but for severity, length of time
            \item Excessive reliance on construction/automobile industries; decline in late 1920s $\to$ collapse w/ new growing markets (plastics, petroleum, chemicals) not developed enough to compensate
            \item Poor distribution of wealth $\to$ weak consumer demand \underline{among majority} $\to$ general market not wealthy enough to purchase consumer goods
            \item Economic credit structure saw farmers, small banks (particularly agricultural) in great debt w/ customers defaulting on loans; large banks making some very reckless investments
            \item $\downarrow$ Euro. demand for American goods due to high American tariffs, recovering Euro. industry/agri., impoverished Euro. nations like Germany due to WWI
            \item Euro. nations owed large amounts to U.S. banks, reparations from weak Germany/Austria too small to completely pay off
            \begin{itemize}
                \item U.S. govt. would not reduce debts, instead making loans to Euro. governments to pay off previous loans $\to$ increasing debt
                \item Weakening American economy, $\uparrow$ protective tariffs $\to$ Euro. nations struggled to borrow money, sell goods to U.S. $\to$ defaulted on loans w/ collapse of international credit 
            \end{itemize}
        \end{itemize}
        \textbf{Long-term causes of the Great Depression included an excessive reliance on the faltering construction and automobile industries, maldistribution of wealth, reckless investments by large banks and growing debts for farmers and small banks, reduced European demand for U.S. goods due to high tariffs and poverty, as well as the collapse of the international credit structure after World War I.}}
        \cornell{How did the depression progress over time?}{\begin{itemize}
            \item Crash of 1929 was stimulus for several events which revealed great faults of U.S. economy
            \item > 9k U.S. banks out of business betw. 1930-1933 $\to$ depositors lost \$2.5b in deposits, $\downarrow$ money supply $\to$ $\downarrow$ purchasing power, deflation 
            \begin{itemize}
                \item Forced manufacturers to reduce prices, production, workers
                \item Federal Reserve Board raised interest rates in 1931 to protect its own wealth $\to$ money supply furteher reduced
            \end{itemize}
            \item U.S. GDP plummeted by 25\%, investments more than decimated, farm income crushed, reduced consumer price index
        \end{itemize}
        \textbf{The Depression immediately caused several banks to go out of business; the overall money supply became severely reduced, causing layoffs and an overall reduction in purchasing power and the U.S. GDP.}}
        \end{document}