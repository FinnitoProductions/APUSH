\documentclass[a4paper]{article}
    \usepackage[T1]{fontenc}
    \usepackage{tcolorbox}
    \usepackage{amsmath}
    \tcbuselibrary{skins}
    
    \usepackage{background}
    \SetBgScale{1}
    \SetBgAngle{0}
    \SetBgColor{red}
    \SetBgContents{\rule[0em]{4pt}{\textheight}}
    \SetBgHshift{-2.3cm}
    \SetBgVshift{0cm}
    \usepackage[margin=2cm]{geometry} 
    
    \makeatletter
    \def\cornell{\@ifnextchar[{\@with}{\@without}}
    \def\@with[#1]#2#3{
    \begin{tcolorbox}[enhanced,colback=gray,colframe=black,fonttitle=\large\bfseries\sffamily,sidebyside=true, nobeforeafter,before=\vfil,after=\vfil,colupper=blue,sidebyside align=top, lefthand width=.3\textwidth,
    opacityframe=0,opacityback=.3,opacitybacktitle=1, opacitytext=1,
    segmentation style={black!55,solid,opacity=0,line width=3pt},
    title=#1
    ]
    \begin{tcolorbox}[colback=red!05,colframe=red!25,sidebyside align=top,
    width=\textwidth,nobeforeafter]#2\end{tcolorbox}%
    \tcblower
    \sffamily
    \begin{tcolorbox}[colback=blue!05,colframe=blue!10,width=\textwidth,nobeforeafter]
    #3
    \end{tcolorbox}
    \end{tcolorbox}
    }
    \def\@without#1#2{
    \begin{tcolorbox}[enhanced,colback=white!15,colframe=white,fonttitle=\bfseries,sidebyside=true, nobeforeafter,before=\vfil,after=\vfil,colupper=blue,sidebyside align=top, lefthand width=.3\textwidth,
    opacityframe=0,opacityback=0,opacitybacktitle=0, opacitytext=1,
    segmentation style={black!55,solid,opacity=0,line width=3pt}
    ]
    
    \begin{tcolorbox}[colback=red!05,colframe=red!25,sidebyside align=top,
    width=\textwidth,nobeforeafter]#1\end{tcolorbox}%
    \tcblower
    \sffamily
    \begin{tcolorbox}[colback=blue!05,colframe=blue!10,width=\textwidth,nobeforeafter]
    #2
    \end{tcolorbox}
    \end{tcolorbox}
    }
    \makeatother

    \parindent=0pt
    \usepackage[normalem]{ulem}

    \newcommand{\chapternumber}{24}
    \newcommand{\chaptertitle}{The Great Depression}
    \title{\vspace{-3em}
    \begin{tcolorbox}
    \Huge\sffamily \begin{center} AP US History  \\
    \LARGE Chapter \chapternumber \, - \chaptertitle \\
    \Large Finn Frankis \end{center} 
    \end{tcolorbox}
    \vspace{-3em}
    }
    \date{}
    \author{}
    
    \begin{document}
        \maketitle
        \SetBgContents{\rule[0em]{4pt}{\textheight}}
        \cornell[Key Concepts]{What are this chapter's key concepts?}{\begin{itemize}
            \item \textbf{7.1.I.C} - Credit/market instability during the Great Depression $\to$ demand for stronger financial system 
            \item \textbf{7.1.III.A} - FDR's New Deal used govt. power to attempt to end Great Depression w/ relief to poor, economic reform
            \item \textbf{7.1.III.B} - Radical/populist/union movements $\to$ FDR pressed to make more radical changes to econ. system; conservatives in Congress attempted to limit efforts
            \item \textbf{7.1.III.C} - New Deal did not end Depression; left reform/regulatory legacy, encouraged long-term pol. realignment w/ Afr. Americans, oppressed grps. turning to Democratic Party
        \end{itemize}}
        \cornell[Launching the New Deal]{What were Roosevelt's early measures after taking office?}{\textbf{After taking office, Roosevelt initially attempted to restore confidence with his fireside chats; he also reformed the agricultural system with the Agricultural Adjustment Administration to prevent overproduction, attempted to recover industry with the National Industry Recovery Act to generate new jobs through industrial regulation and price raising. There were some attempts to restructure regional development, notably with the Tennessee Valley Association. He also took far more federal intervention in the economy by directly experimenting with the currency and policing the markets. The wide majority of relief was given in the form of new jobs for temporary projects, both rural and urban.}}
        \cornell{How did Roosevelt seek to restore confidence within the American population?}{\begin{itemize}
            \item Inaugaural address focused on public image w/ "the only thing we have to fear is fear itself," promising immediate strong action 
            \begin{itemize}
                \item Built confidence w/ frequent "fireside chats" to explain programs to ppl., chronicle progress
                \item Unwritten agreement $\to$ never photographed getting out of car/in wheelchair $\to$ few Americans knew abt. paralysis
            \end{itemize}
            \item Immediately instituted banking reform by closing all U.S. banks for four days until Congress special session
            \begin{itemize}
                \item Known as "bank holiday"; three days in, produced \textbf{Emergency Banking Act}, conservative bill somewhat inspired by Hoover intended to protect stronger banks from weakness of smaller ones
                \item Treasury Department given required inspection of all banks before reopening; federal govt. would assist failing institutions; most demolished would be reorganized
                \item Helped to limit panic w/ three-quarters of banks reopening w/in 3 days, currency soon flowing back in
            \end{itemize}
            \item Soon after Emergency Banking Act, Roosevelt sent \textbf{Economy Act} to convince fiscally conservative Americans that federal govt. remained safe/prosperous
            \begin{itemize}
                \item Cut salaries of govt. employees, reduced pensions to balance budget away from \$1b debt 
            \end{itemize}
            \item Created new constitutional amendment to repeal Prohibition, ratified in 1933
        \end{itemize}
        \textbf{Roosevelt sought to restore public confidence through a strong, hard-working public image, the closing and reopening of all banks with a "bank holiday" to give Congress time to institute reform, and the balancing of the federal government by cutting employee salaries. He also repealed the Prohibition, a very popular choice.}}
        \cornell{How did Roosevelt reform the agricultural system?}{\begin{itemize}
            \item First comprehensive program: \textbf{Agricultural Adjustment Act} in May 1933 designed to limit overproduction
            \item Created Agricultural Adjustment Administration (AAA) to pay farmers subsidies to not use entirety of land, set specific production limits on crops 
            \begin{itemize}
                \item Created new tax on food processing to find sufficient funds to bring crop prices to parity levels, pay farmers
                \item Encouraged rise in prices for farm commodities w/ gross farm income greatly increasing, agricultural economy flourishing and reaching far more stable state
                \item Favored larger farmers because local administration of programs often delegated to largest producers; created no incentive for planters not to fire their smaller tenant farmers $\to$ many lost jobs 
                \item Supreme Court shrunk scope of AAA in 1936, arguing govt. had no constitutional authority to restrict farmers; govt. soon secured new legislation to reduce production w/ outward purpose of "conserving soil"; no more court interference
            \end{itemize}
            \item Administration began efforts for poorer farmers w/ \textbf{Resettlement Administration} (1935) and successor, \textbf{Farm Security Administration} (1937) provided loans for farmers on poor soil 
            \begin{itemize}
                \item Relatively unsuccessful, moving only a few thousand
                \item \textbf{Rural Electrification Administration} (1935) far more successful in distributing power to farmers
            \end{itemize}
        \end{itemize}
        \textbf{FDR created the Agricultural Adjustment Act to limit overproduction by encouraging farmers to produce fewer crops through subsidies, though its policies generally favored larger farmers and it faced some resistance from the Supreme Court. Some relatively unsuccessful efforts to assist poorer farmers began in 1935; most successful was the process of distributing power to these regions.}}
        \cornell{How did FDR initiate programs to recover industry?}{\begin{itemize}
            \item U.S. Chamber of Commerce had been pushing for antideflation since 1931 to allow trade associations to deliberately stabilize prices w/in their own industries
            \begin{itemize}
                \item Forbidden by antitrust laws $\to$ Hoover strongly opposed; Roosevelt far more willing to compromise
                \item Roosevelt wanted business leaders to support workers by recognizing bargaining right through unions to allow incomes to increase w/ prices 
                \item Passed \textbf{National Industrial Recovery Act} to create new public works projects, generate new jobs
            \end{itemize}
            \item Initially seemed successful w/ \textbf{National Recovery Administration} (NRA) led by Hugh S. Johnson 
            \begin{itemize}
                \item Encouraged "blanket code" for \underline{all businesses}: basic minimum wage, no child labor, maximum workweek of 40 hrs. all to raise consumer purchasing power 
                \item Worked w/ leaders of major industries to set lowpoint for wages (often desired for competitive advantage); formed comprehensive agreements
                \item Faced major difficulties due to poorly written codes, inexperienced federal officials selected to administer, producers generally given control over codes $\to$ heavily favored larger industry, codes often artificially set prices to unrealistic values
                \item Unable to enforce several codes
                \begin{itemize}
                    \item Promised right to form unions in 7(a) w/o enforcement mechanisms $\to$ no employer recognition
                    \item \textbf{Public Works Administration} (PWA) had large amount of funds but spending programs required slow trickle $\to$ unable to use most of money until 1938
                \end{itemize}
                \item Visibly unsuccessful w/ industrial production declining despite rise in prices; large criticism $\to$ FDR pushed Johnson to resign, created entirely new board
            \end{itemize}
        \end{itemize}
        \textbf{FDR attempted to recover industrial growth through the NRA, intended to regulate industrial prices. Led by Hugh S. Johnson, it worked directly with the largest companies and also established "blanket codes" to regulate all business. However, due to its poor administration, heavy support of big business, and inability to enforce many of its codes, it caused industrial production to decline despite raising prices.}}
        \cornell{How did the NRA ultimately end?}{\begin{itemize}
            \item Supreme Court intervened in 1935: after Schechter brothers accused of breaking NRA codes
            \begin{itemize}
                \item Supreme Court deemed brothers \underline{not engaged in interstate commerce} $\to$ not subject to federal regulation; felt NRA to be unconstitutional $\to$ ended despite FDR's resistance (though he knew it was a failed program)
            \end{itemize}
        \end{itemize}
        \textbf{It ended after Supreme Court intervention deemed it an unconstitutional application of federal authority.}}
        \cornell{How did the New Deal plan regional development?}{\begin{itemize}
            \item AAA/NRA reflected support of economic growth in public sphere but desiring planning process for regional development to remain restricted to private sphere
            \item Some reformers sought fed. control over regional planning: \textbf{Tennessee Valley Authority} (TVA) originated after large private utility company collapsed
            \begin{itemize}
                \item Utilities companies unable to resist public discontent $\to$ TVA created to regulate Tennesee Valley, completing unfinished dam in AL on Tennessee River, generating electricity and encouraging redevelopment
                \item Stopped major flooding in TN Valley, encouraging local industry through reforestation programs, assisted farmer productivity
                \item Improved water transportation, provided electricity; region remained impoverished due to unwillingness to challenge local customs/prejudices 
            \end{itemize}
        \end{itemize}
        \textbf{Several New Deal reformers sought government control over regional planning: the Tennesee Valley Authority constructed a major dam in the Tennessee Valley, bringing electricity to the region, stopped flooding throughout the region, and encouraged local industry. However, the industry remained impoverished due to the administration's unwillingness to challenge local norms.}}
        \cornell{How did the New Deal approach currency reform?}{\begin{itemize}
            \item Roosevelt felt gold standard hurt restoration of prices $\to$ executive end to gold standard
            \begin{itemize}
                \item Allowed for price experimentation w/ administration initially purchasing gold/silver, later lowering gold content of single dollar value 
                \item Currency manipulation on federal level set significant precedent but had no signif. effect
            \end{itemize}
            \item 1933: \textbf{Glass-Steagall Act} allowed govt. to prevent bank overspeculation; created wall betw. commercial/investment banking
            \begin{itemize}
                \item Established \textbf{Federal Deposit Insurance Corporation}, guaranteeing all bank deposits 
            \end{itemize}
            \item 1935: govt. transferred authority once held by regional Federal Reserve banks to Federal Reserve Board 
            \item 1933: \textbf{Truth in Securities Act of 1933} required corporations issuing securities to be transparent to public
            \item June 1934: \textbf{Securities and Exchange Commission} created to police stock market; indicated that public opinion of economy had collapsed
        \end{itemize}
        \textbf{Roosevelt first eliminated the gold standard to allow for freer currency manipulation; he also issued federal reforms to more effectively police the economy by limiting bank overspeculation, requiring corporate transparency for securities, and creating a committee to police the stock market.}}
        \cornell{How did Roosevelt expand federal relief efforts?}{\begin{itemize}
            \item \textbf{Federal Emergency Relief Administration} provided cash to states to support bankrupt relief agencies
            \begin{itemize}
                \item Led by Hopkins of NY relief agency
                \item Hopkins/Roosevelt feared too much direct relief
            \end{itemize}
            \item FERA grants insufficient $\to$ Roosevelt turned to more comfortable \textbf{Civil Works Administration} w. > 4m ppl. on temporary projects 
            \begin{itemize}
                \item Many projects, like roads/schools/parks, very important; others simply to create work
                \item Hopkins satisfied w/ process of pouring money into econ. 
            \end{itemize}
            \item \textbf{Civilian Conservation Corps} (CCC) one of earliest and Roosevelt's most liked programs: employed \underline{young men} to work in national parks/forests
            \begin{itemize}
                \item Worked in military-like environment to plant trees, build resevoirs, create parks, improve agri. irrigation
                \item Race segregation w/ mostly white men; some positions for Afr. Americans/Mexicans/natives 
            \end{itemize}
            \item \textbf{Farm Credit Administration} to refinance farm mortgages very successful; Frazier-Lemke Farm Bankruptcy Act of 1933 allowed some farmers to regain land after foreclosure 
            \begin{itemize}
                \item 25\% of farm owners lost land regardless
                \item Homeowners frequently lost mortgages
            \end{itemize}
        \end{itemize}
        \textbf{Roosevelt generally shied away from direct cash relief, creating the relatively small FERA to provide cash to states; most relief was in the form of new jobs: the CWA created temporary urban projects and the CCC, with its military-like structure, supported young men in rural projects. Farmers and homeowners alike often lost their mortgages: Roosevelt instituted some government relief but it was unable to save all farmers.}}

    \end{document}