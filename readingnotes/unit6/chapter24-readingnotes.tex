\documentclass[a4paper]{article}
    \usepackage[T1]{fontenc}
    \usepackage{tcolorbox}
    \usepackage{amsmath}
    \tcbuselibrary{skins}
    
    \usepackage{background}
    \SetBgScale{1}
    \SetBgAngle{0}
    \SetBgColor{red}
    \SetBgContents{\rule[0em]{4pt}{\textheight}}
    \SetBgHshift{-2.3cm}
    \SetBgVshift{0cm}
    \usepackage[margin=2cm]{geometry} 
    
    \makeatletter
    \def\cornell{\@ifnextchar[{\@with}{\@without}}
    \def\@with[#1]#2#3{
    \begin{tcolorbox}[enhanced,colback=gray,colframe=black,fonttitle=\large\bfseries\sffamily,sidebyside=true, nobeforeafter,before=\vfil,after=\vfil,colupper=blue,sidebyside align=top, lefthand width=.3\textwidth,
    opacityframe=0,opacityback=.3,opacitybacktitle=1, opacitytext=1,
    segmentation style={black!55,solid,opacity=0,line width=3pt},
    title=#1
    ]
    \begin{tcolorbox}[colback=red!05,colframe=red!25,sidebyside align=top,
    width=\textwidth,nobeforeafter]#2\end{tcolorbox}%
    \tcblower
    \sffamily
    \begin{tcolorbox}[colback=blue!05,colframe=blue!10,width=\textwidth,nobeforeafter]
    #3
    \end{tcolorbox}
    \end{tcolorbox}
    }
    \def\@without#1#2{
    \begin{tcolorbox}[enhanced,colback=white!15,colframe=white,fonttitle=\bfseries,sidebyside=true, nobeforeafter,before=\vfil,after=\vfil,colupper=blue,sidebyside align=top, lefthand width=.3\textwidth,
    opacityframe=0,opacityback=0,opacitybacktitle=0, opacitytext=1,
    segmentation style={black!55,solid,opacity=0,line width=3pt}
    ]
    
    \begin{tcolorbox}[colback=red!05,colframe=red!25,sidebyside align=top,
    width=\textwidth,nobeforeafter]#1\end{tcolorbox}%
    \tcblower
    \sffamily
    \begin{tcolorbox}[colback=blue!05,colframe=blue!10,width=\textwidth,nobeforeafter]
    #2
    \end{tcolorbox}
    \end{tcolorbox}
    }
    \makeatother

    \parindent=0pt
    \usepackage[normalem]{ulem}

    \newcommand{\chapternumber}{24}
    \newcommand{\chaptertitle}{The Great Depression}
    \title{\vspace{-3em}
    \begin{tcolorbox}
    \Huge\sffamily \begin{center} AP US History  \\
    \LARGE Chapter \chapternumber \, - \chaptertitle \\
    \Large Finn Frankis \end{center} 
    \end{tcolorbox}
    \vspace{-3em}
    }
    \date{}
    \author{}
    
    \begin{document}
        \maketitle
        \SetBgContents{\rule[0em]{4pt}{\textheight}}
        \cornell[Key Concepts]{What are this chapter's key concepts?}{\begin{itemize}
            \item \textbf{7.1.I.C} - Credit/market instability during the Great Depression $\to$ demand for stronger financial system 
            \item \textbf{7.1.III.A} - FDR's New Deal used govt. power to attempt to end Great Depression w/ relief to poor, economic reform
            \item \textbf{7.1.III.B} - Radical/populist/union movements $\to$ FDR pressed to make more radical changes to econ. system; conservatives in Congress attempted to limit efforts
            \item \textbf{7.1.III.C} - New Deal did not end Depression; left reform/regulatory legacy, encouraged long-term pol. realignment w/ Afr. Americans, oppressed grps. turning to Democratic Party
        \end{itemize}}
        \cornell[Launching the New Deal]{What were Roosevelt's early measures after taking office?}{\textbf{After taking office, Roosevelt initially attempted to restore confidence with his fireside chats; he also reformed the agricultural system with the Agricultural Adjustment Administration to prevent overproduction, attempted to recover industry with the National Industry Recovery Act to generate new jobs through industrial regulation and price raising. There were some attempts to restructure regional development, notably with the Tennessee Valley Association. He also took far more federal intervention in the economy by directly experimenting with the currency and policing the markets. The wide majority of relief was given in the form of new jobs for temporary projects, both rural and urban.}}
        \cornell{How did Roosevelt seek to restore confidence within the American population?}{\begin{itemize}
            \item Inaugaural address focused on public image w/ "the only thing we have to fear is fear itself," promising immediate strong action 
            \begin{itemize}
                \item Built confidence w/ frequent "fireside chats" to explain programs to ppl., chronicle progress
                \item Unwritten agreement $\to$ never photographed getting out of car/in wheelchair $\to$ few Americans knew abt. paralysis
            \end{itemize}
            \item Immediately instituted banking reform by closing all U.S. banks for four days until Congress special session
            \begin{itemize}
                \item Known as "bank holiday"; three days in, produced \textbf{Emergency Banking Act}, conservative bill somewhat inspired by Hoover intended to protect stronger banks from weakness of smaller ones
                \item Treasury Department given required inspection of all banks before reopening; federal govt. would assist failing institutions; most demolished would be reorganized
                \item Helped to limit panic w/ three-quarters of banks reopening w/in 3 days, currency soon flowing back in
            \end{itemize}
            \item Soon after Emergency Banking Act, Roosevelt sent \textbf{Economy Act} to convince fiscally conservative Americans that federal govt. remained safe/prosperous
            \begin{itemize}
                \item Cut salaries of govt. employees, reduced pensions to balance budget away from \$1b debt 
            \end{itemize}
            \item Created new constitutional amendment to repeal Prohibition, ratified in 1933
        \end{itemize}
        \textbf{Roosevelt sought to restore public confidence through a strong, hard-working public image, the closing and reopening of all banks with a "bank holiday" to give Congress time to institute reform, and the balancing of the federal government by cutting employee salaries. He also repealed the Prohibition, a very popular choice.}}
        \cornell{How did Roosevelt reform the agricultural system?}{\begin{itemize}
            \item First comprehensive program: \textbf{Agricultural Adjustment Act} in May 1933 designed to limit overproduction
            \item Created Agricultural Adjustment Administration (AAA) to pay farmers subsidies to not use entirety of land, set specific production limits on crops 
            \begin{itemize}
                \item Created new tax on food processing to find sufficient funds to bring crop prices to parity levels, pay farmers
                \item Encouraged rise in prices for farm commodities w/ gross farm income greatly increasing, agricultural economy flourishing and reaching far more stable state
                \item Favored larger farmers because local administration of programs often delegated to largest producers; created no incentive for planters not to fire their smaller tenant farmers $\to$ many lost jobs 
                \item Supreme Court shrunk scope of AAA in 1936, arguing govt. had no constitutional authority to restrict farmers; govt. soon secured new legislation to reduce production w/ outward purpose of "conserving soil"; no more court interference
            \end{itemize}
            \item Administration began efforts for poorer farmers w/ \textbf{Resettlement Administration} (1935) and successor, \textbf{Farm Security Administration} (1937) provided loans for farmers on poor soil 
            \begin{itemize}
                \item Relatively unsuccessful, moving only a few thousand
                \item \textbf{Rural Electrification Administration} (1935) far more successful in distributing power to farmers
            \end{itemize}
        \end{itemize}
        \textbf{FDR created the Agricultural Adjustment Act to limit overproduction by encouraging farmers to produce fewer crops through subsidies, though its policies generally favored larger farmers and it faced some resistance from the Supreme Court. Some relatively unsuccessful efforts to assist poorer farmers began in 1935; most successful was the process of distributing power to these regions.}}
        \cornell{How did FDR initiate programs to recover industry?}{\begin{itemize}
            \item U.S. Chamber of Commerce had been pushing for antideflation since 1931 to allow trade associations to deliberately stabilize prices w/in their own industries
            \begin{itemize}
                \item Forbidden by antitrust laws $\to$ Hoover strongly opposed; Roosevelt far more willing to compromise
                \item Roosevelt wanted business leaders to support workers by recognizing bargaining right through unions to allow incomes to increase w/ prices 
                \item Passed \textbf{National Industrial Recovery Act} to create new public works projects, generate new jobs
            \end{itemize}
            \item Initially seemed successful w/ \textbf{National Recovery Administration} (NRA) led by Hugh S. Johnson 
            \begin{itemize}
                \item Encouraged "blanket code" for \underline{all businesses}: basic minimum wage, no child labor, maximum workweek of 40 hrs. all to raise consumer purchasing power 
                \item Worked w/ leaders of major industries to set lowpoint for wages (often desired for competitive advantage); formed comprehensive agreements
                \item Faced major difficulties due to poorly written codes, inexperienced federal officials selected to administer, producers generally given control over codes $\to$ heavily favored larger industry, codes often artificially set prices to unrealistic values
                \item Unable to enforce several codes
                \begin{itemize}
                    \item Promised right to form unions in 7(a) w/o enforcement mechanisms $\to$ no employer recognition
                    \item \textbf{Public Works Administration} (PWA) had large amount of funds but spending programs required slow trickle $\to$ unable to use most of money until 1938
                \end{itemize}
                \item Visibly unsuccessful w/ industrial production declining despite rise in prices; large criticism $\to$ FDR pushed Johnson to resign, created entirely new board
            \end{itemize}
        \end{itemize}
        \textbf{FDR attempted to recover industrial growth through the NRA, intended to regulate industrial prices. Led by Hugh S. Johnson, it worked directly with the largest companies and also established "blanket codes" to regulate all business. However, due to its poor administration, heavy support of big business, and inability to enforce many of its codes, it caused industrial production to decline despite raising prices.}}
        \cornell{How did the NRA ultimately end?}{\begin{itemize}
            \item Supreme Court intervened in 1935: after Schechter brothers accused of breaking NRA codes
            \begin{itemize}
                \item Supreme Court deemed brothers \underline{not engaged in interstate commerce} $\to$ not subject to federal regulation; felt NRA to be unconstitutional $\to$ ended despite FDR's resistance (though he knew it was a failed program)
            \end{itemize}
        \end{itemize}
        \textbf{It ended after Supreme Court intervention deemed it an unconstitutional application of federal authority.}}
        \cornell{How did the New Deal plan regional development?}{\begin{itemize}
            \item AAA/NRA reflected support of economic growth in public sphere but desiring planning process for regional development to remain restricted to private sphere
            \item Some reformers sought fed. control over regional planning: \textbf{Tennessee Valley Authority} (TVA) originated after large private utility company collapsed
            \begin{itemize}
                \item Utilities companies unable to resist public discontent $\to$ TVA created to regulate Tennesee Valley, completing unfinished dam in AL on Tennessee River, generating electricity and encouraging redevelopment
                \item Stopped major flooding in TN Valley, encouraging local industry through reforestation programs, assisted farmer productivity
                \item Improved water transportation, provided electricity; region remained impoverished due to unwillingness to challenge local customs/prejudices 
            \end{itemize}
        \end{itemize}
        \textbf{Several New Deal reformers sought government control over regional planning: the Tennesee Valley Authority constructed a major dam in the Tennessee Valley, bringing electricity to the region, stopped flooding throughout the region, and encouraged local industry. However, the industry remained impoverished due to the administration's unwillingness to challenge local norms.}}
        \cornell{How did the New Deal approach currency reform?}{\begin{itemize}
            \item Roosevelt felt gold standard hurt restoration of prices $\to$ executive end to gold standard
            \begin{itemize}
                \item Allowed for price experimentation w/ administration initially purchasing gold/silver, later lowering gold content of single dollar value 
                \item Currency manipulation on federal level set significant precedent but had no signif. effect
            \end{itemize}
            \item 1933: \textbf{Glass-Steagall Act} allowed govt. to prevent bank overspeculation; created wall betw. commercial/investment banking
            \begin{itemize}
                \item Established \textbf{Federal Deposit Insurance Corporation}, guaranteeing all bank deposits 
            \end{itemize}
            \item 1935: govt. transferred authority once held by regional Federal Reserve banks to Federal Reserve Board 
            \item 1933: \textbf{Truth in Securities Act of 1933} required corporations issuing securities to be transparent to public
            \item June 1934: \textbf{Securities and Exchange Commission} created to police stock market; indicated that public opinion of economy had collapsed
        \end{itemize}
        \textbf{Roosevelt first eliminated the gold standard to allow for freer currency manipulation; he also issued federal reforms to more effectively police the economy by limiting bank overspeculation, requiring corporate transparency for securities, and creating a committee to police the stock market.}}
        \cornell{How did Roosevelt expand federal relief efforts?}{\begin{itemize}
            \item \textbf{Federal Emergency Relief Administration} provided cash to states to support bankrupt relief agencies
            \begin{itemize}
                \item Led by Hopkins of NY relief agency
                \item Hopkins/Roosevelt feared too much direct relief
            \end{itemize}
            \item FERA grants insufficient $\to$ Roosevelt turned to more comfortable \textbf{Civil Works Administration} w. > 4m ppl. on temporary projects 
            \begin{itemize}
                \item Many projects, like roads/schools/parks, very important; others simply to create work
                \item Hopkins satisfied w/ process of pouring money into econ. 
            \end{itemize}
            \item \textbf{Civilian Conservation Corps} (CCC) one of earliest and Roosevelt's most liked programs: employed \underline{young men} to work in national parks/forests
            \begin{itemize}
                \item Worked in military-like environment to plant trees, build resevoirs, create parks, improve agri. irrigation
                \item Race segregation w/ mostly white men; some positions for Afr. Americans/Mexicans/natives 
            \end{itemize}
            \item \textbf{Farm Credit Administration} to refinance farm mortgages very successful; Frazier-Lemke Farm Bankruptcy Act of 1933 allowed some farmers to regain land after foreclosure 
            \begin{itemize}
                \item 25\% of farm owners lost land regardless
                \item Homeowners frequently lost mortgages
            \end{itemize}
        \end{itemize}
        \textbf{Roosevelt generally shied away from direct cash relief, creating the relatively small FERA to provide cash to states; most relief was in the form of new jobs: the CWA created temporary urban projects and the CCC, with its military-like structure, supported young men in rural projects. Farmers and homeowners alike often lost their mortgages: Roosevelt instituted some government relief but it was unable to save all farmers.}}
        \cornell[The New Deal in Transition]{How did the Second New Deal come to be?}{\textbf{Roosevelt was forced to shift his New Deal policies after great criticism emerged from those seeking heavier currency reforms, action against big business, wealth distribution, and support for the elderly. He began to attack big business and further support workers, with the federal government rarely intervening on the behalf of employers and supporting sit-down strikes in the automobile industry. He also created an expansive social security program to ensure that the elderly were supported.}}
        \cornell{How did several begin to criticize the New Deal?}{\begin{itemize}
            \item American conservatives generally opposed New Deal: Du Pont family transformed American Liberty League (init. to oppose prohibition) into anti-New Deal policies
            \begin{itemize}
                \item Felt New Deal directly attacked free enterprise
                \item Unsuccessful beyond northern industry
            \end{itemize}
            \item Far left had similarly limited strength w/ occasional Communist/Socialist criticism
            \item Broad dissident movements most successful in opposition 
            \begin{itemize}
                \item \textbf{Francis E. Townsend} pushed for Townsend Plan to provide pensions to elderly after age of 60 if retired, willing to spend money in full to give more jobs to young, boost economy; set precedent for Social Security
                \item \textbf{Charles E. Coughlin}, Catholic priest, created weekly radio sermons; pushed for banking/currency reforms
                \begin{itemize}
                    \item Sought greenbacks, remonetization of silver, nationalization of banking
                    \item Initially supported Roosevelt but soon felt unable to deal harshly enough w/ corporations $\to$ created National Union for Social Justice
                \end{itemize}
                \item \textbf{Huey Long} known for national popularity, constant attacks on banks/oil companies/utilities/conservatives 
                \begin{itemize}
                    \item Rose to prominence as LA governor; assaulted opponents to remove political power while remaining in popular support due to progressive accomplishments of construction, tax codes, free goods, lowered utility
                    \item Easily won U.S. Senate seat in 1930
                    \item Initially supported Roosevelt but soon created \textbf{Share-Our-Wealth Plan} to easily end Depression by taking wealth from richest, distributing to poor, guaranteeing eevry family annual wage of \$2.5k 
                    \item Large national following $\to$ Democrats determined could pose risk to next election if independent candidate
                \end{itemize}
            \end{itemize}
        \end{itemize}
        \textbf{American conservatives and the far left both struggled to generate significant opposition to the New Deal. More significant on the national level were three powerful men: Francis E. Townsend, advocating for pension reform; Charles E. coughlin, advocating for banking and currency reforms; and the very successful Huey Long, attacking large corporations and seeking to distribute the wealth of the richest among the poor.}}
        \cornell{How did Roosevelt's "Second New Deal" represent a response to his critics?}{\begin{itemize}
            \item "Second New Deal" represented shift most notably in attitude to big business
            \begin{itemize}
                \item Roosevelt openly attacked largest businesses
                \item 1935: passed \textbf{Holding Company Act} to break up large utilities companies; harsh lobbying $\to$ effect limited
            \end{itemize}
            \item Tax reforms somewhat aligned w/ Long's Share-Our-Wealth Plan: created large tax rates on rich (little effect)
            \item After Supreme Court struck down National Industrial Recovery Act, section guaranteeing worker right to unionize also invalidated $\to$ Robert F. Wagner responded
            \begin{itemize}
                \item Wagner Act created \textbf{National Labor Relations Board} with power to compel employers to allow unions
                \item Roosevelt reluctantly signed w/ recognition that workers had become powerful force
            \end{itemize}
        \end{itemize}}
        \cornell{How did labor unions become a powerful force?}{\begin{itemize}
            \item 1934: labor movement showed new power; Wagner Act gave more legal power $\to$ even stronger
            \item AFL continually supported concept of craft union; left unskilled workers out $\to$ 1930s concept of industrial unionism
            \begin{itemize}
                \item Argued that all workers in an industry should form only one union to be united
                \item Industrial unionism leader John L. Lewis (from United Mine Workers) tried to work directly w/ AFL
                \begin{itemize}
                    \item AFL continually opposed, many arguments/fights $\to$ Lewis walked out, forming Congress of Industrial Organizations (CIO) to rival AFL 
                \end{itemize}
            \end{itemize}
            \item CIO open to women/blacks (partly due to larger proportion in unskilled jobs); far more militent
        \end{itemize}
        \textbf{Led by John L. Lewis, the concept of industrial unionism emerged, which believed that all workers in a given industry should form a union. It directly opposed the concepts of the AFL, leading to a direct schism; the Congress of Industrial Organizations was soon formed.}}
        \cornell{How did several battles between employers and workers unfold?}{\begin{itemize}
            \item \textbf{United Auto Workers} emerged as dominant in automobile industry; still struggled
            \begin{itemize}
                \item Made signif. progress w/ sit-down strike (unwilling to work or leave $\to$ no strikebreakers) starting at GM plant in Detroit; rapidly spread
                \item Men stayed in factories; women lobbied directly w/ officials, supported men 
                \item MI governor (liberal Democrat) refused to call National Guard, federal govt. would not intervene $\to$ GM relented, recognizing UAW 
                \item Sit-down strike worked in rubber industries and others; only worked for short time due to illegality
            \end{itemize}
            \item Steel industry more challenging: \textbf{Steel Workers' Organizing Committee} launched powerful drive $\to$ U.S. Steel recognized in 1937; smaller companies ("Little Steel") less willing
            \begin{itemize}
                \item Republic Steel workers gathered on Memorial Day 1937 for picnic/demonstration; marched legally toward steel plant $\to$ police opened fire, killing 10 demonstrators
                \item Public outcry ignored; 1937 strike failed
                \item Memorial Day Massacre last of brutal strikebreaking; 1937 saw 4.7k strikes, mostly favoring unions; union membership swelled and even Little Steel began to turn
            \end{itemize}
        \end{itemize}
        \textbf{The United Auto Workers won rapid recognition by automobile companies after a sitdown strike in Detroit which the government refused to end. U.S. Steel, the largest steel company, recognized the Steel Workers' Organizing Committee relatively soon; smaller steel companies were far more unrelenting, even allowing a police massacre to occur against a public strike with little effect.}}
        \cornell{How did Roosevelt initiate a program of social security?}{\begin{itemize}
            \item 1935: Roosevelt supported Social Security Act w/ several programs
            \item Elderly currently destitute would receive \$15 each month; many Americans entered pension system w/ employers creating payroll tax to receive income after retirement
            \begin{itemize}
                \item Pension payments delayed until 1942; broad categories of work remained excluded (domestic servants/agricultural laborers)
                \item Act very imporatnt step to create support system for elderly
            \end{itemize}
            \item Social Security Act created unemployment insurance to allow temporary relief to workers; created federal aid for disablities/dependent children
            \item Social security aimed for "insurance," \underline{not "welfare"}; directly assisted those unable to support themselves w/ support expanding over time
        \end{itemize}
        \textbf{Roosevelt's social security program assisted elderly Americans by providing the most destitute with a monthly income but also developing a pension system where a portion of all employed Americans' paychecks would go toward a pension to support them in retirement. He also created unemployment insurance and assisted those with disabilities.}}
        \cornell{How did Roosevelt address Americans with more immediate needs?}{\begin{itemize}
            \item Primary source of aid for most Americans: Works Progress Administration (WPA), succeeding the Civil Works Administration, led by Harry Hopkins
            \begin{itemize}
                \item Far larger than previous agencies w/ budget, energy of operators
                \item Constructed airports, roads, bridges w/ 2.1m workers employed $\to$ new money for economy
                \item WPA etremely flexible in assistance: writers given opportunities to work for govt. in Federal Arts Project, artists/actors/musicians assisted by Federal Arts/Music/Theater Projects 
                \item National Youth Administration created alongside WPA to assist high-school/college-age ppl.
            \end{itemize}
            \item New organizations created work/unemployment insurance for men; women given cash assistance in Aid to Dependent Children program for single mothers
            \begin{itemize}
                \item Reflected incorrect belief that only men were in the workforce
            \end{itemize}
        \end{itemize}
        \textbf{Roosevelt addressed the most immediate needs of Americans through the Works Progress Administration led by Harry Hopkins, with a large budget and a far wider range of jobs for those of several skills. The National Youth Administration supported student-age citizens. These organizations gave work, pensions, and insurance to men and cash to women.}}
        \cornell{What was the result of the 1936 election?}{\begin{itemize}
            \item Economic growth/revival in 1936 $\to$ president had solid chance; bolstered by destruction of opposition w/ Huey Long assassinated, Coughlin/Townsend/Smith (Long's henchman) forming failed Union Party 
            \item Republican party nominated Alf M. Landon; Roosevelt got 61\% of pop. vote, Democrats further increased majorities in both houses; Union Party fewer than 900k votes 
            \item Election revealed that Democrats controlled coalition of west/south farmers, urban workers, poor/unemployed, northern blacks, traditional progressives 
            \begin{itemize}
                \item Republicans unable to create coalition of similar size for many years 
            \end{itemize}
        \end{itemize}
        \textbf{Roosevelt carried the 1936 election due to the significant economic growth which preceded it; nearly all opposition had collapsed. The Democrats further carried both houses.}}
        \cornell[The New Deal in Disarray]{What caused the New Deal to encounter many difficulties?}{\textbf{The Supreme Court had long been one of Roosevelt's biggest obstacles; he attempted to pack the court with new justices but lacked justification after the Court made growingly moderate decisions. The New Deal faced great opposition after Roosevelt, seeing an economic boom, cut government spending to prevent inflation, ultimately creating another crash nearing 1932-33 levels. Furthermore, impending war forced Roosevelt to reevaluate his priorities.}}
        \cornell{How did Roosevelt attack the Supreme Court structure?}{\begin{itemize}
            \item 1936 win $\to$ Roosevelt felt powerful against force of Supreme Court which he saw as great obstacle (struck down NRA/AAA)
            \begin{itemize}
                \item Feb. 1937: sent message to Capitol Hill proposing overhaul of federal court system by adding up to six new justices, one for each justice over seventy (felt they were overworked $\to$ fresh ideas required)
                \item Wanted power to add liberal justices to change ideological balance
                \item Conservatives greatly angered by "court-packing"; even supporters felt reflected hunger for power
            \end{itemize}
            \item Court seen as more moderate w/ three consistent New-Dealers, four consistent opposers, Chief Justice Charles Hughes often supporting and Associate Justice Owen Roberts often opposing
            \begin{itemize}
                \item Upheld state minimum wage law, Wagner Act, Social Security Act $\to$ court packing seen as unnecessary w/ bill not making it through Congress
            \end{itemize}
        \end{itemize}
        \textbf{Roosevelt sought more power within the courts, attempting to add up to six new justices by propounding his belief that courts were "overworked"; however, they began to make more moderate decisions regarding the New Deal than before (previously having struck down the NRA and the AAA), making court-packing seem relatively unnecessary.}}
        \cornell{How did a temporary economic boom soon end?}{\begin{itemize}
            \item GDP had risen close to 1929 levels $\to$ Roosevelt used as excuse to balance federal budget w/ much of administration fearing inflation
            \begin{itemize}
                \item Cut WPA in half w/ layoffs for 1.5m relief workers $\to$ boom soon collapsed
                \item Industrial production nearly halved; 4m workers lost jobs; conditions neared 1932-1933 levels
            \end{itemize}
            \item Critics saw as "Roosevelt recession"; many, including Roosevelt, saw as due to decreased spending $\to$ asked for \$5b from Congress to fund public works $\to$ recovery seemed to begin
            \item Many felt more long-term issue was concentrated econ. strength $\to$ Roosevelt sent letter to Congress asking for antitrust reform $\to$ temporary committee for reform
            \begin{itemize}
                \item \textbf{Temporary National Economic Committee} included representatives from both houses of Congress, several agencies
                \item Antitrust division of Justice Department led by Thurman Arnold of Yale Law School
                \item 1938: passed successful \textbf{Labor Standards Act} for forty-hour workweek, limited child labor; excluded women/minorities
            \end{itemize}
            \item New Deal came to an end due to Congressional opposition, growing global crisis
        \end{itemize}}
        
    \end{document}