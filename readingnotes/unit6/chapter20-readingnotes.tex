\documentclass[a4paper]{article}
    \usepackage[T1]{fontenc}
    \usepackage{tcolorbox}
    \usepackage{amsmath}
    \tcbuselibrary{skins}
    
    \usepackage{background}
    \SetBgScale{1}
    \SetBgAngle{0}
    \SetBgColor{red}
    \SetBgContents{\rule[0em]{4pt}{\textheight}}
    \SetBgHshift{-2.3cm}
    \SetBgVshift{0cm}
    \usepackage[margin=2cm]{geometry} 
    
    \makeatletter
    \def\cornell{\@ifnextchar[{\@with}{\@without}}
    \def\@with[#1]#2#3{
    \begin{tcolorbox}[enhanced,colback=gray,colframe=black,fonttitle=\large\bfseries\sffamily,sidebyside=true, nobeforeafter,before=\vfil,after=\vfil,colupper=blue,sidebyside align=top, lefthand width=.3\textwidth,
    opacityframe=0,opacityback=.3,opacitybacktitle=1, opacitytext=1,
    segmentation style={black!55,solid,opacity=0,line width=3pt},
    title=#1
    ]
    \begin{tcolorbox}[colback=red!05,colframe=red!25,sidebyside align=top,
    width=\textwidth,nobeforeafter]#2\end{tcolorbox}%
    \tcblower
    \sffamily
    \begin{tcolorbox}[colback=blue!05,colframe=blue!10,width=\textwidth,nobeforeafter]
    #3
    \end{tcolorbox}
    \end{tcolorbox}
    }
    \def\@without#1#2{
    \begin{tcolorbox}[enhanced,colback=white!15,colframe=white,fonttitle=\bfseries,sidebyside=true, nobeforeafter,before=\vfil,after=\vfil,colupper=blue,sidebyside align=top, lefthand width=.3\textwidth,
    opacityframe=0,opacityback=0,opacitybacktitle=0, opacitytext=1,
    segmentation style={black!55,solid,opacity=0,line width=3pt}
    ]
    
    \begin{tcolorbox}[colback=red!05,colframe=red!25,sidebyside align=top,
    width=\textwidth,nobeforeafter]#1\end{tcolorbox}%
    \tcblower
    \sffamily
    \begin{tcolorbox}[colback=blue!05,colframe=blue!10,width=\textwidth,nobeforeafter]
    #2
    \end{tcolorbox}
    \end{tcolorbox}
    }
    \makeatother

    \parindent=0pt
    \usepackage[normalem]{ulem}

    \newcommand{\chapternumber}{20}
    \newcommand{\chaptertitle}{The Progressives}
    \title{\vspace{-3em}
    \begin{tcolorbox}
    \Huge\sffamily \begin{center} AP US History  \\
    \LARGE Chapter \chapternumber - \chaptertitle \\
    \Large Finn Frankis \end{center} 
    \end{tcolorbox}
    \vspace{-3em}
    }
    \date{}
    \author{}
    
    \begin{document}
        \maketitle
        \SetBgContents{\rule[0em]{4pt}{\textheight}}
        \cornell[Key Concepts]{What are this chapter's key concepts?}{\begin{itemize}
            \item \textbf{6.3.I.C} - Artists/critics (notably agrarians, utopians, socialists, Social Gospel advocates) sought change for U.S. socio-economic state
            \item \textbf{6.3.II.B} - Women sought $\uparrow$ equality w/ men $\to$ joined orgs., attended college, promoted reform, worked in settlement houses w/ immigrants to help adaptation
            \item \textbf{7.1.II.A} - Some journalists attacked corruption, social injustice, econ. inequality; reformers hoped to find reform among immigrants 
            \item \textbf{7.1.II.B} - On national level, sought legislation to regulate econ., expand democracy, generate moral reform $\to$ taclked prohibition/suffrage
            \item \textbf{7.1.II.C} - Conservationists supported nat. parks, sought govt. responses to overuse of natural resources 
            \item \textbf{7.1.II.D} - Progressive movement never uniform: some supported segregation, others ignored; some sought wide govt. participation, others sought only experts; disagreed on immigration 
        \end{itemize}}
        \cornell[The Progressive Impulse]{What moral background underlaid the Progressive movement?}{\textbf{The Progressives pushed for constant progress in society, social cohesion, and widespread knowledge at the expense of concentrated power. They fought for social justice, which generally meant supporting the urban poor through direct aid or the creation of settlement houses to provide a safe place for impoverished immigrants. Furthermore, expertise was critical: law, medicine, business, and farming became increasingly restricted and professionalized; women fit into this society primarily as teachers and nurses.}}
        \cornell{How did the Progressives advocate progress?}{\textbf{The Progressives believed in constant social progress and growth; however, celebrated capitalist ideals of \textit{laissez-faire} and "natural laws" of the marketplace were insufficient: direct intervention was critical.}}
        \cornell{What were the various forms of Progressivism?}{\begin{itemize}
            \item \textbf{Antimonopoly} spirit feared concentrated power, sought wealth/authority dispersal w/ govt. regulation of trusts $\to$ widespread interest from farmers
            \item \textbf{Social cohesion} movement analyzed web of complex social relationships, saw well-being as closely tied to greater well-being of society
            \item Faith in knowledge saw knowledge as great equalizer $\to$ modernized govt. applying principles of natural/social sciences would succeed in regulation
        \end{itemize}
        \textbf{The antimonopolists greatly feared the concentration of power, seeking government dispersal of power. The social cohesion movement saw the individual welfare as directly linked to social welfare. Finally, several believed in the importance of knowledge and the scientific method to modernize society.}}
        \cornell{Who were the muckrakers?}{\begin{itemize}
            \item \textbf{Muckrakers} first to bring Progressive issues to social attention by publicizing corruption
            \begin{itemize}
                \item Name arose after Roosevelt accused of "raking up muck" through his writings
            \end{itemize}
            \item Targeted corrupt trusts/railroads, starting w/ \textbf{Charles Francis Adams Jr.} against railroad barons, \textbf{Ida Tarbell} against Stanadrd Oil
            \item Began to turn attention to govt., notably political machines; \textbf{Lincoln Steffens} wrote \textit{The Shame of the Cities} portraying "machine govt." and "boss rule"
            \begin{itemize}
                \item Called people to bring direct intervention into public life
            \end{itemize}
            \item Reached peak in first dedcade of 20th century
        \end{itemize}
        \textbf{The muckrakers sought to create widespread attention for Progressive social issues by exposing corruption in trusts, railroads, and urban political machines.}}
        \cornell{How did several begin to fight for social justice?}{\begin{itemize}
            \item Many reformers dedicated to "social justice," or goal to benefit all of society through egalitarianism and support for poor $\to$ \textbf{Social Gospel} within Protestantism to reform cities
            \item Salvation Army blended religion w/ reform: structure somewhat resembled military, provided relief for urban poor
            \item Charles Sheldon's \textit{In His Steps} described story of young minister abandoning work to help poor
            \item \textbf{Walter Rauschenbusch}, Protestant theologian, believed Darwinism described evolution of society as a whole, requiring effort of all
            \item Some Catholics followed Pope Leo XIII's \textit{Rerum Novarum}, w/ liberal \textbf{Father John Ryan} likening conditions of poor to slavery
        \end{itemize}
        \textbf{The Social Gospel movement was dedicated to assisting the urban poor; the Salvation Army blended religion with relief for the urban poor, and Catholics and Protestants alike found religious justifications for assisting the less fortunate.}}
        \cornell{What characterized the growth of settlement houses?}{\begin{itemize}
            \item Progressivists felt poverty was due to unfortunate environment $\to$ $\uparrow$ living conditions required
            \item Crowded immigrant neighborhoods greatest cause of distress $\to$ English idea of settlement house
            \begin{itemize}
                \item \textbf{Hull House} in Chicago (1889) after work of Jane Addams became universal model for ones through the nation
                \item Workers were educated middle class; encouraged assimilation without condescension/shunning of past beliefs
                \item Felt middle class were responsible for passing on knowledge to lower classes
            \end{itemize}
            \item Young women played major role: aligned w/ society's belief of requiring women to be consistently sheltered in clean buildings
            \item Social work became critical profession aligned w/ study of sociology $\to$ universities began to take far more seriously 
        \end{itemize}
        \textbf{Settlement houses emerged due to the anti-Social Darwinist belief that poverty was due to a poor environment. They provided a safe, clean place for immigrants to live and receive guidance from the educated middle class. They provided great opportunities for young women and created a new profession: social work.}}
        \cornell{How did the progressives place an emphasis on expertise?}{\textbf{The Progressives believed that all problems, whether scientific or not, could be approached analytically; some sought a new society where scientists and engineers guided all others.}}
        \cornell{What main professions did the Progressives advocate?}{\begin{itemize}
            \item Factories $\to$ $\uparrow$ admin. tasks like managers, technicians, accountants; cities $\to$ $\uparrow$ commercial/medical/legal/education; new technology $\to$ teachers
            \item Middle class placed individualistic focus on accomplishment $\to$ all worked to secure social position
            \begin{itemize}
                \item With limited training, professionalism very rare: anyone could claim themselves a lawyer/doctor w/ limited training
            \end{itemize}
            \item Medical first to professionalize (1901): American Medical Association $\to$ national professional society w/ specific standards $\to$ states began to pass licensing laws 
            \begin{itemize}
                \item Several medical schools began to parallel those in Europe
            \end{itemize}
            \item By 1916, all states had professionalized law w/ expansion of law schoools 
            \item Businessmen formed National Association of Manufacturers, U.S. Chamber of Commerce
            \item Long-famed individualistic farmers formed Farm Bureau Federation to spread scientific methods
            \item Requirements protected those already professional while eliminating fakes; sometimes created strict requirements to exclude blacks/women/immigrants or to keep numbers down for high demand
        \end{itemize}
        \textbf{The Progressives, placing a focus on individual accomplishment, strongly pushed the professionalization of professions like medicine, law, business, and even farming to ensure a strict standard for workers. This professionalization often excluded more than necessary.}}
        \cornell{How did women fit into the new professional system?}{\begin{itemize}
            \item Restrictions/prejudice $\to$ women generally excluded; several middle class women worked to earn education, enter careers
            \item Some women became physicians/lawyers/engineers/scientists/managers w/ 5\% of physicians women due to several medical schools admitting women
            \item Most turned to "helping" domestic professions pushed for by society, like social work, \underline{teaching}
            \begin{itemize}
                \item 90\% of professional women were teachers
                \item Educated black women found jobs in segregated schools of South
            \end{itemize}
            \item During/post-Civil War, women dominated nursing w/ professionalization $\to$ several earned advanced degrees 
        \end{itemize}
        \textbf{legal restrictions and social prejudice meant that women were generally excluded from the new professional system: only a small proportion were able to become physicians or other professional jobs. Most turned to "helping" professions, with teaching by far the most popular, followed by nursing.}}
        \cornell[Women and Reform]{How did women fight for reform?}{\textbf{Women's rights were severely limited at the onset of the 20th century; however, a reduced focus on the home allowed several women to enter the public sphere and fight for women's rights. Women's clubs stimulated reform on a legislative level for female and child labor, alcohol, pensions for widowed mothers, and more. The greatest accomplishment in women's rights of the early 20th century came in the Nineteenth Amendment, where women received suffrage after centuries of pushing for it.}}
        \cornell{What was the state of women's rights at the turn of the century?}{\textbf{Generally, women could not vote, hold public office, or take on professional jobs apart from teaching and nursing. Society continually pushed them out of the public world.}}
        \cornell{How did the "new woman" emerge?}{\begin{itemize}
            \item Socio-economic changes $\to$ private world reformed to same degree as public one w/ income out of home, children entering school at younger ages and spending more time there, innovations $\to$ less housework
            \item $\downarrow$ family size, longevity $\to$ women spent fewer years w/ younger children, lived longer after children had grown
            \item Some educated women renounced marriage: remaining single was essential for public role
            \begin{itemize}
                \item Single women generally most prominent female reformers (like Jane Addams in settlement houses, Frances Willard in temperance, Anna Howard Shaw in suffrage)
                \item Some lived alone, others w/ other women $\to$ occasional romantic relations
            \end{itemize}
        \end{itemize}
        \textbf{Socio-economic changes greatly impacted the private world, ultimately placing far less emphasis on the home as an economic unit and requiring women to spend fewer years nurturing children. As a result, several dedicated their lives to reform, with many of the greatest reformers never marrying.}}
        \cornell{How did women's clubs stimulate the women's reform movement?}{\begin{itemize}
            \item Women's clubs began as intellectual orgs. for middle/upper class women 
            \item By turn of century, clubs focused more on social change
            \begin{itemize}
                \item Full of wealthy women $\to$ significant money dedicated to change
                \item Women could not vote $\to$ nonpartisan image $\to$ politicians forced to recognize
            \end{itemize}
            \item Most clubs excluded blacks $\to$ \textbf{National Association of Colored Women} often fusing issues for Afr. Americans (like lynching/segregation) w/ those for women 
            \item Club movement rarely made controversial changes to challenge social assumptions abt. women: instead sought for public place for women w/in traditional gender roles
            \item Some uncontroversial construction projects; signif. change to legislation overall
            \begin{itemize}
                \item Woman + child labor reduced, food/drug industry regulated, tribal policies reformed, \underline{alcohol outlawed}
                \item Created "mother's pensions" on state level to assist widows w/ small children; became part of Social Security
                \item Children's Bureau created as part of Labor Department to protect children
            \end{itemize}
            \item Often allied w/ other women's groups like Women's Trade Union League to convince women to unionize, strike 
        \end{itemize}
        \textbf{Women's clubs quickly shifted from a place of cultural discussion to one of social reform: significant wealth allowed for great women's reform through legislation changes, including for female and child labor, alcohol, tribes, and unionization.}}
        \cornell{How did women fight for the right to vote?}{\begin{itemize}
            \item Prevailing view that suffrage was radical demand due to early supporters
            \begin{itemize}
                \item Presented as "natural right" to be equal w/ men rather than in separate spheres $\to$ male-dominated antisuffrage movement 
                \item Antisuffrage movement associated w/ divorce, promiscuity, immorality, neglect of children (some women on board)
            \end{itemize}
            \item Suffrage movement slowly overcame opposition in early 20th c. due to $\uparrow$ org. under \textbf{Anna Howard Shaw}, \textbf{Carrie Chapman Catt}, far less controversial justifications
            \begin{itemize}
                \item Argued would not challenge "separate sphere" ideology; bc. such a different sphere, women would provide new insights
                \item Tied directly into temperance movement bc. women were strongest advocates
                \item Argued war would end after women given more power bc. would provide calming factor $\to$ WWI gave final push
                \item Others felt that if black/immigrants could vote, women deserved it too 
            \end{itemize}
            \item In 1910, WA granted right to vote; CA soon followed; IL became first state east of MS River in 1913
            \item Culminated in 1920 passing of Nineteenth Amendment to guarantee pol. rights to women throughout nation
            \item \textbf{Alice Paul}, part of National Woman's Party, hated "seperate sphere" justification and sought legislation giving \underline{complete equality}; received little support
        \end{itemize}
        \textbf{The suffrage movement started off slow due to the early justification that it would bring a complete transformation to society; however, greater organization and far more careful justifications, such as the preservation of the "separate spheres" construct and the practical benefits of women having a voice, ultimately allowed the 1920 Nineteenth Amendment to guarantee political rights to women throughout the nation.}}
        \cornell[The Assault on the Parties]{How did progressives attack the seemingly corrupt American government?}{\textbf{Progressives pushed for the secret ballot and encouraged voter turnout in municipal elections to ultimately reform city governments away from political machines toward those with far less partisan goals. Furthermore, they influenced state legislatures by reducing the influence of corporations; they bypassed state governments through direct primaries and recall. Ultimately, the push for reduced partisanism meant that the voter turnout in federal elections was severely reduced due to the decline of boss rule and the formation of interest groups.}}
        \cornell{What characterized the early attacks on the federal government?}{\begin{itemize}
            \item Greenbackism and Populism were earliest successes of breaking Democrat/Republican hold; Independent Republicans made significant attempt
            \item Secret ballot adopted in 1880s/1890s replacing method of parties distributing filled-in tickets to supporters $\to$ bosses unable to monitor voting, ticket could be "split" between parties
        \end{itemize}
        \textbf{Greenbackism, Populism, and Independent Republicanism were the earliest assaults on the political parties. The greatest success against partisan dominance was the adoption of the secret ballot, preventing bosses from monitoring voting and allowing far greater individual agency in voting.}}
        \cornell{How did the progressives make significant reforms on the municipal level?}{\begin{itemize}
            \item Muckrakers encouraged "respectable" citizens who stayed away from voting due to "vulgar," corrupt bosses to take greater role $\to$ middle class began to take on $\uparrow$ pol. role 
            \item Activists faced city bosses, saloon/brothel owners, businessmen relying on profits and urban immigrants relying on relief from pol. machines 
        \end{itemize}
        \textbf{The muckrakers removed the stigma surrounding voting as a "vulgar" activity rampant with corruption. Activists who began to take a greater role in voting faced city bosses as well as all who relied on political machines.}}
        \cornell{What new forms of governance emerged to better rule cities?}{\begin{itemize}
            \item First govt. success in Galveston, TX w/ new city charter after govt. unable to deal w/ 1900 tidal wave 
            \begin{itemize}
                \item Mayor/council elected by nonpartisan commission (same in Des Moines)
            \end{itemize}
            \item \textbf{City-manager plan} entailed bringing in outside expert to take charge w/o corruption of politics
            \item Most reformers settled for more modest goals 
            \begin{itemize}
                \item Some city elections became nonpartisan; several others remained partisan but never ran during same years as presidential/congressional elections
                \item Attempted to force city councillors to run at-large to prevent boss rule; tried to strengthen power of mayor against council 
            \end{itemize}
            \item \textbf{Tom Johnson}, mayor of Cleveland, fought long battle in city to $\downarrow$ streetcar fare to 3 cents; successor held up reputation of Cleveland as best-governed city 
        \end{itemize}
        \textbf{Several cities reformed their systems of government, whether by creating a new charter, adopting the city-manager plan where a non-partisan expert would take control, or by changing city elections to be nonpartisan or never occur during federal elections. Tom Johnson, the powerful mayor of Cleveland, epitomized the urban political reform movement.}}
        \cornell{How did progressives influence state governments?}{\begin{itemize}
            \item Progressives felt members of state legislatures generally incompetent, underpaid, dominated by party bosses 
            \item Two most significant changes came from Populists: sought legislation changes to go directly to ppl., sent legislation changes to electorate for approval
            \begin{itemize}
                \item \textbf{Direct primary} sought to limit party power, create more competent elected officials; also used to limit black power in South
                \item Recall $\to$ public officials could be removed at special election if petition signed by citizens; adopted by few states (notably CA)
            \end{itemize}
            \item Other reform measures directly cleaned legislatures by preventing lobbying by businesses, corporate endorsements of campaigns, public officials freom receiving free passes from railroads (in many states)
            \begin{itemize}
                \item Increased workers' compensation, created pensions for widows
            \end{itemize}
            \item Reform most successful in states w/ dedicated politicians
            \begin{itemize}
                \item NY: governor \textbf{Charles Evans Hughes} created commission to regulate public utilities
                \item CA: \textbf{Hiram Johnson} limited Southern Pacific Railroad
                \item NJ: \textbf{Woodrow Wilson} reduced power of trusts
                \item WI: \textbf{Robert M. La Folette} tured state into "laboratory of progressivism" by implementing direct primaries, initiatives, referendums; limited railroads/utilities; brought worker compensations; raised taxes; widened public awareness through personal magnetism
            \end{itemize}
        \end{itemize}
        \textbf{Progressives targeted state legislatures through the direct primary, allowing people to vote directly on legislation changes, as well as the recall, allowing incompetent individuals to be removed from office at a special election. Others prevented the influence of big businesses on the legislature directly. Reform was most successful in states with dedicated politicians, notably the Progressive Robert M. La Folette, known for personal magnetism.}}
        \cornell{How did parties decline in overall influence?}{\begin{itemize}
            \item Weakening of parties $\to$ $\downarrow$ voter turnout, having already reached peak 
            \item Secret ballot $\to$ bosses unable to influence vote through force, illiterate voters unable to read ballot
            \item \textbf{Interest groups} replaced party bosses by unifying people by profession, trade, culture
        \end{itemize}
        \textbf{The decline of parties was seen most clearly in the decline in voter turnout due to the secret ballot reducing the influence of bosses and interest groups unifying people under other fronts.}}
        \cornell[Sources of Progressive Reform]{What were the demographics of most Progressive reformers?}{\textbf{Unions and machines worked for progressive reform, most notably the changing Tammany Hall. Western progressives targeted the federal government due to its strong influence over western territories. African Americans, under W.E.B. Du Bois, sought social change for blacks through federal laws.}}
        \cornell{How did workers and machines seek progressive reform?}{\begin{itemize}
            \item Some unions (not AFL) critical to reform: Union Labor Party $\to$ CA child-labor, workers comp., reduced female hours laws 
            \item NYC's infamous Tammany Hall (city machine) fused desires w/ social reformers: pushed for working conditions/child labor legislation
            \begin{itemize}
                \item Fire at Triangle Shirtwaist Company in NYC, killing 146; state commission pushed by workers (and somewhat by Tammany) issued technical report calling for reforms
                \item Recommendations of state commission most heavily supported by working-class Tammany Democrats in state legislature
            \end{itemize}
        \end{itemize}
        \textbf{Unions were significant to pushing new legislature. Surprisingly, several political machines adapted their intentions to align with the goals of reformers in order to preserve their influence. Tammany Hall, most notably, pushed for child labor and working conditions legislation and called for major political reforms after a deadly fire.}}
        \cornell{How did some Westerners embody the progressive movement?}{\begin{itemize}
            \item West produced some of greatest progressive leaders (CA's Hiram Johnson, NE's George Norris, ID's William Borah)
            \item Greatest focus on federal change due to disproportionate authority relative to that in East
            \begin{itemize}
                \item Disputes over waterways typically traversed multiple states (like CO River)
                \item Western states relied on fed. govt. for land grants, railroad/waterproject subsidies
                \item Large portions of unsettled land controlled by the govt. (far more than in the East)
            \end{itemize}
        \end{itemize}
        \textbf{The West produced some of the most powerful progressive leaders, who generally focused on federal change due to the massive influence of the federal government over waterways and land grants.}}
        \cornell{How did African Americans fit into the progressive movement?}{\begin{itemize}
            \item Afr. Americans faced greatest struggle for reform $\to$ \textbf{Booker T. Washington} pushed for short-term self-improvement rather than significant effort on long-term change
            \item \textbf{W.E.B. Du Bois} championed new movement for long-term change
            \begin{itemize}
                \item Never known slavery (unlike Booker T. Washington); educated at Fisk University, Harvard $\to$ far broader view of racial issues for blacks
                \item Directly attacked Washington: felt no progress would ever be made if blacks continually servile w/o fight; should work for civil rights rather than hope to be given them as a result of patience
                \item Launched Niagara Movement, which sooned joined forces w/ white progressives $\to$ National Asociation for the Advancement of Colored People 
                \begin{itemize}
                    \item Most admins. white, but Du Bois, publicity/research director, championed black perspective
                    \item Major progress thru. fed. cases: \textit{Guinn v. U.S.} declared grandfather clause unconstitutional; \textit{Buchanan v. Worley} ended KY law requiring residential segregation 
                \end{itemize}
            \end{itemize}
            \item NAACP criticized lynching; southern white and black women also came together to discredit lynching, segregation
        \end{itemize}
        \textbf{W.E.B. Du Bois opposed Booker T. Washington's belief in short-term self-improvement instead of long-term social change; his education gave him a far broader view of race issues. Du Bois teamed with white progressives to form the NAACP, which make progerss through federal cases and strongly opposed lynching.}}
    \cornell[Crusade for Social Order and Reform]{How did progressives fight for their beliefs in temperance and immigration?}{\textbf{Progressives began to advocate the temperance movement for its negative effects on productivity and industry; temperance advocates eventually pushed for the Eighteenth Amendment to the Constitution in the 1920, banning alcohol. Some progresssives began to push for immigration restrictions, with eugenic arguments of racial inferiority used as justification.}}
    \cornell{How did the progressives fight for the temperance movement?}{\begin{itemize}
        \item \sout{Alcohol} $\to$ critical for order: workers spent wages in saloons, drunkenness $\to$ violence w/in urban families, employers saw reduced overall productivity, industry represented large trust, tied to pol. machines
        \item Temperance essential during antebellum period; new strength in 1873 w/ Women's Christian Temperance Union becoming largest women's organization, 1893 Anti-Saloon League seeking specific abolition of saloons
        \item States began to issue temperance laws in 1916; advocates pushed for national law $\to$ Eighteenth Amendment to Constitution ratified by all but CT/RI (Catholic immigrants) in 1920
    \end{itemize}
    \textbf{Alcohol was seen as detrimental to economic success, productivity, family life; the industry was a large trust and was closely tied to political machines. As a result, women came together in the Women's Christian Temperance Union, soon joined by the Anti-Saloon League in seeking prohibition of alcohol. The 1920 Eighteenth Amendment met their desires.}}
    \cornell{How did several begin to call for restrictions on immigrants?}{\begin{itemize}
        \item All reformers agreed current handling of immigrant pop. $\to$ social problems; some sought to restrict immigration while others sought assimilation
        \item First decades saw \textbf{eugenics} movement, originating in animals but soon being used to grade races/ethnicities genetically; forced sterilization of mentally ill, criminals
        \begin{itemize}
            \item Spread belief that inequalities were hereditary; immigration $\to$ unfit able to reproduce 
            \item \textbf{Madison Grant} warned against racial "mongrelization," arguing purity of Anglo-Saxons
            \item "Experts" chaired by VT Senator \textbf{William P. Dillingham} argued new immigrants (southern/eastern Europe) less assimilable than earlier ones $\to$ immigration should be restricted by nationality
            \item Many others opposed immigration to solve urban issues of overcrowding
        \end{itemize}
        \item Concerns became enough to convince many progressives importance of restriction, including Theodore Roosevelt, by WWI 
    \end{itemize}
    \textbf{Although all reformers agreed that the current method of handling the immigrant population was sub-par, only some sought to restrict immigration. The eugenics movement propounded the view that certain races were genetically inferior to others, pushing for immigration to be restricted by race. These nativist concerns eventually reached progressives; several took them on by WWI.}}
    \cornell[Challenging the Capitalist Order]{How did some progressives defy capitalism?}{\textbf{Some progressives turned to Socialism, with some turning to Marxist views and others seeking more moderate reforms. The IWW sought a single union and fought for workers. Most members of the Socialist Party were relatively moderate. Most progressives, however, desired reform within the capitalist system, with some attacking big business and others simply desiring increased regulation.}}
    \cornell{How did some progressives turn to socialism?}{\begin{itemize}
        \item Greatest critique against capitalism \underline{ever} in pd. 1900-1914; Socialist Party became somewhat strong (though not enough)
        \begin{itemize}
            \item 1912: Eugene V. Debs attracted 1m ballots as Socialist candidate from urban immigrants (Germans/Jews) and Protestant farmers; won over 1k state/local offices
            \item Intellectuals like \textbf{Walter Lippman}, \textbf{Lincoln Steffens} began to support movement; women reformers like \textbf{Florence Kelley} and \textbf{Frances Willard} endorsed
        \end{itemize}
        \item Agreed on basic structural changes to economy; differed on extent
        \begin{itemize}
            \item Some sought radical Marxist views of Europe
            \item Others sought moderate reform w/ some private enterprise; all major industries nationalized 
            \item Some sought reform thru. electoral politics; others thru. direct action 
        \end{itemize}
        \item \textbf{Industrial Workers of the World}, known as "Wobblies" to opponents, advocated single union for all, no "wage slave" system; promoted strikes over political change
        \begin{itemize}
            \item Suspected of terrorist acts like bombing railroad lines/power stations
            \item One of few orgs. to fight for Western + unskilled workers
            \item Decline after shutting down timber industry $\to$ fed. govt., preparing to enter war, imprisoned leaders, legally banned
        \end{itemize}
        \item Socialist Party dominated by moderate Socialists seeking slow, peaceful change
        \begin{itemize}
            \item Sought gradual public education, thereby weakening system from within constraints
            \item Would not support war effort $\to$ shunned by most
        \end{itemize}
    \end{itemize}
    \textbf{The Socialists reached their peak in the period 1900-1914, receiving over one million votes in the 1912 election. They greatly differed in the extent of their views: some were Marxists, others sought moderate reform; some sought electoral reform and others through direct action. The IWW advocated a single working union for all and pushed for strikes. However, the Socialist Party which dominated the 1912 election consisted mainly of more moderate Socialists.}}
    \cornell{How did progressives push for long-term decentralization?}{\begin{itemize}
        \item Most progressives sought reform w/in capitalist system by encouraging some consolidation but breaking up largest ones for greater balance betw. size/competition 
        \item \textbf{Louis D. Brandeis}, lawyer, wrote abt. "curse of bigness" due to inefficiency and immorality of encouraging power abuse, preventing individuals from pursuing destinies
        \item Other progressives believed efficiency \textit{did} lie in size; however, govt. should work to regulate businesses and only condone the least corrupt ones
        \begin{itemize}
            \item \textbf{Herbert Croly} pushed for modernized govt. to adapt to capitalist world
        \end{itemize}
        \item Progressives began to shift attention toward econ. coordination to encourage businesses to regulate, learn new forms of cooperation; others sought more active govt. role in planning econ. life (embodied by Theodore Roosevelt)
    \end{itemize}
    \textbf{Most progressives sought reform within the capitalist structure. Some sought to reduce the overall size of businesses and government regulation to prevent corruption; others only sought government regulation. Progressives ultimately shifted towards intelligent economic coordination to promote regulation.}}
    \cornell[Theodore Roosevelt and the Modern Presidency]{How did Roosevelt transform the presidency into an office far more akin to what we know today?}{\textbf{Roosevelt, despite assuming the presidency unintentionally, was known for his significant energy. As a progressive, he sought trust regulation and sided with the common laborer more often than his predecessors. He initially sought equitable deals which woudl support all parties; after reelection, however, he attacked the railroad industry. He was both a conservationist and a preservationist, admiring the beauty of the natural landscape and taking vast expanses of land under federal control to prevent its destruction; however, he supported the construction of a dam in the Hetch Hetchy Valley for the sake of San Francisco's long-term success. He recovered from a brief economic recession by striking a deal with J.P. Morgan.}}
    \cornell{What response did Roosevelt inspire amidst the general public?}{\textbf{Roosevelt was seen as an inspirational leader worthy of great devotion. Although he was relatively conservative, Roosevelt's greatest successes were in his formal definition of the power of the executive branch.}}
    \cornell{How did Roosevelt take on the presidency?}{\begin{itemize}
        \item VP to President McKinley; after McKinley's 1901 assassination $\to$ youngest president ever 
        \item Rep. as "wild man" not due to substance of career, instead style: known for high-energy attitude in NY legislature, Badlands rancher, ardent imperialism, military successes w/ Rough Riders in Cuba
        \item Rarely rebelled against party politically
    \end{itemize}
    \textbf{Roosevelt assumed the presidency after McKinley was assassinated; although he was commonly seen as wild and out-of-control, this reputation came not from his direct rebellion but instead from his constant energy.}}
    \cornell{What was Roosevelt's perspective on economic issues?}{\begin{itemize}
        \item Allied w/ progressives seeking regulation but not destruction of trusts; sought to explore breadth of govt. power in investigating corporations
        \begin{itemize}
            \item Pioneered creation of \textbf{Department of Commerce and Labor}
            \item Some publicized efforts to destroy combinations; 1902: invoked Sherman Antitrust Act to destroy railroad monopoly in NW led by J.P. Morgan and others 
            \begin{itemize}
                \item Morgan shocked by harshness of Republican admin. $\to$ initially sought compromise
                \item Roosevelt pursued case, w/ Supr. Court ruling dissolution
            \end{itemize}
        \end{itemize}
        \item Past govts. had supported employers rather than laborers; Roosevelt considered positions of laborers
        \begin{itemize}
            \item 1902 strike by United Mine Workers $\to$ Roosevelt requested owners + miners to accept federal arbitration; owners balked $\to$ fed. troops seized mines $\to$ owners allowed 10\% wage increase 
            \item Championed both labor/management depending on case
        \end{itemize}
    \end{itemize}
    \textbf{Roosevelt generally sought to decrease the overall size of trusts without destroying them; he pioneered the Department of Commerce and Labor for government investigation in corporations. Roosevelt sided with laborers far more often than did past presidents.}}
    \cornell{What characterized the "Square Deal"?}{\begin{itemize}
        \item First yrs. as president: main concern was reelection $\to$ supported conservatives (Repub. Old Guard) + progressives, business owners + farmers $\to$ easily won 1904 election 
        \begin{itemize}
            \item Boasted during campaign abt. "square deal" (fair for \underline{everyone}) during anthracite coal strike
        \end{itemize}
        \item First targets post-election: railroad industry
        \begin{itemize}
            \item Courts had regulated power of Interstate Commerce Act $\to$ Roosevelt passed legislation to increase govt. power to oversee railroad rates (\textbf{Hepburn Railroad Regulation}), but very cautious
        \end{itemize}
        \item Pushed Congress for \textbf{Pure Food and Drug Act}, restricting dangerous medicines
        \item Upton Sinclair's novel depicting terrible conditions of meatpacking industry $\to$ \textbf{Meat Inspection Act} to eliminate diseases w/in meat
        \item 1907: several strict reforms but relatively unsuccessful, like eight-hour workday, workers' compensation, inheritance, stock market regulation 
        \item Openly criticized conservatives
    \end{itemize}
    \textbf{Roosevelt's first years as president were dedicated to winning reelection: he emphasized his focus on a "square deal," or an agreement fair for all parties. However, after he easily won reelection, he began to directly target the railroad, medical, and meatpacking industries. He also attempted some large labor and economic reforms.}}
    \cornell{How did Roosevelt's policies reflect his conservationist beliefs?}{\begin{itemize}
        \item Restricted private development on millions of Western land, adding to national forest system; when Congress restricted control over forests, seized all remaining public forests in 1907 before law passed 
        \item In Roosevelt's time, conservation relatively unpopular: most conservationists promoted policies to regulate development, not stop it
        \item Old Guard supported public reclamation/irrigation projects: National Reclamation Act (Newlands Act) federally backed Western dams, resevoirs, canals later cheap electric power
    \end{itemize}
    \textbf{Roosevelt restricted development on public Western lands despite the relatively unpopular conservationist movement. However, most of the Republican party supported his reclamation and irrigation projects.}}
    \cornell{How did Roosevelt espouse preservationist ideals?}{\textbf{Roosevelt believed strongly in the natural beauty of the land as well as its wildlife, spending days with naturalist John Muir. He built up the National Park system to protect portions of public land, adding to the existing Yellowstone, Yosemite, and Sequoia with Crater Lake, Mesa Verde, and more.}}
    \cornell{What was the Hetch Hetchy controversy?}{\begin{itemize}
        \item Yosemite's Hetch Hetchy Valley could be great dam to bring water to San Francisco, after 1906 SFO earthquake, sympathy turned toward dam; Roosevelt's admin. supported construction against Muir's naturalist ideals 
        \item Roosevelt's chief forester, Gifford Pinchot, placed city's needs above nature; Muir did opposite $\to$ constant battle
        \begin{itemize}
            \item Muir put question onto ballot, expecting overwhelming rejection of dam; instead, dam won by large margin
        \end{itemize}
        \item Muir's loss did not cripple naturalist movement, instead inspiring several more to take action
    \end{itemize}
    \textbf{The Hetch Hetchy Valley, despite being a beautiful piece of natural scenery, would provide a much-needed source of water for San Francisco. After a devastating earthquake, Roosevelt's administration supported the construction of a dam and ultimately won against the naturalists.}}
    \cornell{What caused the Panic of 1907?}{\begin{itemize}
        \item Conservatives blamed Roosevelt's econ. policies $\to$ Roosevelt quickly assured business leaders that he would not interfere w/ recovery
        \begin{itemize}
            \item J.P. Morgan constructed major pool of assets, recovering several shaky institutions w/ Roosevelt's agreement that \underline{no antitrust action} would be taken against his purchase by U.S. Steel of TN Coal/Iron Company shares
        \end{itemize}
        \item Roosevelt greatly enjoyed presidency but made promise to step down after 1904 term $\to$ reluctantly retired from public life
    \end{itemize}
    \textbf{Conservatives blamed Roosevelt for the emergence of a great economic recession; Roosevelt condoned business leaders, with J.P. Morgan striking a deal with him to recovery the economy in exchange for no antitrust action in one of Morgan's major purchases.}}
    \cornell[The Troubled Succession]{What characterized the troubled succession from Roosevelt to Taft?}{\textbf{Taft, despite seeming appealing to everyone, soon reversed many of Roosevelt's progressive actions and worked far more directly to please the Old Guard. Roosevelt, furious with Taft, returned to politics and led the angry progressives; he decided to run for the presidency after the midterms showed the great extent of the progressive movement. However, the conservative-dominated nomination process meant that Taft received the Republican nomination; Roosevelt's new Progressive Party, despite espousing several progressive ideals, had too few supporters to win the election.}}
    \cornell{What was Taft's legacy?}{\textbf{Taft was heavily favored by Roosevelt and seen as the champion of progressive ideals; however, he was also far more moderate, so conservatives expected he'd be far less aggressive as a president. Almost everyone favored him; he easily won the 1908 election. However, after four years, he left his party divided; Democrats succeeded him for the first time in 20 years.}}
    \cornell{How did Taft interact with the progressives?}{\begin{itemize}
        \item Taft called special Congress session to discuss lower tariff rates; no effort to overcome Old Guard opposition (unconstitutional) $\to$ weak Payne-Aldrich Tariff w/ very little change
        \item 1912: Created Children's Bureau to address important matters for children; led by \textbf{Julia Lathrop}, Hull House veteran
        \item 1909 controversy where Taft replaced Roosevelt's sec. of the interior (conservationist James R. Garfield) w/ conservative Richard A. Ballinger, who attempted to invalidate Roosevelt's private development restrictions
        \begin{itemize}
            \item Louis Glavis charged Ballinger w/ conniving to turn public coal lands into private enterprise $\to$ Glavis brought evidence to Gifford Pinchot (still head of Forest Service) $\to$ Pinchot brought to Taft, but Taft saw as groundless
            \item Pinchot, angry at Taft's decision, leaked story to press, asked Congress to investigate $\to$ Taft fired for insubordination; Old Guard Repubs. supported Ballinger but progressives throughout the nation supported Pinchot
        \end{itemize}
    \end{itemize}
    \textbf{Taft was known for his passivity after he refused to overcome Old Guard opposition to a tariff bill, resulting in an extremely weak result; he created some change, like in the Children's Bureau. However, controversy emerged after he condoned his new secretary of the interior's past conniving to transform public coal lands into a corporate operation, alienating all of Taft's potential Progressive supporters.}}
    \cornell{How did Roosevelt return to politics?}{\begin{itemize}
        \item Roosevelt distanced himself from most controversies, travelling on African safari, European tour; remained presence due to constant news coverage of voyages
        \item Returned to NY in 1910: angry w/ Taft's decisions, felt his leadership was required to reunite Repub. party $\to$ reentered politics 
        \item Took on leadership of Repub. reformers in 1910 speech describing "New Nationalism"
        \begin{itemize}
            \item Emphasized that he was no longer cautious conservative, instead seeking social justice
            \item Placed human welfare above personal profit, income
        \end{itemize}
    \end{itemize}
    \textbf{Roosevelt returned to politics in 1910, furious with Taft's policies. Feeling he was the sole force capable of reuniting the Republicans, he assumed the position as the leader of the progressives with a powerful speech describing his support for "New Nationalism" and denouncing his former caution.}}
    \cornell{How did the progressive cause spread throughout the nation?}{\begin{itemize}
        \item 1910 midterms indicated extent of progressives: conservative Republicans experienced major defeats, increasingly progressive Democrats won House
        \begin{itemize}
            \item Roosevelt initially denied any goal of running for president
            \item Changed mind after Taft admin. denounced Roosevelt's condonement of U.S. Steel's TN share purchases during Panic of 1907, and primary progressive candidate, WI's Robert La Follette, suffered setback $\to$ Roosevelt announced on Feb. 22, 1912 
        \end{itemize}
    \end{itemize}
    \textbf{The 1910 midterms revealed the overall extent of the progressives within the Republican party; conservatives faced huge losses, with Republican progressives and Democrat progressives alike taking great control. Roosevelt eventually announced his plan to run for the presidency in 1912.}}
    \cornell{How did Roosevelt face off against Taft for the nomination?}{\begin{itemize}
        \item Repub. nomination represented battle betw. Roosevelt, Taft; Roosevelt won major victories in primaries but nomination dominated by Repub. Old Guard $\to$ Taft received large majority of disputed seats
        \item Roosevelt took supporters, launched new Progressive Party w/ himself as candidate
        \begin{itemize}
            \item Known as "Bull Moose" party: represented culmination of growing progressive causes
            \item Roosevelt knew cause was hopeless bc. not many insurgents followed him
        \end{itemize}
    \end{itemize}
    \textbf{Taft won the nomination despite Roosevelt being heavily supported in the primaries because the nomination process remained heavily controlled by the Old Guard conservative Republicans. Roosevelt took several supporters and formed the Progressive Party, but immediately realized his chance was hopeless.}}
    \cornell[Woodrow Wilson and the New Freedom]{How did Woodrow Wilson win the presidency?}{\textbf{Wilson represented a new brand of Democratic reform. He easily won the election and began to centralize the banks under government control through the Federal Reserve Act; he surprisingly went against his direct attack on monopoly, instead creating an agency to regulate it directly. He momentarily backed down, feeling complacent; after midterm losses, however, his reforms continued, and he attacked child labor and supported progressives through the Supreme Court appointment of Louis Brandeis.}}
    \cornell{How did Woodrow Wilson rise to power?}{\begin{itemize}
        \item Wilson, representing a new brand of progressivism, emerged as the Democratic nominee
        \item Became president of Princeton in 1902, governor of NJ in 1910; continually showed commitment to reform 
        \item Espoused "New Freedom" during 1912 election: \underline{did not} condone econ. concentration, felt monopoly should be destroyed 
        \item Campaign anticlimactic: Taft barely campaigned, Roosevelt campaigned energetically until gunshot wound from attempted assassination, Wilson kept most of Democratic progressives 
        \begin{itemize}
            \item Wilson won election by preserving entire Democrat vote, while Taft/Roosevelt split Repub. vote
        \end{itemize}
    \end{itemize}
    \textbf{Wilson represented a Democratic brand of progressivism; he had shown a commitment to reform throughout his years in positions of varied leadership. He believed in the complete destruction of big business, never condoning economic concentration. Wilson easily won the 1912 election.}}
    \cornell{How did Wilson's scholarly nature impact his presidency?}{\begin{itemize}
        \item Known for great force, giving authority only to direct and clearly loyal supporters; most powerful adviser (Edward M. House) never held office
        \item Created coalition to support program w/ majority in both houses; first success was signif. lowering of protective tariff to destroy power of trusts; passed first modern income tax 
        \item Passed \textbf{Federal Reserve Act} for regional banks which would be collectively controlled by banks of its respective district
        \begin{itemize}
            \item Federal Reserve Banks would hold assets of member banks, using to support loans to private banks (w/ some interest)
            \item Federal Reserve notes (new form of currency) would become critical to basic national trade 
            \item Could easily shift funds throughout nation depending on present issue
        \end{itemize}
        \item 1914: proposed two measures to handle monopoly
        \begin{itemize}
            \item Incoporated proposal to create federal agency which would allow business to police itself (Federal Trade Commission Act) with one to directly strengthen govt. power to destroy trusts (Clayton Antitrust Act)
            \item Federal Trade Commission Act created agency to help businesses determine whether action would be acceptable to govt.; authority to prosecute unfair practices and investigate corporations at will
            \item Clayton Antitrust Bill far less appealing to him, representing reversal of initial campaign
        \end{itemize}
    \end{itemize}
    \textbf{Wilson was known as a forceful president, with the Federal Reserve Act to create regional banks, create a national currency, and easily shift money throughout the nation. Furthermore, he backed down on his initial strong stance against monopoly, instead promoting an agency allowing businesses to police themselves.}}
    \cornell{How did Wilson begin to back down on reforms?}{\begin{itemize}
        \item Wilson felt New Freedom was complete $\to$ little more reform
        \begin{itemize}
            \item Refused woman suffrage, supported segregation in South (unlike Roosevelt), dismissed all reform legislation
        \end{itemize}
        \item Midterm elections $\to$ Democrats suffered major losses $\to$ Wilson forced to continue reforms
        \begin{itemize}
            \item Appointed Louis Brandeis to Supreme Court, first Jew and first progressive justice 
            \item Supported measure to allow farmers to easily receive credit; created compensation for fed. employees
            \item Supported child labor regulation w/ 1916 \textbf{Keating-Owen Act}, preventing children from shipping goods across state lines while giving Congress authority to regulate interstate commerce
            \item Court struck down Keating-Owen Act $\to$ passed new law placing tax on child labor (also cut down)
            \item \textbf{Smith-Lever Act} matched agricultural education w/ federal grants
        \end{itemize}
        \item Fostered long-term federal growth
    \end{itemize}
    \textbf{Wilson's many successes made him complacent toward reforms; however, his significant setbacks in the midterm elections inspired far more reforms, including the appointment of a Jewish progressive lawyer to the Supreme Court, supporting farmers, regulating child labor with taxes and formal legislation. His reforms ultimately stimulated long-term federal development.}}
    \end{document}