\documentclass[a4paper]{article}
    \input{../notesheader.tex}
    \usepackage[normalem]{ulem}

    \newcommand{\chapternumber}{20}
    \newcommand{\chaptertitle}{The Progressives}
    \title{\vspace{-3em}
    \begin{tcolorbox}
    \Huge\sffamily \begin{center} AP US History  \\
    \LARGE Chapter \chapternumber - \chaptertitle \\
    \Large Finn Frankis \end{center} 
    \end{tcolorbox}
    \vspace{-3em}
    }
    \date{}
    \author{}
    
    \begin{document}
        \maketitle
        \SetBgContents{\rule[0em]{4pt}{\textheight}}
        \cornell[Key Concepts]{What are this chapter's key concepts?}{\begin{itemize}
            \item \textbf{6.3.I.C} - Artists/critics (notably agrarians, utopians, socialists, Social Gospel advocates) sought change for U.S. socio-economic state
            \item \textbf{6.3.II.B} - Women sought $\uparrow$ equality w/ men $\to$ joined orgs., attended college, promoted reform, worked in settlement houses w/ immigrants to help adaptation
            \item \textbf{7.1.II.A} - Some journalists attacked corruption, social injustice, econ. inequality; reformers hoped to find reform among immigrants 
            \item \textbf{7.1.II.B} - On national level, sought legislation to regulate econ., expand democracy, generate moral reform $\to$ taclked prohibition/suffrage
            \item \textbf{7.1.II.C} - Conservationists supported nat. parks, sought govt. responses to overuse of natural resources 
            \item \textbf{7.1.II.D} - Progressive movement never uniform: some supported segregation, others ignored; some sought wide govt. participation, others sought only experts; disagreed on immigration 
        \end{itemize}}
        \cornell[The Progressive Impulse]{What moral background underlaid the Progressive movement?}{\textbf{The Progressives pushed for constant progress in society, social cohesion, and widespread knowledge at the expense of concentrated power. They fought for social justice, which generally meant supporting the urban poor through direct aid or the creation of settlement houses to provide a safe place for impoverished immigrants. Furthermore, expertise was critical: law, medicine, business, and farming became increasingly restricted and professionalized; women fit into this society primarily as teachers and nurses.}}
        \cornell{How did the Progressives advocate progress?}{\textbf{The Progressives believed in constant social progress and growth; however, celebrated capitalist ideals of \textit{laissez-faire} and "natural laws" of the marketplace were insufficient: direct intervention was critical.}}
        \cornell{What were the various forms of Progressivism?}{\begin{itemize}
            \item \textbf{Antimonopoly} spirit feared concentrated power, sought wealth/authority dispersal w/ govt. regulation of trusts $\to$ widespread interest from farmers
            \item \textbf{Social cohesion} movement analyzed web of complex social relationships, saw well-being as closely tied to greater well-being of society
            \item Faith in knowledge saw knowledge as great equalizer $\to$ modernized govt. applying principles of natural/social sciences would succeed in regulation
        \end{itemize}
        \textbf{The antimonopolists greatly feared the concentration of power, seeking government dispersal of power. The social cohesion movement saw the individual welfare as directly linked to social welfare. Finally, several believed in the importance of knowledge and the scientific method to modernize society.}}
        \cornell{Who were the muckrakers?}{\begin{itemize}
            \item \textbf{Muckrakers} first to bring Progressive issues to social attention by publicizing corruption
            \begin{itemize}
                \item Name arose after Roosevelt accused of "raking up muck" through his writings
            \end{itemize}
            \item Targeted corrupt trusts/railroads, starting w/ \textbf{Charles Francis Adams Jr.} against railroad barons, \textbf{Ida Tarbell} against Stanadrd Oil
            \item Began to turn attention to govt., notably political machines; \textbf{Lincoln Steffens} wrote \textit{The Shame of the Cities} portraying "machine govt." and "boss rule"
            \begin{itemize}
                \item Called people to bring direct intervention into public life
            \end{itemize}
            \item Reached peak in first dedcade of 20th century
        \end{itemize}
        \textbf{The muckrakers sought to create widespread attention for Progressive social issues by exposing corruption in trusts, railroads, and urban political machines.}}
        \cornell{How did several begin to fight for social justice?}{\begin{itemize}
            \item Many reformers dedicated to "social justice," or goal to benefit all of society through egalitarianism and support for poor $\to$ \textbf{Social Gospel} within Protestantism to reform cities
            \item Salvation Army blended religion w/ reform: structure somewhat resembled military, provided relief for urban poor
            \item Charles Sheldon's \textit{In His Steps} described story of young minister abandoning work to help poor
            \item \textbf{Walter Rauschenbusch}, Protestant theologian, believed Darwinism described evolution of society as a whole, requiring effort of all
            \item Some Catholics followed Pope Leo XIII's \textit{Rerum Novarum}, w/ liberal \textbf{Father John Ryan} likening conditions of poor to slavery
        \end{itemize}
        \textbf{The Social Gospel movement was dedicated to assisting the urban poor; the Salvation Army blended religion with relief for the urban poor, and Catholics and Protestants alike found religious justifications for assisting the less fortunate.}}
        \cornell{What characterized the growth of settlement houses?}{\begin{itemize}
            \item Progressivists felt poverty was due to unfortunate environment $\to$ $\uparrow$ living conditions required
            \item Crowded immigrant neighborhoods greatest cause of distress $\to$ English idea of settlement house
            \begin{itemize}
                \item \textbf{Hull House} in Chicago (1889) after work of Jane Addams became universal model for ones through the nation
                \item Workers were educated middle class; encouraged assimilation without condescension/shunning of past beliefs
                \item Felt middle class were responsible for passing on knowledge to lower classes
            \end{itemize}
            \item Young women played major role: aligned w/ society's belief of requiring women to be consistently sheltered in clean buildings
            \item Social work became critical profession aligned w/ study of sociology $\to$ universities began to take far more seriously 
        \end{itemize}
        \textbf{Settlement houses emerged due to the anti-Social Darwinist belief that poverty was due to a poor environment. They provided a safe, clean place for immigrants to live and receive guidance from the educated middle class. They provided great opportunities for young women and created a new profession: social work.}}
        \cornell{How did the progressives place an emphasis on expertise?}{\textbf{The Progressives believed that all problems, whether scientific or not, could be approached analytically; some sought a new society where scientists and engineers guided all others.}}
        \cornell{What main professions did the Progressives advocate?}{\begin{itemize}
            \item Factories $\to$ $\uparrow$ admin. tasks like managers, technicians, accountants; cities $\to$ $\uparrow$ commercial/medical/legal/education; new technology $\to$ teachers
            \item Middle class placed individualistic focus on accomplishment $\to$ all worked to secure social position
            \begin{itemize}
                \item With limited training, professionalism very rare: anyone could claim themselves a lawyer/doctor w/ limited training
            \end{itemize}
            \item Medical first to professionalize (1901): American Medical Association $\to$ national professional society w/ specific standards $\to$ states began to pass licensing laws 
            \begin{itemize}
                \item Several medical schools began to parallel those in Europe
            \end{itemize}
            \item By 1916, all states had professionalized law w/ expansion of law schoools 
            \item Businessmen formed National Association of Manufacturers, U.S. Chamber of Commerce
            \item Long-famed individualistic farmers formed Farm Bureau Federation to spread scientific methods
            \item Requirements protected those already professional while eliminating fakes; sometimes created strict requirements to exclude blacks/women/immigrants or to keep numbers down for high demand
        \end{itemize}
        \textbf{The Progressives, placing a focus on individual accomplishment, strongly pushed the professionalization of professions like medicine, law, business, and even farming to ensure a strict standard for workers. This professionalization often excluded more than necessary.}}
        \cornell{How did women fit into the new professional system?}{\begin{itemize}
            \item Restrictions/prejudice $\to$ women generally excluded; several middle class women worked to earn education, enter careers
            \item Some women became physicians/lawyers/engineers/scientists/managers w/ 5\% of physicians women due to several medical schools admitting women
            \item Most turned to "helping" domestic professions pushed for by society, like social work, \underline{teaching}
            \begin{itemize}
                \item 90\% of professional women were teachers
                \item Educated black women found jobs in segregated schools of South
            \end{itemize}
            \item During/post-Civil War, women dominated nursing w/ professionalization $\to$ several earned advanced degrees 
        \end{itemize}
        \textbf{legal restrictions and social prejudice meant that women were generally excluded from the new professional system: only a small proportion were able to become physicians or other professional jobs. Most turned to "helping" professions, with teaching by far the most popular, followed by nursing.}}
        \cornell[Women and Reform]{How did women fight for reform?}{\textbf{Women's rights were severely limited at the onset of the 20th century; however, a reduced focus on the home allowed several women to enter the public sphere and fight for women's rights. Women's clubs stimulated reform on a legislative level for female and child labor, alcohol, pensions for widowed mothers, and more. The greatest accomplishment in women's rights of the early 20th century came in the Nineteenth Amendment, where women received suffrage after centuries of pushing for it.}}
        \cornell{What was the state of women's rights at the turn of the century?}{\textbf{Generally, women could not vote, hold public office, or take on professional jobs apart from teaching and nursing. Society continually pushed them out of the public world.}}
        \cornell{How did the "new woman" emerge?}{\begin{itemize}
            \item Socio-economic changes $\to$ private world reformed to same degree as public one w/ income out of home, children entering school at younger ages and spending more time there, innovations $\to$ less housework
            \item $\downarrow$ family size, longevity $\to$ women spent fewer years w/ younger children, lived longer after children had grown
            \item Some educated women renounced marriage: remaining single was essential for public role
            \begin{itemize}
                \item Single women generally most prominent female reformers (like Jane Addams in settlement houses, Frances Willard in temperance, Anna Howard Shaw in suffrage)
                \item Some lived alone, others w/ other women $\to$ occasional romantic relations
            \end{itemize}
        \end{itemize}
        \textbf{Socio-economic changes greatly impacted the private world, ultimately placing far less emphasis on the home as an economic unit and requiring women to spend fewer years nurturing children. As a result, several dedicated their lives to reform, with many of the greatest reformers never marrying.}}
        \cornell{How did women's clubs stimulate the women's reform movement?}{\begin{itemize}
            \item Women's clubs began as intellectual orgs. for middle/upper class women 
            \item By turn of century, clubs focused more on social change
            \begin{itemize}
                \item Full of wealthy women $\to$ significant money dedicated to change
                \item Women could not vote $\to$ nonpartisan image $\to$ politicians forced to recognize
            \end{itemize}
            \item Most clubs excluded blacks $\to$ \textbf{National Association of Colored Women} often fusing issues for Afr. Americans (like lynching/segregation) w/ those for women 
            \item Club movement rarely made controversial changes to challenge social assumptions abt. women: instead sought for public place for women w/in traditional gender roles
            \item Some uncontroversial construction projects; signif. change to legislation overall
            \begin{itemize}
                \item Woman + child labor reduced, food/drug industry regulated, tribal policies reformed, \underline{alcohol outlawed}
                \item Created "mother's pensions" on state level to assist widows w/ small children; became part of Social Security
                \item Children's Bureau created as part of Labor Department to protect children
            \end{itemize}
            \item Often allied w/ other women's groups like Women's Trade Union League to convince women to unionize, strike 
        \end{itemize}
        \textbf{Women's clubs quickly shifted from a place of cultural discussion to one of social reform: significant wealth allowed for great women's reform through legislation changes, including for female and child labor, alcohol, tribes, and unionization.}}
        \cornell{How did women fight for the right to vote?}{\begin{itemize}
            \item Prevailing view that suffrage was radical demand due to early supporters
            \begin{itemize}
                \item Presented as "natural right" to be equal w/ men rather than in separate spheres $\to$ male-dominated antisuffrage movement 
                \item Antisuffrage movement associated w/ divorce, promiscuity, immorality, neglect of children (some women on board)
            \end{itemize}
            \item Suffrage movement slowly overcame opposition in early 20th c. due to $\uparrow$ org. under \textbf{Anna Howard Shaw}, \textbf{Carrie Chapman Catt}, far less controversial justifications
            \begin{itemize}
                \item Argued would not challenge "separate sphere" ideology; bc. such a different sphere, women would provide new insights
                \item Tied directly into temperance movement bc. women were strongest advocates
                \item Argued war would end after women given more power bc. would provide calming factor $\to$ WWI gave final push
                \item Others felt that if black/immigrants could vote, women deserved it too 
            \end{itemize}
            \item In 1910, WA granted right to vote; CA soon followed; IL became first state east of MS River in 1913
            \item Culminated in 1920 passing of Nineteenth Amendment to guarantee pol. rights to women throughout nation
            \item \textbf{Alice Paul}, part of National Woman's Party, hated "seperate sphere" justification and sought legislation giving \underline{complete equality}; received little support
        \end{itemize}
        \textbf{The suffrage movement started off slow due to the early justification that it would bring a complete transformation to society; however, greater organization and far more careful justifications, such as the preservation of the "separate spheres" construct and the practical benefits of women having a voice, ultimately allowed the 1920 Nineteenth Amendment to guarantee political rights to women throughout the nation.}}
        \cornell[The Assault on the Parties]{How did progressives attack the seemingly corrupt American government?}{\textbf{Progressives pushed for the secret ballot and encouraged voter turnout in municipal elections to ultimately reform city governments away from political machines toward those with far less partisan goals. Furthermore, they influenced state legislatures by reducing the influence of corporations; they bypassed state governments through direct primaries and recall. Ultimately, the push for reduced partisanism meant that the voter turnout in federal elections was severely reduced due to the decline of boss rule and the formation of interest groups.}}
        \cornell{What characterized the early attacks on the federal government?}{\begin{itemize}
            \item Greenbackism and Populism were earliest successes of breaking Democrat/Republican hold; Independent Republicans made significant attempt
            \item Secret ballot adopted in 1880s/1890s replacing method of parties distributing filled-in tickets to supporters $\to$ bosses unable to monitor voting, ticket could be "split" between parties
        \end{itemize}
        \textbf{Greenbackism, Populism, and Independent Republicanism were the earliest assaults on the political parties. The greatest success against partisan dominance was the adoption of the secret ballot, preventing bosses from monitoring voting and allowing far greater individual agency in voting.}}
        \cornell{How did the progressives make significant reforms on the municipal level?}{\begin{itemize}
            \item Muckrakers encouraged "respectable" citizens who stayed away from voting due to "vulgar," corrupt bosses to take greater role $\to$ middle class began to take on $\uparrow$ pol. role 
            \item Activists faced city bosses, saloon/brothel owners, businessmen relying on profits and urban immigrants relying on relief from pol. machines 
        \end{itemize}
        \textbf{The muckrakers removed the stigma surrounding voting as a "vulgar" activity rampant with corruption. Activists who began to take a greater role in voting faced city bosses as well as all who relied on political machines.}}
        \cornell{What new forms of governance emerged to better rule cities?}{\begin{itemize}
            \item First govt. success in Galveston, TX w/ new city charter after govt. unable to deal w/ 1900 tidal wave 
            \begin{itemize}
                \item Mayor/council elected by nonpartisan commission (same in Des Moines)
            \end{itemize}
            \item \textbf{City-manager plan} entailed bringing in outside expert to take charge w/o corruption of politics
            \item Most reformers settled for more modest goals 
            \begin{itemize}
                \item Some city elections became nonpartisan; several others remained partisan but never ran during same years as presidential/congressional elections
                \item Attempted to force city councillors to run at-large to prevent boss rule; tried to strengthen power of mayor against council 
            \end{itemize}
            \item \textbf{Tom Johnson}, mayor of Cleveland, fought long battle in city to $\downarrow$ streetcar fare to 3 cents; successor held up reputation of Cleveland as best-governed city 
        \end{itemize}
        \textbf{Several cities reformed their systems of government, whether by creating a new charter, adopting the city-manager plan where a non-partisan expert would take control, or by changing city elections to be nonpartisan or never occur during federal elections. Tom Johnson, the powerful mayor of Cleveland, epitomized the urban political reform movement.}}
        \cornell{How did progressives influence state governments?}{\begin{itemize}
            \item Progressives felt members of state legislatures generally incompetent, underpaid, dominated by party bosses 
            \item Two most significant changes came from Populists: sought legislation changes to go directly to ppl., sent legislation changes to electorate for approval
            \begin{itemize}
                \item \textbf{Direct primary} sought to limit party power, create more competent elected officials; also used to limit black power in South
                \item Recall $\to$ public officials could be removed at special election if petition signed by citizens; adopted by few states (notably CA)
            \end{itemize}
            \item Other reform measures directly cleaned legislatures by preventing lobbying by businesses, corporate endorsements of campaigns, public officials freom receiving free passes from railroads (in many states)
            \begin{itemize}
                \item Increased workers' compensation, created pensions for widows
            \end{itemize}
            \item Reform most successful in states w/ dedicated politicians
            \begin{itemize}
                \item NY: governor \textbf{Charles Evans Hughes} created commission to regulate public utilities
                \item CA: \textbf{Hiram Johnson} limited Southern Pacific Railroad
                \item NJ: \textbf{Woodrow Wilson} reduced power of trusts
                \item WI: \textbf{Robert M. La Folette} tured state into "laboratory of progressivism" by implementing direct primaries, initiatives, referendums; limited railroads/utilities; brought worker compensations; raised taxes; widened public awareness through personal magnetism
            \end{itemize}
        \end{itemize}
        \textbf{Progressives targeted state legislatures through the direct primary, allowing people to vote directly on legislation changes, as well as the recall, allowing incompetent individuals to be removed from office at a special election. Others prevented the influence of big businesses on the legislature directly. Reform was most successful in states with dedicated politicians, notably the Progressive Robert M. La Folette, known for personal magnetism.}}
        \cornell{How did parties decline in overall influence?}{\begin{itemize}
            \item Weakening of parties $\to$ $\downarrow$ voter turnout, having already reached peak 
            \item Secret ballot $\to$ bosses unable to influence vote through force, illiterate voters unable to read ballot
            \item \textbf{Interest groups} replaced party bosses by unifying people by profession, trade, culture
        \end{itemize}
        \textbf{The decline of parties was seen most clearly in the decline in voter turnout due to the secret ballot reducing the influence of bosses and interest groups unifying people under other fronts.}}
        \cornell[Sources of Progressive Reform]{What were the demographics of most Progressive reformers?}{\textbf{Unions and machines worked for progressive reform, most notably the changing Tammany Hall. Western progressives targeted the federal government due to its strong influence over western territories. African Americans, under W.E.B. Du Bois, sought social change for blacks through federal laws.}}
        \cornell{How did workers and machines seek progressive reform?}{\begin{itemize}
            \item Some unions (not AFL) critical to reform: Union Labor Party $\to$ CA child-labor, workers comp., reduced female hours laws 
            \item NYC's infamous Tammany Hall (city machine) fused desires w/ social reformers: pushed for working conditions/child labor legislation
            \begin{itemize}
                \item Fire at Triangle Shirtwaist Company in NYC, killing 146; state commission pushed by workers (and somewhat by Tammany) issued technical report calling for reforms
                \item Recommendations of state commission most heavily supported by working-class Tammany Democrats in state legislature
            \end{itemize}
        \end{itemize}
        \textbf{Unions were significant to pushing new legislature. Surprisingly, several political machines adapted their intentions to align with the goals of reformers in order to preserve their influence. Tammany Hall, most notably, pushed for child labor and working conditions legislation and called for major political reforms after a deadly fire.}}
        \cornell{How did some Westerners embody the progressive movement?}{\begin{itemize}
            \item West produced some of greatest progressive leaders (CA's Hiram Johnson, NE's George Norris, ID's William Borah)
            \item Greatest focus on federal change due to disproportionate authority relative to that in East
            \begin{itemize}
                \item Disputes over waterways typically traversed multiple states (like CO River)
                \item Western states relied on fed. govt. for land grants, railroad/waterproject subsidies
                \item Large portions of unsettled land controlled by the govt. (far more than in the East)
            \end{itemize}
        \end{itemize}
        \textbf{The West produced some of the most powerful progressive leaders, who generally focused on federal change due to the massive influence of the federal government over waterways and land grants.}}
        \cornell{How did African Americans fit into the progressive movement?}{\begin{itemize}
            \item Afr. Americans faced greatest struggle for reform $\to$ \textbf{Booker T. Washington} pushed for short-term self-improvement rather than significant effort on long-term change
            \item \textbf{W.E.B. Du Bois} championed new movement for long-term change
            \begin{itemize}
                \item Never known slavery (unlike Booker T. Washington); educated at Fisk University, Harvard $\to$ far broader view of racial issues for blacks
                \item Directly attacked Washington: felt no progress would ever be made if blacks continually servile w/o fight; should work for civil rights rather than hope to be given them as a result of patience
                \item Launched Niagara Movement, which sooned joined forces w/ white progressives $\to$ National Asociation for the Advancement of Colored People 
                \begin{itemize}
                    \item Most admins. white, but Du Bois, publicity/research director, championed black perspective
                    \item Major progress thru. fed. cases: \textit{Guinn v. U.S.} declared grandfather clause unconstitutional; \textit{Buchanan v. Worley} ended KY law requiring residential segregation 
                \end{itemize}
            \end{itemize}
            \item NAACP criticized lynching; southern white and black women also came together to discredit lynching, segregation
        \end{itemize}
        \textbf{W.E.B. Du Bois opposed Booker T. Washington's belief in short-term self-improvement instead of long-term social change; his education gave him a far broader view of race issues. Du Bois teamed with white progressives to form the NAACP, which make progerss through federal cases and strongly opposed lynching.}}
    \cornell[Crusade for Social Order and Reform]{How did progressives fight for their beliefs in temperance and immigration?}{\textbf{Progressives began to advocate the temperance movement for its negative effects on productivity and industry; temperance advocates eventually pushed for the Eighteenth Amendment to the Constitution in the 1920, banning alcohol. Some progresssives began to push for immigration restrictions, with eugenic arguments of racial inferiority used as justification.}}
    \cornell{How did the progressives fight for the temperance movement?}{\begin{itemize}
        \item \sout{Alcohol} $\to$ critical for order: workers spent wages in saloons, drunkenness $\to$ violence w/in urban families, employers saw reduced overall productivity, industry represented large trust, tied to pol. machines
        \item Temperance essential during antebellum period; new strength in 1873 w/ Women's Christian Temperance Union becoming largest women's organization, 1893 Anti-Saloon League seeking specific abolition of saloons
        \item States began to issue temperance laws in 1916; advocates pushed for national law $\to$ Eighteenth Amendment to Constitution ratified by all but CT/RI (Catholic immigrants) in 1920
    \end{itemize}
    \textbf{Alcohol was seen as detrimental to economic success, productivity, family life; the industry was a large trust and was closely tied to political machines. As a result, women came together in the Women's Christian Temperance Union, soon joined by the Anti-Saloon League in seeking prohibition of alcohol. The 1920 Eighteenth Amendment met their desires.}}
    \cornell{How did several begin to call for restrictions on immigrants?}{\begin{itemize}
        \item All reformers agreed current handling of immigrant pop. $\to$ social problems; some sought to restrict immigration while others sought assimilation
        \item First decades saw \textbf{eugenics} movement, originating in animals but soon being used to grade races/ethnicities genetically; forced sterilization of mentally ill, criminals
        \begin{itemize}
            \item Spread belief that inequalities were hereditary; immigration $\to$ unfit able to reproduce 
            \item \textbf{Madison Grant} warned against racial "mongrelization," arguing purity of Anglo-Saxons
            \item "Experts" chaired by VT Senator \textbf{William P. Dillingham} argued new immigrants (southern/eastern Europe) less assimilable than earlier ones $\to$ immigration should be restricted by nationality
            \item Many others opposed immigration to solve urban issues of overcrowding
        \end{itemize}
        \item Concerns became enough to convince many progressives importance of restriction, including Theodore Roosevelt, by WWI 
    \end{itemize}
    \textbf{Although all reformers agreed that the current method of handling the immigrant population was sub-par, only some sought to restrict immigration. The eugenics movement propounded the view that certain races were genetically inferior to others, pushing for immigration to be restricted by race. These nativist concerns eventually reached progressives; several took them on by WWI.}}
    \end{document}