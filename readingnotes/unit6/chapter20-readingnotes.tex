\documentclass[a4paper]{article}
    \usepackage[T1]{fontenc}
    \usepackage{tcolorbox}
    \usepackage{amsmath}
    \tcbuselibrary{skins}
    
    \usepackage{background}
    \SetBgScale{1}
    \SetBgAngle{0}
    \SetBgColor{red}
    \SetBgContents{\rule[0em]{4pt}{\textheight}}
    \SetBgHshift{-2.3cm}
    \SetBgVshift{0cm}
    \usepackage[margin=2cm]{geometry} 
    
    \makeatletter
    \def\cornell{\@ifnextchar[{\@with}{\@without}}
    \def\@with[#1]#2#3{
    \begin{tcolorbox}[enhanced,colback=gray,colframe=black,fonttitle=\large\bfseries\sffamily,sidebyside=true, nobeforeafter,before=\vfil,after=\vfil,colupper=blue,sidebyside align=top, lefthand width=.3\textwidth,
    opacityframe=0,opacityback=.3,opacitybacktitle=1, opacitytext=1,
    segmentation style={black!55,solid,opacity=0,line width=3pt},
    title=#1
    ]
    \begin{tcolorbox}[colback=red!05,colframe=red!25,sidebyside align=top,
    width=\textwidth,nobeforeafter]#2\end{tcolorbox}%
    \tcblower
    \sffamily
    \begin{tcolorbox}[colback=blue!05,colframe=blue!10,width=\textwidth,nobeforeafter]
    #3
    \end{tcolorbox}
    \end{tcolorbox}
    }
    \def\@without#1#2{
    \begin{tcolorbox}[enhanced,colback=white!15,colframe=white,fonttitle=\bfseries,sidebyside=true, nobeforeafter,before=\vfil,after=\vfil,colupper=blue,sidebyside align=top, lefthand width=.3\textwidth,
    opacityframe=0,opacityback=0,opacitybacktitle=0, opacitytext=1,
    segmentation style={black!55,solid,opacity=0,line width=3pt}
    ]
    
    \begin{tcolorbox}[colback=red!05,colframe=red!25,sidebyside align=top,
    width=\textwidth,nobeforeafter]#1\end{tcolorbox}%
    \tcblower
    \sffamily
    \begin{tcolorbox}[colback=blue!05,colframe=blue!10,width=\textwidth,nobeforeafter]
    #2
    \end{tcolorbox}
    \end{tcolorbox}
    }
    \makeatother

    \parindent=0pt
    \newcommand{\chapternumber}{20}
    \newcommand{\chaptertitle}{The Progressives}
    \title{\vspace{-3em}
    \begin{tcolorbox}
    \Huge\sffamily \begin{center} AP US History  \\
    \LARGE Chapter \chapternumber - \chaptertitle \\
    \Large Finn Frankis \end{center} 
    \end{tcolorbox}
    \vspace{-3em}
    }
    \date{}
    \author{}
    
    \begin{document}
        \maketitle
        \SetBgContents{\rule[0em]{4pt}{\textheight}}
        \cornell[Key Concepts]{What are this chapter's key concepts?}{\begin{itemize}
            \item \textbf{6.3.I.C} - Artists/critics (notably agrarians, utopians, socialists, Social Gospel advocates) sought change for U.S. socio-economic state
            \item \textbf{6.3.II.B} - Women sought $\uparrow$ equality w/ men $\to$ joined orgs., attended college, promoted reform, worked in settlement houses w/ immigrants to help adaptation
            \item \textbf{7.1.II.A} - Some journalists attacked corruption, social injustice, econ. inequality; reformers hoped to find reform among immigrants 
            \item \textbf{7.1.II.B} - On national level, sought legislation to regulate econ., expand democracy, generate moral reform $\to$ taclked prohibition/suffrage
            \item \textbf{7.1.II.C} - Conservationists supported nat. parks, sought govt. responses to overuse of natural resources 
            \item \textbf{7.1.II.D} - Progressive movement never uniform: some supported segregation, others ignored; some sought wide govt. participation, others sought only experts; disagreed on immigration 
        \end{itemize}}
        \cornell[The Progressive Impulse]{What moral background underlaid the Progressive movement?}{\textbf{The Progressives pushed for constant progress in society, social cohesion, and widespread knowledge at the expense of concentrated power. They fought for social justice, which generally meant supporting the urban poor through direct aid or the creation of settlement houses to provide a safe place for impoverished immigrants. Furthermore, expertise was critical: law, medicine, business, and farming became increasingly restricted and professionalized; women fit into this society primarily as teachers and nurses.}}
        \cornell{How did the Progressives advocate progress?}{\textbf{The Progressives believed in constant social progress and growth; however, celebrated capitalist ideals of \textit{laissez-faire} and "natural laws" of the marketplace were insufficient: direct intervention was critical.}}
        \cornell{What were the various forms of Progressivism?}{\begin{itemize}
            \item \textbf{Antimonopoly} spirit feared concentrated power, sought wealth/authority dispersal w/ govt. regulation of trusts $\to$ widespread interest from farmers
            \item \textbf{Social cohesion} movement analyzed web of complex social relationships, saw well-being as closely tied to greater well-being of society
            \item Faith in knowledge saw knowledge as great equalizer $\to$ modernized govt. applying principles of natural/social sciences would succeed in regulation
        \end{itemize}
        \textbf{The antimonopolists greatly feared the concentration of power, seeking government dispersal of power. The social cohesion movement saw the individual welfare as directly linked to social welfare. Finally, several believed in the importance of knowledge and the scientific method to modernize society.}}
        \cornell{Who were the muckrakers?}{\begin{itemize}
            \item \textbf{Muckrakers} first to bring Progressive issues to social attention by publicizing corruption
            \begin{itemize}
                \item Name arose after Roosevelt accused of "raking up muck" through his writings
            \end{itemize}
            \item Targeted corrupt trusts/railroads, starting w/ \textbf{Charles Francis Adams Jr.} against railroad barons, \textbf{Ida Tarbell} against Stanadrd Oil
            \item Began to turn attention to govt., notably political machines; \textbf{Lincoln Steffens} wrote \textit{The Shame of the Cities} portraying "machine govt." and "boss rule"
            \begin{itemize}
                \item Called people to bring direct intervention into public life
            \end{itemize}
            \item Reached peak in first dedcade of 20th century
        \end{itemize}
        \textbf{The muckrakers sought to create widespread attention for Progressive social issues by exposing corruption in trusts, railroads, and urban political machines.}}
        \cornell{How did several begin to fight for social justice?}{\begin{itemize}
            \item Many reformers dedicated to "social justice," or goal to benefit all of society through egalitarianism and support for poor $\to$ \textbf{Social Gospel} within Protestantism to reform cities
            \item Salvation Army blended religion w/ reform: structure somewhat resembled military, provided relief for urban poor
            \item Charles Sheldon's \textit{In His Steps} described story of young minister abandoning work to help poor
            \item \textbf{Walter Rauschenbusch}, Protestant theologian, believed Darwinism described evolution of society as a whole, requiring effort of all
            \item Some Catholics followed Pope Leo XIII's \textit{Rerum Novarum}, w/ liberal \textbf{Father John Ryan} likening conditions of poor to slavery
        \end{itemize}
        \textbf{The Social Gospel movement was dedicated to assisting the urban poor; the Salvation Army blended religion with relief for the urban poor, and Catholics and Protestants alike found religious justifications for assisting the less fortunate.}}
        \cornell{What characterized the growth of settlement houses?}{\begin{itemize}
            \item Progressivists felt poverty was due to unfortunate environment $\to$ $\uparrow$ living conditions required
            \item Crowded immigrant neighborhoods greatest cause of distress $\to$ English idea of settlement house
            \begin{itemize}
                \item \textbf{Hull House} in Chicago (1889) after work of Jane Addams became universal model for ones through the nation
                \item Workers were educated middle class; encouraged assimilation without condescension/shunning of past beliefs
                \item Felt middle class were responsible for passing on knowledge to lower classes
            \end{itemize}
            \item Young women played major role: aligned w/ society's belief of requiring women to be consistently sheltered in clean buildings
            \item Social work became critical profession aligned w/ study of sociology $\to$ universities began to take far more seriously 
        \end{itemize}
        \textbf{Settlement houses emerged due to the anti-Social Darwinist belief that poverty was due to a poor environment. They provided a safe, clean place for immigrants to live and receive guidance from the educated middle class. They provided great opportunities for young women and created a new profession: social work.}}
        \cornell{How did the progressives place an emphasis on expertise?}{\textbf{The Progressives believed that all problems, whether scientific or not, could be approached analytically; some sought a new society where scientists and engineers guided all others.}}
        \cornell{What main professions did the Progressives advocate?}{\begin{itemize}
            \item Factories $\to$ $\uparrow$ admin. tasks like managers, technicians, accountants; cities $\to$ $\uparrow$ commercial/medical/legal/education; new technology $\to$ teachers
            \item Middle class placed individualistic focus on accomplishment $\to$ all worked to secure social position
            \begin{itemize}
                \item With limited training, professionalism very rare: anyone could claim themselves a lawyer/doctor w/ limited training
            \end{itemize}
            \item Medical first to professionalize (1901): American Medical Association $\to$ national professional society w/ specific standards $\to$ states began to pass licensing laws 
            \begin{itemize}
                \item Several medical schools began to parallel those in Europe
            \end{itemize}
            \item By 1916, all states had professionalized law w/ expansion of law schoools 
            \item Businessmen formed National Association of Manufacturers, U.S. Chamber of Commerce
            \item Long-famed individualistic farmers formed Farm Bureau Federation to spread scientific methods
            \item Requirements protected those already professional while eliminating fakes; sometimes created strict requirements to exclude blacks/women/immigrants or to keep numbers down for high demand
        \end{itemize}
        \textbf{The Progressives, placing a focus on individual accomplishment, strongly pushed the professionalization of professions like medicine, law, business, and even farming to ensure a strict standard for workers. This professionalization often excluded more than necessary.}}
        \cornell{How did women fit into the new professional system?}{\begin{itemize}
            \item Restrictions/prejudice $\to$ women generally excluded; several middle class women worked to earn education, enter careers
            \item Some women became physicians/lawyers/engineers/scientists/managers w/ 5\% of physicians women due to several medical schools admitting women
            \item Most turned to "helping" domestic professions pushed for by society, like social work, \underline{teaching}
            \begin{itemize}
                \item 90\% of professional women were teachers
                \item Educated black women found jobs in segregated schools of South
            \end{itemize}
            \item During/post-Civil War, women dominated nursing w/ professionalization $\to$ several earned advanced degrees 
        \end{itemize}
        \textbf{legal restrictions and social prejudice meant that women were generally excluded from the new professional system: only a small proportion were able to become physicians or other professional jobs. Most turned to "helping" professions, with teaching by far the most popular, followed by nursing.}}
        \cornell[Women and Reform]{How did women fight for reform?}{\textbf{Women's rights were severely limited at the onset of the 20th century; however, a reduced focus on the home allowed several women to enter the public sphere and fight for women's rights. Women's clubs stimulated reform on a legislative level for female and child labor, alcohol, pensions for widowed mothers, and more. The greatest accomplishment in women's rights of the early 20th century came in the Nineteenth Amendment, where women received suffrage after centuries of pushing for it.}}
        \cornell{What was the state of women's rights at the turn of the century?}{\textbf{Generally, women could not vote, hold public office, or take on professional jobs apart from teaching and nursing. Society continually pushed them out of the public world.}}
        \cornell{How did the "new woman" emerge?}{\begin{itemize}
            \item Socio-economic changes $\to$ private world reformed to same degree as public one w/ income out of home, children entering school at younger ages and spending more time there, innovations $\to$ less housework
            \item $\downarrow$ family size, longevity $\to$ women spent fewer years w/ younger children, lived longer after children had grown
            \item Some educated women renounced marriage: remaining single was essential for public role
            \begin{itemize}
                \item Single women generally most prominent female reformers (like Jane Addams in settlement houses, Frances Willard in temperance, Anna Howard Shaw in suffrage)
                \item Some lived alone, others w/ other women $\to$ occasional romantic relations
            \end{itemize}
        \end{itemize}
        \textbf{Socio-economic changes greatly impacted the private world, ultimately placing far less emphasis on the home as an economic unit and requiring women to spend fewer years nurturing children. As a result, several dedicated their lives to reform, with many of the greatest reformers never marrying.}}
        \cornell{How did women's clubs stimulate the women's reform movement?}{\begin{itemize}
            \item Women's clubs began as intellectual orgs. for middle/upper class women 
            \item By turn of century, clubs focused more on social change
            \begin{itemize}
                \item Full of wealthy women $\to$ significant money dedicated to change
                \item Women could not vote $\to$ nonpartisan image $\to$ politicians forced to recognize
            \end{itemize}
            \item Most clubs excluded blacks $\to$ \textbf{National Association of Colored Women} often fusing issues for Afr. Americans (like lynching/segregation) w/ those for women 
            \item Club movement rarely made controversial changes to challenge social assumptions abt. women: instead sought for public place for women w/in traditional gender roles
            \item Some uncontroversial construction projects; signif. change to legislation overall
            \begin{itemize}
                \item Woman + child labor reduced, food/drug industry regulated, tribal policies reformed, \underline{alcohol outlawed}
                \item Created "mother's pensions" on state level to assist widows w/ small children; became part of Social Security
                \item Children's Bureau created as part of Labor Department to protect children
            \end{itemize}
            \item Often allied w/ other women's groups like Women's Trade Union League to convince women to unionize, strike 
        \end{itemize}
        \textbf{Women's clubs quickly shifted from a place of cultural discussion to one of social reform: significant wealth allowed for great women's reform through legislation changes, including for female and child labor, alcohol, tribes, and unionization.}}
        \cornell{How did women fight for the right to vote?}{\begin{itemize}
            \item Prevailing view that suffrage was radical demand due to early supporters
            \begin{itemize}
                \item Presented as "natural right" to be equal w/ men rather than in separate spheres $\to$ male-dominated antisuffrage movement 
                \item Antisuffrage movement associated w/ divorce, promiscuity, immorality, neglect of children (some women on board)
            \end{itemize}
            \item Suffrage movement slowly overcame opposition in early 20th c. due to $\uparrow$ org. under \textbf{Anna Howard Shaw}, \textbf{Carrie Chapman Catt}, far less controversial justifications
            \begin{itemize}
                \item Argued would not challenge "separate sphere" ideology; bc. such a different sphere, women would provide new insights
                \item Tied directly into temperance movement bc. women were strongest advocates
                \item Argued war would end after women given more power bc. would provide calming factor $\to$ WWI gave final push
                \item Others felt that if black/immigrants could vote, women deserved it too 
            \end{itemize}
            \item In 1910, WA granted right to vote; CA soon followed; IL became first state east of MS River in 1913
            \item Culminated in 1920 passing of Nineteenth Amendment to guarantee pol. rights to women throughout nation
            \item \textbf{Alice Paul}, part of National Woman's Party, hated "seperate sphere" justification and sought legislation giving \underline{complete equality}; received little support
        \end{itemize}
        \textbf{The suffrage movement started off slow due to the early justification that it would bring a complete transformation to society; however, greater organization and far more careful justifications, such as the preservation of the "separate spheres" construct and the practical benefits of women having a voice, ultimately allowed the 1920 Nineteenth Amendment to guarantee political rights to women throughout the nation.}}
    \end{document}