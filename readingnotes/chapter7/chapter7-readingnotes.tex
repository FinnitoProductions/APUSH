\documentclass[a4paper]{article}
    \usepackage[T1]{fontenc}
    \usepackage{tcolorbox}
    \usepackage{amsmath}
    \tcbuselibrary{skins}
    
    \title{
    \vspace{-3em}
    \begin{tcolorbox}
    \Huge\sffamily \begin{center} AP US History  \\
    \LARGE Chapter 7 - The Jeffersonian Era \\
    \Large Finn Frankis \end{center} 
    \end{tcolorbox}
    \vspace{-3em}
    }
    \date{}
    \author{}
    
    \usepackage{background}
    \SetBgScale{1}
    \SetBgAngle{0}
    \SetBgColor{red}
    \SetBgContents{\rule[0em]{4pt}{\textheight}}
    \SetBgHshift{-2.3cm}
    \SetBgVshift{0cm}
    \usepackage[margin=2cm]{geometry} 
    
    \makeatletter
    \def\cornell{\@ifnextchar[{\@with}{\@without}}
    \def\@with[#1]#2#3{
    \begin{tcolorbox}[enhanced,colback=gray,colframe=black,fonttitle=\large\bfseries\sffamily,sidebyside=true, nobeforeafter,before=\vfil,after=\vfil,colupper=blue,sidebyside align=top, lefthand width=.3\textwidth,
    opacityframe=0,opacityback=.3,opacitybacktitle=1, opacitytext=1,
    segmentation style={black!55,solid,opacity=0,line width=3pt},
    title=#1
    ]
    \begin{tcolorbox}[colback=red!05,colframe=red!25,sidebyside align=top,
    width=\textwidth,nobeforeafter]#2\end{tcolorbox}%
    \tcblower
    \sffamily
    \begin{tcolorbox}[colback=blue!05,colframe=blue!10,width=\textwidth,nobeforeafter]
    #3
    \end{tcolorbox}
    \end{tcolorbox}
    }
    \def\@without#1#2{
    \begin{tcolorbox}[enhanced,colback=white!15,colframe=white,fonttitle=\bfseries,sidebyside=true, nobeforeafter,before=\vfil,after=\vfil,colupper=blue,sidebyside align=top, lefthand width=.3\textwidth,
    opacityframe=0,opacityback=0,opacitybacktitle=0, opacitytext=1,
    segmentation style={black!55,solid,opacity=0,line width=3pt}
    ]
    
    \begin{tcolorbox}[colback=red!05,colframe=red!25,sidebyside align=top,
    width=\textwidth,nobeforeafter]#1\end{tcolorbox}%
    \tcblower
    \sffamily
    \begin{tcolorbox}[colback=blue!05,colframe=blue!10,width=\textwidth,nobeforeafter]
    #2
    \end{tcolorbox}
    \end{tcolorbox}
    }
    \makeatother

    \parindent=0pt
    
    \begin{document}
    \maketitle
    \SetBgContents{\rule[0em]{4pt}{\textheight}}
    \cornell[The Rise of Cultural Nationalism]{What led to the growth of cultural nationalism in the U.S. and what were its effects?}{}
    \cornell{How did American life in the early nineteenth century reflect Republican ideals?}{\textbf{Educational opportunities grew, culture began to be freed of European influences, and religion adjusted to Enlightenment ideals.}}
    \cornell{What were the main patterns of white educational development in nineteenth century America?}{\begin{itemize}
        \item Jefferson, aligning with Republican ideals of enlightened populace, began "crusade against ignorance," hoping to establish nation-wide public schools with education free to all white male citizens
        \item Some states endorsed in early years (MA law required each town to support school), but many others ignored enforcement
        \begin{itemize}
            \item VA ignored Jefferson's demand for basic education for all and advanced education for the gifted
            \item Even by 1815, no state had a complete public school system 
        \end{itemize}
        \item Schooling became far more prominent in private institutions (who could afford them)
        \begin{itemize}
            \item South and mid-Atlantic states saw religious-run schools
            \item New England schools generally secular, modeled on Phillips family at Andover, MA and in Exeter, NH
            \begin{itemize}
                \item Often aristocratic, seeking to train students to become elite
                \item Few schools were open to poor
            \end{itemize}
            \item Most private secondary schools accepted only males, but some growth in female opportunities
            \begin{itemize}
                \item Female illiteracy during Revolution exceeded 50\%; ideal of new mother to train children in liberty led to call for education to train mothers
                \item In 1770s, network of female academies emerged (normally for affluent daughters)
                \item MA required in 1789 that schools serve females and males (many states followed)
            \end{itemize}
            \item Most men assumed that female education intended only to create better wives/mothers -> no growth in college education
                \begin{itemize}
                    \item Judith Sargent Murray, in 1784, wrote essay defending rights to education with viewpoint opposing that of most men: argued equal potential/intellect, need for all people to earn their own living
                    \begin{itemize}
                        \item Little support during lifetime; far more successful posthumously
                    \end{itemize}
                \end{itemize}
        \end{itemize}
    \end{itemize}
    \textbf{For whites, Republicans (particularly Jefferson) began to push for an advanced educational system required in all towns. However, this goal was not truly met, with the majority of successful schools private institutions focused on teaching the children of aristocrats. Additionally, women began to receive more educational opportunties with many states requiring that public schools serve both sexes and some women pushing for college education.}}
    \cornell{What were the educational opportunities available to other races?}{\textbf{Because Jefferson and his followers believed natives had potential to become on-par with whites given education (unlike African Americans),  mission schools began to spread throughout the tribes. African-Americans, however, generally acquired little education due to fear of an uprising and the widespread belief of innate inferiority; however, some managed to teach themselves.}}
    \cornell{What was the state of higher education?}{\begin{itemize}
        \item Higher education far less widely available, although numbers of colleges continued to rise slowly 
        \item Even state-established colleges relied on private contributions/fees
        \item Accessible to fewer than one in one thousand white men, required a prosperous family
        \item Education was extremely narrow, with classics and theology; clergy was only profession requiring college education
        \begin{itemize}
            \item Some law schools emerged in College of William and Mary (VA) and Columbia (NY) pre-1800; most lawyers trained through apprenticeships
        \end{itemize}
    \end{itemize}
    \textbf{Higher education was not very widely available, with only the most wealthy, specialized men able to access it due to the high fees to maintain. Furthermore, most curricula were very narrow in scope, applying only to a few specific professions (like the clergy).}}
    \cornell{How did medicine and science develop in the colonies?}{\begin{itemize}
        \item First medical school emerged in University of Philadelphia in eighteenth century, a time when most doctors learned through apprenticeships
        \begin{itemize}
            \item Faced great challenges from long-held medical prejudices
            \item Municipal authorities took significant time to respond to epidemics, only gradually began to listen to knowledgeable doctors like Benjamin Rush
        \end{itemize}
        \item Even leading advocates of scientific medicine continued to follow incorrect practices
        \begin{itemize}
            \item Even Rush followed practice of bleeding/purging
            \item Medical communities used reason of "scientific method" to expand to non-traditional domains (like childbirths -> fewer women's opportunities due to elimination of midwives, higher cost of physicians)
        \end{itemize}
        \item In all, promotion of education saw positions of elites bolstered against Republican ideals
    \end{itemize}
    \textbf{Medicine and science developed slowly despite the emergence of a medical school in Philadelphia. Bad practices continued to be employed and even the leading advocates of scientific medicine (like Rush) utilized poor techniques.}}
    \cornell{What were the significant cultural developments in American society?}{
        Despite the Republican push against Federalist political/economic centralization, most pushed for an alternate form of nationalism: cultural independence.
        \begin{itemize}
            \item Sought American literary/artistic culture rivaling the best of Europe (like in "Poem on the Rising Glory of America")
            \item Schoolbooks saw growing nationalism
            \begin{itemize}
                \item MA geographer Morse created \textit{Geography Made Easy}, demanding that nation use its own textbooks to prevent British influence
                \item Noah Webster, to create distinctive American culture, encouraged simplified spelling system
                \begin{itemize}
                    \item Removed "u" from many words (like honour) in \textit{American Spelling Book}, best-selling American book behind Bible
                    \item Work enlarged to become greater dictionary
                \end{itemize}
            \end{itemize}
            \item Those seeking national literary life faced obstacles
            \begin{itemize}
                \item Challenging to publish work: most printers favored English works (no royalties), and most magazines used British periodicals
                \begin{itemize}
                    \item Only those authors willing to pay a cost could compete for public attention
                \end{itemize}
                \item Many authors sought to create native literature to push American ideals, including Barlow and Brockden Brown
                \begin{itemize}
                    \item Barlow pushed for ideas of glory to be implanted to remove false prejudices
                    \item Brown intrigued by newly popular idea of novel but with distinct American themes
                    \begin{itemize}
                        \item Obsession with originality -> lacked large popular following
                    \end{itemize}
                \end{itemize}
                \item Most successful author: Washington Irving, producing satirical histories of early American life, New World society
                \begin{itemize}
                    \item Wrote stories about adventures of early Americans, with work still read today 
                \end{itemize}
            \end{itemize}
            \item Most successful literature: those glorifying nation's past
            \begin{itemize}
                \item Mercy Otis Warren produced \textit{History of the Revolution}
                \item Mason Weems created \textit{Life of Washington}, with no concern for historical accuracy (only nationalism)
            \end{itemize}
        \end{itemize}
        \textbf{Many Americans pushed for a distinctive American culture reflected in literary works and pieces of art. Schoolbooks (like Morse's \textit{Geography Made Easy} and Webster's \textit{Early Spelling Book}) saw the earliest shift to American styles to root out English influence, but literature soon followed, with the most successful authors focusing on the American past with a satirical twist.}}
        \cornell{What were the religious changes from the American Revolution?}{\begin{itemize}
            \item By detaching church from government and pushing liberty and reason, Revolution weakened traditional religion
            \begin{itemize}
                \item Traditionalists began to fear "rational" theologies focusing on scientific developments
                \item Many Americans began to follow deism (including Jefferson, Franklin), accepting God as a remote being without direct involvement
                \begin{itemize}
                    \item Denounced superstitions, with Thomas Paine producing \textit{The Age of Reason} against Christianity 
                \end{itemize}
                \item Ideas of universalism/unitarianism emerged as dissenting views in New England churches 
                \begin{itemize}
                    \item Argued that Jesus was a mere religious teacher, not the son of God; all could receive salvation
                    \item James Murray founded in Gloucester, MA in 1779; Unitarian Church in Boston
                \end{itemize}
            \end{itemize}
            \item Most Americans continued to hold strong religious beliefs
            \begin{itemize}
                \item Although formality of church had collapsed, most continued to harbor original feelings
                \item Deism, unitarianism, universalism seemed to be dominant because evangelists had lost organization
            \end{itemize}
        \end{itemize}
        \textbf{The American Revolution led to the detachment of church and state, causing evangelists to lose their organization and alternate faiths, including deism, pushing the lack of God's direct involvement, and unitarianism/universalism, arguing that all could receive salvation.}}
        \cornell{What led up the the Second Great Awakening?}{\begin{itemize}
            \item Multiple denominations fought rationalist revival 
        \end{itemize}}
        \cornell[Stirrings of Industrialism]{What led to the industrialization of America?}{\textbf{Many key American inventors ushered in the American Industrial Revolution, like Eli Whitney, who created the cotton gin. Transportation, too, was revolutionized: steamships became prominent with a new engine and turnpikes began to grow in prominence. Despite this, American cities still had a long way to come to rival Europe's largest cities.}}
        \cornell{What was the technological state of America in the early nineteenth century?}{\begin{itemize}
            \item Many advances imported from England despite British attempts to prevent export of machinery, skilled mechanics 
            \begin{itemize}
                \item Samuel Slater travelled from England despite emigration restrictions, built spinning mill for Quaker merchant in RI - first modern factory
            \end{itemize}
            \item America also had numerous inventors
            \begin{itemize}
                \item Oliver Evans (DE) developed flour mill, card-making machine; improved existing steam engine; produced mechanical enginering textbook 
                \begin{itemize}
                    \item Flour mill required only two men to operate
                \end{itemize}
                \item Eli Whitney (MA) even more influential: revolutionized cotton production/weapons
                \begin{itemize}
                    \item English textile industry -> great demand for cotton which could not be met by southern planters due to challenge of separating seeds from fiber (only northern crops were easy to separate)
                    \item Due to experience of working on Georgia plantation, created cotton gin to easily/efficiently perform
                    \begin{itemize}
                        \item Toothed rollers pulled fibers between grating, catching seeds
                        \item One operator could complete tasks within hours which previously required entire day to complete
                        \item Cotton spread beyond South with eightfold production increase; slavery began to grow 
                        \item In north, cotton gin had effect of promoting textile industry, with most northern plantation owners focusing on manufacturing textiles -> greater regional split
                    \end{itemize}
                    \item Revolutionized warfare with machine to systematically create guns by dividing tasks between workers
                    \begin{itemize}
                        \item Idea spread beyond to other industries
                    \end{itemize}
                \end{itemize}
            \end{itemize}
        \end{itemize}
        \textbf{Although a large part of American technology was imported from England, American inventors like Oliver Evans, who developed the flour mill and a card-making machine, and Eli Whitney, who produced the cotton gin, which revolutionized the south, and developed an assembly-line method of producing weapons.}}
        \cornell{What were the major developments in transportation?}{\begin{itemize}
            \item Industrialization requires efficient system of transport; U.S. lacked early system
            \item Small American market solved by customers overseas (but affected by Congress' reduced tariffs), but also by domestic growth, which was accelerated by a war with Europe
            \item Even by 1793, America had merchant marine, foreign trade rivaling all countries but England; rapid growth in number of vessels
            \item New markets emerged locally with trade between states: emerged as a result of steam power
            \begin{itemize}
                \item Oliver Evans' high-pressure engine (more efficient than Watt's) improved feasability of steam
                \item Oliver Fulton/Livingston perfected steamboat design with \textit{Clermont}, using English-built engine
                \begin{itemize}
                    \item Sailed along Hudson in 1807, demonstrating potential 
                    \item Design quickly introduced to West by Livingston's partner
                \end{itemize}
            \end{itemize}
            \item Turnpike era had begun: toll roads with crushed rock between states connected distant towns (only those where construction costs were low -> few roads over mountains into interior)
        \end{itemize}
        \textbf{The issue of the limited market was addressed over water by finding overseas customers and by a growing trade connection between states over water. This was assisted by the development of steampower. The turnpike era also soon began with the building of roads.}}
        \cornell{What marked the growth of early American cities?}{\begin{itemize}
            \item America remained primarily agricultural: only 3\% of non-natives lived in towns (population > 8000) by 1800; 10\% west of Appalachians; largest cities were nowhere near London/Paris
            \item Larger cities had begun to rival secondary European cities (like Philadelphia, NY, Baltimore, Boston, Charleston)
            \item Urban lifestyle very different: more elegant, diversions for enjoyment (like music, dancing, theatre, horce racing)
        \end{itemize}
        \textbf{Although even the largest cities paled in comparison to Europe's best, they began to rival Europe's secondary cities and develop their own distinctive culture, with a focus on elegance and diversions from everyday life.}}
        \cornell[Jefferson the President]{What changes did Jefferson implement while in office?}{\textbf{Jefferson ruled from the small town of Washington closely with the people, presenting himself as ordinary and against aristocratic customs. He pushed for economic reduction, cutting the national debt and reducing spending through military measures. However, a great judiciary conflict emerged between the Federalists and Jefferson's Republicans, but the Federalists ultimately resisted Jefferson's attempts, expanding the judiciary influence.}}
        \cornell{What was Jefferson's disposition when he first entered office?}{\textbf{Although Jefferson believed his victory was truly a revolution, he remained restrained in addressing the disparate parties, focusing on the similarities of all Americans to calm the various factions.}}
        \cornell{What were the key characteristics of Washington when Jefferson came to power?}{\begin{itemize}
            \item Created by Pierre L'Enfant whith grand aspirations, Capitol building remained uncompleted
            \item Many Americans were confident that Washington had potential to develop into Paris of U.S.
            \item Washington remained provincial village throughout nineteenth century: never approached NY/Philadelphia
            \begin{itemize}
                \item Congress-members viewed Washington as place for meetings, not to live; most lived in Capitol boardinghouses
                \item Many quickly departed if state legislature offered better position
            \end{itemize}
        \end{itemize}
        \textbf{Washington remained a small town throughout the nineteenth century despite L'Enfant's aspirations: most members of government viewed it as a place of work, not a true home.}}
        \cornell{How did Jefferson's policies reflect his close connection to the American populace?}{\begin{itemize}
            \item Despite Jefferson's status as wealthy planter with over 100 slaves, disparaged pretensiousness and behaved as a normal citizen without regard for customs; rarely dressed up
            \item Powerful prose, creativity impressed many (unlike Adams), pragmatic politician
            \begin{itemize}
                \item Despite demystification of elite government, acted as party leader favoring Republicans, giving secret directions to build network of influence
                \item Won for reelection in 1804 over Charles Pinckney, increasing Republican house majority
            \end{itemize}
        \end{itemize}
        \textbf{Jefferson behaved as a regular citizen with little regard for traditional customs; however, he was revered for his inspiring prose, creativity, overwhelming knowledge, and ability as a politician. He heavily favored the Republican party, often giving them special advantages.}}
        \cornell{How did the Republican party change the American economy?}{\begin{itemize}
            \item Jefferson felt previous administrations had been overly extravagant due to Hamilton's public debt, whiskey tax
            \begin{itemize}
                \item Abolished trend, convincing Congress to abolish internal taxes -> only revenue source were western lands/customs duties
                \item Secretary of Treasury cut spending significantly, cutting national debt in half
                \item Jefferson scaled down infantry, navy due to fear of rebellion while helping to establish West Point Military Academy 
            \end{itemize}
            \item Economic and military challenge appeared in Mediterranean with Barbary states of North America demanding money from all ships sailing through (even England paid)
            \begin{itemize}
                \item Jefferson agreed to treaty, reluctant to continue with appeasement
                \item Leader of Tripoli cut down American flag at consulate due to insufficient pay -> American fleet built up, leading to cessation of tribute but ransom money for POWs
            \end{itemize}
        \end{itemize}
        \textbf{Jefferson, convinced that the Federalist administrations had implemented overly extravagant economic policies, cut spending significantly and cut internal taxes, causing the national debt to halve. However, he poured resources into the development of West Point, which provided well-trained officers against the Tripoli pirates.}}
        \cornell{What marked the conflicts between the Federalist judiciary branch and the Republican executive and legislative branches?}{\begin{itemize}
            \item Jefferson's first step was to repeal Judiciary Act of 1801, eliminating Adams' last minute appointments
            \item Federalists pushed that Supreme Court had right to nullify congressional acts; up to 1803, had only ever enforced validity of existing laws
            \begin{itemize}
                \item In \textit{Marbury v. Madison}, where Marbury, one of Adams' last minute appointments, did not receive commission of approval before Adams departed and James Madison refused to give to him
                \begin{itemize}
                    \item Marbury appealed to Supreme Court -> ruling that Madison was legally obligated but that they lacked authority to enforce
                \end{itemize}
                \item Decision to relinquish right to deliver commissions paved way for much larger right: that to nullify Congress' acts 
            \end{itemize}
            \item John Marshall was chief justice, Federalist and VA lawyer under Adams, appointed at last minute; immediately made judiciary branch coequal with executive and legislature
            \item Jefferson recognized direct threat -> prepared to take control of judiciary
            \begin{itemize}
                \item Urged Congress to impeach judges, first removing a district judge then seeing House impeach an injudicious yet noncriminal Supreme Court justice Samuel Chase (but Senate votes were not sufficient)
                \item Judiciary survived and precedent was set that partisan differences could not lead to impeachment 
            \end{itemize}
        \end{itemize}
        \textbf{Jefferson immediately set to work on reducing the power of the Federalist judiciary branch, but \textit{Marbury v. Madison} showed that they had the right to nullify acts of Congress. Under John Marshall, the judiciary branch was developed greatly, and survived most of Jefferson's major hits, including a dramatic impeachment attempt.}}
    \end{document}