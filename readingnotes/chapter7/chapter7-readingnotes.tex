\documentclass[a4paper]{article}
    \usepackage[T1]{fontenc}
    \usepackage{tcolorbox}
    \usepackage{amsmath}
    \tcbuselibrary{skins}
    
    \title{
    \vspace{-3em}
    \begin{tcolorbox}
    \Huge\sffamily \begin{center} AP US History  \\
    \LARGE Chapter 7 - The Jeffersonian Era \\
    \Large Finn Frankis \end{center} 
    \end{tcolorbox}
    \vspace{-3em}
    }
    \date{}
    \author{}
    
    \usepackage{background}
    \SetBgScale{1}
    \SetBgAngle{0}
    \SetBgColor{red}
    \SetBgContents{\rule[0em]{4pt}{\textheight}}
    \SetBgHshift{-2.3cm}
    \SetBgVshift{0cm}
    \usepackage[margin=2cm]{geometry} 
    
    \makeatletter
    \def\cornell{\@ifnextchar[{\@with}{\@without}}
    \def\@with[#1]#2#3{
    \begin{tcolorbox}[enhanced,colback=gray,colframe=black,fonttitle=\large\bfseries\sffamily,sidebyside=true, nobeforeafter,before=\vfil,after=\vfil,colupper=blue,sidebyside align=top, lefthand width=.3\textwidth,
    opacityframe=0,opacityback=.3,opacitybacktitle=1, opacitytext=1,
    segmentation style={black!55,solid,opacity=0,line width=3pt},
    title=#1
    ]
    \begin{tcolorbox}[colback=red!05,colframe=red!25,sidebyside align=top,
    width=\textwidth,nobeforeafter]#2\end{tcolorbox}%
    \tcblower
    \sffamily
    \begin{tcolorbox}[colback=blue!05,colframe=blue!10,width=\textwidth,nobeforeafter]
    #3
    \end{tcolorbox}
    \end{tcolorbox}
    }
    \def\@without#1#2{
    \begin{tcolorbox}[enhanced,colback=white!15,colframe=white,fonttitle=\bfseries,sidebyside=true, nobeforeafter,before=\vfil,after=\vfil,colupper=blue,sidebyside align=top, lefthand width=.3\textwidth,
    opacityframe=0,opacityback=0,opacitybacktitle=0, opacitytext=1,
    segmentation style={black!55,solid,opacity=0,line width=3pt}
    ]
    
    \begin{tcolorbox}[colback=red!05,colframe=red!25,sidebyside align=top,
    width=\textwidth,nobeforeafter]#1\end{tcolorbox}%
    \tcblower
    \sffamily
    \begin{tcolorbox}[colback=blue!05,colframe=blue!10,width=\textwidth,nobeforeafter]
    #2
    \end{tcolorbox}
    \end{tcolorbox}
    }
    \makeatother

    \parindent=0pt
    
    \begin{document}
    \maketitle
    \SetBgContents{\rule[0em]{4pt}{\textheight}}
    \cornell[The Rise of Cultural Nationalism]{What led to the growth of cultural nationalism in the U.S. and what were its effects?}{\textbf{Educational opportunities grew with an increase in public schools (but most were private), more women being offered schooling, and specialized professions like medicine and law seeing specialized training beyond apprenticeships, culture began to be freed of European influences with powerful literary advancements and artistic strides, and religion adjusted to eliminate skepticism with the Second Great Awakening, pushing for the readmittance of God and Christ, which had broad affect on whites, natives, and African Americans.}}
    \cornell{What were the educational changes implemented by the Republicans?}{\begin{itemize}
        \item Jefferson, aligning with Republican ideals of enlightened populace, began "crusade against ignorance," hoping to establish nation-wide public schools with education free to all white male citizens
        \item Some states endorsed in early years (MA law required each town to support school), but many others ignored enforcement
        \begin{itemize}
            \item VA ignored Jefferson's demand for basic education for all and advanced education for the gifted
            \item Even by 1815, no state had a complete public school system 
        \end{itemize}
        \item Schooling became far more prominent in private institutions (who could afford them)
        \begin{itemize}
            \item South and mid-Atlantic states saw religious-run schools
            \item New England schools generally secular, modeled on Phillips family at Andover, MA and in Exeter, NH
            \begin{itemize}
                \item Often aristocratic, seeking to train students to become elite
                \item Few schools were open to poor
            \end{itemize}
            \item Most private secondary schools accepted only males, but some growth in female opportunities
            \begin{itemize}
                \item Female illiteracy during Revolution exceeded 50\%; ideal of new mother to train children in liberty led to call for education to train mothers
                \item In 1770s, network of female academies emerged (normally for affluent daughters)
                \item MA required in 1789 that schools serve females and males (many states followed)
            \end{itemize}
            \item Most men assumed that female education intended only to create better wives/mothers -> no growth in college education
                \begin{itemize}
                    \item Judith Sargent Murray, in 1784, wrote essay defending rights to education with viewpoint opposing that of most men: argued equal potential/intellect, need for all people to earn their own living
                    \begin{itemize}
                        \item Little support during lifetime; far more successful posthumously
                    \end{itemize}
                \end{itemize}
        \end{itemize}
    \end{itemize}
    \textbf{For whites, Republicans (particularly Jefferson) began to push for an advanced educational system required in all towns. However, this goal was not truly met, with the majority of successful schools private institutions focused on teaching the children of aristocrats. Additionally, women began to receive more educational opportunties with many states requiring that public schools serve both sexes and some women pushing for college education.}}
    \cornell{What were the educational opportunities available to other races?}{\textbf{Because Jefferson and his followers believed natives had potential to become on-par with whites given education (unlike African Americans),  mission schools began to spread throughout the tribes. African-Americans, however, generally acquired little education due to fear of an uprising and the widespread belief of innate inferiority; however, some managed to teach themselves.}}
    \cornell{What was the state of higher education?}{\begin{itemize}
        \item Higher education far less widely available, although numbers of colleges continued to rise slowly 
        \item Even state-established colleges relied on private contributions/fees
        \item Accessible to fewer than one in one thousand white men, required a prosperous family
        \item Education was extremely narrow, with classics and theology; clergy was only profession requiring college education
        \begin{itemize}
            \item Some law schools emerged in College of William and Mary (VA) and Columbia (NY) pre-1800; most lawyers trained through apprenticeships
        \end{itemize}
    \end{itemize}
    \textbf{Higher education was not very widely available, with only the most wealthy, specialized men able to access it due to the high fees to maintain. Furthermore, most curricula were very narrow in scope, applying only to a few specific professions (like the clergy).}}
    \cornell{How did medicine and science develop in the colonies?}{\begin{itemize}
        \item First medical school emerged in University of Philadelphia in eighteenth century, a time when most doctors learned through apprenticeships
        \begin{itemize}
            \item Faced great challenges from long-held medical prejudices
            \item Municipal authorities took significant time to respond to epidemics, only gradually began to listen to knowledgeable doctors like Benjamin Rush
        \end{itemize}
        \item Even leading advocates of scientific medicine continued to follow incorrect practices
        \begin{itemize}
            \item Even Rush followed practice of bleeding/purging
            \item Medical communities used reason of "scientific method" to expand to non-traditional domains (like childbirths -> fewer women's opportunities due to elimination of midwives, higher cost of physicians)
        \end{itemize}
        \item In all, promotion of education saw positions of elites bolstered against Republican ideals
    \end{itemize}
    \textbf{Medicine and science developed slowly despite the emergence of a medical school in Philadelphia. Bad practices continued to be employed and even the leading advocates of scientific medicine (like Rush) utilized poor techniques.}}
    \cornell{What were the significant cultural developments in American society?}{
        Despite the Republican push against Federalist political/economic centralization, most pushed for an alternate form of nationalism: cultural independence.
        \begin{itemize}
            \item Sought American literary/artistic culture rivaling the best of Europe (like in "Poem on the Rising Glory of America")
            \item Schoolbooks saw growing nationalism
            \begin{itemize}
                \item MA geographer Morse created \textit{Geography Made Easy}, demanding that nation use its own textbooks to prevent British influence
                \item Noah Webster, to create distinctive American culture, encouraged simplified spelling system
                \begin{itemize}
                    \item Removed "u" from many words (like honour) in \textit{American Spelling Book}, best-selling American book behind Bible
                    \item Work enlarged to become greater dictionary
                \end{itemize}
            \end{itemize}
            \item Those seeking national literary life faced obstacles
            \begin{itemize}
                \item Challenging to publish work: most printers favored English works (no royalties), and most magazines used British periodicals
                \begin{itemize}
                    \item Only those authors willing to pay a cost could compete for public attention
                \end{itemize}
                \item Many authors sought to create native literature to push American ideals, including Barlow and Brockden Brown
                \begin{itemize}
                    \item Barlow pushed for ideas of glory to be implanted to remove false prejudices
                    \item Brown intrigued by newly popular idea of novel but with distinct American themes
                    \begin{itemize}
                        \item Obsession with originality -> lacked large popular following
                    \end{itemize}
                \end{itemize}
                \item Most successful author: Washington Irving, producing satirical histories of early American life, New World society
                \begin{itemize}
                    \item Wrote stories about adventures of early Americans, with work still read today 
                \end{itemize}
            \end{itemize}
            \item Most successful literature: those glorifying nation's past
            \begin{itemize}
                \item Mercy Otis Warren produced \textit{History of the Revolution}
                \item Mason Weems created \textit{Life of Washington}, with no concern for historical accuracy (only nationalism)
            \end{itemize}
        \end{itemize}
        \textbf{Many Americans pushed for a distinctive American culture reflected in literary works and pieces of art. Schoolbooks (like Morse's \textit{Geography Made Easy} and Webster's \textit{Early Spelling Book}) saw the earliest shift to American styles to root out English influence, but literature soon followed, with the most successful authors focusing on the American past with a satirical twist.}}
        \cornell{What were the religious changes from the American Revolution?}{\begin{itemize}
            \item By detaching church from government and pushing liberty and reason, Revolution weakened traditional religion
            \begin{itemize}
                \item Traditionalists began to fear "rational" theologies focusing on scientific developments
                \item Many Americans began to follow deism (including Jefferson, Franklin), accepting God as a remote being without direct involvement
                \begin{itemize}
                    \item Denounced superstitions, with Thomas Paine producing \textit{The Age of Reason} against Christianity 
                \end{itemize}
                \item Ideas of universalism/unitarianism emerged as dissenting views in New England churches 
                \begin{itemize}
                    \item Argued that Jesus was a mere religious teacher, not the son of God; all could receive salvation
                    \item James Murray founded in Gloucester, MA in 1779; Unitarian Church in Boston
                \end{itemize}
            \end{itemize}
            \item Most Americans continued to hold strong religious beliefs
            \begin{itemize}
                \item Although formality of church had collapsed, most continued to harbor original feelings
                \item Deism, unitarianism, universalism seemed to be dominant because evangelists had lost organization
            \end{itemize}
        \end{itemize}
        \textbf{The American Revolution led to the detachment of church and state, causing evangelists to lose their organization and alternate faiths, including deism, pushing the lack of God's direct involvement, and unitarianism/universalism, arguing that all could receive salvation.}}
        \cornell{What led up the the Second Great Awakening?}{\begin{itemize}
            \item Multiple denominations fought rationalist revival, including Presbyterians on western fringe and church conservatives denouncing New Light dissenters, Methodism sending preachers throughout, Baptists finding southern following
            \item Combination of fervor inspired by all denominations saw great surge of evangelism, rapidly spreading throughout the country
            \begin{itemize}
                \item Large proportion mobilized, most embracing Methodism, Baptists, Presbyterianism
                \item Cane Ridge, KY: evangelical ministers led large camp meeting (> 25,000 attendees) which represented larger evangelical fervor; tactics continued in following years by Methodists especially (like Peter Cartwright, national preacher with great frenzy)
            \end{itemize}
            \item Message of Second Great Awakening to return God/Christ into daily lives of all, rejecting rationslism 
            \begin{itemize}
                \item Many past beliefs (like predestination/unchanging destiny) were revoked to push for more active piety with emphasis on good deeds
            \end{itemize}
            \item Led to growth of other sects/denominations, establishing new sense of order, including counterparts to male-only churches for women
            \begin{itemize}
                \item Women began to outnumber men in many regions due to numerous adventurous departures to West; found religion to build lives 
                \item Religious beliefs reflected changes in economic roles, using religion to compensate for older women who lost work due to movement of industry to outside home
            \end{itemize}
        \end{itemize}
        \textbf{The Second Great Awakening emerged as a result of the disorganization of evangelical communities and the growth of religious skepticism. It hoped to return God and Christ to the daily lives of all people and reject rationalist beliefs, while also eliminating the past belief of predestination. The awakening also saw great participation by women, who began to take a greater religious role as a result of their reduced importance in economic structure.}}
        \cornell{How did the Great Awakening influence minority races and skeptics?}{\begin{itemize}
            \item African Americans embraced fervor, leading to growth of black preacher class within slave community; translated salvation into desire for egalitarian community for blacks
            \begin{itemize}
                \item Plans for a rebellion emerged but were quickly put down; represented risk posed by religion for slave uprisings
            \end{itemize}
            \item Deeply affected natives in a different way, blending Christian faith with tribal cultures
            \begin{itemize}
                \item Began with Delaware prophet Neolin mixing tribal culture with personal God involved in affairs while denouncing white encroachment
                \item Crises brought about by numerous native defeats -> new era of religious fervor, particularly Presbyterian/Baptist missionaries 
                \item Handsome Lake, Seneca prophet who had managed to transform life after years of alcoholism, claimed to have met Jesus and inspired natives to give up white customs of whiskey/gambling
                \begin{itemize}
                    \item Encouraged missionaries to become active within tribes, Iroquois men to abandon hunting roles for farming
                    \item Those who resisted (mostly women due to shift from farming to domestic roles) classified as witches, many killed
                \end{itemize}
            \end{itemize}
            \item "Freethinkers" who were skeptical toward religious revivals diminished in numbers
        \end{itemize}
        \textbf{Many African Americans embraced the Christian fervor, producing a black preacher class; this set the precedent for many future uprisings. Native Americans, too, due to prophets like Neolin and Handsome Lake, began to accept words of missionaries and blend their culture with that of white society. Finally, skeptics became a relatively small group after the awakening.}}
        \cornell[Stirrings of Industrialism]{What led to the industrialization of America?}{\textbf{Many key American inventors ushered in the American Industrial Revolution, like Eli Whitney, who created the cotton gin. Transportation, too, was revolutionized: steamships became prominent with a new engine and turnpikes began to grow in prominence. Despite this, American cities still had a long way to come to rival Europe's largest cities.}}
        \cornell{What was the technological state of America in the early nineteenth century?}{\begin{itemize}
            \item Many advances imported from England despite British attempts to prevent export of machinery, skilled mechanics 
            \begin{itemize}
                \item Samuel Slater travelled from England despite emigration restrictions, built spinning mill for Quaker merchant in RI - first modern factory
            \end{itemize}
            \item America also had numerous inventors
            \begin{itemize}
                \item Oliver Evans (DE) developed flour mill, card-making machine; improved existing steam engine; produced mechanical enginering textbook 
                \begin{itemize}
                    \item Flour mill required only two men to operate
                \end{itemize}
                \item Eli Whitney (MA) even more influential: revolutionized cotton production/weapons
                \begin{itemize}
                    \item English textile industry -> great demand for cotton which could not be met by southern planters due to challenge of separating seeds from fiber (only northern crops were easy to separate)
                    \item Due to experience of working on Georgia plantation, created cotton gin to easily/efficiently perform
                    \begin{itemize}
                        \item Toothed rollers pulled fibers between grating, catching seeds
                        \item One operator could complete tasks within hours which previously required entire day to complete
                        \item Cotton spread beyond South with eightfold production increase; slavery began to grow 
                        \item In north, cotton gin had effect of promoting textile industry, with most northern plantation owners focusing on manufacturing textiles -> greater regional split
                    \end{itemize}
                    \item Revolutionized warfare with machine to systematically create guns by dividing tasks between workers
                    \begin{itemize}
                        \item Idea spread beyond to other industries
                    \end{itemize}
                \end{itemize}
            \end{itemize}
        \end{itemize}
        \textbf{Although a large part of American technology was imported from England, American inventors like Oliver Evans, who developed the flour mill and a card-making machine, and Eli Whitney, who produced the cotton gin, which revolutionized the south, and developed an assembly-line method of producing weapons.}}
        \cornell{What were the major developments in transportation?}{\begin{itemize}
            \item Industrialization requires efficient system of transport; U.S. lacked early system
            \item Small American market solved by customers overseas (but affected by Congress' reduced tariffs), but also by domestic growth, which was accelerated by a war with Europe
            \item Even by 1793, America had merchant marine, foreign trade rivaling all countries but England; rapid growth in number of vessels
            \item New markets emerged locally with trade between states: emerged as a result of steam power
            \begin{itemize}
                \item Oliver Evans' high-pressure engine (more efficient than Watt's) improved feasability of steam
                \item Oliver Fulton/Livingston perfected steamboat design with \textit{Clermont}, using English-built engine
                \begin{itemize}
                    \item Sailed along Hudson in 1807, demonstrating potential 
                    \item Design quickly introduced to West by Livingston's partner
                \end{itemize}
            \end{itemize}
            \item Turnpike era had begun: toll roads with crushed rock between states connected distant towns (only those where construction costs were low -> few roads over mountains into interior)
        \end{itemize}
        \textbf{The issue of the limited market was addressed over water by finding overseas customers and by a growing trade connection between states over water. This was assisted by the development of steampower. The turnpike era also soon began with the building of roads.}}
        \cornell{What marked the growth of early American cities?}{\begin{itemize}
            \item America remained primarily agricultural: only 3\% of non-natives lived in towns (population > 8000) by 1800; 10\% west of Appalachians; largest cities were nowhere near London/Paris
            \item Larger cities had begun to rival secondary European cities (like Philadelphia, NY, Baltimore, Boston, Charleston)
            \item Urban lifestyle very different: more elegant, diversions for enjoyment (like music, dancing, theatre, horce racing)
        \end{itemize}
        \textbf{Although even the largest cities paled in comparison to Europe's best, they began to rival Europe's secondary cities and develop their own distinctive culture, with a focus on elegance and diversions from everyday life.}}
        \cornell[Jefferson the President]{What changes did Jefferson implement while in office?}{\textbf{Jefferson ruled from the small town of Washington closely with the people, presenting himself as ordinary and against aristocratic customs. He pushed for economic reduction, cutting the national debt and reducing spending through military measures. However, a great judiciary conflict emerged between the Federalists and Jefferson's Republicans, but the Federalists ultimately resisted Jefferson's attempts, expanding the judiciary influence.}}
        \cornell{What was Jefferson's disposition when he first entered office?}{\textbf{Although Jefferson believed his victory was truly a revolution, he remained restrained in addressing the disparate parties, focusing on the similarities of all Americans to calm the various factions.}}
        \cornell{What were the key characteristics of Washington when Jefferson came to power?}{\begin{itemize}
            \item Created by Pierre L'Enfant whith grand aspirations, Capitol building remained uncompleted
            \item Many Americans were confident that Washington had potential to develop into Paris of U.S.
            \item Washington remained provincial village throughout nineteenth century: never approached NY/Philadelphia
            \begin{itemize}
                \item Congress-members viewed Washington as place for meetings, not to live; most lived in Capitol boardinghouses
                \item Many quickly departed if state legislature offered better position
            \end{itemize}
        \end{itemize}
        \textbf{Washington remained a small town throughout the nineteenth century despite L'Enfant's aspirations: most members of government viewed it as a place of work, not a true home.}}
        \cornell{How did Jefferson's policies reflect his close connection to the American populace?}{\begin{itemize}
            \item Despite Jefferson's status as wealthy planter with over 100 slaves, disparaged pretensiousness and behaved as a normal citizen without regard for customs; rarely dressed up
            \item Powerful prose, creativity impressed many (unlike Adams), pragmatic politician
            \begin{itemize}
                \item Despite demystification of elite government, acted as party leader favoring Republicans, giving secret directions to build network of influence
                \item Won for reelection in 1804 over Charles Pinckney, increasing Republican house majority
            \end{itemize}
        \end{itemize}
        \textbf{Jefferson behaved as a regular citizen with little regard for traditional customs; however, he was revered for his inspiring prose, creativity, overwhelming knowledge, and ability as a politician. He heavily favored the Republican party, often giving them special advantages.}}
        \cornell{How did the Republican party change the American economy?}{\begin{itemize}
            \item Jefferson felt previous administrations had been overly extravagant due to Hamilton's public debt, whiskey tax
            \begin{itemize}
                \item Abolished trend, convincing Congress to abolish internal taxes -> only revenue source were western lands/customs duties
                \item Secretary of Treasury cut spending significantly, cutting national debt in half
                \item Jefferson scaled down infantry, navy due to fear of rebellion while helping to establish West Point Military Academy 
            \end{itemize}
            \item Economic and military challenge appeared in Mediterranean with Barbary states of North America demanding money from all ships sailing through (even England paid)
            \begin{itemize}
                \item Jefferson agreed to treaty, reluctant to continue with appeasement
                \item Leader of Tripoli cut down American flag at consulate due to insufficient pay -> American fleet built up, leading to cessation of tribute but ransom money for POWs
            \end{itemize}
        \end{itemize}
        \textbf{Jefferson, convinced that the Federalist administrations had implemented overly extravagant economic policies, cut spending significantly and cut internal taxes, causing the national debt to halve. However, he poured resources into the development of West Point, which provided well-trained officers against the Tripoli pirates.}}
        \cornell{What marked the conflicts between the Federalist judiciary branch and the Republican executive and legislative branches?}{\begin{itemize}
            \item Jefferson's first step was to repeal Judiciary Act of 1801, eliminating Adams' last minute appointments
            \item Federalists pushed that Supreme Court had right to nullify congressional acts; up to 1803, had only ever enforced validity of existing laws
            \begin{itemize}
                \item In \textit{Marbury v. Madison}, where Marbury, one of Adams' last minute appointments, did not receive commission of approval before Adams departed and James Madison refused to give to him
                \begin{itemize}
                    \item Marbury appealed to Supreme Court -> ruling that Madison was legally obligated but that they lacked authority to enforce
                \end{itemize}
                \item Decision to relinquish right to deliver commissions paved way for much larger right: that to nullify Congress' acts 
            \end{itemize}
            \item John Marshall was chief justice, Federalist and VA lawyer under Adams, appointed at last minute; immediately made judiciary branch coequal with executive and legislature
            \item Jefferson recognized direct threat -> prepared to take control of judiciary
            \begin{itemize}
                \item Urged Congress to impeach judges, first removing a district judge then seeing House impeach an injudicious yet noncriminal Supreme Court justice Samuel Chase (but Senate votes were not sufficient)
                \item Judiciary survived and precedent was set that partisan differences could not lead to impeachment 
            \end{itemize}
        \end{itemize}
        \textbf{Jefferson immediately set to work on reducing the power of the Federalist judiciary branch, but \textit{Marbury v. Madison} showed that they had the right to nullify acts of Congress. Under John Marshall, the judiciary branch was developed greatly, and survived most of Jefferson's major hits, including a dramatic impeachment attempt.}}
        \cornell[Doubling the National Domain]{How did Jefferson double the national domain?}{\textbf{Jefferson, through his agreement with France and Spain, purchased the large Louisiana Territory from Napoleon despite his initial flagrant support for the French Empire. The Louisiana Purchase further stimulated western exploration which had already been growing in popularity, like in Lewis and Clark's western expedition and Pike's southern expedition, both authorized by Jefferson. These new policies ultimately catalyzed a series of events which led to Hamilton's death and Burr's failed indictment, which represented the weakness of the central government over their larger territories.}}
        \cornell{What were Jefferson's policies toward France?}{\begin{itemize}
            \item Napoleon, naming himself emperor in 1804 (when Jefferson was reelected), had little in common with Jefferson, yet they were close allies for a time
            \item After failing to seize India from the British empire, Napoleon turned attention to New World, hoping to regain territory west of Mississippi (belonging to Spain)
            \begin{itemize}
                \item Close allies, French and Spain signed a secret treaty giving possession of Louisiana to France, which Napoleon hoped would be the base for a large empire
                \item Napoleon also hoped to build empire on West Indian islands already belonging to France despite slave tensions
                \begin{itemize}
                    \item Put down rebellion led by Toussaint L'Ouverture 
                \end{itemize}
            \end{itemize}
            \item Jefferson initially unaware of Napoleon's imperial desires, continuing previous policy of admiration for France
            \begin{itemize}
                \item Appointed prudent Livingston as minister to Paris
                \item Retracted Adams alliance with British against L'Ouverture, revealing fear of black revolutionary for potential slave uprisings
                \item Hesitant after hearing about Louisiana transfer; 1802: alarmed New Orleans went against Pinckney Treaty in forbidding temporary cargo deposit in city, closing lower Mississippi to Americans 
            \end{itemize}
            \item Westerners demanded reopening of southern Mississippi river
            \begin{itemize}
                \item Jefferson sent Livingston to negotiate purchase of New Orleans; Livingston asked independently for Louisiana entirely
                \item U.S. began to build up army with Congressional approval, pushing Napoleon to accept Louisiana Purchase (in part due to fear of army, also due to epidemic -> army collapsing, potential for war in Europe)
            \end{itemize}
        \end{itemize}
        \textbf{Jefferson, initially supporting the French due to the long-held Republican respect, supported Napoleon in colonizing the West Indies and appointing a very pro-French minister. Things soon took a turn, however, when Napoleon covertly received Louisiana from their Spanish allies and forced the Spanish to revoke Pinckney's Treaty by blocking cargo deposit. Jefferson had no choice but to request the purchase of New Orleans, which Livingston negotiated into the entirety of Louisiana.}}
        \cornell{What was the Louisiana Purchase?}{\begin{itemize}
            \item Despite lacking authorization from government, Livingston / Monroe (sent by Jefferson) signed treaty due to fear of withdrawal
            \begin{itemize}
                \item Required American payment of \$15 million to French government, exclusive trading priviliges to France in New Orleans, treatment of Louisiana residents as citizens of Union
                \item Borders defined those previously ruled over by France
            \end{itemize}
            \item President pleased with results but hesitant: believed that federal government lacked authority to sign (not provided in Constitution), but soon convinced that treaty-making powers authorized him to sign
            \item French assumed formal control from Spanish very briefly to officially turn over power to U.S. commissioner James Wilkinson
            \item Louisiana Territory treated as NW Territory: potential for territories to become states (first was LA in 1812)
        \end{itemize}
        \textbf{The Louisiana Purchase, despite the initial signers lacking authorization from the federal government, was generally well-received and quickly approved by Republican Congress. It gave a large block of territory called the "Louisiana Territory" to the Americans, which were administered as the Northwest Territory.}}
        \cornell{What marked American westward exploration?}{\begin{itemize}
            \item Jefferson planned expedition even before LA Purchase intended to cross to Pacific Ocean, gathering key information about geography and native trade; led by Meriwether Lewis, native war veteran, choosing William Clark as colleague
            \begin{itemize}
                \item Lewis and Clark began at Missouri River in 1804, guided by Shoshone woman Sacajawea, crossing Rocky Mountains and reaching Pacific coast by Autumn 1805; returned to St. Louis in 1806
            \end{itemize}
            \item In parallel, Jefferson sent Zebulon Montgomery Pike to lead expedition into upper Mississippi Valley, failing to climb peak in modern Colorado -> impression that American east was uninhabitable
        \end{itemize}
        \textbf{Jefferson authorized the western expeditions of Lewis and Clark, who traveled to the Pacific Coast and back, gathering key geographic facts, and Pike, who traveled through the upper Mississippi Valley and created the impression that the East was uninhabitable and treacherous.}}
        \cornell{What political controversies emerged as a result of the new land acquisition policies?}{\begin{itemize}
            \item MA Federalists realized that new territories would create new Republican states and further reduce their power -> "Essex Junto" hoped to secede from Union, seeking to gain support of NY/NJ
            \begin{itemize}
                \item Hamilton refused to involve NY, believing that dismantling empire would have no benefit 
                \item VP Burr, who had lost prestige after the 1800 election controversy, agreed to become Federalist candidate for NY governor
                \begin{itemize}
                    \item Hamilton accused Burr of treason, Burr lost election
                    \item Blamed loss on Hamilton, challenged to ultimate duel
                \end{itemize}
            \end{itemize} 
            \item Hamilton did not wish to be seen as cowardly -> accepted duel and died the following day
            \item Burr was aware of potential for murder indictment -> fled from NY to Southwest, allying with Wilkinson (LA Territory governor) and hoping to capture Mexico 
            \begin{itemize}
                \item Jefferson believed rumors that Burr hoped to create independent empire, Wilkinson turned against him -> Burr brought to Richmond for trial
                \item Marshall, despite Jefferson's attempts to dominate trial and win, saw little evidence and acquitted Burr 
            \end{itemize}
            \item Conspiracy represented larger issue of weak central government unable to control vast pieces of land
        \end{itemize}
        \textbf{After Burr allied with a small group of New England Federalists who sought to secede from the Union and failed to win Federalist governor of New York, he blamed his loss on Hamilton's harsh words about him and challenged him to a duel. Hamilton accepted and died the following day. Fearing murder charges, Burr fled to the Southwest and began to take on bolder aspirations to take over Mexico - rumors led Jefferson to indict him but to his ultimate acquittal.}}
    \end{document}