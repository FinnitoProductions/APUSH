\documentclass[a4paper]{article}
\usepackage{tcolorbox}
\usepackage{amsmath}
\tcbuselibrary{skins}

\title{
\vspace{-3em}
\begin{tcolorbox}
\Huge\sffamily \begin{center} Chapter 31  \mbox{} \\ \huge From the "Age of Limits" to the Age of Reagan \mbox{} \\
\LARGE Finn Frankis \mbox{} \\
\Large AP US History - July 25{$^\text{th}$}, 2018 \end{center}
\end{tcolorbox}
\vspace{-3em}
}
\date{}
\author{}

\usepackage{background}
\SetBgScale{1}
\SetBgAngle{0}
\SetBgColor{red}
\SetBgContents{\rule[0em]{4pt}{\textheight}}
\SetBgHshift{-2.3cm}
\SetBgVshift{0cm}
\usepackage[margin=2cm]{geometry}

\makeatletter
\def\cornell{\@ifnextchar[{\@with}{\@without}}
\def\@with[#1]#2#3{
\begin{tcolorbox}[enhanced,colback=gray,colframe=black,fonttitle=\large\bfseries\sffamily,sidebyside=true, nobeforeafter,before=\vfil,after=\vfil,colupper=blue,sidebyside align=top, lefthand width=.3\textwidth,
opacityframe=0,opacityback=.3,opacitybacktitle=1, opacitytext=1,
segmentation style={black!55,solid,opacity=0,line width=3pt},
title=#1
]
\begin{tcolorbox}[colback=red!05,colframe=red!25,sidebyside align=top,
width=\textwidth,nobeforeafter]#2\end{tcolorbox}%
\tcblower
\sffamily
\begin{tcolorbox}[colback=blue!05,colframe=blue!10,width=\textwidth,nobeforeafter]
#3
\end{tcolorbox}
\end{tcolorbox}
}
\def\@without#1#2{
\begin{tcolorbox}[enhanced,colback=white!15,colframe=white,fonttitle=\bfseries,sidebyside=true, nobeforeafter,before=\vfil,after=\vfil,colupper=blue,sidebyside align=top, lefthand width=.3\textwidth,
opacityframe=0,opacityback=0,opacitybacktitle=0, opacitytext=1,
segmentation style={black!55,solid,opacity=0,line width=3pt}
]

\begin{tcolorbox}[colback=red!05,colframe=red!25,sidebyside align=top,
width=\textwidth,nobeforeafter]#1\end{tcolorbox}%
\tcblower
\sffamily
\begin{tcolorbox}[colback=blue!05,colframe=blue!10,width=\textwidth,nobeforeafter]
#2
\end{tcolorbox}
\end{tcolorbox}
}
\makeatother

\parindent=0pt

\begin{document}
\maketitle
\SetBgContents{\rule[0em]{4pt}{\textheight}}

\cornell[Politics and Diplomacy after Watergate (pgs. 838 - 841)]{How did American presidents following Nixon attempt to recover from the disaster of Watergate?}{\textbf{Ford and Carter, the two presidents who served after Nixon, took different approaches in recovering from US economic and political turmoil: Ford relied more on the policies of his predecessor; Carter, on the other hand, seems to have been more strong-minded and  fought for his bold ideals.}}
\cornell{How did Gerald Ford attempt to recover the nation's prosperity in the aftermath of Watergate?}{
    \begin{itemize}
        \item Major setback emerged immediately due to poor decision to completely pardon Nixon
        \begin{itemize}
            \item Led many to suspect collusion between Nixon and Ford
            \item Immediate decline in popularity
        \end{itemize}
        \item Economic policies relatively unsuccessful
        \begin{itemize}
            \item Attempted to curb inflation by calling for voluntary efforts of people, rejecting idea of price/wage controls
            \item Struggled with recession intensified by energy crisis
            \begin{itemize}
                \item \textbf{Arab oil embargo} of 1973 led to extreme increase in price of oil
            \end{itemize}
        \end{itemize}
        \item Political policies often simply continuations of Nixon administration
        \begin{itemize}
            \item Signed \textbf{SALT II}, arms control accord desired by Nixon
            \item Secretary of state \textbf{Henry Kissinger} required Israel to return parts of Sinai to Egypt
            \item Heavily questioned by both right and left: faced challenge from conservative \textbf{Ronald Reagan} for party nomination
            \begin{itemize}
                \item Democrats united before \textbf{Jimmy Carter}, praised for candor, piety; beat Ford in narrow victory
            \end{itemize}
        \end{itemize}
    \end{itemize}
    \textbf{Gerald Ford attempted to recover from the damage done by the Nixon administration by making new economic and political strides but was ultimately unsuccessful due to his close ties with and similar strides to Nixon.}
}
\cornell{What changes in policy did Carter make following the Ford presidency and were they more successful?}{
    \begin{itemize}
        \item Known for extreme intelligence and bold promises; Congress passed few promised reforms
        \item Devoted to improvement of economy amidst recession through modified energy use
        \begin{itemize}
            \item Oil prices rose during final years of presidency; interest rates rose to highest in American history
            \item Gave \textbf{"malaise"} speech after 10 days at Camp David (presidential retreat) describing American \textbf{energy crisis} and potential solutions
            \begin{itemize}
                \item Criticized for blaming of American people for state of nation
            \end{itemize}
        \end{itemize}
        \item Focused on \textbf{human rights}, criticizing many other nations (including the Soviet Union) for violations
        \item Frequently dealt with more traditional concerns
        \begin{itemize}
            \item Returned \textbf{Panama Canal} to Panamanian government
            \item Greatest achievement was \textbf{peace treaty} between Egypt and Israel
            \begin{itemize}
                \item Encouraged dialogue between Egyptian president and Israeli PM at Camp David
                \begin{itemize}
                    \item Helped to mediate disputes 
                \end{itemize}
                \item Leaders later returned to sign \textbf{Camp David accords}
            \end{itemize}
            \item Tried to improve relationship with China, promoting Deng Xiaoping's overtures
            \item Completed SALT II w/ USSR started by Ford, limiting missiles and nuclear warheads for both nations
        \end{itemize}
        \item When Iranian people rebelled against US-promoted government and the shah fled to the US for health care, 53 hostages taken at American embassy
        \item Carter retaliated against USSR invasion of Afghanistan with Olympic withdrawal and cancellation of SALT II
        \item Carter finally fell out of popularity due to domestic economic troubles, international crises
    \end{itemize}
    \textbf{Democrat Jimmy Carter focused on energy use, human rights, and peace between disparate nations; he strongly stood by US traditional ideals and rebuked nations seeking to disrupt those ideals. In all, Carter seemed to have been more popular than Ford among the people:  he voted in for president rather than promoted by virtue of rank.}
}
\cornell[The Rise of the New American Right (pgs. 841 - 846)]{What is the new American Right and how did it influence American society?}{\textbf{The New American Right was the greater wealth of right wing politicians and their staunch opposition toward many liberal policies, including high taxes - it allowed Ronald Reagan to come to power.}}
\cornell{What is the Sunbelt and what was its political condition?}{\textbf{The Sunbelt was the region including the Southeast, Southwest, and California. It changed the political climate by fighting against governmental growth and regulations (often environmental ones like a reduced speed limit). In the late 1970s, it experienced the Sagebrush rebellion, a deliberate conservative opposition against regulation, criticizing the government for its large swathes of land. The most conservative communities were suburbs, which were isolated from diverse contact due to the relative homogeneity of the population.}}
\cornell{How did religion influence American politics in the 1970s?}{
\begin{itemize}
    \item America experienced a major religious revival in the 1970s
    \begin{itemize}
        \item Often materialized in cults and pseudo-faiths like \textbf{Scientology} or the People's Temple
    \end{itemize}
    \item Most significant: \textbf{evangelical Christians}, unified by the belief that all should be converted or "born again"
    \begin{itemize}
        \item Entire section of society, including newspapers, schools, radio stations
        \item Some interpreted as commitment to economic justice, others for world peace
        \item Others saw as duty to prevent social disorder, including feminism, lack of required religion in schools, or right to abortion
        \item Evangelism unified long disparate sects, including Mormons, Protestants, and Catholics
    \end{itemize}
\end{itemize}
\textbf{Religion was extremely influential in 1970s America, particularly evangelical Christianity, a growing religion encouraging conversion to all which slowly began to dominate large portions of society.}
}
\cornell{What is the American "new right"?}{
    \begin{itemize}
        \item The "New Right" was a diverse, powerful coalition originating in 1964 election; boomed in late '70s
        \begin{itemize}
            \item Goldwater campaign, which promoted fund-raising for conservatives, left conservatives better-funded than opponents
        \end{itemize}
        \item Right heavily promoted by conservative film actor Reagan, inspiring people with powerful species on freedom, private enterprise
        \begin{itemize}
            \item Took opportunity of Goldwater's defeat to rise up as governor of California, leader of conservative Republican Wing
        \end{itemize}
        \item Also promoted by presidency of Gerald Ford, which eliminated equilibrium between moderate and right wings of Republican party
        \begin{itemize}
            \item Rockefeller's position as VP offended conservatives
            \item Ford only secured role as party leader by dropping Rockefeller, taking advice from Reagan's allies
        \end{itemize}
    \end{itemize}
    \textbf{The "New Right" was the powerful and wealthy coalition of Republicans who staunchly opposed liberal policies.}
}
\cornell{What was the American tax revolt of 1978 and what caused it?}{
    \begin{itemize}
        \item Tax Revolt of 1978 was essential to success of New Right
        \begin{itemize}
            \item Began with Howard Jarvis' Proposition 13, questioning a referendum on increased property tax
            \item Tackled problem not by speaking against federal government, but instead against major, expensive programs like Medicare
        \end{itemize}
        \item Right separated issue of taxes from issue of programs requiring taxes
        \begin{itemize}
            \item Controversially limited government ability to launch new programs
            \item Generally left previous programs intact
        \end{itemize}
    \end{itemize}
    \textbf{The Tax Revolt of 1978 was the right wing's strong opposition to the paying of taxes as a way of garnering the majority vote. It was only successful because it united people in their opposition to paying taxes.}
}
\cornell{What was the result of the 1980 electoral campaign?}{
    \begin{itemize}
        \item Carter entered 1980 election in political trouble due to Iranian hostage crisis, barely able to secure party's nomination
        \item Reagan won election 51\% to 41\% for Carter, promoting tax revolt, freeing of hostages in Iran
        \begin{itemize}
            \item Despite relatively small percentage greater than Carter, won majority of Senate seats in almost all states (45)
            \item Republican Party earned majority of Senate seats for first time since 1952
            \item Inaugaration of Reagan: hostages immediately freed as Reagan released Iranian government's assets which had been frozen by Carter
        \end{itemize}
    \end{itemize}
}
\cornell[The "Reagan Revolution"]{What was Reagan's political impact and how did he change American society in the long term?}{}
\cornell{How did Reagan come to power?}{}
\cornell{What was Reagan's image while in the White House?}{}
\cornell{What caused the fiscal crisis in the mid-1980s?}{}
\cornell{What is supply-side economics?}{}
\cornell{What were Reagan's diplomatic connections and how did his doctrine influence other nations?}{}
\cornell{What was the result of the 1984 election? Why?}{}
\cornell[America and the Waning of the Cold War]{What led to the end of the Cold War? How did this change American society?}{}
\cornell{How did the Soviet Union fall?}{}
\cornell{What led Reagan's influence to wane in American politics?}{}
\cornell{How did George H.W. Bush come to power?}{}
\cornell{What factors led to the first Gulf War?}{}
\cornell{What was the result of the 1992s election? Why?}{}
\end{document}