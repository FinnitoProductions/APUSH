\documentclass[a4paper]{article}
\usepackage{tcolorbox}
\usepackage{amsmath}
\tcbuselibrary{skins}

\title{
    \vspace{-3em}
    \begin{tcolorbox}
    \Huge\sffamily \begin{center} AP US History  \\
    \LARGE Chapter 31 - From the "Age of Limits" to the Age of Reagan\\
    \Large Finn Frankis \end{center} 
    \end{tcolorbox}
    \vspace{-3em}
    }
    \date{}
    \author{}

\usepackage{background}
\SetBgScale{1}
\SetBgAngle{0}
\SetBgColor{red}
\SetBgContents{\rule[0em]{4pt}{\textheight}}
\SetBgHshift{-2.3cm}
\SetBgVshift{0cm}
\usepackage[margin=2cm]{geometry}

\makeatletter
\def\cornell{\@ifnextchar[{\@with}{\@without}}
\def\@with[#1]#2#3{
\begin{tcolorbox}[enhanced,colback=gray,colframe=black,fonttitle=\large\bfseries\sffamily,sidebyside=true, nobeforeafter,before=\vfil,after=\vfil,colupper=blue,sidebyside align=top, lefthand width=.3\textwidth,
opacityframe=0,opacityback=.3,opacitybacktitle=1, opacitytext=1,
segmentation style={black!55,solid,opacity=0,line width=3pt},
title=#1
]
\begin{tcolorbox}[colback=red!05,colframe=red!25,sidebyside align=top,
width=\textwidth,nobeforeafter]#2\end{tcolorbox}%
\tcblower
\sffamily
\begin{tcolorbox}[colback=blue!05,colframe=blue!10,width=\textwidth,nobeforeafter]
#3
\end{tcolorbox}
\end{tcolorbox}
}
\def\@without#1#2{
\begin{tcolorbox}[enhanced,colback=white!15,colframe=white,fonttitle=\bfseries,sidebyside=true, nobeforeafter,before=\vfil,after=\vfil,colupper=blue,sidebyside align=top, lefthand width=.3\textwidth,
opacityframe=0,opacityback=0,opacitybacktitle=0, opacitytext=1,
segmentation style={black!55,solid,opacity=0,line width=3pt}
]

\begin{tcolorbox}[colback=red!05,colframe=red!25,sidebyside align=top,
width=\textwidth,nobeforeafter]#1\end{tcolorbox}%
\tcblower
\sffamily
\begin{tcolorbox}[colback=blue!05,colframe=blue!10,width=\textwidth,nobeforeafter]
#2
\end{tcolorbox}
\end{tcolorbox}
}
\makeatother

\parindent=0pt

\begin{document}
\maketitle
\SetBgContents{\rule[0em]{4pt}{\textheight}}

\cornell[Politics and Diplomacy after Watergate (pgs. 838 - 841)]{How did American presidents following Nixon attempt to recover from the disaster of Watergate?}{\textbf{Ford and Carter, the two presidents who served after Nixon, took different approaches in recovering from US economic and political turmoil: Ford relied more on the policies of his predecessor; Carter, on the other hand, seems to have been more strong-minded and  fought for his bold ideals.}}
\cornell{How did Gerald Ford attempt to recover the nation's prosperity in the aftermath of Watergate?}{
    \begin{itemize}
        \item Major setback emerged immediately due to poor decision to completely pardon Nixon
        \begin{itemize}
            \item Led many to suspect collusion between Nixon and Ford
            \item Immediate decline in popularity
        \end{itemize}
        \item Economic policies relatively unsuccessful
        \begin{itemize}
            \item Attempted to curb inflation by calling for voluntary efforts of people, rejecting idea of price/wage controls
            \item Struggled with recession intensified by energy crisis
            \begin{itemize}
                \item \textbf{Arab oil embargo} of 1973 led to extreme increase in price of oil
            \end{itemize}
        \end{itemize}
        \item Political policies often simply continuations of Nixon administration
        \begin{itemize}
            \item Signed \textbf{SALT II}, arms control accord desired by Nixon
            \item Secretary of state \textbf{Henry Kissinger} required Israel to return parts of Sinai to Egypt
            \item Heavily questioned by both right and left: faced challenge from conservative \textbf{Ronald Reagan} for party nomination
            \begin{itemize}
                \item Democrats united before \textbf{Jimmy Carter}, praised for candor, piety; beat Ford in narrow victory
            \end{itemize}
        \end{itemize}
    \end{itemize}
    \textbf{Gerald Ford attempted to recover from the damage done by the Nixon administration by making new economic and political strides but was ultimately unsuccessful due to his close ties with and similar strides to Nixon.}
}
\cornell{What changes in policy did Carter make following the Ford presidency and were they more successful?}{
    \begin{itemize}
        \item Known for extreme intelligence and bold promises; Congress passed few promised reforms
        \item Devoted to improvement of economy amidst recession through modified energy use
        \begin{itemize}
            \item Oil prices rose during final years of presidency; interest rates rose to highest in American history
            \item Gave \textbf{"malaise"} speech after 10 days at Camp David (presidential retreat) describing American \textbf{energy crisis} and potential solutions
            \begin{itemize}
                \item Criticized for blaming of American people for state of nation
            \end{itemize}
        \end{itemize}
        \item Focused on \textbf{human rights}, criticizing many other nations (including the Soviet Union) for violations
        \item Frequently dealt with more traditional concerns
        \begin{itemize}
            \item Returned \textbf{Panama Canal} to Panamanian government
            \item Greatest achievement was \textbf{peace treaty} between Egypt and Israel
            \begin{itemize}
                \item Encouraged dialogue between Egyptian president and Israeli PM at Camp David
                \begin{itemize}
                    \item Helped to mediate disputes 
                \end{itemize}
                \item Leaders later returned to sign \textbf{Camp David accords}
            \end{itemize}
            \item Tried to improve relationship with China, promoting Deng Xiaoping's overtures
            \item Completed SALT II w/ USSR started by Ford, limiting missiles and nuclear warheads for both nations
        \end{itemize}
        \item When Iranian people rebelled against US-promoted government and the shah fled to the US for health care, 53 hostages taken at American embassy
        \item Carter retaliated against USSR invasion of Afghanistan with Olympic withdrawal and cancellation of SALT II
        \item Carter finally fell out of popularity due to domestic economic troubles, international crises
    \end{itemize}
    \textbf{Democrat Jimmy Carter focused on energy use, human rights, and peace between disparate nations; he strongly stood by US traditional ideals and rebuked nations seeking to disrupt those ideals. In all, Carter seemed to have been more popular than Ford among the people:  he voted in for president rather than promoted by virtue of rank.}
}

\cornell[The Rise of the New American Right (pgs. 841 - 846)]{What is the new American Right and how did it influence American society?}{\textbf{The New American Right was the greater wealth of right wing politicians and their staunch opposition toward many liberal policies, including high taxes - it allowed Ronald Reagan to come to power.}}
\cornell{What is the Sunbelt and what was its political condition?}{\textbf{The Sunbelt was the region including the Southeast, Southwest, and California. It changed the political climate by fighting against governmental growth and regulations (often environmental ones like a reduced speed limit). In the late 1970s, it experienced the Sagebrush rebellion, a deliberate conservative opposition against regulation, criticizing the government for its large swathes of land. The most conservative communities were suburbs, which were isolated from diverse contact due to the relative homogeneity of the population.}}
\cornell{How did religion influence American politics in the 1970s?}{
\begin{itemize}
    \item America experienced a major religious revival in the 1970s
    \begin{itemize}
        \item Often materialized in cults and pseudo-faiths like \textbf{Scientology} or the People's Temple
    \end{itemize}
    \item Most significant: \textbf{evangelical Christians}, unified by the belief that all should be converted or "born again"
    \begin{itemize}
        \item Entire section of society, including newspapers, schools, radio stations
        \item Some interpreted as commitment to economic justice, others for world peace
        \item Others saw as duty to prevent social disorder, including feminism, lack of required religion in schools, or right to abortion
        \item Evangelism unified long disparate sects, including Mormons, Protestants, and Catholics
    \end{itemize}
\end{itemize}
\textbf{Religion was extremely influential in 1970s America, particularly evangelical Christianity, a growing religion encouraging conversion to all which slowly began to dominate large portions of society.}
}
\cornell{What is the American "new right"?}{
    \begin{itemize}
        \item The "New Right" was a diverse, powerful coalition originating in 1964 election; boomed in late '70s
        \begin{itemize}
            \item Goldwater campaign, which promoted fund-raising for conservatives, left conservatives better-funded than opponents
        \end{itemize}
        \item Right heavily promoted by conservative film actor Reagan, inspiring people with powerful species on freedom, private enterprise
        \begin{itemize}
            \item Took opportunity of Goldwater's defeat to rise up as governor of California, leader of conservative Republican Wing
        \end{itemize}
        \item Also promoted by presidency of Gerald Ford, which eliminated equilibrium between moderate and right wings of Republican party
        \begin{itemize}
            \item Rockefeller's position as VP offended conservatives
            \item Ford only secured role as party leader by dropping Rockefeller, taking advice from Reagan's allies
        \end{itemize}
    \end{itemize}
    \textbf{The "New Right" was the powerful and wealthy coalition of Republicans who staunchly opposed liberal policies.}
}
\cornell{What was the American tax revolt of 1978 and what caused it?}{
    \begin{itemize}
        \item Tax Revolt of 1978 was essential to success of New Right
        \begin{itemize}
            \item Began with Howard Jarvis' Proposition 13, questioning a referendum on increased property tax
            \item Tackled problem not by speaking against federal government, but instead against major, expensive programs like Medicare
        \end{itemize}
        \item Right separated issue of taxes from issue of programs requiring taxes
        \begin{itemize}
            \item Controversially limited government ability to launch new programs
            \item Generally left previous programs intact
        \end{itemize}
    \end{itemize}
    \textbf{The Tax Revolt of 1978 was the right wing's strong opposition to the paying of taxes as a way of garnering the majority vote. It was only successful because it united people in their opposition to paying taxes.}
}
\cornell{What was the result of the 1980 electoral campaign?}{
    \begin{itemize}
        \item Carter entered 1980 election in political trouble due to Iranian hostage crisis, barely able to secure party's nomination
        \item Reagan won election 51\% to 41\% for Carter, promoting tax revolt, freeing of hostages in Iran
        \begin{itemize}
            \item Despite relatively small percentage greater than Carter, won majority of Senate seats in almost all states (45)
            \item Republican Party earned majority of Senate seats for first time since 1952
            \item Inaugaration of Reagan: hostages immediately freed as Reagan released Iranian government's assets which had been frozen by Carter
        \end{itemize}
    \end{itemize}
    \textbf{The 1980 campaign saw a major success for Reagan, in large part due to the widespread disillusionment with the Carter administration.}
}
\cornell[The "Reagan Revolution"]{What was Reagan's political impact and how did he change American society in the long term?}{\textbf{Reagan's economic policies helped the economy overall with reduced unemployment, but also caused the budget deficit to further increase; furthermore, his reduced taxation precipitated the fiscal crisis. His outside relations were staunchly anti-communist. Despite this, he continued to garner the popular vote and won the election of 1984.}}
\cornell{How did Reagan come to power?}{
    \begin{itemize}
        \item Reagan owed large part of success to dissatisfaction with Carter administration
        \item Owed success in part to coalition of conservative groups 
        \begin{itemize}
            \item Groups of wealthy Americans unified by devotion to capitalism, continual economic growth, \textit{laissez-faire} economics
            \begin{itemize}
                \item Opposed supposed "anti-business" policies
                \item Reagan carefully gathered free-market conservatives
            \end{itemize}
            \item Second group unified by anti-communist sentiment: intellectuals seeking to return democratic state to American society (while sympathizing with capitalists), known as neo-conservatives
            \item New Right not synonymous with this coalition, differentiated by New Right's distrust of "eastern establishment"
            \begin{itemize}
                \item Feared potential collaboration between Foreign Relations Council and east
            \end{itemize}
            \item Coalition, in all, expressed concerns held by many non-elite Americans
            \begin{itemize}
                \item Testament to Reagan's political ability to appeal to elites while taking on viewpoints of non-elites
            \end{itemize}    
        \end{itemize}
    \end{itemize}
    \textbf{Reagan's success was in part due to Carter's failures, but also in part due to large coalition between capitalist elites and neo-conservative anti-socialists. Furthermore, his ability to sympathisze with public sentiment put him in a position of great popularity.}
}
\cornell{What was Reagan's image while in the White House?}{
    \begin{itemize}
        \item Powerful image as gifted speaker and fearless, impervious man drew to him many people who opposed his policies 
        \begin{itemize}
            \item Known for keen sense of humor, youthful trips to California ranch
        \end{itemize}
        \item Not heavily involved in government, often showing ignorance about own policies 
        \begin{itemize}
            \item Instead used position as president to garner support for energetic administrators' policies
        \end{itemize}
    \end{itemize}
    \textbf{Reagan's image was of a powerful, youthful, resilient man with great charisma and public speaking ability. Although he was not heavily involved in policy-making, his role in office paired with his ability allowed him to garner support among populace.}
}
\cornell{What is supply-side economics?}{
    \begin{itemize}
        \item "Reaganomics"/supply-side economics promoted reduced taxation, especially for corporations, to promote new investments; reduced government income required budget cuts
        \item Those appointed reduced role of government, eliminating environmental enforcement, opening land to free, public development, slowing civil right enforcement, reducing transporation safety
        \item Nation fell into state of severe recession by 1982, with peak unemployment; however, quickly recovered, with large increase and lowered inflation
        \textbf{Supply-side economics entailed reduced taxation to promote investment paired with government budget cuts through deregulation.}
    \end{itemize}
}
\cornell{What caused the fiscal crisis in the mid-1980s?}{
    \begin{itemize}
        \item Reagan accumulated more debt during eight years than American government had in entire history
        \item Budget cuts failed in part due to increased cost of entitlement programs due to aging populations, but also due to tax cuts and increase in military spending promoted by Reagan
    \end{itemize}
    \textbf{The fiscal crisis during Reagan's time of power was in part caused by natural increase in entitlement programs like Medicare/Social Security, but also caused by Reagan's economic failures due to tax cuts and high military spending.}
}
\cornell{What were Reagan's diplomatic connections and how did his policies influence other nations?}{
    \begin{itemize}
        \item Reagan's goal to restore international prestige to the world and continue fighting against communism
        \item Soviet Union relations became even more strained due to Reagan's harsh language accusing the Soviet Union of supporting terrorism
        \begin{itemize}
            \item Denounced SALT II as unfair to U.S. but continued to honor it
            \item Promised Strategic Defense Initiative, claiming to prevent nuclear war through a laser sattelite-driven shield; denounced by USSR for even further escalating arms race
        \end{itemize}
        \item Cold War escalation led to popular movement calling for end to nuclear weapons with goal of "nuclear freeze" 
        \item Reagan doctrine supported all anti-communist nations regardless of relevance to Soviet Union, giving new American influence in Third World
        \begin{itemize}
            \item Ousted anti-American Marxists from Grenada, supported \textit{contras} after pro-American Nicaragua fell to Marxist \textit{Sandinistas}
            \item In Lebanon, supported smaller Palestinian Liberation Organisation forces against Israeli invsion, leading to 1983 terrorist bombing in Beirut
            \item Marked the most significant turn in smaller, otherwise powerless groups using terrorism to advance amidst
        \end{itemize}
    \end{itemize}
    \textbf{Reagan showed strong support for all anti-Communist nations, attempting to implement any possible policy with the potential to limit the influence of Communism and support American ideals.}
}
\cornell{What was the result of the 1984 election? Why?}{
    \begin{itemize}
        \item Reagan faced off against democratic Walter Mondale, spokesperson for poor and minorities
        \begin{itemize}
            \item Mead headlines due to selection of Geraldine Ferraro, the first woman in history to be on a ballot, as running mate
        \end{itemize}
        \item Reagan won decisive victory by 59\% of vote, carrying all but 1 state and D.C.
        \begin{itemize}
            \item Reagan stronger than party - Democrats gained seat in Senate, maintained slightly reduced control of House of Representatives
        \end{itemize}
    \end{itemize}
    \textbf{Reagan won the election against democratic Walter Mondale due to his emphasis on American patriotism.}
}

\cornell[America and the Waning of the Cold War]{What led to the end of the Cold War? How did this change American society?}{\textbf{The Cold War ended as Russia's economy began to collapse as many became disillusioned with communism around the world; Mikhail Gorbachev's attempts at recovery ultimately failed. With the U.S. the only remaining superpower, economic spending increased dramatically, with the U.S. hoping to further their interests worldwide rather than simply boosting their military. This led to the Gulf War under president George H.W. Bush, a war against Iraq for their annexation of Kuwait. However, the economic recession caused under Bush allowed Clinton to take the presidency in the 1992 election.}}
\cornell{How did the Soviet Union fall?}{
    \begin{itemize}
        \item Mikhail Gorbachev, one of the most revolutionary figures in world politics, transformed Soviet politics with two key initiatives
        \begin{itemize}
            \item \textit{Glasnost} was intended to dismantle repressive mechanisms historically characteristic of Soviet life
            \item \textit{Perestroika} attempted to restructure economy by allowing certain elements of capitalism (like private ownership)
            \begin{itemize}
                \item Economic problems convinced Gorbachev to reduce external economic commitments 
                \item Reduced economic influence in eastern Europe, leading to government overthrows within a few months in 1989
            \end{itemize}
        \end{itemize}  
        \item Other anti-communist movements less successful around the rest of the world
        \begin{itemize}
            \item Chinese students launched democratization movements, crushed in bloody assault at Tiananmen
            \begin{itemize}
                \item Despite crushing democracy movement, did not stop Westernization of economy
            \end{itemize}
            \item Government of South Africa began to withdraw from formally strict apartheid policies
            \begin{itemize}
                \item Legalized major black party, released Nelson Mandela -> Nelson Mandela quickly became first black president of South Africa
            \end{itemize}
        \end{itemize}
        \item Communism began to collapse within Soviet Union by 1991 with failed coup by Soviet Leaders
        \begin{itemize}
            \item Gorbachev returned to power, but it quickly became evident that communist party could not return
            \item Soon resigned, marking complete end of Soviet Union
        \end{itemize}
    \end{itemize}
    \textbf{The collapse of the Soviet Union coincided with the weakening of many other anti-democratic states, but was primarily caused by numerous long-term factors including the war in Afghanistan. Gorbachev's last-ditch attempts to recover ultimately failed, causing rebellions in other nearby communist states.}
}
\cornell{What was the relationship between Reagan and Gorbachev?}{
    \begin{itemize}
        \item Reagan was originally skeptical of Gorbachev, but gradually came to accept his desire for reform
        \begin{itemize}
            \item Worked together, holding a summit in Iceland and at their respective capitals aimed to reduce nuclear weapons
        \end{itemize}
        \item Two leaders soon agreed to remove nuclear arsenals from Europe; Gorbachev ended long war in Afghanistan 
    \end{itemize}
    \textbf{Reagan and Gorbachev made significant progress in reaching a peaceful relationship, curbing the nuclear arms race and ending the long war in Afghanistan.}
}
\cornell{What led Reagan's influence to wane in American politics?}{
    \begin{itemize}
        \item Political scandals which had originally been hidden due to Reagan's pure popularity began to come to light
        \begin{itemize}
            \item Included corruption, illegality, ethical lapses, and significant deregulation of loan industry
            \item Most significant scandal: American government had sold weapons to Iran's revolutionary government in a failed attempt to release Americans held in the Middle East
            \item Other money illegaly funneled to assist Nicaraguan \textit{contras}
        \end{itemize}
        \item Numerous other illegal activities were exposed, damaging Reagan presidency
    \end{itemize}
    \textbf{Reagan's influence began to fade after numerous illegal activities were exposed, most significantly the selling of weapons to Iran's revolutionary government.}
}
\cornell{How did George H.W. Bush come to power?}{
    \begin{itemize}
        \item Fraying of Reagan administration gave Democrats significant hope, but very few popular figures ran for the nomination
        \begin{itemize}
            \item Ultimately went to little-known Michael Dukakis, a dull campaigner
            \item Remained confident in prospects: opponent, George H.W. Bush, had failed to spark any public interest
        \end{itemize}
        \item Turn-around beginning at the Republican Convention, where Bush began to heavily attack Dukakis, tying to negative stigma of the "liberal"
        \begin{itemize}
            \item Bush won presidency, but Democrats secured majority in both houses
        \end{itemize}
    \end{itemize}
    \textbf{George H.W. Bush was initially seen as a write-off due to the negative outlook on the Republican party caused by Reagan; however, his eventual harsh words against the Democratic party sparked great interest within the public, earning himself the spot as president.}
}
\cornell{What were the most significant events in the Bush Presidency?}{
    \begin{itemize}
        \item Bush experienced large popularity during first three years due to subdued image, careful responses against major world events
        \begin{itemize}
            \item Eventually cooperated with Gorbachev to reach agreements with Soviet Union in arm-reduction
        \end{itemize}
        \item Much less successful on domestic issues, having to deal with heavy burden of debt created by Reagan
        \begin{itemize}
            \item Pledges to reduce deficit conflicted with promises of no new taxes
            \item Congress and White House agreed on many significant measures
            \begin{itemize}
                \item Cooperated to salvage savings and loan industry, eventually decided on significant tax increase to reduce deficit
            \end{itemize}
            \item Neither Congress nor Bush had solution to recession beginning in 1990, increasing grip on economy by 1992
            \begin{itemize}
                \item Enormous debt had led to numerous bankruptcies, increased fear among American middle/working classes
                \item Put pressure on government to address rising cost of health care
            \end{itemize}
        \end{itemize}
        \item The \textbf{Gulf War} was caused by the United States reducing military spending to focus on economic interests as a result of becoming the only major superpower after USSR collapse
        \begin{itemize}
            \item Invaded Panama, overthrowing military leader and forcibly replacing with elected leader
            \item When Hussein announced plans to annex oil-rich Kuwait, Bush led campaign to force out
            \begin{itemize}
                \item Convinced nearly all important nations (including USSR and other Arab nations) to place trade embargo on Iraq
                \item UN supported expulsion by military invasion; American forces began to bombard Iraqi forces in Kuwait
                \item Bombing continued for six weeks until American ground forces (under Norman Schwarzkopf) invaded from unexpected north without resistance
                \begin{itemize}
                    \item Killed more than one hundred thousand Iraqis, finally led to cease-fire agreement with acceptance of terrorism
                \end{itemize}
                \item Many Muslims, angered by US hostility against religious bretheren combined with female forces, took to terrorism to target Americans
            \end{itemize}
        \end{itemize}
    \end{itemize}
    \textbf{Bush's popularity grew due to his careful responses to important events on the world stage; however, domestic issues, most significantly the 1990 recession, challenged the public opinion of him. Perhaps the most significant event was the brief Gulf War, the bombardment of Iraq for their invasion of oil-rich Kuwait. It marked the rapid growth of hatred for America especially among the most conservative Muslims, beginning an age of frequent terrorism in the Western world.}
}
\cornell{What was the result of the 1992 election?}{
    \begin{itemize}
        \item Bush's popularity peaked in 1991 after the success of the Gulf War
    \begin{itemize} \item Quickly faded after worsening of recession in 1991 paired with few policies to combat \end{itemize}
        \item Early plans for 1992 election began when Bush's popularity had spiked, leading few Democrats to run
    \begin{itemize} \item Gave opportunity to Bill Clinton, young five-term Arkansas governor\end{itemize} 
        \item Both candidates challenged by independent Ross Perot, Texas billionare promising end to fiscal crisis
    \begin{itemize} \item Led Bush and Clinton in public opinion polls through spring; abruptly withdrew after scandal; returned, regaining most of support\end{itemize}
        \item Clinton finally won clear yet undramatic victory
    \end{itemize}
    \textbf{Democrat Clinton won the election against Bush and independent billionaire Perot.}
}
\end{document}