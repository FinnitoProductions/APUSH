\documentclass[a4paper]{article}
    \usepackage[T1]{fontenc}
    \usepackage{tcolorbox}
    \usepackage{amsmath}
    \tcbuselibrary{skins}
    
    \title{
    \vspace{-3em}
    \begin{tcolorbox}
    \Huge\sffamily \begin{center} AP US History  \\
    \LARGE Chapter 9 - Jacksonian America \\
    \Large Finn Frankis \end{center} 
    \end{tcolorbox}
    \vspace{-3em}
    }
    \date{}
    \author{}
    
    \usepackage{background}
    \SetBgScale{1}
    \SetBgAngle{0}
    \SetBgColor{red}
    \SetBgContents{\rule[0em]{4pt}{\textheight}}
    \SetBgHshift{-2.3cm}
    \SetBgVshift{0cm}
    \usepackage[margin=2cm]{geometry} 
    
    \makeatletter
    \def\cornell{\@ifnextchar[{\@with}{\@without}}
    \def\@with[#1]#2#3{
    \begin{tcolorbox}[enhanced,colback=gray,colframe=black,fonttitle=\large\bfseries\sffamily,sidebyside=true, nobeforeafter,before=\vfil,after=\vfil,colupper=blue,sidebyside align=top, lefthand width=.3\textwidth,
    opacityframe=0,opacityback=.3,opacitybacktitle=1, opacitytext=1,
    segmentation style={black!55,solid,opacity=0,line width=3pt},
    title=#1
    ]
    \begin{tcolorbox}[colback=red!05,colframe=red!25,sidebyside align=top,
    width=\textwidth,nobeforeafter]#2\end{tcolorbox}%
    \tcblower
    \sffamily
    \begin{tcolorbox}[colback=blue!05,colframe=blue!10,width=\textwidth,nobeforeafter]
    #3
    \end{tcolorbox}
    \end{tcolorbox}
    }
    \def\@without#1#2{
    \begin{tcolorbox}[enhanced,colback=white!15,colframe=white,fonttitle=\bfseries,sidebyside=true, nobeforeafter,before=\vfil,after=\vfil,colupper=blue,sidebyside align=top, lefthand width=.3\textwidth,
    opacityframe=0,opacityback=0,opacitybacktitle=0, opacitytext=1,
    segmentation style={black!55,solid,opacity=0,line width=3pt}
    ]
    
    \begin{tcolorbox}[colback=red!05,colframe=red!25,sidebyside align=top,
    width=\textwidth,nobeforeafter]#1\end{tcolorbox}%
    \tcblower
    \sffamily
    \begin{tcolorbox}[colback=blue!05,colframe=blue!10,width=\textwidth,nobeforeafter]
    #2
    \end{tcolorbox}
    \end{tcolorbox}
    }
    \makeatother

    \parindent=0pt
    
    \begin{document}
    \maketitle
    \SetBgContents{\rule[0em]{4pt}{\textheight}}
    \cornell[Key Concepts]{What are this chapter's key concepts?}{\begin{itemize}
        \item \textbf{4.1.I.C} - New Whig Party, led by Henry Clay, emerged by 1820s; disagreed with Jackson's Democrats on role/powers of federal government, issues such as national bank, tariffs, federal funding
        \item \textbf{4.1.I.D} - Regional interests often prioritized over national concerns for leaders' positions on slavery/economy
        \item \textbf{4.2.III.D} - American System (among other plans to unify U.S. economy), caused debates over whether agriculture or industry would be favored
        \item \textbf{4.3.I.A} - U.S. government sought to control North America, Western Hemisphere through exploration, removal of natives, Monroe Doctrine
        \item \textbf{4.3.I.B} - Frontier settlers championed expansion efforts; native resistance -> sequence of wars, federal efforts for relocation
    \end{itemize}}
    \cornell[The Rise of Mass Politics]{How did Jackson usher in the rise of politics controlled by the masses?}{\textbf{Jackson, known for his success in the battle of New Orleans, was predated by but also led a great increase in democracy: before him, many states began to expand suffrage beyond propertyholders (with many compromises), but there was still significant room to improve in African American, women, and native suffrage. Jackson, himself, legitimized the partisan system through his election and aimed to appeal to the common people by dethroning the long-standing elites in favor of commoners for rule; ultimately, he appealed more to his followers than the masses.}}
    \cornell{What characterized Jackson's inauguration?}{\textbf{Thousands of Americans from throughout the country of all social ranks crowded before Washington Capitol to watch his inaguration, with the crowd following their hero to the White House after the ceremonies concluded. They followed him into the White House, ruining the elegant furniture and carpetry to meet the new president. Many viewed it as a mob uprising, however.}}
    \cornell{What was Andrew Jackson's history?}{\begin{itemize}
        \item Born in 1767 to Irish parents in Carolinas; captured by British during Revolution at age of 13, injured after refusing to clean boot -> enduring hatred
        \item Sporadic education, various shops/farms; studied law, was admitted to practice
        \item Known for extensive yet dynamic political history
        \begin{itemize}
            \item Early work was basic land claims 
            \item Soon elected as delegate to TN constitutional convention, becoming Congressman in same year (1797)
            \item Resigned as Senator after one year; appointed to TN Supreme Court from 1798-1804
        \end{itemize}
        \item Became prosperous as planter/merchant with slaves in elegant Nashville plantation; among largest in state
        \item Joined militia in 1801, soon became general; fought natives in AL/GA; fought British in War of 1812 with decisive victory in Battle of New Orleans
        \begin{itemize}
            \item War of 1812 -> many hailed as hero, called for him to run for president
        \end{itemize}
    \end{itemize}
    \textbf{Andrew Jackson started with modest beginnings but eventually worked up to a career in law. He was a delegate to the constitutional convention of Tennessee, a Congressman, a Senator, a member of the Tennessee Supreme Court, a planter and merchant, and a militiaman.}}
    \cornell{What were the major democratic changes occurring pre-Jackson?}{
        Jackson had little effect on economic equality, instead giving many more the right to vote. Up to the 1820s, most states had restricted right to vote to white male property owners and taxpayers, causing the less wealthy to be left out. However, change began in the western states and rapidly spread elsewhere to the eastern states fearful of emigration.
        \begin{itemize}
            \item Significant change -> great resistance, as in MA constitutional convention of 1820
            \begin{itemize}
                \item Radicals demanded more poor representation, but conservatives resisted (including Daniel Webster) -> property requirement eliminated
                \item Still had to be taxpayer to vote, own real estate to be governor
            \end{itemize}
            \item Often far more successful, like in NY convention of 1821
            \begin{itemize}
                \item Conservatives insisted that tax-paying requirement was insufficient, at least for state senators
                \item Reformers cited Decl. of Indep., maintaining that property was not a fundamental concern of society -> requirement abolished
            \end{itemize}
            \item Changes which went through often -> great instability
            \begin{itemize}
                \item RI constitution (old colonial charter) had barred half of adult males from voting with conservative legislature blocking any reform efforts
                \item Lawyer Thomas W. Dorr under "People's Party" drafted new constitution, submitted for pop. vote -> overwhelming approval but legislatures would not accept, submitting their own constitution (but narrowly defeated)
                \item Dorr became governor in eyes of followers (Dorr Rebellion) -> old government proclaimed insurrection, began to imprison Dorrites 
                \begin{itemize}
                    \item Unable to capture state arsenal; Dorr surrendered
                    \item Long-term effect of drafting new constitution to expand suffrage
                \end{itemize}
            \end{itemize}
        \end{itemize}
        \textbf{Many began to call for universal suffrage, with significant resistance often leading to compromises (like in MA, with taxpaying requirement still intact), other states experiencing great success (like NY by citing the Declaration of Independence), and others seeing instability from great changes (like in RI with the Dorr Rebellion.}}
        \cornell{How could the U.S. still improve in terms of suffrage?}{\begin{itemize}      
            \item In south, little overall success w/ election laws continually favoring plantation owners, politicians of past
            \item Even in North, blacks were unable to vote (PA removed right in 1838)
            \item Women could vote in no states
            \item Voting was undemocratic: ballot not secret with spoken vote -> politicans could bribe/intimidate
        \end{itemize}
        \textbf{The U.S. still needed many significant improvements to approach universal suffrage, including removing the property restriction/bias in the south, allowing African Americans and women to vote universally (no states allowed it), and changing voting to a more democratic format without rampant bribery.}}
        \cornell{What were some significant democratic reforms in the early nineteenth century?}{\textbf{The method of selecting presidential electors changed drastically, with a gradual shift from legislature selection to popular vote: by 1828, all states had shifted but SC. This allowed the number of voters to increase drastically over time.}}
        \cornell{Who was Tocqueville and how did he view American democracy?}{\begin{itemize}
            \item French aristocrat Alexis de Tocqueville, intrigued, documented growth of electorate, shift to political parties, rapid spread in right to vote after spending two years in U.S. during Jackson's time
            \begin{itemize}
                \item Sent to study American prisons for humane influence; went beyond to write \textit{Democracy in America}, examining daily lives of key Americans and cultures, associations, democratic visions
            \end{itemize}
            \item French democracy had been restricted to landowners/aristocrats; Tocqueville realized failing aristocracy
            \item Tocqueville understood limits of democracy: favored white men, remaining a distant hope for many
            \item Ultimately spread American democracy to France, other European nations
        \end{itemize}
        \textbf{Tocqueville wrote \textit{Democracy in America}, examining American democracy and noting the significance of the collapsing aristocracy but also of the limits, still restricted to mostly white men and leaving out women, blacks, and natives. In the long term, he spread the ideals American democracy to many European nations.}}
        \cornell{How was the party system legitimized over time?}{\begin{itemize}
            \item Voter participation also result of interest in politics, party organization, party loyalty
            \item Initially, parties viewed as evil entities, with many believing nation should come to consensus without factional lines
            \item Quickly fell apart in 1820s, 1830s, believing discordant parties were key to democracy, beginning at state level
            \begin{itemize}
                \item Van Buren's post-War of 1812 political faction (Bucktails) in NY state challenging established aristocratic leadership (under De Witt Clinton)
                \begin{itemize}
                    \item Argued that institutionalized party could secure democracy unlike Clinton's closed elite
                    \item Proposed ideological committments trumped by party loyalty with main goal to preserve party's success
                    \item Competing parties required for any given party to survive -> would force politicians to remain interested in will of people to balance each other
                \end{itemize}
                \item Jackson's election further legitimized on federal level
                \begin{itemize}
                    \item Two-party system became official with legitimate institution under powerful opponents: Whigs
                    \item Jackson's followers were officially Democrats
                \end{itemize}
            \end{itemize} 
        \end{itemize}
        \textbf{Parties were long viewed as evil and destructive, going directly against democracy. However, beginning with Martin Van Buren's political faction, the Bucktails, in New York state, with the argument that competing parties and party loyalty were essential factors to appeal to the people, parties were quickly legitimized on the state level. Jackson's election further legitimized it on the federal level: his followers were the Democrats and the opposition formally became the Whigs.}}
        \cornell{How did Jackson appeal to the common people?}{\begin{itemize}
            \item Jackson not a philosopher (unlike Jefferson) -> no uniform ideology, but embraced simple theory of democracy: to offer equal protection to all white males regardless of class
            \begin{itemize}
                \item Represented direct assault on eastern aristocracy, effort to promote rising westerners/southerners
                \item Justified subjugation of African Americans/natives: to protect white men
            \end{itemize}
            \item Jackson first targeted federal officeholders, many of whom had ruled for over a generation; believed offices belonged to people, not long-term holders
            \begin{itemize}
                \item Removed no more than one-fifth during eight years, mostly for misuse of funds or corruption 
                \item Dismissed no more than Jefferson, but philosophy gave future elected officials right to appoint their own followers to office
            \end{itemize}
            \item Transformed process to win party's nomination, resenting caucuses for favoring elites 
            \begin{itemize}
                \item For 1832 election, established partisan convention to nominate for presidency; creators saw as democratic triumph where power would arise from people
            \end{itemize}
        \end{itemize}
        \textbf{Jackson's main goal was to limit the power of the elites and give the lower classes the opportunity to rise: he accomplished this by targeting long-standing federal officeholders (though few were ultimately removed) and by replacing the caucus system with a partisan convention to nominate the presidential candidate. In all, Jackson did successfully detrench the elites, but mostly transferred power to his own allies: those nominated at the conventions were rarely common men.}}
        \cornell["Our Federal Union"]{What were the critical tensions during Jackson's first term and how did this affect the political standings of many key players?}{\textbf{The most significant tension was Calhoun's proposed theory of nullification, which would allow states to nullify federal law; this tension escalated in the Webster-Hayne debate, where Hayne argued against Northeastern tyranny in favor of independent state control and the Jackson-supported Webster pushed for the Union and federal supremacy. Crisis emerged when SC attempted to nullify the Treaty of 1816, angering Jackson and potentially inciting violence; however, Clay's compromise of a gradual reduction solved this. Ultimately Calhoun's standing was reduced greatly in the eyes of Jackson while Van Buren, his greatest opponent, remained a strong ally of Jackson, and his influence was further strengthened by his social etiquette.}}
        \cornell{What fundamental beliefs pushed Jackson to despite the federal power?}{
        \textbf{Although he despised the concentration of power in Washington, reducing the potential of those without connections, and thus pushed an economic plan to restrict it, Jackson still asserted the overall control and power of the union in facing the theory of nullification.}}
        \cornell{What formed the basis of Calhoun's theory of nullification?}{\begin{itemize}
            \item 46 year-old Calhoun fell out of grace due to support for tariff of 1816 
            \begin{itemize}
                \item South Carolinians blamed tariff for economic stagnation 
                \item Realistically due to farmland exhaustion -> unable to compete w/ SW -> many called for secession 
            \end{itemize}
            \item Calhoun had to solve issue in home state -> secession alternative: nullification where states had could hold special council to nullify federal law
            \begin{itemize}
                \item Supported by many South Carolinians but not by federal government
                \item Rationale was because federal government was creation of states, not other way around
            \end{itemize}
        \end{itemize}
        \textbf{Calhoun, in an attempt to solve the economic stagnation emergent in SC leading to potential secession, introduced the controversial theory of nullification, which allowed states to hold special conventions to nullify federal law if declared unconstitutional.}}
        \cornell{How did Van Buren put himself in a favorable position with Jackson?}{\begin{itemize}
            \item Van Buren was governor of NY from 1828-1829, when Jackson appointed secretary of state 
            \item Quickly established as critical political ally, part of circle (known as "Kitchen Cabinet")
            \item Van Buren's influence over Jackson was immense, but further increased due to etiquette quarrel
            \begin{itemize}
                \item Peggy O'Neale, married daughter of Washington tavernkeeper who had lodged w/ senators Jackson + Eaton
                \item Eaton rumored to have affair w/ O'Neale; husband soon died (1828) and Eaton soon married her
                \item Eaton named secretary of war by Jackson, but other administration wives would not accept O'Neale -> Jackson furious, Van Buren jumped in and accepted family while Calhoun refused -> by 1831, chosen as successor
            \end{itemize}
        \end{itemize}
        \textbf{Van Buren, as a close friend and political ally of Jackson's, further expanded his influence over the presidency by accepting a cabinet wife of Jackson's friend who had committed adulterous acts.}}
        \cornell{What were the critical themes of the Webster-Hayne Debate?}{\begin{itemize}
            \item Senate debate unfolded in early 1830 as nullification crisis intensified where CT senator argued that land sales/surveys be slowed to reduce spread of slavery
            \begin{itemize}
                \item Supported by Daniel Webster, attacking Hayne and inadvertently Calhoun by subtly arguing conflict between state rights, national power
            \end{itemize}
            \item SC senator Hayne argued importance of Western lands to prevent dominance of East and boost position in Congress to lower tariff, prevent alleged joint tyranny of Northeast 
            \begin{itemize}
                \item Coached by Calhoun to argue for nullification; Webster presented "Second Reply to Hayne" over two full afternoons, with powerful conclusion concerning deep connectedness of liberty/union 
            \end{itemize}
            \item Jackson supported Webster's argument against Calhoun, made clear by push for strength of Union at Democratic Party banquet honoring Jefferson while looking directly at Calhoun and receiving Van Buren's support
        \end{itemize}
        \textbf{The Webster-Hayne Debate, despite being based on the surface around whether western expansion should be slowed to reduce slavery, was fundamentally an argument concerning state rights and national power: Hayne, an SC senator, pushed that Western lands were critical to prevent eastern dominance and northeastern tyranny, going against the central government in favor of that of his own state; Webster, supported by Jackson against Hayne and Calhoun, pushed against nullification to argue the conflict between state rights and national power. }}
        \cornell{What crisis emerged over nullification?}{\begin{itemize}
            \item When Congress passed a bill without any change in Tariff of 1816, SC called for state convention to nullify tariffs of 1828, 1832, appoint Hayne as governor and resigned Calhoun as senator
            \item Jackson called nullification treasonous, accused perpetrators of being traitors
            \begin{itemize}
                \item Sent warship to Charleston, bolstered federal military forts
                \item Proposed force bill to allow the president to use military to enforce Congressional acts
            \end{itemize}
            \item Senator Calhoun received no support from other states, even experienced great divisions within state 
            \item Crisis prevented by Clay, producing compromise allowing for steady decrease in tariffs to reach same level as 1816 by 1842 
            \begin{itemize}
                \item Passed on same day as force bill, both signed by Jackson
                \item SC recalled convention, repealed nullification; Calhoun claimed victory due to change in tariff but situation ultimately showed federal dominance
            \end{itemize}
        \end{itemize}
        \textbf{With the Carolinians outraged by Congress' lack of bills designed to limit the Tariff of 1816, nullification was implemented to eliminate the tariff. Violence seemed near: Jackson sent warships to Charleston and devised a bill to allow military authorization for Congressional treason. Calhoun lacked the support of any other state, pushing him to concede simply a change in the tariff which would allow a steady reduction.}}
        \cornell[The Removal of Indians]{What were Jackson's critical policies toward the natives?}{\textbf{Jackson, his war experience giving him a skeptical view on native society, generally shared the general hostile view that natives were complete savages incapable of recovery. These attitudes led to the Black Hawk War in the Northwest, known for vicious white armies slaughtering large numbers even after surrender, and the relocation of the "Five Civilized Tribes," most notably the civilized Cherokees, in the south along the Trail of Tears, brutal for all ages. In summary, many justified the native relocation to such smaller and unfamiliar territory as inevitable; however, this belief was justified against the potential of cohabitation (like in pueblos of NM) by describing the natives as savages without claim to their lands.}}
        \cornell{What were Jackson's personal motivations for relocating the natives?}{\textbf{Especially due to his military campaigns against southern tribes, Jackson sought to relocate the natives west beyond the Mississippi to further expand westward. Ultimately, his views were very similar to most other white Americans.}}
        \cornell{What were the general white attitudes toward native tribes?}{
            \textbf{The paternalistic 18th century attitudes that natives were "noble savages" perpetuated by Jefferson and allies were soon replaced by ones more hostile, with most viewing them as pure "savages." Furthermore, many white westerners feared potential war with the natives and sought to expand.}}
        \cornell{What events unfolded in the Black Hawk War?}{\begin{itemize}
            \item Northwest tensions w/ Indians climaxed in final battle in 1831-1832 between Illinois whites, Sauk/Fox native alliance led by warrior Black Hawk
            \item Treaty, but had been signed by rival tribe -> resentful Black Hawk reoccupied Illinois lands, federal troops saw as invaders
            \item White military especially vicious
            \begin{itemize}
                \item Sought to exterminate natives and ignored even Black Hawk's surrender
                \item Most fled across Mississippi into Iowa, but tribes followed and slaughtered most
                \item Black Hawk captured, sent on tour of east to meet "curious" whites (including Jackson)
            \end{itemize}
        \end{itemize}
        \textbf{The Black Hawk War unfolded in the northwest as tensions between natives and whites escalated - the warrior Black Hawk led the Sauk and Fox tribes into battle to resist a treaty made with an enemy tribe; the settlers responded viciously, exterminating most even after surrender and capturing their leader and parading him around the East.}}
        \cornell{What were the "Five Civilized Tribes" in the South?}{\begin{itemize}
            \item Cherokee, Creek, Seminole, Chikasaw, Choctaw had created agrarian societies with economy in South, particularly Cherokees w/ written languages, formal constitution
            \item Many whites believed civilization of Cherokees -> deserved to keep eastern lands because they had forsaken traditional ways of hunting and gathering and their somewhat egalitarian society in favor of white ways
            \item Federal government sought to negotiate to move all to West, but slow process -> states began to take action
            \begin{itemize}
                \item Georgia independently attempted to encroach Creeks despite Adams' demands
                \item Jackson's early administration saw states taking direct action but endorsed by Congress/Jackson
                \begin{itemize}
                    \item Passed Removal Act in 1830 to finance negotiations with tribes
                    \item As tribes faced federal/state governments, few could resist
                \end{itemize}
            \end{itemize}
            \item Cherokees attempted to stop Jackson's encouraged expansion by appealing to Supreme Court: \textit{Cherokee Nation v. Georgia}, \textit{Worcester v. Georgia} temporarily freed tribe 
            \begin{itemize}
                \item Jackson, seeking southern support and continually expressing distaste for native, even expressing public doubt that Marshall could enforce decision 
                \item Federal government produced treaty w/ minority (not truly representative) faction of Cherokees, providing \$5m and reservation west of Mississippi 
                \begin{itemize}
                    \item Few Cherokees initially followed until army of 7,000 led by Winfield Scott sent to force them out
                \end{itemize}
            \end{itemize}
        \end{itemize}
        \textbf{The "Five Civilized Tribes" were among the most difficult for the whites to uproot and move west: they had adapted to a large degree to agrarian society and white culture, so many felt their land deserved preservation, particularly the Cherokees for their written language and constitution. However, the Cherokee efforts to resist encroachment saw Jackson overpower the Supreme Court's ruling by forming a treaty with a minority group and forcing them to move west.}}
        \cornell{What hardships were endured on the Trail of Tears?}{\begin{itemize}
            \item $\approx 1,000$ Cherokees fled to NC; federal government eventually agreed to form reservation in Smoky Mountains (remains today)
            \item Rest underwent journey to "Indian Territory" in modern Oklahoma; known for brutality and suffering for those of all ages
            \begin{itemize}
                \item $\frac{1}{8}$ or more died before/soon after reaching forced destination 
                \item Journey known as "Trail of Tears"
                \begin{itemize}
                    \item Jackson justified by explaining that the native race was already too far gone
                \end{itemize}
                \item All civilized tribes forced to undergo journey; started w/ Choctaws, then Creeks, then Chickasaws, and finally Cherokees
            \end{itemize}
            \item Most of government believed territory was part of "Great American Desert," land unfit for explorers -> never any risk of additional conflict
            \item Only Seminoles resisted relocation somewhat; eventually, settled to move to territory within 3 years but minority remained behind with staged uprisings
            \begin{itemize}
                \item War continued for years beginning in 1835, with large number of American troops often outsmarted by guerilla warfare
                \item Leader, Osceola, captured while on truce; Americans had spent large money on war and lost many men
                \item Seminoles remained in Florida; although many had been killed or forced, complete migration never occurred
            \end{itemize}
        \end{itemize}
        \textbf{The Trail of Tears was the route taken by the Five Civilized Tribes to the Indian Territor, known for treacherous conditions. The Sminoles were the only tribe with successful attempts at resisting encroachment; the minority which supported preserving the land engaged in the Seminole Wsr with Jackson. Their migration was never truly complete.}}
        \cornell{What were the true impacts of the forced removal of eastern tribes?}{\begin{itemize}
            \item In total, tribes had given 100m acres of land in exchange for \$68m, 32m in less hospitable lands
            \begin{itemize}
                \item Divided by tribe into strict reservations surrounded by U.S. forts to prevent escape or entrance of whites, but would eventually even face white invasion
                \item Topography/climate far from anything of past 
            \end{itemize}
            \item Alternatives to removal of natives were few; westward expansion was too powerful a force to be stopped
        \end{itemize}
        \textbf{The natives were paid a small sum of money and given unfamiliar yet protected terrain sized at around one third of what they had previously owned. The alternatives to this removal, many argued, were impossible.}}
        \cornell{What were the main alternatives to native removal?}{
            The West saw many tribes creating shared world w/ whites, though it was not always completely egalitarian.
            \begin{itemize}
                \item NM pueblos, western fur traders, and TX/CA w/ settlers from Mexico, Canada, U.S. saw relative harmony 
                \item Lewis + Clark Expedition often saw natives as sexual partners w/ mutualistic relationships 
                \item Despite frequent exploitivity in motives for multiracial societies, generally represented how two cultures could interact 
            \end{itemize}
        }
        \cornell{What justifications were made for a changing white view on natives?}{
        \textbf{By the mid-19th century, the whites sought to create plantations much like the early British settlers which had complete separation from the natives, who they believed were not truly tied to or part of the land - western territories were uncolonized and ready for takeover.}}
        \cornell[Jackson and the Bank War]{How did Jackson take on the national bank?}{\textbf{Jackson, greatly fearing an economic aristocracy, took action against the large Bank of the U.S., representing hard-money supporters who believed that the bank should be abolished because it did not back up its distributed cash with physical gold/silver and supporting soft-money supporters with the same goal but with the long-term aim of economic growth through free expansion of currency. Biddle, the president of the bank, fought Jackson's anger with the bank, initially attempting (and failing) to sway the 1832 election against Jackson by making the bank a decisive issue but later responding to Jackson's deliberate attempts to weaken the bank by removing money from it by calling in loans/raising interest. He took his policies too far, however, and ultimately lost the support of even his most trusted allies for his damage to the American economy.}}
        \cornell{What was Jackson's stance on the use of federal power?}{\textbf{He believed that it should be used against states/tribes when necessary but hesitated to have an economic influence, fearing the "economic aristocracy." His economic view was evidenced in his veto of a Congressional measure to subsidize KY Maysville Road, feeling it was unconstitutional as it did not cross between states, and felt it was an unwise expenditure.}}
        \cornell{How did the Bank of the United States represent a powerful economic aristocracy?}{
            The bank had a large HQ in Philadelphia and branches in 29 other cities -> it was regarded as the most powerful financial institution of the nation.
            \begin{itemize}
                \item Only place where government could deposit its own funds, but govt. also owned $\frac{1}{5}$ of bank stock
                \item Issued bank notes for reliable exchange throughout nation, restraining disorganized state banks
                \item Nicholas Biddle had been the president from 1823 onward, led powerfull and allowed the bank  to remain prosperous/sound
            \end{itemize}
            \textbf{The bank, with its large HQ, numerous branches, monopoly over government money, ability to control state notes through bank notes, was regarded as the most powerful economic institution of the U.S. Headed by Nicholas Biddle, it remained the dominant economic force in the nation.}}
            \cornell{What were the main sources of opposition to the national bank?}{\begin{itemize}
                \item "Soft-money" faction believed that more currency should be circulated not necessarily backed up by gold/silver
                \begin{itemize}
                    \item Generally state bankers simply objecting to state's restraining on free issue of notes
                    \item Sought rapid growth
                \end{itemize}
                \item "Hard-money" faction believed that all notes must be backed up by physical reserves
                \begin{itemize}
                    \item Believed that notes were dangerous, shunning speculation and expansion
                    \item Jackson was example due to large speculations undermined by Panic of 1797 -> deep debt 
                    \begin{itemize}
                        \item Acknowledged and supported soft-money supporters, too 
                        \item Made clear that he would not favor renewal of charter in 1836
                    \end{itemize}
                \end{itemize}
            \end{itemize}
            \textbf{The soft-money faction - generally state bankers seeking rapid economic growth - believed in the most currency possible regardless of physical reserves; the hard-money faction - taking on more old-fashioned views shunning speculation/expansion, Jackson included - believed that notes were dangerous without reserves.}}
            \cornell{How did Biddle resist Jackson's efforts to limit the power of the national bank?}{\begin{itemize}
                \item Biddle, with little experience in politics, granted favors to experienced men in hopes of preserving bank
                \begin{itemize}
                    \item Gained allegiance of Daniel Webster by naming legal counsel, director of Boston branch and supporting frequent borrowing 
                    \item Webster assisted in gaining trust of Henry Clay
                \end{itemize}
                \item Clay/Webster pushed Biddle to apply to Congress for renewal bill in 1832 (four years ahead) in hopes of turning into divisive issue come eleciton-time
                \begin{itemize}
                    \item Congress passed but Jackson vetoed with Congress unable to override
                    \item Clay ran for National Republicans w/ nomination from convention in Baltimore in hopes of receiving support of pro-bankers, but Jackson (w/ Van Buren) won w/ 219 electoral votes
                \end{itemize}
            \end{itemize}
            \textbf{Biddle solicited influential men such as Webster and Clay to ensure the preservation of the bank, who encouraged him toapply to Congress for an early renewal bill to mobilize the bank's supporters in the 1832 election of Jackson v. Clay. However, Jackson won and knew Biddle's defeat was imminent.}}
            \cornell{How did Jackson take direct action to take down the bank?}{\begin{itemize}
                \item Despite legal inability to directly abolish, weakened greatly by removing govt. deposits (fired two treasury secretaries before coming upon compliant friend Taney)
                \begin{itemize}
                    \item Taney placed government deposits in state banks instead
                \end{itemize}
                \item Biddle (known as "Czar Nicholas" by Jacksonians) took action after not only did govt. make new deposits in state banks, but also \textit{transfer} money from federal to national by raising interest rates, calling in loans
                \begin{itemize}
                    \item Aware of great risk of recession, but hoped that actions would convince Congress to recharter
                    \item Bank War had turned into more than battle for political beliefs, but instead duel between headstrong men
                \end{itemize}
                \item Bank supporters blamed recession on Jackson's, appealing to Washington for rechatering; Jacksonians blamed on Biddle: Jackson responded to distressed citizens with "Go to Biddle"
                \item When Biddle's contraction of credit began to directly hurt business allies, protested against him
                \begin{itemize}
                    \item NY/Boston merchants protested -> reversed decision with abundant credit
                    \item Actions had lost potential for recharter in 1836: flawed yet valuable institution had been lost, plaguing economy
                \end{itemize}
            \end{itemize}
            \textbf{Before the charter was due to expire, Jackson weakened the bank by transferring money and making new deposits into state banks. Biddle fought by raising interest and calling in loans, and the small recession temporarily caused a great split; however, when Biddle's choices affected even his allies, protests forced him to reverse his decision, losing the potential for recharter in 1836.}}
        \cornell[The Changing Face of American Politics]{How did American politics change following Jackson's rule?}{\textbf{Jackson's first major policy was the appointment of Roger Taney for Supreme Court, who went against many of Marshall's fundamental nationalist beliefs. Jackson's controversial policies promoted the creation of the Whig Party, opponents to Jackson's Democrats, who favored economic union and large institutions and generally had aristocratic supporters, contrasting with Democratic ideals of opportunity for all and common support; religious differences were also prominent. Both parties, however, often forsook philosophical differences to appeal to the voters. Initially, the Whigs were unable to unite under a single leader; they lost the 1836 election primarily because they were only able to narrow their selections to three candidates. Democrat Van Buren won the election, whose presidency was plagued by America's largest (at the time) ecomonic recession as a result of excessive land speculation. This paved the way for Whig Henry Harrison to take over; however, his immediate death saw VP Tyler rule for the majority of the presidency; his policies were far more Democratic, leading his party to abandon him and a portion of conservative Whigs to consider rejoining the Democrats. Under Tyler, diplomatic successes included negotiating the end to significant conflicts with Britain and promoting Chinese trade with the US. However, these were among the few successes of the administration: the Whigs lost the presidency in the following election.}}
            \cornell{What were the critical early decisions of the Taney court?}{\begin{itemize}
                \item 1835 death of Marshall -> Roger B. Taney appointed as new chief justice, modifying Marshall's nationalism
                \item Change in judicial structure most clearly evidenced in \textit{Charles River Bridge v. Warren Bridge} of 1837
                \begin{itemize}
                    \item Two MA companies disputed right to build bridge across Charles River: one had long-standing monopoly charter to operate toll bridge; other had received permission from legislature to bring toll-free bridge (which would hurt business)
                    \item First company argued that Marshall court stuck closely to contracts and that MA legislature's decision was a breach in contract
                    \item Taney (speaking for Democrat majority) believed that more important than contracts was general happiness of people -> states had right to amend if improved well-being
                    \begin{itemize}
                        \item Charles River Bridge represented original company benefitting from unfair privilege (who were also Boston aristocrats from Harvard compared to aspiring entrepreneurs of second company)
                    \end{itemize}
                    \item Decision represented democratic importance of expanding econ. opportunity
                \end{itemize}
            \end{itemize}
            \textbf{Taney, appointed by Jackson after Marshall's 1835 death, made his stance clear in the early case of \textit{Charles River Bridge v. Warren Bridge}, a duel between Boston aristocrats with a long-standing charter-granted monopoly on an MA toll bridge and aspiring entrepreneurs with legislative permission to build a toll-free bridge. Taney's Democratic court ruled that contracts were only significant when they represented the well-being of the people; because the Warren Bridge would greatly benefit the MA people, the legislative breach would be excused.}}
            \cornell{What were the early causes of the creation of a two-party system once again?}{\textbf{Jackson's tactics in preventing nullification and in suppressing the National Bank polarized a large group of opponents who believed his policies were tyrannical; they named themselves the Whigs, after the English party who wanted to limit the power of the monarch.}}
            \cornell{How did the Democrats push for opportunity?}{\textbf{The 1830s Democrats sought continual expansion of opportunity for white males woth a reduced goal of government but also few obstacles to opportunity, defending the Union to promote dynamic economic growth. They believed that no man deserved artificla privilege; instead, all should understand that they rely on their own industry. The most radical Democrats, "Locofocos," sought a violent attack on privilege and monopoly.}}
            \cornell{How did Whigs favor economic union?}{\textbf{The Whigs favored expansion of federal power, commercial development, and an advanced economic union; they believed in the importance of material progress but feared instability brought by western expansion. They thus sought America to grow into an international industrial power with advanced institutions.}}
            \cornell{How did the supporters of the Whigs and Democrats generally differ?}{\textbf{The Whig party had the tighest hold on large Northeastern merchants and manufacuters and Southern planters with major northern ties. Democrats, contrastingly, generally earned the support of smaller merchants/workingmen, those hesitant about industrial expansion, and from westerners based on an agrarian economy. Whigs were generally wealthier, more aristocratic, and economically aspiring.}}
            \cornell{How were Democrats and Whigs most alike?}{
                Both parties sought primarily to win elections, often forsaking basic ideological components for a greater chance of winning. The clearest example is the Whig alliance with the Anti-Masons.
                    \begin{itemize}
                        \item 1820s movement to resist secret and exclusive society of Freemasons, further aggravted in 1826 when William Morgan, former Mason, vanished after soon being due to publish exposing paper
                        \item Whigs harshly rebuked Jackson/Van Buren, both Freemasons, claiming Democrats were part of undemocratic conspiracy
                    \end{itemize}
                \textbf{Democrats and Whigs were most similar in that neither were rigid philosophical believers, instead seeking primarily to win elections. This was most clearly seen in the Whig alliance with the Anti-Masons, a society which opposed the exclusive society of Freemasons; Jackson's status as a Freemason led the Whigs to ally with the Anti-Masons, rebuking the Democrats as exclusive and undemocratic.}}
                \cornell{What were the cultural differences between Whigs and Democrats?}{\textbf{Irish and German Catholics (majority of immigrants) generally supported Democrats due to aversion to widespread commercial development, while Evangelical Protestants generally supported Whigs due to constant improvement. Cultural differences thus began to set a greater precedent for party alliances than political beliefs.}}
                \cornell{Who were the key figures of the Whig Party?}{\begin{itemize}
                    \item Never united under a single figure like the Democrats with Jackson, Whigs divided into "Great Triumvirate" of Clay, Webster, Calhoun
                    \item Clay pushed for American System, but Western identification -> never broad support (failed for president 3 times)
                    \item Webster's passionate speeches defending the Constitution, Union counterracted by connection w/ Bank of U.S. and protective tariff
                    \begin{itemize}
                        \item Depended greatly on wealthy for financial support, known for fondness of brandy
                    \end{itemize}
                    \item Calhoun never true Whig due to nullification controversy, but had strong support in south
                \end{itemize}
                \textbf{The Whigs were divided into the three powerful leaders of Clay, pushing for the American system but limited by widespread perception of deviance and western idefnitifaction; Webster, a brilliant orator but too reliant on on the wealthy and pushing for the protective tarriff; and Calhoun, sharing the other two leaders' hatred for Jackson but undermined by the nullification crisis.}}
                \cornell{What were the effects of the divided Whig leadership?}{\begin{itemize}
                    \item Election of 1836 saw Democrats united behind Van Buren but Whigs unable to select one
                    \item Whigs ultimately selected three candidates (Webster for New England, Hugh L. White for the South, Harrison for middle states and West) each regionally powerful, hoping to prevent electoral majority and send election to House
                    \item Failed, with Van Buren receiving a large electoral majority
                \end{itemize}
                \textbf{The Whigs inability to unite behind one leader led them to send three candidates in hopes of preventing an electoral majority; however, ultimately, the Democratic unity behind Van Buren saw a significant majority.}}
                \cornell{What was the major economic boom before Van Buren's presidency and during the election?}{\begin{itemize}
                    \item Van Buren never reached public popularity of retired AJ: early success due to economic boom due to canals, railroads, easy credit, booming land business w/ large speculator purchases, budget surpluses in treasury 
                    \item Congress unsure how to handle budget surplus, ruling out idea of lowering tariff, instead passed popular Distribution Act to return wealth to states in four quarterly "loan" (not to be repaid) installments
                    \begin{itemize}
                        \item Immediately put to use for state infrastructure, further stimulating economic boom in states
                    \end{itemize}
                    \item Congress left speculative fever unchecked: Jackson was suspicious of paper currency and preferred selling land for state bank notes
                    \begin{itemize}
                        \item Jackson, in 1836 soon before departure from office, issued "specie circular" in hopes of curing speculative fever, allowing only gold/silver coins or currency backed by physical reserves for land payments
                    \end{itemize}
                \end{itemize}
                \textbf{Van Buren's election success was due in large part to Jackson's economic boom with great transporation infrastructure and the booming land business leading to budget surpluses. Congress further stimulated this under Jackson by redistributing the surpluses to the states to further build infrastructure. However, Jackson did little to stop the speculative fever, only requiring that physical metal or currencies backed up by reserved be used to buy land in his "specie circular."}}
                \cornell{What early events plagued Van Buren's presidency?}{\begin{itemize}
                    \item Jackson's specie circular plagued Van Buren's early presidency with the widespread failure of hundreds of banks/businesses, high unemployment, bread riots in cities, falling prices of land, state governments ceased bond payments
                    \begin{itemize}
                        \item Represented largest depression to that point in American history, political disaster for Democrats 
                    \end{itemize}
                    \item Panic fault of both parties, also foreign nations: England/W. Europe faced their own panics -> investors withdrew U.S. funds; paired with crop failures requiring more imports, hurt overall wealth greatly
                    \item Van Buren administration opposed federal intervention, with only steps potentially aggravating situation, like borrowing money to pay debts, accepting only specie for tax payment
                    \item Some policies slightly beneficial, like ten hour workday for federal projects and new financial system known as "independent treasury"
                    \begin{itemize}
                        \item Independent treasury entailed placing all funds in independent treasury w/o any private banks having govt. money: govt./banks completely separate 
                        \item Failed initially in 1837, but finally passed in 1840
                    \end{itemize}
                \end{itemize}
                \textbf{Van Buren's presidency was plagued by economic disasters; although the Whigs, Democrats, and foreign powers were all to blame, the Democratic majority and Van Buren's decision not to intervene saw the Democrats take the majority of the blame. Van Buren's policies often aggravated the situation, but some, including his ten-hour workday and the independent treasury making government funds independent of banks.}}
                \cornell{What was the Whig log cabin campaign?}{\begin{itemize}
                    \item As election of 1840 approached, Whigs understood significance of choosing only one candidate
                    \begin{itemize}
                        \item Held national nominating convention in Harrisburg, PA; completely ignored Clay, instead selecting Harrison, a renowned for president, John Tyler as vice president
                    \end{itemize}
                    \item Newfound Democratic disunity -> Van Buren once again as president but no vice president selected (electors would choose)
                    \item 1840 campaign represented multiple milestones
                    \begin{itemize}
                        \item First with widespread news distributed to people due to "penny press"
                        \item Illustrated true extent of party competition: both Democrats and Whigs appealed to the common people for voter appeal despite Whig reputation as aristocratic
                        \begin{itemize}
                            \item Philosophical purity lost in favor of votes
                            \item Harrison presented as wealthy frontier elite with simple desires, living in a log cabin; Whigs presented Van Buren as drunk, aloof aristocrat
                        \end{itemize}
                    \end{itemize} 
                    \item Democrats unable to resist tactics of Whigs -> Harrison won significantly electorally
                \end{itemize}
                \textbf{The Whigs understood the importance of choosing a single candidate after their 1836 failure, ultimately nominating Harrison; the Democrats renominated Van Buren but were disunited to the extent that no vice president was selected. Their "log cabin campaign" presented Harrison as a simple, wealthy frontiersman who represented the common man; the Democrats' inability to resist these policies led Harrison to win by a significant margin.}}
                \cornell{What were the significant power struggles during the first presidency?}{\begin{itemize}
                    \item Harrison ("Old Tippecanoe"), died of pneumonia one month after taking office -> succeeded by Tyler, with weak ties to party
                    \item Harrison had consulted w/ Henry Clay and secretary of state Daniel Webster; Tyler, former Democrat in reaction to Jackson's policies, let previous alignment influence policies
                    \begin{itemize}
                        \item Despite abolishing Van Buren's independent treasury, raising tariff rates, refused rechartering of Bank of U.S. 
                        \item Vetoed many heavily Whig-supported bills -> finally read out of party, all cabinet members but Webster resigning; Webster finally resigned -> Calhoun (rejoined Democrats) replaced
                    \end{itemize}
                    \item New political group emerged with Tyler, band of conservative southern Whigs preparing to rejoin Democrats, with aristocratic party seeking to expand slavery/push for state rights soon planning to join "common man" Democrats 
                \end{itemize}
                \textbf{After Harrison died within a month and was succeeded by his VP John Tyler, the presidency took a decidedly Democratic turn: as a former Democrat, Tyler let many of his previous beliefs show through, often vetoing heavily-Whig sponsored bills. The Whig party separated from him and all his cabinet members resigned, creating a new political group based on conservative southern Whigs ready to rejoin the Democrats.}}
                \cornell{What were the key tensions with Britain during the Whig government?}{\begin{itemize}
                    \item Late 1830s saw GB/US close to war w/ rebellion in eastern Canada chartering American ship \textit{Caroline} for supplies
                    \begin{itemize}
                        \item British seized ship, burned it; refused to provide compensation to U.S. or reject
                        \item NY retaliated by arresting Canadian Alexander MacLeod, charging with murder of American killed in \textit{Caroline} incident
                        \begin{itemize}
                            \item British felt that because he acted on official orders, he did not deserve to be arrested
                            \item Webster feared foreign secretary's warning of potential war, but powerless to release: under NY jursidiction -> \textit{had} to be tried in NY states; NY finally let free
                        \end{itemize}
                    \end{itemize}
                    \item Canada/ME border frequently debated, with American/Canadian lumberjacks moving into Aroostook area -> "Aroostook War"
                    \item American ship \textit{Creole} from VA to New Orleans w/ 100 slaves who mutinied, sailed to British Bahamas where they were declared free -> angry southerners
                    \item Newly appointed GB government sought to reduce tensions w/ US -> sent Lord Ashburton to negotiate ME boundary; worked w/ Webster, reps. from MA and ME to form Webster-Ashburton Treaty of 1842
                    \begin{itemize}
                        \item Established northern boundary betw. Maine/New Brunswick surviving even today (gave around half of territory to U.S.)
                        \item ME/MA saw critical trade routes protected in northern U.S., southern Canada
                        \item Ashburton expressed reget for \textit{Caroline}/\text{Creole}, promising no future intervention w/ American ships
                    \end{itemize}
                \end{itemize}
                \textbf{Tensions were high between the British and Americans under Whig rule with a U.S. ship chartered for a Canadian rebellion unapologetically attacked by the British and the Americans retaliating by arresting the main perpetrator. These were further aggravated by border conflicts between Maine and Canada and the release of slaves who mutinied on an American ship. These conflicts were mostly resolved by the Webster-Ashburton Treaty of 1842, establishing a formal boundary, protecting important trade routes, and apologizing for British atacks on American ships.}}
                \cornell{What marked the US' first diplomatic relations w/ China?}{\begin{itemize}
                    \item Tyler administration sought to get involved in British opening of China to foreign trade, sending commissioner Caleb Cushing to China to negotiate treaty to allow trade
                    \item Treaty of Wang Hya secured same privileges as British and extraterritoriality, the right of Americans in China to be tried in American courts
                    \begin{itemize}
                        \item Promoted steady increase in trade
                    \end{itemize}
                \end{itemize}
                \textbf{The Tyler administration sent Cushing to China to negotiate the Wang Hya treaty, securing the same trading privileges as the British as well as the right for Americans in China to be tried in American court, seeing an increase in American-Chinese trade.}}
    \end{document}