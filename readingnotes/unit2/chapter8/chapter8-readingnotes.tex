\documentclass[a4paper]{article}
    \usepackage[T1]{fontenc}
    \usepackage{tcolorbox}
    \usepackage{amsmath}
    \tcbuselibrary{skins}
    
    \title{
    \vspace{-3em}
    \begin{tcolorbox}
    \Huge\sffamily \begin{center} AP US History  \\
    \LARGE Chapter 8 - Varieties of American Nationalism \\
    \Large Finn Frankis \end{center} 
    \end{tcolorbox}
    \vspace{-3em}
    }
    \date{}
    \author{}
    
    \usepackage{background}
    \SetBgScale{1}
    \SetBgAngle{0}
    \SetBgColor{red}
    \SetBgContents{\rule[0em]{4pt}{\textheight}}
    \SetBgHshift{-2.3cm}
    \SetBgVshift{0cm}
    \usepackage[margin=2cm]{geometry} 
    
    \makeatletter
    \def\cornell{\@ifnextchar[{\@with}{\@without}}
    \def\@with[#1]#2#3{
    \begin{tcolorbox}[enhanced,colback=gray,colframe=black,fonttitle=\large\bfseries\sffamily,sidebyside=true, nobeforeafter,before=\vfil,after=\vfil,colupper=blue,sidebyside align=top, lefthand width=.3\textwidth,
    opacityframe=0,opacityback=.3,opacitybacktitle=1, opacitytext=1,
    segmentation style={black!55,solid,opacity=0,line width=3pt},
    title=#1
    ]
    \begin{tcolorbox}[colback=red!05,colframe=red!25,sidebyside align=top,
    width=\textwidth,nobeforeafter]#2\end{tcolorbox}%
    \tcblower
    \sffamily
    \begin{tcolorbox}[colback=blue!05,colframe=blue!10,width=\textwidth,nobeforeafter]
    #3
    \end{tcolorbox}
    \end{tcolorbox}
    }
    \def\@without#1#2{
    \begin{tcolorbox}[enhanced,colback=white!15,colframe=white,fonttitle=\bfseries,sidebyside=true, nobeforeafter,before=\vfil,after=\vfil,colupper=blue,sidebyside align=top, lefthand width=.3\textwidth,
    opacityframe=0,opacityback=0,opacitybacktitle=0, opacitytext=1,
    segmentation style={black!55,solid,opacity=0,line width=3pt}
    ]
    
    \begin{tcolorbox}[colback=red!05,colframe=red!25,sidebyside align=top,
    width=\textwidth,nobeforeafter]#1\end{tcolorbox}%
    \tcblower
    \sffamily
    \begin{tcolorbox}[colback=blue!05,colframe=blue!10,width=\textwidth,nobeforeafter]
    #2
    \end{tcolorbox}
    \end{tcolorbox}
    }
    \makeatother

    \parindent=0pt
    
    \begin{document}
    \maketitle
    \SetBgContents{\rule[0em]{4pt}{\textheight}}
    \cornell[Key Concepts]{What are this chapter's key concepts?}{\begin{itemize}
        \item \textbf{4.1.I.A} - Political parties continued to debate tariff, federal power, European relations
        \item \textbf{4.1.I.B} - Supreme Court decisions established the judiciary influence over Constitutional interpretation; federal laws over state laws
        \item \textbf{4.3.I.A} - Louisiana Purchase -> U.S. government sought control over North America through exploration, war, native encroachment, diplomatic expansion
        \item \textbf{4.3.II.A} - Southern overcultivation -> plantations relocations to lands west of Appalachians
        \item \textbf{4.3.II.C} - Attempts at political compromise (like Missouri Compromise) temporarily stemmed tensions between opponents/defenders of slavery
    \end{itemize}}
    \cornell[Building a National Market]{How did the United States build a national market?}{\textbf{Economically, the United States saw great growth in government funds due to a national bank and a manufacturing sector particularly in textiles, which was protected from foreign intervention by Congress' bills. Transportation, too, was expanded: construction began on a National Road and turnpikes began to extend over the mountains; steam power became an essential part of transporting manufactures. However, large gaps still remained as evidenced by the War of 1812, and a constitutional solution remained unclear.}}
    \cornell{What were the economic effects and repercussions of the War of 1812?}{\begin{itemize}
        \item Post war boom of War of 1812 -> economic boom which resulted in economic bust by 1819, evidencing U.S. lack in basic infrastructure
        \item War of 1812 -> growth in manufaccturing by cutting off imports, but also shipping chaos due to inadequate transportation/financial systems 
        \begin{itemize}
            \item Aftermath of war saw multiple political issues related to economic development
        \end{itemize}
        \item Development of national bank prioritized
        \begin{itemize}
            \item After 1811 expiration of Hamilton's, state banks had begun to operate but often provided more bank notes than their true reserves -> multiple notes of different value circulating at once, business mess
            \item Congress chartered second Bank of United States in 1816; only difference to Hamilton's was slightly more capital
            \begin{itemize}
                \item Unable to forbid state banks from issuing currency, but size -> dominated state banks, potentially forcing them out of business
            \end{itemize}
        \end{itemize}
        \item Promoted manufacturing sector: small wartime industry -> any factory even with unskilled labor could start operations
        \begin{itemize}
            \item Textile industry saw particular growth with increase in cotton spindles; factories adapted to British textile machinery to transition from home-loomed goods to factory
            \begin{itemize}
                \item Boston merchant Lowell created power loom superior to British, organizing Boston Manufacturing Company and founding first American mill
                \item Revolutionized manufacturing, shaped early workforce
            \end{itemize}
            \item End of war of 1812 dimmed American prospects with British ships providing low-cost goods to Americans
            \begin{itemize}
                \item Parliament hoped to stifle American industry by initially incurring a slight loss
                \item Congress protectionists saw tariff law to limit competition from abroad, allowing local industries to develop
                \begin{itemize}
                    \item Despite protest from many agriculturalists who would have to pay more, American industry prevailed
                \end{itemize}
            \end{itemize}
        \end{itemize}
    \end{itemize}
    \textbf{After the War of 1812, a national bank was prioritized to replace Hamilton's, with the second Bank of United States given a larger amount of capital; its monopoly allowed it to threaten states out of issuing currency. The manufacturing industry, too, grew, with textiles, specifically, moving form at-home production to factories after the invention of Lowell's power mill. The government also passed policies to prevent American industry from being taken over by foreign powers, particularly the British.}}
    \cornell{How did the United States build their transportation industry in the early nineteenth century?}{\begin{itemize}
        \item Manufacturers needed to access domestic markets for raw materials -> debate emerged whether federal government should finance roads directly
        \item As early as 1803, Ohio made agreement with federal government that some land revenue would go to financing roads
        \begin{itemize}
            \item Jefferson's treasury secretary Gallatin proposed that Ohio land revenues go toward National Road development; approved by both Congress and the president
            \item Construction began in 1811; by 1818, ran to Wheeling, VA
        \end{itemize}
        \item PA funded private company to extend pike to Pittsburgh, allow for stagecoaches, wagons, carriages, and cattle to transport
        \begin{itemize}
            \item High tolls funded efficient route over mountains
            \item Stimulated manufacture transport (especially textiles) from Atlantic seaboard to Ohio Valley
        \end{itemize}
        \item Great Lakes rivers saw steam-powered expansion (temporarily halted by War of 1812)
        \begin{itemize}
            \item By 1816, steamers down Mississippi to Ohio River, up Ohio to Pittsburgh
            \item Carried more cargo than all other previous river transportation combined 
            \item Stimulated western/southern agricultural economy with ready access to markets; enabled eastern manufacturers to send goods west
        \end{itemize}
        \item Major gaps remained, especially seen in War of 1812
        \begin{itemize}
            \item British blockade ended Atlantic shipping -> coastal roads overtaken by large volume of N-S traffic hoping to reach ferries to cross rivers
            \begin{itemize}
                \item Oxcarts sent for emergencies with 6-7 week travel time between Philadelphia/Charleston
                \item Goods normally sent by sea saw many northern states lacking in materials -> prices soared (like rice in NY, flour in Boston)
                \item Military challenges with inability to travel by sea, poor roads -> campaigns often unsuccessful
            \end{itemize}
        \end{itemize}
        \item Madison called Congress to establish national road system backed up by constitutional amendment
        \begin{itemize}
            \item Calhoun proposed bill allowing funds owed to government by U.S. Bank to finance internal improvements
            \item Congress passed, but Madison vetoed on final day in office; despite supporting purpose, believed Congress lacked authority 
            \begin{itemize}
                \item Forced state governments to continue handling task
            \end{itemize}
        \end{itemize}
    \end{itemize}
    \textbf{The early nineteenth century saw great transportation developments, particularly in OH, where the federal government promised to divert some land funds to road construction; in PA, where the government funded a private company to extend a large pike over the mountains to Pittsburgh; and steam transportation from the Great Lakes rivers, where a large number of goods were transported. However, large gaps remained in the system, particularly in the War of 1812, where the British blockade prevented the smooth transfer of goods and military. Madison urged Congress to establish a road system, but ultimately vetoed the relevant bill due to its divergence from the Constitution. }}

    \end{document}