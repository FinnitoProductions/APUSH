\documentclass[a4paper]{article}
    \usepackage[T1]{fontenc}
    \usepackage{tcolorbox}
    \usepackage{amsmath}
    \tcbuselibrary{skins}
    
    \title{
    \vspace{-3em}
    \begin{tcolorbox}
    \Huge\sffamily \begin{center} AP US History  \\
    \LARGE Chapter 9 - Jacksonian America \\
    \Large Finn Frankis \end{center} 
    \end{tcolorbox}
    \vspace{-3em}
    }
    \date{}
    \author{}
    
    \usepackage{background}
    \SetBgScale{1}
    \SetBgAngle{0}
    \SetBgColor{red}
    \SetBgContents{\rule[0em]{4pt}{\textheight}}
    \SetBgHshift{-2.3cm}
    \SetBgVshift{0cm}
    \usepackage[margin=2cm]{geometry} 
    
    \makeatletter
    \def\cornell{\@ifnextchar[{\@with}{\@without}}
    \def\@with[#1]#2#3{
    \begin{tcolorbox}[enhanced,colback=gray,colframe=black,fonttitle=\large\bfseries\sffamily,sidebyside=true, nobeforeafter,before=\vfil,after=\vfil,colupper=blue,sidebyside align=top, lefthand width=.3\textwidth,
    opacityframe=0,opacityback=.3,opacitybacktitle=1, opacitytext=1,
    segmentation style={black!55,solid,opacity=0,line width=3pt},
    title=#1
    ]
    \begin{tcolorbox}[colback=red!05,colframe=red!25,sidebyside align=top,
    width=\textwidth,nobeforeafter]#2\end{tcolorbox}%
    \tcblower
    \sffamily
    \begin{tcolorbox}[colback=blue!05,colframe=blue!10,width=\textwidth,nobeforeafter]
    #3
    \end{tcolorbox}
    \end{tcolorbox}
    }
    \def\@without#1#2{
    \begin{tcolorbox}[enhanced,colback=white!15,colframe=white,fonttitle=\bfseries,sidebyside=true, nobeforeafter,before=\vfil,after=\vfil,colupper=blue,sidebyside align=top, lefthand width=.3\textwidth,
    opacityframe=0,opacityback=0,opacitybacktitle=0, opacitytext=1,
    segmentation style={black!55,solid,opacity=0,line width=3pt}
    ]
    
    \begin{tcolorbox}[colback=red!05,colframe=red!25,sidebyside align=top,
    width=\textwidth,nobeforeafter]#1\end{tcolorbox}%
    \tcblower
    \sffamily
    \begin{tcolorbox}[colback=blue!05,colframe=blue!10,width=\textwidth,nobeforeafter]
    #2
    \end{tcolorbox}
    \end{tcolorbox}
    }
    \makeatother

    \parindent=0pt
    
    \begin{document}
    \maketitle
    \SetBgContents{\rule[0em]{4pt}{\textheight}}
    \cornell[Key Concepts]{What are this chapter's key concepts?}{\begin{itemize}
        \item \textbf{4.1.I.C} - New Whig Party, led by Henry Clay, emerged by 1820s; disagreed with Jackson's Democrats on role/powers of federal government, issues such as national bank, tariffs, federal funding
        \item \textbf{4.1.I.D} - Regional interests often prioritized over national concerns for leaders' positions on slavery/economy
        \item \textbf{4.2.III.D} - American System (among other plans to unify U.S. economy), caused debates over whether agriculture or industry would be favored
        \item \textbf{4.3.I.A} - U.S. government sought to control North America, Western Hemisphere through exploration, removal of natives, Monroe Doctrine
        \item \textbf{4.3.I.B} - Frontier settlers championed expansion efforts; native resistance -> sequence of wars, federal efforts for relocation
    \end{itemize}}
    \cornell[The Rise of Mass Politics]{How did Jackson usher in the rise of politics controlled by the masses?}{\textbf{Jackson, known for his success in the battle of New Orleans, was predated by but also led a great increase in democracy: before him, many states began to expand suffrage beyond propertyholders (with many compromises), but there was still significant room to improve in African American, women, and native suffrage. Jackson, himself, legitimized the partisan system through his election and aimed to appeal to the common people by dethroning the long-standing elites in favor of commoners for rule; ultimately, he appealed more to his followers than the masses.}}
    \cornell{What characterized Jackson's inauguration?}{\textbf{Thousands of Americans from throughout the country of all social ranks crowded before Washington Capitol to watch his inaguration, with the crowd following their hero to the White House after the ceremonies concluded. They followed him into the White House, ruining the elegant furniture and carpetry to meet the new president. Many viewed it as a mob uprising, however.}}
    \cornell{What was Andrew Jackson's history?}{\begin{itemize}
        \item Born in 1767 to Irish parents in Carolinas; captured by British during Revolution at age of 13, injured after refusing to clean boot -> enduring hatred
        \item Sporadic education, various shops/farms; studied law, was admitted to practice
        \item Known for extensive yet dynamic political history
        \begin{itemize}
            \item Early work was basic land claims 
            \item Soon elected as delegate to TN constitutional convention, becoming Congressman in same year (1797)
            \item Resigned as Senator after one year; appointed to TN Supreme Court from 1798-1804
        \end{itemize}
        \item Became prosperous as planter/merchant with slaves in elegant Nashville plantation; among largest in state
        \item Joined militia in 1801, soon became general; fought natives in AL/GA; fought British in War of 1812 with decisive victory in Battle of New Orleans
        \begin{itemize}
            \item War of 1812 -> many hailed as hero, called for him to run for president
        \end{itemize}
    \end{itemize}
    \textbf{Andrew Jackson started with modest beginnings but eventually worked up to a career in law. He was a delegate to the constitutional convention of Tennessee, a Congressman, a Senator, a member of the Tennessee Supreme Court, a planter and merchant, and a militiaman.}}
    \cornell{What were the major democratic changes occurring pre-Jackson?}{
        Jackson had little effect on economic equality, instead giving many more the right to vote. Up to the 1820s, most states had restricted right to vote to white male property owners and taxpayers, causing the less wealthy to be left out. However, change began in the western states and rapidly spread elsewhere to the eastern states fearful of emigration.
        \begin{itemize}
            \item Significant change -> great resistance, as in MA constitutional convention of 1820
            \begin{itemize}
                \item Radicals demanded more poor representation, but conservatives resisted (including Daniel Webster) -> property requirement eliminated
                \item Still had to be taxpayer to vote, own real estate to be governor
            \end{itemize}
            \item Often far more successful, like in NY convention of 1821
            \begin{itemize}
                \item Conservatives insisted that tax-paying requirement was insufficient, at least for state senators
                \item Reformers cited Decl. of Indep., maintaining that property was not a fundamental concern of society -> requirement abolished
            \end{itemize}
            \item Changes which went through often -> great instability
            \begin{itemize}
                \item RI constitution (old colonial charter) had barred half of adult males from voting with conservative legislature blocking any reform efforts
                \item Lawyer Thomas W. Dorr under "People's Party" drafted new constitution, submitted for pop. vote -> overwhelming approval but legislatures would not accept, submitting their own constitution (but narrowly defeated)
                \item Dorr became governor in eyes of followers (Dorr Rebellion) -> old government proclaimed insurrection, began to imprison Dorrites 
                \begin{itemize}
                    \item Unable to capture state arsenal; Dorr surrendered
                    \item Long-term effect of drafting new constitution to expand suffrage
                \end{itemize}
            \end{itemize}
        \end{itemize}
        \textbf{Many began to call for universal suffrage, with significant resistance often leading to compromises (like in MA, with taxpaying requirement still intact), other states experiencing great success (like NY by citing the Declaration of Independence), and others seeing instability from great changes (like in RI with the Dorr Rebellion.}}
        \cornell{How could the U.S. still improve in terms of suffrage?}{\begin{itemize}      
            \item In south, little overall success w/ election laws continually favoring plantation owners, politicians of past
            \item Even in North, blacks were unable to vote (PA removed right in 1838)
            \item Women could vote in no states
            \item Voting was undemocratic: ballot not secret with spoken vote -> politicans could bribe/intimidate
        \end{itemize}
        \textbf{The U.S. still needed many significant improvements to approach universal suffrage, including removing the property restriction/bias in the south, allowing African Americans and women to vote universally (no states allowed it), and changing voting to a more democratic format without rampant bribery.}}
        \cornell{What were some significant democratic reforms in the early nineteenth century?}{\textbf{The method of selecting presidential electors changed drastically, with a gradual shift from legislature selection to popular vote: by 1828, all states had shifted but SC. This allowed the number of voters to increase drastically over time.}}
        \cornell{Who was Tocqueville and how did he view American democracy?}{\begin{itemize}
            \item French aristocrat Alexis de Tocqueville, intrigued, documented growth of electorate, shift to political parties, rapid spread in right to vote after spending two years in U.S. during Jackson's time
            \begin{itemize}
                \item Sent to study American prisons for humane influence; went beyond to write \textit{Democracy in America}, examining daily lives of key Americans and cultures, associations, democratic visions
            \end{itemize}
            \item French democracy had been restricted to landowners/aristocrats; Tocqueville realized failing aristocracy
            \item Tocqueville understood limits of democracy: favored white men, remaining a distant hope for many
            \item Ultimately spread American democracy to France, other European nations
        \end{itemize}
        \textbf{Tocqueville wrote \textit{Democracy in America}, examining American democracy and noting the significance of the collapsing aristocracy but also of the limits, still restricted to mostly white men and leaving out women, blacks, and natives. In the long term, he spread the ideals American democracy to many European nations.}}
        \cornell{How was the party system legitimized over time?}{\begin{itemize}
            \item Voter participation also result of interest in politics, party organization, party loyalty
            \item Initially, parties viewed as evil entities, with many believing nation should come to consensus without factional lines
            \item Quickly fell apart in 1820s, 1830s, believing discordant parties were key to democracy, beginning at state level
            \begin{itemize}
                \item Van Buren's post-War of 1812 political faction (Bucktails) in NY state challenging established aristocratic leadership (under De Witt Clinton)
                \begin{itemize}
                    \item Argued that institutionalized party could secure democracy unlike Clinton's closed elite
                    \item Proposed ideological committments trumped by party loyalty with main goal to preserve party's success
                    \item Competing parties required for any given party to survive -> would force politicians to remain interested in will of people to balance each other
                \end{itemize}
                \item Jackson's election further legitimized on federal level
                \begin{itemize}
                    \item Two-party system became official with legitimate institution under powerful opponents: Whigs
                    \item Jackson's followers were officially Democrats
                \end{itemize}
            \end{itemize} 
        \end{itemize}
        \textbf{Parties were long viewed as evil and destructive, going directly against democracy. However, beginning with Martin Van Buren's political faction, the Bucktails, in New York state, with the argument that competing parties and party loyalty were essential factors to appeal to the people, parties were quickly legitimized on the state level. Jackson's election further legitimized it on the federal level: his followers were the Democrats and the opposition formally became the Whigs.}}
        \cornell{How did Jackson appeal to the common people?}{\begin{itemize}
            \item Jackson not a philosopher (unlike Jefferson) -> no uniform ideology, but embraced simple theory of democracy: to offer equal protection to all white males regardless of class
            \begin{itemize}
                \item Represented direct assault on eastern aristocracy, effort to promote rising westerners/southerners
                \item Justified subjugation of African Americans/natives: to protect white men
            \end{itemize}
            \item Jackson first targeted federal officeholders, many of whom had ruled for over a generation; believed offices belonged to people, not long-term holders
            \begin{itemize}
                \item Removed no more than one-fifth during eight years, mostly for misuse of funds or corruption 
                \item Dismissed no more than Jefferson, but philosophy gave future elected officials right to appoint their own followers to office
            \end{itemize}
            \item Transformed process to win party's nomination, resenting caucuses for favoring elites 
            \begin{itemize}
                \item For 1832 election, established partisan convention to nominate for presidency; creators saw as democratic triumph where power would arise from people
            \end{itemize}
        \end{itemize}
        \textbf{Jackson's main goal was to limit the power of the elites and give the lower classes the opportunity to rise: he accomplished this by targeting long-standing federal officeholders (though few were ultimately removed) and by replacing the caucus system with a partisan convention to nominate the presidential candidate. In all, Jackson did successfully detrench the elites, but mostly transferred power to his own allies: those nominated at the conventions were rarely common men.}}
    \end{document}