\documentclass[a4paper]{article}
    \usepackage[T1]{fontenc}
    \usepackage{tcolorbox}
    \usepackage{amsmath}
    \tcbuselibrary{skins}
    
    \title{
    \vspace{-3em}
    \begin{tcolorbox}
    \Huge\sffamily \begin{center} AP US History  \\
    \LARGE Chapter 9 - Jacksonian America \\
    \Large Finn Frankis \end{center} 
    \end{tcolorbox}
    \vspace{-3em}
    }
    \date{}
    \author{}
    
    \usepackage{background}
    \SetBgScale{1}
    \SetBgAngle{0}
    \SetBgColor{red}
    \SetBgContents{\rule[0em]{4pt}{\textheight}}
    \SetBgHshift{-2.3cm}
    \SetBgVshift{0cm}
    \usepackage[margin=2cm]{geometry} 
    
    \makeatletter
    \def\cornell{\@ifnextchar[{\@with}{\@without}}
    \def\@with[#1]#2#3{
    \begin{tcolorbox}[enhanced,colback=gray,colframe=black,fonttitle=\large\bfseries\sffamily,sidebyside=true, nobeforeafter,before=\vfil,after=\vfil,colupper=blue,sidebyside align=top, lefthand width=.3\textwidth,
    opacityframe=0,opacityback=.3,opacitybacktitle=1, opacitytext=1,
    segmentation style={black!55,solid,opacity=0,line width=3pt},
    title=#1
    ]
    \begin{tcolorbox}[colback=red!05,colframe=red!25,sidebyside align=top,
    width=\textwidth,nobeforeafter]#2\end{tcolorbox}%
    \tcblower
    \sffamily
    \begin{tcolorbox}[colback=blue!05,colframe=blue!10,width=\textwidth,nobeforeafter]
    #3
    \end{tcolorbox}
    \end{tcolorbox}
    }
    \def\@without#1#2{
    \begin{tcolorbox}[enhanced,colback=white!15,colframe=white,fonttitle=\bfseries,sidebyside=true, nobeforeafter,before=\vfil,after=\vfil,colupper=blue,sidebyside align=top, lefthand width=.3\textwidth,
    opacityframe=0,opacityback=0,opacitybacktitle=0, opacitytext=1,
    segmentation style={black!55,solid,opacity=0,line width=3pt}
    ]
    
    \begin{tcolorbox}[colback=red!05,colframe=red!25,sidebyside align=top,
    width=\textwidth,nobeforeafter]#1\end{tcolorbox}%
    \tcblower
    \sffamily
    \begin{tcolorbox}[colback=blue!05,colframe=blue!10,width=\textwidth,nobeforeafter]
    #2
    \end{tcolorbox}
    \end{tcolorbox}
    }
    \makeatother

    \parindent=0pt
    
    \begin{document}
    \maketitle
    \SetBgContents{\rule[0em]{4pt}{\textheight}}
    \cornell[Key Concepts]{What are this chapter's key concepts?}{\begin{itemize}
        \item \textbf{4.1.I.C} - New Whig Party, led by Henry Clay, emerged by 1820s; disagreed with Jackson's Democrats on role/powers of federal government, issues such as national bank, tariffs, federal funding
        \item \textbf{4.1.I.D} - Regional interests often prioritized over national concerns for leaders' positions on slavery/economy
        \item \textbf{4.2.III.D} - American System (among other plans to unify U.S. economy), caused debates over whether agriculture or industry would be favored
        \item \textbf{4.3.I.A} - U.S. government sought to control North America, Western Hemisphere through exploration, removal of natives, Monroe Doctrine
        \item \textbf{4.3.I.B} - Frontier settlers championed expansion efforts; native resistance -> sequence of wars, federal efforts for relocation
    \end{itemize}}
    \cornell[The Rise of Mass Politics]{How did Jackson usher in the rise of politics controlled by the masses?}{\textbf{Jackson, known for his success in the battle of New Orleans, was predated by but also led a great increase in democracy: before him, many states began to expand suffrage beyond propertyholders (with many compromises), but there was still significant room to improve in African American, women, and native suffrage. Jackson, himself, legitimized the partisan system through his election and aimed to appeal to the common people by dethroning the long-standing elites in favor of commoners for rule; ultimately, he appealed more to his followers than the masses.}}
    \cornell{What characterized Jackson's inauguration?}{\textbf{Thousands of Americans from throughout the country of all social ranks crowded before Washington Capitol to watch his inaguration, with the crowd following their hero to the White House after the ceremonies concluded. They followed him into the White House, ruining the elegant furniture and carpetry to meet the new president. Many viewed it as a mob uprising, however.}}
    \cornell{What was Andrew Jackson's history?}{\begin{itemize}
        \item Born in 1767 to Irish parents in Carolinas; captured by British during Revolution at age of 13, injured after refusing to clean boot -> enduring hatred
        \item Sporadic education, various shops/farms; studied law, was admitted to practice
        \item Known for extensive yet dynamic political history
        \begin{itemize}
            \item Early work was basic land claims 
            \item Soon elected as delegate to TN constitutional convention, becoming Congressman in same year (1797)
            \item Resigned as Senator after one year; appointed to TN Supreme Court from 1798-1804
        \end{itemize}
        \item Became prosperous as planter/merchant with slaves in elegant Nashville plantation; among largest in state
        \item Joined militia in 1801, soon became general; fought natives in AL/GA; fought British in War of 1812 with decisive victory in Battle of New Orleans
        \begin{itemize}
            \item War of 1812 -> many hailed as hero, called for him to run for president
        \end{itemize}
    \end{itemize}
    \textbf{Andrew Jackson started with modest beginnings but eventually worked up to a career in law. He was a delegate to the constitutional convention of Tennessee, a Congressman, a Senator, a member of the Tennessee Supreme Court, a planter and merchant, and a militiaman.}}
    \cornell{What were the major democratic changes occurring pre-Jackson?}{
        Jackson had little effect on economic equality, instead giving many more the right to vote. Up to the 1820s, most states had restricted right to vote to white male property owners and taxpayers, causing the less wealthy to be left out. However, change began in the western states and rapidly spread elsewhere to the eastern states fearful of emigration.
        \begin{itemize}
            \item Significant change -> great resistance, as in MA constitutional convention of 1820
            \begin{itemize}
                \item Radicals demanded more poor representation, but conservatives resisted (including Daniel Webster) -> property requirement eliminated
                \item Still had to be taxpayer to vote, own real estate to be governor
            \end{itemize}
            \item Often far more successful, like in NY convention of 1821
            \begin{itemize}
                \item Conservatives insisted that tax-paying requirement was insufficient, at least for state senators
                \item Reformers cited Decl. of Indep., maintaining that property was not a fundamental concern of society -> requirement abolished
            \end{itemize}
            \item Changes which went through often -> great instability
            \begin{itemize}
                \item RI constitution (old colonial charter) had barred half of adult males from voting with conservative legislature blocking any reform efforts
                \item Lawyer Thomas W. Dorr under "People's Party" drafted new constitution, submitted for pop. vote -> overwhelming approval but legislatures would not accept, submitting their own constitution (but narrowly defeated)
                \item Dorr became governor in eyes of followers (Dorr Rebellion) -> old government proclaimed insurrection, began to imprison Dorrites 
                \begin{itemize}
                    \item Unable to capture state arsenal; Dorr surrendered
                    \item Long-term effect of drafting new constitution to expand suffrage
                \end{itemize}
            \end{itemize}
        \end{itemize}
        \textbf{Many began to call for universal suffrage, with significant resistance often leading to compromises (like in MA, with taxpaying requirement still intact), other states experiencing great success (like NY by citing the Declaration of Independence), and others seeing instability from great changes (like in RI with the Dorr Rebellion.}}
        \cornell{How could the U.S. still improve in terms of suffrage?}{\begin{itemize}      
            \item In south, little overall success w/ election laws continually favoring plantation owners, politicians of past
            \item Even in North, blacks were unable to vote (PA removed right in 1838)
            \item Women could vote in no states
            \item Voting was undemocratic: ballot not secret with spoken vote -> politicans could bribe/intimidate
        \end{itemize}
        \textbf{The U.S. still needed many significant improvements to approach universal suffrage, including removing the property restriction/bias in the south, allowing African Americans and women to vote universally (no states allowed it), and changing voting to a more democratic format without rampant bribery.}}
        \cornell{What were some significant democratic reforms in the early nineteenth century?}{\textbf{The method of selecting presidential electors changed drastically, with a gradual shift from legislature selection to popular vote: by 1828, all states had shifted but SC. This allowed the number of voters to increase drastically over time.}}
        \cornell{Who was Tocqueville and how did he view American democracy?}{\begin{itemize}
            \item French aristocrat Alexis de Tocqueville, intrigued, documented growth of electorate, shift to political parties, rapid spread in right to vote after spending two years in U.S. during Jackson's time
            \begin{itemize}
                \item Sent to study American prisons for humane influence; went beyond to write \textit{Democracy in America}, examining daily lives of key Americans and cultures, associations, democratic visions
            \end{itemize}
            \item French democracy had been restricted to landowners/aristocrats; Tocqueville realized failing aristocracy
            \item Tocqueville understood limits of democracy: favored white men, remaining a distant hope for many
            \item Ultimately spread American democracy to France, other European nations
        \end{itemize}
        \textbf{Tocqueville wrote \textit{Democracy in America}, examining American democracy and noting the significance of the collapsing aristocracy but also of the limits, still restricted to mostly white men and leaving out women, blacks, and natives. In the long term, he spread the ideals American democracy to many European nations.}}
        \cornell{How was the party system legitimized over time?}{\begin{itemize}
            \item Voter participation also result of interest in politics, party organization, party loyalty
            \item Initially, parties viewed as evil entities, with many believing nation should come to consensus without factional lines
            \item Quickly fell apart in 1820s, 1830s, believing discordant parties were key to democracy, beginning at state level
            \begin{itemize}
                \item Van Buren's post-War of 1812 political faction (Bucktails) in NY state challenging established aristocratic leadership (under De Witt Clinton)
                \begin{itemize}
                    \item Argued that institutionalized party could secure democracy unlike Clinton's closed elite
                    \item Proposed ideological committments trumped by party loyalty with main goal to preserve party's success
                    \item Competing parties required for any given party to survive -> would force politicians to remain interested in will of people to balance each other
                \end{itemize}
                \item Jackson's election further legitimized on federal level
                \begin{itemize}
                    \item Two-party system became official with legitimate institution under powerful opponents: Whigs
                    \item Jackson's followers were officially Democrats
                \end{itemize}
            \end{itemize} 
        \end{itemize}
        \textbf{Parties were long viewed as evil and destructive, going directly against democracy. However, beginning with Martin Van Buren's political faction, the Bucktails, in New York state, with the argument that competing parties and party loyalty were essential factors to appeal to the people, parties were quickly legitimized on the state level. Jackson's election further legitimized it on the federal level: his followers were the Democrats and the opposition formally became the Whigs.}}
        \cornell{How did Jackson appeal to the common people?}{\begin{itemize}
            \item Jackson not a philosopher (unlike Jefferson) -> no uniform ideology, but embraced simple theory of democracy: to offer equal protection to all white males regardless of class
            \begin{itemize}
                \item Represented direct assault on eastern aristocracy, effort to promote rising westerners/southerners
                \item Justified subjugation of African Americans/natives: to protect white men
            \end{itemize}
            \item Jackson first targeted federal officeholders, many of whom had ruled for over a generation; believed offices belonged to people, not long-term holders
            \begin{itemize}
                \item Removed no more than one-fifth during eight years, mostly for misuse of funds or corruption 
                \item Dismissed no more than Jefferson, but philosophy gave future elected officials right to appoint their own followers to office
            \end{itemize}
            \item Transformed process to win party's nomination, resenting caucuses for favoring elites 
            \begin{itemize}
                \item For 1832 election, established partisan convention to nominate for presidency; creators saw as democratic triumph where power would arise from people
            \end{itemize}
        \end{itemize}
        \textbf{Jackson's main goal was to limit the power of the elites and give the lower classes the opportunity to rise: he accomplished this by targeting long-standing federal officeholders (though few were ultimately removed) and by replacing the caucus system with a partisan convention to nominate the presidential candidate. In all, Jackson did successfully detrench the elites, but mostly transferred power to his own allies: those nominated at the conventions were rarely common men.}}
        \cornell["Our Federal Union"]{What were the critical tensions during Jackson's first term and how did this affect the political standings of many key players?}{\textbf{The most significant tension was Calhoun's proposed theory of nullification, which would allow states to nullify federal law; this tension escalated in the Webster-Hayne debate, where Hayne argued against Northeastern tyranny in favor of independent state control and the Jackson-supported Webster pushed for the Union and federal supremacy. Crisis emerged when SC attempted to nullify the Treaty of 1816, angering Jackson and potentially inciting violence; however, Clay's compromise of a gradual reduction solved this. Ultimately Calhoun's standing was reduced greatly in the eyes of Jackson while Van Buren, his greatest opponent, remained a strong ally of Jackson, and his influence was further strengthened by his social etiquette.}}
        \cornell{What fundamental beliefs pushed Jackson to despite the federal power?}{
        \textbf{Although he despised the concentration of power in Washington, reducing the potential of those without connections, and thus pushed an economic plan to restrict it, Jackson still asserted the overall control and power of the union in facing the theory of nullification.}}
        \cornell{What formed the basis of Calhoun's theory of nullification?}{\begin{itemize}
            \item 46 year-old Calhoun fell out of grace due to support for tariff of 1816 
            \begin{itemize}
                \item South Carolinians blamed tariff for economic stagnation 
                \item Realistically due to farmland exhaustion -> unable to compete w/ SW -> many called for secession 
            \end{itemize}
            \item Calhoun had to solve issue in home state -> secession alternative: nullification where states had could hold special council to nullify federal law
            \begin{itemize}
                \item Supported by many South Carolinians but not by federal government
                \item Rationale was because federal government was creation of states, not other way around
            \end{itemize}
        \end{itemize}
        \textbf{Calhoun, in an attempt to solve the economic stagnation emergent in SC leading to potential secession, introduced the controversial theory of nullification, which allowed states to hold special conventions to nullify federal law if declared unconstitutional.}}
        \cornell{How did Van Buren put himself in a favorable position with Jackson?}{\begin{itemize}
            \item Van Buren was governor of NY from 1828-1829, when Jackson appointed secretary of state 
            \item Quickly established as critical political ally, part of circle (known as "Kitchen Cabinet")
            \item Van Buren's influence over Jackson was immense, but further increased due to etiquette quarrel
            \begin{itemize}
                \item Peggy O'Neale, married daughter of Washington tavernkeeper who had lodged w/ senators Jackson + Eaton
                \item Eaton rumored to have affair w/ O'Neale; husband soon died (1828) and Eaton soon married her
                \item Eaton named secretary of war by Jackson, but other administration wives would not accept O'Neale -> Jackson furious, Van Buren jumped in and accepted family while Calhoun refused -> by 1831, chosen as successor
            \end{itemize}
        \end{itemize}
        \textbf{Van Buren, as a close friend and political ally of Jackson's, further expanded his influence over the presidency by accepting a cabinet wife of Jackson's friend who had committed adulterous acts.}}
        \cornell{What were the critical themes of the Webster-Hayne Debate?}{\begin{itemize}
            \item Senate debate unfolded in early 1830 as nullification crisis intensified where CT senator argued that land sales/surveys be slowed to reduce spread of slavery
            \begin{itemize}
                \item Supported by Daniel Webster, attacking Hayne and inadvertently Calhoun by subtly arguing conflict between state rights, national power
            \end{itemize}
            \item SC senator Hayne argued importance of Western lands to prevent dominance of East and boost position in Congress to lower tariff, prevent alleged joint tyranny of Northeast 
            \begin{itemize}
                \item Coached by Calhoun to argue for nullification; Webster presented "Second Reply to Hayne" over two full afternoons, with powerful conclusion concerning deep connectedness of liberty/union 
            \end{itemize}
            \item Jackson supported Webster's argument against Calhoun, made clear by push for strength of Union at Democratic Party banquet honoring Jefferson while looking directly at Calhoun and receiving Van Buren's support
        \end{itemize}
        \textbf{The Webster-Hayne Debate, despite being based on the surface around whether western expansion should be slowed to reduce slavery, was fundamentally an argument concerning state rights and national power: Hayne, an SC senator, pushed that Western lands were critical to prevent eastern dominance and northeastern tyranny, going against the central government in favor of that of his own state; Webster, supported by Jackson against Hayne and Calhoun, pushed against nullification to argue the conflict between state rights and national power. }}
        \cornell{What crisis emerged over nullification?}{\begin{itemize}
            \item When Congress passed a bill without any change in Tariff of 1816, SC called for state convention to nullify tariffs of 1828, 1832, appoint Hayne as governor and resigned Calhoun as senator
            \item Jackson called nullification treasonous, accused perpetrators of being traitors
            \begin{itemize}
                \item Sent warship to Charleston, bolstered federal military forts
                \item Proposed force bill to allow the president to use military to enforce Congressional acts
            \end{itemize}
            \item Senator Calhoun received no support from other states, even experienced great divisions within state 
            \item Crisis prevented by Clay, producing compromise allowing for steady decrease in tariffs to reach same level as 1816 by 1842 
            \begin{itemize}
                \item Passed on same day as force bill, both signed by Jackson
                \item SC recalled convention, repealed nullification; Calhoun claimed victory due to change in tariff but situation ultimately showed federal dominance
            \end{itemize}
        \end{itemize}
        \textbf{With the Carolinians outraged by Congress' lack of bills designed to limit the Tariff of 1816, nullification was implemented to eliminate the tariff. Violence seemed near: Jackson sent warships to Charleston and devised a bill to allow military authorization for Congressional treason. Calhoun lacked the support of any other state, pushing him to concede simply a change in the tariff which would allow a steady reduction.}}
        \cornell[The Removal of Indians]{What were Jackson's critical policies toward the natives?}{\textbf{Jackson, his war experience giving him a skeptical view on native society, generally shared the general hostile view that natives were complete savages incapable of recovery. These attitudes led to the Black Hawk War in the Northwest, known for vicious white armies slaughtering large numbers even after surrender, and the relocation of the "Five Civilized Tribes," most notably the civilized Cherokees, in the south along the Trail of Tears, brutal for all ages. In summary, many justified the native relocation to such smaller and unfamiliar territory as inevitable; however, this belief was justified against the potential of cohabitation (like in pueblos of NM) by describing the natives as savages without claim to their lands.}}
        \cornell{What were Jackson's personal motivations for relocating the natives?}{\textbf{Especially due to his military campaigns against southern tribes, Jackson sought to relocate the natives west beyond the Mississippi to further expand westward. Ultimately, his views were very similar to most other white Americans.}}
        \cornell{What were the general white attitudes toward native tribes?}{
            \textbf{The paternalistic 18th century attitudes that natives were "noble savages" perpetuated by Jefferson and allies were soon replaced by ones more hostile, with most viewing them as pure "savages." Furthermore, many white westerners feared potential war with the natives and sought to expand.}}
        \cornell{What events unfolded in the Black Hawk War?}{\begin{itemize}
            \item Northwest tensions w/ Indians climaxed in final battle in 1831-1832 between Illinois whites, Sauk/Fox native alliance led by warrior Black Hawk
            \item Treaty, but had been signed by rival tribe -> resentful Black Hawk reoccupied Illinois lands, federal troops saw as invaders
            \item White military especially vicious
            \begin{itemize}
                \item Sought to exterminate natives and ignored even Black Hawk's surrender
                \item Most fled across Mississippi into Iowa, but tribes followed and slaughtered most
                \item Black Hawk captured, sent on tour of east to meet "curious" whites (including Jackson)
            \end{itemize}
        \end{itemize}
        \textbf{The Black Hawk War unfolded in the northwest as tensions between natives and whites escalated - the warrior Black Hawk led the Sauk and Fox tribes into battle to resist a treaty made with an enemy tribe; the settlers responded viciously, exterminating most even after surrender and capturing their leader and parading him around the East.}}
        \cornell{What were the "Five Civilized Tribes" in the South?}{\begin{itemize}
            \item Cherokee, Creek, Seminole, Chikasaw, Choctaw had created agrarian societies with economy in South, particularly Cherokees w/ written languages, formal constitution
            \item Many whites believed civilization of Cherokees -> deserved to keep eastern lands because they had forsaken traditional ways of hunting and gathering and their somewhat egalitarian society in favor of white ways
            \item Federal government sought to negotiate to move all to West, but slow process -> states began to take action
            \begin{itemize}
                \item Georgia independently attempted to encroach Creeks despite Adams' demands
                \item Jackson's early administration saw states taking direct action but endorsed by Congress/Jackson
                \begin{itemize}
                    \item Passed Removal Act in 1830 to finance negotiations with tribes
                    \item As tribes faced federal/state governments, few could resist
                \end{itemize}
            \end{itemize}
            \item Cherokees attempted to stop Jackson's encouraged expansion by appealing to Supreme Court: \textit{Cherokee Nation v. Georgia}, \textit{Worcester v. Georgia} temporarily freed tribe 
            \begin{itemize}
                \item Jackson, seeking southern support and continually expressing distaste for native, even expressing public doubt that Marshall could enforce decision 
                \item Federal government produced treaty w/ minority (not truly representative) faction of Cherokees, providing \$5m and reservation west of Mississippi 
                \begin{itemize}
                    \item Few Cherokees initially followed until army of 7,000 led by Winfield Scott sent to force them out
                \end{itemize}
            \end{itemize}
        \end{itemize}
        \textbf{The "Five Civilized Tribes" were among the most difficult for the whites to uproot and move west: they had adapted to a large degree to agrarian society and white culture, so many felt their land deserved preservation, particularly the Cherokees for their written language and constitution. However, the Cherokee efforts to resist encroachment saw Jackson overpower the Supreme Court's ruling by forming a treaty with a minority group and forcing them to move west.}}
        \cornell{What hardships were endured on the Trail of Tears?}{\begin{itemize}
            \item $\approx 1,000$ Cherokees fled to NC; federal government eventually agreed to form reservation in Smoky Mountains (remains today)
            \item Rest underwent journey to "Indian Territory" in modern Oklahoma; known for brutality and suffering for those of all ages
            \begin{itemize}
                \item $\frac{1}{8}$ or more died before/soon after reaching forced destination 
                \item Journey known as "Trail of Tears"
                \begin{itemize}
                    \item Jackson justified by explaining that the native race was already too far gone
                \end{itemize}
                \item All civilized tribes forced to undergo journey; started w/ Choctaws, then Creeks, then Chickasaws, and finally Cherokees
            \end{itemize}
            \item Most of government believed territory was part of "Great American Desert," land unfit for explorers -> never any risk of additional conflict
            \item Only Seminoles resisted relocation somewhat; eventually, settled to move to territory within 3 years but minority remained behind with staged uprisings
            \begin{itemize}
                \item War continued for years beginning in 1835, with large number of American troops often outsmarted by guerilla warfare
                \item Leader, Osceola, captured while on truce; Americans had spent large money on war and lost many men
                \item Seminoles remained in Florida; although many had been killed or forced, complete migration never occurred
            \end{itemize}
        \end{itemize}
        \textbf{The Trail of Tears was the route taken by the Five Civilized Tribes to the Indian Territor, known for treacherous conditions. The Sminoles were the only tribe with successful attempts at resisting encroachment; the minority which supported preserving the land engaged in the Seminole Wsr with Jackson. Their migration was never truly complete.}}
        \cornell{What were the true impacts of the forced removal of eastern tribes?}{\begin{itemize}
            \item In total, tribes had given 100m acres of land in exchange for \$68m, 32m in less hospitable lands
            \begin{itemize}
                \item Divided by tribe into strict reservations surrounded by U.S. forts to prevent escape or entrance of whites, but would eventually even face white invasion
                \item Topography/climate far from anything of past 
            \end{itemize}
            \item Alternatives to removal of natives were few; westward expansion was too powerful a force to be stopped
        \end{itemize}
        \textbf{The natives were paid a small sum of money and given unfamiliar yet protected terrain sized at around one third of what they had previously owned. The alternatives to this removal, many argued, were impossible.}}
        \cornell{What were the main alternatives to native removal?}{
            The West saw many tribes creating shared world w/ whites, though it was not always completely egalitarian.
            \begin{itemize}
                \item NM pueblos, western fur traders, and TX/CA w/ settlers from Mexico, Canada, U.S. saw relative harmony 
                \item Lewis + Clark Expedition often saw natives as sexual partners w/ mutualistic relationships 
                \item Despite frequent exploitivity in motives for multiracial societies, generally represented how two cultures could interact 
            \end{itemize}
        }
        \cornell{What justifications were made for a changing white view on natives?}{
        \textbf{By the mid-19th century, the whites sought to create plantations much like the early British settlers which had complete separation from the natives, who they believed were not truly tied to or part of the land - western territories were uncolonized and ready for takeover.}}
    \end{document}