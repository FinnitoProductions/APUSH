\documentclass[a4paper]{article}
    \usepackage[T1]{fontenc}
    \usepackage{tcolorbox}
    \usepackage{amsmath}
    \tcbuselibrary{skins}
    
    \title{
    \vspace{-3em}
    \begin{tcolorbox}
    \Huge\sffamily \begin{center} AP US History  \\
    \LARGE Chapter 8 - Varieties of American Nationalism \\
    \Large Finn Frankis \end{center} 
    \end{tcolorbox}
    \vspace{-3em}
    }
    \date{}
    \author{}
    
    \usepackage{background}
    \SetBgScale{1}
    \SetBgAngle{0}
    \SetBgColor{red}
    \SetBgContents{\rule[0em]{4pt}{\textheight}}
    \SetBgHshift{-2.3cm}
    \SetBgVshift{0cm}
    \usepackage[margin=2cm]{geometry} 
    
    \makeatletter
    \def\cornell{\@ifnextchar[{\@with}{\@without}}
    \def\@with[#1]#2#3{
    \begin{tcolorbox}[enhanced,colback=gray,colframe=black,fonttitle=\large\bfseries\sffamily,sidebyside=true, nobeforeafter,before=\vfil,after=\vfil,colupper=blue,sidebyside align=top, lefthand width=.3\textwidth,
    opacityframe=0,opacityback=.3,opacitybacktitle=1, opacitytext=1,
    segmentation style={black!55,solid,opacity=0,line width=3pt},
    title=#1
    ]
    \begin{tcolorbox}[colback=red!05,colframe=red!25,sidebyside align=top,
    width=\textwidth,nobeforeafter]#2\end{tcolorbox}%
    \tcblower
    \sffamily
    \begin{tcolorbox}[colback=blue!05,colframe=blue!10,width=\textwidth,nobeforeafter]
    #3
    \end{tcolorbox}
    \end{tcolorbox}
    }
    \def\@without#1#2{
    \begin{tcolorbox}[enhanced,colback=white!15,colframe=white,fonttitle=\bfseries,sidebyside=true, nobeforeafter,before=\vfil,after=\vfil,colupper=blue,sidebyside align=top, lefthand width=.3\textwidth,
    opacityframe=0,opacityback=0,opacitybacktitle=0, opacitytext=1,
    segmentation style={black!55,solid,opacity=0,line width=3pt}
    ]
    
    \begin{tcolorbox}[colback=red!05,colframe=red!25,sidebyside align=top,
    width=\textwidth,nobeforeafter]#1\end{tcolorbox}%
    \tcblower
    \sffamily
    \begin{tcolorbox}[colback=blue!05,colframe=blue!10,width=\textwidth,nobeforeafter]
    #2
    \end{tcolorbox}
    \end{tcolorbox}
    }
    \makeatother

    \parindent=0pt
    
    \begin{document}
    \maketitle
    \SetBgContents{\rule[0em]{4pt}{\textheight}}
    \cornell[Key Concepts]{What are this chapter's key concepts?}{\begin{itemize}
        \item \textbf{4.1.I.A} - Political parties continued to debate tariff, federal power, European relations
        \item \textbf{4.1.I.B} - Supreme Court decisions established the judiciary influence over Constitutional interpretation; federal laws over state laws
        \item \textbf{4.3.I.A} - Louisiana Purchase -> U.S. government sought control over North America through exploration, war, native encroachment, diplomatic expansion
        \item \textbf{4.3.II.A} - Southern overcultivation -> plantations relocations to lands west of Appalachians
        \item \textbf{4.3.II.C} - Attempts at political compromise (like Missouri Compromise) temporarily stemmed tensions between opponents/defenders of slavery
    \end{itemize}}
    \cornell[Building a National Market]{How did the United States build a national market?}{\textbf{Economically, the United States saw great growth in government funds due to a national bank and a manufacturing sector particularly in textiles, which was protected from foreign intervention by Congress' bills. Transportation, too, was expanded: construction began on a National Road and turnpikes began to extend over the mountains; steam power became an essential part of transporting manufactures. However, large gaps still remained as evidenced by the War of 1812, and a constitutional solution remained unclear.}}
    \cornell{What were the economic effects and repercussions of the War of 1812?}{\begin{itemize}
        \item War of 1812 -> post-war economic boom which resulted in economic bust by 1819, evidencing U.S. lack in basic infrastructure
        \item War of 1812 -> growth in manufacturing by cutting off imports, but also shipping chaos due to inadequate transportation/financial systems 
        \begin{itemize}
            \item Aftermath of war saw multiple political issues related to economic development
        \end{itemize}
        \item Development of national bank prioritized
        \begin{itemize}
            \item After 1811 expiration of Hamilton's, state banks had begun to operate but often provided more bank notes than their true reserves -> multiple notes of different value circulating at once, business mess
            \item Congress chartered second Bank of United States in 1816; only difference to Hamilton's was slightly more capital
            \begin{itemize}
                \item Unable to forbid state banks from issuing currency, but size -> dominated state banks, potentially forcing them out of business
            \end{itemize}
        \end{itemize}
    \end{itemize}
    \textbf{After the War of 1812, a national bank was prioritized to replace Hamilton's, with the second Bank of United States given a larger amount of capital; its monopoly allowed it to threaten states out of issuing currency.}
    }
    \cornell{What characterized the post-war growth of the manufacturing sector?}{
         The government promoted the manufacturing sector after the conclusion of the war. The immensely small wartime industry meant that even the smallest factory with unskilled labor could start operations and be relatively successful.
        \begin{itemize}
            \item Textile industry saw particular growth with increase in cotton spindles; factories adapted to British textile machinery to transition from home-loomed goods to factory
            \begin{itemize}
                \item Boston merchant Lowell created power loom superior to British, organizing Boston Manufacturing Company and founding first American mill
                \item Revolutionized manufacturing, shaped early workforce
            \end{itemize}
            \item End of War of 1812 dimmed American prospects with British ships providing low-cost goods to Americans
            \begin{itemize}
                \item Parliament hoped to stifle American industry by initially incurring a slight loss
                \item Congress protectionists saw passage of tariff law to limit competition from abroad, allowing local industries to develop
                \begin{itemize}
                    \item Despite protest from many agriculturalists who would have to pay more for local goods, American industry prevailed
                \end{itemize}
            \end{itemize}
        \end{itemize}
    \textbf{The manufacturing industry grew significantly after the War of 1812, with textiles, specifically, moving from at-home production to factories after the invention of Lowell's power mill. The government also passed policies to prevent American industry from being taken over by foreign powers, particularly the British.}}
    \cornell{How did the United States build their transportation industry in the early nineteenth century?}{\begin{itemize}
        \item Manufacturers needed to access domestic markets for raw materials -> debate emerged whether federal government should finance roads directly
        \item As early as 1803, Ohio made agreement with federal government that some land revenue would go to financing roads
        \begin{itemize}
            \item Jefferson's treasury secretary Gallatin proposed that Ohio land revenues go toward National Road development; approved by both Congress and the president
            \item Construction began in 1811; by 1818, ran to Wheeling, VA
        \end{itemize}
        \item PA funded private company to extend pike to Pittsburgh, allow for stagecoaches, wagons, carriages, and cattle to transport
        \begin{itemize}
            \item High tolls funded efficient route over mountains
            \item Stimulated manufacture transport (especially textiles) from Atlantic seaboard to Ohio Valley
        \end{itemize}
        \item Great Lakes rivers saw steam-powered expansion (temporarily halted by War of 1812)
        \begin{itemize}
            \item By 1816, steamers down Mississippi to Ohio River, up Ohio to Pittsburgh
            \item Carried more cargo than all other previous river transportation combined 
            \item Stimulated western/southern agricultural economy with ready access to markets; enabled eastern manufacturers to send goods west
        \end{itemize}
    \end{itemize}
    \textbf{The early nineteenth century saw great transportation developments, particularly in OH, where the federal government promised to divert some land funds to road construction; in PA, where the government funded a private company to extend a large pike over the mountains to Pittsburgh; and steam transportation from the Great Lakes rivers, where a large number of goods were transported.}
    }
    \cornell{What major gaps remained in the transportation industry and how were they addressed?}{
        \begin{itemize}
        \item Major gaps remained in transportation, especially seen in War of 1812
        \begin{itemize}
            \item British blockade ended Atlantic shipping -> coastal roads overtaken by large volume of N-S traffic hoping to reach ferries to cross rivers
            \begin{itemize}
                \item Oxcarts sent for emergencies with 6-7 week travel time between Philadelphia/Charleston
                \item Goods normally sent by sea saw many northern states lacking in materials -> prices soared (like rice in NY, flour in Boston)
                \item Military challenges with inability to travel by sea, poor roads -> campaigns often unsuccessful
            \end{itemize}
        \end{itemize}
        \item Madison called Congress to establish national road system backed up by constitutional amendment
        \begin{itemize}
            \item Calhoun proposed bill allowing funds owed to government by U.S. Bank to finance internal improvements
            \item Congress passed, but Madison vetoed on final day in office; despite supporting purpose, believed Congress lacked authority 
            \begin{itemize}
                \item Forced state governments to continue handling task
            \end{itemize}
        \end{itemize}
    \end{itemize}
    \textbf{Large gaps remained in the transportation system, particularly in the War of 1812, where the British blockade prevented the smooth transfer of goods and military. Madison urged Congress to establish a road system, but ultimately vetoed the relevant bill due to its divergence from the Constitution.}}
    \cornell[Expanding Westward]{What were the key traits of U.S. westward expansion after the War of 1812?}{\textbf{Motivated by a growing population and reduced native power, many easterners migrated to the west. In the Southwest, a plantation system emerged led by wealthy farmers, creating a long-standing aristocracy in states like AL and MS. In the Far West, a fur trading economy grew on Mexico's territory, with white trappers contributing greatly to the health of the American economy. Many Americans remained hesitant to migrate further West, however, due to the grim tales of government-sponsored explorers.}}
    \cornell{What were the motivations for the major westward migrations and what were their effects?}{\begin{itemize}
        \item Motivating factors grew more powerful over time
        \begin{itemize}
            \item Pop. growth -> eastern agrarian saturation (some of which was absorbed by cities, but most remained rural) 
            \item Fundamental attractiveness of West due to diminished native opposition made possible by 1815 treaties, forts along Great Lakes, Mississippi to protect frontier
            \begin{itemize}
                \item Established "factor" system to supply tribes with goods at some cost -> Canadian traders driven out, created dependency system forcing reliance
            \end{itemize}
        \end{itemize}
        \item Major routes of migration included Ohio/Monongahela Rivers and later the Erie Canal (1825); used flatboats along Ohio River and finally traversing long on wagons
        \item Economically, brought new regions into capitalist system; politically, stimulated Civil War; culturally, encouraged diffusion and exchange
    \end{itemize}
    \textbf{Significant rural population growth in the east as well as reduced native opposition stimulated by the War of 1812 and the dependency-based factor system led many easterners to migrate rapidly to the west over the Ohio River and the Erie Canal. These migrations had drastic economic, political, and cultural effects.}}
    \cornell{How did the plantation system emerge in the Southwest?}{\begin{itemize}
        \item Center point: cotton due to erosion of critical lands in Old South, continual growth of market; central regions were central AL, MS known as "Black Belt" for limestone
        \item Settlement quickly saw spread of cotton, plantation, slavery; initial arrivals were smaller farmers with rough forest clearing
        \item Wealthy planters soon arrived with larger land clearing, distinct way of making livings
        \begin{itemize}
            \item Made journey over muddy roads with massive caravans containing livestock, goods, slaves, and the family in the back 
            \item Expanded into vast cotton fields 
            \item Initially constructed log cabins, but soon built mansions
            \item After significant time, built long aristocracy; even by Civil War, most planters had been there for fewer than two generations
        \end{itemize}
    \end{itemize}
    \textbf{The plantation system was pioneered by wealthier planters, who soon followed the smaller farmers and quickly overshadowed them. It was centered around cotton production to satisfy the growing market despite the erosion of land in the Southeast. The planters formed a long-standing aristocracy characterized by large mansions and slavery.}
    }
    \cornell{What new states were admitted after the War of 1812?}{\textbf{After the War of 1812, NW/SW growth saw admission of Indiana (1816), Mississippi (1817), Illinois (1818), and Alabama (1819).}}
    \cornell{What forms of trade developed in the Far West?}{\begin{itemize}
        \item Despite lack of widespread knowledge, some trade developed
        \item Mexico (controlling TX, CA, rest of Southwest), won 1821 independence -> opened trade with U.S. 
        \begin{itemize}
            \item U.S. traders poured into region, displacing dominant Native American/Mexican traders 
            \item Network of wagon trains moved steadily between MO and NM along Santa Fe Trail
        \end{itemize}
        \item Fur traders developed new form of commerce, starting with John Jacob Astor's American Fur Company based out of Oregon
        \begin{itemize}
            \item Sold at onset of War of 1812 to British-owned Northwestern Fur Company
            \item Revived operations post-war, centering in Great Lakes, expanding to Rockies
            \item Initial business spurred by purchasing pelts from natives; white trappers began to hunt independently, known as "mountain men"
            \begin{itemize}
                \item Andrew Henry/William Ashley founded Rocky Mountain Fur Company, recruited trappers to move to Rockies (fur was scarce further east) with annual supply train to send goods 
                \item Mountain men critical to economic development, with some earning salary from Rocky Mountain Fur Company, others relying on companies for credit due to debt (very nominal independence), and others trapping on their own and selling to eastern merchants directly
            \end{itemize}
        \end{itemize}
        \item Many white fur trappers lived in harmony with Native Americans/Mexicans, often marrying women of Indian/Spanish backgrounds; some were more peaceful
        \begin{itemize}
            \item Jedediah S. Smith travelled deep into Mexican territory -> battles with Mojaves ended disastrously; travelled further to Oregon, with 16 members of party of 20 killed; finally killed by Comanches for weapons
        \end{itemize}
    \end{itemize}
    \textbf{Fur trade was the most critical form of trade in Mexico's northern states, with the U.S. merchants dominating trade and eventually trapping their own fur. Known as "mountain men," many of them worked for the Ashley Rocky Mountain Fur Company, sending an annual supply; others worked independently. Some conflict arose between trappers and Mexicans/Natives on their shared land, leading to battles and deaths, particularly for Americans.}}
    \cornell{How did Easterners perceive the state of the West?}{\begin{itemize}
        \item Few trappers wrote of their lives; stories of Smith spread rapidly, creating negative images
        \item U.S. government dispatched explorers to chart territories
        \begin{itemize}
            \item Stephen H. Long led soldiers through modern NE/CO, returning through KS, writing powerful and influential report assessing region's potential 
            \item Most reports echoed Pike's, describing region as "Great American Desert" 
        \end{itemize}
    \end{itemize}
    \textbf{Images of Smith's dangerous explorations as well as many explorers' grim evaluations caused most Americans to perpetually view the West as uninhabitable and unsafe.}}
    \cornell[The "Era of Good Feelings"]{What was the "Era of Good Feelings" in America?}{\textbf{Monroe's ending of the partisan system by appointing a diverse cabinet and accepting Federalists warmly ushered in a temporarily positive era of American politics. Monroe's popularity grew as John Quincy Adams annexed Florida as a result of Jackson's military invasion. However, the significant economic bust in 1819 caused many to blame the national bank for the six-year depression which ensued.}}
    \cornell{What brought the first partisan system to an end?}{\begin{itemize}
        \item Many northerners angered by "Virginia Dynasty," with majority of presidents up through 1816 (all but John Adams) from Virginia; unable to mobilize, with Madison-sponsored Monroe easily winning election
        \begin{itemize}
            \item Monroe was a Revolutionary soldier, diplomat, cabinet officer to Madison
            \item Monroe faced no opposition with Federalist decline, War of 1812 put temporary pause to international threats 
        \end{itemize}
        \item Monroe's ultimate goal to put end to partisan divisions
        \begin{itemize}
            \item Reflected in cabinet selection
            \begin{itemize}
                \item Chose former Federalist and New Englander John Quincy Adams for sec. of state to signify end to "Virginia Dynasty" (secretary of state seen as heir to presidency)
                \item John C. Calhoun named sec. of war after Clay's rejection
                \item In all, included members from all parts of U.S. and both partoes 
            \end{itemize}
            \item Embarked on goodwill tour (first of any president after Washington) throughout country, greeting New Englanders warmly -> "Presidential Jubilee" and "era of good feelings"
            \begin{itemize}
                \item On the surface, successful in ending partisan divisions: faced no opposition in 1820 -> Federalist Party had essentially ceased to exist 
            \end{itemize}
        \end{itemize}
    \end{itemize}
    \textbf{Monroe, despite his election as part of the "Virginia Dynasty," went to great lengths to dismantle partisan tensions by selecting a diverse cabinet and embarking on a goodwill tour to unite Federalists and Republicans. On the surface, he was very successful, facing no opposition in 1820.}}
    \cornell{How did John Q. Adams approach Florida?}{
        Like his father, John Adams, John Quincy Adams had dedicated his life to diplomatic service; he was known as a nationalist devoted to continual expansion. His first challenge was Florida, where he set out to solidify the U.S. claim to West Florida and eventually potentially gain the entire peninsula.
        \begin{itemize}
            \item Negotiated with Spanish minister (Luis de Onís in hopes of resolving issue)
            \item Military events took place during negotations, with Andrew Jackson, acting on Calhoun's orders, aiming to stop raids by Seminole's on American territory
            \begin{itemize}
                \item Loosely interpreted, using as excuse to invade Florida, seizing forts and hanging inciting British suspects (Seminole War)
            \end{itemize}
            \item Adams urged government to take responsibility for Jackson's raid, claiming it to be part of defending borders
            \begin{itemize}
                \item Military strength of Americans forced Onís to form Adams-Onís treaty to prevent all-out war; Spain ceded Florida to U.S., gave up claim to north of 42nd parallel in PNW
                \item U.S. gave up claim to Texas in return
            \end{itemize}
        \end{itemize}
    \textbf{John Quincy Adams, an esteemed diplomat, initially attempted to form a reasonable treaty with Onís, the Spanish minister of Florida, but Jackson took Calhoun's orders to protect borders as an opportunity to invade Florida, dominating its inhabitants. Luís, realizing the American military strength, finally ceded Florida to the Americans in exchange for American claims to Texas being curbed.}}
    \cornell{What were the economic repercussions of the Panic of 1819?}{\begin{itemize}
        \item Soon after U.S. earned Florida, high foreign demand for American farm goods due to European agricultural disruptions as a result of Napoleonic Wars -> high prices for local American farmers -> land boom in Western U.S. -> prices soared
        \item Boom fueled by easy credit for settlers/speculators under government land acts (even Bank of U.S.); in 1819, credit tightened, loans called in, mortgages forecosed -> state banks began to fail
        \begin{itemize}
            \item Major financial panic emerged as a result of state banks' collapse
            \item Six years of depression emerged; process began where most blamed economic bust on national bank
        \end{itemize}
    \end{itemize}
    \textbf{The Panic of 1819 was created by western expansion as a result of higher prices for American farm goods due to high foreign demand. Western expansion was facilitated by easy credit for speculators issued by even the Bank of U.S. for some time; however, after credit was tightened and loans called in, a state-based financial crisis emerged where many blamed the effects on the national bank.}}
    \cornell[Sectionalism and Nationalism]{What were the major sectionalist and nationalist tensions in the United States?}{\textbf{The Missouri Compromise momentarily appeased tensions between northern and southern states by compromising on the admission of slave and free states. John Marshall revolutionized the judiciary system by establishing federal control over the economic positions of states as well as creating a robust position for natives. Finally, the Latin American independence movements spurred the Monroe Doctrine, which established the U.S. policy as the protector of the American continent.}}
    \cornell{What was the Missouri Compromise?}{\begin{itemize}
        \item When MO applied for statehood in 1819, slavery already established
        \begin{itemize}
            \item James Tallmadge Jr. of NY proposed amendment to limit further introduction, promote emancipation
        \end{itemize}
        \item Chance/design led new states to mostly come into Union as pairs -> 11 free states, 11 slave states
        \begin{itemize}
            \item Missouri would upset balance -> many opposed it
            \item Application of ME as free state led to compromise with states admitted in single bill, one as free and one as slave, known as "Missouri Compromise"
        \end{itemize}
        \item Jesse B. Thomas proposed amendment to prohibit slavery north of southern border of MO in Louisiana Purchase land; passed and hailed by nationalists in North and South for ability to keep Union together
    \end{itemize}
    \textbf{The Missouri Compromise entailed an informal agreement that, for each admitted slave state, a free state would be admitted, too, after Congress reached an impasse over the independent admission of the slave state Missouri. Following this saw the Thomas Amendment, banning slavery in all territories north of the southern border of MO (apart from MO itself).}}
    \cornell{How did John Marshall revolutionize the American judiciary system, politically?}{\begin{itemize}
        \item Increased power of federal government at expense of states, advanced interests of propertied classes
        \item \textit{Fletcher v. Peck} saw notorious land frauds in Georgia, Marshall held that land grants were valid, unable to be repealed regardless of corruption
        \begin{itemize}
            \item Marshall promoted commerce by strictly abiding by contracts
        \end{itemize}
        \item \textit{Dartmouth College v. Woodward} saw expanded meaning of contract clause w/ Republicans having gained control of NH government, attempting to revise Dartmouth's charter (private -> state-owned )
        \begin{itemize}
            \item Daniel Webster argued that charter was a contract with equal weight to that upheld in \textit{Fletcher vs Peck}
            \item Argued with irrelevant passage bringing jury to tears, ultimately receiving rule for Dartmouth
            \item Restricted state government's ability to control corporations
            \item Justices implicitly claimed right to override decisions of state courts despite advocates of state rights
        \end{itemize}
        \item \textit{Cohens v. Virginia} saw Marshall directly confirming federal ability to review court decisions despite southern challengers
        \begin{itemize}
            \item Argued that states had given up sovereignty in ratification of Constitution
        \end{itemize}
    \end{itemize}
    \textbf{Marshall strictly enforced contractual law and ultimately ensured that the federal government would have control over the states, able to reverse their courts' major decisions.}}
        \cornell{How did John Marshall revolutionize the American judiciary system, economically?}{\begin{itemize}
        \item \textit{McCulloch v. Maryland} allowed Marshall to confirm implied powers by upholding constitutionality of Bank of U.S.
        \begin{itemize}
            \item Unpopularity of bank -> many states tried to prohibit operation to drive out branches
            \item Daniel Webster, bank attorney, argued that it fit within Constitution, had "power to destroy" through taxes if necessary
        \end{itemize}
        \item \textit{Gibbons v. Ogden} allowed Court to strengthen power to regulate interstate commerce
        \begin{itemize}
            \item Aaron Ogden carried passengers across NY/NJ through indirect license; Thomas Gibbons had Congress-granted license and competed w/ Ogden
            \item Ogden challenged Gibbons in state courts, won case; Gibbons appealed to Supreme Court
            \begin{itemize}
                \item Central question: could Congress supersede NY's power to give a monopoly to Ogden?
                \item Marshall believed that Congress had the power to regulate interstate commerce, making the monopoly void
            \end{itemize}
        \end{itemize}
    \end{itemize}
    \textbf{Marshall's position in the court established the dominance of federal government over the states in regulating the economy and promoting economic growth federally. This ensured that local interference could not affect private companies and that the state had control over the economic status of states.}}
    \cornell{How did the court handle relations with the natives?}{\begin{itemize}
        \item Marshall carved distinct Native American place in Constitution
        \item \textit{Johnson v. McIntosh} emerged when leaders of Illinois and Pinakeshaw sold land to white settlers (including Johnson) but later treaty ceded same lands to U.S.
        \begin{itemize}
            \item Government granted settling rights to other white settlers (including McIntosh) -> Court asked to decide more important claim
            \item Marshall favored U.S., with explanation that tribes had rights over all tribal lands and only superior authority was federal government (not American citizens alone)
        \end{itemize}
        \item \textit{Worcester v. Georgia} saw Court invalidating Georgia laws to regulate U.S. access to Cherokee territory due to federal government being only Americans with authority over lands
        \begin{itemize}
            \item Defined native lands to be sovereign entities defined in same manner as states 
            \item Expanded tribal rights at expense of states
        \end{itemize}
        \item In all, cases allowed Marshall to define place for natives with property, federal government as "guardians" with ultimate authority (rarely able to prevent westward expansion, but gave true legal protection)
    \end{itemize}
    \textbf{Marshall defined an established place for Americans in the Constitution, establishing native lands as equal in power to states; thus individual Americans could not encroach or buy native lands: only the federal government had control.}}
    \cornell{How did Monroe approach the revolutions in Latin America?}{\begin{itemize}
        \item U.S. had already established trading relations with Latin American states under Spanish Empire; independence movements -> U.S. hoped to improve standing
        \begin{itemize}
            \item Neutral in wars, partially recognizing rebels as nations and selling ships/supplies
            \item Recognized new nations of La Plata (Argentina), Chile, Peru, Colombia, Mexico
        \end{itemize}
        \item Monroe proclaimed "Monroe Doctrine" (primarily work of John Q. Adams) protecting all American continental states from European colonization
        \begin{itemize}
            \item Emerged out of relations with Spain's allies: Americans feared that Spain would attempt to take back Latin America
            \item U.S. feared that GB sought to colonize Cuba
        \end{itemize}
        \item Represented growing nationalism in U.S., established as dominant hemispheric power
    \end{itemize}
    \textbf{Through the Latin American independence movements, the U.S. declared themselves neutral but assisted the rebels greatly, immediately recognizing them as nations after their independence. Furthermore, under the Monroe Doctrine, the U.S. promised to protect Latin American states from European colonization, representing greater tensions with Europe.}}
    \cornell[The Revival of Opposition]{How did partisan tensions return in the United States?}{\textbf{Partisan tensions returned after the emergence of the "corrupt bargain" claimed by Jacksonians when Clay and Adams made an alleged deal to undermine Jackson. Adams' presidency was somewhat unsuccessful due to the great opposition, which was was channeled into a system with Democratic Republicans under Andrew Jackson and National Republicans under John Quincy Adams. Ultimately, the Democratic Republicans won the election of 1828.}}
    \cornell{What were the pretexts to the second emergence of partisan divisions?}{\textbf{Post-1816, the Federalists had ceased as a national force. The Republican Party emerged the only organized force, but partison divisions returned in the 1820s, with the Republicans beginning to resemble the early Federalists in pushing for centralization and the opposition pushed for the government to have a reduced economic role. However, both parties agreed that the nation should expand, disagreeing only on the exact nature of expansion.}}
    \cornell{What was the "corrupt bargain"?}{\begin{itemize}
        \item Until 1820, vice/presidential nominees decided by caucuses from each party, discussions between congressmen
        \item In 1824, Republican caucus nominated William H. Crawford; other candidates nominated by state legislatures, mass meetings -> less organized, more opportunity
        \begin{itemize}
            \item John Q. Adams, secretary of state, understood forbidding manners/little popular appeal -> nominated not by caucuses 
            \item Henry Clay had built following for "American System," proposing creation of home market by raising tariffs, strengthening bank
            \item Andrew Jackson had no political record but hailed as a military hero of TN
        \end{itemize}
        \item Jackson received popular vote, but not majority -> House forced to pick between candidates with top 3 electoral votes
        \begin{itemize}
            \item Clay was in position to influence result -> endorsed Adams due to nationalist sentiment likely giving easy support for American Plan
            \item Adams won the election, naming Clay secretary of state -> Jacksonians very angry, calling Clay's endorsement and Adams' selection of Clay a "corrupt bargain"
        \end{itemize}
    \end{itemize}
    \textbf{The "corrupt bargain" emerged after the caucus system was expanded, allowing any mass meetings or state legislatures to nominate a presidential candidate. Although Andrew Jackson of TN received the popular vote, he did not receive a majority and thus the House was forced to select between the three candidates with the most electoral votes. Because Clay was out of the running, he endorsed John Quincy Adams, who then selected him as secretary of state. This, Jacksonians believed, was a "corrupt bargain."}}
    \cornell{What did John Q. Adams implement during his presidency?}{\begin{itemize}
        \item Policies were mostly blocked due to Jacksonian anger at his "corrupt bargain," unable to implement a nationalist program like Clay's American System
        \item Hoped to send delegates to conference held by Simón Bolívar in Panama with Haiti as a participant -> Southern delegates felt white Americans should not mingle w/ Haitian delegates
        \begin{itemize}
            \item Stalled until conference was over
        \end{itemize}
        \item Georgia wished to remove remaining natives from states -> Adams rejected due to existing treaty; McIntosh proceeded with plans for removal and Adams unable to stop
        \item Supported import tariff in 1828 to support MA/RI woolen manufacturers with low prices for British goods; acceptied duties on other items to support other states -> New England ended up having to pay more for raw materials 
    \end{itemize}
    \textbf{John Quincy Adams had a relatively unsuccessful presidency - the Jacksonian anger carried over from the election, blocking many of his key policies like a revamped American System. Additionally, southerners in Congress blocked delegates from attending an international conference with black Haitians by stalling. Furthermore, tensions with Georgia over the natives saw a decisive loss for him, and his import tariff was disliked by all.}}
    \cornell{How did Jackson's victory solidify the two-party system once again?}{\begin{itemize}
        \item Supporters of JQA called themselves National Republicans, supported economic nationalism (many of whom were once Federalists); supporters of Jackson called themselves Democratic Republicans, seeking to widen opportunities (hoped to end economic aristocracy)
        \item Campaign very harsh: Jacksonians called Adams a useless president guilty of using funds for gambling; Adams accused Jackson of murder, listing killed militiamen (they were deserters legally executed)
        \begin{itemize}
            \item Called wife bigamist due to legal miscommunication; wife died soon after
        \end{itemize}
        \item  Jackson's victory very sectional but decisive: Adams swept New England, strong in mid-Atlantic, but Jacksonians won in South and West
        \begin{itemize}
            \item Believed victory had driven privilege from Washington, restoring liberty to the "common man"
        \end{itemize}
    \end{itemize}
    \textbf{Jackson's victory saw the division of parties: the National Republicans, under John Quincy Adams, who supported economic nationalism, and the Democratic Republicans, who sought to wide opportunities for the common person. Jackson won the long election full of charged invectives, but the two-party system had been solidified.}}
    \end{document}