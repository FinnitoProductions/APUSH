\documentclass[a4paper]{article}
    \usepackage[T1]{fontenc}
    \usepackage{tcolorbox}
    \usepackage{amsmath}
    \tcbuselibrary{skins}
    
    \title{
    \vspace{-3em}
    \begin{tcolorbox}
    \Huge\sffamily \begin{center} AP US History  \\
    \LARGE Chapter 14 - The Civil War \\
    \Large Finn Frankis \end{center} 
    \end{tcolorbox}
    \vspace{-3em}
    }
    \date{}
    \author{}
    
    \usepackage{background}
    \SetBgScale{1}
    \SetBgAngle{0}
    \SetBgColor{red}
    \SetBgContents{\rule[0em]{4pt}{\textheight}}
    \SetBgHshift{-2.3cm}
    \SetBgVshift{0cm}
    \usepackage[margin=2cm]{geometry} 
    
    \makeatletter
    \def\cornell{\@ifnextchar[{\@with}{\@without}}
    \def\@with[#1]#2#3{
    \begin{tcolorbox}[enhanced,colback=gray,colframe=black,fonttitle=\large\bfseries\sffamily,sidebyside=true, nobeforeafter,before=\vfil,after=\vfil,colupper=blue,sidebyside align=top, lefthand width=.3\textwidth,
    opacityframe=0,opacityback=.3,opacitybacktitle=1, opacitytext=1,
    segmentation style={black!55,solid,opacity=0,line width=3pt},
    title=#1
    ]
    \begin{tcolorbox}[colback=red!05,colframe=red!25,sidebyside align=top,
    width=\textwidth,nobeforeafter]#2\end{tcolorbox}%
    \tcblower
    \sffamily
    \begin{tcolorbox}[colback=blue!05,colframe=blue!10,width=\textwidth,nobeforeafter]
    #3
    \end{tcolorbox}
    \end{tcolorbox}
    }
    \def\@without#1#2{
    \begin{tcolorbox}[enhanced,colback=white!15,colframe=white,fonttitle=\bfseries,sidebyside=true, nobeforeafter,before=\vfil,after=\vfil,colupper=blue,sidebyside align=top, lefthand width=.3\textwidth,
    opacityframe=0,opacityback=0,opacitybacktitle=0, opacitytext=1,
    segmentation style={black!55,solid,opacity=0,line width=3pt}
    ]
    
    \begin{tcolorbox}[colback=red!05,colframe=red!25,sidebyside align=top,
    width=\textwidth,nobeforeafter]#1\end{tcolorbox}%
    \tcblower
    \sffamily
    \begin{tcolorbox}[colback=blue!05,colframe=blue!10,width=\textwidth,nobeforeafter]
    #2
    \end{tcolorbox}
    \end{tcolorbox}
    }
    \makeatother

    \parindent=0pt
    
    \begin{document}
    \maketitle
    \SetBgContents{\rule[0em]{4pt}{\textheight}}
    \cornell[Key Concepts]{What are this chapter's key concepts?}{\begin{itemize}
        \item \textbf{5.1.I.D} - Civil War $\to$ increased Westward migration due to promoting legislation
        \item \textbf{5.2.II.D} - Lincoln won w/o any Southern votes $\to$ majority of slave states seceded $\to$ Civil War
        \item \textbf{5.3.I.A} - Despite opposition on home fronts, both sides of Civil War underwent economic/social preparation to fight
        \item \textbf{5.3.I.B} - Lincoln began war w/ goal to preserve Union, but Emancipation Proclamation $\to$ Europe would not support Confederacy, Afr. Americans fought for Union
        \item \textbf{5.3.I.C} - Lincoln's powerful speeches (like Gettysburg Address) portrayed slavery as in violation of democracy
        \item \textbf{5.3.I.D} - Union won war due to $\uparrow$ leadership, strategy (including destruction of Southern infrastructure), resources, 
    \end{itemize}}
    \cornell[The Secession Crisis]{What caused several Southern states to secede and how did the North react?}{\textbf{Secession began in South Carolina and was sparked by Lincoln's election; six states followed, forming the Confederacy. Crittenden led the effort for compromise which was supported by Southern senators, but Northern Republicans would not sacrifice their ideals. The Civil War began after the South took over Fort Sumter, pitting the Union, advantaged in materials, population, and transportation, against the South, with a firmer committment and the defensive ability of fighting on home territory.}}
    \cornell{Where did secession begin and what were initial reactions?}{\begin{itemize}
        \item SC (known for radical ideas) unanimously seceded in December 1860; MS, FL, AL, GA, LA, TX seceded by Lincoln's inauguration
        \begin{itemize}
            \item February 1861: seven seceded states met in AL, formed Confederate States of America
            \item Federal government indecisive: Buchanan told Congress that fed. govt. could not intervene in secession 
        \end{itemize}
        \item Seceded states took over federal property, justifying with anger/betrayal at Lincoln's election; \textbf{Fort Sumter} (SC)/Fort Pickens (FL), federal forts, not easily given up
        \begin{itemize}
            \item Initial attacks on forts unsuccessful 
            \item Buchanan ordered fortifications in Jan. 1861, even exchanging shots 
            \item Neither side admitted beginning of war
        \end{itemize}
    \end{itemize}
    \textbf{Secession began in South Carolina, quickly spreading to six other Southern states, who soon formed the Confederacy. The federal government was initially powerless to respond: Buchanan felt Congress had no right. After some seceded states began to take over federal property, however, the Union retaliated by fortifying Fort Sumter and Fort Pickens and fighting back.}}
    \cornell{What attempts were made at compromise?}{\begin{itemize}
        \item Sen. John Crittenden of KY formed \textbf{Crittenden Compromise}, constitutional amendments allowing for permanent slavery, Fugitive Slave Act, reestablishment of MO Compromise
        \begin{itemize}
            \item Southerners in Senate accepted; Republicans would not
        \end{itemize}
        \item Lincoln snuck into DC for inauguration due to fear of attack, giving inaugural address reprimanding secession and promising preservation of federal property
    \end{itemize}
    \textbf{Although the Crittenden Compromise aimed to give the South permanent slavery, formally establish the Fugitive Slave Act, and reinstate the Missouri Compromise and the Southern senators accepted it, the Republicans would not due to its fundamental disagreement with their party's ideals. Lincoln stood by his party at his inaugural address, promising that seceding states would be punished.}}
    \cornell{How did the Civil War begin at Fort Sumter?}{\begin{itemize}
        \item Sumter conditions quickly $\downarrow$ but Lincoln felt critical symbol of power of Union $\to$ sent relief supplies (w/o troops)
        \item Confederacy conflicted betw. cowardly decision of submitting to federal govt. and aggressive one of attacking Fort 
        \begin{itemize}
            \item Chose to attack under General Beauregard
            \item Bombarded for 2 days $\to$ Major Robert Anderson forced to surrender, beginning Civil War on Apr. 14, 1861
        \end{itemize}
        \item Lincoln felt secession infringed upon American liberty $\to$ mobilized North in conjunction w/ VA, AR, NC, TN seceding; border slave states pressured by DC to side w/ Union
        \item Central question: could war have been avoided? Sect. tensions had grown so large $\to$ something had to change
        \begin{itemize}
            \item North/South felt civilizations were 100\% incompatible $\to$ both sides supported war
        \end{itemize}
    \end{itemize}
    \textbf{The South chose to continue attacking Fort Sumter for fear of seeming cowardly; after they took it over and drove out the Union forces, Lincoln prepared the North for war as four more states seceded. The war began due to both the North and South feeling they were mutually incompatible.}}
    \cornell{What were the differences between the two sides of the war?}{\begin{itemize}
        \item North had material advantages: double pop., greater army/workforce, able to manufacture war materials (while South relied on Europe)
        \begin{itemize}
            \item North had more reliable transportation system by rail w/ greater integration
        \end{itemize}
        \item South had advantage of fighting war on home turf against North on hostile territory
        \begin{itemize}
            \item Southern whites entirely committed to war while Northern far more divided 
            \item South believed English/French textile industries needed cotton $\to$ instant support
        \end{itemize}
    \end{itemize}
    \textbf{The North had material advantages on paper, with a greater population, larger army, more power to create war materials for themselves, and an integrated transportation network. However, the South generally fought defensively on their home territory, were far more committed to the war, and hoped for the support of Europe.}}
    \cornell[The Mobilization of the North]{How did the North mobilize their troops for battle?}{\textbf{The North implemented critical economic changes under Republican dominance and, as a result, saw the growth of several industries and unions. However, these changes alone were often insufficient to finance the war and raise armies: Congress relied on loans from and conscriptions of the people themselves. After the war began, Lincoln shifted his view on slavery, siding with the growing radicalist movement pushing for immediate emanciaption, eventually signing the Emancipation Proclamation and allowing Southern blacks to join the Union forces without ramification. Despite being free to join Union forces, these blacks were often assigned menial tasks but still took their contributions with pride. Finally, the war promoted the feminist movement, with countless women becoming nurses and feeling empowered by a newfound freedom associated with the emancipation of slavery.}}
    \cornell{What economic changes were implemented by the North given Republican power in Congress?}{\begin{itemize}
            \item \textbf{Homestead Act of 1862} allowed prospective citizens to claim 160 acres of land, purchase cheaply if inhabited for 5 yrs.
            \item \textbf{Morill Land Grant Act} gave public acreage to state govt. for public education $\to$ several state colleges/universities 
            \item Tariff bills $\to$ raised duties to unprecedented levels $\to$ domestic industries protected from foreign competition
            \item Transcontinental railroad w/ two companies
            \begin{itemize}
                \item \textbf{Union Pacific Railroad Company} to build westward from Omaha
                \item \textbf{Central Pacific Railroad Company} to build eastward from CA
                \item Two would meet in the middle, completing link 
            \end{itemize}
            \item \textbf{National Bank Acts of 1863-1864} $\to$ national bank system 
            \begin{itemize}
                \item Existing banks to join if enough capital, willing to invest in govt. securities; allowed to issue U.S. Treasury notes 
            \end{itemize}
        \end{itemize}
        \textbf{The Northern Republicans, with Southern competitors out of the way, made land more accessible both to the public and to state governments for personal use and education, raised duties to support domestic industries, formed two railroad companies to create a transcontinental railroad, and began a national banking system.}}
    \cornell{How did the North finance the war?}{\begin{itemize}
        \item Congress $\uparrow$ taxes on most goods/services, creating income tax in 1861; heavily opposed
        \item Paper currency equally controversial: no gold/silver to back 
        \begin{itemize}
            \item Fluctuated based on army's success $\to$ govt. used limited amounts
        \end{itemize}
        \item Greatest source: loans from ppl. w/ Treasury convincing Americans to buy \$400m in \textbf{bonds}, paired w/ banks/large corporations assisting
        \end{itemize}
    \textbf{Congress implemented some widely opposed changes of raised taxes and paper currency (used in limited amounts), but mainly relied on loans from the people through bonds as well as from banks.}}
    \cornell{How did the North begin to raise armies?}{\begin{itemize}
        \item 2 million men fought in the Union in total, but the U.S. federal army began at only 16k (mostly in the West to prevent native rebellions)
        \item Lincoln raised regular army but knew that state militia volunteers were critical
        \begin{itemize}
            \item Authorized 500k volunteers in Congress: initially adequate but sunk with enthusiasm
        \end{itemize} 
        \item Forced to issue draft for any young adult male; could escape by paying \$300 or hiring someone else
        \begin{itemize}
            \item Conscription odd to ppl. used to remote govt. $\to$ opposition from laborers, immigrants, Democrats often leading to violence
            \item Only $\approx$ 46k were drafted but $\to$ increased voluntary enlistment
            \item Irish workers led one of deadliest American riots in NYC in 1863, lynching blacks due to fear of war $\to$ more competition for jobs
        \end{itemize}
    \end{itemize}
    \textbf{The primary sources for Northern armies were volunteers from state militias and drafted men. The concept of drafting enraged Americans happy with a distant and remote government, sparking a riot led by Irish workers in NYC who opposed the war due to a fear of more African Americans threatening their jobs.}}
    \cornell{What was the political state of the North during the war?}{\begin{itemize}
        \item Lincoln initially seen as inexperienced politician easy to control, but quickly asserted dominance
        \begin{itemize}
            \item Established cabinet made up of all Republican factions, many of whom opposed his presidency
            \item Violated aspects of Constitution (saw as better than losing it all): declared war, grew army, established blockade w/o Congress 
        \end{itemize}
        \item Lincoln experienced great opposition from \textbf{Peace Democrats} fearing reduced influence of agricultural Northwest, states' rights
        \begin{itemize}
            \item Retaliated by arresting dissenters with no right to be released even if arrested unlawfully; initially for border states but soon extended to all 
            \item Congressman \textbf{Vallandigham} of OH arrested, exiled to Confederacy after claiming war intended to free blacks but enslave whites 
            \item Ignored Taney's written demand for MD secessionist leader to be freed 
        \end{itemize}
        \item Lincoln built support w/ pro-war advertisements supported by photography group (led by \textbf{Mathew Brady}) to show terrible images of war
        \begin{itemize}
            \item Images met some w/ revulsion but many w/ patriotism and a desire to preserve the Union
        \end{itemize}
    \end{itemize}
    \textbf{Lincoln asserted a dominant position without difficulty in the Senate, often freely violating the Constitution for the sake of the war. He was opposed by Peace Democrats arguing for states' rights, but persecuted those who spoke against him. To build popular support for the war, he created public advertisements as well as drafted photographers to take jarring images of the destruction of the battles.}}
    \cornell{What was the result of the election of 1864?}{\begin{itemize}
        \item Republicans lost heavily in 1862 midterms $\to$ party leaders created \textbf{Union Party}, linking Republicans w/ War Democrats: nominated Lincoln and TN's \textbf{Andrew Johnson} (War Democrat)
        \item Democrats selected \textbf{George B. McClellan}, Union general relieved by Lincoln; although McClellan disagreed w/ Democratic goal for a truce, party continued to argue
        \item Major Northern victories (capture of Atlanta) $\to$ Republicans empowered w/ large majority of electoral votes but only 10\% greater popular vote 
    \end{itemize}
    \textbf{With the Republicans hurt in the 1862 midterms, they formed the Union Party to join forces with Democrats supporting the war. Lincoln was pitted against Peace Democrat McClellan, a former general. Lincoln won the election in large part due to luck: the election coincided with significant Northern victories in the battle.}}
    \cornell{How did emancipation play a central role in the Civil War?}{\begin{itemize}
        \item Republicans split across lines of slavery: radicals (like Thaddeus Stevens, Sumner, Wade) sought immediate abolishment while conservatives sought more gradual process 
        \item Support grew for emancipation near beginning of war
        \begin{itemize}
            \item Confiscation Act declared slaves used to fight for Confederacy as freed
            \item Radicals pushed second Confiscation Act, declaring slaves of \underline{any person} fighting for the Confederacy as freed and allowing Afr. Americans (including freed slaves) to fight for Union
            \item Radicals gradually grew in Republican party $\to$ Lincoln became their leader 
        \end{itemize}
        \item Sept. 1862: after victory at Antietam, Lincoln announced intention for emancipation; signed \textbf{Emancipation Proclamation} on Jan 1st, 1863
        \begin{itemize}
            \item Effectively freed all slaves in Confederacy (places not already controlled by Union, like border states, WV, southern LA, TN)
            \item Immediate effect insignificant (territories still controlled by Confederacy), but established war as one also against slavery
        \end{itemize}
        \item True liberating factor for slaves was war itself
        \begin{itemize}
            \item Confederacy often took slaves from plantations and employed to build defenses $\to$ close to border $\to$ easily escaped 
            \item Masters immediately lost any right to them $\to$ flocked to Union Army, some joining and others looking to reach free states 
        \end{itemize}
        \item As war ended, MD and MO had already abolished, as had TN, AR, and LA; Thirteenth Amendment did final duty
    \end{itemize}
    \textbf{Republicans were divided into radicals, who sought immediate emancipation, and conservatives, who sought a gradual freeing of slaves. Radicals grew in power as the war progressed, pushing for the Confiscation Act to immediately free almost all slaves, followed up by Lincoln's Emancipation Proclamation freeing all Confederate slaves. Slaves were most directly liberated not by the proclamation but by being allowed to enlist in the Union for the war itself.}}
    \cornell{How did African Americans fight for the Union?}{\begin{itemize}
        \item Emancipated Afr. Americans joined forces w/ free blacks, often facing obstacles
        \begin{itemize}
            \item Initially excluded from military w/ only a few black regiments out of necessity
            \item Emancipation Proclamation $\to$ black numbers swelled w/ active recruitment 
        \end{itemize}
        \item Some divided into fighting units (like \textbf{Fifty-fourth MA Infantry} w/ white commander Robert Gould Shaw)
        \item Most given non-fighting tasks like digging trenches $\to$ black mortality rate higher than white due to long hours, poor conditions, low pay
        \begin{itemize}
            \item Black soldiers still proud of significant contribution to war in long-term
        \end{itemize}
        \item Captured blacks in Confederacy either returned to original masters or executed
    \end{itemize}
    \textbf{African Americans played a crucial part in supporting the Northern cause despite facing obstacles of initially being unable to even enlist. After the Emancipation Proclamation, although enlistment was widespread, blacks were generally assigned to menial and backbreaking tasks.}}
    \cornell{How did the Civil War promote economic development?}{\begin{itemize}
        \item Some slowing of industrial growth w/ markets cut off from Southern goods
        \item Econ. development sped up due in part to Republican dominance but also conditions of war itself
        \begin{itemize}
            \item Coal (w/ $\uparrow$ production due to demand) and railroad industries (w/ standard gauge) forced to improve 
            \item Farms lost labor to armies $\to$ forced to mechanize agriculture
        \end{itemize}
        \item Industrial workers suffered w/ $\uparrow$ prices (70\% rise) but wages unable to meet (40\% rise)
        \begin{itemize}
            \item Liberal immigration $\to$ new workers keeping wage low 
            \item Mechanization eliminated skilled workers 
            \item Unions became far more widespread despite employer suppression
        \end{itemize}
    \end{itemize}
    \textbf{Although the cutoff of Southern raw materials hurt some industries, several grew significantly out of wartime need, like coal, railroads, and agricultural mechanization. However, industrial workers suffered greatly due to a decrease in purchasing power due to freer immigration laws and mechanization.}}
    \cornell{How did the war affect traditional gender roles?}{\begin{itemize}
        \item Women often took on foreign roles out of necessity, taking over male positions but most notably becoming nurses
        \begin{itemize}
            \item Dorothea Dix led \textbf{U.S. Sanitary Commission}: org. of civilian volunteers, pulling female nurses into field hospitals
            \item By 1900, nursing almost entirely female, caring for patients but performing other important tasks for hospital (like cooking/cleaning)
            \item Male doctors often felt women too weak for role but Sanitary Commission claimed nursing to represent manifestation of key domestic aspects of home life 
            \begin{itemize}
                \item Some stood up to male doctors, pushing incompetent ones aside
                \item Critical role $\to$ male complaints ignored
            \end{itemize}
            \item Nurses generally felt freed by war
            \item Nursing changed medical profession w/ wounded soldiers assisted greatly; Commission also appointed women behind the scenes and spread knowledge about hygiene
        \end{itemize}
        \item Feminists (like Cady Stanton/B. Anthony) founded \textbf{Woman's Loyal League} in 1863, fighting both for abolition of slavery and suffrage
        \begin{itemize}
            \item Clara Barton(assisted w/ nursing, Red Cross) felt war pushed women's rights far further than peace ever would have 
        \end{itemize}
    \end{itemize}
    \textbf{Several women felt liberated both during and after the war, becoming nurses on the battlefront under Dorothea Dix' U.S. Sanitary Commission despite the opposition of several men; they were critical to the health of soldiers. Feminists also capitalized on the movement to further their cause, pushing for both the abolition of slavery and for suffrage.}}
    \cornell[The Mobilization of the South]{How did the South mobilize their society for battle?}{\textbf{Although the South instated a political system similar to the North simply with slavery and states' rights more deeply ingrained, their primary source of income came from unstable paper currency leading to inflation, and their primary source of troops was conscription. President Davis struggled to implement a centralized government but was successful in industrializing in the long-term. The war was devastating economically, cutting the South off from markets and leading to many shortages; socially, however, both women and slaves were given new hope.}}
    \cornell{What was the political structure of the Confederate Government?}{\begin{itemize}
        \item Confederate constitution very similar to U.S. but directly acknowledged state sovereignty (w/o secession), made abolishment impossible
        \item Nominated \textbf{Jefferson Davis} of MS as president, Alexander Stephens of GA as VP for 6 yrs.
        \begin{itemize}
            \item Stephens had been against secession and Davis had been moderate: govt. remained moderate 
            \item Dominated by newer Western leaders rather than old aristocracy of East
        \end{itemize}
        \item Davis was skilled admin. + dominant leader as own sec. of war, but rarely made national decisions
        \item Govt. not formally partisan, but great tension remained
        \begin{itemize}
            \item Some whites ("back/upcountry" regions) and many African Americans who fought for Union or refused to assist govt. 
            \item As battle turned against Confederacy, more and more opposed war
        \end{itemize}
    \end{itemize}
    \textbf{The Confederate government bore numerous similarities to the U.S. government, with a similar constitution apart from increased states' rights and permanent slavery as well as a president, Jefferson Davis (a moderate secessionist who rarely made national decisions), and a vice president, Alexander Stephens (an anti-secessionist). The new government was opposed by several African Americans and some whites, particularly those from the "backcountry" and "upcountry" regions.}}
    \cornell{How did the South finance the war?}{\begin{itemize}
        \item Finances required implementing national revenue and banking system in society with few taxes and little capital (most wealth in slaves/land)
        \begin{itemize}
            \item Only gold taken from federal mints in South, worth only $\approx$ \$1m
        \end{itemize} 
        \item Initially tried to get funds from states w/o direct tax but few states would tax citizens $\to$ paid w/ questionable notes/bonds; income tax relatively unsuccessful, too
        \item Borrowing equally unsuccessful: bonds so vast that public lost trust in govt, unable to borrow European money for cotton
        \item Paid primarily through paper currency, issuing more than double of Union's 
        \begin{itemize}
            \item Currency never uniform: govt., states, cities, private banks produced independent notes $\to$ inflation w/ prices rising 9000\%
        \end{itemize}
    \end{itemize}
    \textbf{The Confederate government was faced with the challenging task of earning revenue and establishing banks in a land with little tax and little liquid capital. Attempts at taxing both through states and later through an income tax failed, and borrowing money from the public resulted in a loss of trust. Thus the majority of wealth came through paper currency, but the inconsistency of notes led to great inflation.}}
    \cornell{How did the South rally troops for the war?}{\begin{itemize}
        \item Volunteers initially but numbers dwindled like in North $\to$ 1862 \textbf{Conscription Act} requiring 3 yrs. for all white males
        \begin{itemize}
            \item Successful for some time, w/ $\approx 500,000$ men excluding slave men/women 
            \item Drafted allowed to find substitute but price high, one white man on each plantation w/ > 20 slaves exempt $\to$ poorer whites opposed $\to$ repealed in 1863
        \end{itemize}
        \item Conscription $\to$ less effective as North captured more land $\to$ reduced population 
        \begin{itemize}
            \item Responded by drafting ages 17-50 (wider range than 18 to 35) but nation already too weak
            \item Desertions increased $\to$ turned to slave army, but war already near-over by implementation 
        \end{itemize}
    \end{itemize}
    \textbf{The South relied primarily on conscription; although the high price to resist the draft angered many poorer whites, the system was relatively successful for a time, forming a large army. As the North captured more and more land, the accessible population pool shrunk, reducing the effectiveness and forcing potential alternatives.}}
    \cornell{How did the Confederate government address the conflict between states' rights and central power?}{\begin{itemize}
        \item $\uparrow$ White southerners dedicated to state rights $\to$ resisted any of Davis' efforts for national control (even if for sake of war)
        \begin{itemize}
            \item Limited conscription, states tried to separate troops from federal army, hoarded supplies 
        \end{itemize}
        \item Confederate government did expand greatly, becoming larger than Union's govt. 
        \begin{itemize}
            \item Implemented "\textbf{food draft}," allowing soldiers to eat crops from farms along the way
            \item Successfully mobilized slaves for labor 
            \item Created industrial regulations, took over railroads/shipping
        \end{itemize}
    \end{itemize}
    \textbf{Although a majority of white Southerners opposed centralization and obstructed any efforts, the Confederate government made great strides, implementing several laws promoting the army and long-term industrialization.}}
    \cornell{What were the economic effects of the war?}{\begin{itemize}
        \item War cut off planters from Northern markets $\to$ difficult to sell cotton, removed male workspace
        \begin{itemize}
            \item Northern agriculture grew as a result while Southern declined
        \end{itemize}
        \item Battles on Southern soil $\to$ railroad destroyed, farmland/plantations lost
        \item Significant shortages in most key goods after Northern naval blockade
        \begin{itemize}
            \item Focused so much on cotton, export crops $\to$ not enough food for self-sufficiency 
            \item Women/slaves worked hard to keep farms functioning, but could not make up for reduced workforce
            \item Specialized workers like doctors, blacksmiths, carpenters often drafted 
        \end{itemize}
        \item Great instability emerged w/ food riots (often led by women), large demonstrations against conscription, "food draft," taxation
    \end{itemize}
    \textbf{Economically, the war was devastating for the South: cut off from all surrounding markets due to the Northern blockade, the South lost their primary source of income and were unable to produce sufficient food for themselves. Furthermore, the drafting of white males led to a decrease in specialized workers. Society became greatly unstable with frequent riots.}}
    \cornell{What were the social effects of the war?}{\begin{itemize}
        \item Southern women given far more responsibility
        \begin{itemize}
            \item Slaveowners' wives managed workforces, farmers' wives managed/worked in fields, specialized workers' wives took up jobs as teachers/govt. workers
            \item Women questioned Southern values dictating that certain social spheres were unsuited for women 
            \item Gender imbalance after war w/ women outnumbering men greatly $\to$ several unmarried/widowed, forced to work for themselves
        \end{itemize}
        \item Slaves, despite suffering from tightened slave codes due to fear of rebellion, able to escape in large numbers to the Union
        \begin{itemize}
            \item Those who felt hopeless resistant to owners due to weaker rule without patriarch owner
        \end{itemize}
    \end{itemize}
    \textbf{In the South, women were given far more responsibility, taking on traditionally male jobs, both during and after the war due to a long-term gender imbalance; several Southern women began to question their true place in society. Slaves, too, were given far more hope, resisting their owners' rule and frequently escaping.}}
    \cornell[Strategy and Diplomacy]{How did the North work to create military strategy and the South diplomatic strategy?}{\textbf{The North was commanded by Lincoln and later in the war, Ulysses S. Grant; the South's military strategy remained primarily under Davis' control. One advantage which the North was able to strategically leverage was their naval supremacy, allowing for a blockade which Southern forces were unable to outthink. Diplomatically, the South struggled to earn the support of European powers and were unable to truly leverage their cotton supplies. The West, too, was a region of great tension despite most all but Texas remaining in the Union, with bloody attacks in Kansas and Missouri.}}
    \cornell{Who were the primary war commanders on each side?}{\begin{itemize}
        \item Lincoln commander in chief for North after experience in state militia during Black Hawk War
        \begin{itemize}
            \item Aware of North's material advantages, exploited them; understood primary goal to destroy Southern armies
            \item Struggled to recruit suitable commanders
            \begin{itemize}
                \item Chief of staff initially Winfield Scott of Mexican War but retired
                \item Replaced by George McClellan, Eastern army commander; poor strategy, arrogant $\to$ returned to field in early 1862  
                \item Henry W. Halleck finally appointed at end of 1862 but strategy also poor, leaving decisions to Lincoln
                \item Finally found Ulysses S. Grant in March 1864, giving him relative freedom in decisions (but always consulted for major choices)
            \end{itemize}
            \item Lincoln/Grant's decisions constantly criticized by Committee on Conduct of the War, joining two houses of Congress
            \begin{itemize}
                \item Northern generals always seen as too ruthless
                \item Radicals claimed officers secretly supported slavery 
            \end{itemize}
        \end{itemize}
        \item Southern commands rooted in Davis, trained/professional soldier
        \begin{itemize}
            \item \textbf{Robert E. Lee} military adviser in early 1862, but wanted to keep full control $\to$ Lee commanded on field while Davis planned alone
            \item \textbf{Braxton Bragg} adviser in Feb. 1864, but mostly technical advice
            \item Finally created general in chief (Lee) in Feb. 1865 but wanted to keep all decisions to himself: not enough time for structure to solidify
        \end{itemize}
        \item Northern/Southern commanders on lower levels very similar
        \begin{itemize}
            \item Many graduates from West Point, Naval Academy at Annapolis $\to$ trained similarly, often friends w/ other side
            \item Generally relied on basic battle tactics; best commanders saw beyond to destruction of resources 
            \item Amateur officers critical, too - economic/social leaders in communities $\to$ rallied up groups 
        \end{itemize}
    \end{itemize}
    \textbf{In the North, Lincoln was the commander in chief but struggled to find a suitable commander, with most making poor strategic decisions; he finally settled on Ulysses S. Grant in March 1864. Davis, more well-versed in war than Lincoln, left major decisions to himself, with most of his advisers giving very little true advice. Both Northern and Southern leaders on lower levels implemented very similar strategies due to shared educational backgrounds.}}
    \cornell{How did naval warfare play a part in the Civil War?}{\begin{itemize}
        \item Union dominated in naval power 
        \begin{itemize}
            \item Created naval blockade of coast
            \begin{itemize}
                \item Navy kept large ships from entering but smaller ones could generally slip by for some time
                \item Union retaliated by capturing ports themselves
            \end{itemize}
            \item Critical in West for providing troops w/ supplies via river; South had only land-based forts
        \end{itemize}
        \item Confederates retaliated with newly developed weapons
        \begin{itemize}
            \item \textbf{Iron warship} produced out of \textit{Merrimac} (named \textit{Virginia}, left behind by Union supporters after VA seceded
            \begin{itemize}
                \item Attacked wooden ships of Union effectively 
                \item Union retaliated with \textit{Monitor}, unable to defeat \textit{Virginia} but able to keep blockade
            \end{itemize}
            \item Developed small torpedo boats, submarines, never able to truly overcome
        \end{itemize}
    \end{itemize}
    \textbf{The Union consistently dominated the Confederacy navally, creating a long-standing blockade and transporting supplies via river. However, the Confederacy attempted to fight back by outfitting an abandoned Union ship with iron (which was soon matched by the Union) and developing smaller innovations.}}
    \cornell{How did the South attempt to rally support from European powers?}{\begin{itemize}
        \item Confederate Sec. of State Judah Benjamin generally stuck to mundane; Union counterpart \textbf{William Seward} vital to long-term diplomatic success
        \item England/France aristocrats both supported Confederacy for cotton, goal to weaken U.S., idyllic social order similar to own societies 
        \begin{itemize}
            \item France unwilling to truly take sides unless England did first 
            \item British liberals (Bright + Cobden) opposed due to opposition to slavery, mobilizing lower-class workers particualrly after Emancipation Proclamation
        \end{itemize}
        \item South hoped to focus on "\textbf{King Cotton Diplomacy}" to counter British abolitionist forces
        \begin{itemize}
            \item Failed in that Britain had surplus $\to$ able to resist for some time; kept mills open w/ sources from Egypt/India
            \item Most workers who lost jobs from countless mills closing still supported Union 
            \item Ultimately, South's consistent projection to lose $\to$ European support too risky
        \end{itemize}
        \item Tensions still ran high betw. U.S. and GB
        \begin{itemize}
            \item War began $\to$ GB then France declared neutral $\to$ U.S. angered by their regard of conflict as one betw. \underline{two \textit{nations} of equal stature}
            \item Trent affair emerged when Bostonian Wilkes seized British ship carrying Confederate diplomats (Mason/Slidell), jailed in Boston
            \begin{itemize}
                \item GB demanded release, repayment, formal apology 
                \item Lincoln/Seward aware of violation $\to$ stalled until most Americans forgot, finally subtly releasing/apologizing 
            \end{itemize}
            \item GB sold six ships to South $\to$ US felt against neutrality, claiming British damage after war
        \end{itemize}
    \end{itemize}
    \textbf{Although European powers fundamentally supported the Confederacy due to their dependence on cotton, competition with the US, and belief in a similar social order, both English and France were hesitant and the South's attempt to push the importance of cotton failed due to British surplus supplies and alternate sources. Regardless, tensions were still strong between Britain and the US, with political and naval disputes plaguing the Civil War period.}}
    \cornell{How did the West get involved in the war?}{\begin{itemize}
        \item TX only Western state to join Confederacy, but several western states had many Southerners
        \item KS/MO had greatest fighting 
        \begin{itemize}
            \item \textbf{Quantrill} created Confederate band of teenage boys, killed all in path around KS-MO border (w/ slaughter of 150 in Lawrence, KS)
            \item Unionists known as \textbf{Jayhawkers}: nearly as savage, generally avenging those targeted by Quantrill 
            \item No major battle but consistent region of tension
        \end{itemize}
        \item Confederacy attempted to earn support of natives, who supported South due to poor treatment of U.S. govt. as well as many being slaveholders themselves; others hostile to slavery $\to$ divided 
        \begin{itemize}
            \item Led to Civil War w/in native territories as well as some groups allying w/ Union and others w/ Confederacy
        \end{itemize}
    \end{itemize}
    \textbf{Although the West had no organized battles and all but Texas remained in the Union, Kansas and Missouri were particularly tense due to the high populations of Southeners, with bands supporting each side ruthlessly murdering innocent inhabitants. Furthermore, Confederate forces attempted to enlist natives, but internal divisions made this force relatively ineffective.}}
    \cornell[The Course of Battle]{What was the play-by-play of the Civil War?}{\textbf{Marked by notable developments in technology, the Civil War was characterized by a series of bloody battles. The war began in the East with the First Battle of Bull Run, but remained a stalemate in Virginia for the following two years due to incompetent Northern generals and erroneous interpretations of the war as one of battles rather than one of depleting resources. Most of the Union progress was in the West, with Union forces able to take over the Mississippi River, New Orleans, and later the Tennessee River under Grant. 1863 marked the final attacks on Northern territory and further victories in the South. The stalemate finally concluded in Virginia under Grant in 1864 and 1865, whose siege on Petersburg drove Lee's forces from Richmond where they ultimately surrendered. Sherman took on the South, wreaking havoc on Atlanta and the surrounding regions. Johnston's forces finally surrendered in North Carolina, marking the end of the war with questions of industry and slavery still unanswered.}}
    \cornell{What was the Civil War by the numbers?}{\textbf{The Civil War saw 4 years of combat and 618,000 Americans dying, greater than all American wars combined pre-Vietnam, leading to times of great grief in the North and South alike.}}
    \cornell{What were the critical technologies employed in the Civil War?}{\begin{itemize}
        \item Unprecedented amount of technology $\to$ seen as first modern war 
        \item \textbf{Repeating weapons} (Colt created pistol, Winchester rifle) and improved cannons/artillery $\to$ new strategies emerged
        \begin{itemize}
            \item Former method of lines of soldiers attacking each other until one withdrew outdated and $\to$ complete slaughter for both sides
            \item All should stay low and under cover $\to$ less organized, chaotic battlefields
            \item Both sides began to form forts to defend from sieges (like those in Vicksburg, Petersburg, Richmond)
        \end{itemize}
        \item Newer technologies like hot-air balloons for reconnaissance, ironclad ships, submarines had potential but not yet truly ready
        \item \textbf{Railroads} and \textbf{telegraph} most revolutionary
        \begin{itemize}
            \item Rail travel $\to$ easy mobilization of troops but limited battles to around rail stations (regardless of strategical advantage)
            \item Telegraph initially insignificant due to unskilled operators, but U.S. Military Telegraph Corps $\uparrow$ training 
            \begin{itemize}
                \item Troops on both sides learned to string wires along routes for easy communication betw. field commanders
                \item Both sides learned to intercept messages 
            \end{itemize}
        \end{itemize}
    \end{itemize}
    \textbf{The first truly "modern" war, technology was central to the progress of the Civil War, from armaments like repeating weapons and cannons forcing new strategies to newer technologies like hot air balloons and submarines to truly revolutionary technologies like the railroad and the telegraph.}}
    \cornell{What marked the initial fights of the Civil War?}{\begin{itemize}
        \item First major battle in Northern VA w/ Union army of 30k under McDowell near Washington against slightly smaller Confed. army under \textbf{Beauregard} near Manassas
        \begin{itemize}
            \item \textbf{McDowell} sought to get over with for potentially immediate end to war $\to$ marched troops to Manassas while Beauregard moved behind Bull Run stream, calling for reinforcements to equalize size
            \item \textbf{First Battle of Bull Run} saw near-success by Union in dispersing forces but Southerners began counterattack 
            \begin{itemize}
                \item Panicked in hot weather, retreating without order but difficult due to obstacles of several civilian onlookers nearby 
                \item Confederates equally disordered $\to$ did not pursue
                \item Represented major hit to Lincoln's confidence
            \end{itemize}
        \end{itemize}
        \item MO: rebels seeking to secede supported by governor $\to$ Nathaniel Lyon brought Union troops to take on; killed but weakened Confederacy, allowing Union dominance in majority of state
        \item McClellan met western Virginians $\to$ own state govt. loyal to Union, eventally becoming \textbf{West Virginia} (unimportant strategically but key symbolically)
    \end{itemize}
    \textbf{The first battle was in Northern Virginia with Union McDowell against Confederate Beauregard, facing off with armies of relatively equal size in the First Battle of Bull Run; Union forces nearly won but succumbed to the hot weather and were driven out. In Missouri, Union forces began to dominate the state, and McClellan formed West Virginia.}}
    \cornell{What were the most important battles in the West?}{\begin{itemize}
        \item After First Battle of Bull Run, East remained stalemate for some time
        \item North's first goal in West to seize MS River $\to$ \textbf{Farragut} brought ironclads to Gulf of Mexico and surprised New Orleans from South $\to$ defenseless w/ easy win
        \begin{itemize}
            \item South greatly weakened at having lost key river, largest city 
        \end{itemize}
        \item At TN's Fort Henry + Donelson, Albert Sidney Johnston's forces stretched, far from Southern power centers
        \begin{itemize}
            \item Grant attacked Fort Henry w/ surprising ironclad boats over river $\to$ quick surrender
            \item Donelson more challenging but finally surrendered on Feb. 16, 1862
        \end{itemize}
        \item Grant then travelled over TN River, capturing railroad lines and ending at \textbf{Shiloh}, TN
        \begin{itemize}
            \item Met w/ force equal in size led by Johnston/Beauregard, killing Johnston in first day at Battle of Shiloh but driven back to river
            \item Grant returned w/ 25k extra reinforcements, forcing withdrawal; occupied central MS railroad hub, Corinth
        \end{itemize}
        \item Braxton Bragg succeeded Johnston as Western commander, hoping to regain TN/KY, starting from Chattanooga
        \begin{itemize}
            \item Faced Union army at Chattanooga aiming to capture area; maneuvered each other for several months
            \item Met on Dec. 31st but in Battle of Murfreesboro, Bragg forced to withdraw
        \end{itemize}
    \end{itemize}
    \textbf{In the West, the North easily seized New Orleans and the lower Mississippi River due to their surprise tactics. However, the more northern forces in Tennessee under Johnston put up more of a fight: Fort Henry and Donelson were taken over several days by Grant, who then travelled to Shiloh, initially poised to lose but surprising the Confederacy with reinforcements. Bragg succeeded Johnston after he had died at Shiloh, but quickly lost in a long battle around Chattanooga and was forced to withdraw.}}
    \cornell{How did the eastern front remain a stalemate?}{\begin{itemize}
        \item McClellan led Union operations in 1862: controversial for indecisiveness, often losing out on key strategical plays 
        \begin{itemize}
            \item Sought to capture Confed. capital at \textbf{Richmond} but took circuitous route to bypass defenses (Peninsular campaign)
        \item Brought 100k men; 30k left behind at Washington under McDowell initially but convinced Lincoln to send remainder of forces
        \item \textbf{Thomas "Stonewall" Jackson} changed plans before Lincoln sent rest of forces, travelling north across Shenandoah, appearing to be approaching Washington
        \begin{itemize}
            \item Lincoln sent 30k forces to take on Jackson but Union defeated in Valley campaign
        \end{itemize}
        \item Confederate troops under Joseph Johnston attacked McLennan's forces but initially unable to repel
        \begin{itemize}
            \item Johnston replaced by Lee who returned Stonewall Jackson's troops, started Battle of Seven Days in attempt to cut McClellan off
            \item McClellan reached prime position 25 mi from Richmond but too hesitant to take on $\to$ Lincoln recalled to Northern Virginia to join \textbf{John Pope}'s forces
        \end{itemize}
        \item McClellan departed by water $\to$ Lee moved overland to attack Pope's forces
        \begin{itemize}
            \item Aggressive Pope took on Lee's forces w/o backup, forcing army to retreat back to Washington in \textbf{Second Battle of Bull Run}
        \end{itemize}
        \end{itemize}
        \item Lincoln put Pope's forces in McClellan's hands as Lee began to head through western MD
        \begin{itemize}
            \item McClellan got copy of Lee's orders, learning that Stonewall Jackson had broken off to attack Harpers Ferry, but too slow to attack before forces recombined
            \item \textbf{Sharpsburg} saw bloodiest single-day battle w/ 6,000 killed, Confederate forces near breaking point but after Jackson's troops arrived, McClellan allowed easy retreat $\to$ soon removed permanently
        \end{itemize}
        \item Ambrose E. Burnside, replacement, attempted to cross strongest defense to reach Richmond, engaging in hopeless battles $\to$ requested personally to be removed from power
    \end{itemize}
    \textbf{The attacks on Richmond, the Confederate capital, were unsuccessful primarily due to incompetent generals including the aggressive Pope and the hesitant McClellan, who gave up countless critical opportunities to excuses.}}
    \cornell{Why did the war remain a relative stalemate on the Eastern front for so long?}{\begin{itemize}
        \item War remained undecided due to no single decisive battle: North blamed on timid/incompetent generals
        \item In truth, leaders had not realized that true way to win was by deterioration of resources, not winning battles
        \begin{itemize}
            \item As North continued to build infrastructure, South food supplies dwindled leading to riots 
        \end{itemize} 
    \end{itemize}
    \textbf{The war remained a stalemate primarily because it required the long-term deterioration of resources for either side to truly surrender.}}
    \cornell{What were the significant events of 1863?}{\begin{itemize}
        \item \textbf{Joseph Hooker} took control of McClellan's Army of the Potomac in East but irresolute
        \begin{itemize}
            \item Crossed Fredericksburg to approach Lee's army, but got scared at last moment and drew into defensive
            \item Lee's army half the size of Hooker's but still took on in Battle of Chancellorsville; Jackson soon followed 
            \begin{itemize}
                \item Hooker barely escaped, but objectives defeated, not army
            \end{itemize}
        \end{itemize}
        \item In West, Union still far more successful  
        \begin{itemize}
            \item Grant attacked well-protected stronghold at Vicksburg (MS) despite marshy terrain, sieging and forcing to surrender after six weeks due to starving 
            \item Port Hudson, LA surrendered to Union force from New Orleans, now controlling whole Mississippi River and splitting Confederacy (TX/AR/LA cut off)
        \end{itemize}
        \item Lee began to plan invasion of PN (during Vicksburg) to drive Union troops north from MS, win allegiance of Europe
        \begin{itemize}
            \item June 1863: Lee into PN w/ Union army (first under Hooker, then \textbf{Meade}) paralleling motions to stay between Lee/Washington
            \item Battle began under Meade at Gettysburg, w/ Lee outnumbered $\to$ failed on first day, also failing on second day after charging for nearly a mile under Union fire (Pickett's Charge); surrendered
            \item \textbf{Gettysburg} represented final serious attack on Northern territory
        \end{itemize}
        \item Turning point in TN after Union Rosencrans' forces pursued Bragg's retreating forces, meeting at GA border with Lee's ready to fight too
        \begin{itemize}
            \item Battle of Chickamauga saw Confederates w/ more troops $\to$ Union retreated to Chattanooga
            \item Bragg sieged Chattanooga, cutting off supplies but Grant returned for \textbf{Battle of Chattanooga}, driving back into GA and occupying eastern TN including \textbf{TN River} 
        \end{itemize}
    \end{itemize}
    \textbf{1863 saw a continued stalemate in Virginia, with the indecisive Hooker leaving his troops open to a combined attack by Jackson and Lee in Chancellorsville. In the West, however, the Union was successful with Grant's Siege of Vicksburg and the capture of Port Hudson splitting the Confederacy into two. Lee's attack on Pennsylvania (defeated at Gettysburg) marked the final genuine attempt at taking over the North. A final turning point of the year occured in Tennesse where Rosencrans and Grant were able to defeat Bragg at the Battle of Chattanooga to take over eastern Tennessee.}}
    \cornell{What were Grant and Sherman's key initial strategies in the final stages?}{\begin{itemize}
        \item Grant knew to exploit material advantage and incur casualties if reciprocated; led Army of Potomac to Richmond
        \begin{itemize}
            \item Initially turned back by Lee in \textbf{Battle of the Wilderness}
            \item Immediately resumed march, meeting again in \textbf{Battle of Spotsylvania Court House} w/ large losses for both
            \item Cold Harbor saw loss once again to Lee: had lost far more men than Lee so far
            \item Grant swapped strategy to seige \textbf{Petersburg}, rail center and cut off communications between Richmond and remainder of Confederacy
        \end{itemize}
        \item Western army took on Atlanta and Joseph Johnston under \textbf{William Tecumseh Sherman}
        \begin{itemize}
            \item Far less resistance w/ Johnston attempting to outmaneuver to delay attack; took on at battle of \textbf{Kennesaw Mountain} w/ win for Confederacy
            \item Union forces continued to approach Atlanta, \textbf{John B. Hood} replaced Johnston but far too aggressive, taking on Union twice only to lose both and weaken themselves in process
            \item Took Atlanta on September 2nd, burning the city
            \item Hood attempted to draw out by staging approach to North; Sherman ignored and instead sent forces to take over \textbf{Nashville}
        \end{itemize}
    \end{itemize}
    \textbf{In the final years of the Civil War, Grant and Sherman were the two key commanders. Grant initially went for Richmond but eventually restrategized and sieged Petersburg over nine months (slightly south of Richmond) to cut off supplies to the capital. Sherman led forces to Atlanta, swiftly taking it over and burning it.}}
    \cornell{How did Grant and Sherman wrap up the Civil War?}{\begin{itemize}
        \item Sherman underwent famous \textbf{March to the Sea} in late 1864 after taking over Georgia
        \begin{itemize}
            \item Relied on land for sustenance and destroyed anything else, depriving region of resources, transportation 
            \item Took over Savannah (without destroying significant amounts), continuing into NC unopposed until a brief delay from Johnston's forces
        \end{itemize}
        \item Grant continued at siege at Petersburg in April 1865, finally capturing rail junction linking to South
        \begin{itemize}
            \item Lee saw no point in defending Richmond w/ key link destroyed $\to$ headed South to link up with Johnston in NC; Union met up with forces and blocked route
            \item Lee agreed to meet Grant privately on April 9th, 1865 at the \textbf{Appomattox Court House}, surrendering forces; Johnston surrendered 9 days later in NC
        \end{itemize}
        \item War had ended but Davis initially would not accept; soon captured during escape to GA
    \end{itemize}
    \textbf{After capturing Atlanta, Sherman departed, travelling eastward and wreaking havoc on Southern resources. He then headed north and fought with Johnston's troops in North Carolina, but easily defeated them; Johnston later surrendered. After Grant successfully captured a critical rail junction near Petersburg, he cut off supplies from the South, droving Lee out of Richmond, soon allowing his Confederate forces to surrender. Davis himself would not surrender immediately until his capture soon after.}}
    \cornell{What was the impact of a Northern victory in the Civil War?}{\begin{itemize}
        \item Question of freed slaves (freedmen), future of South-North relations, and Southern industrialization remained unanswered at point of reunification
        \item North saw $\uparrow$ economy w/ expanded rail system while South had been depleted of its young male pop. $\to$ industry remained Northern
        \item African American slaves departed from masters, facing great future hardships but finally, 3.5m slaves were able to live freely and out of bondage
    \end{itemize}
    \textbf{The Northern victory established Northern industrial and economic supremacy for many years to come; it also freed millions of slaves and allowed them to depart from their masters and live comparatively unrestrained lives.}}
    \end{document}