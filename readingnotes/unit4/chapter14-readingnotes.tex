\documentclass[a4paper]{article}
    \usepackage[T1]{fontenc}
    \usepackage{tcolorbox}
    \usepackage{amsmath}
    \tcbuselibrary{skins}
    
    \title{
    \vspace{-3em}
    \begin{tcolorbox}
    \Huge\sffamily \begin{center} AP US History  \\
    \LARGE Chapter 13 - The Impending Crisis \\
    \Large Finn Frankis \end{center} 
    \end{tcolorbox}
    \vspace{-3em}
    }
    \date{}
    \author{}
    
    \usepackage{background}
    \SetBgScale{1}
    \SetBgAngle{0}
    \SetBgColor{red}
    \SetBgContents{\rule[0em]{4pt}{\textheight}}
    \SetBgHshift{-2.3cm}
    \SetBgVshift{0cm}
    \usepackage[margin=2cm]{geometry} 
    
    \makeatletter
    \def\cornell{\@ifnextchar[{\@with}{\@without}}
    \def\@with[#1]#2#3{
    \begin{tcolorbox}[enhanced,colback=gray,colframe=black,fonttitle=\large\bfseries\sffamily,sidebyside=true, nobeforeafter,before=\vfil,after=\vfil,colupper=blue,sidebyside align=top, lefthand width=.3\textwidth,
    opacityframe=0,opacityback=.3,opacitybacktitle=1, opacitytext=1,
    segmentation style={black!55,solid,opacity=0,line width=3pt},
    title=#1
    ]
    \begin{tcolorbox}[colback=red!05,colframe=red!25,sidebyside align=top,
    width=\textwidth,nobeforeafter]#2\end{tcolorbox}%
    \tcblower
    \sffamily
    \begin{tcolorbox}[colback=blue!05,colframe=blue!10,width=\textwidth,nobeforeafter]
    #3
    \end{tcolorbox}
    \end{tcolorbox}
    }
    \def\@without#1#2{
    \begin{tcolorbox}[enhanced,colback=white!15,colframe=white,fonttitle=\bfseries,sidebyside=true, nobeforeafter,before=\vfil,after=\vfil,colupper=blue,sidebyside align=top, lefthand width=.3\textwidth,
    opacityframe=0,opacityback=0,opacitybacktitle=0, opacitytext=1,
    segmentation style={black!55,solid,opacity=0,line width=3pt}
    ]
    
    \begin{tcolorbox}[colback=red!05,colframe=red!25,sidebyside align=top,
    width=\textwidth,nobeforeafter]#1\end{tcolorbox}%
    \tcblower
    \sffamily
    \begin{tcolorbox}[colback=blue!05,colframe=blue!10,width=\textwidth,nobeforeafter]
    #2
    \end{tcolorbox}
    \end{tcolorbox}
    }
    \makeatother

    \parindent=0pt
    
    \begin{document}
    \maketitle
    \SetBgContents{\rule[0em]{4pt}{\textheight}}
    \cornell[Key Concepts]{What are this chapter's key concepts?}{\begin{itemize}
        \item \textbf{5.1.I.D} - Civil War $\to$ increased Westward migration due to promoting legislation
        \item \textbf{5.2.II.D} - Lincoln won w/o any Southern votes $\to$ majority of slave states seceded $\to$ Civil War
        \item \textbf{5.3.I.A} - Despite opposition on home fronts, both sides of Civil War underwent economic/social preparation to fight
        \item \textbf{5.3.I.B} - Lincoln began war w/ goal to preserve Union, but Emancipation Proclamation $\to$ Europe would not support Confederacy, Afr. Americans fought for Union
        \item \textbf{5.3.I.C} - Lincoln's powerful speeches (like Gettysburg Address) portrayed slavery as in violation of democracy
        \item \textbf{5.3.I.D} - Union won war due to $\uparrow$ leadership, strategy (including destruction of Southern infrastructure), resources, 
    \end{itemize}}
    \cornell[The Secession Crisis]{What caused several Southern states to secede and how did the North react?}{\textbf{Secession began in South Carolina and was sparked by Lincoln's election; six states followed, forming the Confederacy. Crittenden led the effort for compromise which was supported by Southern senators, but Northern Republicans would not sacrifice their ideals. The Civil War began after the South took over Fort Sumter, pitting the Union, advantaged in materials, population, and transportation, against the South, with a firmer committment and the defensive ability of fighting on home territory.}}
    \cornell{Where did secession begin and what were initial reactions?}{\begin{itemize}
        \item SC (known for radical ideas) unanimously seceded in December 1860; MS, FL, AL, GA, LA, TX seceded by Lincoln's inauguration
        \begin{itemize}
            \item February 1861: seven seceded states met in AL, formed Confederate States of America
            \item Federal government indecisive: Buchanan told Congress that fed. govt. could not intervene in secession 
        \end{itemize}
        \item Seceded states took over federal property, justifying with anger/betrayal at Lincoln's election; Fort Sumter (SC)/Fort Pickens (FL), federal forts, not easily given up
        \begin{itemize}
            \item Initial attacks on forts unsuccessful 
            \item Buchanan ordered fortifications in Jan. 1861, even exchanging shots 
            \item Neither side admitted beginning of war
        \end{itemize}
    \end{itemize}
    \textbf{Secession began in South Carolina, quickly spreading to six other Southern states, who soon formed the Confederacy. The federal government was initially powerless to respond: Buchanan felt Congress had no right. After some seceded states began to take over federal property, however, the Union retaliated by fortifying Fort Sumter and Fort Pickens and fighting back.}}
    \cornell{What attempts were made at compromise?}{\begin{itemize}
        \item Sen. John Crittenden of KY formed Crittenden Compromise, constitutional amendments allowing for permanent slavery, Fugitive Slave Act, reestablishment of MO Compromise
        \begin{itemize}
            \item Southerners in Senate accepted; Republicans would not
        \end{itemize}
        \item Lincoln snuck into DC for inauguration due to fear of attack, giving inaugural address reprimanding secession and promising preservation of federal property
    \end{itemize}
    \textbf{Although the Crittenden Compromise aimed to give the South permanent slavery, formally establish the Fugitive Slave Act, and reinstate the Missouri Compromise and the Southern senators accepted it, the Republicans would not due to its fundamental disagreement with their party's ideals. Lincoln stood by his party at his inaugural address, promising that seceding states would be punished.}}
    \cornell{How did the Civil War begin at Fort Sumter?}{\begin{itemize}
        \item Sumter conditions quickly $\downarrow$ but Lincoln felt critical symbol of power of Union $\to$ sent relief supplies (w/o troops)
        \item Confederacy conflicted betw. cowardly decision of submitting to federal govt. and aggressive one of attacking Fort 
        \begin{itemize}
            \item Chose to attack under General Beauregard
            \item Bombarded for 2 days $\to$ Major Robert Anderson forced to surrender, beginning Civil War on Apr. 14, 1861
        \end{itemize}
        \item Lincoln felt secession infringed upon American liberty $\to$ mobilized North in conjunction w/ VA, AR, NC, TN seceding; border slave states pressured by DC to side w/ Union
        \item Central question: could war have been avoided? Sect. tensions had grown so large $\to$ something had to change
        \begin{itemize}
            \item North/South felt civilizations were 100\% incompatible $\to$ both sides supported war
        \end{itemize}
    \end{itemize}
    \textbf{The South chose to continue attacking Fort Sumter for fear of seeming cowardly; after they took it over and drove out the Union forces, Lincoln prepared the North for war as four more states seceded. The war began due to both the North and South feeling they were mutually incompatible.}}
    \cornell{What were the differences between the two sides of the war?}{\begin{itemize}
        \item North had material advantages: double pop., greater army/workforce, able to manufacture war materials (while South relied on Europe)
        \begin{itemize}
            \item North had more reliable transportation system by rail w/ greater integration
        \end{itemize}
        \item South had advantage of fighting war on home turf against North on hostile territory
        \begin{itemize}
            \item Southern whites entirely committed to war while Northern far more divided 
            \item South believed English/French textile industries needed cotton $\to$ instant support
        \end{itemize}
    \end{itemize}
    \textbf{The North had material advantages on paper, with a greater population, larger army, more power to create war materials for themselves, and an integrated transportation network. However, the South generally fought defensively on their home territory, were far more committed to the war, and hoped for the support of Europe.}}
    \end{document}