\documentclass[a4paper]{article}
    \usepackage[T1]{fontenc}
    \usepackage{tcolorbox}
    \usepackage{amsmath}
    \tcbuselibrary{skins}
    
    \title{
    \vspace{-3em}
    \begin{tcolorbox}
    \Huge\sffamily \begin{center} AP US History  \\
    \LARGE Chapter 13 - The Impending Crisis \\
    \Large Finn Frankis \end{center} 
    \end{tcolorbox}
    \vspace{-3em}
    }
    \date{}
    \author{}
    
    \usepackage{background}
    \SetBgScale{1}
    \SetBgAngle{0}
    \SetBgColor{red}
    \SetBgContents{\rule[0em]{4pt}{\textheight}}
    \SetBgHshift{-2.3cm}
    \SetBgVshift{0cm}
    \usepackage[margin=2cm]{geometry} 
    
    \makeatletter
    \def\cornell{\@ifnextchar[{\@with}{\@without}}
    \def\@with[#1]#2#3{
    \begin{tcolorbox}[enhanced,colback=gray,colframe=black,fonttitle=\large\bfseries\sffamily,sidebyside=true, nobeforeafter,before=\vfil,after=\vfil,colupper=blue,sidebyside align=top, lefthand width=.3\textwidth,
    opacityframe=0,opacityback=.3,opacitybacktitle=1, opacitytext=1,
    segmentation style={black!55,solid,opacity=0,line width=3pt},
    title=#1
    ]
    \begin{tcolorbox}[colback=red!05,colframe=red!25,sidebyside align=top,
    width=\textwidth,nobeforeafter]#2\end{tcolorbox}%
    \tcblower
    \sffamily
    \begin{tcolorbox}[colback=blue!05,colframe=blue!10,width=\textwidth,nobeforeafter]
    #3
    \end{tcolorbox}
    \end{tcolorbox}
    }
    \def\@without#1#2{
    \begin{tcolorbox}[enhanced,colback=white!15,colframe=white,fonttitle=\bfseries,sidebyside=true, nobeforeafter,before=\vfil,after=\vfil,colupper=blue,sidebyside align=top, lefthand width=.3\textwidth,
    opacityframe=0,opacityback=0,opacitybacktitle=0, opacitytext=1,
    segmentation style={black!55,solid,opacity=0,line width=3pt}
    ]
    
    \begin{tcolorbox}[colback=red!05,colframe=red!25,sidebyside align=top,
    width=\textwidth,nobeforeafter]#1\end{tcolorbox}%
    \tcblower
    \sffamily
    \begin{tcolorbox}[colback=blue!05,colframe=blue!10,width=\textwidth,nobeforeafter]
    #2
    \end{tcolorbox}
    \end{tcolorbox}
    }
    \makeatother

    \parindent=0pt
    
    \begin{document}
    \maketitle
    \SetBgContents{\rule[0em]{4pt}{\textheight}}
    \cornell[Key Concepts]{What are this chapter's key concepts?}{\begin{itemize}
        \item \textbf{5.1.I.D} - Civil War $\to$ increased Westward migration due to promoting legislation
        \item \textbf{5.2.II.D} - Lincoln won w/o any Southern votes $\to$ majority of slave states seceded $\to$ Civil War
        \item \textbf{5.3.I.A} - Despite opposition on home fronts, both sides of Civil War underwent economic/social preparation to fight
        \item \textbf{5.3.I.B} - Lincoln began war w/ goal to preserve Union, but Emancipation Proclamation $\to$ Europe would not support Confederacy, Afr. Americans fought for Union
        \item \textbf{5.3.I.C} - Lincoln's powerful speeches (like Gettysburg Address) portrayed slavery as in violation of democracy
        \item \textbf{5.3.I.D} - Union won war due to $\uparrow$ leadership, strategy (including destruction of Southern infrastructure), resources, 
    \end{itemize}}
    \cornell[The Secession Crisis]{What caused several Southern states to secede and how did the North react?}{\textbf{Secession began in South Carolina and was sparked by Lincoln's election; six states followed, forming the Confederacy. Crittenden led the effort for compromise which was supported by Southern senators, but Northern Republicans would not sacrifice their ideals. The Civil War began after the South took over Fort Sumter, pitting the Union, advantaged in materials, population, and transportation, against the South, with a firmer committment and the defensive ability of fighting on home territory.}}
    \cornell{Where did secession begin and what were initial reactions?}{\begin{itemize}
        \item SC (known for radical ideas) unanimously seceded in December 1860; MS, FL, AL, GA, LA, TX seceded by Lincoln's inauguration
        \begin{itemize}
            \item February 1861: seven seceded states met in AL, formed Confederate States of America
            \item Federal government indecisive: Buchanan told Congress that fed. govt. could not intervene in secession 
        \end{itemize}
        \item Seceded states took over federal property, justifying with anger/betrayal at Lincoln's election; Fort Sumter (SC)/Fort Pickens (FL), federal forts, not easily given up
        \begin{itemize}
            \item Initial attacks on forts unsuccessful 
            \item Buchanan ordered fortifications in Jan. 1861, even exchanging shots 
            \item Neither side admitted beginning of war
        \end{itemize}
    \end{itemize}
    \textbf{Secession began in South Carolina, quickly spreading to six other Southern states, who soon formed the Confederacy. The federal government was initially powerless to respond: Buchanan felt Congress had no right. After some seceded states began to take over federal property, however, the Union retaliated by fortifying Fort Sumter and Fort Pickens and fighting back.}}
    \cornell{What attempts were made at compromise?}{\begin{itemize}
        \item Sen. John Crittenden of KY formed Crittenden Compromise, constitutional amendments allowing for permanent slavery, Fugitive Slave Act, reestablishment of MO Compromise
        \begin{itemize}
            \item Southerners in Senate accepted; Republicans would not
        \end{itemize}
        \item Lincoln snuck into DC for inauguration due to fear of attack, giving inaugural address reprimanding secession and promising preservation of federal property
    \end{itemize}
    \textbf{Although the Crittenden Compromise aimed to give the South permanent slavery, formally establish the Fugitive Slave Act, and reinstate the Missouri Compromise and the Southern senators accepted it, the Republicans would not due to its fundamental disagreement with their party's ideals. Lincoln stood by his party at his inaugural address, promising that seceding states would be punished.}}
    \cornell{How did the Civil War begin at Fort Sumter?}{\begin{itemize}
        \item Sumter conditions quickly $\downarrow$ but Lincoln felt critical symbol of power of Union $\to$ sent relief supplies (w/o troops)
        \item Confederacy conflicted betw. cowardly decision of submitting to federal govt. and aggressive one of attacking Fort 
        \begin{itemize}
            \item Chose to attack under General Beauregard
            \item Bombarded for 2 days $\to$ Major Robert Anderson forced to surrender, beginning Civil War on Apr. 14, 1861
        \end{itemize}
        \item Lincoln felt secession infringed upon American liberty $\to$ mobilized North in conjunction w/ VA, AR, NC, TN seceding; border slave states pressured by DC to side w/ Union
        \item Central question: could war have been avoided? Sect. tensions had grown so large $\to$ something had to change
        \begin{itemize}
            \item North/South felt civilizations were 100\% incompatible $\to$ both sides supported war
        \end{itemize}
    \end{itemize}
    \textbf{The South chose to continue attacking Fort Sumter for fear of seeming cowardly; after they took it over and drove out the Union forces, Lincoln prepared the North for war as four more states seceded. The war began due to both the North and South feeling they were mutually incompatible.}}
    \cornell{What were the differences between the two sides of the war?}{\begin{itemize}
        \item North had material advantages: double pop., greater army/workforce, able to manufacture war materials (while South relied on Europe)
        \begin{itemize}
            \item North had more reliable transportation system by rail w/ greater integration
        \end{itemize}
        \item South had advantage of fighting war on home turf against North on hostile territory
        \begin{itemize}
            \item Southern whites entirely committed to war while Northern far more divided 
            \item South believed English/French textile industries needed cotton $\to$ instant support
        \end{itemize}
    \end{itemize}
    \textbf{The North had material advantages on paper, with a greater population, larger army, more power to create war materials for themselves, and an integrated transportation network. However, the South generally fought defensively on their home territory, were far more committed to the war, and hoped for the support of Europe.}}
    \cornell[The Mobilization of the North]{How did the North mobilize their troops for battle?}{\textbf{The North implemented critical economic changes under Republican dominance and, as a result, saw the growth of several industries and unions. However, these changes alone were often insufficient to finance the war and raise armies: Congress relied on loans from and conscriptions of the people themselves. After the war began, Lincoln shifted his view on slavery, siding with the growing radicalist movement pushing for immediate emanciaption, eventually signing the Emancipation Proclamation and allowing Southern blacks to join the Union forces without ramification. Despite being free to join Union forces, these blacks were often assigned menial tasks but still took their contributions with pride. Finally, the war promoted the feminist movement, with countless women becoming nurses and feeling empowered by a newfound freedom associated with the emancipation of slavery.}}
    \cornell{What economic changes were implemented by the North given Republican power in Congress?}{\begin{itemize}
            \item Homestead Act of 1862 allowed prospective citizens to claim 160 acres of land, purchase cheaply if inhabited for 5 yrs.
            \item Morill Land Grant Act gave public acreage to state govt. for public education $\to$ several state colleges/universities 
            \item Tariff bills $\to$ raised duties to unprecedented levels $\to$ domestic industries protected from foreign competition
            \item Transcontinental railroad w/ two companies
            \begin{itemize}
                \item Union Pacific Railroad Company to build westward from Omaha
                \item Central Pacific Railroad Company to build eastward from CA
                \item Two would meet in the middle, completing link 
            \end{itemize}
            \item National Bank Acts of 1863-1864 $\to$ national bank system 
            \begin{itemize}
                \item Existing banks to join if enough capital, willing to invest in govt. securities; allowed to issue U.S. Treasury notes 
            \end{itemize}
        \end{itemize}
        \textbf{The Northern Republicans, with Southern competitors out of the way, made land more accessible both to the public and to state governments for personal use and education, raised duties to support domestic industries, formed two railroad companies to create a transcontinental railroad, and began a national banking system.}}
    \cornell{How did the North finance the war?}{\begin{itemize}
        \item Congress $\uparrow$ taxes on most goods/services, creating income tax in 1861; heavily opposed
        \item Paper currency equally controversial: no gold/silver to back 
        \begin{itemize}
            \item Fluctuated based on army's success $\to$ govt. used limited amounts
        \end{itemize}
        \item Greatest source: loans from ppl. w/ Treasury convincing Americans to buy \$400m in bonds, paired w/ banks/large corporations assisting
        \end{itemize}
    \textbf{Congress implemented some widely opposed changes of raised taxes and paper currency (used in limited amounts), but mainly relied on loans from the people through bonds as well as from banks.}}
    \cornell{How did the North begin to raise armies?}{\begin{itemize}
        \item 2 million men fought in the Union in total, but the U.S. federal army began at only 16k (mostly in the West to prevent native rebellions)
        \item Lincoln raised regular army but knew that state militia volunteers were critical
        \begin{itemize}
            \item Authorized 500k volunteers in Congress: initially adequate but sunk with enthusiasm
        \end{itemize} 
        \item Forced to issue draft for any young adult male; could escape by paying \$300 or hiring someone else
        \begin{itemize}
            \item Conscription odd to ppl. used to remote govt. $\to$ opposition from laborers, immigrants, Democrats often leading to violence
            \item Only $\approx$ 46k were drafted but $\to$ increased voluntary enlistment
            \item Irish workers led one of deadliest American riots in NYC in 1863, lynching blacks due to fear of war $\to$ more competition for jobs
        \end{itemize}
    \end{itemize}
    \textbf{The primary sources for Northern armies were volunteers from state militias and drafted men. The concept of drafting enraged Americans happy with a distant and remote government, sparking a riot led by Irish workers in NYC who opposed the war due to a fear of more African Americans threatening their jobs.}}
    \cornell{What was the political state of the North during the war?}{\begin{itemize}
        \item Lincoln initially seen as inexperienced politician easy to control, but quickly asserted dominance
        \begin{itemize}
            \item Established cabinet made up of all Republican factions, many of whom opposed his presidency
            \item Violated aspects of Constitution (saw as better than losing it all): declared war, grew army, established blockade w/o Congress 
        \end{itemize}
        \item Lincoln experienced great opposition from Peace Democrats fearing reduced influence of agricultural Northwest, states' rights
        \begin{itemize}
            \item Retaliated by arresting dissenters with no right to be released even if arrested unlawfully; initially for border states but soon extended to all 
            \item Congressman Vallandigham of OH arrested, exiled to Confederacy after claiming war intended to free blacks but enslave whites 
            \item Ignored Taney's written demand for MD secessionist leader to be freed 
        \end{itemize}
        \item Lincoln built support w/ pro-war advertisements supported by photography group (led by Mathew Brady) to show terrible images of war
        \begin{itemize}
            \item Images met some w/ revulsion but many w/ patriotism and a desire to preserve the Union
        \end{itemize}
    \end{itemize}
    \textbf{Lincoln asserted a dominant position without difficulty in the Senate, often freely violating the Constitution for the sake of the war. He was opposed by Peace Democrats arguing for states' rights, but persecuted those who spoke against him. To build popular support for the war, he created public advertisements as well as drafted photographers to take jarring images of the destruction of the battles.}}
    \cornell{What was the result of the election of 1864?}{\begin{itemize}
        \item Republicans lost heavily in 1862 midterms $\to$ party leaders created Union Party, linking Republicans w/ War Democrats: nominated Lincoln and TN's Andrew Johnson (War Democrat)
        \item Democrats selected George B. McClellan, Union general relieved by Lincoln; although McClellan disagreed w/ Democratic goal for a truce, party continued to argue
        \item Major Northern victories (capture of Atlanta) $\to$ Republicans empowered w/ large majority of electoral votes but only 10\% greater popular vote 
    \end{itemize}
    \textbf{With the Republicans hurt in the 1862 midterms, they formed the Union Party to join forces with Democrats supporting the war. Lincoln was pitted against Peace Democrat McClellan, a former general. Lincoln won the election in large part due to luck: the election coincided with significant Northern victories in the battle.}}
    \cornell{How did emancipation play a central role in the Civil War?}{\begin{itemize}
        \item Republicans split across lines of slavery: radicals (like Thaddeus Stevens, Sumner, Wade) sought immediate abolishment while conservatives sought more gradual process 
        \item Support grew for emancipation near beginning of war
        \begin{itemize}
            \item Confiscation Act declared slaves used to fight for Confederacy as freed
            \item Radicals pushed second Confiscation Act, declaring slaves of \underline{any person} fighting for the Confederacy as freed and allowing Afr. Americans (including freed slaves) to fight for Union
            \item Radicals gradually grew in Republican party $\to$ Lincoln became their leader 
        \end{itemize}
        \item Sept. 1862: after victory at Antietam, Lincoln announced intention for emancipation; signed Emancipation Proclamation on Jan 1st, 1863
        \begin{itemize}
            \item Effectively freed all slaves in Confederacy (places not already controlled by Union, like border states, WV, southern LA, TN)
            \item Immediate effect insignificant (territories still controlled by Confederacy), but established war as one also against slavery
        \end{itemize}
        \item True liberating factor for slaves was war itself
        \begin{itemize}
            \item Confederacy often took slaves from plantations and employed to build defenses $\to$ close to border $\to$ easily escaped 
            \item Masters immediately lost any right to them $\to$ flocked to Union Army, some joining and others looking to reach free states 
        \end{itemize}
        \item As war ended, MD and MO had already abolished, as had TN, AR, and LA; Thirteenth Amendment did final duty
    \end{itemize}
    \textbf{Republicans were divided into radicals, who sought immediate emancipation, and conservatives, who sought a gradual freeing of slaves. Radicals grew in power as the war progressed, pushing for the Confiscation Act to immediately free almost all slaves, followed up by Lincoln's Emancipation Proclamation freeing all Confederate slaves. Slaves were most directly liberated not by the proclamation but by being allowed to enlist in the Union for the war itself.}}
    \cornell{How did African Americans fight for the Union?}{\begin{itemize}
        \item Emancipated Afr. Americans joined forces w/ free blacks, often facing obstacles
        \begin{itemize}
            \item Initially excluded from military w/ only a few black regiments out of necessity
            \item Emancipation Proclamation $\to$ black numbers swelled w/ active recruitment 
        \end{itemize}
        \item Some divided into fighting units (like Fifty-fourth MA Infantry w/ white commander Robert Gould Shaw)
        \item Most given non-fighting tasks like digging trenches $\to$ black mortality rate higher than white due to long hours, poor conditions, low pay
        \begin{itemize}
            \item Black soldiers still proud of significant contribution to war in long-term
        \end{itemize}
        \item Captured blacks in Confederacy either returned to original masters or executed
    \end{itemize}
    \textbf{African Americans played a crucial part in supporting the Northern cause despite facing obstacles of initially being unable to even enlist. After the Emancipation Proclamation, although enlistment was widespread, blacks were generally assigned to menial and backbreaking tasks.}}
    \cornell{How did the Civil War promote economic development?}{\begin{itemize}
        \item Some slowing of industrial growth w/ markets cut off from Southern goods
        \item Econ. development sped up due in part to Republican dominance but also conditions of war itself
        \begin{itemize}
            \item Coal (w/ $\uparrow$ production due to demand) and railroad industries (w/ standard gauge) forced to improve 
            \item Farms lost labor to armies $\to$ forced to mechanize agriculture
        \end{itemize}
        \item Industrial workers suffered w/ $\uparrow$ prices (70\% rise) but wages unable to meet (40\% rise)
        \begin{itemize}
            \item Liberal immigration $\to$ new workers keeping wage low 
            \item Mechanization eliminated skilled workers 
            \item Unions became far more widespread despite employer suppression
        \end{itemize}
    \end{itemize}
    \textbf{Although the cutoff of Southern raw materials hurt some industries, several grew significantly out of wartime need, like coal, railroads, and agricultural mechanization. However, industrial workers suffered greatly due to a decrease in purchasing power due to freer immigration laws and mechanization.}}
    \cornell{How did the war affect traditional gender roles?}{\begin{itemize}
        \item Women often took on foreign roles out of necessity, taking over male positions but most notably becoming nurses
        \begin{itemize}
            \item Dorothea Dix led U.S. Sanitary Commission: org. of civilian volunteers, pulling female nurses into field hospitals
            \item By 1900, nursing almost entirely female, caring for patients but performing other important tasks for hospital (like cooking/cleaning)
            \item Male doctors often felt women too weak for role but Sanitary Commission claimed nursing to represent manifestation of key domestic aspects of home life 
            \begin{itemize}
                \item Some stood up to male doctors, pushing incompetent ones aside
                \item Critical role $\to$ male complaints ignored
            \end{itemize}
            \item Nurses generally felt freed by war
            \item Nursing changed medical profession w/ wounded soldiers assisted greatly; Commission also appointed women behind the scenes and spread knowledge about hygiene
        \end{itemize}
        \item Feminists (like Cady Stanton/B. Anthony) founded Woman's Loyal League in 1863, fighting both for abolition of slavery and suffrage
        \begin{itemize}
            \item Clara Barton (assisted w/ nursing, Red Cross) felt war pushed women's rights far further than peace ever would have 
        \end{itemize}
    \end{itemize}
    \textbf{Several women felt liberated both during and after the war, becoming nurses on the battlefront under Dorothea Dix' U.S. Sanitary Commission despite the opposition of several men; they were critical to the health of soldiers. Feminists also capitalized on the movement to further their cause, pushing for both the abolition of slavery and for suffrage.}}
    \end{document}