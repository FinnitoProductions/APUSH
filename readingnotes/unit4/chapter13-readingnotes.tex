\documentclass[a4paper]{article}
    \usepackage[T1]{fontenc}
    \usepackage{tcolorbox}
    \usepackage{amsmath}
    \tcbuselibrary{skins}
    
    \title{
    \vspace{-3em}
    \begin{tcolorbox}
    \Huge\sffamily \begin{center} AP US History  \\
    \LARGE Chapter 13 - The Impending Crisis \\
    \Large Finn Frankis \end{center} 
    \end{tcolorbox}
    \vspace{-3em}
    }
    \date{}
    \author{}
    
    \usepackage{background}
    \SetBgScale{1}
    \SetBgAngle{0}
    \SetBgColor{red}
    \SetBgContents{\rule[0em]{4pt}{\textheight}}
    \SetBgHshift{-2.3cm}
    \SetBgVshift{0cm}
    \usepackage[margin=2cm]{geometry} 
    
    \makeatletter
    \def\cornell{\@ifnextchar[{\@with}{\@without}}
    \def\@with[#1]#2#3{
    \begin{tcolorbox}[enhanced,colback=gray,colframe=black,fonttitle=\large\bfseries\sffamily,sidebyside=true, nobeforeafter,before=\vfil,after=\vfil,colupper=blue,sidebyside align=top, lefthand width=.3\textwidth,
    opacityframe=0,opacityback=.3,opacitybacktitle=1, opacitytext=1,
    segmentation style={black!55,solid,opacity=0,line width=3pt},
    title=#1
    ]
    \begin{tcolorbox}[colback=red!05,colframe=red!25,sidebyside align=top,
    width=\textwidth,nobeforeafter]#2\end{tcolorbox}%
    \tcblower
    \sffamily
    \begin{tcolorbox}[colback=blue!05,colframe=blue!10,width=\textwidth,nobeforeafter]
    #3
    \end{tcolorbox}
    \end{tcolorbox}
    }
    \def\@without#1#2{
    \begin{tcolorbox}[enhanced,colback=white!15,colframe=white,fonttitle=\bfseries,sidebyside=true, nobeforeafter,before=\vfil,after=\vfil,colupper=blue,sidebyside align=top, lefthand width=.3\textwidth,
    opacityframe=0,opacityback=0,opacitybacktitle=0, opacitytext=1,
    segmentation style={black!55,solid,opacity=0,line width=3pt}
    ]
    
    \begin{tcolorbox}[colback=red!05,colframe=red!25,sidebyside align=top,
    width=\textwidth,nobeforeafter]#1\end{tcolorbox}%
    \tcblower
    \sffamily
    \begin{tcolorbox}[colback=blue!05,colframe=blue!10,width=\textwidth,nobeforeafter]
    #2
    \end{tcolorbox}
    \end{tcolorbox}
    }
    \makeatother

    \parindent=0pt
    
    \begin{document}
    \maketitle
    \SetBgContents{\rule[0em]{4pt}{\textheight}}
    \cornell[Key Concepts]{What are this chapter's key concepts?}{\begin{itemize}
        \item \textbf{5.1.I.A} - Sought for natural resources $\to$ settlers migrated Westward; sought econ./relig. opportunities 
        \item \textbf{5.1.I.B} - Manifest Destiny justified expansion to Pacific Ocean  
        \item \textbf{5.1.I.C} - U.S. added large territories in West after winning in Mexican-American War $\to$ questions about slavery, American Indians, Mexicans
        \item \textbf{5.2.I.A} - North's economy required free labor, South required slave labor; some Northerners only objected economically $\to$ free soil movement arguing that slavery was incompatible w/ free labor
        \item \textbf{5.2.I.C} - Slavery defendants focused on racial superiority based on Constitution
        \item \textbf{5.2.II.A} - Mexican Cession $\to$ sectional crisis as to whether to allow slavery in territories
        \item \textbf{5.2.II.B} - Courts/national leaders posed variety of solutions to resolve slavery, but ultimately unsuccessful
        \item \textbf{5.2.II.C} - Second Party System ended after slavery/nativism $\to$ sectional parties w/o strong alliances to either one
    \end{itemize}}
    \cornell[Looking Westward]{What were the critical effects of westward expansion?}{\textbf{Driven by Manifest Destiny, based around racial and political superiority, Americans migrated throughout the territory of modern America. Some migrated to Mexican Texas to form a new cotton kingdom, ultimately defeating Mexico's military dictator to declare independence for Texas; annexation by the United States was not immediate and native Mexicans were often ostracized. Oregon, too, was initially only inhabited by fur traders, but soon became a missionary hotbed and thousands flocked over the Oregon Trail to reach it; along with those who travelled to Santa Fe trail to Mexico, the westward migrants seeking to exploit America's economic freedom faced many hardships but enjoyed the collectivity of a larger group and a transplantation of American society.}}
    \cornell{What was the concept of Manifest Destiny?}{
        \begin{itemize}
            \item Expansion argued to benefit rest of society (never selfish)
            \item Beyond pol. pride: racial justification w/ superiority of white "American race" w/ "nonwhites" excluded 
            \begin{itemize}
                \item O'Sullivan argued "racial purity" was critical
            \end{itemize}
            \item Spread throughout nation by "penny press": inexpensive newspapers
            \item Advocates never uniform in goals w/ some seeking modern U.S., others Canada/Mexico/Carib., some entire world
            \begin{itemize}
                \item Some saw necessity of force
            \end{itemize}
            \item Others feared potential for revival of sectional conflict (valid concern) 
        \end{itemize}
        \textbf{Manifest Destiny reflected a growing pride in American nationalism with an idealistic vision of social perfection stimulating reform. Several argued it was never selfish, instead benefiting the inferior "nonwhites" in the rest of society.}}
        \cornell{What stimulated Americans to migrate to Texas?}{\begin{itemize}
            \item U.S. claimed Texas as Louisiana Purchase but renounced in 1819; Mexico had rejected two requests for purchase
            \item Early 1820s: MX government encouraged American migration to Texas to increase tax revenues, create buffer between natives and southern Mexicans, convert American loyalty to Mexico
            \begin{itemize}
                \item Thousands migrated due to rich soil for cotton; many brought slaves 
            \end{itemize}
            \item Most settlers migrated due to intermediaries, receiving land grants from Mexico in exchange for bringing settlers
            \begin{itemize}
                \item Stephen Austin most successful: first legal American settlement in Texas but created power source competing with Mexican govt.
                \item 1826: one of intermediaries led revolt to establish independent state of Fredonia but quickly crushed 
            \end{itemize}
            \item 1833: Mexicans issued useless immigration ban: Americans had outnumbered Mexicans significantly
        \end{itemize}
        \textbf{After the Mexican government opened American migration to Texas for increased tax revenue and for a buffer between Mexicans and the northern natives, migrants poured in through intermediaries, who received large land grants and encouraged settlers to migrate.}}
        \cornell{How did tensions develop between the United States and Mexico?}{\begin{itemize}
            \item Friction grew due to continuing migration, conflicting laws on slavery (MX govt. made illegal in TX)
            \begin{itemize}
                \item Some Americans sought peaceful settlement, others war for independence
            \end{itemize}
            \item Mid-1830s: General Antonio Lopez de Santa Anna took over Mexico, establishing autocracy
            \begin{itemize}
                \item Stricter laws increasing national power; Texans felt directly targeted, with even Austin imprisoned in Mexico City due to fear of revolt 
                \item By 1836, Americans declared independence from Mexico $\to$ Santa Anna brought army
                \begin{itemize}
                    \item Rebels disorganized on true leader $\to$ destroyed at Alamo w/ very light defense; Goliad, too, annihalated
                    \item General Sam Houston remained strong, defeating Mexican army at Battle of San Jacinto, taking Santa Anna prisoner $\to$ signed independence 
                \end{itemize}
            \end{itemize}
        \end{itemize}
        \textbf{Friction emerged as migration increased and immigrants disagreed with restrictive Mexican slavery laws. After Santa Anna, a military dictator, took over Mexico, establishing stricter nationalist laws, Americans declared independence and, despite some initial losses, ultimately triumphed over the Mexicans under the command of Sam Houston.}}
            \cornell{How did Texas develop soon after declaring independence?}{\begin{itemize}
            \item Mexican residents of Texas ("Tejanos") supported independence in war but mistrusted by Americans $\to$ mostly driven out or given subordinate status 
            \item American Texans sought U.S. annexation - first president, Sam Houston, sent delegates to Wash. to offer joining Union
            \begin{itemize}
                \item Some opposition in U.S. due to large slave territory, increase of Southern votes in Congress
                \item Jackson feared potential war w/ Mexico; Van Buren/Harrison also failed to pass
            \end{itemize}
            \item Texas remained independent for some time; some dreamed of rivaling U.S. $\to$ reached out to European powers
            \begin{itemize}
                \item England/France feared losing remainder of U.S. to Texas $\to$ concluded trade relations
                \item Tyler persuaded reapplication for statehood but Calhoun presented as new slave state $\to$ Northern senators prevented from passing
            \end{itemize}
        \end{itemize}
        \textbf{After declaring independence, many Americans ostracized the remaining Mexicans and drove them out as subordinates. Although under their first president, Sam Houston, they voted for annexation, the U.S. refused, particularly due to the opposition of Andrew Jackson and northerners fearing the expansion of slavery. After England and France concluded trade with Texas to prevent competition, Tyler attempted to integrate Texas into the Union but failed.}}
        \cornell{How did Oregon become a critical part of the sectional conflict?}{\begin{itemize}
            \item Control of Oregon Country (modern OR, WA, ID) in PNW major 1840s pol. issue because both GB/U.S. claimed land
            \begin{itemize}
                \item Agreed in 1818 treaty to allow equal access ("joint occupation")
                \item Neither U.S nor GB had created large presence around time of treaty: mostly American/Canadian fur traders w/ post at John Jacob Astor's company w/ farming + native labour 
            \end{itemize} 
            \item Interest grew in 1820s/30s as missionaries sought to begin evangelical expansion
            \begin{itemize}
                \item Witnessed strange appearance of four natives in Missouri (unable to speak English), beliving in divine need to expand westward 
                \item Sought to counter Catholic missionaries in Canada 
                \item Little success w/ converted tribes: experienced some resistance to efforts particularly among Cayuse (where Marcus Whitman and wife Narcissa established unsuccessful mission)
            \end{itemize}
            \item White Americans began emigration in early 1840s, outnumbering GB settlers and crushing natives w/ disease
            \begin{itemize}
                \item Cayuse tribe blamed Whitman for disease $\to$ killed 13 whites; whites continued to migrate through 1840s
            \end{itemize}
        \end{itemize}
        \textbf{The Oregon Country, jointly controlled by Britain and the United States, only began to truly develop in the 1820s/30s as missionaries like Marcus Whitman began to take interest. Americans quickly outnumbered the British settlers and many began to call for a complete American rule.}}
        \cornell{What were the main westward migrations in the mid-19th century?}{\begin{itemize}
            \item Most Southerners $\to$ Texas, but majority of migrants were prosperous young men from Old Northwest $\to$ California for gold rush
            \item Only poor migrants went in groups as laborers to assist on farms, ranches (men) or as servants, teachers, and prostitutes (women)
            \item Single men $\to$ places w/ lumbering/mining primary activity; families $\to$ farming regions
            \item All migrants sought new life, but differed in precise cause: some sought quick riches; others to take over new federal lands; others mercantile pursuits; others escaped religious persecution; most economic opportunities 
        \end{itemize}
        \textbf{Most Southerners migrated to Texas, but the majority of migrants came from the Northwest: most were prosperous young men but some were poor, mainly travelling in groups. Although every migrant desired to begin a new life, most sought to migrate for economic reasons (but some for religious and territorial purposes).}}
        \cornell{What were the hardships experienced by those who migrated westward?}{\begin{itemize}
            \item $\approx 300,000$ migrants betw. 1840-1860, travelling west after meeting in Iowa/Missouri along Oregon Trail (to OR) or Santa Fe Trail (to NM)
            \item Mountain/desert terrains challenging, constant pressure to reach destination before winter despite slow wagon pace
            \item Migrants often suffered from epidemics contracted while travelling
            \item Few migrants died due to native conflicts
            \begin{itemize}
                \item Most were assisted by tribes, who acted as guides and trading partners
                \item Few stories of conflict $\to$ many white travellers fearful 
            \end{itemize}
            \item Trail life generally based around traditional gender divisions
            \begin{itemize}
                \item Although all migrants walked most of the way, women worked harder than men, unable to rest after stopping
                \item Travellers generally enjoyed collective, group experience after spending months with close friends, relatives, and neighbors
                \item Inteisty w/o human contact $\to$ required close bonds 
            \end{itemize}
        \end{itemize}
        \textbf{The westward travellers struggled with tough terrain, a cooling climate as they approached winter, epidemics, and conficts with the natives (though they more often helped by trading with and guiding the travellers). In general, migrations closely mirrored American life with similar gender divisions and a focus on collectivity.}}
    \end{document}