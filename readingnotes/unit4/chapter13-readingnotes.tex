\documentclass[a4paper]{article}
    \usepackage[T1]{fontenc}
    \usepackage{tcolorbox}
    \usepackage{amsmath}
    \tcbuselibrary{skins}
    
    \title{
    \vspace{-3em}
    \begin{tcolorbox}
    \Huge\sffamily \begin{center} AP US History  \\
    \LARGE Chapter 13 - The Impending Crisis \\
    \Large Finn Frankis \end{center} 
    \end{tcolorbox}
    \vspace{-3em}
    }
    \date{}
    \author{}
    
    \usepackage{background}
    \SetBgScale{1}
    \SetBgAngle{0}
    \SetBgColor{red}
    \SetBgContents{\rule[0em]{4pt}{\textheight}}
    \SetBgHshift{-2.3cm}
    \SetBgVshift{0cm}
    \usepackage[margin=2cm]{geometry} 
    
    \makeatletter
    \def\cornell{\@ifnextchar[{\@with}{\@without}}
    \def\@with[#1]#2#3{
    \begin{tcolorbox}[enhanced,colback=gray,colframe=black,fonttitle=\large\bfseries\sffamily,sidebyside=true, nobeforeafter,before=\vfil,after=\vfil,colupper=blue,sidebyside align=top, lefthand width=.3\textwidth,
    opacityframe=0,opacityback=.3,opacitybacktitle=1, opacitytext=1,
    segmentation style={black!55,solid,opacity=0,line width=3pt},
    title=#1
    ]
    \begin{tcolorbox}[colback=red!05,colframe=red!25,sidebyside align=top,
    width=\textwidth,nobeforeafter]#2\end{tcolorbox}%
    \tcblower
    \sffamily
    \begin{tcolorbox}[colback=blue!05,colframe=blue!10,width=\textwidth,nobeforeafter]
    #3
    \end{tcolorbox}
    \end{tcolorbox}
    }
    \def\@without#1#2{
    \begin{tcolorbox}[enhanced,colback=white!15,colframe=white,fonttitle=\bfseries,sidebyside=true, nobeforeafter,before=\vfil,after=\vfil,colupper=blue,sidebyside align=top, lefthand width=.3\textwidth,
    opacityframe=0,opacityback=0,opacitybacktitle=0, opacitytext=1,
    segmentation style={black!55,solid,opacity=0,line width=3pt}
    ]
    
    \begin{tcolorbox}[colback=red!05,colframe=red!25,sidebyside align=top,
    width=\textwidth,nobeforeafter]#1\end{tcolorbox}%
    \tcblower
    \sffamily
    \begin{tcolorbox}[colback=blue!05,colframe=blue!10,width=\textwidth,nobeforeafter]
    #2
    \end{tcolorbox}
    \end{tcolorbox}
    }
    \makeatother

    \parindent=0pt
    
    \begin{document}
    \maketitle
    \SetBgContents{\rule[0em]{4pt}{\textheight}}
    \cornell[Key Concepts]{What are this chapter's key concepts?}{\begin{itemize}
        \item \textbf{5.1.I.A} - Sought for natural resources $\to$ settlers migrated Westward; sought econ./relig. opportunities 
        \item \textbf{5.1.I.B} - Manifest Destiny justified expansion to Pacific Ocean  
        \item \textbf{5.1.I.C} - U.S. added large territories in West after winning in Mexican-American War $\to$ questions about slavery, American Indians, Mexicans
        \item \textbf{5.2.I.A} - North's economy required free labor, South required slave labor; some Northerners only objected economically $\to$ free soil movement arguing that slavery was incompatible w/ free labor
        \item \textbf{5.2.I.C} - Slavery defendants focused on racial superiority based on Constitution
        \item \textbf{5.2.II.A} - Mexican Cession $\to$ sectional crisis as to whether to allow slavery in territories
        \item \textbf{5.2.II.B} - Courts/national leaders posed variety of solutions to resolve slavery, but ultimately unsuccessful
        \item \textbf{5.2.II.C} - Second Party System ended after slavery/nativism $\to$ sectional parties w/o strong alliances to either one
    \end{itemize}}
    \cornell[Looking Westward]{What were the critical effects of westward expansion?}{\textbf{Driven by Manifest Destiny, based around racial and political superiority, Americans migrated throughout the territory of modern America. Some migrated to Mexican Texas to form a new cotton kingdom, ultimately defeating Mexico's military dictator to declare independence for Texas; annexation by the United States was not immediate and native Mexicans were often ostracized. Oregon, too, was initially only inhabited by fur traders, but soon became a missionary hotbed and thousands flocked over the Oregon Trail to reach it; along with those who travelled to Santa Fe trail to Mexico, the westward migrants seeking to exploit America's economic freedom faced many hardships but enjoyed the collectivity of a larger group and a transplantation of American society.}}
    \cornell{What was the concept of Manifest Destiny?}{
        \begin{itemize}
            \item Expansion argued to benefit rest of society (never selfish)
            \item Beyond pol. pride: racial justification w/ superiority of white "American race" w/ "nonwhites" excluded 
            \begin{itemize}
                \item O'Sullivan argued "racial purity" was critical
            \end{itemize}
            \item Spread throughout nation by "penny press": inexpensive newspapers
            \item Advocates never uniform in goals w/ some seeking modern U.S., others Canada/Mexico/Carib., some entire world
            \begin{itemize}
                \item Some saw necessity of force
            \end{itemize}
            \item Others feared potential for revival of sectional conflict (valid concern) 
        \end{itemize}
        \textbf{Manifest Destiny reflected a growing pride in American nationalism with an idealistic vision of social perfection stimulating reform. Several argued it was never selfish, instead benefiting the inferior "nonwhites" in the rest of society.}}
        \cornell{What stimulated Americans to migrate to Texas?}{\begin{itemize}
            \item U.S. claimed Texas as Louisiana Purchase but renounced in 1819; Mexico had rejected two requests for purchase
            \item Early 1820s: MX government encouraged American migration to Texas to increase tax revenues, create buffer between natives and southern Mexicans, convert American loyalty to Mexico
            \begin{itemize}
                \item Thousands migrated due to rich soil for cotton; many brought slaves 
            \end{itemize}
            \item Most settlers migrated due to intermediaries, receiving land grants from Mexico in exchange for bringing settlers
            \begin{itemize}
                \item Stephen Austin most successful: first legal American settlement in Texas but created power source competing with Mexican govt.
                \item 1826: one of intermediaries led revolt to establish independent state of Fredonia but quickly crushed 
            \end{itemize}
            \item 1833: Mexicans issued useless immigration ban: Americans had outnumbered Mexicans significantly
        \end{itemize}
        \textbf{After the Mexican government opened American migration to Texas for increased tax revenue and for a buffer between Mexicans and the northern natives, migrants poured in through intermediaries, who received large land grants and encouraged settlers to migrate.}}
        \cornell{How did tensions develop between the United States and Mexico?}{\begin{itemize}
            \item Friction grew due to continuing migration, conflicting laws on slavery (MX govt. made illegal in TX)
            \begin{itemize}
                \item Some Americans sought peaceful settlement, others war for independence
            \end{itemize}
            \item Mid-1830s: General Antonio Lopez de Santa Anna took over Mexico, establishing autocracy
            \begin{itemize}
                \item Stricter laws increasing national power; Texans felt directly targeted, with even Austin imprisoned in Mexico City due to fear of revolt 
                \item By 1836, Americans declared independence from Mexico $\to$ Santa Anna brought army
                \begin{itemize}
                    \item Rebels disorganized on true leader $\to$ destroyed at Alamo w/ very light defense; Goliad, too, annihalated
                    \item General Sam Houston remained strong, defeating Mexican army at Battle of San Jacinto, taking Santa Anna prisoner $\to$ signed independence 
                \end{itemize}
            \end{itemize}
        \end{itemize}
        \textbf{Friction emerged as migration increased and immigrants disagreed with restrictive Mexican slavery laws. After Santa Anna, a military dictator, took over Mexico, establishing stricter nationalist laws, Americans declared independence and, despite some initial losses, ultimately triumphed over the Mexicans under the command of Sam Houston.}}
            \cornell{How did Texas develop soon after declaring independence?}{\begin{itemize}
            \item Mexican residents of Texas ("Tejanos") supported independence in war but mistrusted by Americans $\to$ mostly driven out or given subordinate status 
            \item American Texans sought U.S. annexation - first president, Sam Houston, sent delegates to Wash. to offer joining Union
            \begin{itemize}
                \item Some opposition in U.S. due to large slave territory, increase of Southern votes in Congress
                \item Jackson feared potential war w/ Mexico; Van Buren/Harrison also failed to pass
            \end{itemize}
            \item Texas remained independent for some time; some dreamed of rivaling U.S. $\to$ reached out to European powers
            \begin{itemize}
                \item England/France feared losing remainder of U.S. to Texas $\to$ concluded trade relations
                \item Tyler persuaded reapplication for statehood but Calhoun presented as new slave state $\to$ Northern senators prevented from passing
            \end{itemize}
        \end{itemize}
        \textbf{After declaring independence, many Americans ostracized the remaining Mexicans and drove them out as subordinates. Although under their first president, Sam Houston, they voted for annexation, the U.S. refused, particularly due to the opposition of Andrew Jackson and northerners fearing the expansion of slavery. After England and France concluded trade with Texas to prevent competition, Tyler attempted to integrate Texas into the Union but failed.}}
        \cornell{How did Oregon become a critical part of the sectional conflict?}{\begin{itemize}
            \item Control of Oregon Country (modern OR, WA, ID) in PNW major 1840s pol. issue because both GB/U.S. claimed land
            \begin{itemize}
                \item Agreed in 1818 treaty to allow equal access ("joint occupation")
                \item Neither U.S nor GB had created large presence around time of treaty: mostly American/Canadian fur traders w/ post at John Jacob Astor's company w/ farming + native labour 
            \end{itemize} 
            \item Interest grew in 1820s/30s as missionaries sought to begin evangelical expansion
            \begin{itemize}
                \item Witnessed strange appearance of four natives in Missouri (unable to speak English), beliving in divine need to expand westward 
                \item Sought to counter Catholic missionaries in Canada 
                \item Little success w/ converted tribes: experienced some resistance to efforts particularly among Cayuse (where Marcus Whitman and wife Narcissa established unsuccessful mission)
            \end{itemize}
            \item White Americans began emigration in early 1840s, outnumbering GB settlers and crushing natives w/ disease
            \begin{itemize}
                \item Cayuse tribe blamed Whitman for disease $\to$ killed 13 whites; whites continued to migrate through 1840s
            \end{itemize}
        \end{itemize}
        \textbf{The Oregon Country, jointly controlled by Britain and the United States, only began to truly develop in the 1820s/30s as missionaries like Marcus Whitman began to take interest. Americans quickly outnumbered the British settlers and many began to call for a complete American rule.}}
        \cornell{What were the main westward migrations in the mid-19th century?}{\begin{itemize}
            \item Most Southerners $\to$ Texas, but majority of migrants were prosperous young men from Old Northwest $\to$ California for gold rush
            \item Only poor migrants went in groups as laborers to assist on farms, ranches (men) or as servants, teachers, and prostitutes (women)
            \item Single men $\to$ places w/ lumbering/mining primary activity; families $\to$ farming regions
            \item All migrants sought new life, but differed in precise cause: some sought quick riches; others to take over new federal lands; others mercantile pursuits; others escaped religious persecution; most economic opportunities 
        \end{itemize}
        \textbf{Most Southerners migrated to Texas, but the majority of migrants came from the Northwest: most were prosperous young men but some were poor, mainly travelling in groups. Although every migrant desired to begin a new life, most sought to migrate for economic reasons (but some for religious and territorial purposes).}}
        \cornell{What were the hardships experienced by those who migrated westward?}{\begin{itemize}
            \item $\approx 300,000$ migrants betw. 1840-1860, travelling west after meeting in Iowa/Missouri along Oregon Trail (to OR) or Santa Fe Trail (to NM)
            \item Mountain/desert terrains challenging, constant pressure to reach destination before winter despite slow wagon pace
            \item Migrants often suffered from epidemics contracted while travelling
            \item Few migrants died due to native conflicts
            \begin{itemize}
                \item Most were assisted by tribes, who acted as guides and trading partners
                \item Few stories of conflict $\to$ many white travellers fearful 
            \end{itemize}
            \item Trail life generally based around traditional gender divisions
            \begin{itemize}
                \item Although all migrants walked most of the way, women worked harder than men, unable to rest after stopping
                \item Travellers generally enjoyed collective, group experience after spending months with close friends, relatives, and neighbors
                \item Inteisty w/o human contact $\to$ required close bonds 
            \end{itemize}
        \end{itemize}
        \textbf{The westward travellers struggled with tough terrain, a cooling climate as they approached winter, epidemics, and conficts with the natives (though they more often helped by trading with and guiding the travellers). In general, migrations closely mirrored American life with similar gender divisions and a focus on collectivity.}}
        \cornell[Expansion and War]{How did American expansion stimulate the Mexican-American war?}{\textbf{After Democrat Polk won the presidency for his strong stance on annexation, and tensions rose concerning Texas' southern border as well as over ownership of New Mexico and California, Polk encouraged the Mexican-American War despite moral and political opposition. American troops dominated the Mexicans, eventually annexing California and New Mexico; the Mexicans finally surrendered with the Treaty of Guadalupe Hidalgo after American troops seized Mexico City.}}
        \cornell{How did the Democrats stand for expansion?}{\begin{itemize}
            \item Neither Clay (W.)/Van Buren (D.) took stand on annexation; Whigs nominated Clay but Democrats nominated Polk
            \begin{itemize}
                \item Polk represented TN in House of Representatives for 14 yrs.: spent 4 yrs. as Speaker; later governor 
                \item Disappeared for 3 yrs. but returned for election, won nomination and election due to strong support for OR/TX annexation
            \end{itemize}
            \item Tyler accomplished annexation of TX near end of presidency (1845); Polk personally resolved OR 
            \begin{itemize}
                \item GB minister in WA completely ignored initial offer for formal U.S./Canada border at 49th parallel
                \item Tensions rose w/ Americans often asserting "54-40 or fight!" referencing desired border location
                \item Neither U.S./GB wanted war $\to$ GB eventually accepted orig. offer in 1846
            \end{itemize}
        \end{itemize}
            \textbf{After Democrat James Polk won the presidential election due to his strong stance on the annexation of Texas and Oregon, outgoing president Tyler received congressional approval to annex Texas and Polk resolved Oregon with a resolute proposal for the British which was eventually accepted.}}
        \cornell{How did tensions begin to rise in the Southwest?}{\begin{itemize}
            \item U.S. annexation of Texas $\to$ Mexico ended diplomacy w/ U.S.
            \item Tensions grew after Texas asserted Rio Grande marked Southern border (included modern NM) while Texans pushed for Nueces River
            \item Several territories became problematic as Americans began to pour in and outnumber Mexicans
            \begin{itemize}
                \item NM debated due to disruption of long-standing native/Spanish society by pouring of U.S. immigrants, becoming more American than Mexican
                \item CA inhabited by native tribes, small number of Mexicans ($\approx 7,000$)
                \begin{itemize}
                    \item White Americans slowly arrived initially on whaling ships, creating small stores for trade
                    \item Farmers later went to Sacramento Valley 
                \end{itemize}
                \item Polk sought both NM/CA $\to$ told troops sent to annex TX to also seize CA ports if MX declared war, inform CA Americans that U.S. would support revolt
            \end{itemize}
            \end{itemize}
            \textbf{Tensions rose in the Southwest as Texans sought their southern border to include much of modern New Mexico; both New Mexico, where a long-standing social harmony between Spanish and natives was disrupted by Americans, and California, with few Mexicans but growing numbers of American whalers and farmers, became desirable to Polk.}}
            \cornell{What were the critical events of the Mexican-American War?}{\begin{itemize}
                \item Polk initially tried to buy territories from Mexicans thru. diplomat John Slidell; rejected $\to$ Polk sent Taylor's TX army across Nueces to Rio Grande
                \begin{itemize}
                    \item Mexicans initially refused to fight, but Americans claimed eventually crossed Rio Grande and attacked American troops $\to$ Congress approved war
                \end{itemize}
                \item American forces dominated Mexicans but war still time-consuming
                \begin{itemize}
                    \item Taylor ordered to cross Rio Grande, seizing northern MX (first Monterrey, then Mexico City)
                    \begin{itemize}
                        \item Seized Monterrey but allowed army to escape $\to$ Polk feared competence, potential political rivalry if MX City successful
                    \end{itemize} 
                    \item Other troops ordered $\to$ NM/CA, led by Colonel Kearny 
                    \begin{itemize}
                        \item Captured Santa Fe w/o opposition
                        \item Moved into CA, joining existing conflicts where settlers + exploring party (under Frémont) + U.S. Navy initiated Bear Flag Revolution
                    \end{itemize}
                \end{itemize}
                \item U.S. controlled territories which had inspired war but MX refused defeat
                \begin{itemize}
                    \item Polk sent General Scott w/ large army $\to$ Tampico, eventually MX City, winning every battle and finally dominating capital $\to$ MX govt. agreed to form peace treaty
                \end{itemize}
                \item Nearing election $\to$ Polk sought to end war $\to$ sent Nicholas Trist to negotiate Treaty of Guadalupe Hidalgo in Feb. 1848 
                \begin{itemize}
                    \item Ceded NM/CA to U.S., allowed Rio Grande to be border of TX; absorbed debts of citizens to Mexico, paid \$15 million
                    \item Trist had not agreed on additional Mexican territory $\to$ Polk angry but forced to accept due to strong antislavery argument
                \end{itemize}
            \end{itemize}
            \textbf{After Polk's attempt to buy off the Mexicans failed, he iniated war by sending troops into Texan territories which Mexico claimed for themselves; war became official after the Mexicans eventually retaliated. Kearney led troops into New Mexico and California, ultimately annexing those two regions, and Taylor into Monterrey, successfully taking over the city. After the U.S. had annexed all desired territories, Mexico would not concede defeat so Polk sent General Scott to seize Mexico City, finally forming the Treaty of Guadelupe Hidalgo, receiving all desired territories for a fee of \$15 million.}}
        \cornell{What were the main sources of opposition to the Mexican-American War?}{\begin{itemize}
            \item Whig critics felt Polk had manipulated battles, staged inciting incident
            \item Many felt greater focus should be on Pacific NW, arguing that Polk settled finally settled for low amt. w/ GB because focus was on SW
            \item Moral opposition, w/ Ulysses Grant claiming "unjust," Lincoln criticizing for power given to Polk, Thoreau going refusing to pay taxes as pacifist
        \end{itemize}
        \textbf{The war was criticized by Whigs on partisan grounds, others on grounds that the Pacific Northwest should be prioritized over the Southwest, and others on moral grounds.}}
        \cornell[The Sectional Debate]{How was sectionalism further aggravated in the antebellum period?}{\textbf{Sectionalism was further aggravated in the U.S. after Taylor took the presidency in 1848 and he demanded for new territories to be admitted as states, causing the South to balk in fear of losing an equal ratio of slave to free states. Debates ensued, first between the older leaders espousing broad ideas of nationalism, and next between younger leaders who finally formed a compromise in 1850.}}
        \cornell{How did slavery impact the Presidential Election of 1848?}{\begin{itemize}
            \item Aug. 1846 (during MX-AM war): Polk asked Congress for \$2m to purchase peace w/ MX; David Wilmot (D.) added amendment to prohibit slavery in Mexican territories but rejected in Senate 
            \item Sectional debate $\uparrow$ $\to$ Polk sought to extend MO compromise along line to Pacific coast; others sought popular vote of territories to formally decide 
            \item Presidential campaign of 1848 $\to$ weakened controversy as both parties tried to avoid question
            \begin{itemize}
                \item Polk's health $\to$ didn't run: instead Lewis Cass (D.) and Taylor (W.) 
                \item Free-Soil Party populated by Liberty Party, Anti-Slavery Whigs/Dems.; led by Van Buren 
            \end{itemize} 
            \item Taylor won narrowly, but Van Buren impressive in electing 10 to Congress, representing growth of slave-based tensions
        \end{itemize}
        \textbf{Sectionalist tensions rose after David Wilmot created an amendment prohibiting slavery in all Mexican territories; Polk adressed them by proposing to extend the Missouri Compromise westward; others sought to give the choice to each respective territories. Tensions were quelled, however, after the Presidential Campaign of 1848 weakened controversies.}}
        \cornell{How did the California Gold Rush stimulate westward migration?}{\begin{itemize}
            \item Jan. 1848: James Marshall found gold in Sierra Nevadas $\to$ frantic migration in search for gold, stimulating Gold Rush 
            \item Gold Rush based around greed, excitement; SFO depopulated as residents $\to$ mountains
            \begin{itemize}
                \item Migrants generally threw away lives in search of gold (majority men)
                \item Chinese migrants flocked to California after discovery, with many impoverished Chinese hoping to get rich in CA, return later 
                \item 100\% voluntary migration as free laborers/merchants
                \item Labor shortage $\to$ Chinese not seeking gold filled labor shortages created by Gold Rush; natives often exploited to fill jobs as state law allowed indentured servants 
            \end{itemize}
            \item Gold Rush key for California not due to wealth (many did get rich but most either unable to search for it/find any at all) but for population growth
            \begin{itemize}
                \item Most dissatisfied arrivals stayed behind, growing pop. size; SFO grew to $50,000$ 
                \item Migrants included Europeans, Chinese, South Americans, Mexicans, free blacks/slaves
                \item Gold conflicts $\to$ racial + ethnic tensions
            \end{itemize}
        \end{itemize}
        \textbf{The Gold Rush, beginning in early 1848, had the short term effect of emptying out San Francisco but, in the long-term, countless Chinese migrants arrived in California, filling labor shortages and swelling California's population.}}
        \cornell{How did sectional tensions develop in new territories?}{\begin{itemize}
            \item Taylor sought to admit new territories (NM/CA) as states ASAP because while territories, federal government required to resolve issue of slavery but could pass onto states after admittance
            \item Congress rejected due to existing slavery conflicts: abolishment in DC, personal liberty laws in North preventing runaway slaves from being returned; also due to Southern fear of losing equal slave-free balance
            \item Tensions rose as even moderate Southern leaders called for secession; Northern legislatures demanded slavery be prohibited in all territories
        \end{itemize}
        \textbf{Sectional tensions were furthered in that the South feared losing an equal balance between slave and free states whem California and New Mexico were admitted, causing both Northern and Southern states to consider drastic changes on the state level.}}
        \cornell{What was the Compromise of 1850?}{\begin{itemize}
            \item Clay led goal for compromise betw. North/South by combining several smaller measures into large piece of legislation
            \begin{itemize}
                \item Admission of CA as free state, territorial govt. formation in lands from Mexico (w/o slavery restrictions), formalization of fugitive slave laws 
            \end{itemize}
            \item Clay's legislation $\to$ Congressional debate divided into two phases
            \begin{itemize}
                \item First phase: old men dominated (those recalling words of Jefferson/Adams, founding fathers), focusing on broad ideologies
                \begin{itemize}
                    \item Clay focused on nationalism
                    \item Calhoun demanded that North give South equal rights in new territories, stop attacking slavery, create two Presidents of U.S.: one from North and other from South 
                    \item Webster delivered powerful oratory supporting Clay's compromise 
                \end{itemize}
                \item Second phase: after Congress overturned Clay's proposal, Clay became ill + Calhoun died + Webster took on position as sec. of state $\to$ mostly young leaders 
                \begin{itemize}
                    \item William H. Seward (49) abhorred political compromise
                    \item Jefferson Davis of MO (42) represented cotton kingdom 
                    \item Stephen A. Douglas, Democratic senator of IL, pushed self-promotion and economic needs of home state 
                    \item Compromise successful after Taylor, demanding CA and NM as states, died; succeeded by Fillmore, who supported compromise
                    \item Douglas broke up larger bill into separate ideas to form a true compromise; also linked ideas to nonideological (more easily agreed upon) ideas of govt. bonds, railroads
                \end{itemize}
            \end{itemize}
            \item Compromise of 1850 not as resolute as MO Compromise: not based on widespread ideological agreement but instead some personal benefits for both
        \end{itemize}
        \textbf{The Compromise of 1850, spurred after Henry Clay sought to form a large compromise between the North and South by lumping together multiple bills, began with debates by older nationalists who ultimately failed and saw the compromise overturned; it was followed by younger leaders who split the large bill into many and tied ideas to nonideological concepts to promote compromise.}}
        \cornell[The Crises of the 1850s]{What were the main disasters of the 1850s?}{\textbf{The 1850s saw numerous crises all based around sectional differences. For one, the discussion of whether the transcontinental railroad should terminate in the North or the South sparked the Gadsen Purchase as well as the Kansas-Nebraska Act, which violated the Missouri Compromise to create a slave state north of the defined demarcation. Kansas, the result of this, suffered great internal divisions between pro-slavers and free-staters, but ultimately emerged a free state. Antislavers took a big hit when Taney decided that slaves had no Constitutional right to sue for their freedom. As tensions rose, Abraham Lincoln, with a firm yet moderate stance on slavery, took the presidency in the election of 1860, eventually sparking disunion in the South.}}
        \cornell{What was the result of the election of 1852?}{\begin{itemize}
            \item Dems. and Whigs endorsed Compromise of 1850, w/ Franklin Pierce (D.) against General Winfield Scott (W.); both relatively obscure
            \begin{itemize}
                \item Internal divisions among Whigs due to flip-flop stance on slavery $\to$ many quit to join Free-Soil Party, giving Pierce the win 
            \end{itemize}
            \item Pierce tried to avoid issue of slavery, but after Fugitive Slave Act of 1850 allowed southerners to arrive in Northern cities, claim to have found fugitive slaves and bring to South
            \begin{itemize}
                \item North 100\% defiant to concept, with mobs forming to prevent enforcement, some states overturning
                \item Completely overturned Compromise of 1850
            \end{itemize}
        \end{itemize}
        \textbf{After Democrat Pierce won the election as a result of Whig vagueness on the issue of slavery, he attempted to avoid the issue of slavery but struggled as the Fugitive Slave Act, allowing southerners to deport fugitive slaves from the North, was completely defied by most Northern states.}}
        \cornell{What was the "Young America" movement?}{\begin{itemize}
            \item Emerged w/in Democratic Party: pushed for expansion of democracy beyond America to take focus off of slavery
            \item Unable to truly resist sectional influence
            \begin{itemize}
                \item Pierce created plan to take Cuba by force $\to$ North angered by potential of additional slave state 
                \item South equally resistant to expansion of democracy and rule to free states (Hawaii bill overturned; Canada never annexed)
            \end{itemize}
        \end{itemize}
        \textbf{The "Young America" movement pushed to expand American democracy throughout the world without the influence of slavery; ultimately, however, this was a futile goal and both the North and the South resisted the admission of new territories which counterracted their own beliefs.}}
        \cornell{How did the expansion of railroads to the West fuel the sectional crisis?}{\begin{itemize}
            \item American expansion moved beyond MO/IA/MN as farmers saw capability for farming $\to$ Old Northwesterners pushed govt. to open area w/ territorial govt, encroach on native land
            \item Further settlement tied issue of railroads with sectional conflict
            \begin{itemize}
                \item Transcontinental railroad necessary for communication, but eastern terminus to connect to eastern lines either Chicago (North) or Memphis/St. Louis/New Orleans (South)
                \item Jefferson Davis (sec. of state) saw Southern route would cross thru. MX $\to$ sent James Gadsden to purchase modern AZ/NM to allow for railroad development 
            \end{itemize}
        \end{itemize}
        \textbf{As Americans began to migrate to the recently-discovered-to-be prosperous Far West, calls for a formal railroad emerged, but the question remained as to whether the eastern terminus should have been in a Northern or Southern city. Davis' Gadsden Purchase, acquiring a strip of land in modern Arizona and New Mexico, broke one of the barriers toward a Southern terminus and furthered divisions.}}
        \cornell{What was the Kansas-Nebraska Controversy?}{\begin{itemize}
            \item Stephen A. Douglas sought railroad to terminate in North (in his own city)
            \item Douglas saw strong argument against: numerous native tribes in the way of Northern route $\to$ pushed for organization of new territory, Nebraska  
            \item Initial bill prevented Nebraska from being guaranteed free state, instead opening to vote of ppl.;
            \item South pushed him to break Missouri Compromise by dividing into northern free Nebraska territory and Southern slave Kansas territory; Pierce supported 
            \item Kansas-Nebraska Act $\to$ Whigs destroyed, Northern Dems. divided (some angered by breach of MO compromise); new party formed based on sectionalism: Republicans 
        \end{itemize}
        \textbf{After Stephen A. Douglas pushed for the Nebraska Territory in the West to allow for a safe Northern railroad routes, the South convinced him to breach the MO Compromise by breaking the territory (despite being north of the slave-free barrier) into free Nebraska and slave Kansas. This division decimated the Whigs, divided Democrats in the North, and formed the Republican party.}}
        \cornell{How did Kansas become a region of great tension?}{\begin{itemize}
            \item Southerners and Northerners both moved to Kansas; by territorial election, only 1,500 legal voters but Missourians travelled in large \#s to vote on pro-slavery legislature
            \item Free-staters created constitutional convention with own personal governor, legislature, petitioned for statehood, but Pierce denounced as traitors, supporting slave legislature w/ federal marshals arresting free-staters 
            \item John Brown, Kansas abolitionist, moved to Kansas and started a time of great strife by murdering 5 pro-slavers
            \item Charles Sumner of MA gave speech against Andrew P. Butler of SC, defender of slavery, making sexual references about wife 
            \begin{itemize}
                \item Nephew, Preston Brooks, angrily approached Sumner's desk and beat with cane until unconscious $\to$ Sumner became martyr in North, Brooks hero in South 
            \end{itemize}
        \end{itemize}
        \textbf{Kansas became known as "Bleeding Kansas" as tensions rose between free-staters and slave-staters inhabiting it, culminating in physical battles, including John Brown's murder of 5 pro-slavers and Preston Brooks' beating of Charles Sumner after he had spoken out against his uncle for his supporting of slavery.}}
        \cornell{How did the "free-soil" ideology stimulate sectionalist tensions?}{\begin{itemize}
            \item Tensions reflected solidifcation of disparate ideas in both regions 
            \item Northerners believed in "free soil": some argued for moral reasons but most for dangers to whites by threatening democratic threads of society
            \begin{itemize}
                \item Northeners saw South as opposite to democracy: aristocracy-based society stagnating due to rejection of individualism; sought to destroy democracy/capitalism by spreading aristocracy in "slave power conspiracy"
            \end{itemize}
            \item Ideology central to Republican Party, strengthening committment to Union
        \end{itemize}
        \textbf{The "free-soil" ideology focused not on the implication of slavery for blacks, but instead that for whites. Several Northerners believed that slavery threatened democratic and capitalist ideals in favor of the deeply rooted aristocracy of the South. Such Northern beliefs were central to the Republican Party.}}
        \cornell{How did some argue for the continuation of slavery?}{\begin{itemize}
            \item Southerners continued slavery out of fear of blacks from revolts, importance of cotton economy
            \item Intellectuals created formal defense of slavery, starting w/ Thomas R. Dew and culminating 20 yrs. later (1852) in \textit{The Pro-Slavery Argument} - anthology detailing arguments
            \item Calhoun argued slavery should not be seen as necessary sin but instead as positive good
            \begin{itemize}
                \item Slaves enjoyed better conditions then northern workers
                \item Southern economy critical to U.S.
            \end{itemize}
            \item Several argued crucial to Southern way of life
            \begin{itemize}
                \item Saw instability, horrors of Northern factory system with "unruly" immigrants 
                \item Enjoyed peaceful Southern society supposedly w/o feuds, instead with superior culture and way of life
            \end{itemize}
            \item Biological inferiority of Afr. Americans often cited: seen as unfit to live independently 
            \item Religious args. based on Protestant clergy
        \end{itemize}
        \textbf{Those who argued for slavery cited the positive working conditions for slaves, the economic importance, slavery's deeply rooted position in Southern society, black biological inferiority, and biblical justifications.}}
        \cornell{What was the result of the election of 1856?}{\begin{itemize}
            \item Democrats wanted non-sectionalist candidate (unlike Pierce), nominating James Buchanan of PA, minister to England
            \item Republicans shunned KS-NE Act, slavery while taking on Whig goal for internal improvement; nominated Frémont
            \item Buchanan won in heated campaign against Frémont and Fillmore (under shrinking Know-Nothing Party)
            \begin{itemize}
                \item Frémont had received essentially no Southern votes while dominating in the North 
            \end{itemize}
            \item Buchanan, oldest president except for Henry Harrison, timid and indecisive 
            \begin{itemize}
                \item Faced w/ major financial panic then long-lastic depression which led to doubt in Democrats, strengthening Repubs. 
            \end{itemize}
        \end{itemize}
        \textbf{Democrat Buchanan won against Republican Frémont and Know-Nothing Fillmore; one of the oldest presidents, his presidency was marked by indecision in times of great adversity.}}
        \cornell{What was the \textit{Dred Scott} decision?}{\begin{itemize}
            \item 1857: \textit{Dred Scott v. Sandford} w/ Dred Scott, MO slave who migrated w/ owner $\to$ IL; owner died $\to$ sued widow for freedom because now in free state 
            \item Court declared Scott free, but widow's brother, Sanford (mispelled) appealed ruling to Supreme Court, arguing that property could not sue
            \item Supreme Court issues multiple rulings: in summary, Chief Justice Taney argued that Scott was not a citizen by constitutional law $\to$ couldn't sue $\to$ Congress could not make ruling 
            \begin{itemize}
                \item States still had power to decide on slavery, but federal renouncement of power shocking (good for South)
                \item Angry Republicans threatened reversal after coming into power
            \end{itemize}
        \end{itemize}
        \textbf{In the case \textit{Dred Scott v. Sandford}, a Missouri slave sued his deceased owner's widow for freedom due to his moving to a free state; although the courts declared him free, the widow's brother, Sandford, brought the case to the Supreme Court, where Taney argued that Scott was not a citizen and thus could not legally sue.}}
        \cornell{How did Buchanan handle Kansas?}{\begin{itemize}
            \item Buchanan hoped to admit KS as slave state to resolve controversy $\to$ pro-slavery legislature formed constitutional convention
            \begin{itemize}
                \item Free-staters refused to attend due to perceived discrimination in district lines $\to$ pro-slavers won majority in convention and framed pro-slavery "Lecompton Constitution"
                \item Election for new legislature came around $\to$ free-staters won majority $\to$ Lecompton rejected
            \end{itemize}
            \item Buchanan sought quick fix by accepting under Lecompton anyway, but it failed in the House 
            \item Compromise formed w/ Lecompton sent to voters of Kansas: if passed, admitted as slave state; else process postponed
            \begin{itemize}
                \item Rejected $\to$ KS only became \underline{free state} in 1861
            \end{itemize}
        \end{itemize}
        \textbf{Buchanan hoped to admit Kansas as a slave state to resolve the conflict; when the pro-slavery legislature formed a constitution for statehood, it was rejected by the newly elected antislavery legislature, postponing the process of statehood until Kansas was finally admitted as a free state in 1861.}}
        \cornell{How did Lincoln rise to political power?}{\begin{itemize}
            \item In IL, 1858 Senate election pitted Stephen A. Douglas (most known northern Democrat) against unknown Republican Abraham Lincoln
            \begin{itemize}
                \item Experienced lawyer w/ long history in state politics
                \item Relied on eloquent Lincoln-Douglas debates to grow national image
            \end{itemize}
            \item Fundamental difference betw. Douglas/Lincoln was slavery
            \begin{itemize}
                \item Douglas no definite position: believed should be left to states; did make clear blacks did not deserve governmental role nor citizenship
                \item Lincoln opposed slavery because oppressing African Americans opened the door to oppression of other groups (like immigrants, poorer white laborers) $\to$ free labor important 
                \begin{itemize}
                    \item Disagreed w/ moral aspect but not abolitionist due to no logical alternative: felt Afr. Americans remained inferior to whites 
                    \item Sought to prevent further spread
                \end{itemize}
            \end{itemize}
            \item Douglas won election, but Lincoln had gained national image
        \end{itemize}
        \textbf{Lincoln developed his national image by debating eloquently with Stephen Douglas for the Senate position, making clear his opposition to slavery on moral grounds and goal to prevent its further spread.}}
        \cornell{How did John Brown's raid scare white southerners?}{\begin{itemize}
            \item Abolitionist John Brown (who killed KS pro-slavers) received financial backing to create slave insurrection in VA
            \item Brown failed to create, instead met by angry citizens, local militia, local troops $\to$ outnumbered, tried and hanged
            \item South erroneously believed Brown supported by Republican Party $\to$ felt betrayed by North
        \end{itemize}
        \textbf{John Brown's attempt at starting a slave insurrection in Virginia, which the South believed had been backed by the Republican Party, drove white southerners to feel unsafe remaining in the Union.}}
        \cornell{What was the result of the election of 1860?}{\begin{itemize}
            \item Democrats divided into southerners seeking slavery and westerners seeking popular sovereignty
            \begin{itemize}
                \item Convention in April 1860 in SC agreed on popular sovereignty $\to$ eight southern delegates walked out, remaining unable to agree $\to$ smaller group met again in Baltimore, nominating Douglas
                \item Southern Democrats nominated J.C. Breckinridge 
                \item Ex-Whigs formed Constitutional Union Party, nominating John Bell, endorsing Union
            \end{itemize}
            \item Republican leaders sought to expand to entirety of North by focusing on South's harming of North's economy
            \begin{itemize}
                \item Supported internal developments of tariffs, Pacific railroad, individual right of states to decide slavery
                \item Nominated Lincoln due to growing reputation of eloquence; position on slavery moderate yet firm; less well-known $\to$ no bad reputation
            \end{itemize}
            \item Lincoln won election w/ majority of elec. votes, not winning Congress or majority of popular vote 
            \begin{itemize}
                \item Disunion began soon after $\to$ long war
            \end{itemize}
        \end{itemize}
        \textbf{The divided Democrats nominated three candidates independently, with one based around western and Northern beliefs (Douglas), one based around Southern beliefs (Breckinridge), and one based around Whig beliefs (John Bell). Republicans, on the other hand, united behind Abraham Lincoln for his firm stance on slavery and won the election; disunion, however, soon began.}}
    \end{document}