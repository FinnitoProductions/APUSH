\documentclass[a4paper]{article}
    \usepackage[T1]{fontenc}
    \usepackage{tcolorbox}
    \usepackage{amsmath}
    \tcbuselibrary{skins}
    
    \title{
    \vspace{-3em}
    \begin{tcolorbox}
    \Huge\sffamily \begin{center} AP US History  \\
    \LARGE Chapter 15 - Reconstruction and the New South \\
    \Large Finn Frankis \end{center} 
    \end{tcolorbox}
    \vspace{-3em}
    }
    \date{}
    \author{}
    
    \usepackage{background}
    \SetBgScale{1}
    \SetBgAngle{0}
    \SetBgColor{red}
    \SetBgContents{\rule[0em]{4pt}{\textheight}}
    \SetBgHshift{-2.3cm}
    \SetBgVshift{0cm}
    \usepackage[margin=2cm]{geometry} 
    
    \makeatletter
    \def\cornell{\@ifnextchar[{\@with}{\@without}}
    \def\@with[#1]#2#3{
    \begin{tcolorbox}[enhanced,colback=gray,colframe=black,fonttitle=\large\bfseries\sffamily,sidebyside=true, nobeforeafter,before=\vfil,after=\vfil,colupper=blue,sidebyside align=top, lefthand width=.3\textwidth,
    opacityframe=0,opacityback=.3,opacitybacktitle=1, opacitytext=1,
    segmentation style={black!55,solid,opacity=0,line width=3pt},
    title=#1
    ]
    \begin{tcolorbox}[colback=red!05,colframe=red!25,sidebyside align=top,
    width=\textwidth,nobeforeafter]#2\end{tcolorbox}%
    \tcblower
    \sffamily
    \begin{tcolorbox}[colback=blue!05,colframe=blue!10,width=\textwidth,nobeforeafter]
    #3
    \end{tcolorbox}
    \end{tcolorbox}
    }
    \def\@without#1#2{
    \begin{tcolorbox}[enhanced,colback=white!15,colframe=white,fonttitle=\bfseries,sidebyside=true, nobeforeafter,before=\vfil,after=\vfil,colupper=blue,sidebyside align=top, lefthand width=.3\textwidth,
    opacityframe=0,opacityback=0,opacitybacktitle=0, opacitytext=1,
    segmentation style={black!55,solid,opacity=0,line width=3pt}
    ]
    
    \begin{tcolorbox}[colback=red!05,colframe=red!25,sidebyside align=top,
    width=\textwidth,nobeforeafter]#1\end{tcolorbox}%
    \tcblower
    \sffamily
    \begin{tcolorbox}[colback=blue!05,colframe=blue!10,width=\textwidth,nobeforeafter]
    #2
    \end{tcolorbox}
    \end{tcolorbox}
    }
    \makeatother

    \parindent=0pt
    
    \begin{document}
    \maketitle
    \SetBgContents{\rule[0em]{4pt}{\textheight}}
    \cornell[Key Concepts]{What are this chapter's key concepts?}{\begin{itemize}
        \item \textbf{5.3.II.A} - 13th Amendment abolished slavery, 14th/15th provided citizenship, legal rights, vote
        \item \textbf{5.3.II.C} - Short-term success for Republicans hoping to change power balance betw. Congress/presidency; Reconstruction ultimately limited pol. opportunities for slave due to Southern resistance
        \item \textbf{5.3.II.D} - Southern plantation owners continued to own land; slaves generally unable to own large amounts due to difficult upkeep for those starting with limited wealth
        \item \textbf{5.3.II.E} - Segregation/violence/Supreme Court/local tactics gradually removed rights from Afr. Americans; 14th/15th Amendments held strong
        \item \textbf{6.1.II.D} - Some southern industrialization based on "New South" ideals; still primarily based around sharecropping/tenant farming
        \item \textbf{6.3.II.C} - \textit{Plessy v. Ferguson} $\to$ long-term segregation but Afr. Americans began to stand up for rights
    \end{itemize}}
    \cornell[The Problems of Peacemaking]{What immediate challenges were faced by post-Civil War leaders?}{\textbf{After the Civil War devastated the South and left both blacks and whites in mourning, several Southerners began to reconsider their notion of freedom: blacks saw it as a life without slavery but also as equal rights to whites; whites saw it as white supremacy and independence from the North. When plans for Reconstruction emerged, Republicans divided into Conservatives, seeking pardoning for Southerners, and Radicals, seeking hard punishments. Lincoln sided with the Conservatives, creating a moderate plan requiring little for readmission. After his assassination at the hands of Booth, Johnson succeeded him and was known for showing little sympathy to the freedmen. His reconstruction plan was strongly based on the Wade-Davis Bill passed by radicals months earlier, placing far more stringent requirements on states before they could return to the Union.}}
    \cornell{What was the initial aftermath of the Civil War?}{\begin{itemize}
        \item Civil War $\to$ unparalleled devastation for South 
        \begin{itemize}
            \item Land, bridges, railroads destroyed; plantations and fields gutted 
        \end{itemize}
        \item White southerners had lost several close to them as well as way of life
        \begin{itemize}
            \item Most white southerners, without slaves or Confederate bonds/currency or adult males $\to$ rebuilt lives independently 
            \item 258k killed soldiers $\to$ 20\% of adult white male population killed; with nearly everyone having lost men close to them, mourning emerged for multiple years
            \item "Lost Cause" emerged where whites mourned past South, revering Jackson, Lee, later Davis
        \end{itemize}
        \item Conditions worse for blacks: many had served in war and most had left plantations $\to$ no immediate home
        \begin{itemize}
            \item Roamed between cities, camping in countryside; some hoped for help from Union occupation forces
        \end{itemize}
    \end{itemize}
    \textbf{The aftermath led to both infrastructural and territorial destruction but also devastated the lives of whites and blacks alike. White southerners had lost nearly 20\% of adult males in society, and their former way of life was in shambles. Blacks, too, were often left without homes and roamed between cities.}}
    \cornell{How did the African American notion of freedom conflict with the white one?}{\begin{itemize}
        \item African Americans saw freedom as no slavery but also rights/protections equal to whites
        \begin{itemize}
            \item Some sought to achieve by redistributing economy (notably land) because all had worked to build it up
            \item Others sought legal equality: with equal opportunity, could build themselves up to parallel whites
            \item All unified in no white control $\to$ formed Afr. American communities with new churches, aid societies, school
        \end{itemize}
        \item White southerners saw freedom as no Northern/federal intervention and white supremacy 
        \begin{itemize}
            \item Many attempted to restore this notion of freedom post-war by restoring society to antebellum state 
            \item Several kept black workers legally tied to plantations
        \end{itemize}
        \item Union troops remained in South to protect freed slaves $\to$ Freedmen's Bureau under Congress
        \begin{itemize}
            \item Bureau provided food to former slaves, created schools led by Freedmen's Aid Societies
            \item Supported poor whites devastated by war
            \item Never long-term solution
        \end{itemize}
    \end{itemize}
    \textbf{African Americans saw freedom as an unenslaved state with rights equal to whites through legal equality as well as their own independent communities; whites, contrastingly, saw freedom both as independence from the North but also as continued white supremacy. Black freedom was preserved and enforced by Union troops under the Freedmen's Bureau, who also established schools and provided food to freed slaves.}}
    \cornell{What were the critical questions of Reconstruction?}{\begin{itemize}
        \item Reconstruction based around partisan politics: Republican majorities of 1860/1864 due to disunited Democratic Party/no South
        \begin{itemize}
            \item Both parties aware that restored South $\to$ Democrats likely to regain power $\to$ Republican programs issued during majority at risk
            \item Several Northerners felt South should be punished for rebellion, and its society be modeled on urban North's
        \end{itemize}
        \item Republican Party divided about how to approach readmission of South
        \begin{itemize}
            \item All agreed that slavery must be formally abolished in all state codes; conservatives felt little more necessary
            \item Radicals (like \textbf{Thaddeus Stevens}, \textbf{Charles Sumner}) demanded punishment of military leaders, several Southern whites deprived of right to vote, wealth of rich Southerners assisting Confederacy seized 
            \begin{itemize}
                \item Some sought suffrage for blacks but many did not want to conflict with Northern laws 
            \end{itemize}
            \item Moderates sought no punishment but greater concessions for Afr. American rights
        \end{itemize}
    \end{itemize}
    \textbf{Reconstruction was based around party politics and the Republican fear that regained Southern power would mean a Denocrat majority. More radical Republicans thus argued for removal of the right to vote and confiscation of wealth for several Southern rights while conservatives pushed only for the abolition of slavery.}}
    \cornell{What formal plans emerged for Reconstruction?}{\begin{itemize}
        \item Lincoln supported Moderates/Conservatives on issue: felt few demands $\to$ more Southern Republicans; frelt freedman fate could be postponed
        \begin{itemize}
            \item Reconstruction plan pardoned white Southerners (apart from Confederate leaders) who were loyal to govt., agreed with abolishment of slavery; hoped suffrage would be given to educated, property-owning blacks from Union army
            \item After 10\% of voters took Lincoln's oath, allowed to form state govt.; LA, AR, TN formed govt. in 1864
        \end{itemize}
        \item Radicals felt too mild, initially hesitant for solution; passed \textbf{Wade-Davis Bill} through Congress
        \begin{itemize}
            \item Allowed president to select provisional governor; after \textit{majority} pledged oath to U.S., allowed to form constitutional convention with requirements of abolishment, repudiation of debts
            \item Delegates only those who swore they had never borne arms against U.S.
            \item Lincoln quickly vetoed but realized importance of conceding to radicals
        \end{itemize}
    \end{itemize}
    \textbf{Lincoln produced a moderate plan for reconstruction pardoning all non-Confederate leader-white Southerners and allowed a state government after 10\% of voters promised they would abolish slavery, would remain loyal to government. Radicals produced the Wade-Davis Bill, which gave the president the choice of governor; a constitutional convention could be formed only after 50\% pledged oath and could only be populated by those who promised to have never attacked U.S. armies. Lincoln vetoed it.}}
    \cornell{What were the effects of Lincoln's death?}{
        On April 14th, 1865, Lincoln was assassinated at a theatre by Confederate actor John Wilkes Booth.
        \begin{itemize}
            \item Lincoln hailed as martyr throughout North
            \item Several accused Booth of being part of Southern conspiracy; in fact linked with group seeking to assassinate Lincoln, secretary of state Seward, VP Johnson
            \item Booth escaped to VA countryside but pursued and killed by Union forces; four of associates hanged
            \item Republicans interpreted as continued hostility from Southern leaders
        \end{itemize}
        \textbf{Booth, part of a group seeking to kill Lincoln, Seward, and Johnson, assassinated Lincoln in April 1865, but was soon killed in the Virginia countryside. Republicans saw Lincoln's assassination as a sign of continued Southern tensions.}}
    \cornell{How did Andrew Johnson implement restoration plans of his own?}{\begin{itemize}
        \item Johnson, Democrat until joining forces with Lincoln, hostile toward freedmen and felt reconstruction should be driven by South
        \item Developed plan for Reconstruction known as "Restoration"
        \begin{itemize}
            \item Southerners who took oath immediately pardoned; high-ranking officials/wealthy plantation owners forced to appeal directly to him for pardon 
            \item Generally based on Wade-Davis Bill w/ provisional governor to select unspecified \# of voters to elect delegates (w/ majority required)
            \item State would need to abolish slavery, revoke secession, ratify 13th Amendment, reject Confederate/state war debts 
            \item No assistance to former slaves w/ many often remaining in similar conditions
        \end{itemize}
        \item By end of 1865, all seceded states formed new govt. based on Lincoln/Johnson plans; radicals would not accept
        \begin{itemize}
            \item Radicals felt reluctance to abolish slavery + election of Confederate leaders (like Confed. VP to GA Senate) reflected false loyalty
        \end{itemize}
    \end{itemize}
    \textbf{Johnson, a Democrat at heart, was hesitant about reconstruction and developed a plan very similar to the Wade-Davis Bill. Although all states not already following Lincoln's plan formed state governments, radicals refused to accept their admission to Congress as Northern opinion had slowly hardened due to the South reluctance to abolish slavery.}}
    \cornell[Radical Reconstruction]{How did Reconstruction begin to align with Radical perspectives?}{\textbf{After the Radicals took over Reconstruction, they created two constitutional amendments (14th and 15th) making blacks citizens and granting them suffrage and forced several states to accept them for chance at readmission through military districts. To preserve their control, Radicals limited presidential and judicial power and attempted to impeach President Johnson.}}
    \cornell{How did Johnson lose control of Reconstruction?}{\textbf{Congress reconvened in December 1865 and formed the Joint Committee on Reconstruction to override Johnson's policies with their own more radical ones.}}
    \cornell{What were the Black Codes?}{\begin{itemize}
        \item Southern states began to enact \textbf{Black Codes} to allow whites to control former slaves
        \begin{itemize}
            \item Unemployed Afr. Americans could be seized, fined for loitering
            \item Private employers would then hire them to pay off fee (no wage)
            \item Some banned from owning land, required that they become plantation workers/domestic servants
        \end{itemize}
        \item Congress first extended Freedmen's Bureau to limit abuse; 1866: \textbf{first Civil Rights Act}: African Americans were citizens $\to$ federal govt. could intervene 
        \begin{itemize}
            \item Johnson vetoed, but he was overriden by Congress
        \end{itemize}
    \end{itemize}
    \textbf{The "Black Codes" allowed Southern whites to exercise control once again over African Americans, often justifying unfair fines (like for mere vagrancy) and then requiring them to pay them off by working for companies without wage; others prevented them from owning land or working specialized jobs. Congress retaliated with the first Civil Rights Act, naming African Americans citizens.}}
    \cornell{What was the 14th Amendment?}{\begin{itemize}
        \item 14th Amendment passed by Joint Committee on Reconstruction in 1866
        \begin{itemize}
            \item Gave citizenship to all born in U.S. (or naturalized) with no exceptions
            \item Penalized states limiting voting rights of \textit{any} adult male
            \item Former members of Congress/officials who were part of Confederacy could only hold state/fed. office if pardoned by 2/3 of Congress
        \end{itemize}
        \item Radicals readmitted any state ratifying amendment: only TN did, not even DE/KY $\to$ lacking 3/4 approval
        \item Radicals more confident w/ race riots targeting Afr. Americans in Southern cities
        \begin{itemize}
            \item Johnson supported Conservatives in 1866 midterms but weak speeches $\to$ mostly Radical Republicans elected (few Southerners, Democrats)
        \end{itemize}
    \end{itemize}
    \textbf{The 14th Amendment made all people born in America citizens, gave all adult men the right to vote, and limited the potential power held by former Confederate officials. Initially, however, Tennessee was the only former Confederate state to ratify it; however, Radical Republicans gained power in the 1866 midterms.}}
    \cornell{How did Radicals continue with Reconstruction post-midterms?}{\begin{itemize}
        \item Passed 3 more bills $\to$ complete plan, overrode Johnson's vetoes
        \item TN readmitted but Lincoln-Johnson govts. of 10 other states rejected, becoming 5 military districts
        \begin{itemize}
            \item Commander led each district with permission to register qualified voters: adult black males, white males not in Confederate rebellion
            \item Registered voters elected delegates for convention with constitution to be ratified by voters; state government elected after Constitution
            \item Once Congress approved constitution, enough states ratified 14th Amendment to bring to 3/4, states would be readmitted
        \end{itemize}
        \item 1868: AR, NC, SC, LA, AL, GA, FL fulfilled w/ 14th Amendment part of Constitution 
        \begin{itemize}
            \item VA/TX in 1869 and MS in 1870 due to resistance of white conservatives
            \item Congress added new requirement: ratification of 15th Amendment, preventing suffrage discrimination based on race
        \end{itemize}
        \item Radicals limited presidential and judicial power due to risk of interfering w/ plans
        \begin{itemize}
            \item \textbf{Tenure of Office Act} prevented removal of cabinet w/o Senate approval due to War Sec. Edwin M. Stanton's work w/ Radicals
            \item \textbf{Command of the Army Act} prevented president from making military orders unless through commanding general (Grant) who was selected by Senate
            \item \textit{Ex parte Milligan} banned military tribunals in places w/ civil courts $\to$ military government in South at risk $\to$ \textit{proposed} (but never passed) bill requiring 2/3 of justices to agree to overturn Congressional law $\to$ no justices took on Reconstruction cases
        \end{itemize}
    \end{itemize}
    \textbf{The Radicals readmitted Tennessee but turned the remaining 10 states into military districts with a forced path to statehood still requiring the 14th Amendment. By 1870, all states had returned, also required to ratify the 15th Amendment extending suffrage to all races. The Radicals strengthened their position by limiting presidential and judicial power to overturn their rule.}}
    \cornell{How did the Radicals attempt to impeach Johnson?}{\begin{itemize}
        \item Johnson no longer genuine obstacle but Radicals still sought to impeach
        \begin{itemize}
            \item After Johnson dismissed War Sec. Stanton w/o Congressional approval, saw room for impeachment
            \item Easily passed through House
        \end{itemize}
        \item In Senate, Radicals pressured rest of Republicans but some Moderates sided w/ Democrats $\to$ one away from 2/3 majority for impeachment $\to$ dropped efforts
    \end{itemize}
    \textbf{After Johnson violated the Tenure of Office Act, Radicals attempted to impeach him but they were short of one vote in the Senate due to some Moderates' siding with the Democrats.}}
    \cornell[The South in Reconstruction]{What major Reconstruction reforms were made to the South?}{\textbf{Reconstruction entailed creating new Republican-dominated political structures unifying Northern white and Southern black and white Republicans under one political force often seen as corrupt due to heavy spending. Education became far more widespread, particularly for freedmen; although no attempts at forced land redistribution succeeded, inevitable changes occurred with blacks receiving more land over time but also several becoming trapped in cycles of debt as a result of the crop-lien system. However, African American familial structure, too, developed in several ways, often nearing white family structures with a greater focus on women performing stereotypically female jobs of cooking and cleaning; several still had to work due to their abject poverty.}}
    \cornell{What were the political implementations made during Reconstruction?}{\begin{itemize}
        \item $\frac{1}{4}$ of white males init. excluded from voting/holding office $\to$ black majority in SC/MS/LA/AL/FL
        \begin{itemize}
            \item Govt. soon removed restrictions $\to$ nearly all white males could vote
        \end{itemize}
        \item Republicans enjoyed majority support in govt.
        \begin{itemize}
            \item White Southern Republicans known as \textbf{scalawags}
            \begin{itemize}
                \item Generally due to lack of interest in Democrats or were former Whigs
                \item Some were poor farmers hoping to promote internal improvements, serving personal economic goals
            \end{itemize}
            \item Northern Republicans oft. led in South as \textbf{carpetbaggers}
            \begin{itemize}
                \item Well-educated/middle-class, but nickname gave impression of penniless explorers 
                \item Generally Union veterans seeing new opportunity in South
            \end{itemize}
            \item Most Southern Repubs.: black freedmen w/o experience in politics $\to$ sought institutions to express power
            \begin{itemize}
                \item Several formed "colored conventions" (as dubbed by Southern whites), others joined black churches to unify former slaves as political forces
                \item Critical to Southern pol.: delegates at constitutional conventions, held near-every office (incl. House and Senate)
                \item Power lamented by Southern whites but still much lesser than whites w/ no governors, never controlling state legislatures; \% of officeholders far lower than \% in overall pop.
            \end{itemize}
        \end{itemize}
        \item Several Reconstruction govts. seen as corrupt w/ heavy spending, illegal activities to expand budgets $\to$ raised debt; same in North w/ econ. expansion of govt. $\to$ corruption
        \begin{itemize}
            \item Costs much larger than antebellum times because govt. had neglected important services
        \end{itemize}
    \end{itemize}
    \textbf{Reconstruction governments initially saw several whites excluded and thus black majorities in several states; although whites increasingly received their former voting priviliges, blacks still became a force critical to the progression of Southern politics as Republicans. Republicans consistently held majorities, driven by "scalawag" white Southern Republicans and "carpetbagger" Northern Republicans. Under their governance, Reconstruction governments were frequently classified as corrupt due to their incrasingly extravagant spending.}}
    \cornell{How did Reconstruction leaders promote education?}{\begin{itemize}
        \item $\uparrow$ education of former slaves initially driven by North thru. Freedmen's Bureau, private orgs. 
        \begin{itemize}
            \item White Southerners opposed due to risk of "false" equality but lg. networks
        \end{itemize}
        \item Reconstruction govt. developed public schools w/ around $\approx 50\%$ of whites and $\approx 40\%$ of blacks attending school; advanced "academies" $\to$ black colleges like Fisk/Atlanta Universities
        \item Early desegregation efforts unsuccessful w/ Freedmen's Bureau and New Orleans offering schools to all but few whites attended; \textbf{Civil Rights Act of 1875} saw desegregation removed
    \end{itemize}
    \textbf{Education efforts were first driven by Northern organizations and later Reconstruction governments (expanding the public school and university system for whites and blacks), who both made several attempts at creating desegregated schools, but few whites would attend.}}
    \cornell{How did the Reconstruction implement land ownership policies?}{\begin{itemize}
        \item Freedmen's Bureau initially sought to redistribute Southern land but failed in long term
        \begin{itemize}
            \item In final years of war, supplied abandoned plantations in Sea Islands of SC/GA, parts of MS to blacks, providing 10k families w/ land $\to$ great hope
            \item After war, whites returned and demanded return of land $\to$ Johnson supported, returning most of land to white owners
        \end{itemize}
        \item Northern Republicans saw no right to redistribute land 
        \item Southern land distribution patterns changed nonetheless
        \begin{itemize}
            \item Smaller \% of whites owned land by end of Reconstruction ($80\% \to 67\%$) due to debt/taxes; others moved to rented land elsewhere
            \item Afr. Americans rose from $\approx 0\% \to \geq 20\%$; some thru. hard work but others thru. assistance from white institutions (like Freedman's Bank)
            \begin{itemize}
                \item Institutions allowed blacks to invest money but many made poor investments $\to$ Freedman's Bank collapsed in 1874 after depression
            \end{itemize}
        \end{itemize}
        \item Sharecropping w/ some blacks renting lands from white owners, paying rent/giving portion of crops; others worked for wages ($\approx 25\%$)
        \begin{itemize}
            \item Opposite to old gang-labor system w/ no direction of master, instead physical independence 
            \item Landlords benefitted in that no longer needed to care for workers
        \end{itemize}
    \end{itemize}
    \textbf{Although the Freedmen's Bureau attempted to redistribute land, they failed as Northern Republicans in Congress felt they had no right to confiscate land. Regardless, Southern lamd patterns changed, with fewer whites owning and far more African Americans (some through hard work, others through white institutions) owning land. Those African Americans not owning land generally worked as tenants on white lands, paying a fixed rent to work the land.}}
    \cornell{What was the crop-lien system?}{\begin{itemize}
        \item Numerically, Reconstruction seems great success for Afr. Americans w/ earning increasingly greater share of agri. while white share rapidly declined
        \begin{itemize}
            \item Part of agri. decline as a whole w/ $\downarrow$ cotton market
            \item Afr. Americans earned more per hour than equivalent in slavery but worked $\approx \frac{2}{3}$ of original amnt. (to same as white laborers)
            \item Black per capita income went from $\frac{1}{4}$ of white to $\frac{1}{2}$ but stagnated there 
        \end{itemize}
        \item Crop-lien system devastating to low-income farmers
        \begin{itemize}
            \item Unable to pay sufficient money to local stores required for food, clothing, seed, farm tools $\to$ forced to use credit but w/ $\uparrow$ interest $\to$ farmers gave up claim (lien) on crops $\to$ often cycle of debt
            \item Former slaves lost earned land due to high debt; farmers also dependent on cash crops due to greatest immediate revenue $\to$ land destroyed, never diversification
        \end{itemize}
    \end{itemize}
    \textbf{Reconstruction blacks began to earn a greater share of the agricultural economy but this was paired with an overall decline of Southern agriculture; furthermore, their income per capita remained around half of that of whites. The crop-lien was devastating, too, allowing stores to make claims on farmers' crops if they were unable to pay their high-interest debt back in sufficient time, often leading to a cycle of poverty and stimulating a one-dimensional economy reliant only on the most profitable crops.}}
    \cornell{How did African American family structures shift in Reconstruction?}{\begin{itemize}
        \item Blacks attempted to rebuild family structures after destruction from slavery
        \begin{itemize}
            \item Travelled throughout South to find lost relatives; black newspapers published info. abt. lost relatives 
            \item Rapid marriage due to previous ban; relocation to private cabins throughout countryside
        \end{itemize}
        \item Black families often began to mirror white families in gender relations
        \begin{itemize}
            \item Women/children saw field-work as part of slavery $\to$ women cooked/cleaned/gardened/raised children only
            \item Some black men refused wives' working as servants for whites: wanted exclusive servitude for himself
        \end{itemize}
        \item Impoverished families often required black women to work in former activities like domestic servants, laundry, field work; $\approx$ half of black women worked for wages, most wage-earners married
    \end{itemize}
    \textbf{African American family structures were emphasized with a strong effort to rebuild them by finding lost relatives, marrying in large numbers, and moving to private cabins. However, black families increasingly saw women restricting themselves to stereotypically female roles like cooking and cleaning, generally encouraged by men. Regardless, several more impoverished families required women to work in their former activities simply to earn a wage.}}
    \cornell[The Grant Administration]{How did Ulysses S. Grant handle his presidency?}{\textbf{Ulysses S. Grant, a Republican, ruled clumsily, relying on party leaders for assistance. Several economic scandals emerged, including several briberies of high-up officials like his vice president and secretary of war. By issuing the Specie Resumption act, Grant's administration sought to crush the greenback movement fo paper money and prevent inflation by backing all wealth with gold. Finally, under Lincoln, Johnson, and Grant, diplomatic affairs including the purchase of Alaska and the forging of neutrality with England unfolded due to their tactful secretaries of state.}}
    \cornell{What marked Grant's first years as president?}{\begin{itemize}
        \item Grant could have joined either party but chose Republicans due to greater Northern popularity; against Horatio Seymour, won popular vote (thanks to black Republicans) but tight
        \item Policies reflected lack of experience: only Hamilton Fish, sec. of state, was truly remarkable
        \begin{itemize}
            \item Relied on party leaders and spoils system $\to$ many angered
            \item Alienated those who began to oppose Radical Reconstruction as well as corruption
        \end{itemize}
        \item Liberal Republican group emerged opposing "Grantism," joining w/ Democrats to bring Horace Greeley, \textit{NY Tribune} editor, but Grant won
    \end{itemize}
    \textbf{Grant, choosing to side with the Republicans, won the popular vote (but barely) thanks to black Republicans in the South. He had little political experience and his early years reflected this: he relied on party leaders, the spoils system, and alienating his opponents. Although Republicans lost several members to Democrats opposing "Grantism," he won the 1872 election by a significant margin.}}
    \cornell{What were the main scandals during Grant's time in office?}{\begin{itemize}
        \item 1872 campaign saw scandals w/ \textbf{Crédit Mobilier}, company building Union Pacific railroad 
        \begin{itemize}
            \item Mobilier had manipulated stock in Union Pacific to rob both them and fed. govt. of several millions
            \item Prevented investigations by giving stock to Congress; Grant's VP Schuyler Colfax exposed as ownning stock
        \end{itemize}
        \item Benjamin Bristow, Treasury sec., found his officials leading "whiskey ring" to cheat govt. of taxes
        \item William Belknap, war sec., had been bribed to keep Native American trader in office
    \end{itemize}
    \textbf{"Grantism" was seen as an increasingly corrupt philosophy after the emergence of several scandals, including Grant's VP taking stock from a fraudulent company to prevent investigation, Treasury officials working with distillers to stop paying taxes, and the war secretary receiving a bribes to keep a Native American trader in office.}}
    \cornell{How did Grant's administration address the issue of greenbacks?}{\begin{itemize}
        \item Investment banking firm Jay Cooke and Company collapsed $\to$ eventual four-year depression; worst panic yet
        \item Debtors pushed govt. to redeem war bonds w/ paper money but Grant sought "sound" currency based around gold reserves
        \begin{itemize}
            \item Treasury responded to panic by further increasing circulation
            \item Congress retaliated w/ 1875 Specie Resumption act to crush \textbf{greenback} movement $\to$ starting in 1879, greenbacks redeemed by govt. w/ certificates based on \textit{unfluctuating} gold/silver
        \end{itemize}
        \item "Greenbackers" (inflationists) created political party: National Greenback Party, never gaining widespread support but constantly pushing issue of money
    \end{itemize}
    \textbf{Debtors increasingly pushed the government to issue more paper money to redeem their war bonds; though Grant opposed them, the Treasury began to issue in larger amounts. Congress retaliated with the Specie Resumption act to base all certificates of wealth on sound reserves of gold and silver. The angered "greenbackers" formed their own party but never gained widespread support.}}
    \cornell{How did Johnson and Grant handle diplomatic affairs?}{\begin{itemize}
        \item Great diplomacy under Republican leaders all due to sec. of states Seward (Lincoln/Johnson) and Fish (Grant)
        \item Seward purchased AK from Russia for \$7.2m (seen as great mistake); annexed Midway islands west of HI
        \item Fish resolved issues of British violation of neutrality during Civil War after sale of ships to Confederacy
        \begin{itemize}
            \item Americans demanded payment but Fish agreed on \textbf{Treaty of Washington} in which Britain formally expressed regret
        \end{itemize}
    \end{itemize}
    \textbf{Under Lincoln and Johnson, William H. Seward purchased Alaska and annexed the Midway Islands; under Grant, Hamilton Fish sorted out tense relations with Britain concerning neutrality violations by selling ships to the Confederacy in the Treaty of Washington.}}
    \end{document}