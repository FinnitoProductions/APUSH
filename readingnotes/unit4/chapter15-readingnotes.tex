\documentclass[a4paper]{article}
    \usepackage[T1]{fontenc}
    \usepackage{tcolorbox}
    \usepackage{amsmath}
    \tcbuselibrary{skins}
    
    \title{
    \vspace{-3em}
    \begin{tcolorbox}
    \Huge\sffamily \begin{center} AP US History  \\
    \LARGE Chapter 15 - Reconstruction and the New South \\
    \Large Finn Frankis \end{center} 
    \end{tcolorbox}
    \vspace{-3em}
    }
    \date{}
    \author{}
    
    \usepackage{background}
    \SetBgScale{1}
    \SetBgAngle{0}
    \SetBgColor{red}
    \SetBgContents{\rule[0em]{4pt}{\textheight}}
    \SetBgHshift{-2.3cm}
    \SetBgVshift{0cm}
    \usepackage[margin=2cm]{geometry} 
    
    \makeatletter
    \def\cornell{\@ifnextchar[{\@with}{\@without}}
    \def\@with[#1]#2#3{
    \begin{tcolorbox}[enhanced,colback=gray,colframe=black,fonttitle=\large\bfseries\sffamily,sidebyside=true, nobeforeafter,before=\vfil,after=\vfil,colupper=blue,sidebyside align=top, lefthand width=.3\textwidth,
    opacityframe=0,opacityback=.3,opacitybacktitle=1, opacitytext=1,
    segmentation style={black!55,solid,opacity=0,line width=3pt},
    title=#1
    ]
    \begin{tcolorbox}[colback=red!05,colframe=red!25,sidebyside align=top,
    width=\textwidth,nobeforeafter]#2\end{tcolorbox}%
    \tcblower
    \sffamily
    \begin{tcolorbox}[colback=blue!05,colframe=blue!10,width=\textwidth,nobeforeafter]
    #3
    \end{tcolorbox}
    \end{tcolorbox}
    }
    \def\@without#1#2{
    \begin{tcolorbox}[enhanced,colback=white!15,colframe=white,fonttitle=\bfseries,sidebyside=true, nobeforeafter,before=\vfil,after=\vfil,colupper=blue,sidebyside align=top, lefthand width=.3\textwidth,
    opacityframe=0,opacityback=0,opacitybacktitle=0, opacitytext=1,
    segmentation style={black!55,solid,opacity=0,line width=3pt}
    ]
    
    \begin{tcolorbox}[colback=red!05,colframe=red!25,sidebyside align=top,
    width=\textwidth,nobeforeafter]#1\end{tcolorbox}%
    \tcblower
    \sffamily
    \begin{tcolorbox}[colback=blue!05,colframe=blue!10,width=\textwidth,nobeforeafter]
    #2
    \end{tcolorbox}
    \end{tcolorbox}
    }
    \makeatother

    \parindent=0pt
    
    \begin{document}
    \maketitle
    \SetBgContents{\rule[0em]{4pt}{\textheight}}
    \cornell[Key Concepts]{What are this chapter's key concepts?}{\begin{itemize}
        \item \textbf{5.3.II.A} - 13th Amendment abolished slavery, 14th/15th provided citizenship, legal rights, vote
        \item \textbf{5.3.II.C} - Short-term success for Republicans hoping to change power balance betw. Congress/presidency; Reconstruction ultimately limited pol. opportunities for slave due to Southern resistance
        \item \textbf{5.3.II.D} - Southern plantation owners continued to own land; slaves generally unable to own large amounts due to difficult upkeep for those starting with limited wealth
        \item \textbf{5.3.II.E} - Segregation/violence/Supreme Court/local tactics gradually removed rights from Afr. Americans; 14th/15th Amendments held strong
        \item \textbf{6.1.II.D} - Some southern industrialization based on "New South" ideals; still primarily based around sharecropping/tenant farming
        \item \textbf{6.3.II.C} - \textit{Plessy v. Ferguson} $\to$ long-term segregation but Afr. Americans began to stand up for rights
    \end{itemize}}
    \cornell[The Problems of Peacemaking]{What immediate challenges were faced by post-Civil War leaders?}{\textbf{After the Civil War devastated the South and left both blacks and whites in mourning, several Southerners began to reconsider their notion of freedom: blacks saw it as a life without slavery but also as equal rights to whites; whites saw it as white supremacy and independence from the North. When plans for Reconstruction emerged, Republicans divided into Conservatives, seeking pardoning for Southerners, and Radicals, seeking hard punishments. Lincoln sided with the Conservatives, creating a moderate plan requiring little for readmission. After his assassination at the hands of Booth, Johnson succeeded him and was known for showing little sympathy to the freedmen. His reconstruction plan was strongly based on the Wade-Davis Bill passed by radicals months earlier, placing far more stringent requirements on states before they could return to the Union.}}
    \cornell{What was the initial aftermath of the Civil War?}{\begin{itemize}
        \item Civil War $\to$ unparalleled devastation for South 
        \begin{itemize}
            \item Land, bridges, railroads destroyed; plantations and fields gutted 
        \end{itemize}
        \item White southerners had lost several close to them as well as way of life
        \begin{itemize}
            \item Most white southerners, without slaves or Confederate bonds/currency or adult males $\to$ rebuilt lives independently 
            \item 258k killed soldiers $\to$ 20\% of adult white male population killed; with nearly everyone having lost men close to them, mourning emerged for multiple years
            \item "Lost Cause" emerged where whites mourned past South, revering Jackson, Lee, later Davis
        \end{itemize}
        \item Conditions worse for blacks: many had served in war and most had left plantations $\to$ no immediate home
        \begin{itemize}
            \item Roamed between cities, camping in countryside; some hoped for help from Union occupation forces
        \end{itemize}
    \end{itemize}
    \textbf{The aftermath led to both infrastructural and territorial destruction but also devastated the lives of whites and blacks alike. White southerners had lost nearly 20\% of adult males in society, and their former way of life was in shambles. Blacks, too, were often left without homes and roamed between cities.}}
    \cornell{How did the African American notion of freedom conflict with the white one?}{\begin{itemize}
        \item African Americans saw freedom as no slavery but also rights/protections equal to whites
        \begin{itemize}
            \item Some sought to achieve by redistributing economy (notably land) because all had worked to build it up
            \item Others sought legal equality: with equal opportunity, could build themselves up to parallel whites
            \item All unified in no white control $\to$ formed Afr. American communities with new churches, aid societies, school
        \end{itemize}
        \item White southerners saw freedom as no Northern/federal intervention and white supremacy 
        \begin{itemize}
            \item Many attempted to restore this notion of freedom post-war by restoring society to antebellum state 
            \item Several kept black workers legally tied to plantations
        \end{itemize}
        \item Union troops remained in South to protect freed slaves $\to$ Freedmen's Bureau under Congress
        \begin{itemize}
            \item Bureau provided food to former slaves, created schools led by Freedmen's Aid Societies
            \item Supported poor whites devastated by war
            \item Never long-term solution
        \end{itemize}
    \end{itemize}
    \textbf{African Americans saw freedom as an unenslaved state with rights equal to whites through legal equality as well as their own independent communities; whites, contrastingly, saw freedom both as independence from the North but also as continued white supremacy. Black freedom was preserved and enforced by Union troops under the Freedmen's Bureau, who also established schools and provided food to freed slaves.}}
    \cornell{What were the critical questions of Reconstruction?}{\begin{itemize}
        \item Reconstruction based around partisan politics: Republican majorities of 1860/1864 due to disunited Democratic Party/no South
        \begin{itemize}
            \item Both parties aware that restored South $\to$ Democrats likely to regain power $\to$ Republican programs issued during majority at risk
            \item Several Northerners felt South should be punished for rebellion, and its society be modeled on urban North's
        \end{itemize}
        \item Republican Party divided about how to approach readmission of South
        \begin{itemize}
            \item All agreed that slavery must be formally abolished in all state codes; conservatives felt little more necessary
            \item Radicals (like \textbf{Thaddeus Stevens}, \textbf{Charles Sumner}) demanded punishment of military leaders, several Southern whites deprived of right to vote, wealth of rich Southerners assisting Confederacy seized 
            \begin{itemize}
                \item Some sought suffrage for blacks but many did not want to conflict with Northern laws 
            \end{itemize}
            \item Moderates sought no punishment but greater concessions for Afr. American rights
        \end{itemize}
    \end{itemize}
    \textbf{Reconstruction was based around party politics and the Republican fear that regained Southern power would mean a Denocrat majority. More radical Republicans thus argued for removal of the right to vote and confiscation of wealth for several Southern rights while conservatives pushed only for the abolition of slavery.}}
    \cornell{What formal plans emerged for Reconstruction?}{\begin{itemize}
        \item Lincoln supported Moderates/Conservatives on issue: felt few demands $\to$ more Southern Republicans; frelt freedman fate could be postponed
        \begin{itemize}
            \item Reconstruction plan pardoned white Southerners (apart from Confederate leaders) who were loyal to govt., agreed with abolishment of slavery; hoped suffrage would be given to educated, property-owning blacks from Union army
            \item After 10\% of voters took Lincoln's oath, allowed to form state govt.; LA, AR, TN formed govt. in 1864
        \end{itemize}
        \item Radicals felt too mild, initially hesitant for solution; passed \textbf{Wade-Davis Bill} through Congress
        \begin{itemize}
            \item Allowed president to select provisional governor; after \textit{majority} pledged oath to U.S., allowed to form constitutional convention with requirements of abolishment, repudiation of debts
            \item Delegates only those who swore they had never borne arms against U.S.
            \item Lincoln quickly vetoed but realized importance of conceding to radicals
        \end{itemize}
    \end{itemize}
    \textbf{Lincoln produced a moderate plan for reconstruction pardoning all non-Confederate leader-white Southerners and allowed a state government after 10\% of voters promised they would abolish slavery, would remain loyal to government. Radicals produced the Wade-Davis Bill, which gave the president the choice of governor; a constitutional convention could be formed only after 50\% pledged oath and could only be populated by those who promised to have never attacked U.S. armies. Lincoln vetoed it.}}
    \cornell{What were the effects of Lincoln's death?}{
        On April 14th, 1865, Lincoln was assassinated at a theatre by Confederate actor John Wilkes Booth.
        \begin{itemize}
            \item Lincoln hailed as martyr throughout North
            \item Several accused Booth of being part of Southern conspiracy; in fact linked with group seeking to assassinate Lincoln, secretary of state Seward, VP Johnson
            \item Booth escaped to VA countryside but pursued and killed by Union forces; four of associates hanged
            \item Republicans interpreted as continued hostility from Southern leaders
        \end{itemize}
        \textbf{Booth, part of a group seeking to kill Lincoln, Seward, and Johnson, assassinated Lincoln in April 1865, but was soon killed in the Virginia countryside. Republicans saw Lincoln's assassination as a sign of continued Southern tensions.}}
    \cornell{How did Andrew Johnson implement restoration plans of his own?}{\begin{itemize}
        \item Johnson, Democrat until joining forces with Lincoln, hostile toward freedmen and felt reconstruction should be driven by South
        \item Developed plan for Reconstruction known as "Restoration"
        \begin{itemize}
            \item Southerners who took oath immediately pardoned; high-ranking officials/wealthy plantation owners forced to appeal directly to him for pardon 
            \item Generally based on Wade-Davis Bill w/ provisional governor to select unspecified \# of voters to elect delegates (w/ majority required)
            \item State would need to abolish slavery, revoke secession, ratify 13th Amendment, reject Confederate/state war debts 
            \item No assistance to former slaves w/ many often remaining in similar conditions
        \end{itemize}
        \item By end of 1865, all seceded states formed new govt. based on Lincoln/Johnson plans; radicals would not accept
        \begin{itemize}
            \item Radicals felt reluctance to abolish slavery + election of Confederate leaders (like Confed. VP to GA Senate) reflected false loyalty
        \end{itemize}
    \end{itemize}
    \textbf{Johnson, a Democrat at heart, was hesitant about reconstruction and developed a plan very similar to the Wade-Davis Bill. Although all states not already following Lincoln's plan formed state governments, radicals refused to accept their admission to Congress as Northern opinion had slowly hardened due to the South reluctance to abolish slavery.}}
    \end{document}