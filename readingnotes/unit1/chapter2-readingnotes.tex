\documentclass[a4paper]{article}
    \usepackage[T1]{fontenc}
    \usepackage{tcolorbox}
    \usepackage{amsmath}
    \tcbuselibrary{skins}
    
    \usepackage{background}
    \SetBgScale{1}
    \SetBgAngle{0}
    \SetBgColor{red}
    \SetBgContents{\rule[0em]{4pt}{\textheight}}
    \SetBgHshift{-2.3cm}
    \SetBgVshift{0cm}
    \usepackage[margin=2cm]{geometry} 
    
    \makeatletter
    \def\cornell{\@ifnextchar[{\@with}{\@without}}
    \def\@with[#1]#2#3{
    \begin{tcolorbox}[enhanced,colback=gray,colframe=black,fonttitle=\large\bfseries\sffamily,sidebyside=true, nobeforeafter,before=\vfil,after=\vfil,colupper=blue,sidebyside align=top, lefthand width=.3\textwidth,
    opacityframe=0,opacityback=.3,opacitybacktitle=1, opacitytext=1,
    segmentation style={black!55,solid,opacity=0,line width=3pt},
    title=#1
    ]
    \begin{tcolorbox}[colback=red!05,colframe=red!25,sidebyside align=top,
    width=\textwidth,nobeforeafter]#2\end{tcolorbox}%
    \tcblower
    \sffamily
    \begin{tcolorbox}[colback=blue!05,colframe=blue!10,width=\textwidth,nobeforeafter]
    #3
    \end{tcolorbox}
    \end{tcolorbox}
    }
    \def\@without#1#2{
    \begin{tcolorbox}[enhanced,colback=white!15,colframe=white,fonttitle=\bfseries,sidebyside=true, nobeforeafter,before=\vfil,after=\vfil,colupper=blue,sidebyside align=top, lefthand width=.3\textwidth,
    opacityframe=0,opacityback=0,opacitybacktitle=0, opacitytext=1,
    segmentation style={black!55,solid,opacity=0,line width=3pt}
    ]
    
    \begin{tcolorbox}[colback=red!05,colframe=red!25,sidebyside align=top,
    width=\textwidth,nobeforeafter]#1\end{tcolorbox}%
    \tcblower
    \sffamily
    \begin{tcolorbox}[colback=blue!05,colframe=blue!10,width=\textwidth,nobeforeafter]
    #2
    \end{tcolorbox}
    \end{tcolorbox}
    }
    \makeatother

    \parindent=0pt
    \usepackage[normalem]{ulem}

    \newcommand{\chapternumber}{2}
    \newcommand{\chaptertitle}{Transplantations and Borderlands}

    \title{\vspace{-3em}
\begin{tcolorbox}
\Huge\sffamily \begin{center} AP US History  \\
\LARGE Chapter \chapternumber \, - \chaptertitle \\
\Large Finn Frankis \end{center} 
\end{tcolorbox}
\vspace{-3em}
}
\date{}
\author{}
    \begin{document}
    \maketitle
    \SetBgContents{\rule[0em]{4pt}{\textheight}}
    \cornell[Key Concepts]{What are this chapter's key concepts?}{\begin{itemize}
        \item \textbf{2.1.I.C:} English colonization attracted a large number of Europeans (mostly British) seeking social mobility, economic prosperity, religious freedom, improved living conditions; focused on agriculture on land taken from isolated natives
        \item \textbf{2.1.III.B:} Natives continually traded with European settlements, leading to cultural/economic changes and the rapid spread of disease
        \item \textbf{2.1.III.E:} Conflict with natives over land/resources/boundaries sparked military confrontations
        \item \textbf{2.2.I.A:} The unification of mixed ethnic groups through colonization increased intellectual exchange, pluralism enhanced by Enlightenment
        \item \textbf{2.2.II.A:} All colonies participated in slave trade (land abundance); New England/port cities used small numbers of slaves while Chesapeake, southern Atlantic coast had large numbers; greatest numbers sent to West Indies
    \end{itemize}}
    \cornell[The Early Chesapeake]{What were the characteristics of the early Chesapeake settlement?}{\textbf{The initial Chesapeake settlement, was centered in Virginia (led by the Virginia Company and later the crown) and Maryland (led by the Calverts and suffered major religious tensions between Catholics and protestants). Tobacco proved to be an extremely important crop, and, in Virginia especially, tensions with the natives led to great unrest in society, including the Bacon Rebellion.}}
    \cornell{What were the initial relations between the colonists and the natives in Jamestown?}{\begin{itemize}
        \item Journey of 144 men dwindled to 104 by arrival; named colony Jamestown in honor of King James
        \item Initial setup had multiple problems
        \begin{itemize}
            \item Selected swampy peninsula for security from natives, but extremely difficult to harvest; disease rampant
            \item Brought no women, reducing focus on community-building and more on hunt for gold
        \end{itemize}
        \item Colonist survival due to natives' showing agricultural techniques, important crops (\textbf{maize}), technology like canoes for river navigation
        \begin{itemize}
            \item Despite this, English insisted on inferiority of natives, calling them "savages"
        \end{itemize}
        \item Jamestown remained small for over a decade w/ natives far more powerful than colonists, united in large confederacy
        \begin{itemize}
            \item Within a few months, only 38 men remained alive
            \item Colony owed survival to Captain John Smith, age 27; known for powerful leadership (often at native expense)
        \end{itemize}
    \end{itemize}
    \textbf{Jamestown's survival was almost entirely due to the kindness of the natives, showing the colonists agricultural techniques, new crops, and technology. However, the colonists continually referred to the natives as "savages," despite being overwhelmed in population. The colony's population began to dwindle, but it was kept afloat by Captain John Smith.}}
    \cornell{How was Jamestown revived?}{\begin{itemize}
        \item London Company did not give up, obtaining new charter and sending nine new vessels to Jamestown
        \begin{itemize}
            \item Hit by disaster: one vessel ran aground in Bermuda Islands, other lost in hurricane 
            \item New settlers succumbed to fever by wintertime; after natives had realized threat posed by colonists, blocked off from further expansion, food 
        \end{itemize}
        \item Settlers endured terrible winter between 1609 and 1610
        \begin{itemize}
            \item When ship in Bermuda arrived in May, found only 60 people alive, all on the brink of death
            \item Departed for England but ran into another ship on James River led by e La Warr, new governor
            \begin{itemize}
                \item Convinced to return to colony, relief expeditions allowed for beginning of thriving time
            \end{itemize}
        \end{itemize}
        \item Settlers focused on cultivating tobacco, new crop to English
        \begin{itemize}
            \item First profitable crop; encouraged planters to move inland
        \end{itemize}
        \item Working conditions initially extremely harsh for settlers under first few governors, with little clear incentive
        \begin{itemize}
            \item As many began to deliberately dodge work despite punishment of potential death, Governor Thomas Dale allowed private ownership to provide personal incentive
            \item Landowners repaid company with part-time work, grain contributions 
        \end{itemize}
        \item Despite initial rocky leadership, Virginia began to expand, with settlements create beyond Jamestown
        \begin{itemize}
            \item Order imposed by governors essential to success along with profitable tobacco
        \end{itemize}
    \end{itemize}
    \textbf{Attempts by the London Company to continue sending vessels eventually failed, with native blockades reducing the settlers' food supply. Things were turned around when De La Warr, the first governor, arrived, encouraging all settlers to stay and imposing a harsh regime requiring all to work diligently. Although his successors eventually reduced his harsh policies, his rule, along with the importance of the tobacco crop, were critical to Jamestown's success.}}
    \cornell{What was the importance of tobacco farming in Virginia?}{\begin{itemize}
        \item Tobacco was initially discovered by Columbus w/ Cuban natives; had developed large stigma in England
        \begin{itemize}
            \item James I led campaign against it, touting connection to natives, cause of diseases, and profitability for Spanish
        \end{itemize}
        \item John Rolfe of Jamestown experimented w/ native-grown form of tobacco
        \begin{itemize}
            \item Especially harsh, leading to high quality tobacco
            \item Found many English buyers, spreading tobacco farming throughout Chesapeake area
        \end{itemize}
        \item Growth of tobacco farming required territorial expansion, further into native-owned territory
    \end{itemize}
    \textbf{Although tobacco was initially extremely stigmatic in England, even disapproved of by James I, John Rolfe's successful strain (from the natives) led to high demand for English-grown tobacco, turning it into a lucrative colonial industry.}}
    \cornell{What characterized the Virginia settlement after the emergence of tobacco farming?}{\begin{itemize}
        \item Virginia Company, despite emergence of tobacco farming, continued to grow in debt, launching program promoting migration
        \begin{itemize}
            \item Focused on "headright" system, based on land grants to new migrants based on size of family, number of sponsored immigrants
            \begin{itemize}
                \item Allowed many to establish large plantations with servants
            \end{itemize}
            \item Brought over ironworkers, craftsmen, women (generally between free and indentured) to diversify economy
            \begin{itemize}
                \item Promised all new settlers full rights of English citizen, end to strict rule
            \end{itemize}
        \end{itemize}
        \item Meeting in Jamestown church where delegates from each community met as House of Burgesses marked first meeting of elected legislature in U.S. (July 1619)
        \item In August 1619, first Africans came by boat
        \begin{itemize}
            \item Initially not likely treated with full harshness of slaves
            \item Use of black labor limited until 1670s: white indentured servants preferred (eventually scarce, expensive)
        \end{itemize}
        \item Relationships with natives became increasingly strained
        \begin{itemize}
            \item Governor Dale led continues assults against Powhatan group, kidnapping Pocohontas (daughter of chief) and converting her to Christianity; married John Rolfe
            \item Although Pocohontas' seizure led to reduced attacks temporarily; after brother took over, attacks continued
            \begin{itemize}
                \item Called settlers for trading, massacring 347 whites of all ages, sexes
                \item Failed uprising later pushed Powhatans to finally give up
            \end{itemize}
        \end{itemize}
        \item Virginia Company, amidst all these events, had become defunct
        \begin{itemize}
            \item Colony transferred to ownership of Crown
        \end{itemize}
    \end{itemize}
    \textbf{The Virginia Company continued to grow in debt despite the emergence of tobacco farming, forcing them to seek out new settlers and modify their ruling policies to be more democractic. With the settlers' relations with the natives becoming strained, climaxing in the massacre of 347 whites, the Virginia Company went defunct, leading to the colony's transfer into the hands of the crown.}}
    \cornell{What were the important agricultural techniques which the British learned from the natives?}{\begin{itemize}
        \item Despite continual hostility to natives for backwards technology (even blamed for inability to find gold), relied on techniques to farm New World soil 
        \begin{itemize}
            \item Natives had extremely successful farms
        \end{itemize}
        \item Despite not adopting all (like field clearing), learned a great deal, especially importance of maize for high yield, sugar source
    \end{itemize}
    \textbf{Although the British refused to learn a significant amount from the natives due to their insisted backwards technology, they most significantly learned about the importance of maize in the North American soil.}}
    \cornell{What marked the development of society in Maryland?}{\begin{itemize}
        \item Founded by son of Catholic George Calvert as retreat for English Catholics seeking refuge from Anglican-dominated England
        \begin{itemize}
            \item Lord Baltimore (Cecilius, son of Calvert) received grant carrying remarkable power, sent brother Leonard with 200-300 passengers
            \item First village: St. Mary's (named after Queen); native focus on rival tribes allowed Maryland settlers to experience no assaults/plagues/starving
        \end{itemize}
        \item To attract settlers, Calverts understood need to abandon Catholic emphasis, adopting policy of toleration for all
        \begin{itemize}
            \item Population quickly dominated by Protestants, soon appointed as governor 
            \item Political relations remained tense: Protestant majority banned Catholics from voting, repealed Toleration Act
        \end{itemize}
        \item Labor shortage encouraged "headright" system like in Virginia
        \begin{itemize}
            \item Major land grants to settlers; focus on tobacco cultivation, eventually driven by African slaves
        \end{itemize}
    \end{itemize}
    \textbf{Maryland was founded by the Calverts, a Catholic family hoping to create a place away from the Anglican-dominated England. However, when the migrants became majority Protestant, they adopted a policy of religious toleration. Politics quickly became turbulent, with the Protestant majority often discriminating against Catholics.}}
    \cornell{What were the major political tensions in mid-17th century Virginia?}{\begin{itemize}
        \item Having survived key initial hardships and expansion, Virginia began to take on political issues
        \item In 1642, King Charles I appointed William Berkeley as governor, who remained in power to 1670
        \begin{itemize}
            \item Initially popular for expansion rounds, defeating of natives in battle
            \item Part of native defeat involved large territory boost for settlers, but also promise not to expand beyond a certain line
            \begin{itemize}
                \item Challenge created by rapidly growing population (due to success of Oliver Cromwell leading his opponents to flee to colonies)
                \item Established three counties in territory designated for natives only
            \end{itemize}
            \item Berkeley soon ecame autocrat, reducing power to vote to landowners, keeping burgesses in power from year-to-year 
            \begin{itemize}
                \item Led to underrepresentation of those in "backcountry"
            \end{itemize}
        \end{itemize}
        \item By 1670s, many indentured servants had completed terms; left without home/money
                \begin{itemize}
                    \item Led to stealing, begging, working throughout colony
                \end{itemize}
    \end{itemize}
    \textbf{Although William Berkeley was initially a popular governor due to his territorial expansion, his growing autocratic regime began to lead to great discontent due to the underrepresentation of those not connected to him. Furthermore, expired indentured servants began to roam the colony without money or a home.}
    }
    \cornell{What was Bacon's rebellion?}{\begin{itemize}
        \item Nathaniel Bacon, wealthy young university graduate, arrived in Virginia as member of backcountry gentry (in west)
        \item Bacon disagreed with eastern leaders most significantly on native policy
        \begin{itemize}
            \item More directly threatened by native presence, leading him (and other western landowners) to push line of settlement further
            \item Unhappy with Berkeley's choice to exclude him from inner circle governor's council, fur trade
        \end{itemize}
        \item When angry natives struck against western plantation, local groups retaliated, leading to heavy response from natives
        \begin{itemize}
            \item Bacon became natural leader, defying Berkeley to attack Indians: declared as group of rebels 
            \item Transitioned to attack against colonial government
            \begin{itemize}
                \item Most powerful insurrection against authority in colonial history
            \end{itemize}
        \end{itemize}
        \item Bacon led army east, first winning temporary pardon but eventually (after pardon was not honored), poised to take over Jamestown; abruptly died of dysentery
        \begin{itemize}
            \item Troops defeated by arrival of British backup
            \item Did lead to Indian signing of new treaty allowing additional lands for settlement (aware of military power of settler forces)
        \end{itemize}
        \item Rebellion significant for symbolizing settlers' inability to abide by agreements with natives, natives' inability to tolerate additional expansion
        \begin{itemize}
            \item Revealed bitter competition between eastern and western landowners, potential for great instability in colony pushed by landless men
            \item Risk of social unrest from former indentured servants was one of many reasons for promotion of African slave trade (removal of indentured servants)
        \end{itemize}
    \end{itemize}
    \textbf{Bacon, an eastern landowner in Virginia, rebelled against Berkeley by attacking the hostile natives on the western border. He then led a revolution in Jamestown which, although eventually crushed by British troops, represented, in all, the tensions between natives and settlers (especially the inability to abide by agreements), competition between eastern and western landowners, and the risks posed by indentured servants.}}
    \cornell[The Growth of New England]{What characterized the growth of New England?}{\textbf{The New England settlement began with a group of discontent Puritan Separatists fleeing persecution at the hands of the government, for freedom of worship. In summary, New England contained the key colonies of Massachusetts, Connecticut, and Rhode Island, and had a very powerful Puritan undertone, though some areas (like Rhode Island), promoted religious tolerance.}}
    \cornell{What defined the departure to the Plymouth plantation?}{\begin{itemize}
        \item Separatists began to depart England quietly for Holland (known for toleration), but poor job opportunities led many to consider travel to New World
        \item Scrooby group (group of Separatists) received permission from Virginia Company, slight support of king to settle in British America
        \begin{itemize}
            \item Funded by merchants, hoping for funds at end of seven years
            \item Puritans viewed themselves as pilgrims; travelled from Plymouth on \textit{Mayflower}
            \item Long trip forced early settlement on Cape Cod, near Plymouth (established by John Smith years earlier)
            \begin{itemize}
                \item Outside of London Company's territory; despite no legal basis, established civil government and proclaimed allegiance to king 
            \end{itemize}
        \end{itemize}
    \end{itemize}
    \textbf{The Separatist departure to Plymouth was initiated by the Scrooby group, who were funded by merchants to travel on the \textit{Mayflower} on a long trip which ended early on Cape Cod.}}
    \cornell{What were the conditions on the Plymouth plantation?}{\begin{itemize}
        \item Puritans settled on deserted native village (devastated by plague likely brought by Europeans), faced challenging first winter 
        \begin{itemize}
            \item William Bradford, leader in England, became colony governor; faced numerous personal hardships but remained strong
            \item Dramatically changed landscape, including decrease in wild animal population (demand for furs, skins, meats); nearly entirely eliminated native population through smallpox
            \begin{itemize} 
                \item Farmed mix of native crops (corn, potatoes peas) and English crops (wheat, barley, oats)
            \end{itemize}
        \end{itemize}
        \item Experience with natives very different due to devastated population
        \begin{itemize}
            \item Natives understood importance of cooperation, assisted settlers in gathering seafood, cultivating corn, hunting local animals
            \begin{itemize}
                \item Key helpers: Squanto and Samoset (Squanto spoke English due to previous capture by English, time in Europe)
                \item Marked allegiance to natives through invitation to first Thanksgiving in 1621
            \end{itemize}
            \item Good relationship did not last long: second smallpox epidemic wiped out most remaining
        \end{itemize}
        \item Organization poor and profits low until arrival of military officer Miles Standish, leading to trading surplus
        \begin{itemize}
            \item Fur trade emerged with natives of Maine, population grew
        \end{itemize}
        \item "Plymouth Plantation" selected Bradford as governor once again, who persuaded Council for New England (successor to Plymouth Company) to allow legal permission to inhabit lands
        \begin{itemize}
            \item Ended harsh regime, communal labor plan of Standish
            \item Distributed land among families, eventually able to pay off financiers with wealth from fur trade
        \end{itemize}
        \item Pilgrims continually poor, but clung to God's word
        \begin{itemize}
            \item Cared less about how they were viewed by others than Puritans settling futher north
        \end{itemize}
    \end{itemize}
    \textbf{The Plymouth plantation, founded by Pilgrim Separatists, was known for a stronger relationship with the natives and gradual productivity through the fur trade and corn farming. Despite being continually somewhat poor, they remained strong in their conviction that God had called for a New World settlement.}}

    \cornell{What were the religious origins of the Puritan settlers and how did they view their settlement?}{\begin{itemize}
        \item Tension created by King James and later Charles I (by disbanding Parliament, destroying nonconformity)
        \begin{itemize}
            \item Puritans established Massachusetts Bay Company with charter from Charles to establish colony in New World (did not reveal that they were Puritans)
            \begin{itemize}
                \item Supplies provided by defunct fishing/trading company 
            \end{itemize}
        \end{itemize}
        \item Many Puritans viewed enterprise in new colony more than business venture: as haven for religious freedom
        \begin{itemize}
            \item Members began to move en masse to America
            \item John Winthrop chosen to be initial governor for affluence, education, piety, power
            \begin{itemize}
                \item Led first initial migration, bringing mostly family groups 
                \item Carried charter to Massachusetts Bay Company: colonists responsible not to anyone in England, but to themselves
            \end{itemize}
            \item Numerous settlements emerged: Boston (HQ, port), other towns throughout area
        \end{itemize}
    \end{itemize}
    \textbf{The Massachusetts colony, founded by covert Puritans in England, strongly emphasized the Puritan faith with a pious and serious society.}}
    \cornell{What was the political system of the Puritan colonies?}{\begin{itemize}
        \item Established colonial government initially with eight stockholders, but later all male citizens
        \begin{itemize}
            \item Winthrop eventually made to force election each year for role of governor 
        \end{itemize}
        \item Massachusetts founders had no intention from breaking away from English Church
        \begin{itemize}
            \item Had no remaining allegiance, but wanted to remain covert, with liberty to stand alone 
            \item Formed Congregational Church where each church has complete power
        \end{itemize}
        \item Puritans worshippers not of traditional faith of Anglican Church but instead emphasis on personal knowledge/belief of ministers, direct reading of Bible, John Calvin
        \begin{itemize}
            \item Exercised religious authority: dissidents in Massachusetts had no more freedom than the Puritans in England 
            \item Authority stemmed from individual communities, leading to pious society which many hoped would be beacon for New World
        \end{itemize}
        \item Political structure somewhat theocratic: church members were the only people who could vote/hold office, influenced heavily by ministers
        \begin{itemize}
            \item Government taxed all members, protected ministers
        \end{itemize}
        \item Colony had initial difficulties of winter, but greater number of families led to stronger community, more prosperous in long term
        \begin{itemize}
            \item Relied on natives for food/advice as well as Pilgrims
        \end{itemize}
    \end{itemize}
        \textbf{The Puritans had an advanced theocratical political system, including elections, positions of office, and ministers. Their society grew rapidly due to the emphasis on family migrations and the advice from the natives/Pilgrims.}}
    \cornell{What characterized the growth of Connecticut and Rhode Island?}{
        Because Massachusetts required all voters to be Puritan (otherwise they were forced to leave), many new arrivals began to depart and expand outward into the New England area.
        \begin{itemize}
        \item Connecticut Valley attracted numerous English families
        \begin{itemize}
            \item Thomas Hooker defined Massachusetts to created Hartford with colonial government, constitution
            \item Puritan minister, wealthy English merchant created New Haven
            \begin{itemize}
                \item Focused on combatting religious laxity in Massachusetts: extremely strict religious government
                \item Later combined with Hartford by royal decree 
            \end{itemize}
        \end{itemize}        
        \item Rhode Island originated in Roger Williams, an amicable Separatist
        \begin{itemize}
            \item Demanded that Massachusetts abandon all allegiance to Anglicans, separation between church and state; led to banishment
            \item After spending time with Narragansett tribesmen, brought followers to Providence, later receiving charter allowing government
            \item Rhode Island was only colony for some time which emphasized complete religious tolerance
        \end{itemize}
    \end{itemize}
    \textbf{As Massachusetts began to hone in on their religious strictness, many began to depart (often forced). Connecticut had Hartford and strictly religious New Haven, and Rhode Island (w/ Providence) was known for religious tolerance and separation of church and state.}}
    \cornell{Who was Anne Hutchinson and what did she stand for?}{
        Anne Hutchinson posed the greatest challenge to the Massachusetts order of all the dissenters.
        \begin{itemize}
            \item Argued that many members of Massachusetts clergy not part of "elect" (or true conversion experience), no right to hold office
            \begin{itemize}
                \item Charged that all ministers were not among the elect
            \end{itemize}
            \item Focused on proper role of women in Puritan society
            \begin{itemize}
                \item Powerful religious figure
                \item Developed large following among women, merchants, young men, numerous dissidents 
            \end{itemize}
            \item Massachusetts began to observe threat: Hutchinson's followers prevented Winthrop's reelection as governor
            \begin{itemize}
                \item Next year, returned to office; banished Hutchinson for heresy (despite remarkable theological knowledge)
                \item Moved to Rhode Island, New Netherland with followers; later died during native uprising
            \end{itemize}
            \item In Massachusetts, clergy began to further restrict role of women in response to Hutchinson, leading many of her followers to depart, mostly to New Hampshire and Maine
            \begin{itemize}
                \item Colonies had been already established but had failed due to few settlers
                \item Boomed after Hutchinson's followers departed en masse, began to populate region
            \end{itemize}
        \end{itemize}
    \textbf{Anne Hutchinson's emphasis on the role of the "elect" and her push for women's rights made her a very controversial figure. As she became more influential, she was tried for heresy and banished. Her followers travelled to Maine and New Hampshire, beginning new colonies based on Hutchinson's ideals.}}
    \cornell{What were the key relations between settlers and natives in New England?}{\begin{itemize}
        \item Population initially very small due to epidemics; surviving natives had sold much of their land, converted to Christianity 
        \item Native advice and presence crucial to early success of nearly all colonies
        \begin{itemize}
            \item Taught about local food crops, techniques (annual burning for fertilization, beans to replenish soil)
            \item Served as trading partners, particularly in fur trade, manufactured goods (iron pots, arrows, guns, alcohol)
            \begin{itemize}
                \item Commerce w/ natives created wealthiest families
            \end{itemize}
        \end{itemize}
        \item Tensions began to develop as settlers continually expanded land due to agragian economy (domesticated animals as wild ones disappeared)
        \begin{itemize}
            \item Brutality of conflicts encouraged Puritans to view natives as "savages"
            \begin{itemize}
                \item Some sought to "civilize" through conversion (translation of Bible by John Eliot)
                \item Others believed in extermination or displacement
            \end{itemize}
            \item Natives felt English were land-hungry; they frequently let their livestock run wild, destroying crops
            \begin{itemize}
                \item Led to numerous land/food shortages
                \item Decline led to great despair, which often promoted alcoholism
            \end{itemize}
        \end{itemize}
    \end{itemize}
    \textbf{In New England, although the natives were initially essential to the development of society through commerce and knowledge (about crops, agricultural techniques), tensions rose as the settlers continually expanded. While the settlers wanted to "civilize" the natives through Christianity, the natives simply felt the English were land-hungry and their food shortages led to great social despair.}
    }
    \cornell{How did technology play a part in the Pequot War and King Philip's War?}{\begin{itemize}
        \item Pequot War emerged as competition over trade with Dutch, friction over land
        \begin{itemize}
            \item English allied with Mohegan, Narragansett
            \item Most violent act: English setting fire to Pequot stronghold, killing hundreds of natives
            \begin{itemize}
                \item Marked end of war: tribe nearly completely wiped out
            \end{itemize}
        \end{itemize}
        \item Most prolonged encounter: King Philip's War
        \begin{itemize}
            \item Wampanoags, under leader known as King Philip to English (Metacomet) began to resist English, fearing incursion of English law, taking of lands
            \begin{itemize}
                \item Terrorized Massachusetts towns for three years, using organizational skills and guns
            \end{itemize}
            \item Massachusetts society heavily weakened; received aid from Mohawks, rivals of Wampanoags, and converted spies
            \begin{itemize}
                \item White militiamen attacked Indian villages, native food supplies; Mohawks killed and beheaded Metacomet, delivering head to leaders
                \begin{itemize}
                    \item Metacomet's alliance collapsed, Wampanoag tribe had ended, with leaders sold to slavery or executed
                \end{itemize}
            \end{itemize}
        \end{itemize}
        \item Natives continued to attack English colonies; New England settlers also began to face competition from Dutch/French
        \begin{itemize}
            \item French posed threat by allying with Algonquins, later attacking English
        \end{itemize}
        \item Wars with natives heavily characterized by undertone of technology exchange
        \begin{itemize}
            \item While English settlers took time to adapt to new flintlock rifle (far more efficient than matchlock), natives adopted immediately and quickly taught themselves how to handle them
            \begin{itemize}
                \item Built forge for handling, repairing rifles
            \end{itemize}
            \item Natives also relied on traditional technologies
            \begin{itemize}
                \item Narragansetts built enormous fort in Great Swamp of Rhode Island, side of bloody battle (burned down); later built stone fort
            \end{itemize}
            \item Technology ultimately proved no match for numbers, firepower of English settlers
        \end{itemize}
    \end{itemize}
    \textbf{The Pequot War was sparked by a trade competition with the Dutch and friction over land. King Philip's War, with the Wampanoags, was initiated by the natives, but the English quickly retaliated by allying with the Mohawks, wiping out the tribe. Technology played a major part in both wars: the natives had received rifles in trade and taught themselves how to use them, often using more advanced rifles than the English soldiers themselves.}}
    \cornell[The Restoration Colonies]{What were the Restoration Colonies and how were they governed?}{\textbf{The Restoration Colonies began after the English Civil War and included the Carolinas, known for their early involvement in slavery and the divisions between the north and south; New York, known for its diversity and wealth inequality; New Jersey, known for its diversity but no large, wealthy class or city due to the lack of a harbor, and Pennsylvania, a Quaker colony founded by William Penn known for cosmopolitan Philadelphia and strong relations with the natives.}}
    \cornell{How did the English Civil War influence colonialism?}{\begin{itemize}
        \item When Charles I dissolved Parliament then called back into session and dissolved it again twice in two years, military challenge emerged
        \begin{itemize}
            \item English Civil War began, pitting Cavaliers (supporters of king) against Roundheads (mostly Puritan supporters of Parliament)
            \item After seven years, Roundheads captured and beheaded king, replacing with Oliver Cromwell as "protector"
        \end{itemize}
        \item Charles II returned from exile, came to power after Cromwell's heir was unable to maintain authority 
        \begin{itemize}
            \item Faced similar problems to father: many believed he was secretly Catholic
            \item Supported toleration, but Parliament refused to agree (prudently allowed them to proceed)
            \item Return of Charles II (Stuart Restoration) led to resumption of colonization in U.S.
            \begin{itemize}
                \item Rewarded courtiers with land grands (Carolina, New York, New Jersey, Pennsylvania)
                \item New colonies modeled on proprietary ventures (like Maryland), not company-based
                \begin{itemize}
                    \item Marked beginning of colonies being viewed as permanent settlements not for commercial success but for long-term rule
                \end{itemize}    
            \end{itemize}
        \end{itemize}
    \end{itemize}
    \textbf{When Charles I was beheaded and removed from power, Cromwell took over, limiting colonization (which had already been limited by the Civil War). However, once Charles II took power again, colonization resumed, with the four colonies of Carolina, New York, New Jersey, and Pennsylvania. Colonization took a new, long-term focus rather than quick commercial ventures.}}
    \cornell{What were the key characteristics of the Carolinas?}{\begin{itemize}
        \item Carolina awarded to eight court favorites: vast territory from Florida in south to Pacific in west 
        \item Proprietors hoped to profit by implementing headright system with annual tax (quitrents)
        \begin{itemize}
            \item Guaranteed religious freedom, hoping to receive local migrants from other colonies
            \item Political freedom through representative assembly
        \end{itemize}
        \item Initial efforts failed: many initial proprietors gave up
        \item Anthony Ashley Cooper (Earl of Shaftesbury) persisted with Carolinas
        \begin{itemize}
            \item Convinced partners to finance migrations from England
            \begin{itemize}
                \item Very difficult voyage (1/3 survived)
            \end{itemize}
            \item New migrants created town of Charles Town 
            \item Earl focused on order, driven by principles of John Locke
            \begin{itemize}
                \item Elaborate social utopia which never truly came to be
            \end{itemize}
        \end{itemize}
        \item Northern and southern regions extremely different socially and economically
        \begin{itemize}
            \item Northern settlers: backwoods farmers with little connection to outside world, focus on subsistence agriculture
            \item Southern settlers: Charles Town led to prosperous economy, position on Ashley and Cooper rivers led to numerous other settlements
            \begin{itemize}
                \item Rice became crucial crop
                \item Close ties with Barbados (made up majority of new migrants)
                \begin{itemize}
                    \item Many arrived with African workers, becoming landlords
                    \item Barbadian migrants established similar slave-based system to what had existed on Barbados, continuing in African slave trade
                \end{itemize}
            \end{itemize}
        \end{itemize}
        \item Carolina remained among most unstable of English colonies
        \begin{itemize}
            \item Conflict between north and south eventually led to division of region by king
        \end{itemize}
    \end{itemize}
    \textbf{The Carolinas, initially a failure, were taken on by the Earl of Shaftesbury, who financed a major migration from England and attempted to create a utopian society. However, with the northern backwood farmers and the southern wealthy, prosperous plantation owners with slave labor from Barbados, Carolina was never truly united. It was eventually divided into North and South Carolina by the king.}
    }
    \cornell{What were the origins of the New York colony?}{\begin{itemize}
        \item James, Duke of York, received territory between Connecticut/Delaware rivers
        \begin{itemize}
            \item Large portion claimed by Dutch, representing larger commercial rivalry throughout world
            \item England unhappy about Dutch colony in New Amsterdam: broke up English colonies into two groups and allowed smugglers to evae English customs laws
        \end{itemize}
        \item English fleet commanded by Richardd Nicolls sailed into lightly defended port, demanding surrender
        \begin{itemize}
            \item Peter Stuyvesant, unpopular leader, failed to mobilize resistance 
            \item British promised that Dutch settlers would not be displaced
            \item Dutch temporarily regained control, but lost for good after a year 
        \end{itemize}
        \item James renamed colony "New York," known for diversity, tolerance
        \begin{itemize}
            \item Remained in England, delegating power to governor (though no representative assembly due to father's run-in with parliament)
            \item Laws established religious toleration
            \item Tensions emerged over power distribution with wealthy Dutch "patroons" not having lost any power
            \begin{itemize}
                \item James created large class of loyal landowners by donating large portions of land
                \item Wealth unequally distributed between patroons, landlords, fur traders
            \end{itemize}
            \item Colony very prosperous, rapidly growing in size
        \end{itemize}
        \item When James gave part of land to political allies (John Berkeley, George Carteret), New Jersey began
        \begin{itemize}
            \item Carteret named New Jersey, but political tensions quickly returned it to the Crown's control
            \item Despite diversity like New York, developed no large landowner class and no major city (no harbor)
        \end{itemize}
    \end{itemize}
    \textbf{The New York colony, led by the Duke of York, was made possible after the English took control over the Dutch New Amsterdam. It was known for immense religious diversity, religious tolerance, and unequal wealth distribution. Additionally, New Jersey emerged when James donated his southerly land to a pair of Carolina proprietors.}}
    \cornell{Who were the Quakers and what was their colonial significance?}{\begin{itemize}
        \item "Society of Friends," the Quakers were pacifistic, supported gender equality in religion, rejected predestination
        \begin{itemize}
            \item Known for anarchistic/democratic structure, with periodic representative meetings but no paid clergy, church government
            \item Referred to all with informal "thee" and "thou"
            \item Stigma in England developed due to their unorthodox practices, occasional interference with other groups
        \end{itemize}
        \item Quakers turned to America for asylum: in all but RI and NC, met with same struggles as in Englan
        \begin{itemize}
            \item Despite large population in NC, many wanted colony of their own, but needed influential courtsman to attain royal grant
            \begin{itemize}
                \item William Penn, son of admiral in Royal Navy, with George Fox, supported the Quaker's cause as a believer in the doctrine of the Inner Light
            \end{itemize}
            \item Penn first attempted New Jersey, but ultimately, due to large debt owed to father by king, received large grant between New York and Maryland (larger than England/Wales, most naturally prosperous)
            \begin{itemize}
                \item Named Pennsylvania after Penn's father
                \item Hoped to be profitable, promoting throughout England, including a large population of Swedes/Finns
                \item Colony never profitable for him; died in poverty
                \item Created city of Philadelphia (brotherly love), understood that all land belonged to natives (from Quakerism) and wanted to ensure reimbursement: no major conflicts, viewed as honest
                \item Prosperity of Pennsylvania in part due to numerous emigrants, mild climate, fertile soil, planning
            \end{itemize}
        \end{itemize}
        \item Resistance began to emerge against proprietor, with Penn finally agreeing to Charter of Liberties, establishing representative assembly, separate colony of Delaware
    \end{itemize}
    \textbf{The Quakers, pacifist Christians rejecting predestination and known for their democratic structure, migrated to America in hopes of creating a new colony for asylum: they were ultimately successful through Penn's creation of Pennsylvania, a prosperous colony known for the city of Philadelphia.}}
    \cornell[Borderlands and Middle Grounds]{What were the Borderlands colonies?}{\textbf{The Borderlands colonies belonged to the Spanish and were centered in the southwest and southeast. Especially in the southeast, there was great tension (which eventually led the U.S. to win over Florida) between the Spanish and the English. Furthermore, the Caribbean, one of the first locations of colonization, became one of the first locations to become heavily involved in slavery.}}
    \cornell{What were the conditions on the Caribbean Islands?}{\begin{itemize}
        \item Most important early destination during early seventeenth century for migrants
        \begin{itemize}
            \item Influenced development of mainland, but also conflicts with Spanish
        \end{itemize}
        \item Substantial native populations pre-Columbus, but epidemics nearly completely wiped out
        \item Spanish claimed all Caribbean islands, but only settlement on largest few: Hispaniola, Cuba, Puerto Rico
        \item English/French/Dutch began settling on smaller islands (vulnerable to Spanish attack)
        \begin{itemize}
            \item English settlements in Antigua, St. Kitts, Jamaica, Barbados target of constant attack from natives, Spanish, Portuguese, French, Dutch 
        \end{itemize}
        \item Early economies built in export crops (mainly \textbf{sugar}), for distillation into rum
        \begin{itemize}
            \item Demand for sugar cane led to cutting down of forests, destroying of natural lands, limiting agriculture
            \item Need for labor eventually led to indentured servants from England, but arduous work made it unsuccessful
            \item Ultimately relied on African workers, who soon outnumbered them
            \item English planters particularly tough on workers
            \begin{itemize}
                \item Many were extremely wealthy with entire livelihoods relying on sugar farming
            \end{itemize}
        \end{itemize}
    \end{itemize}
    \textbf{The Caribbean Islands, initially claimed by the Spanish (which led to some early conflicts when the English began to settle), were known for a reduced native population and an export-based economy in sugar cane. Ultimately, they were one of the first locations to rely on the African slave trade.}}
    \cornell{What was the master-slave dynamic in the Caribbean?}{\begin{itemize}
        \item Small, wealthy white population, large African population spelled turbulence 
        \begin{itemize}
            \item Landowners began to fear slave revolts (at least seven major ones occured on islands)
            \item For safety, landowners gave themselves complete legal control over slaves, even able to murder them 
            \item Economically, many believed that cheapest solution was to continually buy new slaves after other ones died
            \begin{itemize}
                \item Often worked slaves to death in harsh climate; even whites often died due to tropical diseases
            \end{itemize}
        \end{itemize}
        \item Conditions on Caribbean islands not conducive to stable society: mostly single men, no long-term commitment to lands, often returned to England 
        \begin{itemize}
            \item Those who did not depart were often too poor to maintain society; few women, little intermarriage between blacks/whites 
        \end{itemize}
        \item Africans created world of their own: started families, sustained religions while ignoring Christianity, patterns of resistance
        \item Settlements connected through colonies through trade (sugar/rum for manufactured goods), copying of ideas on mainland (particularly slavery)
    \end{itemize}
    \textbf{A harsh relationship, little intermarriage occurred between blacks and whites. Although the Africans made a world of their own, Caribbean societal conditions were never conducive to stable society, with few women and most men without long-term committment to the land.}}
    \cornell{What were the characteristics of the southwestern borderlands?}{\begin{itemize}
        \item Spain known for large empire but centered in Mexico; much greater prosperity than England
        \begin{itemize}
            \item U.S. colonies of Florida, Texas, New Mexico, Arizona, California brought religious minotirites, Catholic missionaries, ranchers, troops
            \item Extremely weak in northern flank
        \end{itemize}
        \item New Mexico most prosperous after quelling of Pueblo revolt
        \item California soon began at San Diego, Monterey, San Francisco, Los Angeles, Santa Barbara: trading communities
        \begin{itemize}
            \item Native population in California rapidly died out, most remaining forced to convert
            \item Natives forced to work on agricultural economy 
        \end{itemize}
        \item By late seventeeth century, viewed greatest threat to northern borders the French
        \begin{itemize}
            \item Travelled down Mississippi Valley, claiming lands for Louis XIV
            \item Spanish began to fortify existing settlements, especially Arizona, which was known for nomadic natives who would not convert, and California 
        \end{itemize}
        \item Unlike English, Spanish in Southwest saw natives as crucial to success, enlisting in labor, conversion
    \end{itemize}
    \textbf{The southwestern borderlands were known for the Spanish states of New Mexico, Texas, Arizona, and California. The threat of the French promoted significant development in the region, and the natives were a crucial part to the region's success.}}
    \cornell{What were the characteristics of the southeastern borderlands?}{\begin{itemize}
        \item Spain had colonized Florida, began moving into Georgia, westward into modern panhandle
        \begin{itemize}
            \item After English built Jamestown, cut expansion and focused on existing regions, building forts
            \item Area between Carolinas/Florida became region of continuing tension
            \begin{itemize}
                \item Spanish v. English, Spanish v. French in Louisiana area
            \end{itemize}
        \end{itemize}
        \item No formal war, but tensions were high
        \begin{itemize}
            \item English encouraged Spanish natives to revolt; Spanish encouraged slaves to be free in exchange for Catholicism
            \item Spanish population in area dwindled extremely low, forcing them to rely on the natives and Africans for intermarriage
        \end{itemize}
        \item England eventually prevailed in Florida after the conclusion of the Seven Years' War
    \end{itemize}
    \textbf{The southeastern borderlands were a site of far more direction tension between the English and the Spanish. Ultimately, the English prevailed and Florida became part of the British empire.}}
    \cornell{What were the conditions through which Georgia was founded?}{\begin{itemize}
        \item Founded by James Oglethorpe, member of Parliament
        \begin{itemize}
            \item Driven by military, philanthropic motives: hoped to create buffer against Spanish lands, refuge for impoverished English
            \begin{itemize}
                \item Oglethorpe pitied debtors dying in confinement, hoping to transfer them to America for new fate 
            \end{itemize}
            \item Need for buffer created by frequent hostilities in South Carolina driven by conflict between English and Spanish
        \end{itemize}
        \item Georgia involved limited landholdings for more compact settlement, banned Africans for fear of slave revolts or turning against them on side of Spanish, banne Catholics for potential collusion with Spanish 
        \begin{itemize}
            \item Built fortified town on Savannah River
            \item Very few debtors were ultimately freed, mostly impoverished brought from England
            \item Trade with natives reduced due to risk of insurrection 
        \end{itemize}
        \item Most arrivals believed Oglethorpe's hold was too tight, seeking to own slaves as workforce 
        \begin{itemize}
            \item Failure of colony encouraged hold to be loosened, ban on slavery to be removed, prohibition ended
        \end{itemize}
    \end{itemize}
    \textbf{Georgia, founded by military man and philanthropist James Oglethorpe, initially hoped to create a tight land where debtors could start fresh. Ultimately, however, most of the new migrants disliked the restrictions, causing them to slowly loosen, most significantly the allowance of slaves.}}
    \cornell{What were the "middle grounds"?}{\begin{itemize}
        \item Relations between natives and Europeans consistently tense
        \item French much more aware of how to treat correctly, including by treating tribal chiefs respectfully
        \begin{itemize}
            \item English took long time to learn what came naturally to French, and eventually newer settlers lost complex rituals
        \end{itemize}
        \item Natives were often victims, but also obstacles to colonization
    \end{itemize}
    \textbf{The middle grounds was the interior region of the U.S. It was known for a considerable population of natives, with whom the English found it especially difficult to treat cordially.}}
    \cornell[The Evolution of the British Empire]{What was the evolution of the British Empire?}{\textbf{Mercantilist ideals began to influence the British Empire's decisions in the Americas. For instance, New England was temporarily unified and the Navigation Acts were instated to limit trade with other foreign powers. However, the Glorious Revolution reversed many of these decisions, returning the colonies to a state much like their previous one.}}
    \cornell{What major reorganization occurred?}{\begin{itemize}
        \item Many pushed for unification of colonies for profitability through raw materials
        \begin{itemize}
            \item To push mercantilist goals, required abandonment of foreigners in trading process 
        \end{itemize}
        \item Challenging because many goods were unsuitable for export; colonists often traded with Spanish/French/Dutch over English for profitability
        \begin{itemize}
            \item Oliver Cromwell banned Dutch ships from English colonies; Charles II adopted three Navigation Acts
            \item First Navigation Act required trade with English ships only, required certain items to be exported
            \item Second Navigation Act required all other goods going to Europe to pass through England to be taxed
            \item Third Navigation Act prevented common lie that boat was going to other English colony but instead foreign port through strong enforcement
            \item Navigation Acts led to major shipbuilding industry in colonies (to export through ships of their own - British ones)
            \begin{itemize}
                \item Often promoted English to subsidize goods due to importance
            \end{itemize}
        \end{itemize}
    \end{itemize}
    \textbf{The early reorganization was a push for greater unity between the colonies, which began with the Navigation Acts to promote trade with England only and not foreign ships.}}
    \cornell{What was the dominion of New England?}{\begin{itemize}
        \item To enforce Navigation Acts, oversee colonial affairs, limit authority of Massachusetts (often behaved as though independent), Charles II created royal colony in New Hampshire (no longer in control of Massachusetts)
        \begin{itemize}
            \item After Massachusetts defied Navigation Act, revoked charter priviliges and transformed into royal colony
        \end{itemize}
        \item Charles II's successor, James II, created single Dominion of New England, unifying New England colonies, New York, and New Jersey 
        \begin{itemize}
            \item Led by one governor, Edmund Andros, from Boston (known for stern, tactless behavior; highly unpopular in Massachusetts due to Anglican strengthening)
        \end{itemize}
    \end{itemize}
    \textbf{To enforce the Navigation Act, James II unified New England into a single dominion along with New York and New Jersey, ruled by a single governor.}}
    \cornell{What was the "Glorious Revolution"?}{\begin{itemize}
        \item James II, openly Catholic and trying to taking control of Parliament, was voted off the throne by Parliament, leaving without resistance (fearing fate of Charles I)
        \begin{itemize}
            \item Mary II (daughter) and husband, William of Orange, ruled jointly to replace him 
            \item Bostonians immediately removed Andros from power
        \end{itemize}
        \item New monarchs agreed with colonists mostly, removing Dominion of New England, combining Massachusetts and Plymouth, and making voting in Massachusetts not based on church membership but instead on property ownership
        \item Removal of Andros led many to resist rule of lieutenant governor of New York: Francis Nicholson
        \begin{itemize}
            \item Disliked by farmers, mechanics, small traders, shopkeepers; only liked by wealthiest fur traders
            \item Jacob Leisler, German immigrant, had harbored negative feelings; removal of Anros pushed to revolt, exiling Nicholson, ruling the government for two years
            \begin{itemize}
                \item Failed to restore any true unity to colony
                \item William/Mary appointed new governor: Leisler's resistance led to charge of treason, hanging by opponents
            \end{itemize}
        \end{itemize}
        \item In Maryland, many assumed that Lord Baltimore had sided with Catholic James II, leading to the revolt of Protestant John Coode
        \begin{itemize}
            \item Elected government was created; colonists received royal charter (as opposed to proprietry) and established strict anti-Catholic rule
            \item Returned to proprietor (Lord Baltimore V) after joining Anglican Church
        \end{itemize}
    \end{itemize}
    \textbf{The Glorious Revolution, involving the takedown of Catholic King James, led to great turmoil in the colonies, including the dissolution of the Dominion of New England, the takeover of New York by rebels temporarily, and the temporary transformation of Maryland into a royal charter colony.}}
    \end{document} 