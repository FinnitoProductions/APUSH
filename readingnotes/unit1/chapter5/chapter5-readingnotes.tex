\documentclass[a4paper]{article}
    \usepackage[T1]{fontenc}
    \usepackage{tcolorbox}
    \usepackage{amsmath}
    \tcbuselibrary{skins}
    
    \title{
    \vspace{-3em}
    \begin{tcolorbox}
    \Huge\sffamily \begin{center} AP US History  \\
    \LARGE Chapter 5 - The American Revolution \\
    \Large Finn Frankis \end{center} 
    \end{tcolorbox}
    \vspace{-3em}
    }
    \date{}
    \author{}
    
    \usepackage{background}
    \SetBgScale{1}
    \SetBgAngle{0}
    \SetBgColor{red}
    \SetBgContents{\rule[0em]{4pt}{\textheight}}
    \SetBgHshift{-2.3cm}
    \SetBgVshift{0cm}
    \usepackage[margin=2cm]{geometry} 
    
    \makeatletter
    \def\cornell{\@ifnextchar[{\@with}{\@without}}
    \def\@with[#1]#2#3{
    \begin{tcolorbox}[enhanced,colback=gray,colframe=black,fonttitle=\large\bfseries\sffamily,sidebyside=true, nobeforeafter,before=\vfil,after=\vfil,colupper=blue,sidebyside align=top, lefthand width=.3\textwidth,
    opacityframe=0,opacityback=.3,opacitybacktitle=1, opacitytext=1,
    segmentation style={black!55,solid,opacity=0,line width=3pt},
    title=#1
    ]
    \begin{tcolorbox}[colback=red!05,colframe=red!25,sidebyside align=top,
    width=\textwidth,nobeforeafter]#2\end{tcolorbox}%
    \tcblower
    \sffamily
    \begin{tcolorbox}[colback=blue!05,colframe=blue!10,width=\textwidth,nobeforeafter]
    #3
    \end{tcolorbox}
    \end{tcolorbox}
    }
    \def\@without#1#2{
    \begin{tcolorbox}[enhanced,colback=white!15,colframe=white,fonttitle=\bfseries,sidebyside=true, nobeforeafter,before=\vfil,after=\vfil,colupper=blue,sidebyside align=top, lefthand width=.3\textwidth,
    opacityframe=0,opacityback=0,opacitybacktitle=0, opacitytext=1,
    segmentation style={black!55,solid,opacity=0,line width=3pt}
    ]
    
    \begin{tcolorbox}[colback=red!05,colframe=red!25,sidebyside align=top,
    width=\textwidth,nobeforeafter]#1\end{tcolorbox}%
    \tcblower
    \sffamily
    \begin{tcolorbox}[colback=blue!05,colframe=blue!10,width=\textwidth,nobeforeafter]
    #2
    \end{tcolorbox}
    \end{tcolorbox}
    }
    \makeatother

    \parindent=0pt
    
    \begin{document}
    \maketitle
    \SetBgContents{\rule[0em]{4pt}{\textheight}}
    \cornell[Key Concepts]{What are this chapter's key concepts?}{ \begin{itemize}
        \item \textbf{3.1.II.D}: With economic shortages, British military occupation, colonists provided significant financial/material aid to Patriot movement
        \item \textbf{3.1.II.E}: Patriot cause succeeded due to colonial militias under George Washington's leadership, colonists' ideological commitment, resistance from European allies
        \item \textbf{3.2.I.A}: Enlightenment ideals -> many Americans emphasized individual ability over hereditary right; religion -> views of liberty
        \item \textbf{3.2.I.B}: Thomas Paine's \textit{Common Sense}, Declaration of Independence reflected belief in superiority in republican government 
        \item \textbf{3.2.1.C}: The Revolution saw an increase awareness of social inequalities, leading to a growing abolitionist, democratic movement
        \item \textbf{3.2.I.D}: Because women partook in the Revolution and Enlightenment ideals grew in popularity, idea of "republican motherhood" became significant, granting women new importance
        \item \textbf{3.2.I.E}: The American Revolution, D of I influenced revolutions in many other states
        \item \textbf{3.2.II.A}: New state constitutions placed power in legislative branch which maintained property qualifications for voting/citizenship
        \item \textbf{3.2.II.B}: Articles of Confederation created central government, w/ limited power; calls emerged for a stronger central government
        \item \textbf{3.3.I.C}: With growing westward expansion, Congress enacted the Northwest Ordinance for new states with private property, public education, banned slavery
    \end{itemize}
    }
    \cornell[The States United]{What marked the unity between disparate states at the onset of the revolution?}{\textbf{The states' key motivation for war began unclear, however soon centered around independence, which was made official by the Continental Congress' Declaration of Independence, written primarily by Thomas Jefferson. Responses to independence were generally positive, apart from a minority of Loyalists who either opposed war or remained loyal to Britain. America's success in the war was due in large part to the foreign aid provided by France and its allies.}}
    \cornell{What were the main goals of the Americans in war?}{\begin{itemize}
        \item 3 weeks after Lexington/Concord: Continental Congress met again in Philadelphia, with everyone agreeing to support the war
        \item Specific purpose of war remained debated
        \begin{itemize}
            \item Adams cousins of MA, Richard Henry Lee of VA, favored \textit{complete} indepdendence 
            \item John Dickinson of PA at opposite end, hoping for modest imperial reforms 
            \item Most delegates hoping for some middle ground, approving two disparate declarations in a short time; final appeal to king was rejected 
            \item Passed major proclamation "Declaration of the Causes and Necessity of Taking Up Arms," insisting on either submission or force 
        \end{itemize}
        \item Public was equally undecided: originally felt war aims were to redress grievances within empire; soon changed their minds
        \begin{itemize}
            \item High costs of war -> realized greater importance 
            \item Foreign mercenaries, slaves, natives hired by British -> most Patriots lost all loyalty 
            \item British enacted "Prohibitory Act," closing colonies to overseas trade and ignoring most American demands -> Americans felt British had lost interest 
        \end{itemize}
        \item Support for independence remained unspoken until \textit{Common Sense}, by Thomas Paine
        \begin{itemize}
            \item Wrote anonymously at 38; known for growing up in poverty, ridicule of Christianity
            \item Shifted American anger from parliamentary measures to the English constitution and behavior of corrupt king
        \end{itemize}
    \end{itemize}
    \textbf{The purpose of war was initially unclear. In the Continental Congress, some felt it was for complete independence, others for moderate reforms, and most were in the middle. In the public sphere, most initially believed the war was simply revenge against Britain, but the high cost of war, Britain's ruthless use of slaves and natives to fight, the "Prohibitory Act," and Paine's \textit{Common Sense}, soon convinced many that independence was the true goal.}}
    \cornell{What marked the official decision for indepdendence?}{\begin{itemize}
        \item Continental Congress began to make reforms favoring independence
        \begin{itemize}
            \item Opened ports to all but Britain; made direct communication with foreign powers
            \item Encouraged colonies to establish indepdendent governments
            \item Appointed committee to officially declare indepdendence
        \end{itemize}
        \item Declaration of Independence approved on July 4th, 1776, adopting resolution of complete independence
        \begin{itemize}
            \item Written mostly by Thomas Jefferson with help of Benjamin Franklin and John Adams
            \item Little of document was new, with majority expressing widespread beliefs voiced in colonies' local declarations if indepdendence
            \item First part of document restated John Locke's contract theory about rights given by government (made more idealistic), second part listed crimes of king who had violated contract
            \item Central idea that "all men created equal" sparked revolutions in U.S. and abroad
            \item Emphasis on "United States of America" gave much more foreign support to U.S., encouraging Patriots to fight for peace that was independence
        \end{itemize}
    \end{itemize}
    \textbf{The Continental Congress began to make small reforms to encourage independence, including opening ports to all nations but Britain, making direct communication with foreign powers, and encouraging colonies to establish independent governments. However, the most sweeping statement was the Declaration of Independence written mostly by Jefferson, which made clear the rights of government and how the king had blatantly violated those terms.}}
    \cornell{What were the main responses to the possibility of independence?}{\begin{itemize}
        \item Massive crowds gathered to celebrate Declaration of Independence, but minority (called themselves "Loyalists," referred to by Patriots as "Tories") who either opposed war or remained loyal to the king emerged
        \item Colonies officially called themselves "states," representing part of a bigger whole while remaining a sovereign entity
        \begin{itemize}
            \item Even before D of I, colonies had began to operate independently with Parliament having suspended representative government
            \item After D of I, each colony wrote a formal constitution (many of which survived for several decades)
        \end{itemize}
        \item National level far less successful: Americans uncertain whether a national government was necessary (Continental Congress viewed only as coordinating mechanism)
        \begin{itemize}
            \item Realized that fighting a war required central direction, forced to balance local power with centralized authority
            \item In November 1777: Articles of Confederation were adopted, confirming weak, decentralized system
            \begin{itemize}
                \item Continental Congress continued to coordinate war effort (but lacked direct power over individual states)
            \end{itemize}
        \end{itemize}
    \end{itemize}
    \textbf{Publically, most Americans were overjoyed by the possibility of independence; however, the minority Loyalists remained loyal to the king. Colonies began to refer to themselves as "states," with each state adopting a formal constitution. On the national level, however, self-government was far less successful, with little central direction, even from the Continental Congress.}}
    \cornell{What were the government's major challenges in mobilizing for the war?}{
        Without British supplies, the challenge of raising armies, finding supplies and equipment, and paying for everything became increasingly difficult.
        \begin{itemize}
            \item American gunsmiths could not meed wartime demand (despite arsenal at Springfield, MA), forcing reliance on weapons captured from British/provided by France
            \item Financing war perhaps the most difficult, with Congress lacking direct authority to tax people (only through state governments)
            \begin{itemize} 
                \item Even states struggled to raise, with each one far from expected contribution
                \item Government forced to issue paper money to keep up with financial difficulties, leading to inflation
                \begin{itemize}
                    \item Many American farmers/merchants preferred business with British due to gold/silver coin rather than devalued paper money
                    \item Caused many Philadelphia merchants in 1777-1778 not to sell food to George Washington's troops
                \end{itemize}
                \item Congress forced to borrow from other nations
            \end{itemize}
            \item As initial surge of patriotism faded, few Americans volunteered for military service, requiring persuasion or force
            \begin{itemize}
                \item Some militias began to form due to bounties to attract recruits, but entirely coordinated on state-level
                \item Congress attempted to correct with Continental army under control of George Washington in June 1775, early independence advocate and admired for military prowess 
                \begin{itemize}
                    \item Washington proved to be good choice: never gave up despite hardships including morale leading to open mutinies, lack of direct support by Congress (instead attempting to control military operations)
                    \item Received aid from foreign military experts; Washington ultimately prevailed against greatest power in the world, representing symbol of stability which all could trust
                \end{itemize}
            \end{itemize}
        \end{itemize}
        \textbf{Lack of British aid proved to be very challenging for the colonists, but they were ultimately supported by foreign powers in weaponry (American gunsmiths could not meet demand) and finances (tax collection was unsuccessful despite introduction of paper money). Furthermore, with the guidance of powerful military strategists worldwide, George Washington was able to organize a nation-wide army to ultimately triumph against the British.}}
        \cornell[The War for Independence]{What were the key events of the war for independence?}{\textbf{The first phase of the war was centered in New England and marked a period where the British were extremely half-hearted concerning the war efforts. After the British were driven from Boston, however, the second phase began and they regrouped, taking the battle more seriously; they were poised to win in nearly all regards; however, several key errors, including William Howe's decision not to follow his own plan, led to many key American victories. These victories enabled the Americans to rally foreign support (particularly from France), who formally recognized America's diplomatic independence. The British focused more efforts on the south during the third and final phase (where they erroneously believed there were a large number of Loyalists willing to rush to their aid), but the war symbolically ended when the southern general, Cornwallis, was defeated at Yorktown. True peace came from the 1783 Treaty of Paris, which gave Americans large swathes of land.}}
        \cornell{What were the advantages of each side in the battle?}{\textbf{At first, the British seemed to have all the advantages with the greatest navy and army, access to an entire empire, and a coherent structure, compared to the Americans' struggle to mobilize even a single army. However, the Americans were fighting on home ground compared to the British need to mobilize resources, the Patriots were directly committed to the war compared to half-hearted support of British, and they received major aid from abroad. The most direct cause of American success, however, was numerous egregious blunders by the British which prevented an early win.}}
        \cornell{What was the first phase of the war?}{\begin{itemize}
            \item British initially unsure whether they were actually involved in a war or simply quelling rebellion pockets; colonists soon went on offensive
            \item American forces besieged Gage's forces in Boston after Concord/Lexington; suffered severe casualties at Battle of Bunker Hill, ultimately losing
            \begin{itemize}
                \item In fact inflicted more casualties on the British than they suffered themselves 
                \item British concluded revolutionary fervor of Boston made it an unfit battleground, retreating to Nova Scotia with other Loyalists
                \begin{itemize}
                    \item Represented new phase of war, with British realizing severity of rebellion and importance of direct force
                \end{itemize}
            \end{itemize}
            \item War was inconclusive elsewhere, with Patriots crushing Loyalist uprising in NC, British not receiving expected aid from southern Tories
            \item In north, Americans invaded Canada, hoping to remove British threats
            \begin{itemize}
                \item Benedict Arnold paired with Richard Montgomery to invade Quebec; ultimately ended in frustration
                \item Benjamin Franklin headed civilian commission in attempt to earn allegiance; failed
            \end{itemize}
        \end{itemize}
        \textbf{The first phase began in New England, and was marked by British confusion about the exact state and significance of the war. After the Patriots attacked Gage's forces in Boston, the British retreated to Nova Scotia, realizing that Boston was an unsuitable battleground due to immense revolutionary fervor. Elsewhere, Loyalists uprising were crushed (NC) and the British were surprised by the lack of southern loyalist aid. Furthermore, the Americans were unable to win the support of Canada.}}
        \cornell{What marked the beginning of the second phase of war?}{\begin{itemize}
            \item Second phase placed British in clear advantage to easily win conventional war with obvious outmatching of Americans
            \item British regrouped after Boston, sending \textit{most formidable abroad} military force to NYC (32,000), commanded by William Howe
            \begin{itemize}
                \item Goal to awe Americans into submission and reveal hidden loyalty to king 
                \item Met with Congress, offering two choices: submission with royal pardon or direct battle
                \item Washington responded with army of only 19,000; Americans still rejected offer and chose to continue
                \begin{itemize}
                    \item Quickly outmatched, pushed away from Long Island, Manhattan
                \end{itemize}
            \end{itemize}
            \item Europeans viewed war as seasonal activity, with British settling down during winter at various outposts (including one of Hessians (foreign mercenaries) on Delaware River)
            \begin{itemize}
                \item Washington would not rest: occupied Hessian town, drove British forces away from nearby base in Princeton
                \begin{itemize}
                    \item Unable to hold onto claims, ultimately taking refuge for remainder of winter
                \end{itemize}
            \end{itemize}
            \item Key British strategy involved splitting American forces into two: Howe would move north from NYC, another British force would meet him from Canada
            \begin{itemize}
                \item John Burgoyne planned two-prong attack
            \end{itemize}
        \end{itemize}
        \textbf{The second phase began with William Howe, who hoped to shock the Americans into finding their true allegiance to the crown. However, they first fought at Long Island and were outmatched, being pushed back. An important strategy was formed between Howe and John Burgoyne: a two-prong attack coming at the Americans from both sides.}}
        \cornell{What factors allowed the tide to turn in favor of the Americans near the end of the second phase?}{
        \begin{itemize}
            \item Howe abandoned plan shortly after setting into motion, instead targeting Philadelphia
            \begin{itemize}
                \item Removed forces from NYC (by sea), moved them to head of Chesapeake Bay, brushing by Washington and occupying Philadelphia with ease
                \begin{itemize}
                    \item Continental Congress reassembled at York, PA
                \end{itemize}
                \item Burgoyne left to battle alone, advancing down Hudson River with flying start (easily seized Fort Ticonderoga, with it many supplies) -> Congress replaced Philip Schuyler with Horatio Gates
                \begin{itemize}
                    \item Burgoyne experienced two staggering defeats: one in Oriskany, NY w/ Patriot band of German farmers holding off force of natives/Tories, giving Arnold time to relieve Fort Stanwix, prevent advance of St. Leger; other at Bennington, VT where Americans cut off British material supply, forcing Burgoyne to surrender army of nearly 5,000
                \end{itemize}
            \end{itemize}
            \item American victory at Saratoga represented major success in public eye, leading to direct alliance between U.S./France
            \begin{itemize}
                \item British failure to win war during middle phase represented tragedy, failure of Howe (abandoned plan, allowed Washington to regroup by delaying final attack on Continental Army in PA); some believed Howe supported Americans, others credited alcoholism, romantic attachment, failure to understand true nature of war
            \end{itemize}
        \end{itemize}
        \textbf{After Howe abandoned his plan with Burgoyne and instead conquered Philadelphia, Burgoyne was left alone and, despite some early victories, was soon outmatched in Oriskany, NY and Bennington, VT. The American victory represented a major success in the public eye, which allowed the Americans to form key allegiances with foreign nations.}}
        \cornell{How were the Iroquois affected by the Revolution?}{\begin{itemize}
            \item Iroquois had declared themselves neutral, but many hoped that siding with English would limit white movement onto tribal lands
            \item Most significant Iroquois participation from British-supporting Joseph/Mary Brant, people of stature within the Mohawk nation attracting support of Senega, Cayuga as well as Mohawk; played key role in Burgoyne's campaigns
            \begin{itemize}
                \item Represented growing Iroquois divisions, with only three of six nations supporting the British (Oneida/Tuscarora backed Americans; Onondaga split up)
            \end{itemize}
            \item A year after Oriskany, Iroquois sided with British again in raiding white settlements in NY, but Patriots retaliated harshly, patricularly on natives (many fled to Canada, never returning)
        \end{itemize}
        \textbf{The Iroquois Confederation was greatly splintered by the Revolution as support for the British divided them in half. Furthermore, when the tide turned to the Americans, they experienced harsh retaliation and many fled to Canada.}}
        \cornell{How did foreign aid play a role in the war?}{
            The final phase of the war began with the American victories in mid-Atlantic leading to foreign support (most directly from France).
            \begin{itemize}
                \item Began before D of I when Congress sent representatives to European capitals to negotiate commercial treaties with new trading partners (required recognition as independent nation); little experience with Old World diplomacy, forced to make decisions w/o Congress
                \item France was always most supportive from Louis XVI and foreign minister, who hoped to undermine British power
                \begin{itemize}
                    \item Created fictional trading firms, secret agents to allow smooth trade between France/America; still reluctant to represent diplomatic independence
                    \item Benjamin Franklin went to France to represent U.S., earning international praise as a hero; paired with British victory at Saratoga, French finally recognized independence 
                \end{itemize}
                \item French recognition led to international conflict, with Spain/Netherlands all contributing to war against Britain, through munitions and direct naval support
            \end{itemize}
            \textbf{Foreign support was critical to the war: France was most supportive from the beginning, providing goods and eventually recognizing America as an independent nation. Soon after, Spain and the Netherlands joined the American cause, providing direct naval/army support for the Americans.}}
            \cornell{What marked the final phase of the war?}{\begin{itemize}
                \item After British lost at Saratoga, shifted attitude toward undermining effort from within by using "majority" still loyal to crown
                \begin{itemize}
                    \item Believed most of support lay in southern colonies (despite NC failure), shifting majority of effort there
                    \item New strategy was a complete failure
                    \begin{itemize}
                        \item Far fewer Loyalists than expected (a few Tory regulators), but, especially in northern parts of south (VA, MD), resistance was nearly as strong as in MA
                        \item Many Loyalists afraid of Patriot reprisal; cause further undermined when British encouraged southern slaves to flee owners ($\approx 5\%$ did), leading most supporters to join Patriot side
                        \item Logistical problems: American army could blend in with others without British knowledge of friend vs foe
                    \end{itemize}
                \end{itemize}
                \item Northern battles generally limited, ending in stalemates
                \begin{itemize}
                    \item Howe had been replaced by Henry Clinton; moved to NYC but watched carefully by Washington's troops
                    \item Fighting so little that Washington diverted some troops to fight against natives on borders
                    \item In winter, George Rogers Clark (VA) captured British settlements in IL 
                    \item Arnold shocked all by becoming a traitor, convinced that American cause was hopeless
                    \begin{itemize}
                        \item Conspired with British at West Point, but plan failed and he took refuge in British camp
                    \end{itemize}
                \end{itemize}
            \end{itemize}
            \textbf{The final phase saw a major loss by the British as Saratoga, which allowed the colonists to rally significant foreign support. The British formed a new strategy to primarily target the southern region, meaning that the northern battles were reduced and generally ended in stalemates.}
            }
            \cornell{What was the southern characteristic of final phase of the war?}{\begin{itemize}
                \item Southern phase was truly most "revolutionary": mobilized large, formerly aloof groups, with the war expanding into isolated communities; significant fights
                \begin{itemize}
                    \item British had some successes, like capturing Charles Town, encouraging Loyalists to aid in interior
                    \item Americans won unconventional battles through guerilla warfare
                    \item Lord Cornwallis (selected by Clinton as southern commander) crushed Horatio Gates' forces -> replaced with Nathanael Greene, Quaker from RI
                    \begin{itemize}
                        \item Tide had turned before arrival of Greene, but Greene further exasperated Cornwallis with troop divisions into small, fast-moving units
                        \item Carolina campaign ended at Guilford Court House, where, despite modest win, Cornwallis abandoned campaign due to loss of troops; hoped to launch campaigns in VA interior but Clinton relegated to Yorktown
                    \end{itemize}
                \end{itemize}
                \item As Cornwallis fortified Yorktown, GW and French commanders aimed to trap Cornwallis
                \begin{itemize}
                    \item Marched army from NYC to join other forces in Lafayette, VA; other troops arrived at Chesapeake
                    \item Armies trapped Cornwallis between land and sea, forcing him to capitulate; formal surrender of 7,000 men occurred two days later
                    \item Marked end of major fighting, but Britain still held many large port cities
                    \begin{itemize}
                        \item Fighting temporarily ceased between American and British forces; fate of the war unclear for over a year
                    \end{itemize}
                \end{itemize}
            \end{itemize}
            \textbf{The southern portion of the war was the most truly revolutionary, mobilizing large, formerly isolated groups. Lord Cornwallis, the southern commander for the British, retreated to Yorktown after Nathanael Greene overpowered his troops. In Yorktown, Cornwallis was gravely defeated, a battle which marked the symbolic end of the war.}}
            \cornell{What marked the end of the war?}{\begin{itemize}
                \item Cornwallis' defeat -> outcry in England, Lord North resigned to be replaced by Lord Shelbourne
                \item Americans were instructed to cooperate fully with France, but French would not settle war until Gibraltar was returned to ally, Spain (unlikely to happen soon)
                \begin{itemize}
                    \item Americans continued without French while Franklin carefully pacified Vergennes (foreign minister)
                \end{itemize}
                \item Final settlement in Treaty of Paris in 1783: both France and Spain agreed to end hostilities
                \begin{itemize}
                    \item Generous to U.S., ceding all territory from Canada to northern Florida, Atlantic to Mississippi under full control
                \end{itemize}
            \end{itemize}
            \textbf{Although Cornwallis' defeat symbolically represented the conclusion of the war, minor fighting still ensued for around a year. The war was officially ended by the Treaty of Paris in 1783, where France and Spain reluctantly agreed to cease hostilities.}}
        \cornell[War and Society]{What were the social impacts of the war?}{\textbf{The war had severe impacts on many distinct groups: the Loyalists and natives were shunned after the conclusion of the war and often driven to exile due to their assisting the English; some slaves earned their freedom due to the British, but many were simply touched and inspired by ideals of liberty; women, despite often assisting in the war, were left without significant short-term change, but mothers did see an increase in prestige. Economically, the war had a positive effect, sparking industry and spurring the American domestic industry.}}
            \cornell{How were Loyalists and other minorities impacted by the war?}{\begin{itemize}
                \item Between $\frac{1}{5}$ and $\frac{1}{3}$ of Americans loyal to Britain during the war; motivations were mixed 
                \begin{itemize}
                    \item Some due to status as imperial officers, others were merchants closely tied with British trade (though most merchants \textit{supported} the Revolution), others were isolated from British policies
                    \item Minorities feared that independence would reduce protection, social stability
                    \item Others believed the British would certainly win and wanted to side with the winners
                \end{itemize}
                \item Loyalists suffered extreme tribulations during the war, constantly harassed by Patriots and courts; ip to 100,000 fled the country
                \begin{itemize}
                    \item Those who could afford it (including Thomas Hutchinson, controversial MA governor) fled to England
                    \item Less well-off moved to Canada, founding first English-speaking community in Quebec
                    \begin{itemize}
                        \item Some returned after the war, reentered American life as tensions faded; others remained for the rest of their lives
                    \end{itemize}
                \end{itemize}
                \item Most loyalists were of average means, but many were wealthy with large estates, leadership
                \begin{itemize}
                    \item Suffered confiscation of property, forfeiting of positions 
                \end{itemize}
                \item Departure of loyalists did \textit{not} represent social revolution
                \begin{itemize}
                    \item Most who were previously wealthy remained wealthy
                \end{itemize}
                \item Other minorities also suffered, including certain religious groups
                \begin{itemize}
                    \item Anglicans (generally Loyalists) suffered the most: VA and MD removed Anglicanism as official religion, removing clergymen from churches
                    \begin{itemize}
                        \item Nearly eliminated from America in process
                    \end{itemize}
                    \item Pacificism of Quakers also led to hostility, destroying their social prestige
                \end{itemize}
                \item Catholic Church was strengthened, giving America its own hierarchy (numbers did not increase, but prestige did)
                \begin{itemize}
                    \item John Caroll of MD placed at the head
                \end{itemize}
            \end{itemize}
            \textbf{Loyalists suffered extremely, with their property generally taken and their positions forfeited. They were often forced to flee, most to Canada and the most wealthy to Britain. Although many of the Loyalists who fled were of high status, this did not constitute a social revolution. Furthermore, other minorities suffered, too, during the Revolution, including the Anglicans and the Quakers. Catholicism, however, grew in prestige.}}
            \cornell{How did the war impact slavery in the U.S.?}{\begin{itemize}
                \item For some slaves, the war meant freedom (British presence in south allowed them to leave the country)
                \item For many, the concept of liberty drastically changed social views
                \begin{itemize}
                    \item Most were unable to read, but profound ideas began to circulate through towns, cities -> open resistance against white control 
                    \item Thomas Jeremiah, free black, was executed after accusations of conspiring to smuggle British guns to SC slaves
                    \item Northern black writers produced eloquent pieces of writing to articulate lessons of liberty (\textit{ex}: Lemuel Hayes advocated for equal rights)
                \end{itemize}
                \item Ambivalence in SC and GA caused in large part due to slavery: slaveowners opposed emancipation of British but also feared growing possibility of large slave rebellions
                \begin{itemize}
                    \item In the North, antislavery efforts were more pronounced due to evangelical Christian fervor, Revolutionary sentiment
                    \item South remained committed to slavery due to ideals of white superiority
                    \begin{itemize}
                        \item War exposed tension between liberty and slavery, with many whites believing that black slavery would further promote liberty for whites 
                        \item Feared that reduction in black workforce would require whites to become servile 
                    \end{itemize}
                \end{itemize}
            \end{itemize}
            \textbf{For some slaves, the war allowed them to earn their freedom due to the British; however, most were only touched by the ideal of liberty. The war exposed the growing tension in the south between the ideals of liberty and slavery, which, despite being incompatible today, many felt were possible if blacks were not viewed as true American citizens.}
            }
            \cornell{How did the Revolution impact the natives?}{\begin{itemize}
                \item Most natives viewed w/ uncertainty; Patriots tried to persuade neutrality in "family quarrel"
                \item Land was at stake: British believed that expansion should be limited and native lands protected to prevent large wars
                \begin{itemize}
                    \item Revolutionary War fought for freedom to expand westward (directly affected GW, a landowner/speculator himself)
                    \item Cherokee faction led by Dragging Canoe attacked white settlements in western VA, Carolinas -> fought back with severity, ravaging lands and forcing to flee west
                    \begin{itemize}
                        \item Remaining Cherokees agreed to treaty giving up more land
                    \end{itemize}
                    \item Other native campaigns more successful, including Iroquois waging of war against whites in West, destroying agricultural parts in NY, PA 
                \end{itemize}
                \item Revolution ultimately weakened native position
                \begin{itemize}
                    \item Patriot victory led to westward expansion with little regard for tribes; often hostile against tribes due to assistance toward British, others asserted state as savages due to culture
                    \item Revealed + increased divisions between tribes, with revolutions rarely seeing allies/common front against whites (\textit{ex}: Lord Dunmore's War w/ Shawnee Indians) 
                    \item Raids continued on white settlers, but were often used as pretexts for larger attacks
                    \begin{itemize}
                        \item Massacred those who stood in way of expansion
                        \item Most notable: white militias' slaughtering of peaceful Delaware Indians due to likely false accusation of having murdered a family
                        \begin{itemize}
                            \item Killed 96 people
                        \end{itemize}
                        \item Massacres were not the norm, but revealed growing tensions
                    \end{itemize}
                    \item Revolution showed ultimate defeat of tribes with aggressive westward expansion
                \end{itemize}
            \end{itemize}
            \textbf{The Revolution had greatly negative effects on the natives, who had generally sided with the British due to their belief that native land should be protected to prevent wars. After the war, the native position was severely weakened, with westward expansion rampant and all in the way put down, even those who were peaceful (like the Shawnee).}}
            \cornell{How did women take part in the Revolution?}{\begin{itemize}
                \item Departure of many men to fight in armies -> wives, mothers, sisters, daughters left in power
                \begin{itemize}
                    \item Some left without even a farm to fall back on -> large population of impoverished women who protested, attacked British troops
                \end{itemize}
                \item Many others departed with men, mostly out of necessity (disease, takeover)
                \begin{itemize}
                    \item Inhabited Patriot camps; George Washington feared that they would be "disruptive," despite Martha staying with him 
                    \item Many officers complained about reversal of gender roles (women lived in tandem with men, working hard)
                    \begin{itemize}
                        \item Women were still of significant value as auxiliary services which had not yet been developed in U.S., increasing morale, performing key tasks
                        \item Rough camps often required women to become involved in Combat (\textit{ex}: legendary Molly Pitcher, others who disguised as men)
                    \end{itemize}
                \end{itemize}
            \end{itemize}
            \textbf{Many women were left in positions of power in place of their husbands, which was often disastrous due to their at times lacking even a farm to fall back on - this sparked large women-led urban protests. However, many women also accompanied their husbands to war, often helping as auxiliary services (food, clothing, medicine) and sometimes fighting directly in combat.}
            } 
            \cornell{How were women's rights affected by the revolution?}{\begin{itemize}
                \item \textit{Combat} had little immediate role on female roles; Revolution empowered many women with beliefs taht ideals of liberty extended beyond men
                \begin{itemize}
                    \item Abigail Adams insisted that husband John show generosity for women (modest expansion against tyrannical men)
                    \item Some more radical, like Judith Sargent Murray, leading essayist, who pushed that women's minds were equal to men, deserved full access to education
                    \item Key political leaders like B.F./Benjamin Rush questioned need for feminine reforms; Yale students debated question in 1780s, but few concrete reforms emerged
                \end{itemize}
                \item Unmarried women had some rights (property, contracts), but married had none
                \begin{itemize}
                    \item No property, independent wages, authority over children, right to vote, divorce (in most states)
                    \item Abigail Adams hoped to limit above basic restrictions for women
                \end{itemize}
                \item Revolution had little effect on legal customs (some increased right to divorce, but also setbacks like no dowries for widows -> no support after death of husband)
                \item Patriarchal structure only strengthened; most women accepted existing places; subtle change was slight reevaluation
                \begin{itemize}
                    \item After Revolution, many Americans believed in new type of citizen empowered by liberty; mothers were key to instruct children in important ideals
                \end{itemize}
            \end{itemize}
            \textbf{The Revolution ultimately saw little legal change for women; in fact, patriarchal society was only strengthened by social stability. However, many mothers saw an increase in prestige as they were given the important role of training their children in the principles of liberty.}}
            \cornell{What were the economic impacts of the war?}{\begin{itemize}
                \item With direct hostility from the British, the Americans were forced to fend for themselves and avoid all markets of the empire
                \item In the long term, Revolution had positive effects on the American economy
                \begin{itemize}
                    \item Even before the conclusion of the war, Americans developed maneuverable vessels to evade British navy
                    \item End of imperial restrictions on commerce saw trade flourish with many new areas, like in the Caribbean, South America, and eventually Asia via Cape Horn; trade even grew between states
                    \item English imports cut off -> domestic manufacturing forced to grow (but gradually): "homespun" cloth seen as more desirable to British fabrics
                \end{itemize}
                \item War did not revolutionize economy, but it released entrepreneurial energies
            \end{itemize}
            \textbf{Although the war forced the Americans to find new trading partners, its effects were ultimately positive, seeing a growth in domestic industry, increased trade between formerly restricted nations, and a growth in shipbuilding. However, the economic changes were nowhere near as revolutionary as the 19th century Industrial Revolution.}}
        \cornell[The Creation of State Governments]{What characterized the growth of state governments following the Revolution?}{\textbf{States created constitutions during the war which emphasized a republican system giving all men equal rights and placing true authority in the people. However, they were soon ratified to reduce some of the more radical changes (like reduced executive power). Most states emphasized the need for religious freedom, but the northern and southern states were greatly divided over the issue of slavery.}}
        \cornell{What were the basic tenets of a republican governmental system?}{\begin{itemize}
            \item Meant that all power came from people, meaning success/nature of government depended on nature of citizenry
            \begin{itemize}
                \item If people mostly wealthy planters, government would be safe; if mostly impoverished workers, potentially at risk
                \item Key ideal in American government: independent landowner
            \end{itemize}
            \item D of I emphasized that all men deserved equal chance at government regardless of birth: all success \textit{earnt}
            \item U.S. was never nation entirely of property holders -> many tenets fell short
            \begin{itemize}
                \item White laborers enjoyed some priviliges of citizenship; blacks allowed none
                \item Women remained subordinate 
                \item Although society was more fluid than any European one, birth still determined success to some extent
            \end{itemize}
            \item Main ideals revolutionary -> most admired, studied nation on the planet
        \end{itemize}
        \textbf{The republican government system required that all power came from the people (this meant that the small freeholder's opinion was critical) and that all men were born equal. Although birth still determined the success of all to some extent (especially blacks, women), America was one of the most revolutionary, mobile societies for a long time.}}
        \cornell{What were the characteristics of the initial state constitutions?}{\textbf{Due to the vagueness of England's constitution, all state constitutions were formally written; furthermore, they emphasized reduction in power of the executive (eliminated in PA) and limiting governor power (unable to hold legislature seat, dismiss legislature, veto). However, direct, popular rule was never adopted; GA and PN had one house while all others had an upper ("higher order")/lower chamber.}}
        \cornell{How were state governments revised following the Revolution?}{\begin{itemize}
            \item Initial constitutions made governments unstable, unable to accomplish anything -> constitutions generally ratified to limit popular power
            \item Key differences (starting w/ MA in 1780) included preventing constitutions from being easily amended/violated in favor of special assembly which would most likely never meet again, strengthening of executive to give governor more power (fixed salary, veto, appointment powers)
            \begin{itemize}
                \item Upper houses emerged; PA produced strong executive branch in governor
            \end{itemize}
        \end{itemize}
        \textbf{Because most states were able to accomplish very little, ratifications were made to the constitutions, starting in Massachussetts, which limited popular control over the constitution and gave more power to the executive branch.}}
        \cornell{How did state governments approach the issues of religion and slavery?}{\begin{itemize}
            \item Most states heavily believed in complete freedom, hoping to give no special power to any denomination
            \begin{itemize}
                \item Church powers were stripped away; VA saw Statute of Religious Liberty (Jefferson) \textit{completely} separating church and state
            \end{itemize}
            \item Slavery remained a challenging question
            \begin{itemize}
                \item States with weak slavery saw early abolishment (NE, PA)
                \item Although the south (all but SC and GA) banned further importation of slaves (SC did during the war), slavery survived in all southern/border states
                \begin{itemize}
                    \item Caused by racist assumptions, inability of GW/TJ to provide reasonable alternatives to slavery
                    \item Few whites believed blacks could integrate into American society
                \end{itemize}
            \end{itemize}
        \end{itemize}
        \textbf{Most states encouraged complete religious freedom, often separating church and state. As for slavery, most northern states whose slavery industry was already limited abolished it quite early; all southern/border states kept the industry alive for racist reasons and the inability to find a reasonable alternative.}}
        \cornell[The Search for a National Government]{What was the state of the national government in the U.S.?}{\textbf{Under the Confederation, the national government was extremely weak, unable to enact taxes on states directly or make significant decisions to the Articles of Confederation without the approval of \textit{all states}. In part because of this, they were unable to present a united front to expel remaining British troops. However, they were able to create relatively well-organized land distribution policies in the northwest through a series of ordinances. However, their inability to directly tax meant that they were unable to pay off their war debt, increasing the urgency of a new Constitution.}}
        \cornell{What were the key points of the Articles of Confederation?}{\begin{itemize}
            \item Articles of Confederation hoped to unify states (who mostly viewed themselves as sovereign nations) under common ideals of liberty, making Congress the sole institution of national authority
            \begin{itemize}
                \item Expanded congressional powers to include control over money, foreign relations, war
                \item Unable to levy taxes, draft troops without formal requests to legislatures (who could easily refuse) or regulate trade 
                \item No main executive; "president" was the presiding officer at Congress; each state had a single vote, at least nine had to approve new state while all had to approve any amendment of Articles
            \end{itemize}
            \item Broad disagreements over Articles of Confederation became clear as many states attempted ratification
            \begin{itemize}
                \item Larger states pushed for population-based vote while smaller enjoyed equal votes (this prevailed)
                \item States claiming western lands wished to keep them; rest of them demanded turning over to national government
                \begin{itemize}
                    \item NY/VA forced to give up
                \end{itemize}
            \end{itemize}
            \item Confederation (1781 - 1789) was not a failure, but not successful due to lack of ability to deal with interstate issues
        \end{itemize}
        \textbf{The Confederation hoped to unify all states under one central government with the power to control money, foreign relations, and war; however, it would be unable to levy taxes, draft troops, or regulate trade. This government had no main executive: the "president" was the presiding officer at Congress. Vast disagreements over the Articles of Confederation made clear the many issues and places for improvement.}}
        \cornell{What were the diplomatic failures of the Confederation?}{
            \textbf{Confederation was unable to force the British to evacuate troops from American territory (especially in the north, along the Great Lakes) or to return confiscated slaves to their slaveowners. Furthermore, they were unable to receive permission to trade with the British. When John Adams was sent to London to resolve these issues, he made no headway: the British were unsure of whether he represented a whole nation or thirteen different ones.}}
        \cornell{How did the Confederation handle the Northwest territories?}{
            The biggest accomplishment of the Confederation was their resolution of controversies over western lands: they were forced to deal with over 120,000 whites west of the Appalachians while preventing any one state from claiming too much.
            \begin{itemize}
                \item Ordinance of 1784 divided western territory into ten self-governing districts
                \begin{itemize}
                    \item Able to petition for statehood when population exceeded number of free inhabitants in smallest state 
                \end{itemize}
                \item Ordinance of 1785 formally defined surveying/selling western lands
                \begin{itemize}
                    \item Territory north of Ohio River divided into 36 rectangular townships
                    \item Each township had four sections set off for U.S.; any other section, when auctioned off (starting at a dollar an acre), required revenue to go toward public school
                    \item Set precedent for dividing up land: unlike in the past, w/ natural barriers or landlords' holdings, Enlightenment ideals pushed for mathematical division of land
                    \begin{itemize}
                        \item Model became most common form for Americans to establish ownership over landscape
                    \end{itemize}
                \end{itemize}
                \item Original ordinances favored land speculators, not settlers
                \begin{itemize}
                    \item Congress sold best land to companies rather than settlers
                    \item Resolved with "Northwest Ordinance" -> ten districts created in 1784 were abandoned, single Northwest Territory created out of lands north of Ohio River
                    \begin{itemize}
                        \item Equally divided into 3-5 territories, required 60,000 people for statehood, freedom of religion, no slavery, right to trial by jury
                    \end{itemize}
                \end{itemize}
                \item Lands south of Ohio River never as regulated, leading to creation of TN and KY 
            \end{itemize}
            \textbf{The Confederation approached these territories with multiple ordinances, which emphasized (initially) their division into neat, rectangular townships. However, because this system favored large companies and land speculators over everyday settlers, they created a more fluid system of 3-5 main territories which were eligible for statehood.}
        }
        \cornell{What were the tensions between natives and settlers on the western lands?}{\begin{itemize}
            \item On paper, system of Ordinances seemed to bring great stability; in reality, natives had claimed most of western lands
            \begin{itemize}
                \item Congress attempted to resolve by forming agreements with Iroquois, Choctaw, Chickasaw, Cherokee; ultimately ineffective
                \item Iroquois threatened to attack white settlements; other tribes continued resistance to settlement
            \end{itemize}
            \item Violence peaked in early 1790s w/ group of tribes led by Miami warrior Little Turtle
            \begin{itemize}
                \item Defeated forces in two battles near western Ohio; 630 white Americans died in second battle
                \item Miami would not form agreement with Americans until agreement formed forbidding white settlement west of Ohio River
                \begin{itemize}
                    \item Negotiations resumed when Anthony Wayne led 4,000 soldiers to defeat natives in Battle of Fallen Timbers
                \end{itemize}
            \end{itemize}
            \item Miami finally signed treaty of Grenville, ceding new lands to U.S. in exchange for \textit{formal acknowledgement} of territory which they had fought to retain
            \begin{itemize}
                \item U.S. theoretically affirmed that land could only be ceded by tribes themselves; proved to be frail protection in later years
            \end{itemize}
        \end{itemize}
        \textbf{Westward expansion promoted by the Ordinances saw great violence between settlers and natives, peaking in 1790s with dramatic white American defeat. However, negotiations soon resumed after the natives were defeated, with a contract theoretically acknowledging the control that the western Miami tribe had over their land (this was not much protection, however).}}
        \cornell{What characterized the postwar depression?}{\begin{itemize}
            \item Lasted from 1784 to 1787, further aggravated American issue of inadequate money supply
            \item Congress suffered from enormous debt as a result of war
            \begin{itemize}
                \item Without power to tax, had few means to pay; received only one-sixth of requisitioned money -> faced with prospect of defaulting
            \end{itemize}
            \item Grim prospect pushed new leaders to forefront who would shape republic for many decades
            \begin{itemize}
                \item Nationalist like Robert Morris (head of treasury), Alexander Hamilton (young protégé), James Madison called for duty on imported goods to be levied by Congress
                \item Americans feared that impost plan would concentrate power in central government, leading impost not to be passed
            \end{itemize}
            \item States handled war debts with increased taxations; angry, impoverished farmers demanded paper currency to increase wealth and counteract effects of taxation (especially those in NE, who felt that new policies would strengthen bondholders)
            \begin{itemize}
                \item Late 1780s saw mobs of distressed MA farmers rioting periodically
                \item Many rioters were veterans of the war behind Daniel Shays (former captain)
                \begin{itemize}
                    \item Shays demanded tax relief, postponement of debts, relocation of MA capital from Boston to interior
                    \item Concentrated on reduced debt collection, preventing sheriffs from selling confiscated property
                    \item Shays was denounced by legislature members (even Samuel Adams) as traitors, quickly dispersing their attacks on Springfield in January 1787
                    \item Rebellion was military failure but did produce some concessions as Shays and lieutenants were eventually pardoned and protestors were provided some tax reliefs; added urgency to movement to produce new national constitution
                \end{itemize}
            \end{itemize}
        \end{itemize}
        \textbf{The postwar depression had drastic effects on the national level, with Congress' inability to tax preventing them from paying of their war debt. Furthermore, many of their essential policies were rejected on the state level, making the prospect of defaulting on loans more and more real. States, too, struggled with this issue: although they were able to increase taxation to combat these war debts, they struggled with angry farmers, which culminated in Shays' rebellion.}}
    \end{document}