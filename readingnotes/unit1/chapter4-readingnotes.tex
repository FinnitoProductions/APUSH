\documentclass[a4paper]{article}
    \usepackage[T1]{fontenc}
    \usepackage{tcolorbox}
    \usepackage{amsmath}
    \tcbuselibrary{skins}
    
    \usepackage{background}
    \SetBgScale{1}
    \SetBgAngle{0}
    \SetBgColor{red}
    \SetBgContents{\rule[0em]{4pt}{\textheight}}
    \SetBgHshift{-2.3cm}
    \SetBgVshift{0cm}
    \usepackage[margin=2cm]{geometry} 
    
    \makeatletter
    \def\cornell{\@ifnextchar[{\@with}{\@without}}
    \def\@with[#1]#2#3{
    \begin{tcolorbox}[enhanced,colback=gray,colframe=black,fonttitle=\large\bfseries\sffamily,sidebyside=true, nobeforeafter,before=\vfil,after=\vfil,colupper=blue,sidebyside align=top, lefthand width=.3\textwidth,
    opacityframe=0,opacityback=.3,opacitybacktitle=1, opacitytext=1,
    segmentation style={black!55,solid,opacity=0,line width=3pt},
    title=#1
    ]
    \begin{tcolorbox}[colback=red!05,colframe=red!25,sidebyside align=top,
    width=\textwidth,nobeforeafter]#2\end{tcolorbox}%
    \tcblower
    \sffamily
    \begin{tcolorbox}[colback=blue!05,colframe=blue!10,width=\textwidth,nobeforeafter]
    #3
    \end{tcolorbox}
    \end{tcolorbox}
    }
    \def\@without#1#2{
    \begin{tcolorbox}[enhanced,colback=white!15,colframe=white,fonttitle=\bfseries,sidebyside=true, nobeforeafter,before=\vfil,after=\vfil,colupper=blue,sidebyside align=top, lefthand width=.3\textwidth,
    opacityframe=0,opacityback=0,opacitybacktitle=0, opacitytext=1,
    segmentation style={black!55,solid,opacity=0,line width=3pt}
    ]
    
    \begin{tcolorbox}[colback=red!05,colframe=red!25,sidebyside align=top,
    width=\textwidth,nobeforeafter]#1\end{tcolorbox}%
    \tcblower
    \sffamily
    \begin{tcolorbox}[colback=blue!05,colframe=blue!10,width=\textwidth,nobeforeafter]
    #2
    \end{tcolorbox}
    \end{tcolorbox}
    }
    \makeatother

    \parindent=0pt
    \usepackage[normalem]{ulem}

    \newcommand{\chapternumber}{4}
    \newcommand{\chaptertitle}{The Empire in Transition}

    \title{\vspace{-3em}
\begin{tcolorbox}
\Huge\sffamily \begin{center} AP US History  \\
\LARGE Chapter \chapternumber \, - \chaptertitle \\
\Large Finn Frankis \end{center} 
\end{tcolorbox}
\vspace{-3em}
}
\date{}
\author{}
    \begin{document}
    \maketitle
    \SetBgContents{\rule[0em]{4pt}{\textheight}}
    \cornell[Key Concepts]{What are this chapter's key concepts?}{\begin{itemize}
        \item British government increasingly attempted to incorporate North American colonies into coherent, imperial structure with mercantilist aims; failed due to conflict w/ colonists and natives 
        \item Colonists drew on local self-government to resist imperial control: liberty began to grow as a major concept
        \item Colonial rivalry between Britain and France intensified, threatening French-native trade networks and nativr autonomy
        \item Britain defeated the French, expanding territorial holdings
        \item Imperial officials attempted to prevent westward expansion - natives hoped to continue trading with Europeans while resisting colonial expansion 
        \item Colonial leaders based calls for resistance on arguments about rights of British subjects
        \item American independence effort pushed by powerful leaders like Franklin
    \end{itemize}}
    \cornell[Loosening Ties]{What changes marked the weakening of ties between the British mainland and overseas colonies?}{
        \textbf{British Parliamentary control saw limited rule over the colonies; the London Privy Council, too, as well as royal governors, were often bribed and controlled by colonial officials. In part because of this, the colonies remained divided, viewing each other as foreigners.}
    }
    \cornell{What was the overall context behind England's reduced attention to the colonies?}{\textbf{The British government made no serious effort to govern the colonies due to great divisions within Parliament concerning the extent of interference. Although many colonies soon became royal colonies and some new laws were implemented (mainly economic), colonies were generally left to diverge from British rule.}}
    \cornell{What, specifically, gave the colonies freedom to diverge from British rule?}{\begin{itemize}
        \item Reigns of German-born George I, George II saw monarchical alienation; prime minister/cabinet ministers became true executives
        \begin{itemize}
            \item Parliamentary leaders depended on landholders and merchants, leading them to keep loose control over colonies to reduce expenditure
            \item Navigation Act often ignored to stimulate commerce
        \end{itemize}
        \item Day-to-day administration decentralized: some power in Board of Trade and Plantations, an advisory body; real power in Privy Council
        \begin{itemize}
            \item Focus on mainland affairs led to reduced concentration on colonial affairs
            \item Most London officials had little experience with America: only knowledge came from overseas agents (who rarely encouraged interference)
            \item Royal officials (mostly governors) in America generally succumbed to bribery, favoritism
            \begin{itemize}
                \item Often hired substitutes to take places in America with low wages, encouraging bribery(\textit{ex}: customs collectors often waived duties if paid directly)
            \end{itemize}
        \end{itemize}
        \item Main, deliberate resistance to authority seen in colonial legislatures
        \begin{itemize}
            \item Gave themselves right to levy taxes, pass laws with potential veto by Privy Council
            \item Often found loopholes such as slight changes to laws; leverage over Privy Council due to control of budget gave little power to Council itself
        \end{itemize}
    \end{itemize}
    \textbf{Decentralized control was rooted both in mainland England and the colonies. The Parliament's control over English politics and their ties with merchants meant that rarely would laws be enforced for economic reasons; furthermore, the London Privy Council had little overall control due to frequent bribery of royal officials and the legislatures' leverage over them.}}
    \cornell{Did the colonies began to converge into one cohesive unit?}{\begin{itemize}
        \item Many colonists felt greater ties to England than to each other; nearby colonies viewed each other as foreigners tied only by geography
        \item Economic connections often emerged especially with roads along coast allowing for intercolonial trade, postal service promoting communication
        \item Cooperation was often challenging even amidst threat of adversaries
        \begin{itemize}
            \item Threat from French/native allies led to Albany Plan
            \begin{itemize}
                \item Treaty with the Iroquois and a potential colonial federation with one general, elected government for native relations
                \item Allowed present constitutions to remain intact
            \end{itemize}
            \item Albany Plan approved by no colonial assemblies
        \end{itemize}
    \end{itemize}
    \textbf{No - despite some economic ties, the colonists generally remained split by regional differences. Even in the threat of their French adversaries, all legislatures refused to cooperate under the proposed Albany Plan.}}
    \cornell[The Struggle for the Continent]{What were the major struggles for control over the American continent?}{\textbf{The most significant struggle was the French and Indian War, preceded by smaller conflicts between the British, French, and Iroquois over the large territories held by each group. It began with Washington's invasion of Fort Duquesne and was split into three phases: one of British disorganization and little assistance, one of colonial assistance and a connection to Europe but aggravated tensions between the British and the colonists, and the third with the tide turned toward the British and a victory which provided large swathes of land to the English and Spanish but sparked more conflicts between the British and the colonists.}}
    \cornell{What was the background to the French and Indian War?}{\textbf{A global war, the French and Indian War saw a rearrangement of global power with England at the head. However, the conflict was long-coming, with an uneasy balance of power between natives (particularly the Iroquois), the French, and the English colonists finally settled.}}
    \cornell{What were the basic traits of New France and how were the natives involved?}{\begin{itemize}
        \item French/English had coexisted peacefully, but religious/commercial tensions produced new conflicts
        \begin{itemize}
            \item Expansion of French presence under Louis XIV with growing fur trade, presence of French Jesuits, farmers travelling south from Canada 
        \end{itemize}
        \item New France comprised vast territory, with explorers claiming Louisiana (named for King Louis); subsequent explorers reached as far south as Rio Grande
        \begin{itemize}
            \item Territory held together only by widely separated communities, fortresses, trading posts 
            \item In north, fortified city of Quebec with Montreal to south; Mississippi plantations further south worked by black slaves; Louisiana contained large cities like New Orleans, Mobile
            \item Large territory shared with natives, English settlers 
            \begin{itemize}
                \item French/English both understood importance of forging relations with native tribes for power
                \item Natives concerned with protecting independence; often marrying for convenience
                \begin{itemize}
                    \item English attracted natives with advanced goods, but forced to conform to standards
                    \item French adjusted behavior to native patterns, marrying women and adopting tribal ways; Jesuits allowed Catholicism
                \end{itemize}
                \item Iroquois confederacy (Mohawk, Seneca, Cayuga, Onondaga, Oneida) had unique relationship with both groups
                \begin{itemize}
                    \item Economic relations with English/Dutch and some with French
                    \item Avoided allying to closely with either side, often playing both groups against each other
                \end{itemize}
                \item Ohio Valley location of principal conflict; claimed by French but natives hoped to control, too (especially Iroquois)
            \end{itemize}
        \end{itemize}
    \end{itemize}
    \textbf{New France expanded rapidly after the intervention of Louis XIV, but remained fragmented and held together by distant (yet often large communities). The territory was shared with the natives: unlike the English, the French often adopted native ways through marriage without infringing on their rights and desires. One key group allied with neither the English nor the French: the Iroquois, who, for economic reasons, remained neutral.}}
    \cornell{What were the major initial conflicts between the English and the French?}{\begin{itemize}
        \item Greatly dependent on relations between English/French throne
        \begin{itemize}
            \item Aggravated after Glorious Revolution brought Louis XIV's enemies into power (opposed expansionism)
            \item Continued as successor of William III (Anne), continued struggle against Spain, French ally 
            \item King William's War saw a few small clashes in NE
            \item Queen Anne's War saw conflicts between southern Spaniards and northern French
            \begin{itemize}
                \item Resolution to Queen Anne's War (Treaty of Ulrecht) gave substantial French territory to English (including Newfoundland, Acadia)
            \end{itemize}
        \end{itemize}
        \item More conflicts followed, most notably King George's War
        \begin{itemize}
            \item Based around conflicting trading rights of English settlers in Spanish territories
            \item Merged with larger war: on opposite sides of conflicts between Austrians and Prussians
            \item War ended after New Englanders captured French bastion at Louisbourg on Cape Breton w/ peace treaty forcing abandonment
        \end{itemize}
        \item Aftermath led to deterioration between French-English-Iroquois relations
        \begin{itemize}
            \item Iroquois granted trading concessions to English merchants, leading the French to fear English goal for expansion
            \item French created more fortresses in Ohio Valley, leading the English to interpret activity as threat to western settlements, building their own fortresses
            \item Iroquois power balance had collapsed, forcing them to ally with the English
            \item Tensions increased in following five years, starting with 1754 Virginia militia into Ohio Valley under command of George Washington
            \begin{itemize}
                \item Attack unsuccessful: Fort Necessity (crude stockade constructed by Washington) raided with more than $\frac{1}{3}$ of men killed 
                \item Washington forced to surrender
            \end{itemize}
        \end{itemize}
    \end{itemize}
    \textbf{The initial conflicts were dependent on French-English relations in Europe, most notably Queen Anne's War, which involved the Spanish, an ally of the French. King George's War, too, was significant, representative of a greater war in Europe between the Prussians and the Austrians. After King George's War, Iroquois relations deteriorated due to a series of misinterpretations. The French and Indian War began with Washington's attack on Fort Duquesne in modern Pittsburgh.}}
    \cornell{What defined the first phase of the French and Indian War?}{The war lasted nearly nine years, and can be divided into three phases. The first phase began with Washington's attack on Fort Desquene and ended when the Europeans began to become directly involved.
    \begin{itemize}
        \item British initially provided modest assistance but was poorly organized, failing to prevent more French fleets from arriving and a British army failing in their attempt to take over the Fort Necessity area
        \item Local forces preoccupied with defense on western front, defending against \textit{all} native tribes but Iroquois (viewed English loss as weakness)
        \item Iroquois themselves feared French - despite having declared war, made few advances on Canada
    \end{itemize}
    \textbf{During the first phase, British assistane was limited and disorganized, local forces were unable to focus on anything but the western front, and the Iroquois, the only native alliance of the English, were too afraid of France's power to fight directly.}
    }
    \cornell{What defined the second phase of the French and Indian War?}{The second phase opened in 1756 when governments of England and France directly began hostilities with one another.
    \begin{itemize}
        \item Marked beginning of British Seven Years' War, where France allied with Austria, former enemy, and England allied with Prussia, former French ally
        \item Fighting had reached West Indies, India, Europe; primary struggle remained in North America
        \item English had been continually defeated until intervention of William Pitt, secretary of state
        \begin{itemize}
            \item Planned advaned military strategy with forced enlistment ("impressment"), seizure of supplies from farmers, tradesmen; demanded complementary shelter for troops 
            \item Greater independence of Americans led to stark resistance to Pitt's advancements
            \begin{itemize}
                \item 1757: major riot in New York City threatened to bring English war effort to conclusion
            \end{itemize}
        \end{itemize}
    \end{itemize}
    \textbf{The second phase was when the war truly became global (while remaining centered in North America). It was defined by great internal conflicts between the English (under William Pitt) and the colonists due to controversial policies like impressment and required hospitality toward troops, greatly weakening the English resistance.}
    }
    \cornell{What defined the third phase of the French and Indian War?}{
        The third phase was initiated by William Pitt in relaxing many oppressive policies, with compensation for colonist supplies and new troops.
        \begin{itemize}
            \item Tide began to favor England, partly due to many troops and partly due to poor French harvests
            \item Key generals Jeffrey Amherst, James Wolfe captured fortress at Louisbourg; defeated Montcalm in Quebec in 1759 where both commanders died, fell dramatically
            \item Many other aspects of the war far more brutal
            \begin{itemize}
                \item Uprooted Nova Scotians (Acadians), forcing migration throughout English colonies due to suspected disloyalty
                \item Offered "scalp bounties" for natives 
                \begin{itemize}
                    \item French/native allies retaliated with massacre of hundreds of English families
                \end{itemize}
            \end{itemize}
        \end{itemize}
        \textbf{The third phase was initiated by greater cooperation between the colonists and the English, with many successful battles for the England (including the graceful defeat in Quebec, but it was also marked by great brutality on both sides, including practices of scalping and forced relocation.}
    }
    \cornell{What was the aftermath of the Seven Years' War?}{\begin{itemize}
        \item Peace came after George III reached throne, Pitt resigned (he had hoped to continue hostilities)
        \begin{itemize}
            \item French ceded West Indies islands, most of Indian colonies, Canada to GB; all territory east of Mississippi but New Orleans to Spain in Peace of Paris
        \end{itemize}
        \item War had profound effects on various involved groups
        \begin{itemize}
            \item War gave Britain many new territories while enlarging debt
            \begin{itemize}
                \item Led British leaders to house resentment for Americans due to few financial contributions (even some merchants continued selling goods to French)
                \item Many leaders felt reorganization was essential
            \end{itemize}
            \item For American colonists, began to feel sense of unity against common foe 
            \begin{itemize}
                \item Established British illegitimacy: viewed war as voluntary, communal while British regulars viewed as hierarchal and coercive
            \end{itemize}
            \item Natives of Ohio Valley saw British victory as disastrous: allegiance to French led to English enmity
            \begin{itemize}
                \item Iroquois's passivity angered colonists, leading to unraveling of alliance, loss of control over territories
            \end{itemize}
        \end{itemize}
    \end{itemize}
    \textbf{Peace was found after Pitt resigned; it entailed the cession of the majority of French land to the English and Spanish. The war caused the British and the colonists to feel greater mutual hostility and the natives to completely unravel due to the enmity of the British.}}
    \cornell[The New Imperialism]{What marked Britain's decision to take a greater role in governing the colonies?}{\textbf{The British suffered from many major burdens after the war, including a redesigned imperial structure, challenges with larger area, a large war debt, and struggles with George III, an ill-fitted monarch. Grenville, a new prime minister, hoped to seize control over the colonies with many major changes, which included greater support of the natives to control colonial westward expansion and also many major economic changes. Although the colonists were initially unable to resist due to local divisions, they were eventually unified by an economic boon and the ideals of self-government.}}
    \cornell{What were some political burdens of Britain's large empire?}{
    \begin{itemize}
        \item Britain initially hesitant to intervene in colonies due to great conflicts during war between Pitt and colonists 
        \item Major 1763 shift in imperial design led to further challenges
        \begin{itemize}
            \item British initially viewed colonies as key to trade; many leaders argued the land was the true value for taxes, imperial splendor 
            \begin{itemize}
                \item Mercantilists, at conclusion of Seven Years' War, hoped to return Canada to France for Guadeloupe, productive sugar island
                \item Franklin felt larger territories were critical for limitless growth
            \end{itemize}
        \end{itemize}
        \item Area had grown twice as large by 1763, posing major challenges
        \begin{itemize}
            \item Some British sought restrained settlement due to potential conflict with natives if settlemetn was rapid
            \item Many colonists wanted to expand existing colonies westward into new lands; others felt new colonies should emerge
        \end{itemize}
        \item Accession of George III led to further issues
        \begin{itemize}
            \item Strongly determined to increase power of monarch (pressured by mother)
            \begin{itemize}
                \item Removed stable Whig coalition from Parliamentary power, creating new bribed coalition with uneasy control over Parliament; very unstable 
            \end{itemize}
            \item Intellectual/physcological limitations led to major political difficulties
            \begin{itemize}
                \item Disease produced intermettent bouts of insanity
                \item Often behaved irrationally: ill-fitted for role but immaturely constantly attempted to show fitness
            \end{itemize}
        \end{itemize}
    \end{itemize}
    \textbf{British political challenges included a new shift in imperial design emphasizing land over trade,and a large growth in area leading to internal and external administrative conflicts concerning the nature of expansion. Furthermore, George III, an unstable monarch, hoped to be active as a ruler but was generally ill-fitted for the role.}
    }
    \cornell{What were Britain's major economic burdens and how did they address them?}{\begin{itemize}
        \item War debt of England could not be improved
        \begin{itemize}
            \item Landlords/merchants objected to tax increases
            \item Necessity to station troops on American-Indian borders led to further costs
            \item Little response of colonial assemblies in war efforts gave Britian little reason to rely on cooperation
            \item Believed that only solution would be new system of taxation
        \end{itemize}
        \item George Grenville, new prime minister, disagreed with Pitt: felt colonists had been long overindulged, required to pay part of cost
        \begin{itemize}
            \item Hoped to implement major control over America
        \end{itemize}
    \end{itemize}
    \textbf{The significant war debt created by the Seven Years' War was one very difficult to recover from: landlords (who were closely tied to Parliament)rejected tax increases, and the need to maintain troops in North America further aggravted it. The only solution, especially in the eyes of George Grenville, Britain's new prime minister, was taxation and control over the colonies.}}
    \cornell{What were the relations between the natives and the British?}{\begin{itemize}
        \item After French departed Ohio Valley, British immediately streamed in, angering natives
        \begin{itemize}
            \item Ottawa chieftain Pontiac struck back against British; war ended immediately by government to prevent major conflicts 
            \item Proclamation of 1764 prevented expansion beyond Appalachians
            \begin{itemize}
                \item Allowed London to control westward movement in hopefully orderly manner
                \item Allowed coastal colonies to remain strong, promoting English interests
            \end{itemize}
        \end{itemize}
        \item Tribes generally discontent with Proclamation due to required cession of more land
        \begin{itemize}
            \item Many accepted as only option, especially Cherokee 
            \item Relations improved with natives in large part due to British-appointed superintendents Stuart (south) and Johnson (north)
            \begin{itemize}
                \item Sympathetic to natives, living among tribes
            \end{itemize}
            \item Proclamation ultimately failed to meet most needs of natives
            \begin{itemize}
                \item Line of settlement never enforced, with settlers continually expanding beyond despite restrictions by British authorities
            \end{itemize}
        \end{itemize}
    \end{itemize}
    \textbf{The British government issued Proclamation of 1764, restricting expansion beyond the Appalachians, in an attempt to appease the natives and take direct control over westward expansion. However, most tribes were discontent as it required cession of more land and it was poorly enforced.}}
    \cornell{What changes did Grenville implement to tighten his hold over the colonies?}{\begin{itemize}
        \item Required permanent troops in colonies, mandatory colonial assistance in maintaining armies, sent ships to patrol for smugglers 
        \item Manufacturing restricted as to not compete with expanding industries of Britain
        \begin{itemize}
            \item Sugar Act of 1764 strengthened duty enforcement on sugar (limiting illegal sugar trade), establishing new courts for smugglers
            \item Currency Act of 1764 banned paper money
            \item Stamp Act of 1765 issued new taxes on printed documents
            \item Ultimate goal to strengthen mercantilism in colonies
        \end{itemize}
    \end{itemize}
    \textbf{Grenville's initial changes involved the requirement for permanent stationing of troops within colonies, mandatory colonial assistance for army maintenance (Mutiny Act), and ships to patrol for smugglers. Furthermore, numerous policies limited colonial commerce as to not threaten British industry.}}
    \cornell{What caused the colonists' initial resistance to Grenville to fail?}{
        Major tensions between the colonies themselves prevented a rapid revolt against the newly implemented policies.
        \begin{itemize}
        \item Primary tensions between "backcountry" and coastal societies w/ isolation and underrepresentation
        \begin{itemize}
            \item Western societies far closer to native tribes, often left unable to defend by government
        \end{itemize}
        \item Civil war in North Carolina emerged
        \begin{itemize}
            \item Caused by Regulator movement: farmer movement to oppose high taxes collected by local sheriffs
            \item Failed with local assembly, so began resisting by force; suppressed by governor William Tyron with army of militiamen
            \begin{itemize}
                \item Hanged seven dissidents for treason; nine killed on each side
            \end{itemize}
        \end{itemize}
    \end{itemize}
    \textbf{Initially, the colonies were not sufficiently cohesive to resist Grenville's changes. The main tensions were between the "backcountry" and coastal societies due to underrepresentation, and a major civil war emerged in NC due to the opposition to local sherrifs' high tax collection.}}
    \cornell{What marked the cohesive colonial resistance to Grenville's policies?}{
    Tensions with British began to overshadow divisions between colonies, forcing them to unite against a common enemy.
    \begin{itemize}
        \item Everyone felt antagonized by Grenville program
        \begin{itemize}
            \item Commercial restraints, increased taxes angered northern merchants
            \item Northern backcountry angered by reduced western expansion
            \item Southern planters feared additional taxes
            \item Professionals depended on merchants/planters for livelihood
            \item Small farmers feared abolition of paper money
        \end{itemize}
        \item New restrictions arrived in British attempt to restrict postwar debt
        \begin{itemize}
            \item After having poured money into colonies for wartime success, bust followed with declined standard of living
        \end{itemize}
        \item Most Americans circumvented policies; economy not destroyed
        \item Political consequences viewed as most damaging with Enlightenment ideals of self-government threatened
        \begin{itemize}
            \item Home rule was seen as an essential trait of colonial life
            \item Movement democratic and conservative: to conserve existing liberties
        \end{itemize}
    \end{itemize}
    \textbf{Nearly all colonists felt threatened by Grenville's policies with a close connection between all levels of society. Although the policies themselves were not devastating to the economy, the political consequences of reduced self-rule truly angered many Americans, forcing them to unite against a common enemy.}}
    \cornell[Stirrings of Revolt]{What were the major fundamental causes of revolt in the colonies?}{\textbf{Colonial revolt was sparked for multiple reasons, including the Stamp Act, which represented an infringement on American liberty,the Townshend Program, which raised taxes greatly and saw an increase in customs inforcement, and the Boston Tea Party, which saw Bostonians revolt against reduced taxes on British tea. Philosophically, revolt emerged from fundamental differences in American and British perspectives on sovereignty.}}
    \cornell{What was the context to and true economic burden of the Stamp Act?}{
        \begin{itemize}
            \item Sugar Act from a year earlier had little effect on non-merchant colonists, but Stamp Act tax increased taxes for all Americans
            \begin{itemize}
                \item Opposition from most powerful members of population, especially tavern owners and printers (who were thereafter required to purchase stamps for business operation)
            \end{itemize}
            \item Economic burden relatively light; true issue was precedent
            \begin{itemize}
                \item Most Americans had seen previous taxes as intended to regulate commerce; many believed that Sugar Act (despite true purpose) was simply part of imperial duties 
                \item Stamp Act could only be interpreted as direct grab for wealth; many believed it would open the door to more burdensome taxes
            \end{itemize}
        \end{itemize}
        \textbf{The Stamp Act was seen as a truly "internal" conflict (purely for wealth of British, not success of commerce), which angered many Americans. Although the act itself had little effect on the American economy, the idea of greater regulation led many to fear for the future.}
    }
    \cornell{What was the initial resistance to the Stamp Act?}{\begin{itemize}
        \item Initial resistance was light (what could they do?) until House of Burgesses inspired great sedition among colonists
        \begin{itemize}
            \item Collective voice of younger aristocrats hoped to challenge power of planters 
            \item Foremost was Patrick Henry, having already achieved fame for oratory, defiance against British authority
            \begin{itemize}
                \item In 1765, made vague threat to decapitate George III
                \item Introduced set of revolutions that Americans and British possessed equal rights, including taxation only by representatives
                \item Virginians, like other Brits, should pay no taxes apart from those induced by their own representatives
            \end{itemize}
            \item Burgesses rejected most extreme of resolutions, but kept most as "Virginia Resolves"
        \end{itemize}
        \item In Massachusetts, James Otis persuaded for intercolonial congress to take action against new tax
        \begin{itemize}
            \item Met in NY with delegates from 9 colonies, petitioned king/Parliament with emphasis on rightful subordination but insisted that all taxes come from provincial assemblies
        \end{itemize}
        \item Many cities began to resist, like a 1765 riot in Boston where men in "Sons of Liberty" terrorized stamp agents and burned stamps
        \begin{itemize}
            \item Riots caused agents to hastily resign, ceasing sale of stamps in continental colonies
            \item Pillagers attacked pro-British "aristocrats" like lieutenant governor Hutchinson, pillaging and destryong his house
        \end{itemize}
    \end{itemize}
    \textbf{The initial resistance was relatively light, but started at Virginia's House of Burgesses with the treasonous Patrick Henry and continued in Massachussetts, both of whom believed that taxes should be implemented not by the government but by provincial assemblies. Furthermore, many cities began to resist, eventually causing the circulation of stamps to stop completely in the colonies.}
    }
    \cornell{What was the conclusion of the Stamp Act crisis?}{\begin{itemize}
        \item Crisis subsided after England backed down, w/ London authorities changing attitude due to economic pressure (from colonial boycotts by most citizens, where those loyal to Britain were pressured by Sons of Liberty)
        \begin{itemize}
            \item English merchants begged Parliament to repeal; English ports began to experience great poverty
        \end{itemize}
        \item Stamp Act finally ended by Marquis of Rockingham, Grenville's successor
        \begin{itemize}
            \item Parliament repealed by 1766, but Rockingham's opponents demanded that Parliament had to assert control
            \item Passed Declaratory Act, asserting authority over colonies in all cases (but few colonists noted)
        \end{itemize}
    \end{itemize}
    \textbf{After England finally backed down due to major boycotts threatening commerce, the colonists were mostly appeased. However, the Declaratory Act, which asserted complete authority, was mostly ignored in the colonists' time of jubilance.}}
    \cornell{What were examples of internal rebellions in the colonies?}{\textbf{Other significant \textit{internal} conflicts emerged in the 1760s, based on the class system in NY and NE. The small farmers who had long been tenants of rented land owned by larger farmers began to demand ownership, ceasing rent payments. It was mostly a failure, but in the NY-governed Vermont, an entirely new state broke off.}}
    \cornell{What was the aftermath of the Marquis of Rockingham's repeal of the Stamp Act?}{
        \textbf{Rockingham's appeasement program greatly angered English merchants, who feared that taxes would have to increase on them. Thus he was dismissed and replaced with the aging yet powerful Pitt.}
    }
    \cornell{What were Townshend's main policies?}{\begin{itemize}
        \item Pitt's poor health (gout/mental illness) gave majority of power to chancellor, Charles Townshend
        \begin{itemize}
            \item First major challenges: Parliamentary grievances of Americans Mutiny Act of 1765, forcing colonists to provide quarters/supplies for troops
            \begin{itemize}
                \item Troops stationed for defense of colonists -> British felt price was fair
                \item Colonists not angry about quartering but instead \textit{mandatory nature}
                \item NY/MA refused mandated supplies (despite army HQ in NYC)
            \end{itemize}
            \item Townshend made two key changes to enforce law, raise revenues
            \begin{itemize}
                \item Singled out NY, disbanded assembly to ensure not all colonies were aroused at once
                \item Levied higher taxes on goods imported from England (lead, paint, paper, tea)
                \begin{itemize}
                    \item Colonists could not reject in his eyes due to their seeming acceptance of "external" taxes
                    \item Taxes equally unacceptable to Stamp Act
                \end{itemize}
            \end{itemize}
        \end{itemize}
        \item Townshend implemented new customs commissioners board to reduce corruption
        \begin{itemize}
            \item In some sense, goals were met: smuggling ended entirely in Boston
            \item Boston merchants very angry about strict enforcement of Navigation Acts, led another boycott
            \begin{itemize}
                \item Later joined by merchants of Philadelphia, NYC
                \item English luxuries quickly fell out of favor
            \end{itemize}
        \end{itemize}
    \end{itemize}
    \textbf{Townshend, the true ruler under Pitt, angered colonists by singling out NY in disbanding assembly and levying higher taxes on most goods imported by England. Furthermore, he brought in customs commissioners to ensure that economic laws were carefully followed.}
    }
    \cornell{How did the colonists respond to the Townshend program and what were the actions of his successor?}{\begin{itemize}
        \item Townshend's major changes viewed as assault on government, annihilation of rights
        \begin{itemize}
            \item Massachussetts Assembly started with letter to all colonial governments to stand up against every Parliamentary tax 
            \item Letter had little effect w/ some opposition but also large fear with threat of dissolution from England, but MA stood by
        \end{itemize}
        \item Townshend suddenly died in 1767; solution fell onto Lord North, new PM
        \begin{itemize}
            \item Repealed all acts but tea taxes in 1770
        \end{itemize} 
    \end{itemize}
    \textbf{The colonists viewed Townshend's programs as a complete infringement on their rights, particularly the Massachussetts Assembly (who failed to petition support from other colonies due to fear). After Townshend's death, Lord North took over, repealing all but the tea taxes.}}

    \cornell{What was the Boston Massacre?}{\begin{itemize}
        \item Withdrawal of Townshend Duties unable to pacify colonial opinion; major event in MA before news of repeal reached America brought revolt to new intensity
        \item Bostonians had been harassing customs commissioners in Boston -> British government placed troops ("redcoats") within city to monitor
        \begin{itemize}
            \item Low pay meant they often competed with Bostonians for jobs
        \end{itemize}
        \item In March 1770, dockworkers pelted sentries at customs house w/ rocks/snowballs
        \begin{itemize}
            \item Significant scuffling with shots fired by British soldiers, killing many
            \item Events seen as "Boston Massacre" by colonists, perpetuated by Paul Revere's artistic depiction as organized attack on peaceful crowd
        \end{itemize} 
        \item In aftermath, two soldiers found guilty of manslaughter but six others acquitted; many felt that all deserved punishment
        \begin{itemize}
            \item Older Samuel Adams with significant ties to NE's Puritan past; failure in business, spoke of corruption/sin of Britain, attracting large following and becoming first head of "committee of correspondence"
            \item Other colonies followed Adams' footsteps with large network of dissent organizations forming
        \end{itemize}
    \end{itemize}
    \textbf{The Boston Massacre was the result of Bostonians' anger at British troops being directly stationed in the city, leading dockworkers to pelt the sentries with rocks and snowballs. After thr British shot at the crowd in defense, many Americans felt their actions were unnecessary, further aggravating dissent (like w/ Samuel Adams' committee of correspondendce).}}
    \cornell{What was the main philosophy behind the revolutions?}{\begin{itemize}
        \item Temporary calm emerged for about three years after massacre; colonial political structure soon began to shift to justify revolt
        \item Revolutionary ideals emerged from numerous sources, including religious (mostly Puritan) experiences, ideas from abroad like from the Scots about GB, and the country westerners who felt left out of political structure
        \item New concept of true meaning of government emerged based around ideal of inherent corruption of humans
        \begin{itemize}
            \item English Constitution long considered best solution to provide safeguard for inherently corrupt officials with fair power distribution between three elements (monarchy, aristocracy, common) for checks and balances
            \begin{itemize}
                \item Americans felt single center of power (king/monarchy) had begun to grow too strong
                \item American arguments found little sympathy: general practice was to inscribe constitutions while English used vague system always subject to change
            \end{itemize}
            \item Idea of "taxation without representation" foreign to most English: believed that Parliament represented not individuals but the entire empire (even if not all counties had elected representatives)
            \begin{itemize}
                \item Americans felt, due to personal colonial assembly meetings, that direct representation was key (they had none) 
            \end{itemize}
        \end{itemize}
        \item Contrasting ideals reflected fundamental differences between American/English view on sovereignty
        \begin{itemize}
            \item Americans sought \textbf{division of sovereignty}, with some power to Parliament but majority of power given to local assemblies
            \item British felt all government required single, ultimate authority
        \end{itemize}
    \end{itemize}
    \textbf{The key pholosophical differences between the British and the Americans were based fundamentally around the difference between uniform sovereignty with a single, ultimate power and divided sovereignty, with most power given to state-based assemblies. Americans drew their revolutionary ideals both from their own experiences (need for representation) and from other radicals, like the Scots.}}
    \cornell{What was the major initial dissent in the colonies?}{\begin{itemize}
        \item Growing dissent despite calm start to 1770s; customs commissioners seen as inept and arrogant
        \begin{itemize}
            \item Often levied unfair taxes, illegally seized merchandise
        \end{itemize}
        \item Writing/talking quickly spread dissent, including discussions in towns, churches, and \textbf{taverns}
        \begin{itemize}
            \item Ordinary people began to absorb new ideas
        \end{itemize}
        \item Direct forms of rebellion emerged
        \begin{itemize}
            \item Colonists seized British revenue ship on Delaware River
            \item Angry Rhode Islanders boarded and set fire to British ship \textit{Gaspée}; British sent commission to America to return perpetrators for British trial
        \end{itemize}
    \end{itemize}
    \textbf{Initial forms of descent in the British colonies spread through written and spoken means, particularly in taverns. Numerous direct rebellions emerged, including the incendiary torching of a British ship in RI.}}
    \cornell{What was the Boston Tea Party?}{
        \begin{itemize}
        \item Most significant revival of revolutionary fervor sparked by business of selling tea
        \begin{itemize}
            \item EIC was on the verge of bankruptcy with large stocks of tea
            \item British authorized direct export of tea without any navigation taxes, allowing for monopoly
            \item Enraged colonists for many reasons
            \begin{itemize}
                \item Merchants feared replacement due to monopoly; certain ones were excluded in exclusive EIC selection of merchants
                \item Revived anger for taxation without representation: EIC was not required to pay standard customs
            \end{itemize}
            \item Response: boycott of tea entirely which mobilized entire population in mass protest, particularly women, who led effort
            \begin{itemize}
                \item Women had always been key players in resistance activities (\textit{ex}: Mercy Otis Warren, with satirical plays)
                \begin{itemize}
                    \item Formed "Daughters of Liberty" in 1760s, who often mocked men for insufficient fervor; challenged tea in 1773
                \end{itemize}
                \item Plans emerged to prevent cargoes of EIC from arriving in ports 
                \begin{itemize}
                    \item In Philadelphia/NYC, tea never left ships; in Charles Town, stored in public warehouses
                    \item Boston, failing to prevent ships, saw group of Patriots break into three ships and toss tea chests into harbor 
                \end{itemize}
            \end{itemize}
        \end{itemize}
    \end{itemize}
    \textbf{The Boston Tea Party, was part of a larger group of states' resistance to Britain's reduced taxes on their large tea supply from India travelling to America, where Bostonians stormed British boats, throwing tea into the harbor.}}
    \cornell{What was the aftermath of the Tea Party?}{\begin{itemize}
        \item Aftermath of Tea Party, due to Boston's denial to pay for property damages, saw Coercive Acts in MA
        \begin{itemize}
            \item Reduced self-government, closed port, tried officers in other colonies
            \item Sparked new resistance along entire coast, with legislatures installing numerous resolves to support Massachusetts
            \item Women's groups continued to boycott British goods, creating popular substitutes
        \end{itemize}
        \item Quebec Act, giving long overdue civil government and rights to French Roman Catholic inhabitants of Quebec (and slightly south) seen as threat by English-speaking colonies
        \begin{itemize}
            \item Feared bishop would be appointed to push Anglican authority on all; or, with the shrinking line between Anglicanism and Catholicism, the order of the pope being returned to America
        \end{itemize}
    \end{itemize}
    \textbf{When they refused to pay for the property damage, Massachusetts saw its rights severely restricted, but this only further mobilized resistance within the colonies. Furthermore. the Quebec Act, giving more rights to the French Catholics in Quebec, inspired fear in many Americans concerning the return of Catholicism.}}
    \cornell[Cooperation and War]{What factors led to the beginning of the Revolution?}{\textbf{The Revolution was impacted significantly by developing sources of authority, including those within the colonies beginning to simulate closely the British Parliament but also the Continental Congress, a meeting of all colonial leaders to make key decisions about Britain, including an economic boycott. The battles began at Lexington and Concord with General Gage's troops sent on a guerilla mission to confiscate gunpowder but being met with trained minutemen.}}
    \cornell{What were the emergent local sources of authority before the Revolution?}{
        On the local level, authority transitioned slowly from the royal government to the colonists themselves.
        \begin{itemize}
            \item Local institutions began to seize authority independently, forming \textbf{extralegal bodies} to simulate government
            \item In MA, Samuel Adams (1768) called convention of delegates to replace formerly dissolved General Court
            \begin{itemize}
                \item Sons of Liberty, also organized by Adams, became another significant source of power 
                \item Prominent citizens began to meet frequently to perform major political functions
                \item Most effective: committees of correspondence
            \end{itemize}
            \item Virginia later established intercolonial committees of correspondence, encouraging mutual cooperation between colonies
            \begin{itemize}
                \item Took major step of united action in 1774 after declaring that Intolerable (Coercive) Acts put all colonies at risk, calling all colonies together (except for Georgia)
            \end{itemize}
        \end{itemize}
    \textbf{Locally, authority began to become more and more distant from the crown as new Parliament-groups began to form in each colony, particularly in Massachusetts under Samuel Adams. This emergence of local authority culminated in Virginia's intercolonial committees, which ultimately led to significant unity between the colonies.}}
    \cornell{What were the first signs of intercolonial unity pre-Revolution and how did Americans react to their decisions?}{\begin{itemize}
        \item First meeting of country-wide Continental Congress occurred in Philadelphia, making five key decisions
        \begin{enumerate}
            \item Rejected colonial union under Britain (like Albany Plan)
            \item Endorsed official statement of grievances for England
            \begin{itemize}
                \item Conceded Parliament's right to limit trade
                \item Requested that all oppressive legislation be repealed
            \end{itemize}
            \item Approved resolutions for military preparation against British troops in Boston
            \item Agreed to stop \textit{all trade} with Great Britain
            \begin{itemize}
                \item Continental Assocation formed to enforce agreements
            \end{itemize}
            \item Agreed to meet again in following spring
            \begin{itemize}
                \item Through meeting, had formed agreement to remain autonomy while declaring economic war on GB
            \end{itemize}
        \end{enumerate}
        \item More optimistic Americans hoped economic war would be sufficient to earn independence; many felt direct war was necessary
        \item Great contemplations made within Britain as to how to handle colonists
        \begin{itemize}
            \item Pitt urged withdrawal of troops; Burke called for repeal of Coercive Acts
            \item Lord North won approval by early 1775 for Concillatory Proposition, which required taxes but allowed colonists to tax themselves under demand of Parliament
            \begin{itemize}
                \item Aimed to divide moderates from extremists, but too late
            \end{itemize}
        \end{itemize}
    \end{itemize}
    \textbf{A significant step for intercolonial unity was made under the Continental Congress, where all colonies but Georgia met in Philadelphia to make key decisions including resolutions for military preparation, conclusion of all trade with Britain, and an official demand to repeal all oppressive legislation. In England, Lord North implemented the Concillatory Proposition, which, while giving the colonists more autonomy, was too late: war had already begun.}}
    \cornell{What occurred in Lexington and Concord?}{\begin{itemize}
        \item MA "minutemen" had been training for months to fight on a moment's notice
        \begin{itemize}
            \item General Thomas Gage, commanding troops, felt minutemen were too small, but remained cautious (ignored advice of peers that American colonists would immediately bow down to authority)
            \item After Gage received orders to arrest Adams/Hancock, hesitated; finally acted after hearing of large supply of gunpowder
            \begin{itemize}
                \item Hoped to raid minutemen (w/ 1000 soldiers) at night to take goods without bloodshed
                \item Patriots watched closely, with farms and towns already warned of British invasion
            \end{itemize}
            \item Arrival in Lexington met with prepared minutemen who quickly fell; reached Concord, burning remaining gunpowder supply; on return to Boston, shot at for entire trip, losing three times as many men as Americans
        \end{itemize}
        \item Source of first shots disputed (both blamed each other), but tales began to rapidly spread to rally thousands of colonists in north and south
        \item Battles in Lexington and Concord may have seemed simply more battles as part of greater tensions; retrospectively, clearly recognized as decisive step for both, with war having begun
    \end{itemize}
    \textbf{The skirmishes in Lexington and Concord are seen today as having been the first battles of the Revolution. They emerged after General Gage, commanding British troops, sent soldiers at night to confiscate an illegal gunpowder supply. The prepared minutemen fought back, seeing many soldiers on both sides killed.}}
    \end{document}