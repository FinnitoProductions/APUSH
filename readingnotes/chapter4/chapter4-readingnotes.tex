\documentclass[a4paper]{article}
    \usepackage[T1]{fontenc}
    \usepackage{tcolorbox}
    \usepackage{amsmath}
    \tcbuselibrary{skins}
    
    \title{
    \vspace{-3em}
    \begin{tcolorbox}
    \Huge\sffamily \begin{center} Chapter 4  \mbox{} \\ \huge The Empire in Transition \mbox{} \\
    \LARGE Finn Frankis \mbox{} \\
    \Large AP US History - September 14{$^\text{th}$}, 2018 \end{center} 
    \end{tcolorbox}
    \vspace{-3em}
    }
    \date{}
    \author{}
    
    \usepackage{background}
    \SetBgScale{1}
    \SetBgAngle{0}
    \SetBgColor{red}
    \SetBgContents{\rule[0em]{4pt}{\textheight}}
    \SetBgHshift{-2.3cm}
    \SetBgVshift{0cm}
    \usepackage[margin=2cm]{geometry} 
    
    \makeatletter
    \def\cornell{\@ifnextchar[{\@with}{\@without}}
    \def\@with[#1]#2#3{
    \begin{tcolorbox}[enhanced,colback=gray,colframe=black,fonttitle=\large\bfseries\sffamily,sidebyside=true, nobeforeafter,before=\vfil,after=\vfil,colupper=blue,sidebyside align=top, lefthand width=.3\textwidth,
    opacityframe=0,opacityback=.3,opacitybacktitle=1, opacitytext=1,
    segmentation style={black!55,solid,opacity=0,line width=3pt},
    title=#1
    ]
    \begin{tcolorbox}[colback=red!05,colframe=red!25,sidebyside align=top,
    width=\textwidth,nobeforeafter]#2\end{tcolorbox}%
    \tcblower
    \sffamily
    \begin{tcolorbox}[colback=blue!05,colframe=blue!10,width=\textwidth,nobeforeafter]
    #3
    \end{tcolorbox}
    \end{tcolorbox}
    }
    \def\@without#1#2{
    \begin{tcolorbox}[enhanced,colback=white!15,colframe=white,fonttitle=\bfseries,sidebyside=true, nobeforeafter,before=\vfil,after=\vfil,colupper=blue,sidebyside align=top, lefthand width=.3\textwidth,
    opacityframe=0,opacityback=0,opacitybacktitle=0, opacitytext=1,
    segmentation style={black!55,solid,opacity=0,line width=3pt}
    ]
    
    \begin{tcolorbox}[colback=red!05,colframe=red!25,sidebyside align=top,
    width=\textwidth,nobeforeafter]#1\end{tcolorbox}%
    \tcblower
    \sffamily
    \begin{tcolorbox}[colback=blue!05,colframe=blue!10,width=\textwidth,nobeforeafter]
    #2
    \end{tcolorbox}
    \end{tcolorbox}
    }
    \makeatother

    \parindent=0pt
    
    \begin{document}
    \maketitle
    \SetBgContents{\rule[0em]{4pt}{\textheight}}
    \cornell[Key Concepts]{What are this chapter's key concepts?}{\begin{itemize}
        \item British government increasingly attempted to incorporate North American colonies into coherent, imperial structure with mercantilist aims; failed due to conflict w/ colonists and natives 
        \item Colonists drew on local self-government to resist imperial control: liberty began to grow as a major concept
        \item Colonial rivalry between Britain and France intensified, threatening French-native trade networks and nativr autonomy
        \item Britain defeated the French, expanding territorial holdings
        \item Imperial officials attempted to prevent westward expansion - natives hoped to continue trading with Europeans while resisting colonial expansion 
        \item Colonial leaders based calls for resistance on arguments about rights of British subjects
        \item American independence effort pushed by powerful leaders like Franklin
    \end{itemize}}
    \cornell[Loosening Ties]{What changes marked the weakening of ties between the British mainland and overseas colonies?}{
        \textbf{British Parliamentary control saw limited rule over the colonies; the London Privy Council, too, as well as royal governors, were often bribed and controlled by colonial officials. In part because of this, the colonies remained divided, viewing each other as foreigners.}
    }
    \cornell{What was the overall context behind England's reduced attention to the colonies?}{\textbf{The British government made no serious effort to govern the colonies due to great divisions within Parliament concerning the extent of interference. Although many colonies soon became royal colonies and some new laws were implemented (mainly economic), colonies were generally left to diverge from British rule.}}
    \cornell{What, specifically, gave the colonies freedom to diverge from British rule?}{\begin{itemize}
        \item Reigns of German-born George I, George II saw monarchical alienation; prime minister/cabinet ministers became true executives
        \begin{itemize}
            \item Parliamentary leaders depended on landholders and merchants, leading them to keep loose control over colonies to reduce expenditure
            \item Navigation Act often ignored to stimulate commerce
        \end{itemize}
        \item Day-to-day administration decentralized: some power in Board of Trade and Plantations, an advisory body; real power in Privy Council
        \begin{itemize}
            \item Focus on mainland affairs led to reduced concentration on colonial affairs
            \item Most London officials had little experience with America: only knowledge came from overseas agents (who rarely encouraged interference)
            \item Royal officials (mostly governors) in America generally succumbed to bribery, favoritism
            \begin{itemize}
                \item Often hired substitutes to take places in America with low wages, encouraging bribery(\textit{ex}: customs collectors often waived duties if paid directly)
            \end{itemize}
        \end{itemize}
        \item Main, deliberate resistance to authority seen in colonial legislatures
        \begin{itemize}
            \item Gave themselves right to levy taxes, pass laws with potential veto by Privy Council
            \item Often found loopholes such as slight changes to laws; leverage over Privy Council due to control of budget gave little power to Council itself
        \end{itemize}
    \end{itemize}
    \textbf{Decentralized control was rooted both in mainland England and the colonies. The Parliament's control over English politics and their ties with merchants meant that rarely would laws be enforced for economic reasons; furthermore, the London Privy Council had little overall control due to frequent bribery of royal officials and the legislatures' leverage over them.}}
    \cornell{Did the colonies began to converge into one cohesive unit?}{\begin{itemize}
        \item Many colonists felt greater ties to England than to each other; nearby colonies viewed each other as foreigners tied only by geography
        \item Economic connections often emerged especially with roads along coast allowing for intercolonial trade, postal service promoting communication
        \item Cooperation was often challenging even amidst threat of adversaries
        \begin{itemize}
            \item Threat from French/native allies led to Albany Plan
            \begin{itemize}
                \item Treaty with the Iroquois and a potential colonial federation with one general, elected government for native relations
                \item Allowed present constitutions to remain intact
            \end{itemize}
            \item Albany Plan approved by no colonial assemblies
        \end{itemize}
    \end{itemize}
    \textbf{No - despite some economic ties, the colonists generally remained split by regional differences. Even in the threat of their French adversaries, all legislatures refused to cooperate under the proposed Albany Plan.}}
    \cornell[The Struggle for the Continent]{What were the major struggles for control over the American continent?}{\textbf{The most significant struggle was the French and Indian War, preceded by smaller conflicts between the British, French, and Iroquois over the large territories held by each group. It began with Washington's invasion of Fort Duquesne and was split into three phases: one of British disorganization and little assistance, one of colonial assistance and a connection to Europe but aggravated tensions between the British and the colonists, and the third with the tide turned toward the British and a victory which provided large swathes of land to the English and Spanish but sparked more conflicts between the British and the colonists.}}
    \cornell{What was the background to the French and Indian War?}{\textbf{A global war, the French and Indian War saw a rearrangement of global power with England at the head. However, the conflict was long-coming, with an uneasy balance of power between natives (particularly the Iroquois), the French, and the English colonists finally settled.}}
    \cornell{What were the basic traits of New France and how were the natives involved?}{\begin{itemize}
        \item French/English had coexisted peacefully, but religious/commercial tensions produced new conflicts
        \begin{itemize}
            \item Expansion of French presence under Louis XIV with growing fur trade, presence of French Jesuits, farmers travelling south from Canada 
        \end{itemize}
        \item New France comprised vast territory, with explorers claiming Louisiana (named for King Louis); subsequent explorers reached as far south as Rio Grande
        \begin{itemize}
            \item Territory held together only by widely separated communities, fortresses, trading posts 
            \item In north, fortified city of Quebec with Montreal to south; Mississippi plantations further south worked by black slaves; Louisiana contained large cities like New Orleans, Mobile
            \item Large territory shared with natives, English settlers 
            \begin{itemize}
                \item French/English both understood importance of forging relations with native tribes for power
                \item Natives concerned with protecting independence; often marrying for convenience
                \begin{itemize}
                    \item English attracted natives with advanced goods, but forced to conform to standards
                    \item French adjusted behavior to native patterns, marrying women and adopting tribal ways; Jesuits allowed Catholicism
                \end{itemize}
                \item Iroquois confederacy (Mohawk, Seneca, Cayuga, Onondaga, Oneida) had unique relationship with both groups
                \begin{itemize}
                    \item Economic relations with English/Dutch and some with French
                    \item Avoided allying to closely with either side, often playing both groups against each other
                \end{itemize}
                \item Ohio Valley location of principal conflict; claimed by French but natives hoped to control, too (especially Iroquois)
            \end{itemize}
        \end{itemize}
    \end{itemize}
    \textbf{New France expanded rapidly after the intervention of Louis XIV, but remained fragmented and held together by distant (yet often large communities). The territory was shared with the natives: unlike the English, the French often adopted native ways through marriage without infringing on their rights and desires. One key group allied with neither the English nor the French: the Iroquois, who, for economic reasons, remained neutral.}}
    \cornell{What were the major initial conflicts between the English and the French?}{\begin{itemize}
        \item Greatly dependent on relations between English/French throne
        \begin{itemize}
            \item Aggravated after Glorious Revolution brought Louis XIV's enemies into power (opposed expansionism)
            \item Continued as successor of William III (Anne), continued struggle against Spain, French ally 
            \item King William's War saw a few small clashes in NE
            \item Queen Anne's War saw conflicts between southern Spaniards and northern French
            \begin{itemize}
                \item Resolution to Queen Anne's War (Treaty of Ulrecht) gave substantial French territory to English (including Newfoundland, Acadia)
            \end{itemize}
        \end{itemize}
        \item More conflicts followed, most notably King George's War
        \begin{itemize}
            \item Based around conflicting trading rights of English settlers in Spanish territories
            \item Merged with larger war: on opposite sides of conflicts between Austrians and Prussians
            \item War ended after New Englanders captured French bastion at Louisbourg on Cape Breton w/ peace treaty forcing abandonment
        \end{itemize}
        \item Aftermath led to deterioration between French-English-Iroquois relations
        \begin{itemize}
            \item Iroquois granted trading concessions to English merchants, leading the French to fear English goal for expansion
            \item French created more fortresses in Ohio Valley, leading the English to interpret activity as threat to western settlements, building their own fortresses
            \item Iroquois power balance had collapsed, forcing them to ally with the English
            \item Tensions increased in following five years, starting with 1754 Virginia militia into Ohio Valley under command of George Washington
            \begin{itemize}
                \item Attack unsuccessful: Fort Necessity (crude stockade constructed by Washington) raided with more than $\frac{1}{3}$ of men killed 
                \item Washington forced to surrender
            \end{itemize}
        \end{itemize}
    \end{itemize}
    \textbf{The initial conflicts were dependent on French-English relations in Europe, most notably Queen Anne's War, which involved the Spanish, an ally of the French. King George's War, too, was significant, representative of a greater war in Europe between the Prussians and the Austrians. After King George's War, Iroquois relations deteriorated due to a series of misinterpretations. The French and Indian War began with Washington's attack on Fort Duquesne in modern Pittsburgh.}}
    \cornell{What defined the conflicts of the French and Indian War?}{\begin{itemize}
        \item Lasted nearly nine years, divided into three key phases
        \item First phase began with Fort Necessity and ended with eventual involvement of Europeans
        \begin{itemize}
            \item British initially provided modest assistance but was poorly organized, failing to prevent more French fleets from arriving and a British army failing in their attempt to take over the Fort Necessity area
            \item Local forces preoccupied with defense on western front, defending against \textit{all} native tribes but Iroquois (viewed English loss as weakness)
            \item Iroquois themselves feared French - despite having declared war, made few advances on Canada
        \end{itemize}
        \item Second phase opened in 1756 when governments of England and France directly began hostilities 
        \begin{itemize}
            \item Marked beginning of British Seven Years' War, where France allied with Austria, former enemy, and England allied with Prussia, former French ally
            \item Fighting had reached West Indies, India, Europe; primary struggle remained in North America
            \item English had been continually defeated until intervention of William Pitt, secretary of state
            \begin{itemize}
                \item Planned advaned military strategy with forced enlistment ("impressment"), seizure of supplies from farmers, tradesmen; demanded complementary shelter for troops 
                \item Greater independence of Americans led to stark resistance to Pitt's advancements
                \begin{itemize}
                    \item 1757: major riot in New York City threatened to bring English war effort to conclusion
                \end{itemize}
            \end{itemize}
        \end{itemize}
        \item Third phase initiated by William Pitt in relaxing many oppressive policies, with compensation for colonist supplies, new troops
        \begin{itemize}
            \item Tide began to favor England, partly due to many troops and partly due to poor French harvests
            \item Key generals Jeffrey Amherst, James Wolfe captured fortress at Louisbourg; defeated Montcalm in Quebec in 1759 where both commanders died, fell dramatically
            \item Many other aspects of the war far more brutal
            \begin{itemize}
                \item Uprooted Nova Scotians (Acadians), forcing migration throughout English colonies due to suspected disloyalty
                \item Offered "scalp bounties" for natives 
                \begin{itemize}
                    \item French/native allies retaliated with massacre of hundreds of English families
                \end{itemize}
            \end{itemize}
        \end{itemize}
    \end{itemize}
    \textbf{The conflicts of the French and Indian War can be divided into three phases: the first without colonial intervention and great British disorganization and numerous losses, the second tied with the Seven Years' War in England (known for British harshness agaisnt colonists under William Pitt), and the third initiated by greater cooperation between the colonists and the English, with many successful English battles (including that in Quebec), but also great brutality on both sides.}}
    \cornell{What was the aftermath of the Seven Years' War?}{\begin{itemize}
        \item Peace came after George III reached throne, Pitt resigned (he had hoped to continue hostilities)
        \begin{itemize}
            \item French ceded West Indies islands, most of Indian colonies, Canada to GB; all territory east of Mississippi but New Orleans to Spain in Peace of Paris
        \end{itemize}
        \item War had profound effects on various involved groups
        \begin{itemize}
            \item War gave Britain many new territories while enlarging debt
            \begin{itemize}
                \item Led British leaders to house resentment for Americans due to few financial contributions (even some merchants continued selling goods to French)
                \item Many leaders felt reorganization was essential
            \end{itemize}
            \item For American colonists, began to feel sense of unity against common foe 
            \begin{itemize}
                \item Established British illegitimacy: viewed war as voluntary, communal while British regulars viewed as hierarchal and coercive
            \end{itemize}
            \item Natives of Ohio Valley saw British victory as disastrous: allegiance to French led to English enmity
            \begin{itemize}
                \item Iroquois's passivity angered colonists, leading to unraveling of alliance, loss of control over territories
            \end{itemize}
        \end{itemize}
    \end{itemize}
    \textbf{Peace was found after Pitt resigned; it entailed the cession of the majority of French land to the English and Spanish. The war caused the British and the colonists to feel greater mutual hostility and the natives to completely unravel due to the enmity of the British.}}
    \cornell[The New Imperialism]{What marked Britain's decision to take a greater role in governing the colonies?}{\textbf{The British suffered from many major burdens after the war, including a redesigned imperial structure, challenges with larger area, a large war debt, and struggles with George III, an ill-fitted monarch. Grenville, a new prime minister, hoped to seize control over the colonies with many major changes, which included greater support of the natives to control colonial westward expansion and also many major economic changes. Although the colonists were initially unable to resist due to local divisions, they were eventually unified by an economic boon and the ideals of self-government.}}
    \cornell{What were the major burdens of Britain's large empire and how did they solve them?}{\begin{itemize}
        \item Britain initially hesitant due to great conflicts during war between Pitt and colonists 
        \item Major 1763 shift in imperial design led to further challenges
        \begin{itemize}
            \item British initially viewed colonies as key to trade; many leaders argued the land was the true value for taxes, imperial splendor 
            \begin{itemize}
                \item Mercantilists, at conclusion of Seven Years' War, hoped to return Canada to France for Guadeloupe, productive sugar island
                \item Franklin felt larger territories were critical for limitless growth
            \end{itemize}
        \end{itemize}
        \item Area had grown twice as large by 1763, posing major challenges
        \begin{itemize}
            \item Some British sought restrained settlement due to potential conflict with natives if settlemetn was rapid
            \item Many colonists wanted to expand existing colonies westward into new lands; others felt new colonies should emerge
        \end{itemize}
        \item War debt of England could not be improved
        \begin{itemize}
            \item Landlords/merchants objected to tax increases
            \item Necessity to station troops on American-Indian borders led to further costs
            \item Little response of colonial assemblies in war efforts gave Britian little reason to rely on cooperation
            \item Believed that only solution would be new system of taxation
        \end{itemize}
        \item Accession of George III led to further issues
        \begin{itemize}
            \item Strongly determined to increase power of monarch (pressured by mother)
            \begin{itemize}
                \item Removed stable Whig coalition from Parliamentary power, creating new bribed coalition with uneasy control over Parliament; very unstable 
            \end{itemize}
            \item Intellectual/physcological limitations led to major political difficulties
            \begin{itemize}
                \item Disease produced intermettent bouts of insanity
                \item Often behaved irrationally: ill-fitted for role but immaturely constantly attempted to show fitness
            \end{itemize}
        \end{itemize}
        \item George Grenville, new prime minister, disagreed with Pitt: felt colonists had been long overindulged, required to pay part of cost
        \begin{itemize}
            \item Hoped to implement major control over America
        \end{itemize}
    \end{itemize}
    \textbf{British challenges included a new shift in imperial design emphasizing land over trade, the large growth in area leading to internal and external administrative conflicts, and a great war debt which many felt required inevitable taxation. Furthermore, George III, an unstable monarch, hoped to be active as a ruler but was generally ill-fitted for the role. Finally, George Grenville, the British prime minister, hoped to restore control over the colonies.}}
    \cornell{What were the relations between the natives and the British?}{\begin{itemize}
        \item After French departed Ohio Valley, British immediately streamed in, angering natives
        \begin{itemize}
            \item Ottawa chieftain Pontiac struck back against British; war ended immediately by government to prevent major conflicts 
            \item Proclamation of 1764 prevented expansion beyond Appalachians
            \begin{itemize}
                \item Allowed London to control westward movement in hopefully orderly manner
                \item Allowed coastal colonies to remain strong, promoting English interests
            \end{itemize}
        \end{itemize}
        \item Tribes generally discontent with Proclamation due to required cession of more land
        \begin{itemize}
            \item Many accepted as only option, especially Cherokee 
            \item Relations improved with natives in large part due to British-appointed superintendents Stuart (south) and Johnson (north)
            \begin{itemize}
                \item Sympathetic to natives, living among tribes
            \end{itemize}
            \item Proclamation ultimately failed to meet most needs of natives
            \begin{itemize}
                \item Line of settlement never enforced, with settlers continually expanding beyond despite restrictions by British authorities
            \end{itemize}
        \end{itemize}
    \end{itemize}
    \textbf{The British government issued Proclamation of 1764, restricting expansion beyond the Appalachians, in an attempt to appease the natives and take direct control over westward expansion. However, most tribes were discontent as it required cession of more land and it was poorly enforced.}}
    \cornell{What was the colonial response to British intervention?}{\begin{itemize}
        \item Grenville tightened hold over colonies
        \begin{itemize}
            \item Required permanent troops in colonies, mandatory colonial assistance in maintaining armies, sent ships to patrol for smugglers 
            \item Manufacturing restricted as to not compete with expanding industries of Britain
            \begin{itemize}
                \item Sugar Act of 1764 strengthened duty enforcement on sugar (limiting illegal sugar trade), establishing new courts for smugglers
                \item Currency Act of 1764 banned paper money
                \item Stamp Act of 1765 issued new taxes on printed documents
                \item Ultimate goal to strengthen mercantilism in colonies
            \end{itemize}
        \end{itemize}
        \item Colonists initially struggled to resist Grenville's changes as frequent local grievances between colonies limited cohesiveness
        \begin{itemize}
            \item Primary tensions between "backcountry" and coastal societies w/ isolation and underrepresentation
            \begin{itemize}
                \item Western societies far closer to native tribes, often left unable to defend by government
            \end{itemize}
            \item Civil war in North Carolina emerged
            \begin{itemize}
                \item Caused by Regulator movement: farmer movement to oppose high taxes collected by local sheriffs
                \item Failed with local assembly, so began resisting by force; suppressed by governor William Tyron with army of militiamen
                \begin{itemize}
                    \item Hanged seven dissidents for treason; nine killed on each side
                \end{itemize}
            \end{itemize}
        \end{itemize}
        \item Tensions with British began to overshadow divisions between colonies
        \begin{itemize}
            \item Everyone felt antagonized by Grenville program
            \begin{itemize}
                \item Commercial restraints, increased taxes angered northern merchants
                \item Northern backcountry angered by reduced western expansion
                \item Southern planters feared additional taxes
                \item Professionals depended on merchants/planters for livelihood
                \item Small farmers feared abolition of paper money
            \end{itemize}
            \item New restrictions arrived in British attempt to restrict postwar debt
            \begin{itemize}
                \item After having poured money into colonies for wartime success, bust followed with declined standard of living
            \end{itemize}
            \item Most Americans circumvented policies; economy not destroyed
            \item Political consequences viewed as most damaging with Enlightenment ideals of self-government threatened
            \begin{itemize}
                \item Home rule was seen as an essential trait of colonial life
                \item Movement democratic and conservative: to conserve existing liberties
            \end{itemize}
        \end{itemize}
    \end{itemize}
    \textbf{As Grenville tightened his hold over the colonies, the colonists initially struggled to respond due to local tensions including those between the western farmers and coastal cities. However, major tensions with the British as Grenville's acts threatened everyone paired with Enlightenment ideals of self-government began to unify the colonies.}}
    \end{document}