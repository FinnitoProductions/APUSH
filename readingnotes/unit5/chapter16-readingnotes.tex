\documentclass[a4paper]{article}
    \usepackage[T1]{fontenc}
    \usepackage{tcolorbox}
    \usepackage{amsmath}
    \tcbuselibrary{skins}
    
    \title{
    \vspace{-3em}
    \begin{tcolorbox}
    \Huge\sffamily \begin{center} AP US History  \\
    \LARGE Chapter 16 - The Conquest of the Far West \\
    \Large Finn Frankis \end{center} 
    \end{tcolorbox}
    \vspace{-3em}
    }
    \date{}
    \author{}
    
    \usepackage{background}
    \SetBgScale{1}
    \SetBgAngle{0}
    \SetBgColor{red}
    \SetBgContents{\rule[0em]{4pt}{\textheight}}
    \SetBgHshift{-2.3cm}
    \SetBgVshift{0cm}
    \usepackage[margin=2cm]{geometry} 
    
    \makeatletter
    \def\cornell{\@ifnextchar[{\@with}{\@without}}
    \def\@with[#1]#2#3{
    \begin{tcolorbox}[enhanced,colback=gray,colframe=black,fonttitle=\large\bfseries\sffamily,sidebyside=true, nobeforeafter,before=\vfil,after=\vfil,colupper=blue,sidebyside align=top, lefthand width=.3\textwidth,
    opacityframe=0,opacityback=.3,opacitybacktitle=1, opacitytext=1,
    segmentation style={black!55,solid,opacity=0,line width=3pt},
    title=#1
    ]
    \begin{tcolorbox}[colback=red!05,colframe=red!25,sidebyside align=top,
    width=\textwidth,nobeforeafter]#2\end{tcolorbox}%
    \tcblower
    \sffamily
    \begin{tcolorbox}[colback=blue!05,colframe=blue!10,width=\textwidth,nobeforeafter]
    #3
    \end{tcolorbox}
    \end{tcolorbox}
    }
    \def\@without#1#2{
    \begin{tcolorbox}[enhanced,colback=white!15,colframe=white,fonttitle=\bfseries,sidebyside=true, nobeforeafter,before=\vfil,after=\vfil,colupper=blue,sidebyside align=top, lefthand width=.3\textwidth,
    opacityframe=0,opacityback=0,opacitybacktitle=0, opacitytext=1,
    segmentation style={black!55,solid,opacity=0,line width=3pt}
    ]
    
    \begin{tcolorbox}[colback=red!05,colframe=red!25,sidebyside align=top,
    width=\textwidth,nobeforeafter]#1\end{tcolorbox}%
    \tcblower
    \sffamily
    \begin{tcolorbox}[colback=blue!05,colframe=blue!10,width=\textwidth,nobeforeafter]
    #2
    \end{tcolorbox}
    \end{tcolorbox}
    }
    \makeatother

    \parindent=0pt
    
    \begin{document}
    \maketitle
    \SetBgContents{\rule[0em]{4pt}{\textheight}}
    \cornell[Key Concepts]{What are this chapter's key concepts?}{\begin{itemize}
        \item \textbf{5.1.II.C} - U.S. government restricted culture/independence of Mexicans/natives by encroaching on territories
        \item \textbf{6.1.I.A} - Government support of transportation $\to$ emergence of new markets post-Civil War
        \item \textbf{6.1.III.A} - Mechanization $\to$ $\uparrow$ food production, $\downarrow$ prices
        \item \textbf{6.2.II.A} - Transcontinental railroads, minerals, govt. policies $\to$ growth of new commercial regions
        \item \textbf{6.2.II.B} - Migrants went to both rural/commercial west for new opportunities in railroad construction/farming/ranching/mining
        \item \textbf{6.2.II.C} - $\uparrow$ migrant populations $\to$ $\downarrow$ bison population $\to$ growing competition for resources $\to$ conflict w/ natives/Mexicans
        \item \textbf{6.2.II.D} - U.S. govt. continually broke treaties w/ natives $\to$ lost independence through confinement to reservations
    \end{itemize}}
    \cornell[The Societies of the Far West]{How did Western societies develop and change over time?}{\textbf{Indigenous populations in the west were composed mainly of Pueblo natives, based around small commerce-based towns, and Plains natives, a diverse group generally hunting buffaloes for subsistence. The Hispanic populations in New Mexico, California, and Texas, were oppressed and forced into subordinacy by migrating Anglo-Americans. Several Chinese migrants arrived for the fleeting gold rush, quickly transitioning from prospecting to railroad construction to urban life, where anti-Chinese sentiments due to economic and cultural fear culminated in the Chinese Exclusion Act of 1882, banning Chinese migration. Several Americans, many immigrants from Europe, migrated westward in the late 19th century, forming new political structures.}}
    \cornell{What was the state of the western tribes before the arrival of the western migrants?}{\begin{itemize}
        \item Largest pop. in west were natives, some resettled but some indigenous
        \item Pueblo societies mainly agricultural, permanent settlements even before Spanish
        \begin{itemize}
            \item Corn, adobe houses, irrigation, trade
            \item Allied w/ Spanish against other tribes of region
            \item Formed complex social system w/ Spanish on top, Pueblos relatively free, captured members of other tribes and those who had escaped tribes (\textit{genizaros}) at bottom
        \end{itemize}
        \item Plains Indians most widespread group
        \begin{itemize}
            \item Known for diverse traits w/ some alliances; others conflicted, some sedentary; others nomadic
            \item Shared extended family networks, connection w/ nature, tribal divisions into smaller "bands" w/ council
            \begin{itemize}
                \item Tasks by gender w/ women domestic roles, some work in gardens; men hunters, trade, religious/mil. leaders
                \item Religion typically spiritual based around rhythms
            \end{itemize}
            \item Economic basis of society: buffalo hunting for flesh (food), skin (clothes), manure (fuel), bones (knives)
            \begin{itemize}
                \item Hunted w/ small/powerful horses (from Spain), following stock and constructing teepees along way
                \item Never disrupted landscape
            \end{itemize}
            \item Powerful warriors w/ majority of males warrior class, competition; \textbf{Sioux} most powerful
            \begin{itemize}
                \item Posed greatest threat to white settlers but lost due to disunity, internal conflict
                \item Sioux, Arapaho, Cheyenne temporarily posed threat w/ alliance but lost due to disease, limited industry
            \end{itemize}
        \end{itemize}
    \end{itemize}
    \textbf{The natives were the largest population in the west, with the Pueblo people known for a strict racial hierarchy and the Plains natives dividing tasks by gender and known for a spiritual religion based around the natural rhythms. Their economy was based around buffalo hunyinh, with several parts of the buffalo critical for success; furthermore, nearly all the men made up a warrior class, posing a great threat to whites but ultimately failing due to disunity, internal conflict, and disease.}}
    \cornell{How was the Hispanic population of New Mexico transformed by the arrival of Anglo-Americans?}{\begin{itemize}
        \item Land acquired by U.S. not most populous in Mexico but still many Mexicans stayed behind
        \begin{itemize}
            \item Spanish-speaking communities transformed by Anglo-American migrants w/ expansion of capitalist econ.; brought wealth to some but ruined society for others
            \item New Mexico had farming/trading communities established by Spanish, eventually consisting of Pueblos, American traders, Mexican peasants
        \end{itemize}
        \item Stephen Kearny established govt. after conquering NM excluding Mexicans, natives despite being \underline{large majority}
        \begin{itemize}
            \item Taos natives rebelled in 1847, killing governor but eventually being subdued by U.S. Army w/ mil. govt.
            \item Territorial govt. reestablished in 1850, known as "territorial ring" by 1870s w/ businessppl. taking over and focusing on profitability thru. expansion 
        \end{itemize}
        \item Hispanic societies remained sizable 
        \begin{itemize}
            \item Largest growth after U.S. finally defeated Navajo/Apache tribes who had been terrorizing NM inhabitants (mostly peasants/tradesppl. looking for commerce)
            \item Survived amidst Anglo-Americans due to distance from U.S. English-speaking centers but also willingness to fight for control (\textit{ex}: Mexican peasants in modern NV prevented cattle ranchers)
            \item Generally grew in subordinacy over time w/ Anglo-Americans arriving by railroad restricting to lowest-paid jobs
        \end{itemize}
    \end{itemize}
    \textbf{The Hispanic population of New Mexico remained sizable even after the U.S. took over the region from Mexico, but their small farming communities were quickly transformed by the arrival of Anglo-Americans, with a territorial government barring natives and Mexicans. Although Hispanic society continued to grow, most were restricted to subordinate positions.}}
    \cornell{How was the Hispanic population of California transformed by the arrival of Anglo-Americans?}{\begin{itemize}
        \item Spanish settlement in California began in 18th c. w/ Christian missions along coast, pulling surrounding natives into communities as a labor source and for conversion  
        \begin{itemize}
            \item Workers received few profits from their herding of animals, brickmaking, farming, etc.
        \end{itemize}
        \item Mexicans restricted power of church in 1830s $\to$ mission society collapsed w/ emergence of aristocracy and large estates; soon transformed by Anglo-Americans
        \begin{itemize}
            \item Hispanic \textit{californios} harmed by arrival of Anglo-Americans, forcibly excluded and losing their land thru. corrupt deals 
            \item Southern California saw large market for cattle raised by \textit{rancheros}, culture soon collapsed due to drought/debt/recklessness
            \begin{itemize}
                \item Became lower class workers clustering in cities; farmers who remained became dominated by ranchers
            \end{itemize}
        \end{itemize}
    \end{itemize}
    \textbf{The Spanish settlement of California began as a missionary community based around the enslavement and conversion of natives; however, the secular push from Mexicans soon transformed society into a short-lived aristocracy later dominated by Anglo-Americans, excluding Mexicans and encroaching on their land and eventually relegating them to a state of abject poverty.}}
    \cornell{How was the Hispanic population of Texas transformed by the arrival of Anglo-Americans?}{\begin{itemize}
        \item Texas saw even most dominant ranchers unable to compete with Anglo-American counterparts, also losing their land through fraud/coercion
        \begin{itemize}
            \item Most became unskilled farmers/industrial laborers
        \end{itemize}
        \item 1859: Anger culminated in raid on Brownsville led by Juan Cortina, freeing Mexican prisoners; no long-term effect due to quick imprisonment
    \end{itemize}
    \textbf{Texan Hispanics saw their land encroached by Anglo-Americans, slowly being relegated to unskilled farmers. Although a brief raid emerged in 1859, it was quickly put down and Hispanic subordinacy preserved.}}
    \cornell{How were the natives affected by Anglo-American western migration?}{\textbf{Natives were even more dramatically affected than Hispanics (who were sometimes able to rise the social ladder), remaining in the lower classes and, along with many Hispanics, becoming drawn into a growing capitalist society.}}
    \cornell{What characterized the Chinese migration to the Americas?}{\begin{itemize}
        \item Many Chinese $\to$ HI, Australia, S./C. America, Caribbean, S. Africa; some to America pre-gold rush but  majority $\to$ California with gold rush
        \begin{itemize}
            \item Initially welcomed by whites as hard workers; quick hostility due to perceived threat for hard work $\to$ great discrimination
            \item Many originally saw success, but CA "foreign miners" tax limited profitability for Chinese/Mexicans $\to$ generally abandoned prospecting work, with few remaining joining large, deep operations
        \end{itemize}
        \item Most Chinese shifted to railroad work, making up 90\% of Central Pacific labor; several directly recruited
        \begin{itemize}
            \item No experience w/ organized labor $\to$ easily exploitable by owners 
            \item Arduous work, tunneling through wintry mountains and often suffocating in sleep due to volatile resting places in snowbanks
            \item 1886: 5k workers went on strike but quickly put down and starved 
        \end{itemize}
        \item 1869: railroad completed $\to$ out of work, w/ some working in agriculture (tenant farmers, irrigation) but most $\to$ cities
        \begin{itemize}
            \item Formed powerful clan-based Chinatowns led by merchants ("Six Companies") addressing issues of inhabitants and preserving culture; largest in San Francisco
            \item Some organizations secret, violent societies in opium/prostitution, known as \textbf{tongs}
            \item Urban Chinese made up lower classes: most were laborers, some established small businesses, typically laundries due to exclusion from all other areas
            \item Early Chinese migrant women generally originated from prostitution; sex ratio eventually balanced $\to$ more families
        \end{itemize}
    \end{itemize}
    \textbf{Most early Chinese migrants to America went to California for the gold rush, but "foreign miners" taxes quickly forced them into arduous railroad work for the Central Pacific. After railroad construction was completed, though some worked in agriculture, most flocked to cities, forming merchant-led Chinatowns. Urban Chinese typically occupied the lower classes, with most working as industrial laborers but some establishing small laundry businesses; women initially struggled due to their origins in prostitution but the sex ratio eventually balanced.}}
    \cornell{What characterized the opposition toward the Chinese?}{\begin{itemize}
        \item Several white residents formed "anti-coolie" clubs against Chinese, seeking work ban, attacking workers, and framing for crimes 
        \begin{itemize}
            \item Originated out of anger for acceptance of lower wages 
            \item Democratic Party and Irish-immigrant founded Workingmen's Party of California known for strongly anti-Chinese sentiment 
        \end{itemize}
        \item Others opposed Chinese for perceived cultural inferiority/savagery 
        \item Opposition culminated in 1882 Chinese Exclusion Act, barring emigration for 10 years, preventing citizenship
        \begin{itemize}
            \item Gained widespread acceptance for desire to protect "American" workers 
            \item Renewed in 1892 for another 10 years; made permanent in 1902 
        \end{itemize}
        \item Chinese in America argued origin from intelligent civilization; resented grouping w/ natives/Afr. Americans rather than w/ Irish/Jews/Italians, but little effect
    \end{itemize}
    \textbf{Several white Americans opposed the Chinese for fear of industrial competition due to their general acceptance of lower wages; others felt Chinese culture was inferior and dismissed immigrants as savages. The Chinese Exclusion Act of 1882, barring immigration and preventing citizenship, was widely accepted; any Chinese opposition was stifled.}}
    \cornell{What characterized the postwar migration to the west?}{\begin{itemize}
        \item Migrants to western U.S. came in millions; most from eastern U.S. but many from Europe
        \begin{itemize}
            \item Immigrants motivated by potential for cattle grazing, gold/silver, meadowlands for farming, railroad travel 
        \end{itemize}
        \item Homestead Act of 1862 provided western land for small fee given that purchaser would occupy, develop land for $\geq$ 5 yrs.; essentially free farm
        \begin{itemize}
            \item Possession of land far from sufficient to start farm w/ $\uparrow$ mechanization, costs to run; 160 acres of land often too small due to western terrain
            \item Large \#s abandoned land before 5 yrs.
        \end{itemize}
        \item Govt. increased allotments of Homestead Act in response to suffering
        \begin{itemize}
            \item \textbf{Timber Culture Act} of 1873 allowed 160 extra acres if inhabitants planted 40 acres of trees
            \item \textbf{Desert Land Act} of 1877 allowed 640 total acres w/ \$1.25 per acre if irrigation system implemented within three years 
            \item \textbf{Timber and Stone Act} of 1878 allowed general (nonarable) land for \$2.50 per acre
            \item Acts oft. encouraged fraud, but many settlers fairly earned large tracts of land
        \end{itemize}
        \item New territories quickly received political organization
        \begin{itemize}
            \item After Kansas' statehood (1861), WA, UT, NM, NB divided into smaller units for organization
            \item Territorial governments operated in several western states (NV, CO, Dakota, AZ, ID, MT, WY) by end of 1860s
            \item NV admitted in 1864; NB in 1867; CO in 1876; ND/SD, MT, WA in 1889; WY/ID in 1890; UT in 1896 after promising end of polygamy
            \item AZ/NM excluded at turn of century due to small white pop., Dem. majority in Repub. era; OK only opened to white settlement in 1890
        \end{itemize}
    \end{itemize}
    \textbf{Several migrants to the west were immigrants from Europe motivated by several factors. All western migrants were encouraged by the Homestead Act of 1862, which initially provided 160 acres for a small fee but evolved to provide more land for smaller fees to address workers' complaints. The new western territories were quickly admitted to statehood, with only three territories not having been made states by 1900.}}
    \cornell[The Changing Western Economy]{How did the economic structure of the West change over time?}{\textbf{Western labor was generally more volatile, profitable, and racially diverse than that of the East. The most notable industries were mining, which boomed after several strikes on valuable minerals like gold and silver but soon focused on cheaper ores like copper, zinc, and quartz and turned individual prospectors into wage laborers with terrible working conditions, and cattle ranching, a male-dominated profession based around the transport of Texan cattle but soon focused more on the sedentary ranch system instead. Western women played a part in the cattle industry and were also granted suffrage far earlier than in the East.}}    
    \cornell{What characterized the labor structure in the west?}{\begin{itemize}
        \item $\uparrow$ commerce $\to$ farmers/ranchers/miners recruited labor, but shortage due to remoteness, unwillingness/inability to hire natives $\to$ $\uparrow$ wages but $\downarrow$ job security, working conditions
        \begin{itemize}
            \item After projects completed, often lost jobs immediately; competition w/ Chinese workers
            \item Landless generally male, nomadic, unmarried w/ little social mobility apart from for those with initial wealth 
        \end{itemize}
        \item Working class generally multiracial w/ African Americans, Europeans, Chinese, Filipinos, Mexicans, natives working in unison
        \begin{itemize}
            \item White workers always placed above others; lower tiers generally nonwhite
            \item Employers pushed stereotype that Chinese/Mexicans/Filipinos more naturally suited to manual labor in mines due to smaller average size, acclimation to heat, supposed lack of amition
            \item All mobility reserved for whites, with great mobility within working class; Chinese/Mexicans/Filipinos rarely rose in society 
        \end{itemize}
    \end{itemize}
    \textbf{Labor was far less plentiful in the west, leading to higher wages; however, a project-based structure gave essentially no job security, leaving most workers single and nomadic. Racially, the working class of the west was very diverse, but white workers were generally allowed to rise up to higher positions and given far greater mobility than Chinese, Mexicans, and Filipinos due to racial stereotypes.}}
    \cornell{How did mining transform the western economic structure?}{\begin{itemize}
        \item Mining boom very brief ($\approx$ 30 yrs.) yet impactful: all strikes began w/ individual prospectors clearing out shallow areas by hand, followed by arrival of corporation to dig deeper, then depart
        \begin{itemize}
            \item First post-gold rush strike: Pike's Peak, CO in 1859 w/ 50k prospectors from throughout U.S.; rapidly ended $\to$ corporations arrived, reviving some profits
            \item Comstock Lode, NV discovered by Californians in 1859, dominating settlement and importing \underline{all goods}; capitalists followed initial boom w/ difficult quartz mining 
            \item 1874: Black Hills, southwest Dakota also saw very rapid corporate takeover
        \end{itemize}
        \item More long-term prosperity found in copper (\textbf{Anaconda copper mine} - 1881 in Montana), lead, tin, quartz, zinc
        \item Boomtowns quickly became lawless w/ outlaws forming gangs $\to$ vigilante groups formed to enforce law
        \item Men outnumbered women in boomtowns; most women came w/ husbands for domestic tasks; few single women worked as cooks, tavernkeepers, laundresses, or prostitutes 
        \item Post-boom, most prospectors worked as wage laborers for corporations; known for terrible conditions
        \begin{itemize}
            \item Generally very hot ($\geq 100^{\text{o}}$ F) w/ little ventilation $\to$ poisonous CO$_2$, dust, explosions
            \item Especially early on, machinery killed large numbers
        \end{itemize}
    \end{itemize}
    \textbf{The initial mining boom was brief, with strikes popping up in various places and characterized by a rush of independent prospectors followed by corporations; less lucrative metals formed a far stabler industry. Boomtowns were generally lawless, male-dominated, and, after the boom subsided, filled with wage laborers working in terrible conditions.}}
    \cornell{How was cattle ranching significant to the western economy?}{\begin{itemize}
        \item Open range free of charge $\to$ cattle ranching critical industry 
        \item Cattle industry instituted by Mexicans and Texans: pre-Anglo-Americans, had developed techniques of \textbf{branding}, \textbf{roundups}, \textbf{roping}, as well as gears like \textbf{saddles} and \textbf{spurs}
        \begin{itemize}
            \item Americans adopted similar methods after arrival 
            \item Texan animals generally descended from strong Spanish stock $\to$ known for toughness
        \end{itemize}
        \item Large portion of cattle died along rough journey from TX to MO for MO-Pacific Railroad, but trip represented beginning of link betw. TX cattle industry and Eastern markets
        \begin{itemize}
            \item KS-Pacific railroad $\to$ \textbf{Abilene, KS} remained railhead of cattle industry for some time (Chisolm Trail from TX to KS)
            \item Cattle ranchers forced to expand further west for new grazing-land
        \end{itemize}
        \item "Long drive" journey w/ cattle between places romanticized throughout U.S. w/ thousands of animals travelling at once
        \begin{itemize}
            \item Leading cowboys Confederate veterans; next-largest group Afr. Americans
            \item Foreigners assigned jobs of herder, cook 
        \end{itemize}
        \item Ranch emerged as permanent cattle base: grew in size as competition for open land grew
        \begin{itemize}
            \item Eastern farmers typically surrounded ranches w/ fences $\to$ range wars for land 
        \end{itemize}
        \item Famed profits $\to$ growing migrants from eastern U.S., England, Scotland $\to$ increasingly corporate 
        \begin{itemize}
            \item Growing size $\to$ grazing space shrunk, w/ long drives and open ranges less viable
            \item After two severe winters w/ a very hot summer in between, long drive ended, open-range industry collapsed; only settled ranches producing beef remained
        \end{itemize}
    \end{itemize}
    \textbf{The cattle industry, with technology introduced by the earlier Texans and Mexicans, pushed increasingly westward for new grazing lands and was famed for the "long drives" in which the large numbers of cattle in Texas travelled overland to northern rail stations. However, as corporations began to take over, the industry became oversaturated: there was little free grazing room, and two poor winters ultimately ended the open-range industry, with established permanent ranches taking over.}}
    \cornell{How did women play an integral part in the western economy?}{\begin{itemize}
        \item Early cattle industry saw few female ranchers/drivers; birth of ranch w/ sedentary lifestyle $\to$ far more women led cattle ranches than before
        \item Women received suffrage in the West far earlier than in East (starting in WY)
        \begin{itemize}
            \item UT: Mormons granted to prevent criticism of polygamy
            \item Many places granted women suffrage to raise population for easier statehood 
            \item Women often argued for inherent virtue and morality $\to$ deserved to have voice
        \end{itemize}
    \end{itemize}
    \textbf{The cattle industry became increasingly female after the concept of the ranch grew in popularity. Furthermore, most western women received suffrage notably earlier than their eastern counterparts, for reasons including inherent virtue as well as swelling population size to meet statehood requirements.}}
\end{document}