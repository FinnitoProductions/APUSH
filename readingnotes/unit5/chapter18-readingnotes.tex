\documentclass[a4paper]{article}
    \input{../notesheader.tex}
    \newcommand{\chapternumber}{17}
    \newcommand{\chaptertitle}{Industrial Supremacy}
    \title{\vspace{-3em}
    \begin{tcolorbox}
    \Huge\sffamily \begin{center} AP US History  \\
    \LARGE Chapter \chapternumber$\text{ }$- \chaptertitle \\
    \Large Finn Frankis \end{center} 
    \end{tcolorbox}
    \vspace{-3em}
    }
    \date{}
    \author{}
    
    \begin{document}
        \maketitle
        \SetBgContents{\rule[0em]{4pt}{\textheight}}
        \cornell[Key Concepts]{What are this chapter's key concepts?}{\begin{itemize}
            \item \textbf{6.1.I.C} - $\downarrow$ prices for goods $\to$ $\uparrow$ worker wages $\to$ $\uparrow$ std. of living, access to new goods, but also $\uparrow$ gap betw. rich/poor
            \item \textbf{6.1.I.A} - Cities brought Asian/S. and E. European/Afr. American migrants to escape poverty, religious persecution, restricted social mobility 
            \item \textbf{6.2.I.B} - Urban neighborhoods formed for specific ethnicities/races/classes
            \item \textbf{6.2.I.C} - International migration $\to$ debates over assimilation w/ many immigrants finding compromise
            \item \textbf{6.2.I.D} - Govt. thrived by providing social services to immigrants/poor in cities w/ unequal power distribution
            \item \textbf{6.2.I.E} - Corporations needed managers/clerical workers $\to$ middle class w/ leisure time
            \item \textbf{6.3.1.C} - Artists/critics like agrarians, utopians, socialists, Social Gospel advocates emphasized different visions for U.S. society
        \end{itemize}}
    \end{document}