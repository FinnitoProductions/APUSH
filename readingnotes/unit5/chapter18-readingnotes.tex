\documentclass[a4paper]{article}
    \input{../notesheader.tex}
    \usepackage[normalem]{ulem}

    \newcommand{\chapternumber}{18}
    \newcommand{\chaptertitle}{The Age of the City}

    \title{\vspace{-3em}
\begin{tcolorbox}
\Huge\sffamily \begin{center} AP US History  \\
\LARGE Chapter \chapternumber \, - \chaptertitle \\
\Large Finn Frankis \end{center} 
\end{tcolorbox}
\vspace{-3em}
}
\date{}
\author{}
    \begin{document}
        \maketitle
        \SetBgContents{\rule[0em]{4pt}{\textheight}}
        \cornell[Key Concepts]{What are this chapter's key concepts?}{\begin{itemize}
            \item \textbf{6.1.I.C} - $\downarrow$ prices for goods $\to$ $\uparrow$ worker wages $\to$ $\uparrow$ std. of living, access to new goods, but also $\uparrow$ gap betw. rich/poor
            \item \textbf{6.1.I.A} - Cities brought Asian/S. and E. European/Afr. American migrants to escape poverty, religious persecution, restricted social mobility 
            \item \textbf{6.2.I.B} - Urban neighborhoods formed for specific ethnicities/races/classes
            \item \textbf{6.2.I.C} - International migration $\to$ debates over assimilation w/ many immigrants finding compromise
            \item \textbf{6.2.I.D} - Govt. thrived by providing social services to immigrants/poor in cities w/ unequal power distribution
            \item \textbf{6.2.I.E} - Corporations needed managers/clerical workers $\to$ middle class w/ leisure time
            \item \textbf{6.3.1.C} - Artists/critics like agrarians, utopians, socialists, Social Gospel advocates emphasized different visions for U.S. society
        \end{itemize}}
        \cornell[The Urbanization of America]{What characterized the growing urban migrations to America?}{\textbf{Immigrants, attracted to cities on new transportation lines for new opportunities, consisted in part of rural women and Southern blacks; most, however, were international migrants from Europe. They formed ethnic communities encouraging specific cultural values and reminders of home. Most immigrants sought to assimilate as "Americans"; many were forced by the conditions of American society. The nativist movement continued, but politicians never came to a united front.}}
        \cornell{What was most attractive about cities to migrants?}{\begin{itemize}
            \item Throughout nation, cities grew significantly w/ majority in urban areas by 1920
            \item Natural $\uparrow$ insignificant due to higher infant mortality, disease, reduced fertility rate; most were rural migrants seeking for new opportunities and safety
            \begin{itemize}
                \item Women able to behave in ways shunned by countryside for impropriety
                \item Gay men/lesbian women able to build growing culture (but still mostly hidden)
                \item Jobs far more plentiful, better-paying 
            \end{itemize}
            \item Transportation far more rapid: railroads and ocean liners $\to$ competitive shipping, easy transportation
        \end{itemize}
        \textbf{Cities grew primarily due to immigrants from other regions taking advantage of rapid railroad or ocean transportation to find new, more prosperous job opportunities or to find cultural havens to support lifestyles deemed improper in the countryside.}}
        \cornell{What characterized the migrations to cities?}{\begin{itemize}
            \item Urbanization $\to$ great geographical mobility w/ countless departing agricultural regions of East (some to West, but many to East/Midwest cities)
            \item Young rural women departed farms for cities because mechanization $\to$ increasingly dominated by single men; most home-produced goods available in widespread quantities
            \item Southern blacks departed to escape oppression, violence, debt, poverty despite limited opportunities in factory/professional work
            \begin{itemize}
                \item Generally janitors, domestic servants, low-paying jobs; women often outnumbered men
                \item Several cities established African-American communities, but most Afr. American migration occurred post-WWI
            \end{itemize}
            \item \underline{Largest source of pop. growth: immigrants from abroad}; \underline{Europe was greatest source}
            \begin{itemize}
                \item S./E. Europeans (Italians, Greeks, Slavs, Slovaks, Russian Jews, Armenians, etc.) came in largest numbers
                \item Most earlier N. European migrants had been wealthy/educated (like Germans/Scandinavians), but limited wealth of new migrants $\to$ mainly settled in cities as unskilled laborers
            \end{itemize}
        \end{itemize}
        \textbf{Several young rural women as well as southern blacks departed for cities to escape a changing job market offering them increasingly fewer opportunities; however, the majority of population growth came from immigrants, of which Eastern and Southern Europe were the greatest sources.}}
        \cornell{What characterized the ethnic communities within cities?}{\begin{itemize}
            \item Several cities had majority of pop. foreign-born immigrants + children; group greatly diverse within itself
            \begin{itemize}
                \item Most countries dominated by Italian/Spanish migrants but U.S. had no dominating group
            \end{itemize}
            \item Most struggled to adapt to urban life $\to$ ethnic communities formed to re-create features from homelands
            \begin{itemize}
                \item Sometimes ppl. from same province/town/village; often more diverse but always offered familiar settings (like newspapers in native languages, stores w/ native foods, religion, chance to communicate w/ relatives)
                \item Some groups advanced more rapidly than others (like Jews, Germans), perhaps due to ethnic neighborhoods enforcing specific cultural values (like Jews and education)
            \end{itemize}
        \end{itemize}
        \textbf{With most U.S. cities characterized by a large, very diverse group of foreign-born immigrants, ethnic communities often formed, offering their inhabitants a unique taste of culture from their homelands and eased the transition from rural to urban lifestyles. They are one explanation for the prosperity of certain groups over others (like the Jews compared to the Irish).}}
        \cornell{How did immigrants assimilate to city life?}{\begin{itemize}
            \item Nearly all immigrants relatively young (15-45), faced conflict between ethnic roots and assimilation
            \item Most sought to become "Americans" w/ many first-generation immigrants actively losing old culture, even more second-generation; often shunned parents who preserved old values
            \item Male-female relations far more liberal in U.S. than in many other nations $\to$ several families struggled with reduced control over daughers/wives, but family structures remained strong
            \item Assimilation often encouraged by native-born Americans w/ English at school/work, majority of stores selling American products, foreign church leaders encouraging American ways to attract native-born Americans 
        \end{itemize}
        \textbf{Most immigrants to America sought assimilation, often ridding themselves of past cultural ties to become "true Americans." Assimilation often conflicted with ingrained beliefs, notably the increasing American liberality of women. Assimilation was often a necessity: English remained essential, and diets and clothing habits often had to be adapted to align with what was accessible.}}
        \cornell{How did some exclude immigrants?}{\begin{itemize}
            \item Nativism continued w/ scorn for Europe on East Coast, Asia on West Coast due to fear of losing jobs
            \begin{itemize}
                \item Henry Bowers formed xenophobia/conspiracy-based \textbf{American Protective Association} to stop immigration w/ membership 500k by 1894
                \item Harvard alumni formed sophisticated \textbf{Immigration Restriction League} requiring literacy tests and other methods 
            \end{itemize}
            \item Politicians struggled w/ question of immigration: Chinese Exclusion Act in 1882, later "undesirables" banned along with tax, literacy requirement passed but vetoed in 1897
            \begin{itemize}
                \item Conflicted due to many native-born Americans welcoming cheap labor sources 
            \end{itemize}
        \end{itemize}
        \textbf{The nativist movement scorned new immigrants and attempted to exclude them through literacy tests and xenophobic organizations. Politicians never came to a united front on immigration; although the Chinese Exclusion Act as well as some immigrant taxes were passed, most were unable to bypass the desire of several Americans for cheap labor.}}
        \cornell[The Urban Landscape]{What were the major characteristics of how cities were laid out?}{\textbf{Cities were generally known for expansive public spaces like parks, libraries, and art galleries; reconstruction projects were undertaken to revamp old neighborhoods. The wealthiest lived in fashionable districts of the city; the moderately wealthy lived in suburbs and commuted to work; the vast majority lived in densely packed buildings with low rent and poor living conditions. New transportation systems developed, notably the cable car and the elevated railway, to serve the needs of the masses; skyscrapers began to emerge after the Civil War.}}
        \cornell{How did cities begin to develop public spaces?}{\begin{itemize}
            \item Urban parks reflected goal to limit congestion of city, starting w/ Olmsted/Vaux w/ Central Park to serve as truly nautral contrast 
            \item Buildings like libraries, museums, theatres, concert halls, art galleries developed as centers of culture and knowledge
            \begin{itemize}
                \item Driven by wealthiest inhabitants of cities, seeking to produce city fit for their lifestyle
                \item Philanthropy often drove even to create parks; had positive effects on cities overall
            \end{itemize}
            \item Growing size $\to$ large rebuilding projects like in older Euro. cities to wipe out old neighborhoods, create more elegant areas
            \begin{itemize}
                \item Most notable: "Great White City" for Chicago World's Fair; symmetrical city surrounding lagoon to contrast widespread assupmtion of disorder
                \item 1850s Boston: marshy tital land filled in over 50 yrs. to create "Back Bay," landfill producing new neighborhood
                \item Several other cities followed by filling in lakes to combat limited space; NYC relied more on expanding territory
            \end{itemize}
        \end{itemize}
        \textbf{Parks brought a natural contrast to city life, most notably New York's Central Park; furthermore, the philanthropic efforts of the wealthy, often out of a desire to create a more refined city, led to several public buildings like libraries and art galleries. Larger-scale rebuilding projects were undertaken too, with new land being annexed, old neighborhoods being torn down, and lakes being filled in.}}
        \cornell{How did the wealthy find suitable housing accomodations?}{\begin{itemize}
            \item Wealthiest lived in central mansions in central parts of city, like Nob Hill of SF, Fifth Avenue of NYC, Back Bay of Boston
            \item Moderately wealthy expanded to suburbs, relying on cheaper land to purchase larger properties, transportation systems like streetcars/trains to reach downtown 
            \begin{itemize}
                \item "Streetcar suburbs" popular in Boston, Chicago
                \item Surburbs emphasized potential to own large tracts of land
            \end{itemize}
        \end{itemize}
        \textbf{Although society's wealthiest lived in dedicated "fashion districts" in the city centers, the moderately wealthy members of society generally flocked to the suburbs for the cheap land and easy transportation to the city.}}
        \cornell{How did workers and the poor find housing accomodations in cities?}{\begin{itemize}
            \item Most urban residents rented apartments from landlords seeking to squeeze as many into small space $\to$ high density, but high demand $\to$ little bargaining power
            \item Little money invested to improve housings: profits were easy to come by; Southern cities known for Afr. Americans living in former slave quarters, narrow brick houses
            \item "Tenement": used to refer to slum dwellings despite advertising practical and cheap design; often windowless, lacking plumbing, cramming several into a room 
            \item \textbf{Jacob Riis}, Danish immigrant/NY reporter publicized poor conditions of slums; few could find viable, low-cost alternatives 
        \end{itemize}
        \textbf{The lower classes generally rented from landlords, who squeezed several into single units and invested little money for improvement: tenements, common in New York, were slum dwellings often lacking windows or plumbing. Few could find a viable solution without raising costs.}}
        \cornell{How did transportation systems connect the nation?}{\begin{itemize}
            \item Urban growth $\to$ great challenges w/ old streets narrow, dusted, muddy; paving unable to keep up with expansion
            \item Large volume of ppl. $\to$ need for mass transit
            \begin{itemize}
                \item Horse-driven streetcars too slow to be viable
                \item 1870: NYC opened elevated railway w/ filthy yet rapid trains speeding through city; cable cars soon explored
                \item 1897: Boston opened first subway by moving some trolley lines underground
                \item \textbf{John A. Roebling} completed the Brooklyn Bridge in NYC in the 1880s
            \end{itemize}
        \end{itemize}
        \textbf{Urban growth led to significant transportation challenges; the large volume of traffic meant that mass transit had to be explored. The most viable methods were elevated railways and cable cars; the subway grew in popularity by the turn of the century.}}
        \cornell{How did the skyscraper begin to advance in relevance?}{
        \textbf{Cities began to grow upward soon after the Civil War, with new cast iron and steel beam techniques emerging by the 1870s and elevators emerging by the 1850s. The Equitable Building of New York was 7.5 stories, and the tallest building grew significantly in the coming years.}}
        \cornell[Strains of Urban Life]{What events placed increasing tension on urban lifestyles?}{\textbf{Fire, disease, and environmental degradation through pollution ultimately encouraged action, from reformed city infrastructure to governmental changes. Poverty remained high with few relief organizations; violence and crime spiked in cities. City government were often dominated by political machines, backed up by the voting power of immigrants looking for basic needs.}}
        \cornell{How did fire and disease ravage modern cities?}{\textbf{Chicago and Boston experienced large fires in 1871; San Francisco saw an earthquake lead to mass fires. Though initially devastating, the encouraged the creation of fire departments as well as the rebuilding of cities with architectural innovations like fireproof buildings.}}
        \cornell{What factors caused the environment to degrade in cities?}{\begin{itemize}
            \item Few widespread environmental movements
            \item Great degradation to cities due to improper disposal of waste from domestic animals, city production (often in rivers) $\to$ pollution, poor air quality 
            \begin{itemize}
                \item Far better than most European cities, but respiratory infection common 
            \end{itemize}
            \item Reform movements emerged by early 20th c.
            \begin{itemize}
                \item Some devoted to sewage disposal for clean water 
                \item \textbf{Alice Hamilton} identified workplace pollution, facing off against factory owners to pass legislation to enforce proper disposal $\to$ Public Health Service, able to reduce workplace disease
                \begin{itemize}
                    \item Greatest legacy was making public health a responsibility of the fed. govt. 
                \end{itemize}
            \end{itemize}
        \end{itemize}
        \textbf{Cities experienced great environmental degradation due to poor waste management and reduced air quality from factories. Reform movements soon emerged to focus on workplace pollution, with Alice Hamilton forming the Public Health Service, making health a federal responsibility.}}
        \cornell{What characterized urban poverty?}{\begin{itemize}
            \item Impoverished rarely enjoyed significant relief because relief organizations dominated by middle class $\to$ most of aid went to those who were truly unable to recover
            \item \textbf{Salvation Army} began in 1879, but often dominated by religious tensions betw. Protestant philanthropists and Catholic immigrants
            \item Poor children most heavily noticed by reformers, but little action taken
        \end{itemize}
        \textbf{Dominated by the middle classes, relief organizations generally only assisted the most destitute, physically incapable of supporting themselves. Impoverished children living on streets were most heavily noticed.}}
        \cornell{How did cities spur a growth of crime?}{\begin{itemize}
            \item Most crime small: pickpocketing, artists, thieves; some more dangerous w/ high murder rate (but less than nonurban regions of South w/ lynching, instable western villages); immigrants often blamed
            \item Violence/crime $\to$ police forces from informal, private to professionalized, public; worked with attorneys but often $\to$ corruption, favoring whites and the wealthy
            \begin{itemize}
                \item Middle classes formed national guard groups to complement police work
            \end{itemize}
        \end{itemize}
        \textbf{Although most urban crime was minor, notably pickpocketing, the murder rate rose significantly; immigrants were often blamed. Police forces became far more formal and public with the rise of crime, but corruption was rampant.}}
        \cornell{How did the political machine aid immigrants and lower classes?}{\begin{itemize}
            \item Urban machine created by power vacuum from constant urban growth, weak govt., \underline{voting power of immigrants}
            \begin{itemize}
                \item "Bosses" relied on immigrant vote (often immigrants, notably Irish, themselves) $\to$ provided relief for target voters, saving from jail and rewarding w/ jobs
            \end{itemize}
            \item Machines important money-makers
            \begin{itemize}
                \item Some overt like bosses purchasing land in advance near secretive planned streetcar line
                \item Some covert like manipulation of contractors to build projects in specific locations, control of public utilities
                \item Most famously corrupt: \textbf{William M. Tweed} (eventually jailed)
            \end{itemize} 
            \item Middle classes criticized as obstacles to progress, but ultimately led to expansion of public works, infrastructure despite corrupt intentions
            \begin{itemize}
                \item Critics often able to mobilize public to reduce political power of bosses
            \end{itemize}
            \item Boss rule promoted by immigrants often less concerned w/ political morality, more w/ subsistence; connection betw. wealthy and organizations; weakness of city govt. 
        \end{itemize}
        \textbf{The political machine, driven by a growing power vacuum as well as the voting power of the immigrant population concerned with receiving basic needs, brought in great wealth for the "bosses" who led them but also for the city itself.}}
        \cornell[The Rise of Mass Consumption]{How did mass consumption grow as a significant social force?}{\textbf{With income rising significantly for nearly all Americans, consumerism emerged, with consumers, most often women, flocking to chain or department stores to purchase ready-made clothing and goods.}}
        \cornell{How did income patterns affect consumption of goods?}{\begin{itemize}
            \item Expansion of markets $\to$ rise in income for all (though uneven)
            \begin{itemize}
                \item Vast fortunes most notable result
                \item Rising wages for white-collar middle class workers, doctors, lawyers
                \item Working classes, notably iron and steel workers, saw significant hourly increases; female-dominated textile indusry saw little growth, as did South
            \end{itemize}
            \item Affordable products, ready-made clothing $\to$ private tailors, homespun clothes quickly replaced
            \begin{itemize}
                \item Led to greater focus on personal style: middle class/working class began to concern themselves w/ women's fashion, stylish wardrobes
            \end{itemize}
            \item Tin cans in 1880s $\to$ industry for packing/selling in refridgerated rail cars $\to$ improved diets
        \end{itemize}
        \textbf{The rise in income for nearly all Americans, rich and poor was accompanied by a rise in demand for stylish ready-made clothing as well as more balanced diets through tin cans.}}
        \cornell{How did chain stores and mail-order house dominate the consumer industry?}{\begin{itemize}
            \item Chains, stores creating several locations throughout a city or the nation (like the Great Atlantic and Pacific Tea Company in groceries), began to dominate smaller stores 
            \begin{itemize}
                \item Volume of goods $\to$ prices far lower $\to$ displaced merchants w/ great opposition, customers unable to resist 
            \end{itemize}
            \item Rural areas relied on mail-order houses (like Chicago's Montgomery Ward, Sears Roebuck), distributing catalogues o farmers offering wide arrays of goods
        \end{itemize}
        \textbf{Chain stores, those with several locations, offered goods at far lower prices due to their high volume, causing merchants to be rapidly displaced. Rural areas relied on mail-order houses, custom catalogues offering wide arrays of goods.}}
        \cornell{How did department stores transform buying habits?}{\begin{itemize}
            \item Began w/ Marshall Field in Chicago, soon NYC's Macy's, Brooklyn's Abraham and Straus, etc.
            \item Brought large array of products under single roof of elegant excitement/wonder w/ female salesclerks catering to female audience, offered lower prices due to volume
        \end{itemize}
        \textbf{Department stores were renowed for their elegant, refined female salesclerks and their large array of products under a single roof. Like chains, they were able to offer goods at far lower prices.}}
        \cornell{How did women grow as a consumer force?}{\begin{itemize}
            \item Rapid change in vogue women's styles, need to purchase food for family $\to$ far more purchases made than men 
            \item Women found new opportunities as salesclerks, waitresses
            \item Consumer protection movement emerged w/ \textbf{National Consumers League} fighting for women workers under common front of "consumers"
        \end{itemize}
        \textbf{Women purchased far more than men due to their rapidly changing styles as well as their need to purchase food. As they found jobs as salesclerks and waitresses, the consumer protection movement emerged to fight for female working conditions under a common front: consumerism.}}
        \cornell[Leisure in the Consumer Society]{How did notions of leisure change in a growingly consumerist society?}{\textbf{Far more time was available to all for public leisure; spectator sports like baseball, football, and boxing (particularly for lower classes) grew rapidly. Furthermore, ethnic theaters developed unique new musical and theatrical style. For the working class, most leisure time was spent engaging in lower-cost activities, like exploring the streets with friends or spending time at the saloon; the Fourth of July holiday was treasured as a symbol of ethnic pride. Mass communication, too, was stimulated by a uniform national press service through the telegraph. }}
        \cornell{What factors allowed for greater leisure time?}{\textbf{With larger blocks of time afforded to middle classes to rest (like evenings, weekends, some vacations) as well as reduced working hours in factories and farms (mechanization of agriculture), Americans began to find far more free time.}}
        \cornell{How was leisure redefined during the Gilded Age?}{\begin{itemize}
            \item Formerly viewed as laziness; most productive use of free time seen as "rest": spiritual contemplation for Sabbath
            \item Economic expansion, reducing working hours $\to$ increasingly viewed as normal part of life critical to emotional health 
            \item \textbf{Simon Patton} wrote texts like \textit{The Theory of Prosperity}, tying consumerism into new view of leisure 
            \item Public leisure often critical (going out amidst other people), like amusement parks (NYC's Coney Island), movies, sports
            \begin{itemize}
                \item Entertainment often broken by gender: sports/saloons mostly male activities, shopping/luncheonettes for females; and by race: theaters specific to ethnic backgrounds
                \item Public activity where classes/races/genders intermingled often $\to$ conflict, w/ wealthier scorning lower classes for sports in Central Park
            \end{itemize}
        \end{itemize}
        \textbf{Leisure transformed from an activity viewed as lazy and wasteful to a critical part of spiritual and emotional health. Public leisure was very common, with entertainment broken by gender, race, and class and conflict emerging when these various groups intermingled.}}
        \cornell{How did spectator sports grow in relevance over time?}{\begin{itemize}
            \item Baseball soon became national pastime w/ versions in 1830s/40s, but largest interest post-Civil War
            \begin{itemize}
                \item Cincinnati Red Stockings first salaried team in 1869; National League created in 1876 w/ competing American Association which collapsed but replaced in 1901 by American League
                \item 1903: first World Series saw American League Boston Red Sox beat National League Pittsburgh Pirates; great crowds
            \end{itemize}
            \item Football initially for elite males in universities (first game betw. Princeton and Rutgers in 1869), with early football resembling rugby but modernization $\to$ more rules
            \begin{itemize}
                \item Spread to midwestern state universities, soon beginning to dominate sport
                \item Unprofessionalism emerged, w/ "ringers" (athletes not part of school); Big Ten conference in 1896 aimed to formalize rules of eligibility
                \item Football known for signif. violence w/ several injuries, deaths $\to$ President Theodore Roosevelt called convention, forming NCAA (National College Athletic Assocation) to revise rules for safety
            \end{itemize}
            \item Basketball invented by Canadian working in MA 
            \item Boxing became increasingly popular, slowly overcoming great stigma w/ safer rules (remained illegal in several states)
            \item Horse-racing also grew w/ Kentucky Derby
            \item Sports connected w/ gambling: controversy in 1919 when Chicago White Sox threw World Series for gambling purposes; boxing often saw "fixed" matches 
            \item Most spectator sports male-dominated, but tennis/golf participated by wealthy men/women; women's colleges began to introduce more intensive sports
        \end{itemize}
        \textbf{Spectator sports like baseball, the most significant American pastime, football, known for violence and unprofessionalism, boxing, overcoming the stigma of the lower classes, and horse-racing all grew over time. Sports were often connected with gambling, leading to several matches being controversially "thrown." Women also began to take a greater involvement in specific sports like golf and tennis.}}
        \cornell{How did the performing arts grow in the Gilded Age?}{\begin{itemize}
            \item Theaters usually ethnic-driven (like Italian theaters drawing on Italian opera; Yiddish theaters building on American Jewish experience $\to$ large number of talented artists went to English-speaking theater)
            \item Musical comedy emerged, evolving from Euroepan comic operettas, pioneered by \textbf{George M. Cohan}; \textbf{Irving Berlin} in Yiddish theater
            \item \textbf{Vaudeville} most popular in early 20th c. due to ease of access and low costs due to simple combination of various intriguing acts (magicians, jugglers, musicians, comedians)
            \begin{itemize}
                \item Open to black performers 
                \item Some transformed into more elaborate shows (\textbf{Florenz Ziegfeld})
            \end{itemize}
        \end{itemize}
        \textbf{Theaters were predominantly divided into ethnic groups, often bringing in unique elements from the homeland. Musical comedy and vaudeville (an inexpensive combination of musical/magic/juggling acts) were the most popular forms.}}
        \cornell{How did film grow as a source of entertainment?}{\begin{itemize}
            \item Movies, created by Thomas Edison (1880s), began w/ smaller "peep shows" but transformed into larger screens
            \item By 1900, movies greatly popular: initially plotless, showing off technology; silent epics by \textbf{D.W. Griffith} brought into a new era of significant plot, meaning (like \textit{Birth of a Nation} celebrating KKK)
        \end{itemize}
        \textbf{Created by Thomas Edison, moving pictures were initially plotless, displaying natural imagery to show off the new technology. However, D.W. Griffith brought film into a new era with far more meaningful pieces.}}
        \cornell{How did the working class spend their leisure time?}{\begin{itemize}
            \item Workers often spent leisure time walking on streets w/ friends, watching entertainers, or walking alone due to low cost
            \item Neighborhood saloon attracted many men due to regular friend circle in specific ethnic groups
            \begin{itemize}
                \item Saloonkeepers became political mbosses due to great influence over men in neighborhood $\to$ Anti-Saloon League (primarily temperance) attacked for political power 
                \item Often places of crime
            \end{itemize}
            \item Boxing most popular for working-class men due to opportunity to show off strength
        \end{itemize}
        \textbf{Workers often simply explored the streets with friends, watching entertainers due to the little cost; saloons, too, were attractive for men, but they often became political machines and places of significant crime. Boxing was the most popular sport for the working class.}}
        \cornell{How was the Fourth of July a significant holiday?}{\begin{itemize}
            \item Few holidays (often 7-day workweeks) $\to$ celebrations for 4th of July greatly celebrated among working-class communities
            \begin{itemize}
                \item Ancient Order of Hibernians (Irish org.) in Worcester, MA created large picnics; competed against Irish temperance organizations
            \end{itemize}
            \item Came to represent ethnic pride alongside nation's independence
        \end{itemize}
        \textbf{For workers with few days off from work, Fourth of July celebraions were greatly valued by workers; they came to represent ethnic pride.}}
        \cornell{How did urban society stimulate mass communication?}{\begin{itemize}
            \item 1870-1910: rapid raise in wages of reporters, great spread of newspapers 
            \item National press service used telegraph to distribute news throughout nation to produce uniform newspapers
            \begin{itemize}
                \item \textbf{William Randolph Hearst} controlled several newspapers, magazines; published "yellow journalism" w/ graphic style designed to appeal to mass readers
                \item Magazines more designed for mass audiences (like \textit{Ladies' Home Journal})
            \end{itemize}
        \end{itemize}
        \textbf{Mass communication rose greatly through the spread of newspapers and uniform news with companies creating national press services through telegraphs.}}
        \cornell[High Culture in the Age of the City]{What were the key cultural elements advanced in the urban lifestyle?}{\textbf{Literature and art emerged to accurately depict the state of cities, showing the plight of the working class and their great struggles. Darwinism, too, became a notable social force, creating a split between urban and rural and spawning new movements like pragmatism, emphasizing the constant application of the scientific method. Educational opportunities expanded, too, particularly in urban public education as well as higher education through the Morrill Land Grant Act. Higher education opportunities for women remained limited, but an increasing number of institutions emerged devoted to female higher education: coeducational schools, female colleges on existing campuses, or institutions entirely dedicated to female education. }}
        \cornell{What were the key elements of literature created for urban audiences?}{\begin{itemize}
            \item Stephen Crane recreated urban society in \textit{The Red Badge of Courage}, showing plight of working class; \textit{Maggie: A Girl of the Streets}, showing slum life
            \item Theodore Dreiser encouraged abandonment of genteel traditions, instead focus on social injustices
            \item Frank Norris: \textit{The Octopus}, showing struggle betw. CA wheat farmers and railroad; Upton Sinclair: \textit{The Jungle}, showing cons of capitalism in meatpacking industry; Kate Chopin: \textit{The Awakening}, showing wife abandoning family; William Dean Howells: \textit{The Rise of Silas Lapham} to show corruption of wealth
            \item Henry Adams withdrew from American society, showing disillusionment and inability to relate w/ society in \textit{The Education of Henry Adams}
            \item Book clubs, mostly for women (both white and Afr. American) grew in popularity
        \end{itemize}
        \textbf{American urban literature generally focused on the changes in society unfolding in cities, often encouraging the raw struggles of the working class and impoverished in cities in graphic detail. Some chose to denounce urban society despite being surrounded by it. The expansion of literature led to the creation of many book clubs.}}
        \cornell{How did the urban lifestyle transform American art?}{\begin{itemize}
            \item Several artists broke from Old World traditions, w/ Winslow Homer known for American-style New England Paintings, James McNeil Whistler for east Asian concepts in American art
            \item Several artists broke free of academic style, instead focusing on grim modern life (\textbf{Ashcan School}), showing slums, loneliness of cities through expressionism/abstraction w/ constant goal to find new forms of art (\textit{ex}: John Sloan, George Bellows)
            \begin{itemize}
                \item Known today as modernist movement, seeing counterparts in other parts of life: general rejection of glorified elite civilization, instead showing standards of average person 
                \item Focus on individual creativity
            \end{itemize}
        \end{itemize}
        \textbf{The greatest artistic breakthroughs were in the modernist movement, where the Ashcan school showed city life as it was rather than glorifying the genteel elite lifestyles around which so many past artists centered their work.}}
        \cornell{How did Darwinism transform society?}{\begin{itemize}
            \item Concept of natural selection, divine plan challenged story of Creation w/ life representing random process
            \item Initially resisted by even scientists, but soon accepted by elite classes, even middle-class Protestant leaders 
            \item Led to schism betw. rural/urban: respected scientists/schools in cities came to except, while rural regions remained enshrined in traditional Creationist values
            \item Led to Social Darwinism of William Graham Sumner, pragmatism (all guidance should come from personal scientific exploraition, not long-held ideas; best institutions based on experience) of Harvard's William James
            \begin{itemize}
                \item Sociologists (Lester Frank Ward, Edward A. Ross) applying sci. method to political issues
                \item Historians (Frederick Jackson Turner, Charles Beard) to explain past events/reactions, expaining economy > religion in influencing history
                \item John Dewey created more democratic approach to education
            \end{itemize}
            \item Darwinism encouraged anthropology $\to$ small minority group began to view Native Americans as equals w/ social norms worthy of respect
        \end{itemize}
        \textbf{Darwinism, though initially post, soon came to be accepted by most urban inhabitants, including Protestant leaders; rural areas remained left out. It spawned several other movements, most notably pragmatism, which encouraged the scientific method for the study of all things, including sociology, history, education, and anthropology.}}
        \cornell{How did several seek universal schooling?}{\begin{itemize}
            \item Public education expanded \underline{rapidly}, most notably in public high schools; laws put in place in 31 states by 1900 requiring school attendance; rural areas still behind, w/ blacks lacking any school access
            \item Reformers sought to "civilize" tribes w/ residential schools; Carlisle School attempted to assimilate but failed due to little funding, native resistance
            \item Colleges expanded rapidly w/ Morrill Land Grant Act $\to$ states able to easily establish colleges; financial tycoons like Rockefeller, Carnegie donated to existing schools or founded new universities
            \begin{itemize}
                \item Seen most in South/West w/ 69 institutions created in final decades of century 
            \end{itemize}
        \end{itemize}
        \textbf{Public education expanded rapidly, with several compulsory attendance laws and public high schools proliferating throughout the nation. Native American residential schools largely failed; colleges, however, expanded rapidly due to external funding and the Morrill Land Grant Act, most notably in the South and the West.}}
        \cornell{What educational opportunities were afforded to women?}{\begin{itemize}
            \item Few higher education opportunities: post-Civil War, only 3 co-ed universities; land-grant colleges $\to$ some development but relatively few; Mount Holyoke became fully-fledged female institution 
            \item Some private universities created dedicated women's colleges on campuses; importance stressed to ensure women would be treated equally
            \item Represented emergence of women's community w/ most faculty members unmarried $\to$ sorority, close bonds; several did not marry, those who did marrying much later
            \begin{itemize}
                \item Felt liberated by education w/ less urgency to marry $\to$ independence from man 
            \end{itemize}
        \end{itemize}
        \textbf{Few opportunities were afforded to women in higher education; few universities were coeducational, but some institutions began to emerge for females only (like Mount Holyoke, Bryn Mawr, Goucher, etc.), and some colleges formed female colleges on their campus. Women who attended college often felt liberated by the experience, with a significantly lower rate of marriage.}}
    \end{document}