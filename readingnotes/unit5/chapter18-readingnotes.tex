\documentclass[a4paper]{article}
    \input{../notesheader.tex}
    \newcommand{\chapternumber}{17}
    \newcommand{\chaptertitle}{Industrial Supremacy}
    \title{\vspace{-3em}
    \begin{tcolorbox}
    \Huge\sffamily \begin{center} AP US History  \\
    \LARGE Chapter \chapternumber$\text{ }$- \chaptertitle \\
    \Large Finn Frankis \end{center} 
    \end{tcolorbox}
    \vspace{-3em}
    }
    \date{}
    \author{}
    
    \begin{document}
        \maketitle
        \SetBgContents{\rule[0em]{4pt}{\textheight}}
        \cornell[Key Concepts]{What are this chapter's key concepts?}{\begin{itemize}
            \item \textbf{6.1.I.C} - $\downarrow$ prices for goods $\to$ $\uparrow$ worker wages $\to$ $\uparrow$ std. of living, access to new goods, but also $\uparrow$ gap betw. rich/poor
            \item \textbf{6.1.I.A} - Cities brought Asian/S. and E. European/Afr. American migrants to escape poverty, religious persecution, restricted social mobility 
            \item \textbf{6.2.I.B} - Urban neighborhoods formed for specific ethnicities/races/classes
            \item \textbf{6.2.I.C} - International migration $\to$ debates over assimilation w/ many immigrants finding compromise
            \item \textbf{6.2.I.D} - Govt. thrived by providing social services to immigrants/poor in cities w/ unequal power distribution
            \item \textbf{6.2.I.E} - Corporations needed managers/clerical workers $\to$ middle class w/ leisure time
            \item \textbf{6.3.1.C} - Artists/critics like agrarians, utopians, socialists, Social Gospel advocates emphasized different visions for U.S. society
        \end{itemize}}
        \cornell[The Urbanization of America]{What characterized the growing urban migrations to America?}{\textbf{Immigrants, attracted to cities on new transportation lines for new opportunities, consisted in part of rural women and Southern blacks; most, however, were international migrants from Europe. They formed ethnic communities encouraging specific cultural values and reminders of home. Most immigrants sought to assimilate as "Americans"; many were forced by the conditions of American society. The nativist movement continued, but politicians never came to a united front.}}
        \cornell{What was most attractive about cities to migrants?}{\begin{itemize}
            \item Throughout nation, cities grew significantly w/ majority in urban areas by 1920
            \item Natural $\uparrow$ insignificant due to higher infant mortality, disease, reduced fertility rate; most were rural migrants seeking for new opportunities and safety
            \begin{itemize}
                \item Women able to behave in ways shunned by countryside for impropriety
                \item Gay men/lesbian women able to build growing culture (but still mostly hidden)
                \item Jobs far more plentiful, better-paying 
            \end{itemize}
            \item Transportation far more rapid: railroads and ocean liners $\to$ competitive shipping, easy transportation
        \end{itemize}
        \textbf{Cities grew primarily due to immigrants from other regions taking advantage of rapid railroad or ocean transportation to find new, more prosperous job opportunities or to find cultural havens to support lifestyles deemed improper in the countryside.}}
        \cornell{What characterized the migrations to cities?}{\begin{itemize}
            \item Urbanization $\to$ great geographical mobility w/ countless departing agricultural regions of East (some to West, but many to East/Midwest cities)
            \item Young rural women departed farms for cities because mechanization $\to$ increasingly dominated by single men; most home-produced goods available in widespread quantities
            \item Southern blacks departed to escape oppression, violence, debt, poverty despite limited opportunities in factory/professional work
            \begin{itemize}
                \item Generally janitors, domestic servants, low-paying jobs; women often outnumbered men
                \item Several cities established African-American communities, but most Afr. American migration occurred post-WWI
            \end{itemize}
            \item \underline{Largest source of pop. growth: immigrants from abroad}; \underline{Europe was greatest source}
            \begin{itemize}
                \item S./E. Europeans (Italians, Greeks, Slavs, Slovaks, Russian Jews, Armenians, etc.) came in largest numbers
                \item Most earlier N. European migrants had been wealthy/educated (like Germans/Scandinavians), but limited wealth of new migrants $\to$ mainly settled in cities as unskilled laborers
            \end{itemize}
        \end{itemize}
        \textbf{Several young rural women as well as southern blacks departed for cities to escape a changing job market offering them increasingly fewer opportunities; however, the majority of population growth came from immigrants, of which Eastern and Southern Europe were the greatest sources.}}
        \cornell{What characterized the ethnic communities within cities?}{\begin{itemize}
            \item Several cities had majority of pop. foreign-born immigrants + children; group greatly diverse within itself
            \begin{itemize}
                \item Most countries dominated by Italian/Spanish migrants but U.S. had no dominating group
            \end{itemize}
            \item Most struggled to adapt to urban life $\to$ ethnic communities formed to re-create features from homelands
            \begin{itemize}
                \item Sometimes ppl. from same province/town/village; often more diverse but always offered familiar settings (like newspapers in native languages, stores w/ native foods, religion, chance to communicate w/ relatives)
                \item Some groups advanced more rapidly than others (like Jews, Germans), perhaps due to ethnic neighborhoods enforcing specific cultural values (like Jews and education)
            \end{itemize}
        \end{itemize}
        \textbf{With most U.S. cities characterized by a large, very diverse group of foreign-born immigrants, ethnic communities often formed, offering their inhabitants a unique taste of culture from their homelands and eased the transition from rural to urban lifestyles. They are one explanation for the prosperity of certain groups over others (like the Jews compared to the Irish).}}
        \cornell{How did immigrants assimilate to city life?}{\begin{itemize}
            \item Nearly all immigrants relatively young (15-45), faced conflict between ethnic roots and assimilation
            \item Most sought to become "Americans" w/ many first-generation immigrants actively losing old culture, even more second-generation; often shunned parents who preserved old values
            \item Male-female relations far more liberal in U.S. than in many other nations $\to$ several families struggled with reduced control over daughers/wives, but family structures remained strong
            \item Assimilation often encouraged by native-born Americans w/ English at school/work, majority of stores selling American products, foreign church leaders encouraging American ways to attract native-born Americans 
        \end{itemize}
        \textbf{Most immigrants to America sought assimilation, often ridding themselves of past cultural ties to become "true Americans." Assimilation often conflicted with ingrained beliefs, notably the increasing American liberality of women. Assimilation was often a necessity: English remained essential, and diets and clothing habits often had to be adapted to align with what was accessible.}}
        \cornell{How did some exclude immigrants?}{\begin{itemize}
            \item Nativism continued w/ scorn for Europe on East Coast, Asia on West Coast due to fear of losing jobs
            \begin{itemize}
                \item Henry Bowers formed xenophobia/conspiracy-based \textbf{American Protective Association} to stop immigration w/ membership 500k by 1894
                \item Harvard alumni formed sophisticated \textbf{Immigration Restriction League} requiring literacy tests and other methods 
            \end{itemize}
            \item Politicians struggled w/ question of immigration: Chinese Exclusion Act in 1882, later "undesirables" banned along with tax, literacy requirement passed but vetoed in 1897
            \begin{itemize}
                \item Conflicted due to many native-born Americans welcoming cheap labor sources 
            \end{itemize}
        \end{itemize}
        \textbf{The nativist movement scorned new immigrants and attempted to exclude them through literacy tests and xenophobic organizations. Politicians never came to a united front on immigration; although the Chinese Exclusion Act as well as some immigrant taxes were passed, most were unable to bypass the desire of several Americans for cheap labor.}}
    \end{document}