\documentclass[a4paper]{article}
    \usepackage[T1]{fontenc}
    \usepackage{tcolorbox}
    \usepackage{amsmath}
    \tcbuselibrary{skins}
    
    \usepackage{background}
    \SetBgScale{1}
    \SetBgAngle{0}
    \SetBgColor{red}
    \SetBgContents{\rule[0em]{4pt}{\textheight}}
    \SetBgHshift{-2.3cm}
    \SetBgVshift{0cm}
    \usepackage[margin=2cm]{geometry} 
    
    \makeatletter
    \def\cornell{\@ifnextchar[{\@with}{\@without}}
    \def\@with[#1]#2#3{
    \begin{tcolorbox}[enhanced,colback=gray,colframe=black,fonttitle=\large\bfseries\sffamily,sidebyside=true, nobeforeafter,before=\vfil,after=\vfil,colupper=blue,sidebyside align=top, lefthand width=.3\textwidth,
    opacityframe=0,opacityback=.3,opacitybacktitle=1, opacitytext=1,
    segmentation style={black!55,solid,opacity=0,line width=3pt},
    title=#1
    ]
    \begin{tcolorbox}[colback=red!05,colframe=red!25,sidebyside align=top,
    width=\textwidth,nobeforeafter]#2\end{tcolorbox}%
    \tcblower
    \sffamily
    \begin{tcolorbox}[colback=blue!05,colframe=blue!10,width=\textwidth,nobeforeafter]
    #3
    \end{tcolorbox}
    \end{tcolorbox}
    }
    \def\@without#1#2{
    \begin{tcolorbox}[enhanced,colback=white!15,colframe=white,fonttitle=\bfseries,sidebyside=true, nobeforeafter,before=\vfil,after=\vfil,colupper=blue,sidebyside align=top, lefthand width=.3\textwidth,
    opacityframe=0,opacityback=0,opacitybacktitle=0, opacitytext=1,
    segmentation style={black!55,solid,opacity=0,line width=3pt}
    ]
    
    \begin{tcolorbox}[colback=red!05,colframe=red!25,sidebyside align=top,
    width=\textwidth,nobeforeafter]#1\end{tcolorbox}%
    \tcblower
    \sffamily
    \begin{tcolorbox}[colback=blue!05,colframe=blue!10,width=\textwidth,nobeforeafter]
    #2
    \end{tcolorbox}
    \end{tcolorbox}
    }
    \makeatother

    \parindent=0pt
    \newcommand{\chapternumber}{17}
    \newcommand{\chaptertitle}{Industrial Supremacy}
    \title{\vspace{-3em}
    \begin{tcolorbox}
    \Huge\sffamily \begin{center} AP US History  \\
    \LARGE Chapter \chapternumber$\text{ }$- \chaptertitle \\
    \Large Finn Frankis \end{center} 
    \end{tcolorbox}
    \vspace{-3em}
    }
    \date{}
    \author{}
    
    \begin{document}
        \maketitle
        \SetBgContents{\rule[0em]{4pt}{\textheight}}
        \cornell[Key Concepts]{What are this chapter's key concepts?}{\begin{itemize}
            \item \textbf{6.1.I.C} - $\downarrow$ prices for goods $\to$ $\uparrow$ worker wages $\to$ $\uparrow$ std. of living, access to new goods, but also $\uparrow$ gap betw. rich/poor
            \item \textbf{6.1.I.A} - Cities brought Asian/S. and E. European/Afr. American migrants to escape poverty, religious persecution, restricted social mobility 
            \item \textbf{6.2.I.B} - Urban neighborhoods formed for specific ethnicities/races/classes
            \item \textbf{6.2.I.C} - International migration $\to$ debates over assimilation w/ many immigrants finding compromise
            \item \textbf{6.2.I.D} - Govt. thrived by providing social services to immigrants/poor in cities w/ unequal power distribution
            \item \textbf{6.2.I.E} - Corporations needed managers/clerical workers $\to$ middle class w/ leisure time
            \item \textbf{6.3.1.C} - Artists/critics like agrarians, utopians, socialists, Social Gospel advocates emphasized different visions for U.S. society
        \end{itemize}}
        \cornell[The Urbanization of America]{What characterized the growing urban migrations to America?}{\textbf{Immigrants, attracted to cities on new transportation lines for new opportunities, consisted in part of rural women and Southern blacks; most, however, were international migrants from Europe. They formed ethnic communities encouraging specific cultural values and reminders of home. Most immigrants sought to assimilate as "Americans"; many were forced by the conditions of American society. The nativist movement continued, but politicians never came to a united front.}}
        \cornell{What was most attractive about cities to migrants?}{\begin{itemize}
            \item Throughout nation, cities grew significantly w/ majority in urban areas by 1920
            \item Natural $\uparrow$ insignificant due to higher infant mortality, disease, reduced fertility rate; most were rural migrants seeking for new opportunities and safety
            \begin{itemize}
                \item Women able to behave in ways shunned by countryside for impropriety
                \item Gay men/lesbian women able to build growing culture (but still mostly hidden)
                \item Jobs far more plentiful, better-paying 
            \end{itemize}
            \item Transportation far more rapid: railroads and ocean liners $\to$ competitive shipping, easy transportation
        \end{itemize}
        \textbf{Cities grew primarily due to immigrants from other regions taking advantage of rapid railroad or ocean transportation to find new, more prosperous job opportunities or to find cultural havens to support lifestyles deemed improper in the countryside.}}
        \cornell{What characterized the migrations to cities?}{\begin{itemize}
            \item Urbanization $\to$ great geographical mobility w/ countless departing agricultural regions of East (some to West, but many to East/Midwest cities)
            \item Young rural women departed farms for cities because mechanization $\to$ increasingly dominated by single men; most home-produced goods available in widespread quantities
            \item Southern blacks departed to escape oppression, violence, debt, poverty despite limited opportunities in factory/professional work
            \begin{itemize}
                \item Generally janitors, domestic servants, low-paying jobs; women often outnumbered men
                \item Several cities established African-American communities, but most Afr. American migration occurred post-WWI
            \end{itemize}
            \item \underline{Largest source of pop. growth: immigrants from abroad}; \underline{Europe was greatest source}
            \begin{itemize}
                \item S./E. Europeans (Italians, Greeks, Slavs, Slovaks, Russian Jews, Armenians, etc.) came in largest numbers
                \item Most earlier N. European migrants had been wealthy/educated (like Germans/Scandinavians), but limited wealth of new migrants $\to$ mainly settled in cities as unskilled laborers
            \end{itemize}
        \end{itemize}
        \textbf{Several young rural women as well as southern blacks departed for cities to escape a changing job market offering them increasingly fewer opportunities; however, the majority of population growth came from immigrants, of which Eastern and Southern Europe were the greatest sources.}}
        \cornell{What characterized the ethnic communities within cities?}{\begin{itemize}
            \item Several cities had majority of pop. foreign-born immigrants + children; group greatly diverse within itself
            \begin{itemize}
                \item Most countries dominated by Italian/Spanish migrants but U.S. had no dominating group
            \end{itemize}
            \item Most struggled to adapt to urban life $\to$ ethnic communities formed to re-create features from homelands
            \begin{itemize}
                \item Sometimes ppl. from same province/town/village; often more diverse but always offered familiar settings (like newspapers in native languages, stores w/ native foods, religion, chance to communicate w/ relatives)
                \item Some groups advanced more rapidly than others (like Jews, Germans), perhaps due to ethnic neighborhoods enforcing specific cultural values (like Jews and education)
            \end{itemize}
        \end{itemize}
        \textbf{With most U.S. cities characterized by a large, very diverse group of foreign-born immigrants, ethnic communities often formed, offering their inhabitants a unique taste of culture from their homelands and eased the transition from rural to urban lifestyles. They are one explanation for the prosperity of certain groups over others (like the Jews compared to the Irish).}}
        \cornell{How did immigrants assimilate to city life?}{\begin{itemize}
            \item Nearly all immigrants relatively young (15-45), faced conflict between ethnic roots and assimilation
            \item Most sought to become "Americans" w/ many first-generation immigrants actively losing old culture, even more second-generation; often shunned parents who preserved old values
            \item Male-female relations far more liberal in U.S. than in many other nations $\to$ several families struggled with reduced control over daughers/wives, but family structures remained strong
            \item Assimilation often encouraged by native-born Americans w/ English at school/work, majority of stores selling American products, foreign church leaders encouraging American ways to attract native-born Americans 
        \end{itemize}
        \textbf{Most immigrants to America sought assimilation, often ridding themselves of past cultural ties to become "true Americans." Assimilation often conflicted with ingrained beliefs, notably the increasing American liberality of women. Assimilation was often a necessity: English remained essential, and diets and clothing habits often had to be adapted to align with what was accessible.}}
        \cornell{How did some exclude immigrants?}{\begin{itemize}
            \item Nativism continued w/ scorn for Europe on East Coast, Asia on West Coast due to fear of losing jobs
            \begin{itemize}
                \item Henry Bowers formed xenophobia/conspiracy-based \textbf{American Protective Association} to stop immigration w/ membership 500k by 1894
                \item Harvard alumni formed sophisticated \textbf{Immigration Restriction League} requiring literacy tests and other methods 
            \end{itemize}
            \item Politicians struggled w/ question of immigration: Chinese Exclusion Act in 1882, later "undesirables" banned along with tax, literacy requirement passed but vetoed in 1897
            \begin{itemize}
                \item Conflicted due to many native-born Americans welcoming cheap labor sources 
            \end{itemize}
        \end{itemize}
        \textbf{The nativist movement scorned new immigrants and attempted to exclude them through literacy tests and xenophobic organizations. Politicians never came to a united front on immigration; although the Chinese Exclusion Act as well as some immigrant taxes were passed, most were unable to bypass the desire of several Americans for cheap labor.}}
        \cornell[The Urban Landscape]{What were the major characteristics of how cities were laid out?}{\textbf{Cities were generally known for expansive public spaces like parks, libraries, and art galleries; reconstruction projects were undertaken to revamp old neighborhoods. The wealthiest lived in fashionable districts of the city; the moderately wealthy lived in suburbs and commuted to work; the vast majority lived in densely packed buildings with low rent and poor living conditions. New transportation systems developed, notably the cable car and the elevated railway, to serve the needs of the masses; skyscrapers began to emerge after the Civil War.}}
        \cornell{How did cities begin to develop public spaces?}{\begin{itemize}
            \item Urban parks reflected goal to limit congestion of city, starting w/ Olmsted/Vaux w/ Central Park to serve as truly nautral contrast 
            \item Buildings like libraries, museums, theatres, concert halls, art galleries developed as centers of culture and knowledge
            \begin{itemize}
                \item Driven by wealthiest inhabitants of cities, seeking to produce city fit for their lifestyle
                \item Philanthropy often drove even to create parks; had positive effects on cities overall
            \end{itemize}
            \item Growing size $\to$ large rebuilding projects like in older Euro. cities to wipe out old neighborhoods, create more elegant areas
            \begin{itemize}
                \item Most notable: "Great White City" for Chicago World's Fair; symmetrical city surrounding lagoon to contrast widespread assupmtion of disorder
                \item 1850s Boston: marshy tital land filled in over 50 yrs. to create "Back Bay," landfill producing new neighborhood
                \item Several other cities followed by filling in lakes to combat limited space; NYC relied more on expanding territory
            \end{itemize}
        \end{itemize}
        \textbf{Parks brought a natural contrast to city life, most notably New York's Central Park; furthermore, the philanthropic efforts of the wealthy, often out of a desire to create a more refined city, led to several public buildings like libraries and art galleries. Larger-scale rebuilding projects were undertaken too, with new land being annexed, old neighborhoods being torn down, and lakes being filled in.}}
        \cornell{How did the wealthy find suitable housing accomodations?}{\begin{itemize}
            \item Wealthiest lived in central mansions in central parts of city, like Nob Hill of SF, Fifth Avenue of NYC, Back Bay of Boston
            \item Moderately wealthy expanded to suburbs, relying on cheaper land to purchase larger properties, transportation systems like streetcars/trains to reach downtown 
            \begin{itemize}
                \item "Streetcar suburbs" popular in Boston, Chicago
                \item Surburbs emphasized potential to own large tracts of land
            \end{itemize}
        \end{itemize}
        \textbf{Although society's wealthiest lived in dedicated "fashion districts" in the city centers, the moderately wealthy members of society generally flocked to the suburbs for the cheap land and easy transportation to the city.}}
        \cornell{How did workers and the poor find housing accomodations in cities?}{\begin{itemize}
            \item Most urban residents rented apartments from landlords seeking to squeeze as many into small space $\to$ high density, but high demand $\to$ little bargaining power
            \item Little money invested to improve housings: profits were easy to come by; Southern cities known for Afr. Americans living in former slave quarters, narrow brick houses
            \item "Tenement": used to refer to slum dwellings despite advertising practical and cheap design; often windowless, lacking plumbing, cramming several into a room 
            \item \textbf{Jacob Riis}, Danish immigrant/NY reporter publicized poor conditions of slums; few could find viable, low-cost alternatives 
        \end{itemize}
        \textbf{The lower classes generally rented from landlords, who squeezed several into single units and invested little money for improvement: tenements, common in New York, were slum dwellings often lacking windows or plumbing. Few could find a viable solution without raising costs.}}
        \cornell{How did transportation systems connect the nation?}{\begin{itemize}
            \item Urban growth $\to$ great challenges w/ old streets narrow, dusted, muddy; paving unable to keep up with expansion
            \item Large volume of ppl. $\to$ need for mass transit
            \begin{itemize}
                \item Horse-driven streetcars too slow to be viable
                \item 1870: NYC opened elevated railway w/ filthy yet rapid trains speeding through city; cable cars soon explored
                \item 1897: Boston opened first subway by moving some trolley lines underground
                \item \textbf{John A. Roebling} completed the Brooklyn Bridge in NYC in the 1880s
            \end{itemize}
        \end{itemize}
        \textbf{Urban growth led to significant transportation challenges; the large volume of traffic meant that mass transit had to be explored. The most viable methods were elevated railways and cable cars; the subway grew in popularity by the turn of the century.}}
        \cornell{How did the skyscraper begin to advance in relevance?}{
            \textbf{Cities began to grow upward soon after the Civil War, with new cast iron and steel beam techniques emerging by the 1870s and elevators emerging by the 1850s. The Equitable Building of New York was 7.5 stories, and the tallest building grew significantly in the coming years.}}
            \cornell[Strains of Urban Life]{What events placed increasing tension on urban lifestyles?}{\textbf{Fire, disease, and environmental degradation through pollution ultimately encouraged action, from reformed city infrastructure to governmental changes. Poverty remained high with few relief organizations; violence and crime spiked in cities. City government were often dominated by political machines, backed up by the voting power of immigrants looking for basic needs.}}
            \cornell{How did fire and disease ravage modern cities?}{\textbf{Chicago and Boston experienced large fires in 1871; San Francisco saw an earthquake lead to mass fires. Though initially devastating, the encouraged the creation of fire departments as well as the rebuilding of cities with architectural innovations like fireproof buildings.}}
            \cornell{What factors caused the environment to degrade in cities?}{\begin{itemize}
                \item Few widespread environmental movements
                \item Great degradation to cities due to improper disposal of waste from domestic animals, city production (often in rivers) $\to$ pollution, poor air quality 
                \begin{itemize}
                    \item Far better than most European cities, but respiratory infection common 
                \end{itemize}
                \item Reform movements emerged by early 20th c.
                \begin{itemize}
                    \item Some devoted to sewage disposal for clean water 
                    \item \textbf{Alice Hamilton} identified workplace pollution, facing off against factory owners to pass legislation to enforce proper disposal $\to$ Public Health Service, able to reduce workplace disease
                    \begin{itemize}
                        \item Greatest legacy was making public health a responsibility of the fed. govt. 
                    \end{itemize}
                \end{itemize}
            \end{itemize}
            \textbf{Cities experienced great environmental degradation due to poor waste management and reduced air quality from factories. Reform movements soon emerged to focus on workplace pollution, with Alice Hamilton forming the Public Health Service, making health a federal responsibility.}}
            \cornell{What characterized urban poverty?}{\begin{itemize}
                \item Impoverished rarely enjoyed significant relief because relief organizations dominated by middle class $\to$ most of aid went to those who were truly unable to recover
                \item \textbf{Salvation Army} began in 1879, but often dominated by religious tensions betw. Protestant philanthropists and Catholic immigrants
                \item Poor children most heavily noticed by reformers, but little action taken
            \end{itemize}
            \textbf{Dominated by the middle classes, relief organizations generally only assisted the most destitute, physically incapable of supporting themselves. Impoverished children living on streets were most heavily noticed.}}
            \cornell{How did cities spur a growth of crime?}{\begin{itemize}
                \item Most crime small: pickpocketing, artists, thieves; some more dangerous w/ high murder rate (but less than nonurban regions of South w/ lynching, instable western villages); immigrants often blamed
                \item Violence/crime $\to$ police forces from informal, private to professionalized, public; worked with attorneys but often $\to$ corruption, favoring whites and the wealthy
                \begin{itemize}
                    \item Middle classes formed national guard groups to complement police work
                \end{itemize}
            \end{itemize}
            \textbf{Although most urban crime was minor, notably pickpocketing, the murder rate rose significantly; immigrants were often blamed. Police forces became far more formal and public with the rise of crime, but corruption was rampant.}}
            \cornell{How did the political machine aid immigrants and lower classes?}{\begin{itemize}
                \item Urban machine created by power vacuum from constant urban growth, weak govt., \underline{voting power of immigrants}
                \begin{itemize}
                    \item "Bosses" relied on immigrant vote (often immigrants, notably Irish, themselves) $\to$ provided relief for target voters, saving from jail and rewarding w/ jobs
                \end{itemize}
                \item Machines important money-makers
                \begin{itemize}
                    \item Some overt like bosses purchasing land in advance near secretive planned streetcar line
                    \item Some covert like manipulation of contractors to build projects in specific locations, control of public utilities
                    \item Most famously corrupt: \textbf{William M. Tweed} (eventually jailed)
                \end{itemize} 
                \item Middle classes criticized as obstacles to progress, but ultimately led to expansion of public works, infrastructure despite corrupt intentions
                \begin{itemize}
                    \item Critics often able to mobilize public to reduce political power of bosses
                \end{itemize}
                \item Boss rule promoted by immigrants often less concerned w/ political morality, more w/ subsistence; connection betw. wealthy and organizations; weakness of city govt. 
            \end{itemize}
            \textbf{The political machine, driven by a growing power vacuum as well as the voting power of the immigrant population concerned with receiving basic needs, brought in great wealth for the "bosses" who led them but also for the city itself.}}
            \cornell[The Rise of Mass Consumption]{How did mass consumption grow as a significant social force?}{\textbf{With income rising significantly for nearly all Americans, consumerism emerged, with consumers, most often women, flocking to chain or department stores to purchase ready-made clothing and goods.}}
            \cornell{How did income patterns affect consumption of goods?}{\begin{itemize}
                \item Expansion of markets $\to$ rise in income for all (though uneven)
                \begin{itemize}
                    \item Vast fortunes most notable result
                    \item Rising wages for white-collar middle class workers, doctors, lawyers
                    \item Working classes, notably iron and steel workers, saw significant hourly increases; female-dominated textile indusry saw little growth, as did South
                \end{itemize}
                \item Affordable products, ready-made clothing $\to$ private tailors, homespun clothes quickly replaced
                \begin{itemize}
                    \item Led to greater focus on personal style: middle class/working class began to concern themselves w/ women's fashion, stylish wardrobes
                \end{itemize}
                \item Tin cans in 1880s $\to$ industry for packing/selling in refridgerated rail cars $\to$ improved diets
            \end{itemize}
            \textbf{The rise in income for nearly all Americans, rich and poor was accompanied by a rise in demand for stylish ready-made clothing as well as more balanced diets through tin cans.}}
            \cornell{How did chain stores and mail-order house dominate the consumer industry?}{\begin{itemize}
                \item Chains, stores creating several locations throughout a city or the nation (like the Great Atlantic and Pacific Tea Company in groceries), began to dominate smaller stores 
                \begin{itemize}
                    \item Volume of goods $\to$ prices far lower $\to$ displaced merchants w/ great opposition, customers unable to resist 
                \end{itemize}
                \item Rural areas relied on mail-order houses (like Chicago's Montgomery Ward, Sears Roebuck), distributing catalogues o farmers offering wide arrays of goods
            \end{itemize}
            \textbf{Chain stores, those with several locations, offered goods at far lower prices due to their high volume, causing merchants to be rapidly displaced. Rural areas relied on mail-order houses, custom catalogues offering wide arrays of goods.}}
            \cornell{How did department stores transform buying habits?}{\begin{itemize}
                \item Began w/ Marshall Field in Chicago, soon NYC's Macy's, Brooklyn's Abraham and Straus, etc.
                \item Brought large array of products under single roof of elegant excitement/wonder w/ female salesclerks catering to female audience, offered lower prices due to volume
            \end{itemize}
            \textbf{Department stores were renowed for their elegant, refined female salesclerks and their large array of products under a single roof. Like chains, they were able to offer goods at far lower prices.}}
            \cornell{How did women grow as a consumer force?}{\begin{itemize}
                \item Rapid change in vogue women's styles, need to purchase food for family $\to$ far more purchases made than men 
                \item Women found new opportunities as salesclerks, waitresses
                \item Consumer protection movement emerged w/ \textbf{National Consumers League} fighting for women workers under common front of "consumers"
            \end{itemize}
            \textbf{Women purchased far more than men due to their rapidly changing styles as well as their need to purchase food. As they found jobs as salesclerks and waitresses, the consumer protection movement emerged to fight for female working conditions under a common front: consumerism.}}
    \end{document}