\documentclass[a4paper]{article}
    \input{../notesheader.tex}
    \newcommand{\chapternumber}{19}
    \newcommand{\chaptertitle}{From Crisis to Empire}
    \title{\vspace{-3em}
    \begin{tcolorbox}
    \Huge\sffamily \begin{center} AP US History  \\
    \LARGE Chapter \chapternumber$\text{ }$- \chaptertitle \\
    \Large Finn Frankis \end{center} 
    \end{tcolorbox}
    \vspace{-3em}
    }
    \date{}
    \author{}
    
    \begin{document}
        \maketitle
        \SetBgContents{\rule[0em]{4pt}{\textheight}}
        \cornell[Key Concepts]{What are this chapter's key concepts?}{\begin{itemize}
            \item \textbf{6.1.I.E} - Businesses, foreign policy $\to$ outside borders for influence over Asia, Latin America, Pacific rim 
            \item \textbf{6.1.III.B} - Agricultural market consolidation, heavy dependence on railroad $\to$ farmers formed organizations
            \item \textbf{6.1.III.C} - Economic instability $\to$ agrarian People's/Populist Party seeking $\uparrow$ govt. regulation
            \item \textbf{6.3.1.C} - Artists/critics like agrarians, utopians, socialists, Social Gospel advocates emphasized different visions for U.S. society
            \item \textbf{6.3.II.A} - Major political parties divided along Civil War boundaries; fought over tariffs/currency despite reformers arguing greed/self-interest was corrupting govt.
            \item \textbf{7.3.I.A} - Imperialists stressed econ. opportunities, racial theories, Euro. competition, 1890 idea of "closed" frontier to stress importance of expanding American culture across globe
            \item \textbf{7.3.I.B} - Anti-imperialists stressed self-determination, contrasting racial theories, isolationism to argue no need to extend territory
            \item \textbf{7.3.I.C} - Amer. win in Sp.-Amer. war $\to$ U.S. acquired Carib./Pacific territories, suppressed Filipino nationalism
        \end{itemize}}
        \cornell[The Politics of Equilibrium]{What factors altered the equilibrium of the partisan system?}{\textbf{The party system, split relatively equally along regional, ethnic, gender, and class lines, was characterized by strong party loyalty. The federal government did relatively little; presidents had to appease factionalism within their parties inspired by Hayes. Garfield and Arthur were greatly weakened by their actions' angering a significant part of the Republican party, both seeking civil service systems; Democrat Grover Cleveland sought to fight machine politics and cut tariffs; he served two terms with a one-term interruption from Republican Benjamin Harrison, who failed to rally public opinion in raising protective tariffs. Public opinion rallied toward antitrust reform as well as federal restriction of railroad fares.}}
        \cornell{How was the party system divided along regional, ethnic, gender, and class lines?}{\begin{itemize}
            \item End of Reconstruction $\to$ 1890s: electorate divided betw. Republicans (sixteen northern states) and Democrats (fourteen Southern states) w/ four undecided states 
            \begin{itemize}
                \item Repub. typically won presidency, Senate; Dems. won House
                \item Party loyalty v. strong w/ voter turnout $\uparrow$ 78\% of eligible; most blacks/woman disenfranchised
            \end{itemize}
            \item Loyalty to parties often result of undebated faith w/ white Southerners $\to$ Democrats, northerners $\to$ Republicans
            \item Ethnic diffs. w/ most Catholics, immigrants, impoverished $\to$ Democrats; middle class, Protestants, older citizens $\to$ Republicans 
            \begin{itemize}
                \item Republicans generally sought temperance (against Catholics) and anti-immigration laws while Democrats aligned w/ pro-Catholic views
            \end{itemize}
            \item Party selection generlaly to favor economic desires or to align w/ traditions of ancestors/relatives
        \end{itemize}
        \textbf{The party system was divided relatively equally among Republican and Democrat lines. Southern whites, northern Catholics, immigrants, and the lower classes usually aligned with the Democrats; northern Protestants, middle classes, and African Americans typically aligned with the Republicans. Party alignments were often primarily due to economic interest.}}
        \cornell{What was the relation of the federal government to partisan ties?}{\begin{itemize}
            \item Federal govt. generally did little: mail, military, foreign policy, taxes
            \item Several exceptions mainly in national econ. development 
            \begin{itemize}
                \item Railroad subsidies, land grants, brought up miltiary to put down strikes
                \item Pensions for retired Union Civil War veterans to pay majority of male citizens and several women; ultimately failed due to corruption
            \end{itemize}
            \item Pol. parties far more influential than nat. govt.
        \end{itemize}
        \textbf{The federal government kept stability by doing relatively little in the grand scheme of things; however, their industrial subsidies, land grants, and pensions ultimately made a significant economic impact. Ultimately, political parties and machines as well as federal courts were most powerful.}}
        \cornell{What was the significance of the presidential office during the Gilded Age?}{\begin{itemize}
            \item Party bosses $\to$ presidency generally symbolic w/ $\uparrow$ task distributing appointments ($\approx 100k$)
            \begin{itemize}
                \item Had to avoid angering factions w/in parties $\to$ generally careful
            \end{itemize}
            \item Rutherford B. Hayes (1877-1881) created great factionalism w/in party: \textbf{Stalwarts} competed for Republican power w/ \textbf{Half Breeds} 
            \begin{itemize}
                \item Stalwarts: traditional political machines, Half-Breeds: reform; neither satisfied by Hayes
                \item Hayes' attempt at civil service system not supported by either; wife's temperance reform angered many
            \end{itemize}
            \item Repubs. compromised w/ James A. Garfield (Stalwart) for president, Chester A. Arthur for vice president
            \begin{itemize}
                \item Won both houses of Congress against Democrat Hancock
                \item Garfield initially favored civil service reform, Half Breeds $\to$ direct defiance of Stalwarts $\to$ great conflict, assassinated four months after inauguration
            \end{itemize}
            \item Chester A. Arthur, successor, closely allied w/ Stalwart leader; attempted reform against Stalwarts w/ civil service measure \textbf{Pendleton Act} to require merit for job appointment 
        \end{itemize}
        \textbf{Gilded Age Presidents had to cater directly to factions within their parties; Rutherford B. Hayes inspired great Republican factionalism between the traditional Stalwarts and the reforming Half Breeds. Garfield, a Half Breed, won the presidency but was assassinated for his defiance of the Stalwarts; his successor, Arthur, continued Garfield's reform against the Stalwarts, with significant focus on a civil service program.}}
        \cornell{How did tariffs play a role in the presidencies of Cleveland and Harrison?}{\begin{itemize}
            \item 1884: Repub. candidate (James G. Blaine) represented epitome of partisan divisions $\to$ group of Republicans supported Democrat candidate, \textbf{Grover Cleveland}
            \begin{itemize}
                \item Blaine suffered religious hit after Protestant spokesman supporting Blaine denounced Catholics $\to$ Cleveland won tight last-minute victory
            \end{itemize}
            \item Cleveland known for strong opposition to machine politics, willingness to veto (particularly tariffs) due to belief that high federal revenues corrupted legislation 
            \item 1888: Democrats nominated Cleveland, supported lowered tariffs; Republicans selected \textbf{Benjamin Harrison} w/ clear diff. betw. parties $\to$ decisive victory for Harrison
        \end{itemize}
        \textbf{Democrat Grover Cleveland won the 1884 election thanks to a religious controversy which put Republicans out of Catholic favor; he was known particularly for his opposition to high tariffs. The tariff issue became a critical issue in the 1888 election, allowing Republican Benjamin Harrison to clinch a decisive win.}}
        \cornell{What were significant public issues emerging during and after Harrison's presidency?}{\begin{itemize}
            \item Harrison generally passive w/ little attempt to influence Congress; public influence became far more powerful force 
            \item Mid-1880s: 15 western/southern states banned combinations that reduced competition $\to$ most corporation went to NJ/Delaware to bypass
            \begin{itemize}
                \item \textbf{Sherman Antitrust Act} in 1890 widely supported, having \underline{limited impact} but passed as a symbolic attack against corporations
            \end{itemize}
            \item Republicans focused on tariff (believed to be winning factor of election) w/ \textbf{McKinley Tariff} sent to Congress in 1890 
            \begin{itemize}
                \item Misunderstood public opinion: 1890 midterms saw destruction in House, reduced majority in Senate 
            \end{itemize}
            \item 1892 elec.: Harrison continued to fight for tariffs, Cleveland opposed it, People's Party wanted more major reform
            \begin{itemize}
                \item Repubs. too badly weakened w/ Dem. majority in both houses, Cleveland winning presidency 
                \item Cleveland passed tariff production; made it through but greatly weakened by Senate/
            \end{itemize}
            \item Public pressure for railroad regulation from Midwest farm organizations
            \begin{itemize}
                \item Midwest state legislatures passed several legislations in 1870s; Supreme Court ruled one unconstitutional in 1886 bc. represented control of interstate commerce
                \begin{itemize}
                    \item Supreme Court later displayed power even more w/ limitation of state's power to reg. commerce w/in boundaries
                \end{itemize}
                \item Regulation had to come from fed. govt. $\to$ 1887 \textbf{Interstate Commerce Act} 
                \begin{itemize}
                    \item Banned rate diffs. betw. long/short hauls; required published rate schedules w/ govt.; required all rates to be fair
                    \item Interstate Commerce Commission/ICC would administer but relied on judicial support $\to$ little effect
                \end{itemize}
            \end{itemize}
        \end{itemize}
        \textbf{After Harrison took control, he passed high protective tariffs in the McKinley Tariff as well as the Sherman Antitrust Act as a symbolic show of power against corporations, but his misinterpretation of public opinion cost the Republicans the next election, putting Cleveland back in power. Furthermore, federal regulation of railroads became increasingly desired; the Interstate Commerce Act aimed to accomplish this but lacked practical effect.}}  
        \cornell[The Agrarian Revolt]{How did farmers oppose the Gilded Age federal government?}{\textbf{Farmers initially formed the Grange organization, dedicated to economic improvement to prevent the influence of middlemen and corporate monopolies; they made some political strides, but collapsed due to the return of agricultural prosperity. The Farmers' Alliance succeeded it, with similar short-term goals but eventually expanding to merge the two largest groups and forming the Populist party. Populists were typically isolated, impoverished farmers unable to keep up with the expansion of agricultural technology; they sought to provide cooperatives for farmers and reject government concepts of \textit{laissez-faire} and stress absolute ownership.}}
        \cornell{How did the farmers united in the Grange?}{\begin{itemize}
            \item Despite stereotype of farmer independence, formed Grange organization post-Civil War after Agri. Dept. official appalled by isolation of rural life
            \begin{itemize}
                \item Created National Grange of the Patrons of Husbandry, leading to networks of smaller organization
                \item Initially modestly defined goal as keeping up w/ advancing agri. techniques; also creating community to combat loneliness
            \end{itemize}
            \item Depression of 1873 $\to$ $\downarrow$ farm prices $\to$ $\uparrow$ Grange membership concentrated in South/Midwest 
            \begin{itemize}
                \item Rising numbers $\to$ focused more on economic benefits, seeking to bypass "middlemen," who took cut of profits from selling crops, and reduce railroad/warehouse monopolies
                \item Grangers created several stores, warehouses, factories to cater to farmers; founded mail-order Montgomery Ward and Company; most failed due to operator inexperience
            \end{itemize}
            \item Grangers sought to elect state legislators; generally worked through existing policies but a few attempts at "Antimonopoly" and "Reform" 
            \begin{itemize}
                \item At peak, controlled some Midwest legislatures w/ strict 1870s railroad regulation
                \item Temporary revival of agri. prosperity, judicial destruction of Grange laws $\to$ organization shrunk rapidly
            \end{itemize}
        \end{itemize}
        \textbf{The Grange organization united farmers with the common economic goal of cutting out middlemen and limiting railroad monopolies. They created enterprises to cater directly to farmers and elected state legislators, even gaining control of some midwestern legislatures to restrict railroad power. A return of agricultural prosperity caused the organization to rapidly decline.}}
        \cornell{How did the Farmers' Alliances succeed the Grange?}{\begin{itemize}
            \item Emerging before decline of Grange in South, farmers banded together in \textbf{Farmers' Alliances}
            \begin{itemize}
                \item 1880 Southern Alliance (concentrated in TX) known for more than 4m members; Northwestern Alliance grew in plains states
            \end{itemize}
            \item Sought to solve local problems w/ cooperatives, marketing mechanisms, stores, banks, processing plants all to avoid middlemen
            \item Some sought to build society w/ econ. competition replaced by cooperation through neighborly responsibility; travelled throughout rural areas attacking corporations
            \item Women generally afforded great power in organizations
            \begin{itemize}
                \item \textbf{Mary E. Lease}, lecturer, became Populist orator bringing farmers to action
                \item Several focused on temperance 
            \end{itemize}
            \item Alliances suffered from poorly-functioning cooperatives w/ market forces too strong 
            \item Weakening of Farmers' Alliances $\to$ political organization
            \begin{itemize}
                \item Southern/Northwestern Alliances agreed to weak merger w/ national convention in Ocala, FL to create \underline{party platform}
                \begin{itemize}
                    \item 1890 mid-terms: Alliances won control (mostly by endorsing Democrats) of legislatures in 12 states; some Senate/House seats 
                \end{itemize}
                \item Northwestern Alliance (and some from Southern Alliance) sought third party, including \textbf{Tom Watson} of GA, \textbf{Leonidas L. Polk}, both heavily involved in Alliance
                \begin{itemize}
                    \item Formed People's Party (commonly known as Populist) w/ great early success in 1892: 8.5\% of pop. vote, 22 elec. votes, some state legislatures, governors, congressmen 
                \end{itemize}
            \end{itemize}
        \end{itemize}
        \textbf{The Farmers' Alliances banded together primarily in the large Southern Alliance as well as the Northwestern Alliance; their initial goals were similar to the Grangers, seeking to solve local problems by limiting middlemen and railroad monopolies. They allowed several women to participate. However, the power of market forces meant they were often unsuccessful, encouraging a Northwestern/Southern merger and the creation of a third party: the Populist Party. It was very successful even in its early years.}}
        \cornell{What were the main demographics of the Populists?}{\begin{itemize}
            \item Populists sought coalition primarily of farmers w/ limited economic security, antiquated farming methods in the face of mechanization 
            \item Midwest Populists family farmers losing land; Southern Populists often sharecroppers/tenant farmers 
            \item Geographically isolated turned to Populism for sense of community, purpose 
            \item Attempted actively to attract labor unions w/ attempts at mergers w/ Knights of Labor but never widely successful due to conflicting interests 
            \item In Rocky Mountains, attracted large numbers of miners
            \begin{itemize}
                \item Supported "free silver," where gold/silver would become fundamental to currency (but temporary success)
            \end{itemize}
            \item In South, struggled w/ question of admitting Afr. Americans; "Colored Alliances" formed with similar structure to Farmers Alliances
            \begin{itemize}
                \item Most white populists accepted blacks if control remained guaranteed
            \end{itemize}
            \item Populist leaders part of rural middle class, consisting of both men and women; some extremely serious but others crazy
            \begin{itemize}
                \item Inspired idea of "southern demagogue"
            \end{itemize}
        \end{itemize}
        \textbf{Populists were typically isolated farmers using archaic agricultural techniques who sought to join the party for economic security and a sense of community. Attempts were made to attract labor forces, but these were never widely successful; African Americans were usually accepted but forced to remain subordinate. Populist leaders were adiverse population, with some somber and serious but others crazy.}}
        \cornell{What were the central ideas of the Populist platform?}{\begin{itemize}
            \item Suggested "subtreasuries" to strengthen cooperatives of Grangers/Alliances, govt. warehouses for farmers to deposit crops as collateral to borrow money 
            \item Called for end of national banks, absentee land ownership, direction electon of senators; sought regulation and eventually govt. ownership of railroads
            \item Some were Anti-Semitic, blaming Jews for economic expansion; others anti-intellectual, anti-eastern, anti-urban, mystical, deranged
            \item Despite some discrim., Populists overall sought solutions to economic issues (like end of \textit{laissez-faire} and absolute ownership rights)
        \end{itemize}
        \textbf{Populists sought to strengthen cooperatives through subtreasuries, form a national warehouse network, end national banks, directly elect senators, and prevent absentee land ownership; they eventually sought railroad regulation. Though some were greatly discriminatory, most sought practical solutions to economic problems.}}
        \cornell[The Crisis of the 1890s]{What major controversies unfolded in the 1890s?}{\textbf{The major controversies were the Panic of 1893, where rapid railroad expansion ultimately led to collapse. Silver was a very important specie; when the government phased it out but prices happed and opportunities for unions at the same time, they addressed Ayush; angry farmers rose up, opposing 3ws Cohn's book, providing detailed descriptions of farmer communities to stimulate economic change. }}
        \cornell{What unfolded during the Panic of 1893?}{\begin{itemize}
            \item Most severe depression nation had ever emerged in 1893
            \begin{itemize}
                \item Short-term: two successive corporate failures $\to$ stock market collapsed $\to$ banks failed $\to$ credit contracted $\to$ loan-depended businesses collapsed
                \item Long-term: lowering agriculture prices, European depression, rapid expansion of railroads beyond market demand, market interconnectedness $\to$ widespread collapse mainly due to railroads
            \end{itemize}
            \item Panic $\to$ collapse of 8k businesses w/in 6 months, even lower agri. prices, 20\% of agri. workers lost jobs 
            \item Caused social unrest w/ Populist Coxey proposing public works program to create jobs for unemployed
            \begin{itemize}
                \item Coxey marched to capital to advocate cause but soon arrested 
                \item Americans saw labor turmoil (paired w/ strikes) as sign of instability, revoltuin w/ potential for radicalism
            \end{itemize}
        \end{itemize}
        \textbf{The Panic of 1893 was ultimately stimulated by the too-rapid growth of the railroad industry beyond market demand leading to the stock market collapse with several large companies following. It led to great social unrest and unemployment; Coxey attempted to solve it with a large public works program, but was rejected by the government.}}
        \cornell{What was the significance of silver?}{\begin{itemize}
            \item Panic weakened monetary system $\to$ Cleveland saw unstable currency as main cause of conflict
            \begin{itemize}
                \item Modern dollar based on public confidence in govt.; many used to belief in importance of physical reserves (specie) of precious metal to back up worth
                \item U.S. initially accepted gold and silver; gold was given a value of 16x of silver despite being far more $\to$ silver fashioned into objects by jewelers for easy wealth $\to$ silver stopped flowing to mint $\to$ stopped
            \end{itemize}
            \item Congress dicontinued silver coinage: no longer recognized; opposition init. weak but resumed after siver grew greatly in worth $\to$ felt money had been taken 
            \item Classified as "Crime of '73" w/ free silverers feeling silver was tyranny; William H. Harveu Clay convinced many of ideology
            \begin{itemize}
                \item Saw silver as collective award $\to$ wrote great book abt. lectures w/ students, ginances; generally persuaded students of to learn more abt. individual
                \item Led imaginary school to prepare in the long-term
            \end{itemize}
        \end{itemize}
        \textbf{Silver, a very important free metal, was argued by many to be critical to back up the economy rather htna relying on popular interest. After the panic, Smith created a legislators to ban it; little support was met apart from for hs }}
    \end{document}