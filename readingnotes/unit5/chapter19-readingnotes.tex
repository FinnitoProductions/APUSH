\documentclass[a4paper]{article}
    \input{../notesheader.tex}
    \newcommand{\chapternumber}{19}
    \newcommand{\chaptertitle}{From Crisis to Empire}
    \title{\vspace{-3em}
    \begin{tcolorbox}
    \Huge\sffamily \begin{center} AP US History  \\
    \LARGE Chapter \chapternumber$\text{ }$- \chaptertitle \\
    \Large Finn Frankis \end{center} 
    \end{tcolorbox}
    \vspace{-3em}
    }
    \date{}
    \author{}
    
    \begin{document}
        \maketitle
        \SetBgContents{\rule[0em]{4pt}{\textheight}}
        \cornell[Key Concepts]{What are this chapter's key concepts?}{\begin{itemize}
            \item \textbf{6.1.I.E} - Businesses, foreign policy $\to$ outside borders for influence over Asia, Latin America, Pacific rim 
            \item \textbf{6.1.III.B} - Agricultural market consolidation, heavy dependence on railroad $\to$ farmers formed organizations
            \item \textbf{6.1.III.C} - Economic instability $\to$ agrarian People's/Populist Party seeking $\uparrow$ govt. regulation
            \item \textbf{6.3.1.C} - Artists/critics like agrarians, utopians, socialists, Social Gospel advocates emphasized different visions for U.S. society
            \item \textbf{6.3.II.A} - Major political parties divided along Civil War boundaries; fought over tariffs/currency despite reformers arguing greed/self-interest was corrupting govt.
            \item \textbf{7.3.I.A} - Imperialists stressed econ. opportunities, racial theories, Euro. competition, 1890 idea of "closed" frontier to stress importance of expanding American culture across globe
            \item \textbf{7.3.I.B} - Anti-imperialists stressed self-determination, contrasting racial theories, isolationism to argue no need to extend territory
            \item \textbf{7.3.I.C} - Amer. win in Sp.-Amer. war $\to$ U.S. acquired Carib./Pacific territories, suppressed Filipino nationalism
        \end{itemize}}
        \cornell[The Politics of Equilibrium]{What factors altered the equilibrium of the partisan system?}{\textbf{The party system, split relatively equally along regional, ethnic, gender, and class lines, was characterized by strong party loyalty. The federal government did relatively little; presidents had to appease factionalism within their parties inspired by Hayes. Garfield and Arthur were greatly weakened by their actions' angering a significant part of the Republican party, both seeking civil service systems; Democrat Grover Cleveland sought to fight machine politics and cut tariffs; he served two terms with a one-term interruption from Republican Benjamin Harrison, who failed to rally public opinion in raising protective tariffs. Public opinion rallied toward antitrust reform as well as federal restriction of railroad fares.}}
        \cornell{How was the party system divided along regional, ethnic, gender, and class lines?}{\begin{itemize}
            \item End of Reconstruction $\to$ 1890s: electorate divided betw. Republicans (sixteen northern states) and Democrats (fourteen Southern states) w/ four undecided states 
            \begin{itemize}
                \item Repub. typically won presidency, Senate; Dems. won House
                \item Party loyalty v. strong w/ voter turnout $\uparrow$ 78\% of eligible; most blacks/woman disenfranchised
            \end{itemize}
            \item Loyalty to parties often result of undebated faith w/ white Southerners $\to$ Democrats, northerners $\to$ Republicans
            \item Ethnic diffs. w/ most Catholics, immigrants, impoverished $\to$ Democrats; middle class, Protestants, older citizens $\to$ Republicans 
            \begin{itemize}
                \item Republicans generally sought temperance (against Catholics) and anti-immigration laws while Democrats aligned w/ pro-Catholic views
            \end{itemize}
            \item Party selection generlaly to favor economic desires or to align w/ traditions of ancestors/relatives
        \end{itemize}
        \textbf{The party system was divided relatively equally among Republican and Democrat lines. Southern whites, northern Catholics, immigrants, and the lower classes usually aligned with the Democrats; northern Protestants, middle classes, and African Americans typically aligned with the Republicans. Party alignments were often primarily due to economic interest.}}
        \cornell{What was the relation of the federal government to partisan ties?}{\begin{itemize}
            \item Federal govt. generally did little: mail, military, foreign policy, taxes
            \item Several exceptions mainly in national econ. development 
            \begin{itemize}
                \item Railroad subsidies, land grants, brought up miltiary to put down strikes
                \item Pensions for retired Union Civil War veterans to pay majority of male citizens and several women; ultimately failed due to corruption
            \end{itemize}
            \item Pol. parties far more influential than nat. govt.
        \end{itemize}
        \textbf{The federal government kept stability by doing relatively little in the grand scheme of things; however, their industrial subsidies, land grants, and pensions ultimately made a significant economic impact. Ultimately, political parties and machines as well as federal courts were most powerful.}}
        \cornell{What was the significance of the presidential office during the Gilded Age?}{\begin{itemize}
            \item Party bosses $\to$ presidency generally symbolic w/ $\uparrow$ task distributing appointments ($\approx 100k$)
            \begin{itemize}
                \item Had to avoid angering factions w/in parties $\to$ generally careful
            \end{itemize}
            \item Rutherford B. Hayes (1877-1881) created great factionalism w/in party: \textbf{Stalwarts} competed for Republican power w/ \textbf{Half Breeds} 
            \begin{itemize}
                \item Stalwarts: traditional political machines, Half-Breeds: reform; neither satisfied by Hayes
                \item Hayes' attempt at civil service system not supported by either; wife's temperance reform angered many
            \end{itemize}
            \item Repubs. compromised w/ James A. Garfield (Stalwart) for president, Chester A. Arthur for vice president
            \begin{itemize}
                \item Won both houses of Congress against Democrat Hancock
                \item Garfield initially favored civil service reform, Half Breeds $\to$ direct defiance of Stalwarts $\to$ great conflict, assassinated four months after inauguration
            \end{itemize}
            \item Chester A. Arthur, successor, closely allied w/ Stalwart leader; attempted reform against Stalwarts w/ civil service measure \textbf{Pendleton Act} to require merit for job appointment 
        \end{itemize}
        \textbf{Gilded Age Presidents had to cater directly to factions within their parties; Rutherford B. Hayes inspired great Republican factionalism between the traditional Stalwarts and the reforming Half Breeds. Garfield, a Half Breed, won the presidency but was assassinated for his defiance of the Stalwarts; his successor, Arthur, continued Garfield's reform against the Stalwarts, with significant focus on a civil service program.}}
        \cornell{How did tariffs play a role in the presidencies of Cleveland and Harrison?}{\begin{itemize}
            \item 1884: Repub. candidate (James G. Blaine) represented epitome of partisan divisions $\to$ group of Republicans supported Democrat candidate, \textbf{Grover Cleveland}
            \begin{itemize}
                \item Blaine suffered religious hit after Protestant spokesman supporting Blaine denounced Catholics $\to$ Cleveland won tight last-minute victory
            \end{itemize}
            \item Cleveland known for strong opposition to machine politics, willingness to veto (particularly tariffs) due to belief that high federal revenues corrupted legislation 
            \item 1888: Democrats nominated Cleveland, supported lowered tariffs; Republicans selected \textbf{Benjamin Harrison} w/ clear diff. betw. parties $\to$ decisive victory for Harrison
        \end{itemize}
        \textbf{Democrat Grover Cleveland won the 1884 election thanks to a religious controversy which put Republicans out of Catholic favor; he was known particularly for his opposition to high tariffs. The tariff issue became a critical issue in the 1888 election, allowing Republican Benjamin Harrison to clinch a decisive win.}}
        \cornell{What were significant public issues emerging during and after Harrison's presidency?}{\begin{itemize}
            \item Harrison generally passive w/ little attempt to influence Congress; public influence became far more powerful force 
            \item Mid-1880s: 15 western/southern states banned combinations that reduced competition $\to$ most corporation went to NJ/Delaware to bypass
            \begin{itemize}
                \item \textbf{Sherman Antitrust Act} in 1890 widely supported, having \underline{limited impact} but passed as a symbolic attack against corporations
            \end{itemize}
            \item Republicans focused on tariff (believed to be winning factor of election) w/ \textbf{McKinley Tariff} sent to Congress in 1890 
            \begin{itemize}
                \item Misunderstood public opinion: 1890 midterms saw destruction in House, reduced majority in Senate 
            \end{itemize}
            \item 1892 elec.: Harrison continued to fight for tariffs, Cleveland opposed it, People's Party wanted more major reform
            \begin{itemize}
                \item Repubs. too badly weakened w/ Dem. majority in both houses, Cleveland winning presidency 
                \item Cleveland passed tariff production; made it through but greatly weakened by Senate/
            \end{itemize}
            \item Public pressure for railroad regulation from Midwest farm organizations
            \begin{itemize}
                \item Midwest state legislatures passed several legislations in 1870s; Supreme Court ruled one unconstitutional in 1886 bc. represented control of interstate commerce
                \begin{itemize}
                    \item Supreme Court later displayed power even more w/ limitation of state's power to reg. commerce w/in boundaries
                \end{itemize}
                \item Regulation had to come from fed. govt. $\to$ 1887 \textbf{Interstate Commerce Act} 
                \begin{itemize}
                    \item Banned rate diffs. betw. long/short hauls; required published rate schedules w/ govt.; required all rates to be fair
                    \item Interstate Commerce Commission/ICC would administer but relied on judicial support $\to$ little effect
                \end{itemize}
            \end{itemize}
        \end{itemize}
        \textbf{After Harrison took control, he passed high protective tariffs in the McKinley Tariff as well as the Sherman Antitrust Act as a symbolic show of power against corporations, but his misinterpretation of public opinion cost the Republicans the next election, putting Cleveland back in power. Furthermore, federal regulation of railroads became increasingly desired; the Interstate Commerce Act aimed to accomplish this but lacked practical effect.}}  
    \end{document}