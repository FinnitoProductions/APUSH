\documentclass[a4paper]{article}
    \input{../notesheader.tex}
    \newcommand{\chapternumber}{19}
    \newcommand{\chaptertitle}{From Crisis to Empire}
    \title{\vspace{-3em}
    \begin{tcolorbox}
    \Huge\sffamily \begin{center} AP US History  \\
    \LARGE Chapter \chapternumber$\text{ }$- \chaptertitle \\
    \Large Finn Frankis \end{center} 
    \end{tcolorbox}
    \vspace{-3em}
    }
    \date{}
    \author{}
    
    \begin{document}
        \maketitle
        \SetBgContents{\rule[0em]{4pt}{\textheight}}
        \cornell[Key Concepts]{What are this chapter's key concepts?}{\begin{itemize}
            \item \textbf{6.1.I.E} - Businesses, foreign policy $\to$ outside borders for influence over Asia, Latin America, Pacific rim 
            \item \textbf{6.1.III.B} - Agricultural market consolidation, heavy dependence on railroad $\to$ farmers formed organizations
            \item \textbf{6.1.III.C} - Economic instability $\to$ agrarian People's/Populist Party seeking $\uparrow$ govt. regulation
            \item \textbf{6.3.1.C} - Artists/critics like agrarians, utopians, socialists, Social Gospel advocates emphasized different visions for U.S. society
            \item \textbf{6.3.II.A} - Major political parties divided along Civil War boundaries; fought over tariffs/currency despite reformers arguing greed/self-interest was corrupting govt.
            \item \textbf{7.3.I.A} - Imperialists stressed econ. opportunities, racial theories, Euro. competition, 1890 idea of "closed" frontier to stress importance of expanding American culture across globe
            \item \textbf{7.3.I.B} - Anti-imperialists stressed self-determination, contrasting racial theories, isolationism to argue no need to extend territory
            \item \textbf{7.3.I.C} - Amer. win in Sp.-Amer. war $\to$ U.S. acquired Carib./Pacific territories, suppressed Filipino nationalism
        \end{itemize}}
        \cornell[The Politics of Equilibrium]{What factors altered the equilibrium of the partisan system?}{\textbf{The party system, split relatively equally along regional, ethnic, gender, and class lines, was characterized by strong party loyalty. The federal government did relatively little; presidents had to appease factionalism within their parties inspired by Hayes. Garfield and Arthur were greatly weakened by their actions' angering a significant part of the Republican party, both seeking civil service systems; Democrat Grover Cleveland sought to fight machine politics and cut tariffs; he served two terms with a one-term interruption from Republican Benjamin Harrison, who failed to rally public opinion in raising protective tariffs. Public opinion rallied toward antitrust reform as well as federal restriction of railroad fares.}}
        \cornell{How was the party system divided along regional, ethnic, gender, and class lines?}{\begin{itemize}
            \item End of Reconstruction $\to$ 1890s: electorate divided betw. Republicans (sixteen northern states) and Democrats (fourteen Southern states) w/ four undecided states 
            \begin{itemize}
                \item Repub. typically won presidency, Senate; Dems. won House
                \item Party loyalty v. strong w/ voter turnout $\uparrow$ 78\% of eligible; most blacks/woman disenfranchised
            \end{itemize}
            \item Loyalty to parties often result of undebated faith w/ white Southerners $\to$ Democrats, northerners $\to$ Republicans
            \item Ethnic diffs. w/ most Catholics, immigrants, impoverished $\to$ Democrats; middle class, Protestants, older citizens $\to$ Republicans 
            \begin{itemize}
                \item Republicans generally sought temperance (against Catholics) and anti-immigration laws while Democrats aligned w/ pro-Catholic views
            \end{itemize}
            \item Party selection generlaly to favor economic desires or to align w/ traditions of ancestors/relatives
        \end{itemize}
        \textbf{The party system was divided relatively equally among Republican and Democrat lines. Southern whites, northern Catholics, immigrants, and the lower classes usually aligned with the Democrats; northern Protestants, middle classes, and African Americans typically aligned with the Republicans. Party alignments were often primarily due to economic interest.}}
        \cornell{What was the relation of the federal government to partisan ties?}{\begin{itemize}
            \item Federal govt. generally did little: mail, military, foreign policy, taxes
            \item Several exceptions mainly in national econ. development 
            \begin{itemize}
                \item Railroad subsidies, land grants, brought up miltiary to put down strikes
                \item Pensions for retired Union Civil War veterans to pay majority of male citizens and several women; ultimately failed due to corruption
            \end{itemize}
            \item Pol. parties far more influential than nat. govt.
        \end{itemize}
        \textbf{The federal government kept stability by doing relatively little in the grand scheme of things; however, their industrial subsidies, land grants, and pensions ultimately made a significant economic impact. Ultimately, political parties and machines as well as federal courts were most powerful.}}
        \cornell{What was the significance of the presidential office during the Gilded Age?}{\begin{itemize}
            \item Party bosses $\to$ presidency generally symbolic w/ $\uparrow$ task distributing appointments ($\approx 100k$)
            \begin{itemize}
                \item Had to avoid angering factions w/in parties $\to$ generally careful
            \end{itemize}
            \item Rutherford B. Hayes (1877-1881) created great factionalism w/in party: \textbf{Stalwarts} competed for Republican power w/ \textbf{Half Breeds} 
            \begin{itemize}
                \item Stalwarts: traditional political machines, Half-Breeds: reform; neither satisfied by Hayes
                \item Hayes' attempt at civil service system not supported by either; wife's temperance reform angered many
            \end{itemize}
            \item Repubs. compromised w/ James A. Garfield (Stalwart) for president, Chester A. Arthur for vice president
            \begin{itemize}
                \item Won both houses of Congress against Democrat Hancock
                \item Garfield initially favored civil service reform, Half Breeds $\to$ direct defiance of Stalwarts $\to$ great conflict, assassinated four months after inauguration
            \end{itemize}
            \item Chester A. Arthur, successor, closely allied w/ Stalwart leader; attempted reform against Stalwarts w/ civil service measure \textbf{Pendleton Act} to require merit for job appointment 
        \end{itemize}
        \textbf{Gilded Age Presidents had to cater directly to factions within their parties; Rutherford B. Hayes inspired great Republican factionalism between the traditional Stalwarts and the reforming Half Breeds. Garfield, a Half Breed, won the presidency but was assassinated for his defiance of the Stalwarts; his successor, Arthur, continued Garfield's reform against the Stalwarts, with significant focus on a civil service program.}}
        \cornell{How did tariffs play a role in the presidencies of Cleveland and Harrison?}{\begin{itemize}
            \item 1884: Repub. candidate (James G. Blaine) represented epitome of partisan divisions $\to$ group of Republicans supported Democrat candidate, \textbf{Grover Cleveland}
            \begin{itemize}
                \item Blaine suffered religious hit after Protestant spokesman supporting Blaine denounced Catholics $\to$ Cleveland won tight last-minute victory
            \end{itemize}
            \item Cleveland known for strong opposition to machine politics, willingness to veto (particularly tariffs) due to belief that high federal revenues corrupted legislation 
            \item 1888: Democrats nominated Cleveland, supported lowered tariffs; Republicans selected \textbf{Benjamin Harrison} w/ clear diff. betw. parties $\to$ decisive victory for Harrison
        \end{itemize}
        \textbf{Democrat Grover Cleveland won the 1884 election thanks to a religious controversy which put Republicans out of Catholic favor; he was known particularly for his opposition to high tariffs. The tariff issue became a critical issue in the 1888 election, allowing Republican Benjamin Harrison to clinch a decisive win.}}
        \cornell{What were significant public issues emerging during and after Harrison's presidency?}{\begin{itemize}
            \item Harrison generally passive w/ little attempt to influence Congress; public influence became far more powerful force 
            \item Mid-1880s: 15 western/southern states banned combinations that reduced competition $\to$ most corporation went to NJ/Delaware to bypass
            \begin{itemize}
                \item \textbf{Sherman Antitrust Act} in 1890 widely supported, having \underline{limited impact} but passed as a symbolic attack against corporations
            \end{itemize}
            \item Republicans focused on tariff (believed to be winning factor of election) w/ \textbf{McKinley Tariff} sent to Congress in 1890 
            \begin{itemize}
                \item Misunderstood public opinion: 1890 midterms saw destruction in House, reduced majority in Senate 
            \end{itemize}
            \item 1892 elec.: Harrison continued to fight for tariffs, Cleveland opposed it, People's Party wanted more major reform
            \begin{itemize}
                \item Repubs. too badly weakened w/ Dem. majority in both houses, Cleveland winning presidency 
                \item Cleveland passed tariff production; made it through but greatly weakened by Senate/
            \end{itemize}
            \item Public pressure for railroad regulation from Midwest farm organizations
            \begin{itemize}
                \item Midwest state legislatures passed several legislations in 1870s; Supreme Court ruled one unconstitutional in 1886 bc. represented control of interstate commerce
                \begin{itemize}
                    \item Supreme Court later displayed power even more w/ limitation of state's power to reg. commerce w/in boundaries
                \end{itemize}
                \item Regulation had to come from fed. govt. $\to$ 1887 \textbf{Interstate Commerce Act} 
                \begin{itemize}
                    \item Banned rate diffs. betw. long/short hauls; required published rate schedules w/ govt.; required all rates to be fair
                    \item Interstate Commerce Commission/ICC would administer but relied on judicial support $\to$ little effect
                \end{itemize}
            \end{itemize}
        \end{itemize}
        \textbf{After Harrison took control, he passed high protective tariffs in the McKinley Tariff as well as the Sherman Antitrust Act as a symbolic show of power against corporations, but his misinterpretation of public opinion cost the Republicans the next election, putting Cleveland back in power. Furthermore, federal regulation of railroads became increasingly desired; the Interstate Commerce Act aimed to accomplish this but lacked practical effect.}}  
        \cornell[The Agrarian Revolt]{How did farmers oppose the Gilded Age federal government?}{\textbf{Farmers initially formed the Grange organization, dedicated to economic improvement to prevent the influence of middlemen and corporate monopolies; they made some political strides, but collapsed due to the return of agricultural prosperity. The Farmers' Alliance succeeded it, with similar short-term goals but eventually expanding to merge the two largest groups and forming the Populist party. Populists were typically isolated, impoverished farmers unable to keep up with the expansion of agricultural technology; they sought to provide cooperatives for farmers and reject government concepts of \textit{laissez-faire} and stress absolute ownership.}}
        \cornell{How did the farmers united in the Grange?}{\begin{itemize}
            \item Despite stereotype of farmer independence, formed Grange organization post-Civil War after Agri. Dept. official appalled by isolation of rural life
            \begin{itemize}
                \item Created National Grange of the Patrons of Husbandry, leading to networks of smaller organization
                \item Initially modestly defined goal as keeping up w/ advancing agri. techniques; also creating community to combat loneliness
            \end{itemize}
            \item Depression of 1873 $\to$ $\downarrow$ farm prices $\to$ $\uparrow$ Grange membership concentrated in South/Midwest 
            \begin{itemize}
                \item Rising numbers $\to$ focused more on economic benefits, seeking to bypass "middlemen," who took cut of profits from selling crops, and reduce railroad/warehouse monopolies
                \item Grangers created several stores, warehouses, factories to cater to farmers; founded mail-order Montgomery Ward and Company; most failed due to operator inexperience
            \end{itemize}
            \item Grangers sought to elect state legislators; generally worked through existing policies but a few attempts at "Antimonopoly" and "Reform" 
            \begin{itemize}
                \item At peak, controlled some Midwest legislatures w/ strict 1870s railroad regulation
                \item Temporary revival of agri. prosperity, judicial destruction of Grange laws $\to$ organization shrunk rapidly
            \end{itemize}
        \end{itemize}
        \textbf{The Grange organization united farmers with the common economic goal of cutting out middlemen and limiting railroad monopolies. They created enterprises to cater directly to farmers and elected state legislators, even gaining control of some midwestern legislatures to restrict railroad power. A return of agricultural prosperity caused the organization to rapidly decline.}}
        \cornell{How did the Farmers' Alliances succeed the Grange?}{\begin{itemize}
            \item Emerging before decline of Grange in South, farmers banded together in \textbf{Farmers' Alliances}
            \begin{itemize}
                \item 1880 Southern Alliance (concentrated in TX) known for more than 4m members; Northwestern Alliance grew in plains states
            \end{itemize}
            \item Sought to solve local problems w/ cooperatives, marketing mechanisms, stores, banks, processing plants all to avoid middlemen
            \item Some sought to build society w/ econ. competition replaced by cooperation through neighborly responsibility; travelled throughout rural areas attacking corporations
            \item Women generally afforded great power in organizations
            \begin{itemize}
                \item \textbf{Mary E. Lease}, lecturer, became Populist orator bringing farmers to action
                \item Several focused on temperance 
            \end{itemize}
            \item Alliances suffered from poorly-functioning cooperatives w/ market forces too strong 
            \item Weakening of Farmers' Alliances $\to$ political organization
            \begin{itemize}
                \item Southern/Northwestern Alliances agreed to weak merger w/ national convention in Ocala, FL to create \underline{party platform}
                \begin{itemize}
                    \item 1890 mid-terms: Alliances won control (mostly by endorsing Democrats) of legislatures in 12 states; some Senate/House seats 
                \end{itemize}
                \item Northwestern Alliance (and some from Southern Alliance) sought third party, including \textbf{Tom Watson} of GA, \textbf{Leonidas L. Polk}, both heavily involved in Alliance
                \begin{itemize}
                    \item Formed People's Party (commonly known as Populist) w/ great early success in 1892: 8.5\% of pop. vote, 22 elec. votes, some state legislatures, governors, congressmen 
                \end{itemize}
            \end{itemize}
        \end{itemize}
        \textbf{The Farmers' Alliances banded together primarily in the large Southern Alliance as well as the Northwestern Alliance; their initial goals were similar to the Grangers, seeking to solve local problems by limiting middlemen and railroad monopolies. They allowed several women to participate. However, the power of market forces meant they were often unsuccessful, encouraging a Northwestern/Southern merger and the creation of a third party: the Populist Party. It was very successful even in its early years.}}
        \cornell{What were the main demographics of the Populists?}{\begin{itemize}
            \item Populists sought coalition primarily of farmers w/ limited economic security, antiquated farming methods in the face of mechanization 
            \item Midwest Populists family farmers losing land; Southern Populists often sharecroppers/tenant farmers 
            \item Geographically isolated turned to Populism for sense of community, purpose 
            \item Attempted actively to attract labor unions w/ attempts at mergers w/ Knights of Labor but never widely successful due to conflicting interests 
            \item In Rocky Mountains, attracted large numbers of miners
            \begin{itemize}
                \item Supported "free silver," where gold/silver would become fundamental to currency (but temporary success)
            \end{itemize}
            \item In South, struggled w/ question of admitting Afr. Americans; "Colored Alliances" formed with similar structure to Farmers Alliances
            \begin{itemize}
                \item Most white populists accepted blacks if control remained guaranteed
            \end{itemize}
            \item Populist leaders part of rural middle class, consisting of both men and women; some extremely serious but others crazy
            \begin{itemize}
                \item Inspired idea of "southern demagogue"
            \end{itemize}
        \end{itemize}
        \textbf{Populists were typically isolated farmers using archaic agricultural techniques who sought to join the party for economic security and a sense of community. Attempts were made to attract labor forces, but these were never widely successful; African Americans were usually accepted but forced to remain subordinate. Populist leaders were adiverse population, with some somber and serious but others crazy.}}
        \cornell{What were the central ideas of the Populist platform?}{\begin{itemize}
            \item Suggested "subtreasuries" to strengthen cooperatives of Grangers/Alliances, govt. warehouses for farmers to deposit crops as collateral to borrow money 
            \item Called for end of national banks, absentee land ownership, direction electon of senators; sought regulation and eventually govt. ownership of railroads
            \item Some were Anti-Semitic, blaming Jews for economic expansion; others anti-intellectual, anti-eastern, anti-urban, mystical, deranged
            \item Despite some discrim., Populists overall sought solutions to economic issues (like end of \textit{laissez-faire} and absolute ownership rights)
        \end{itemize}
        \textbf{Populists sought to strengthen cooperatives through subtreasuries, form a national warehouse network, end national banks, directly elect senators, and prevent absentee land ownership; they eventually sought railroad regulation. Though some were greatly discriminatory, most sought practical solutions to economic problems.}}
        \cornell[The Crisis of the 1890s]{What major controversies unfolded in the 1890s?}{\textbf{The major controversies were the Panic of 1893, where rapid railroad expansion ultimately led to collapse. Silver was a very important specie; when the government phased it out due to its reducing use but it suddenly spiked in value, farmers and silver-mine owners responded with great passion; both sides passionately debated the issue, with the nation's honor on the line. }}
        \cornell{What unfolded during the Panic of 1893?}{\begin{itemize}
            \item Most severe depression nation had ever emerged in 1893
            \begin{itemize}
                \item Short-term: two successive corporate failures $\to$ stock market collapsed $\to$ banks failed $\to$ credit contracted $\to$ loan-depended businesses collapsed
                \item Long-term: lowering agriculture prices, European depression, rapid expansion of railroads beyond market demand, market interconnectedness $\to$ widespread collapse mainly due to railroads
            \end{itemize}
            \item Panic $\to$ collapse of 8k businesses w/in 6 months, even lower agri. prices, 20\% of agri. workers lost jobs 
            \item Caused social unrest w/ Populist Coxey proposing public works program to create jobs for unemployed
            \begin{itemize}
                \item Coxey marched to capital to advocate cause but soon arrested 
                \item Americans saw labor turmoil (paired w/ strikes) as sign of instability, revolution w/ potential for radicalism
            \end{itemize}
        \end{itemize}
        \textbf{The Panic of 1893 was ultimately stimulated by the too-rapid growth of the railroad industry beyond market demand leading to the stock market collapse with several large companies following. It led to great social unrest and unemployment; Coxey attempted to solve it with a large public works program, but was rejected by the government.}}
        \cornell{What was the significance of silver?}{\begin{itemize}
            \item Panic weakened monetary system $\to$ Cleveland saw unstable currency as main cause of conflict
            \begin{itemize}
                \item Modern dollar based on public confidence in govt.; many used to belief in importance of physical reserves (specie) of precious metal to back up worth
                \item U.S. initially accepted gold and silver; gold was given a value of 16x of silver despite being far more $\to$ silver fashioned into objects by jewelers for easy wealth $\to$ silver stopped flowing to mint $\to$ stopped
            \end{itemize}
            \item Congress discontinued silver coinage: no longer recognized; opposition init. weak but resumed after siver grew greatly in worth $\to$ felt money had been taken 
            \item Classified as "Crime of '73" w/ those seeking to use silver as "free silver" as currency seeing as tyrannical
            \begin{itemize}
                \item Silver-mine owners wanted govt. to take surplus, pay more than market price
                \item Discontented farmers wanted increase in quantity of money through inflation to raise farm prices, ease debts
                \item Congress made little response
            \end{itemize}
            \item Gold reserves falling; Cleveland blamed on Sherman Silver Purchase Act of 1890 (designed to help silver miners) $\to$ had repealed w/ great battle leading to split in Dem. Party
            \begin{itemize}
                \item Southern Dems. opposed him
            \end{itemize}
            \item Silver became issue of great debate w/ those on both sides arguing w/ signif. passion
            \begin{itemize}
                \item Supporters of gold standard felt it was critical to the nation's honor/success
                \item William H. Harvey Clay convinced many of free silver ideology
                \begin{itemize}
                    \item Saw silver as collective finance for all to share $\to$ wrote book arguing economic importance
                    \item Led imaginary school to prepare students in long-term with financial knowledge which persuaded students of key importance of silver
                \end{itemize}
            \end{itemize}
        \end{itemize}
        \textbf{After the government discontinued silver currency to rely on gold alone due to its low financial worth, few responded; however, silver quickly became far more important, causing silver-mine owners and discontented farmers to rise up against the government in the "Crime of '73". It became an issue of passionate debate with the nation's honor on the line; Clay convinced many of the importance of free silver in his powerful book and through his imaginary financial school.}}
    \cornell["A Cross of Gold"]{What was the "Cross of Gold"?}{\textbf{For the election of 1896, the Republicans selected McKinley, whose traditional campaigning (very little) won out against powerful orator Bryan's more radical, widespread campaigning; he had been selected for his powerful "Cross of Gold" speech emphasizing the gold standard. Because the Populists had endorsed Bryan for his stressing the gold standard, Bryan's loss led to their collapse. McKinley's presidency was focused on restoring economic prosperity from the Panic of 1893; he raised tariffs and assured the gold standard, ultimately returning prosperity to America.}}
    \cornell{How did Bryan transform public opinion on the free silver issue?}{\begin{itemize}
        \item Election of 1896: Republicans confident of success due to Democrat failure to handle depression; nominated \textbf{William McKinley}, involved in tariff act 
        \begin{itemize}
            \item Opposed free coinage of silver unless leading commercial nations formed agreement $\to$ 34 mountains/plains delegates walked out to join Dems. 
        \end{itemize}
        \item Southern/western Democrats sought to neutralize Populists by absorbing into their own thru. acceptance of some demands (notably free silver)
        \begin{itemize}
            \item Convention saw several defenders of gold standard; final speech ("Cross of Gold") given by \textbf{William Jennings Bryan} of NE powerfully defended free silver through power of words, delivery 
            \item Bryan nominated for Dem. president; youngest nominee at 36; some saw as too powerful, many admirers saw as symbol of rural, middle-class America
        \end{itemize}
        \item Populists had expected both parties to support gold standard, Democrats stole thunder $\to$ could either continue w/ reduced voters or endorse Bryan and lose partisan identity
        \begin{itemize}
            \item Ultimately endorsed Bryan
        \end{itemize}
    \end{itemize}
    \textbf{The Republicans were confident of their win in the 1896 election, nominating gold standard William McKinley, who had been heavily involved in the protective tariff acts. The Democrats, on the other hand, surprisingly nominated youthful William Jennings Bryan for his "Cross of Gold" speech emphasizing the importance of free silver. Populists, surprised by having lost their critical argument against the gold standard, were forced to lose their identity by endorsing Bryan.}}
    \cornell{How did the Republicans win the 1896 election?}{\begin{itemize}
        \item Republicans feared Bryan victory $\to$ stuck w/ tradition 
        \begin{itemize}
            \item Poured money into campaign (\$7m to \$300k)
            \item McKinley followed traditional path of not taking active role in campaigning, instead receiving Republicans at door and speaking w/ them
        \end{itemize}
        \item Bryan violated standard by actively campaigning throughout nation in several villages and by addressing millions
        \begin{itemize}
            \item Despite pioneering modern presidential politics, did more short-term harm than good w/ many voters lost due to belief that Bryan's behavior was undignified
        \end{itemize}
        \item McKinley won election by signif. electoral margin; Democrats' concentrated/narrow campaign won Southern/Western vote but too small 
        \item Populist party had given up partisan identity through fusion for success of Democrats; dissolved soon after the election
    \end{itemize}
    \textbf{McKinley stuck to tradition, receiving large sums of money for the campaign and rarely campaigning, instead receiving Republicans at his home for discussion. Bryan, on the other hand, pioneered modern presidential politics by campaigning throughout the nation; several deemed his behavior undignified. McKinley ultimately won the election due to Bryan's loss of several voters, leading to the collapse of the Populist Party.}}
    \cornell{How did McKinley work to recover from the economic devastation from the Panic of 1893?}{\begin{itemize}
        \item Little dissent to McKinley's administration w/ 1897 labor unrest $\to$ most Americans frightened, agrarian protest collapsed w/ loss of Populist party 
        \item McKinley dedicated to preserving stability amidst economic turmoil 
        \begin{itemize}
            \item Central issue: higher tariffs w/ \textbf{Dingley Tariff} $\uparrow$ duties, reaching peak in U.S. history 
            \item Little concern for issue of silver: McKinley considered some agreements w/ Europe, sending commission to discuss w/ Britain and France; no agreement reached $\to$ Gold Standard Act of 1900 confirming gold standard 
            \begin{itemize}
                \item Effectively ended debate over free silver and gold standard
            \end{itemize}
            \item Prosperity returned in 1898
            \begin{itemize}
                \item Foreign crop failures $\to$ rising prices for American crops 
                \item American business exited "bust" part of cycle to resume expansion 
            \end{itemize}
        \end{itemize}
        \item Gold standard not flawless: Western countries experienced $\uparrow$ growth but money supply had not kept up bc. gold supply remained near-constant until late 1890s w/ gold deposits in Australia, South Africa, Alaska
        \begin{itemize}
            \item Populists would have likely been correct abt. economic collapse if prosperous gold deposits not discovered
        \end{itemize}
    \end{itemize}
    \textbf{McKinley entered his presidency with little dissent due to fear of another economic collapse as well as the weakening of the agrarian revolt. He centered his campaign on the preservation of economic stability through increased tariffs and the federal finalization of the gold standard; furthermore, the tides turned in 1898 to bring prosperity back, with crop failures leading to rising prices for American crops. Several gold deposits were discovered in the late 1890s: had they not been found, the reliance of the currency on the quantity of gold would surely have eventually backfired.}}
    \cornell[Stirrings of Imperialism]{How did the U.S. begin to morph into a growing imperial power?}{\textbf{Two decades after the Civil War, Americans began to turn to imperialism once again, stressing a new Manifest Destiny justified by their success with the natives, the closing of the frontier, Social Darwinism, the superiority of English speakers, and the power of the U.S. Navy. American leaders made several negotiations with Latin American nations to expand their markets. However, they chose to settle on Hawaii for trade with China, destroying the native Hawaiian population and culture with disease, forming a prosperous plantation-based sugar market and eventually encroaching on Hawaiian independence by forcibly annexating the island. Samoa, too, was soon settled for trade with the Pacific, with Germany and the U.S. splitting the islands among themselves.}}
    \cornell{What principles resumed the fixation on Manifest Destiny?}{\begin{itemize}
        \item Encroachment on lands of natives $\to$ Americans inspired to continue exerting control over dependent grps.
        \begin{itemize}
            \item Supposed closing of frontier $\to$ fears that resources would become scarce, new markets needed to recover from depression
            \item Populist revolts $\to$ politicians sought foreign policy as outlet for destabilizing domestic conflict
            \item Foreign trade $\uparrow$ in importance w/ $\uparrow$ exports $\to$ desire for new markets
        \end{itemize}
        \item Americans witnessed imperialism in Europe w/ powers partitioning Africa, Far East in Chinese Empire, becoming increasingly inspired to take up imperialism themselves
        \begin{itemize}
            \item Senator \textbf{Henry Cabot Lodge} of MA applied Social Darwinism to justify need for Americans to subjugate nations to dominance
            \item \textbf{John Fiske} produced 1885 magazine article predicting that English-speakers would control all lands not already established as "civilization" w/ natives as prime example
            \item \textbf{John W. Burgess} supported imperialism w/ study claiming Anglo-Saxon/Teutonic nations were strongest pol. administrators; owed it to other societies to pull out of "barbarism" w/ guidance
            \item \textbf{Alfred Thayer Mahan}, captain in the U.S. Navy, argued that countries w/ sea access were naturally dominant if access to strong domestic economy, foreign trade, merchant marine, navy; argued for Panama Canal to join oceans
            \begin{itemize}
                \item Feared U.S. Navy too small, but govt. launched shipbuilding program
            \end{itemize}
        \end{itemize}
    \end{itemize}
    \textbf{Americans sought to engage in imperialist ventures once again due to the closing of the frontier, the importance of foreign trade, and their past successes with the natives. Several arguments were made to justify imperialism, including Social Darwinism, the natural dominance of English speakers, and the U.S. naval prowess.}}
    \cornell{How did the United States assert their dominance over Latin America?}{\begin{itemize}
        \item James G. Blaine, Repub. sec. of state, sought to expand to Latin America to find markets for surplus goods
        \begin{itemize}
            \item Organized 1889 Pan-American Congress w/ 19 nations agreeing to create Pan-American Union in DC to spread information and communicate w/ member nations; rejected Blaine's customs union, arbitration procedures
        \end{itemize}
        \item Cleveland administration supported Venezuela against GB in 1895, arguing that British were violating Monroe Doctrine in dispute
        \begin{itemize}
            \item Special commission created to settle w/ resistance $\to$ war; GB quickly agreed to discuss
        \end{itemize}
    \end{itemize}
    \textbf{Early efforts to expand markets to Latin America were made in James G. Blaine's Pan-American Congress, which formed a weak Pan-American Union for easy communication. Cleveland showed his support for Latin America by supporting Venezuela in a dispute against Britain, even claiming to be willing to go to war for them. }}
    \cornell{How did Hawaii fit into America's larger imperialistic goals?}{\begin{itemize}
        \item Islands of Hawaii critical to U.S. for trade w/ China $\to$ U.S. Navy sought Hawaii's Pearl Harbor as permanent naval base, Americans who had settled pushing for American control
        \item Hawaii had long been self-sufficient w/ agricultural/fishing society; pop. of $\approx$ 500k when first Americans arrived in late 18th c. 
        \begin{itemize}
            \item Battles betw. chieftains $\to$ King Kamehameha I emerged as dominant, welcoming U.S. traders for prosperous trade w/ China due to mutual benefits
            \item American missionaries soon began to settle to convert natives; Boston trader William Hooper created sugar plantation in 1830s
        \end{itemize}
        \item American arrival devastated Hawaiian society
        \begin{itemize}
            \item Brought diseases to vulnerable Hawaiians $\to$ more than half died by 1850, another half by 1900 
            \item Missionaries sought to stamp out native religion
            \item Settlers brought firearms, liquor, widespread commerce $\to$ traditional character 
            \item King Kamehameha III agreed to est. constitutional monarchy w/ white American G.P. Judd becoming prime minister
        \end{itemize}
        \item 1887: treaty allowed naval base at Pearl Harbor w/ sugar export from Hawaii becoming principal part of economy, dominating native Hawaiians
        \begin{itemize}
            \item Several employers exclusively employed Asian immigrants as workers due to belief of greater reliability
            \item Some planters created mixed-race workforce to prevent unified group from challenging authority
        \end{itemize}
        \item Hawaiians protested w/ election of \textbf{Queen Liluokalani}, nationalist seeking to resist U.S. control
        \begin{itemize}
            \item Quickly lost control after U.S. eliminated need for sugar industry $\to$ economy devastated
            \item Amer. planters desired U.S. control to eliminate tariffs for trade $\to$ staged 1893 revolution, called for protection; navy supported rebels $\to$ queen submitted
            \item Provisional govt. dominated by Americans despite < 5\% of total pop.; negotation for annexation finally approved in 1898 when Repubs. returned to power
        \end{itemize}
    \end{itemize}
    \textbf{Hawaii had long been critical for American trade with China; the Hawaiian was initially accepting. However, Americans began to settle rapidly, with disease destroying the native population and culture. The decline of the sugar industry further curtailed the power of Hawaiians; an American revolt supported by the U.S. forced submission and acceptance of annexation.}}
    \cornell{How did the U.S. exert imperialistic control over Samoa?}{\begin{itemize}
        \item Samoa critical for trade w/ Pacific $\to$ businesses, navy sought annexation of Samoan harbor; Hayes formed 1878 treaty w/ leaders to create naval station 
        \item GB/Germany also sought Samoa $\to$ formed similar treaties $\to$ conflict over Samoan dominance, sometimes nearing war 
        \begin{itemize}
            \item Finally agreed to share power, but three-way control unsuccessful $\to$ Germany/U.S. divided islands, giving other Pacific islands to GB as compensation 
            \item U.S. kept important harbor at Pago Pago
        \end{itemize}
    \end{itemize}
    \textbf{Samoa was essential for U.S. trade with the Pacific, leading businesses and the navy to seek annexation. Hayes formed a treaty for harbor use around the same time as Britain and Germany, leading to conflict over the islands; Britain eventually departed in exchange for other islands while Germany and the U.S. divided the islands among themselves.}}
    \end{document}