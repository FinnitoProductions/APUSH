\documentclass[a4paper]{article}
    \usepackage[T1]{fontenc}
    \usepackage{tcolorbox}
    \usepackage{amsmath}
    \tcbuselibrary{skins}
    
    \usepackage{background}
    \SetBgScale{1}
    \SetBgAngle{0}
    \SetBgColor{red}
    \SetBgContents{\rule[0em]{4pt}{\textheight}}
    \SetBgHshift{-2.3cm}
    \SetBgVshift{0cm}
    \usepackage[margin=2cm]{geometry} 
    
    \makeatletter
    \def\cornell{\@ifnextchar[{\@with}{\@without}}
    \def\@with[#1]#2#3{
    \begin{tcolorbox}[enhanced,colback=gray,colframe=black,fonttitle=\large\bfseries\sffamily,sidebyside=true, nobeforeafter,before=\vfil,after=\vfil,colupper=blue,sidebyside align=top, lefthand width=.3\textwidth,
    opacityframe=0,opacityback=.3,opacitybacktitle=1, opacitytext=1,
    segmentation style={black!55,solid,opacity=0,line width=3pt},
    title=#1
    ]
    \begin{tcolorbox}[colback=red!05,colframe=red!25,sidebyside align=top,
    width=\textwidth,nobeforeafter]#2\end{tcolorbox}%
    \tcblower
    \sffamily
    \begin{tcolorbox}[colback=blue!05,colframe=blue!10,width=\textwidth,nobeforeafter]
    #3
    \end{tcolorbox}
    \end{tcolorbox}
    }
    \def\@without#1#2{
    \begin{tcolorbox}[enhanced,colback=white!15,colframe=white,fonttitle=\bfseries,sidebyside=true, nobeforeafter,before=\vfil,after=\vfil,colupper=blue,sidebyside align=top, lefthand width=.3\textwidth,
    opacityframe=0,opacityback=0,opacitybacktitle=0, opacitytext=1,
    segmentation style={black!55,solid,opacity=0,line width=3pt}
    ]
    
    \begin{tcolorbox}[colback=red!05,colframe=red!25,sidebyside align=top,
    width=\textwidth,nobeforeafter]#1\end{tcolorbox}%
    \tcblower
    \sffamily
    \begin{tcolorbox}[colback=blue!05,colframe=blue!10,width=\textwidth,nobeforeafter]
    #2
    \end{tcolorbox}
    \end{tcolorbox}
    }
    \makeatother

    \parindent=0pt
    \newcommand{\chapternumber}{19}
    \newcommand{\chaptertitle}{From Crisis to Empire}
    \title{\vspace{-3em}
    \begin{tcolorbox}
    \Huge\sffamily \begin{center} AP US History  \\
    \LARGE Chapter \chapternumber$\text{ }$- \chaptertitle \\
    \Large Finn Frankis \end{center} 
    \end{tcolorbox}
    \vspace{-3em}
    }
    \date{}
    \author{}
    
    \begin{document}
        \maketitle
        \SetBgContents{\rule[0em]{4pt}{\textheight}}
        \cornell[Key Concepts]{What are this chapter's key concepts?}{\begin{itemize}
            \item \textbf{6.1.I.E} - Businesses, foreign policy $\to$ outside borders for influence over Asia, Latin America, Pacific rim 
            \item \textbf{6.1.III.B} - Agricultural market consolidation, heavy dependence on railroad $\to$ farmers formed organizations
            \item \textbf{6.1.III.C} - Economic instability $\to$ agrarian People's/Populist Party seeking $\uparrow$ govt. regulation
            \item \textbf{6.3.1.C} - Artists/critics like agrarians, utopians, socialists, Social Gospel advocates emphasized different visions for U.S. society
            \item \textbf{6.3.II.A} - Major political parties divided along Civil War boundaries; fought over tariffs/currency despite reformers arguing greed/self-interest was corrupting govt.
            \item \textbf{7.3.I.A} - Imperialists stressed econ. opportunities, racial theories, Euro. competition, 1890 idea of "closed" frontier to stress importance of expanding American culture across globe
            \item \textbf{7.3.I.B} - Anti-imperialists stressed self-determination, contrasting racial theories, isolationism to argue no need to extend territory
            \item \textbf{7.3.I.C} - Amer. win in Sp.-Amer. war $\to$ U.S. acquired Carib./Pacific territories, suppressed Filipino nationalism
        \end{itemize}}
    \end{document}