\documentclass[a4paper]{article}
    \usepackage[T1]{fontenc}
    \usepackage{tcolorbox}
    \usepackage{amsmath}
    \tcbuselibrary{skins}
    
    \usepackage{background}
    \SetBgScale{1}
    \SetBgAngle{0}
    \SetBgColor{red}
    \SetBgContents{\rule[0em]{4pt}{\textheight}}
    \SetBgHshift{-2.3cm}
    \SetBgVshift{0cm}
    \usepackage[margin=2cm]{geometry} 
    
    \makeatletter
    \def\cornell{\@ifnextchar[{\@with}{\@without}}
    \def\@with[#1]#2#3{
    \begin{tcolorbox}[enhanced,colback=gray,colframe=black,fonttitle=\large\bfseries\sffamily,sidebyside=true, nobeforeafter,before=\vfil,after=\vfil,colupper=blue,sidebyside align=top, lefthand width=.3\textwidth,
    opacityframe=0,opacityback=.3,opacitybacktitle=1, opacitytext=1,
    segmentation style={black!55,solid,opacity=0,line width=3pt},
    title=#1
    ]
    \begin{tcolorbox}[colback=red!05,colframe=red!25,sidebyside align=top,
    width=\textwidth,nobeforeafter]#2\end{tcolorbox}%
    \tcblower
    \sffamily
    \begin{tcolorbox}[colback=blue!05,colframe=blue!10,width=\textwidth,nobeforeafter]
    #3
    \end{tcolorbox}
    \end{tcolorbox}
    }
    \def\@without#1#2{
    \begin{tcolorbox}[enhanced,colback=white!15,colframe=white,fonttitle=\bfseries,sidebyside=true, nobeforeafter,before=\vfil,after=\vfil,colupper=blue,sidebyside align=top, lefthand width=.3\textwidth,
    opacityframe=0,opacityback=0,opacitybacktitle=0, opacitytext=1,
    segmentation style={black!55,solid,opacity=0,line width=3pt}
    ]
    
    \begin{tcolorbox}[colback=red!05,colframe=red!25,sidebyside align=top,
    width=\textwidth,nobeforeafter]#1\end{tcolorbox}%
    \tcblower
    \sffamily
    \begin{tcolorbox}[colback=blue!05,colframe=blue!10,width=\textwidth,nobeforeafter]
    #2
    \end{tcolorbox}
    \end{tcolorbox}
    }
    \makeatother

    \parindent=0pt
    \newcommand{\chapternumber}{17}
    \newcommand{\chaptertitle}{Industrial Supremacy}
    \title{\vspace{-3em}
    \begin{tcolorbox}
    \Huge\sffamily \begin{center} AP US History  \\
    \LARGE Chapter \chapternumber$\text{ }$- \chaptertitle \\
    \Large Finn Frankis \end{center} 
    \end{tcolorbox}
    \vspace{-3em}
    }
    \date{}
    \author{}
    
    \begin{document}
        \maketitle
        \SetBgContents{\rule[0em]{4pt}{\textheight}}
        \cornell[Key Concepts]{What are this chapter's key concepts?}{\begin{itemize}
            \item \textbf{6.1.I.B} - Businesses utilized technology, resources, financial systems, marketing, and labor to produce more goods 
            \item \textbf{6.1.I.D} - Business magnates sought to raise profits by combining corporations into larger trusts/holding companies $\to$ wealth further concentrated
            \item \textbf{6.1.II.A} - Several supported laissez-faire economic policies for long-term economic growth, opposing govt. intervention even during crises
            \item \textbf{6.1.II.B} - The industrial workforce expanded and diversified due to migration, but child labor also increased
            \item \textbf{6.1.II.C} - Labor forces fought w/ managers over wages, conditions $\to$ formations of local/national unions to rise against business leaders
            \item \textbf{6.3.I.A} - Social Darwinism further used to justify inevitability of success of certain
            \item \textbf{6.3.I.B} - Some business leaders used Gospel of Wealth argument to push wealthy to assist poor through philanthropy
        \end{itemize}}
        \cornell[Sources of Industrial Growth]{What spurred long-term industrial development in the U.S.?}{\textbf{Industrial development was spurred greatly by the rapidly growing steel industry, which was tightly connected to the trailblazing railroad industry; the creation of the automobile and aeroplane revolutionized transportation. Corporations began to jump on the technology boom, funding research in labs and universities; several turned to scientifically optimizing the manufacturing process for efficiency. The stock-based corporation developed in the 1840s, with corporations growing rapidly through horizontal and vertical integration, most notably seen in Rockefeller's Standard Oil; trusts and mergers allowed companies to come together. Though several criticized the dominance which these companies began to hold over American society, they inevitably stimulated economic growth.}}
        \cornell{How did technology encourage long-term industrial growth?}{\begin{itemize}
            \item Iron/steel production most important w/ \underline{rapid growth} post-Civil War
            \begin{itemize}
                \item Henry Bessemer (GB) and William Kelly (US) simultaneously found how to convert iron $\to$ steel, known as Bessemer process, complimented by Mushet's idea of adding to melted iron
                \item 1868: Hewitt (US) introduced open-hearth process, soon replacing Bessemer process
                \item Steel dev. allowed rapid production of large pieces for rail cars, rails, girders for buildings
                \item Emerged in W. PA and E. OH due to natural iron ore, demand for anthracite fuel (common in PA) $\to$ Pittsburgh early center
                \item Rapid growth of industry $\to$ new sources found in MI, MN, AL w/ Cleveland, Chicago, Detroit, Birmingham growing 
                \item Stone furnaces for steel production replaced w/ brick in 1870s $\to$ far larger amounts could be produced at once
            \end{itemize}
            \item Transportation emerged to cater to steel industry
            \begin{itemize}
                \item Freighters to Great Lakes allowed growth of industry there
                \item Steam engines unloaded ore far more rapidly than humans
                \item Railroads closely connected to steel industry w/ steel used for rails/cars and rail companies providing instant market 
            \end{itemize}
            \item Lubrication in steel production $\to$ oil grew in relevance (not as fuel until much later) w/ petroleum reserves in PA producing large amts.
        \end{itemize}
        \textbf{The production of steel from iron was arguably the most important technological advancement spurring industrial growth; starting in Pennsylvania and Ohio but rapidly expanding outward, it was critical for the construction of railroads and also relied on railroads to reach new markets. Additionally, the need for lubrication led to the growth of the petroleum industry.}}
        \cornell{What characterized the development of the airplane and automobile industries?}{\begin{itemize}
            \item Automobile made possible by process of separating petrol from crude oil, German development of gas-powered engine (not initially portable) 
            \begin{itemize}
                \item Charles/Frank Duryea built first gasoline-driven vehicle in 1893
                \item Henry Ford built vehicle in 1896
                \item Industry grew extremely rapidly
            \end{itemize}
            \item Potential for flight became viable only in late 19th c. w/ experiments using balloons, kites, gliders
            \begin{itemize}
                \item Wilbur/Orville Wright began to work on glider powered by combustion engine; first test flight in 1903 at Kitty Hawk; could travel 23 miles by 1905
                \item Most substantial airplane design in France due to govt. funding; U.S. created National Advisory Committee on Aeronautics in 1915, but commercial flight saw hope only after Lindbergh's first intercontinental flight
            \end{itemize}
        \end{itemize}
        \textbf{The automobile industry was spurred by the process of retrieving usable fuel from crude oil as well as the development of the gas-powered engine; Henry Ford built his first vehicle in 1896. The flight industry first saw potential after by the Wright brothers had their first successful flight, but most early development came from France.}}
        \cornell{How did research and development spur industrial growth?}{\begin{itemize}
            \item Businesses began to sponsor research to face competition
            \begin{itemize}
                \item GE made corporate laboratory in 1900; several other large companies had followed by 1913
                \item $\uparrow$ corporate interest $\Leftrightarrow$ $\downarrow$ govt. interest $\to$ skilled scientists moved from govt. to corporations, research far more free-moving and decentralized
            \end{itemize}
            \item Scientists/engineers became increasingly divided as engineers worked at forefront of tech. for corps. while many scientists insisted on studying less directly practical subjects (though far fewer than in Europe)
            \item U.S. universities funded by corps $\to$ developed research institutes for industrial economy (not the case in Europe)
        \end{itemize}
        \textbf{Several large U.S. corporations sponsored research both at laboratories and universities to cater to a rapidly changing market. The increased corporatization of science created a significant rift between knowledge-driven scientists and market-driven engineers.}}
        \cornell{How did several great thinkers turn to optimizing manufacturing?}{\begin{itemize}
            \item Led by Frederick Winslow Taylor, "Taylorists" stressed subdivision of tasks to limit dependence on single employee, reduce required training to optimize efficiency through "scientific management"
            \item Henry Ford introduced mass production through moving assembly line in automobile factories to rapidly cut production times
        \end{itemize}
        \textbf{The Taylorists stressed the concept of "scientific management", subdividing tasks to optimize overall efficiency. Henry Ford similarly revolutionized mass production with the moving assembly line to cut production times.}}
        \cornell{How did the railroad remain the primary agent of industry?}{\begin{itemize}
            \item Railroads blazed way for new development, spurring commercial activity anywhere they expanded
            \begin{itemize}
                \item Passing thru. forests $\to$ lumberers; West $\to$ buffalo hunters to kill buffalo, bring cattle
                \item As railroad hub, Chicago became slaughterhouse of nation w/ most cattle going there
            \end{itemize}
            \item Pre-1880s, time determined by sun position $\to$ even neighbouring towns had diff. times $\to$ railroad companies agreed in 1883 on four time zones each an hour apart for scheduling
            \item Railroads expanded rapidly due to fed./state/local subsidies, investors from abroad, railroad combinations $\to$ power restricted to hands of few 
            \begin{itemize}
                \item Tycoons like Cornelius Vanderbilt, James J. Hill, Collis P. Huntingon represented nation w/ power in hands of few 
                \item Led to creation of modern corporation
            \end{itemize}
        \end{itemize}
        \textbf{Commercial activity continued to follow railroads wherever they emerged, throughout the nation. Railroads also led to the creation of four standard time zones and were spurred by government subsidies, foreign investors, and railroad combinations.}}
        \cornell{What was the structure of the U.S. corporation?}{\begin{itemize}
            \item Post-Civil War, railroad leaders/industrialists realized that no single person could dominate economy
            \item Incorporation laws of 1830s/1840s allowed businesses to earn money by selling stock to public with "limited liability" (investors could only lose amount invested, not having to cover company's debt) 
            \begin{itemize}
                \item Wealth from stocks $\to$ corporations could take on large projects
            \end{itemize}
            \item First corporations were railroads, but quickly expanded outward 
            \begin{itemize}
                \item Modest immigrant Andrew Carnegie dominated steel industry, buying out competitors, negotiating railroad deals, buying/leasing coal mines w/ associate Henry Clay Frick
                \begin{itemize}
                    \item Financed projects w/ wealth from sale of stock 
                    \item Sold to banker J. Pierpont Morgan in 1901 for \$450m, who created U.S. Steel Corporation, near-monopolizing enterprise
                \end{itemize}
                \item Gustavus Swift turned Chicago meatpacking company into national corporation by selling to mil. in Civil War
                \item Isaac Singer patended sewing machine in 1851, creating manufacturing corporation 
            \end{itemize}
            \item Corporations began to formally approach management w/ systematic techniques creating hierarchy of control, concept of "middle manager" between owners and workers 
        \end{itemize}
        \textbf{Incorporation laws in the 1830s allowed businesses to safely sell stock to the public, creating the modern corporation. Although corporations began with railroads, they soon expanded outward, with Andrew Carnegie dominating the steel industry. As corporations expanded, new managerial techniques emerged, most notably a formal hierarchy of control with "middle managers" between owners and workers.}}
        \cornell{How did businesses consolidate power in corporate America?}{\begin{itemize}
            \item Businesses consolidated power through \textbf{horizontal integration} (merging similar firms in same enterprise) and \textbf{vertical integration} (taking over businesses on which the corporation relied)
            \item Rockefeller's Standard Oil, created in Cleveland post-Civil War, initially expanded horizontally w/ rapid purchase of competitors; soon expanded vertically w/ purchase of factory, freight cars, warehouses, pipelines to prevent reliance on other companies
            \item Consolidation accepted as method of preventing social instability from \underline{too much} competition
            \item Railroads made pool arrangements betw. companies to stabilize rates, divide markets (known as cartels), but rarely viable due to requirement for cooperation from all companies 
        \end{itemize}
        \textbf{Rockefeller's Standard Oil company is the greatest example of an expanding company, initially absorbing competitors in the same market (horizontal integration), and later taking over operations on which it depended (vertical integration). Consolidation was so popular because it was seen as a solution to the inevitable instability of an overcompetitive market.}}
        \cornell{How did the ideas of the trust and the holding company emerge?}{\begin{itemize}
            \item Failure of pools $\to$ new consolidation methods relying on central control 
            \item "Trust" began w/ Standard Oil, perfected by J.P. Morgan; allowed stockholders to form trust agreements, transferring stocks to small grp. of trustees in exchange for shares in trust
            \begin{itemize}
                \item Owners relied on trustees to bring in profits
                \item Several trustees exercised great power over major corporations
            \end{itemize}
            \item 1889: NJ allowed companies to buy other ones $\to$ corporate mergers possible
            \begin{itemize}
                \item Standard Oil relocated to NJ, created "holding company" to buy stock from trustees 
            \end{itemize}
            \item By turn of century, 1\% of corporations controlled more than 33\% of manufacturing, in hands of large bankers and industrial magnates; methods frequently criticized but indubitably spurred growth
        \end{itemize}
        \textbf{The "trust" allowed investors to transfer their stocks to trustees in exchange for shares in a trust, giving a small group of trustees great power. Corporate mergers were first made possible in New Jersey and overshadowe the trust, instead creating "holding companies" which controlled several others.}}
        \cornell[Capitalism and Its Critics]{What were the main supporting arguments for and opposing arguments to capitalism?}{\textbf{Capitalist supporters incorporated its tenets into ideals of individualism and Social Darwinism, stressing that the most successful had reached such prominence out of their own hard work, independence, and grit. Furthermore, the concentration of great wealth in the hands of the fittest, many argued, allowed them to redistribute it back into society through philanthropy and public works, allowing all to be successful. However, several Americans resented the laissez-faire policies of capitalism, the creation of a large impoverished class, ideas of competition, and large monopolies due to their high prices, restrictions on individualism, and evidencing of the gap between rich and poor.}}
        \cornell{How was capitalism tied into ideas of individualism?}{\begin{itemize}
            \item Capitalist defenders argued industry allowed every individual chance to succeed
            \begin{itemize}
                \item Nearly all millionaries claimed to be "self-made men"; very few truly were (such as Carnegie, Rockefeller, Harriman)
                \item Several rose to power thru. ruthlessness, arrogance, corruption (\textit{ex}: Cornelius Vanderbilt publicly claimed disregard for law)
                \item Greatest industrialists enjoyed heavy govt. support
            \end{itemize}
            \item Most businesspeople ultimately failed due to absorption by another company, great competition
            \begin{itemize}
                \item Many industries never monopolized, w/ smaller companies constantly fighting for dominance
            \end{itemize}
        \end{itemize}
        \textbf{Supporters of capitalism argued that it gave every individual a chance at success; truly, however, most had started with wealth, rose to power in ruthless or corrupt ways, and were consistently supported by the government. Most businesspeople were quickly put out of business in the competitive industrial climate.}}
        \cornell{How did some supporters of capitalism stress Social Darwinism?}{\begin{itemize}
            \item Most magnates stressed success was due to personal virtue, worthiness; Rockefeller felt all impoverished were that way by their own fault
            \item Social Darwinists stressed that only the fittest flourished in the industrial world
            \begin{itemize}
                \item \textbf{Herbert Spencer}, most prominent Social Darwinist, argued society benefited by survival of the fittest; supported by several intellectuals (like Yale's William Graham Sumner)
                \item Businessmen supported due to pairing of idea of freedom to struggle/succeed with their success; supported lower wages, reduced govt. influence 
            \end{itemize}
            \item Social Darwinism complimented by supply and demand of Adam Smith, idea of "invisible hand" pushing market forward
            \item Main theories far from realities of economy w/ constant work to protect from competition, going against natural laws of marketplace
        \end{itemize}
        \textbf{Social Darwinism was supported by several business magnates, who used it to justify their success out of virtue and natural worth; furthermore, classical economists pushed concepts of supply and demand and the idea of natural market forces encouraging competition. However, the active attempts by business magnates to limit competition fundamentally contradicted the ideas they so strongly supported.}}
        \cornell{What was the Gospel of Wealth?}{\begin{itemize}
            \item Andrew Carnegie stressed in \textit{Gospel of Wealth} importance for wealthy to use money for good, creating "trust funds" for good of community, philanthropy to help poor thru. schools and libraries
            \item Wealth eventually seen as a commodity accessible to all; \textbf{Russell H. Conwell}, Baptist minister, delivered "Acres of Diamonds" countless times, arguing that all people can access great wealth if done in the right way
            \item \textbf{Horatio Alger} promoted success story through novels about impoverished boys working hard to become prominent
        \end{itemize}
        \textbf{The Gospel of Wealth was the notion that all wealthy people should use their excess money to support the public good; the idea of public wealth paired closely with the concept that all could access wealth if done in the right way. Though rarely the case, several popularized the success story, giving hope to many.}}
        \cornell{How did some argue against capitalism?}{\begin{itemize}
            \item \textbf{Lester Frank Ward} was a Darwinist, but not in social context: felt society molded not by natural processes, but instead by active human (govt.) intervention and intelligence
            \item Radical \textbf{Socialist Labor Party} by Daniel De Leon never became significant political force; eventually, dissidents broke off, creating long-lasting \textbf{American Socialist Party}
            \item \textbf{Henry George} of CA wrote \textit{Progress and Poverty}, argued that society could not progressed with the presence of a growing impoverished class
            \begin{itemize}
                \item Felt those who grew wealthy from rising land values benefitted from growth of surrounding society $\to$ undeserved wealth should be distributed to society with "single tax" equalizer
            \end{itemize}
            \item \textbf{Edward Bellamy} wrote utopian \textit{Looking Backward}, telling story of Bostonian travelling to 2000, seeing all corporations having formed into single, peaceful trust controlled by govt., equally distributing industrial benefits among all w/o class divisions
        \end{itemize}
        \textbf{Ward felt society required more direct and active governmental intervention; George felt that the impoverished class prevented any true progress and a single equalizer tax should be implemented on all to prevent a small group from dominating; Bellamy sought a peaceful society without class divisions and led by a single, government-led corporation looking out for the good of the people.}}
        \cornell{How did several Americans begin to oppose the growing monopoly?}{\begin{itemize}
            \item Lower classes blamed monopoly for creating unstable (production > demand) economy due to freedom of monopolist companies to set \underline{artifical prices} (most notably in railroad industry)
            \begin{itemize}
                \item Supported by several recessions between 1873 and 1893, with the economy seeming close to collapse
            \end{itemize}
            \item Others feared monopoly for restriction of individualism: corporate dominance of an industry prevented simply anyone from rising up against them
            \item Resentment for wealthy emerged due to many having famously ornate and ostentatious lifestyles (like Vanderbilts w/ large mansions) 
            \begin{itemize}
                \item Impoverished saw visible indication of growing gap betw. rich and poor 
            \end{itemize}
        \end{itemize}
        \textbf{The American lower classes blamed the monopoly for economic instability with artificial, unfair prices, for the restriction of individualism with limited opportunities for small businesses to face magnates, and for the creation of a small yet ostenatious wealthy classes frequently parading their wealth.}}
    \end{document}