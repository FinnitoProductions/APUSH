\documentclass[a4paper]{article}
    \input{../notesheader.tex}
    \newcommand{\chapternumber}{17}
    \newcommand{\chaptertitle}{Industrial Supremacy}
    \title{\vspace{-3em}
    \begin{tcolorbox}
    \Huge\sffamily \begin{center} AP US History  \\
    \LARGE Chapter \chapternumber$\text{ }$- \chaptertitle \\
    \Large Finn Frankis \end{center} 
    \end{tcolorbox}
    \vspace{-3em}
    }
    \date{}
    \author{}
    
    \begin{document}
        \maketitle
        \SetBgContents{\rule[0em]{4pt}{\textheight}}
        \cornell[Key Concepts]{What are this chapter's key concepts?}{\begin{itemize}
            \item \textbf{6.1.I.B} - Businesses utilized technology, resources, financial systems, marketing, and labor to produce more goods 
            \item \textbf{6.1.I.D} - Business magnates sought to raise profits by combining corporations into larger trusts/holding companies $\to$ wealth further concentrated
            \item \textbf{6.1.II.A} - Several supported laissez-faire economic policies for long-term economic growth, opposing govt. intervention even during crises
            \item \textbf{6.1.II.B} - The industrial workforce expanded and diversified due to migration, but child labor also increased
            \item \textbf{6.1.II.C} - Labor forces fought w/ managers over wages, conditions $\to$ formations of local/national unions to rise against business leaders
            \item \textbf{6.3.I.A} - Social Darwinism further used to justify inevitability of success of certain
            \item \textbf{6.3.I.B} - Some business leaders used Gospel of Wealth argument to push wealthy to assist poor through philanthropy
        \end{itemize}}
    
    \end{document}