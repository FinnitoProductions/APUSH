\documentclass[a4paper]{article}
\usepackage{tcolorbox}
\usepackage{amsmath}
\tcbuselibrary{skins}

\title{
\vspace{-3em}
\begin{tcolorbox}
\Huge\sffamily \begin{center} Chapter 32  \mbox{} \\ \huge The Age of Globalization \mbox{} \\
\LARGE Finn Frankis \mbox{} \\
\Large AP US History - August 3{$^\text{rd}$}, 2018 \end{center}
\end{tcolorbox}
\vspace{-3em}
}
\date{}
\author{}

\usepackage{background}
\SetBgScale{1}
\SetBgAngle{0}
\SetBgColor{red}
\SetBgContents{\rule[0em]{4pt}{\textheight}}
\SetBgHshift{-2.3cm}
\SetBgVshift{0cm}
\usepackage[margin=2cm]{geometry}

\makeatletter
\def\cornell{\@ifnextchar[{\@with}{\@without}}
\def\@with[#1]#2#3{
\begin{tcolorbox}[enhanced,colback=gray,colframe=black,fonttitle=\large\bfseries\sffamily,sidebyside=true, nobeforeafter,before=\vfil,after=\vfil,colupper=blue,sidebyside align=top, lefthand width=.3\textwidth,
opacityframe=0,opacityback=.3,opacitybacktitle=1, opacitytext=1,
segmentation style={black!55,solid,opacity=0,line width=3pt},
title=#1
]
\begin{tcolorbox}[colback=red!05,colframe=red!25,sidebyside align=top,
width=\textwidth,nobeforeafter]#2\end{tcolorbox}%
\tcblower
\sffamily
\begin{tcolorbox}[colback=blue!05,colframe=blue!10,width=\textwidth,nobeforeafter]
#3
\end{tcolorbox}
\end{tcolorbox}
}
\def\@without#1#2{
\begin{tcolorbox}[enhanced,colback=white!15,colframe=white,fonttitle=\bfseries,sidebyside=true, nobeforeafter,before=\vfil,after=\vfil,colupper=blue,sidebyside align=top, lefthand width=.3\textwidth,
opacityframe=0,opacityback=0,opacitybacktitle=0, opacitytext=1,
segmentation style={black!55,solid,opacity=0,line width=3pt}
]

\begin{tcolorbox}[colback=red!05,colframe=red!25,sidebyside align=top,
width=\textwidth,nobeforeafter]#1\end{tcolorbox}%
\tcblower
\sffamily
\begin{tcolorbox}[colback=blue!05,colframe=blue!10,width=\textwidth,nobeforeafter]
#2
\end{tcolorbox}
\end{tcolorbox}
}
\makeatother

\parindent=0pt

\begin{document}
\maketitle
\SetBgContents{\rule[0em]{4pt}{\textheight}}
\cornell[A Resurgence of Partisanship]{What defined the American resurgence of partisanship?}{\textbf{The Clinton administration struggled with the Republican dominance in Congress and was plagued by numerous scandals. Despite this, Clinton ushered in a time of budget surplus and resolved foreign disputes. George W. Bush succeeded Clinton, winning a tight election against Al Gore. Bush relied on Clinton's surplus to implement a tax reduction and implement staunchly Republican policies. In 2004, Bush won another election against John Kerry, once again by a small margin.}}
\cornell{How did the Clinton presidency start off?}{
    \begin{itemize}
        \item Clinton administration initially defined by many misfortunes, requiring numerous withdrawals
        \item Defined by a few important achievements
        \begin{itemize}
            \item Included budget approval away from Reagan-Bush years, including tax increase on wealthy, reduction in government spending, expansion of tax credits to working people
            \item Advocated free trade, globalism
            \begin{itemize}
                \item Seen in long battle for approval of NAFTA, eliminating trade barriers
                \item Received approval of far-reaching trade agreement in GATT
            \end{itemize}
            \item Saw major reform of health care system, supervised by task force led by wife
            \begin{itemize}
                \item Reform promised to guarantee coverage to all Americans
            \end{itemize}
            \item Some foreign successes, including negotation to end war between Muslims and Christians in Bosnia through partitioning
        \end{itemize}
    \end{itemize}
    \textbf{The Clinton presidency was initially defined by a few major setbacks requiring major changes in policy; however, Clinton later achieved free trade agreements, began major reform of health care system, and foreign successes.}
}
\cornell{What led to the Republican resurgence in 1994?}{
    \begin{itemize}
        \item Republicans gained both houses of Congress in 1994, taking advantage of this to construct ambitious legislative program
        \begin{itemize}
            \item Proposed measures to transfer power from federal government to states to consequently reduce federal spending
            \item Hoped to restructure Medicare program
        \end{itemize}
        \item Clinton responded to Republican majority by shifting agenda to center
        \begin{itemize}
            \item Proposed tax cuts and budget balances to align with Republicans
            \item Still challenging to find compromise, leading to federal shutdown for several days due to inability to agree on budget 
            \begin{itemize}
                \item Discredited Republican leadership, improving Clinton's standings
            \end{itemize}
        \end{itemize}
    \end{itemize}
    \textbf{The Republican resurgence of 1994 was initially caused by a Republican majority in both houses of Congress. Although the Republicans took this opportunity to construct a new legislative program, their inability to agree with Clinton on key matters culminating in a federal government shutdown ultimately discredited their leadership and improved Clinton's standings.}}
\cornell{What was the result of the election of 1996?}{
    \begin{itemize}
        \item Clinton reached commanding position for reelection by 1996, unopposed for nomination
        \begin{itemize}
            \item Faced \textbf{Robert Dole}, senator unable to inspire enthusiasm even within party
            \item Clinton reached position of high popularity due to centrist stance, undermining Republicans and championing ideals promoted by Reagan such as peace and prosperity
        \end{itemize}
        \item Congress passed many important bills as election neared
        \begin{itemize}
            \item Raised legal minimum wage
            \item Clinton reluctantly signed \textbf{welfare reform} bill 
            \begin{itemize}
                \item Ended guaranteed federal assistance to families with dependent children
                \item Transferred majority of power to state governments
                \item Shifted welfare benefits to those with low-wage jobs rather than those without jobs
            \end{itemize}
        \end{itemize}
        \item Clinton won election despite slight campaign flagging by conclusion
        \begin{itemize}
            \item Failed to regain either house of Congress
            \item First Republican president to win two terms since Franklin Roosevelt
        \end{itemize}
    \end{itemize}
    \textbf{Clinton won the election against Robert Dole by a significant amount due to his centrist stance and important bills passed as the election approached, including the raise of minimum wage.}
}
\cornell{What major events marked Clinton's second term as president?}{
    \begin{itemize}
        \item Clinton still faced hostile Republican Congress
        \begin{itemize}
            \item Forced to propose modest tax agenda with tax cuts, credits for middle-class Americans
            \item Negotiated balanced budget with Republicans, generating first surplus in 30 years by 1998
        \end{itemize}
        \item Despite having been faced with many scandals, most extreme was denied sexual relations with young intern \textbf{Lewinsky}
        \begin{itemize}
            \item Charged for having lied about events in deposition
            \item Continued to deny charges while heavily backed by public
            \begin{itemize}
                \item Popularity soared to record levels
            \end{itemize}
            \item Scandal revived after Lewinsky testified about relationship with Clinton 
            \begin{itemize}
                \item After \textbf{special counsel Starr} subpoenaed Clinton, president finally agreed to "improper relationship"
                \item Recommended impeachment to Congress
            \end{itemize}
            \item Full House approved impeachment by 1998, finally moved to Senate which ended in acquittal
        \end{itemize}
        \item Serious foreign policy crisis emerged in 1999 in Balkans
        \begin{itemize}
            \item Serbian government and Kosovo separatists engaged in bitter civil war
            \item NATO forces dominated by U.S. began to bomb Serbians, leading to cease-fire in exchange for Serbian withdrawals
            \item Precarious peace followed
        \end{itemize}
        \item Despite numerous scandals, Clinton ended eight years with popularity higher than initially due to overall stability and prosperity
    \end{itemize}
    \textbf{Clinton's second term began with an important budget agreement leading to a major surplus, followed up by a major scandal concerning a sexual relation with an intern, Monica Lewinsky. Finally, he authorized NATO forces to bomb Serbia, marking an end to the Serbia-Kosovo separation crisis.}
}
\cornell{What was the result of the election of 2000?}{
    \begin{itemize}
        \item Republican \textbf{George W. Bush} and Democrat \textbf{Al Gore} both easily won party nominations
        \item Both ran centrist campaigns, with polls showing extremely tight race even up to end
        \item After the eleciton, neither candidate immediately won due to inaccuracy in Florida
        \begin{itemize}
            \item Led to recount, resulting in Bush leading by no more than 300 votes
            \item When court deadline came, recount had not yet been complete; Republican Floridian secretary of state claimed that Bush had won
            \item Gore campaign contested, leading to 5-4 Supreme Court decision in favor of Bush
        \end{itemize}
    \end{itemize}
    \textbf{The election of 2000 was extremely controversial due to the approximately equal popularity of both Gore and Bush. Ultimately, however, after a recount in Florida and a Supreme Court decision, it was decided that Bush won an extremely tight race.}
}
\cornell{What defined Bush's first term in office?}
{
    \begin{itemize}
        \item Principal campaign promise to use budget surplus to finance tax reduction; become narrowly possible
        \item Despite campaign as moderate centrist hoping to bridge gap between parties, governed as staunch conservative
        \begin{itemize}
            \item Refused to support renewal of Clinton's assault weapons ban
            \item Mobilized evangelical Christians as part of coalition
        \end{itemize}
        \item Entirety of presidency ultimately defined by September 11 attacks
    \end{itemize}
    \textbf{Bush's presidency was marked by a major tax reduction and staunchly Republican policies despite centrist campaign, including limited gun control. However, Bush's presidency was, in all, defined by the September 11 attacks.}
}
\cornell{What was the result of the election of 2004?}{
    \textbf{Bush won the election against uncontested John Kerry, once again by a very small amount with the votes approximately equal.}
}
\cornell[The Economic Boom]{What caused the dramatic transformation to the American economy?}{\textbf{The economic boom emerged due to reduced labor costs and the rapid growth of the technology sector; it led the wage gap to further increase, and also coincided with the globalization of the American economy.}}
\cornell{What were the roots of the economic boom in America?}{
    \begin{itemize}
        \item Roots of economic growth of '80s onward lay in troubled years of '70s
        \begin{itemize}
            \item Stagnation encouraged American businesses to adopt new practices, most significantly investment in technology
            \item Sought to reduce labor costs, with many comparisons being drawn to increasingly prosperous nations with low-wage workers
            \begin{itemize}
                \item Implemented by taking harsher actions against unions or moving where union activity was low
                \item Often out-sourced production to China, Mexico
            \end{itemize}
        \end{itemize}
        \item Technology boom created many more jobs, but not as many as originally in the industrial sector
        \item Began to experience great prosperity at unprecedented levels, including booming stock prices, rapidly growing GDP, and significantly lowered rate of inflation 
        \item Lasted for long period of time
    \end{itemize}
    \textbf{The economic boom in the U.S. emerged from the great troubles in the '70s: it was primarily caused by reduced labor costs and the rapid growth of the technology sector.}
}
\cornell{What defines the American two-tiered economy?}{
    \textbf{The two-tiered economy emerged significantly with the economic boom as only those talented enough to profit from areas of growth were able to earn large incomes. For most Americans, income was unchanged or even reduced, with the poverty rate beginning to increase significantly.}
}
\cornell{What were the effects of the globalization of the American economy?}{
    \textbf{In the '50s/'60s, the economy prospered with little external influence; however, by the late '70s, the American economy had become heavily import-oriented, leading to a great trade imbalance with American products facing competition from within U.S.}
}
\cornell[Science and Technology in the New Economy]{How did the booming economy drive the furthering of science and technology?}{\textbf{The booming American economy allowed the digital revolution to unfold in America, tranforming the world by connecting people worldwide through the internet. Furthermore, the economy allowed the government to devote significant money to genetic engineering and, specifically, mapping the human genome.}}
\cornell{What was the Digital Revolution?}{
    \begin{itemize}
        \item Development of microprocessor revolutionized American life, allowing small machines to perform large computations
        \item Microprocessor served as basis for personal computer, first by Apple and later IBM with PC (OS by Microsoft)
        \item Led to numerous major businesses, including computer manufacturers, silicon chip creators
    \end{itemize}
    \textbf{The Digital Revolution started with Intel's creation of the microprocessor, which ultimately gave way to the personal computer, a device which transformed the American lifestyle and economy.}
}
\cornell{What were the origins and initial impact of the Internet?}{
    \begin{itemize}
        \item Internet began in 1963 in U.S. government's Advanced Research Projects Agency, ARPA
        \begin{itemize}
            \item Created for defense-related purposes
        \end{itemize}
        \item By 1971, 23 computers had been linked; rapidly expanded afterward
        \begin{itemize}
            \item Widespread interest led to new technologies included e-mail, personal computers
            \item Users went from less than 1000 in 1984 to 2 billion in 2013
        \end{itemize}
        \item World Wide Web emerged in 1989, allowing for easy sharing of information
    \end{itemize}
    \textbf{The Internet began as a U.S. defense tool to link multiple computers for convenient communication; however, it quickly spread and allowed anyone with a computer to access a much larger network of other users.}
}
\cornell{What were the major early breakthroughs in genetics?}{
    \begin{itemize}
        \item Computer technology proved essential in growth of scientific research
        \begin{itemize}
            \item Predated by discoveries of DNA, double helix structure, genetic codes
            \item Science of genetic engineering emerged 
        \end{itemize}
        \item Scientists could slowly identify genes in humans, other creatures which determined key traits
        \begin{itemize}
            \item Process sped up gradually after government investment in Human Genome Project to continue to map the complete human Genome
            \item DNA attracted public attention for ability to uniquely identify a human
        \end{itemize}
    \end{itemize}
    \textbf{The early breakthroughs in genetics were generally assisted by the emerging computer technology,including the gradual identification of the complete human genome and the specific traits which DNA dictates uniquely.}
}
\end{document} 