\documentclass[a4paper]{article}
    \usepackage[T1]{fontenc}
    \usepackage{tcolorbox}
    \usepackage{amsmath}
    \tcbuselibrary{skins}
    
    \title{
    \vspace{-3em}
    \begin{tcolorbox}
    \Huge\sffamily \begin{center} AP US History  \\
    \LARGE Chapter 11 - Cotton, Slavery, and The Old South \\
    \Large Finn Frankis \end{center} 
    \end{tcolorbox}
    \vspace{-3em}
    }
    \date{}
    \author{}
    
    \usepackage{background}
    \SetBgScale{1}
    \SetBgAngle{0}
    \SetBgColor{red}
    \SetBgContents{\rule[0em]{4pt}{\textheight}}
    \SetBgHshift{-2.3cm}
    \SetBgVshift{0cm}
    \usepackage[margin=2cm]{geometry} 
    
    \makeatletter
    \def\cornell{\@ifnextchar[{\@with}{\@without}}
    \def\@with[#1]#2#3{
    \begin{tcolorbox}[enhanced,colback=gray,colframe=black,fonttitle=\large\bfseries\sffamily,sidebyside=true, nobeforeafter,before=\vfil,after=\vfil,colupper=blue,sidebyside align=top, lefthand width=.3\textwidth,
    opacityframe=0,opacityback=.3,opacitybacktitle=1, opacitytext=1,
    segmentation style={black!55,solid,opacity=0,line width=3pt},
    title=#1
    ]
    \begin{tcolorbox}[colback=red!05,colframe=red!25,sidebyside align=top,
    width=\textwidth,nobeforeafter]#2\end{tcolorbox}%
    \tcblower
    \sffamily
    \begin{tcolorbox}[colback=blue!05,colframe=blue!10,width=\textwidth,nobeforeafter]
    #3
    \end{tcolorbox}
    \end{tcolorbox}
    }
    \def\@without#1#2{
    \begin{tcolorbox}[enhanced,colback=white!15,colframe=white,fonttitle=\bfseries,sidebyside=true, nobeforeafter,before=\vfil,after=\vfil,colupper=blue,sidebyside align=top, lefthand width=.3\textwidth,
    opacityframe=0,opacityback=0,opacitybacktitle=0, opacitytext=1,
    segmentation style={black!55,solid,opacity=0,line width=3pt}
    ]
    
    \begin{tcolorbox}[colback=red!05,colframe=red!25,sidebyside align=top,
    width=\textwidth,nobeforeafter]#1\end{tcolorbox}%
    \tcblower
    \sffamily
    \begin{tcolorbox}[colback=blue!05,colframe=blue!10,width=\textwidth,nobeforeafter]
    #2
    \end{tcolorbox}
    \end{tcolorbox}
    }
    \makeatother

    \parindent=0pt
    
    \begin{document}
    \maketitle
    \SetBgContents{\rule[0em]{4pt}{\textheight}}
    \cornell[Key Concepts]{What are this chapter's key concepts?}{\begin{itemize}
        \item \textbf{4.1.II.B} - A culture formed blending national American and European elements with regional sections
        \item \textbf{4.1.II.C} - Romantic and liberal social beliefs influenced literature, art, philosophy, and architecture
        \item \textbf{4.1.II.D} - Enslaved/free Afr. Americans $\to$ communities to protect dignity, joining pol. efforts to change status
        \item \textbf{4.1.III.A} - Americans created organizations to improve society thru. reform
        \item \textbf{4.1.III.C} - Women's rights movement developed for gender equality, culminating in Seneca Falls Convention
        \item \textbf{4.3.II.B} - North saw $\uparrow$ antislavery while South (despite few owning slaves) saw $\uparrow$ slavery as natural way of life
    \end{itemize}}
    \cornell[The Romantic Impulse]{How did a unique culture develop in American society?}{\textbf{Painting reflected nationalistic ideals through the power of wild natural environments, Northern literature reflecting ideals of independence, liberty, and democracy, Southern literature based either around the wealthy aristocrats or those inhabiting the fringes of society as rural peasants, and the Transcendentalists, focusing on an appreciation of the surrounding world from a personal standpoint, came together to form a unique American literary style.}}
    \cornell{How did American painting reflect nationalism and romanticism?}{\begin{itemize}
        \item Sydney Smith, English wit, expressed that no one outside of America enjoyed American art; but U.S. enjoyed significantly
        \item Most popular aimed to show landscapes: not mere countryside, but instead wildest places (w/ "sublime": awe/fear of nature)
        \begin{itemize}
            \item Frederic Church, Thomas Cole, Thomas Doughty, Asher Durand, all of NY, known as Hudson River School, painted rugged Hudson Valley
            \begin{itemize}
                \item Felt nature greatest source of wisdom; stre
            \end{itemize}
            \item Many began to travel westward to witness spectacular world of Yosemite Valley, Yellowstone, Rockey Mountains
            \begin{itemize}
                \item Thomas Moran, Albert Bierstadt traveled throughout country
            \end{itemize}
        \end{itemize}
    \end{itemize}
    \textbf{American painting, despite not having reached an international audience, appealed greatly to Americans themselves. It generally focused on the idea of "wild nature" and the sublime, initially centered around the Hudson Valley in NY but extending westward.}}
    \cornell{How did American writers generally emphasize ideas of liberty?}{\begin{itemize}
        \item GB's Sir Walter Scott most popular in early nineteenth century; most common American novels were "sentimental novels" of women
        \item James Fenimore Cooper stressed ideals of wilderness, adventure, growing up on frontier NY ("Leatherstocking Tales")
        \begin{itemize}
            \item Represented ideal for true American literature, also depicting central social concerns like fear of disorder, ideal of independence
        \end{itemize}
        \item Walt Whitman, "poet of American democracy," born in 1819 w/ start as newspaper apprentice
        \begin{itemize}
            \item Founded and led NY newspaper \textit{Long Islander}
            \item Printed first volume of work \textit{Leaves of Grass} in 1855, celebrating democracy/individualism; reflected homosexuality in intolerant society
        \end{itemize}
        \item Herman Melville sailed world before rooting back in U.S. and publishing \textit{Moby Dick}, portraying Ahab, captain of whaling vessel, seeking violent whale Moby Dick for fulfillment
        \begin{itemize}
            \item Spirit ultimately -> annihilation
        \end{itemize}
        \item Edgar Allen Poe, one of few southern writers, created sad stories, with books and famous poem "The Raven," seeking to transcend from intellect, explore emotion; had great effect on other poets
    \end{itemize}
    \textbf{Although British writer Walter Scott was popular, American writers like Cooper, describing the independence of the wilderness, Whitman, stressing individualism and democracy through his poems reflecting his troubled state as a homosexual man in an intolerant society, Melville, whose \textit{Moby Dick} revealed the potential destructive nature of the human spirit, and Poe's sad poems exploring true emotion beyond intellect gradually grew in popularity.}}
    \cornell{What were the critical ideals of literature in the South?}{
        Antebellum Southern literature was based around defining the American nation, but often contradicted the true state of society.
        \begin{itemize}
            \item Novelists created romances/eulogies describing upper South plantation system 
            \begin{itemize}
                \item Early (1830s) from Richmond, including Beverly Tucker, William Alexander Caruthers, John Pendleton Kennedy 
                \item Literary capital moved to Charleston in 1840s w/ William Gilmore Simms expressing nationalism initially hoping to transcend regional diffs. but soon defended slavery 
            \end{itemize}
            \item Writers on fringes of plantation society depicted backwoods societies
            \begin{itemize}
                \item Included Augustus B. Longstreet, Joseph G. Baldwin, Johnson J. Hooper
                \item Centered around ordinary, poor people with unique humor; Mark Twain most powerful of group
            \end{itemize}
        \end{itemize}
        \textbf{Southern literature aimed to define the American nation: novelists generally focused on the cavaliers, initially mostly from Richmond but shifting to Charleston. Another group of fringe writers described the impoverished instead of the aristocratic class.}}
        \cornell{Who were the Transcendentalists?}{\begin{itemize}
            \item Transcendentalists focused on individualism by distinguishing "reason" (innate ability of all to understand beauty/truth with full expression of emotions) and "understanding" (intellect applied to narrow confines of society)
            \item Leader was Emerson, lecturer devoted to sharing beliefs, speaking with intellectuals daily 
            \begin{itemize}
                \item Produced some poetry but known for essays/lectures like "Nature" and "Self-Reliance"
                \item Nationalist: believed in cultural independence - lecture "American Scholar" argued that European cultural heritage be ignored and instictive genius be harnessed
            \end{itemize}
            \item Henry David Thoreau significant, too, arguing repression of society $\to$ desperation w/ no one conforming to social pressures
            \begin{itemize}
                \item Went to Walden Pond in Concord Woods and lived in cabin to live deliberately and simply
                \item Resisted slave-allowing govt. by not paying poll tax -> jailed briefly in 1846
                \begin{itemize}
                    \item Argued in "Resistance to Civil Government" that morality > legal codes
                \end{itemize}
            \end{itemize}
        \end{itemize}
        \textbf{The Transcendentalists argued for individualism through an innate personal inderstanding of the world over absorption of mere knowledge and its application to narrow fields. Led by Emerson, a lecturer producing powerful essays like "Nature and Self-Reliance" and a nationalist believing in cultural independence and also supported by Henry David Thoreau, who isolated himself in nature and resisted the government's allowance of slavery, the Transcendentalists produced a powerful repretoire of literature.}}
    \end{document}