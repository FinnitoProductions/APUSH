\documentclass[a4paper]{article}
    \usepackage[T1]{fontenc}
    \usepackage{tcolorbox}
    \usepackage{amsmath}
    \tcbuselibrary{skins}
    
    \title{
    \vspace{-3em}
    \begin{tcolorbox}
    \Huge\sffamily \begin{center} AP US History  \\
    \LARGE Chapter 11 - Cotton, Slavery, and The Old South \\
    \Large Finn Frankis \end{center} 
    \end{tcolorbox}
    \vspace{-3em}
    }
    \date{}
    \author{}
    
    \usepackage{background}
    \SetBgScale{1}
    \SetBgAngle{0}
    \SetBgColor{red}
    \SetBgContents{\rule[0em]{4pt}{\textheight}}
    \SetBgHshift{-2.3cm}
    \SetBgVshift{0cm}
    \usepackage[margin=2cm]{geometry} 
    
    \makeatletter
    \def\cornell{\@ifnextchar[{\@with}{\@without}}
    \def\@with[#1]#2#3{
    \begin{tcolorbox}[enhanced,colback=gray,colframe=black,fonttitle=\large\bfseries\sffamily,sidebyside=true, nobeforeafter,before=\vfil,after=\vfil,colupper=blue,sidebyside align=top, lefthand width=.3\textwidth,
    opacityframe=0,opacityback=.3,opacitybacktitle=1, opacitytext=1,
    segmentation style={black!55,solid,opacity=0,line width=3pt},
    title=#1
    ]
    \begin{tcolorbox}[colback=red!05,colframe=red!25,sidebyside align=top,
    width=\textwidth,nobeforeafter]#2\end{tcolorbox}%
    \tcblower
    \sffamily
    \begin{tcolorbox}[colback=blue!05,colframe=blue!10,width=\textwidth,nobeforeafter]
    #3
    \end{tcolorbox}
    \end{tcolorbox}
    }
    \def\@without#1#2{
    \begin{tcolorbox}[enhanced,colback=white!15,colframe=white,fonttitle=\bfseries,sidebyside=true, nobeforeafter,before=\vfil,after=\vfil,colupper=blue,sidebyside align=top, lefthand width=.3\textwidth,
    opacityframe=0,opacityback=0,opacitybacktitle=0, opacitytext=1,
    segmentation style={black!55,solid,opacity=0,line width=3pt}
    ]
    
    \begin{tcolorbox}[colback=red!05,colframe=red!25,sidebyside align=top,
    width=\textwidth,nobeforeafter]#1\end{tcolorbox}%
    \tcblower
    \sffamily
    \begin{tcolorbox}[colback=blue!05,colframe=blue!10,width=\textwidth,nobeforeafter]
    #2
    \end{tcolorbox}
    \end{tcolorbox}
    }
    \makeatother

    \parindent=0pt
    
    \begin{document}
    \maketitle
    \SetBgContents{\rule[0em]{4pt}{\textheight}}
    \cornell[Key Concepts]{What are this chapter's key concepts?}{\begin{itemize}
        \item \textbf{4.1.II.B} - A culture formed blending national American and European elements with regional sections
        \item \textbf{4.1.II.C} - Romantic and liberal social beliefs influenced literature, art, philosophy, and architecture
        \item \textbf{4.1.II.D} - Enslaved/free Afr. Americans $\to$ communities to protect dignity, joining pol. efforts to change status
        \item \textbf{4.1.III.A} - Americans created organizations to improve society thru. reform
        \item \textbf{4.1.III.C} - Women's rights movement developed for gender equality, culminating in Seneca Falls Convention
        \item \textbf{4.3.II.B} - North saw $\uparrow$ antislavery while South (despite few owning slaves) saw $\uparrow$ slavery as natural way of life
    \end{itemize}}
    \cornell[The Romantic Impulse]{How did a unique culture develop in American society?}{\textbf{Painting reflected nationalistic ideals through the power of wild natural environments, Northern literature reflecting ideals of independence, liberty, and democracy, Southern literature based either around the wealthy aristocrats or those inhabiting the fringes of society as rural peasants, and the Transcendentalists, focusing on an appreciation of the surrounding world from a personal standpoint, came together to form a unique American literary style. Nature became the focal point of a large part of society, a belief which often manifested itself in utopian societies where inhabitants worked in harmony with their natural environment. These societies, and others, often distorted gender roles by going against social norms to give women significantly more, or different forms of, power. Mormonism attracted those who felt out of touch or increasingly distant from material society, and sought to create an idealistic society based around human perfection.}}
    \cornell{How did American painting reflect nationalism and romanticism?}{\begin{itemize}
        \item Sydney Smith, English wit, expressed that no one outside of America enjoyed American art; but U.S. enjoyed significantly
        \item Most popular aimed to show landscapes: not mere countryside, but instead wildest places (w/ "sublime": awe/fear of nature)
        \begin{itemize}
            \item Frederic Church, Thomas Cole, Thomas Doughty, Asher Durand, all of NY, known as Hudson River School, painted rugged Hudson Valley
            \begin{itemize}
                \item Felt nature greatest source of wisdom; stre
            \end{itemize}
            \item Many began to travel westward to witness spectacular world of Yosemite Valley, Yellowstone, Rockey Mountains
            \begin{itemize}
                \item Thomas Moran, Albert Bierstadt traveled throughout country
            \end{itemize}
        \end{itemize}
    \end{itemize}
    \textbf{American painting, despite not having reached an international audience, appealed greatly to Americans themselves. It generally focused on the idea of "wild nature" and the sublime, initially centered around the Hudson Valley in NY but extending westward.}}
    \cornell{How did American writers generally emphasize ideas of liberty?}{\begin{itemize}
        \item GB's Sir Walter Scott most popular in early nineteenth century; most common American novels were "sentimental novels" of women
        \item James Fenimore Cooper stressed ideals of wilderness, adventure, growing up on frontier NY ("Leatherstocking Tales")
        \begin{itemize}
            \item Represented ideal for true American literature, also depicting central social concerns like fear of disorder, ideal of independence
        \end{itemize}
        \item Walt Whitman, "poet of American democracy," born in 1819 w/ start as newspaper apprentice
        \begin{itemize}
            \item Founded and led NY newspaper \textit{Long Islander}
            \item Printed first volume of work \textit{Leaves of Grass} in 1855, celebrating democracy/individualism; reflected homosexuality in intolerant society
        \end{itemize}
        \item Herman Melville sailed world before rooting back in U.S. and publishing \textit{Moby Dick}, portraying Ahab, captain of whaling vessel, seeking violent whale Moby Dick for fulfillment
        \begin{itemize}
            \item Spirit ultimately -> annihilation
        \end{itemize}
        \item Edgar Allen Poe, one of few southern writers, created sad stories, with books and famous poem "The Raven," seeking to transcend from intellect, explore emotion; had great effect on other poets
    \end{itemize}
    \textbf{Although British writer Walter Scott was popular, American writers like Cooper, describing the independence of the wilderness, Whitman, stressing individualism and democracy through his poems reflecting his troubled state as a homosexual man in an intolerant society, Melville, whose \textit{Moby Dick} revealed the potential destructive nature of the human spirit, and Poe's sad poems exploring true emotion beyond intellect gradually grew in popularity.}}
    \cornell{What were the critical ideals of literature in the South?}{
        Antebellum Southern literature was based around defining the American nation, but often contradicted the true state of society.
        \begin{itemize}
            \item Novelists created romances/eulogies describing upper South plantation system 
            \begin{itemize}
                \item Early (1830s) from Richmond, including Beverly Tucker, William Alexander Caruthers, John Pendleton Kennedy 
                \item Literary capital moved to Charleston in 1840s w/ William Gilmore Simms expressing nationalism initially hoping to transcend regional diffs. but soon defended slavery 
            \end{itemize}
            \item Writers on fringes of plantation society depicted backwoods societies
            \begin{itemize}
                \item Included Augustus B. Longstreet, Joseph G. Baldwin, Johnson J. Hooper
                \item Centered around ordinary, poor people with unique humor; Mark Twain most powerful of group
            \end{itemize}
        \end{itemize}
        \textbf{Southern literature aimed to define the American nation: novelists generally focused on the cavaliers, initially mostly from Richmond but shifting to Charleston. Another group of fringe writers described the impoverished instead of the aristocratic class.}}
        \cornell{Who were the Transcendentalists?}{\begin{itemize}
            \item Transcendentalists focused on individualism by distinguishing "reason" (innate ability of all to understand beauty/truth with full expression of emotions) and "understanding" (intellect applied to narrow confines of society)
            \item Leader was Emerson, lecturer devoted to sharing beliefs, speaking with intellectuals daily 
            \begin{itemize}
                \item Produced some poetry but known for essays/lectures like "Nature" and "Self-Reliance"
                \item Nationalist: believed in cultural independence - lecture "American Scholar" argued that European cultural heritage be ignored and instictive genius be harnessed
            \end{itemize}
            \item Henry David Thoreau significant, too, arguing repression of society $\to$ desperation w/ no one conforming to social pressures
            \begin{itemize}
                \item Went to Walden Pond in Concord Woods and lived in cabin to live deliberately and simply
                \item Resisted slave-allowing govt. by not paying poll tax -> jailed briefly in 1846
                \begin{itemize}
                    \item Argued in "Resistance to Civil Government" that morality > legal codes
                \end{itemize}
            \end{itemize}
        \end{itemize}
        \textbf{The Transcendentalists argued for individualism through an innate personal inderstanding of the world over absorption of mere knowledge and its application to narrow fields. Led by Emerson, a lecturer producing powerful essays like "Nature and Self-Reliance" and a nationalist believing in cultural independence and also supported by Henry David Thoreau, who isolated himself in nature and resisted the government's allowance of slavery, the Transcendentalists produced a powerful repretoire of literature.}}
        \cornell{How did the Transcendentalists defend critical concepts of nature?}{\textbf{The Transcendentalists and others felt that nature was not a scientific virtue or an economic stimulus, but instead a place for spirituality and inspiration forming a basic part of humanity. Their work marked the beginning of the environmental movement.}}
        \cornell{How did several utopian societies emerge?}{\begin{itemize}
            \item Brook Farm: Boston Transcend. George Ripley created equal social organization
            \begin{itemize}
                \item One of first thinkers to establish leisure as beneficial, restorative practice
                \item Manual labor -> gap slowly bridged betw. nature and instinct
                \item Realistically, tensions began to rise and large fire finally split up group for good
                \item Writher N. Hawthorne one of original inhabitants, wrote \textit{The Blithesdale Romance} to describe terrrible consequences to most devout
            \end{itemize}
            \item Charles Fourier, French philosopher detailing socialist communities known as "phalanxes," inspired numerous communities
            \item Philanthropist Robert Owen founded New Harmony in Indiana devoted to cooperation; despite economic failure, continued to inspire
        \end{itemize}
        \textbf{Brook Farm, one of the earliest utopian experiments, placed numerous inhabitants together on a farm, required to perform manual labor to appreciate their natural surroundings. Fourier, a French philosopher, inspired multiple communities and Owen, a Scottish philanthropist, created the New Harmony reservation in Indiana. Although nearly all utopian societies failed economically, they continued to inspire countless.}}
        \cornell{How did changing social philosophies transform gender roles?}{\begin{itemize}
            \item Margaret Fuller, Transcendentalist close to Emerson, shunned domestic female stereotype and encouraged intellectual power and social dominance
            \item Oneida Community by John Humphrey Noys rejected traditional family ideals with everyone married to each other
            \begin{itemize}
                \item Sexual activity monitored to prevent rape; children generally raised by multiple parents
                \item Believed in ability to limit male desire by removing family bonds
            \end{itemize}
            \item Shakers, by "Mother" Ann Lee in 1770s, received their peak population in the antebellum period; known for commitment to abstinence
            \begin{itemize}
                \item Never passed onto children; always voluntary choice
                \item Men and women generally segregated but regarded as equal with gender-ambiguous God
                \item Women had majority of power
                \item Primary goal not for equitable gender roles but for society distinct from chaos of normal life
            \end{itemize}
            \item Amana Community by 1843 German immigrants began in Iowa
        \end{itemize}
        \textbf{Fuller, a Transcendentalist; the Oneida Community, replacing the notion of marriage with one of universal union; and the Shakers, requiring complete abstinence all pushed for increased women's rights to varying degrees.}}
        \cornell{Who were the Mormons?}{\begin{itemize}
            \item Mormons (Church of Jesus Christ of Latter Day Saints) aimed to create new ordered society; led by young Joseph Smith with Book of Mormon
            \begin{itemize}
                \item Smith claimed ancient prophet had written words on golden tablets shown to him by God; he had simply translated 
                \item Book argued that group of Israelites had formed a fruitful sociey in America; Jesus resurrected there
                \item After early Americans began to stray from righteous beliefs, punished by God with darkened skin, cleared memory as natives
            \end{itemize}
            \item Smith established sizable following by 1831, but continually persecuted for radical religious beliefs and goal to find isolated community
            \begin{itemize}
                \item Polygamy, social rigidity, and secrecy damaged reputation
                \item Sought human perfection achieved through social organization based around a tight hierarchical structure 
            \end{itemize}
            \item Settled in Nauvoo, IL, but Smith arrested in 1844 and imprisoned and killed by angry mob
            \item Smith's successor, Brigham Young, led society of 12,000 ppl. across desert to Salt Lake City 
            \begin{itemize}
                \item Majority of converts those who were displaced in modern society/felt lacking control in material world
            \end{itemize}
        \end{itemize}
        \textbf{Mormonism began with Joseph Smith, who claimed to have been shown works detailing the arrival of a group of Israelites to America and the formation of a long-standing, fruitful society. Hoping to recreate this, Smith created a rigid social order based around a polygamous society hoping to reach human perfection. Although Smith was imprisoned and killed during his first true attempt at settlement, his successor, Brigham Young, led 12,000 to Salt Lake City.}}
        \cornell[The Crusade Against Slavery]{How did Americans begin to more directly oppose the institution of slavery?}{\textbf{The abolitionist movement, despite starting humbly with ambitions for relocation of slaves to Africa, was transformed by Garrison, demanding universal black citizenship; he was supported by countless free blacks, who not also assisted in selling subscriptions to his newspapers but also united behind powerful leaders like Frederick Douglass. Often-violent anti-abolitionists created strains within the movement, dividing them between the more moderate, gradual advocates with slow political reform and those who sought immediate emancipation (united under Garrison). Many abolitionists became increasingly radical, turning to violence and propaganda.}}
        \cornell{What were the earliest forms of opposition to slavery?}{\begin{itemize}
            \item Early opponents rarely truly inflamed or overt
            \begin{itemize}
                \item Most opposed colonization, w/ ACS (American Colonization Society) seeking to resettle African Americans in Africa/Caribbean w/o angering Southerners
                \item Sought gradual manumission with assistance in forming new slave-based societies 
                \item Some private donors, Congressional support, forming nation of Liberia for exported slaves, soon becoming independent; but ultimately negligible
            \end{itemize}
            \item Antislavery movement had begun to decline by 1830s, with number of "colonized" fewer than number born; many African Americans did not want relocation 
        \end{itemize}
        \textbf{The earliest opponents to slavery generally remained powerless, with the most poweful movement "colonization" led by the ACS, where African Americans were shipped to Africa for resettlement. However, its long-term contributions were near-negligible.}}
        \cornell{How did William Lloyd Garrison fight for abolition?}{\begin{itemize}
            \item Garrison initially assisted NJ Quaker Lundy with antislavery newspaper \textit{Genius of Universal Emancipation}
            \item Tired of Lundy's slow/gradual desire for reform $\to$ formed own newspaper, \textit{Liberator}
            \begin{itemize}
                \item Argued that slavery opponents should sympathize directly with blacks, talking not about slavery's impact on white society but instead on African society
                \item Felt proponents of colonization merely strengthened slavery by ridding country of free blacks 
                \item Believed all African Americans deserved immediate, universal citizenship 
            \end{itemize}
            \item Attracted large numbers $\to$ founded American Anti-Slavery Society in 1832, Philadelphia convention
            \begin{itemize}
                \item Formed chapters throughout nation
            \end{itemize}
        \end{itemize}
        \textbf{Garrison, despite initially advocating for gradualism, or the slow, natural emancipation, realized that slavery was to be abolished immediately with citizenship given directly to blacks. Only by sympathizing with the slaves themselves, he argued, can one become a truly powerful abolitionist. He created the American Anti-Slavery Society in 1832, which spread rapidly throughout the nation.}}
        \cornell{How did northern free blacks partake in the abolitionist movement?}{\begin{itemize}
            \item Free blacks of North generally impoverished, oppressed worse than slaves; suffered from mob violence, unable to obtain education, able to vote in only a few states; often kidnapped and returned to slavery
            \item Northern blacks consistently proud of freedom, hoping to support Southern slaves in any means possible
            \begin{itemize}
                \item Many joined Garrison in 1830s, selling \textit{Liberator} subscriptions w/in communities
            \end{itemize}
            \item Several African Americans rose up within communities to shun slavery
            \begin{itemize}
                \item David Walker created violent pamphlet shunning whites (particularly slaveowners)
                \item Sojourner Truth created religious cult in upstate NY, speaking out against slavery but promoting less violent means
                \item Frederick Douglass most notable, escaping from MD slavery to MA
                \begin{itemize}
                    \item Lectured in England, earning enough wealth to buy freedom from owner, establish antislavery newspaper \textit{North Star}
                    \item Wrote autobiography overtly shunning state of U.S..
                \end{itemize} 
            \end{itemize}
        \end{itemize}
        \textbf{Northern blacks, despite often living in conditions more oppressed than many Southern slaves, valued their liberty and supporter slaves in multiple ways. Some joined white reformers, like Garrison, to spread the message within their communities; others, however, became powerful leaders, most notably Frederick Dougalss, an escaped slave who truly mobilized the abolitionist movement within free blacks.}}
        \cornell{To what degree was anti-abolitionism a powerful social force?}{\begin{itemize}
            \item Most critics feared social turmoil brought on by abolition, with free blacks pouring into north and threatening northern stability; others feared reduction in successful trade with South
            \item Many anti-abolitionists took violent actions
            \begin{itemize}
                \item Prudence Crandall tried to open school to African American girls $\to$ arrested, humiliated with filth
                \item Mob burned down "Temple of Liberty," abolitionist headquarters, in 1834
                \item Garrison himself captured and threatened with hanging; situation only remedied after being jailed by authorities 
                \item Elijah Lovejoy, abolitionist newspaper editor, saw presses smashed three times by angry whites
            \end{itemize}
            \item Continual resistance to abolitionism revealed that movement itself was truly passionate and transformational, resisting frequent threat and standing for beliefs despite great risk
        \end{itemize}
        \textbf{Most anti-abolitionists of the North feared the inevitable social turmoil due to the influx of free blacks into the North as well as reduced Southern productivity. Many took violent actions, humiliating, attacking, and destroying possessions of leading abolitionists. However, the movement remained strong in the face of adversity.}}
        \cornell{How did the abolitionist movement began to show internal strains?}{\begin{itemize}
            \item By the mid-1830s, violence of anti-abolitionists $\to$ many sought more moderate approaches
            \begin{itemize}
                \item Sought peaceful struggle for abolition with gradual convincing of slaveholders that instituion was fundamentally sinful 
                \item Failure to persuade $\to$ turned to govt., joined Garrisonians w/ underground railroad
                \begin{itemize}
                    \item Supreme Court w/ \textit{Prigg v. Pennsylvania} ruling that states needn't enforce law to return fugitive slaves to owners
                    \item "Personal liberty laws" forbade officials from assisting in capture of runaways 
                    \item Abolished in federal regions like D.C. and territories, but most felt Congress lacked Constitutionality
                \end{itemize}
            \end{itemize}
            \item Many others grew in radicalism, notably Garrison
            \begin{itemize}
                \item Shocked allies (like Douglass) by attacking Constitution, churches
                \item Anti-Slavery Society split up in 1840 after Garrison demanded that women be given full equality in eyes of society 
                \item Post-1840, argued for pacificism, opposition to any forms of coercion like prisons and asylums, and eventually a split between the North and the South
                \item Remained influential with unwavering morality
            \end{itemize}
            \item Never unified in political party, but inspired Liberty Party, led by Birney of KY 
            \begin{itemize}
                \item Campaigned for "free soil": no slavery in territories, but no direct abolition
                \item Ultimately attracted support in large numbers 
            \end{itemize}
        \end{itemize}
        \textbf{The abolitionist movement was divided by the mid-1840s between the more moderate ones who sought a peaceful struggle through political reform or "moral suasion," and the radicalists under Garrison, who demanded strict morals to ultimately abolish slavery. They inspired the Liberty Party, which campaigned for free soil.}}
        \cornell{How did abolitionists turn to more drastic measures to further their cause?}{\begin{itemize}
            \item Some abolitionists $\to$ more drastic measures, w/ violence by funding arms purchases, propaganda through distorted images of slavery (like in Angelica Grimke's/Theodore Dwight Weld's \textit{American Slavery as It Is: Testimony of a Thousand Witnesses})
            \item Most powerful work was fictional: Harriet Beecher Stowe's \textit{Uncle Tom's Cabin}
            \begin{itemize}
                \item Appeared first in antisalvery newspaper, but published in 1852, selling rapidly
                \item Based around tradition of sentimental novels for women, embedding message of antislavery in known form of literature
                \begin{itemize}
                    \item Expanded movement to new audience, describing victimized slaves under cruel master 
                \end{itemize}
                \item Stowe seen as northern hero
            \end{itemize}
            \item Abolitionism remained powerful despite divisions
        \end{itemize}
        \textbf{Many abolitionists turned to more drastic measures like violence and propaganda, leading uprisings by funding arms purchases and spreading both propaganda through work claimed to be nonfiction (incorrect depictions) and fiction work, notably Stowe's \textit{Uncle Tom's Cabin}.}}
    \end{document}