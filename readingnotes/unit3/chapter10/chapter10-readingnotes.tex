\documentclass[a4paper]{article}
    \usepackage[T1]{fontenc}
    \usepackage{tcolorbox}
    \usepackage{amsmath}
    \tcbuselibrary{skins}
    
    \title{
    \vspace{-3em}
    \begin{tcolorbox}
    \Huge\sffamily \begin{center} AP US History  \\
    \LARGE Chapter 10 - America's Economic Revolution \\
    \Large Finn Frankis \end{center} 
    \end{tcolorbox}
    \vspace{-3em}
    }
    \date{}
    \author{}
    
    \usepackage{background}
    \SetBgScale{1}
    \SetBgAngle{0}
    \SetBgColor{red}
    \SetBgContents{\rule[0em]{4pt}{\textheight}}
    \SetBgHshift{-2.3cm}
    \SetBgVshift{0cm}
    \usepackage[margin=2cm]{geometry} 
    
    \makeatletter
    \def\cornell{\@ifnextchar[{\@with}{\@without}}
    \def\@with[#1]#2#3{
    \begin{tcolorbox}[enhanced,colback=gray,colframe=black,fonttitle=\large\bfseries\sffamily,sidebyside=true, nobeforeafter,before=\vfil,after=\vfil,colupper=blue,sidebyside align=top, lefthand width=.3\textwidth,
    opacityframe=0,opacityback=.3,opacitybacktitle=1, opacitytext=1,
    segmentation style={black!55,solid,opacity=0,line width=3pt},
    title=#1
    ]
    \begin{tcolorbox}[colback=red!05,colframe=red!25,sidebyside align=top,
    width=\textwidth,nobeforeafter]#2\end{tcolorbox}%
    \tcblower
    \sffamily
    \begin{tcolorbox}[colback=blue!05,colframe=blue!10,width=\textwidth,nobeforeafter]
    #3
    \end{tcolorbox}
    \end{tcolorbox}
    }
    \def\@without#1#2{
    \begin{tcolorbox}[enhanced,colback=white!15,colframe=white,fonttitle=\bfseries,sidebyside=true, nobeforeafter,before=\vfil,after=\vfil,colupper=blue,sidebyside align=top, lefthand width=.3\textwidth,
    opacityframe=0,opacityback=0,opacitybacktitle=0, opacitytext=1,
    segmentation style={black!55,solid,opacity=0,line width=3pt}
    ]
    
    \begin{tcolorbox}[colback=red!05,colframe=red!25,sidebyside align=top,
    width=\textwidth,nobeforeafter]#1\end{tcolorbox}%
    \tcblower
    \sffamily
    \begin{tcolorbox}[colback=blue!05,colframe=blue!10,width=\textwidth,nobeforeafter]
    #2
    \end{tcolorbox}
    \end{tcolorbox}
    }
    \makeatother

    \parindent=0pt
    
    \begin{document}
    \maketitle
    \SetBgContents{\rule[0em]{4pt}{\textheight}}
    \cornell[Key Concepts]{What are this chapter's key concepts?}{\begin{itemize}
        \item \textbf{4.2.I.A} - Entrepreneurs stimulated production and commerce revolutions with organized relations between producers and consumers
        \item \textbf{4.2.I.B} - Textile machinery, steam engines, interchangeable parts, telegraph, agricultural innovations led to greater efficiency
        \item \textbf{4.2.I.C} - Judicial systems -> transportation networks linking North/Midwest but still limited connections between those regions/South
        \item \textbf{4.2.II.A} - Americans began to support themselves w/ production (frequently working in factories rather than w/ agriculture
        \item \textbf{4.2.II.B} - Manufacturing growth -> many more prosperous with larger middle class, but also larger poor class
        \item \textbf{4.2.II.C} - Market revolution changed gender/family roles w/ domestic ideals emphasizing domestic spheres (public v. private)
        \item \textbf{4.2.III.A} - International migrants -> industrial north while many Americans west of Appalachians -> OH/MS rivers
        \item \textbf{5.1.II.A} - International migrants from Europe/Asia (predom. Ireland/Germany) in etrhnic communities preserving languages/customs
        \item \textbf{5.1.II.B} - Anti-Catholic antivist movement developed to curb political power of new immigrants
    \end{itemize}}
    \cornell[The Changing American Population]{How did American demographic changes serve to stimulate economic and social changes in American society?}{\textbf{The American population became increasingly characterized by immigration, particularly Irish Catholics and Germans; furthermore, free states began to shift to an increasingly urban lifestyle. However, accompanying this rise in immigration was the rise of a more sinister movement: nativism: the belief in the superiority of native American people over foreign immigrants, often fueled by racism but also economic competition. Nativismn manifested itself in multiple covert societies.}}
    \cornell{What were the key characteristics of American population growth between 1820 and 1840?}{\begin{itemize}
        \item Population increased rapidly and many moved from countryside to cities, others westward
        \item Native population growth more rapid than in Europe due to public health 
        \begin{itemize}
            \item Fewer epidemics
            \item High birth rate from white women, higher likelihood of surviving -> adulthood
        \end{itemize}
        \item Immigration relatively insignificant through beginning of nineteenth century; returned in 1830s
        \begin{itemize}
            \item Stimulated by reduced transportation costs, economic opportunities paired with deterioration in rest of Europe
            \begin{itemize}
                \item Irish Catholics were particularly large new group of migrants
            \end{itemize}
            \item Most traveled to cities of Northeast
            \begin{itemize}
                \item Complemented urban travel of agricultural New England inhabitants (some to west, but many to cities) 
                \item NYC saw particular growth: largest US city by 1810 due to natural harbor, Erie Canal, liberal commercial state laws 
            \end{itemize}
        \end{itemize}
    \end{itemize}
    \textbf{The American population grew due to growing public health leading to higher birth rates and a greater child survival rate as well as significant immigration after the 1830s due to reduced transportation costs. The Northeast cities grew most significantly both from immigrants and arriving New England farmers, with New York City growing the most dramatically.}}
    \cornell{What were the key characteristics of American population growth between 1840 and 1860?}{\begin{itemize}
        \item Urban growth became even more rapid betw. 1840-1860 
        \begin{itemize}
            \item In east, particularly NYC, but also Philadelphia, Boston; 26\% of free state population lived in towns by 1860 
            \item Western regions saw urban growth, too, like St. Louis, Pittsburgh, Cincinnati, Louisville due to MS river; connected New Orleans with Midwestern farmers and Northeastern merchants
            \item Great Lakes to MS river saw growth of \textbf{Chicago}, Buffalo, Detroit, Milwaukee, Cleveland
        \end{itemize}
        \item Urban growth remained stimulated by partly migration of New England farmers but mostly immigration from abroad
        \begin{itemize}
            \item Over 1.5m Europeans betw. 1840-1850; rising numbers in 1850s w/ over half of NYC pop. immigrants; very few in South 
            \item Some newcomers from England/France/Italy/Scandinavia/Poland/Holland
            \item Most from Ireland (oppressive GB rule, potato famine -> widespread death) and Germany (industrial revolution -> great poverty)
            \begin{itemize}
                \item Irish -> eastern cities as unskilled laborers due to low initial money, mostly young, single women (able to find factory/domestic work)
                \item Germans -> NW as farmers due to higher initial wealth, mostly either \underline{single men or families} to whom farm life was more accessible
            \end{itemize}
        \end{itemize}
        \item National population grew rapidly, too, surpassing Britain by 1860 and nearing France/Germany
    \end{itemize}
    \textbf{The American population became even more urban between 1840 and 1860, with eastern (like NYC/Philadelphia/Boston), western (like Pittsburgh/St. Louis), and Great Lake cities (like Chicago) growing most significantly. Immigration remained significant, mostly stimulated by the German single men and families, who became northwestern farmers, and the Irish young women, who took on unskilled jobs in factories.}}
    \cornell{What factors catalyzed the rise of nativism?}{\begin{itemize}
        \item Many Americans welcomed immigration: cheap labor -> low wages, land speculators saw potential for westward immigration, political leaders hoped to grow population and thus state influence (\textit{ex}: WI allowed immediate voting to immigrants promising citizenship, inhabiting for 1 year)
        \item "Nativism," defense of native-born, grew significantly with hostility, desire to stop immigration
        \begin{itemize}
            \item Often out of racism: comparable to views of natives/African Americans
            \item Others felt immigrants were unfit for unique American society due to many originating from poor cities, assuming it was a choice
            \item Workers feared low-wage foreigners would steal jobs from natives
            \item Protestants feared greater Catholic influence, Whigs feared Democratic influence, many feared immigrants were bribed for votes, many older feared radical ideals
        \end{itemize}
        \item Secret societies formed to combat immigration
        \begin{itemize}
            \item Began in Northeast but soon spread to West/South
            \item Native American Association began in 1837 (w/ 1845 convention in Philadelphia)
            \item Supreme Order of the Star-Spangled Banner in 1850 attempted to ban Catholics/foreigners from public office, require literacy tests to vote; code of secrecy w/ password "I know nothing" -> known as "Know-Nothings"
        \end{itemize}
        \item Know-Nothings in particular turned to partisan politics -> created American Party after election of 1852 w/ very successful vote in 1854, winning MA state government; eventually declined
    \end{itemize}
    \textbf{Although many Americans appreciated the cheap labor costs and population growth accompanied by immigration, many others feared the immigrants due to racism, a belief in civil superiority, low-wages potentially stealing jobs, as well as shifts in religious and partisan demographics. Nativism began to grow, manifested in parties like the Native American Association and the Know-Nothings, who even had a significant political influence.}}
    \cornell[The Changing American Population]{What were the most significant transportation, technological, and communication booms?}{\textbf{Transportation innovations including canals traversed by steamboats and railroads linking distant parts of the North through consolidation began to greatly overshadow highways, with railroads ultimately becoming the most significant form of transportation. The telegraph, too, initially significant for its linking of railroad stations to allow for scheduling, allowed for significant communication and paired with Hoe's steam cylinder, allowed journalism to take off with the Associated Press.}}
    \cornell{What were the early forms of transportation?}{\begin{itemize}
        \item 1790s - 1820s: turnpike era, relying on roads
        \item MS/OH had been significant for years, but most traffic was from flat barges (little more than rafts) with cargo, which were torn up at end of journey (could only travel downstream); upstream vessels far more time-consuming
        \item By 1820s, steamboat industry had expanded significantly
        \begin{itemize}
            \item Corn/wheat of NW farmers and cotton/tobacco of South carried to New Orleans far more rapidly 
            \item Passenger industry grew w/ countries building lavish ships
        \end{itemize}
        \item Highways developed across mountains; though costs for overland transportation lowered, remained high
        \end{itemize}
        \textbf{The predominant forms of travel were by steamboat over rivers, over roads (the turnpike era), and over the Mississippi and Ohio Rivers (these were expensive and circuitous).}}
        \cornell{How did canals become the dominant form of transportation for some time?}{
        \textbf{Western farmers and Eastern merchants were unsatisfied with the circuitous route based on existing rivers and roads due to its high price: they sought to send goods more directly.}
        \begin{itemize}
            \item Canals extremely advantageous economically -> interest to expand to West; canals too expensive for companies -> states took burden
            \begin{itemize}
                \item Began in NY due to existing land route making it easier to build; underestimated forestry between points -> many questioned viability, but governor De Witt Clinton advocated for it
                \begin{itemize}
                    \item Erie Canal largest construction project to date w/ forty foot-wide ditch four feet deep; required difficult cuts and aqueducts to carry canal across streams, heavy masonry w/ wooden gates
                    \item Instant financial success w/ extremely heavy traffic -> repaid entire cost w/in 7 years 
                    \item Route to Great Lakes provided direct access to Chicago from NYC -> NYC began to compete w/ New Orleans
                \end{itemize}
                \item Extended by OH and IN w/ connection betw. Lake Erie, Ohio River creating inland water route to NY 
                \begin{itemize}
                    \item Still required frequent transfer of goods betw. lake, river, and canal vessels
                    \item New canal -> increased white settlement in NW due to ease of travel for migrants
                \end{itemize}
                \item Rival cities unable to catch up to NY: Boston blocked by Berkshire mountains, Philadelphia/Baltimore made effort to cross larger Allegheny Mountains but too expensive for PN and unable to cross mountains for MD
            \end{itemize}
        \end{itemize}
    \textbf{Canals soon overshadowed turnpikes for their great efficiency of transferring goods using steamboats. Almost entirely state-funded, the Erie Canal from New York City to Lake Erie was by far the most successful, establishing NYC as a city formidable to New Orleans. Ohio and Indiana followed with a water route between Lake Erie and the Ohio River, but most large cities were unable to compete with NYC remaining the most dominant.}}
    \cornell{What marked the early development of railroads?}{\begin{itemize}
        \item Railroads were combination of tracks, steam power, railroad cars as public carriers
        \begin{itemize}
            \item Developed by both English and American inventors by 1804
            \item John Stevens formed first in 1820 around NJ estate 
            \item Short Stockton/Darlington Railroad created in England in 1825
        \end{itemize}
        \item American entrepreneurs intrigued by English experiment -> formed companies, first being Baltimore and Ohio creating thirteen-mile stretch in 1830; Mohawk and Hudson in NY in 1831; over 1k total miles by 1836
        \item Railroads remained small and insignificant
        \begin{itemize}
            \item Mostly designed to connect water routes
            \item Tracks often differed in width between connecting lines, preventing travel of one continuous train
            \item Schedules inconsistent, frequent wrecks
        \end{itemize}
        \item Saw advances of iron rails, redesigned passenger cars by 1830s and 1840s
        \item Competition between companies grew
        \begin{itemize}
            \item \textit{ex}: Chesapeake/Ohio Canal Company prevented Baltimore/Ohio Railroad  from travelling through Potomac
            \item NY prohibited from competing w/ Erie Canal
        \end{itemize}
    \end{itemize}
    \textbf{Railroads became truly significant in the 1830s, when mere experiments began to transform into genuinely lucrative industries; although they remained insignificant for some time, with frequent wrecks and inconsistent service, advances by the 1840s led to great competition between companies as well as from states.}}
    \cornell{How did railroads continue to expand?}{\begin{itemize}
        \item Northheast had most efficient system (more than NW and significantly more than South); began to reach west of MS River over iron bridges (St. Louis and Kansas City)
        \item Key trend: \textbf{consolidation} w/ shorter lines combining to form larger lines; 1853 saw connection of four tracks between Northeast and Northwest over Appalachians 
        \begin{itemize}
            \item NY Central / NY and Erie connected NYC w/ Lake Erie 
            \item PN railroad linked Philadelphia/Pittsburgh
            \item OH connected Baltimore w/ OH River at Wheeling
            \item Railroads into interior touched MS river at 8 points, predominantly Chicago as central western rail center
            \item Trunk lines diverted traffic from primary water routes (like Erie Canal / MS River), reducing NW connection to South 
        \end{itemize}
        \item Railroads funded in part by private investors, with railroad companies receiving loans from abroad; assostamce from local government and federal government as land grants
        \begin{itemize}
            \item 1850: Stephen A. Douglas convinced Congress to grant federal lands to Illinois Central; other states soon followed, demanding privileges
        \end{itemize}
    \end{itemize}
    \textbf{The Northeast enjoyed the most efficient rail system, with multiple routes beginning to cross the MS River. Consolidation, or the combination of multiple shorter lines to form larger lines, became critical to expanding the rail system: four critical lines began to reach westward. Funding for railroads came from both private investors (local and abroad) as well as from state/federal governments.}}
    \cornell{How did the telegraph revolutionze communication in the United States?}{\begin{itemize}
        \item \textbf{Magnetic telegraph} created by Samuel F.B. Morse, after sending news of Polk's nomination from Baltimore to DC; low cost made Morse telegraph system seem ideal
        \begin{itemize}
            \item Expanded rapidly: more than 50k miles by 1860; Pacific telegraph connected NYC and SFO
            \item Joined in Western Union Telegraph Company
        \end{itemize}
        \item Telegraphs had wide-reaching effects
        \begin{itemize}
            \item Extended along railroad tracks to connect stations and coordinate train scheduling
            \item Significant for communication between cities
            \item Further aggravated schism betw. South and North, connecting Northeast and Northwest because lines were far more extensive (mostly followed railroad tracks)
        \end{itemize}
    \end{itemize}
    \textbf{The magnetic telegraph expanded rapidly in the late 1850s, with important effects of coordinating railroad scheduling, encouraging communication between distant cities, and ultimately further separating the North and the South.}}
    \cornell{How did journalism change in the United States?}{\begin{itemize}
        \item 1846: \textbf{Richard Hoe} invented steam cylinder rotary press, allowing rapid/cheap printing of newspapers; paired with telegraph to share news far more easily betw. cities, revolutionized communication w/ formation of Associated Press
        \item Northeast created early metropolitan newspapers: NY saw Horace Greeley's \textit{Tribune}, James Gordon Bennett's \textit{Herald}, Henry J. Raymond's \textit{Times}; all detailed international events
        \item Journalism -> sectionalism in 1840s/1850s w/ most major magazines in North -> South felt subjugated due to smaller budgets for newspapers with little impact outside of communities 
    \end{itemize}
    \textbf{Journalism, revolutionized by the pairing of the telegraph and the steam cylinder rotary press, led to the formation of the Associated Press, dedicated to expanding communication throughout the nation. However, because the Northeast had the most significant and far-reaching newspapers, journalism further fueled sectionalism.}}
    \cornell[Commerce and Industry]{What were the primary developments in commerce and industry during the market revolution?}{\textbf{During the market revolution, business began to be increasingly dominated by larger, freer corporations due to reduced legal restrictions. The factory system, too, became particularly significant, particulary in the Northeast, beginning in the textile and shoe industries. Finally, technology advanced greatly to meet the growing industry, particularly in armories sponsored by the federal government, spurred on by interchangeable parts, new energy sources, and brilliant inventors.}}
    \cornell{How did business expand between 1820 and 1840?}{\begin{itemize}
        \item Grew rapidly due to population growth, transportation, new entrepreneur class w/ growing wealth
        \item Retail grew significantly: larger cities saw stores specializing in groceries, hardware, etc.
        \begin{itemize}
            \item Smaller towns still relied on "general stores"; some even performed business through barter
        \end{itemize}
        \item Business organization changed: although most remained founded by individuals/limited partnerships dominated by merchant capitalists, larger business saw \textbf{corporation}
        \begin{itemize}
            \item 1830s saw reduced legal restrictions for corporations (previously required special state charter to grow, replaced w/ general incorporation for small fee)
            \begin{itemize}
                \item Limited liability system meant that stockholders never liable for losses of corporation, only their own stock
            \end{itemize}
            \item Businesses could not survive on investment alone -> relied on credit
            \begin{itemize}
                \item Borrowing often -> instability due to weak credit system (govt. could only issue gold/silver, too little in quantity to meet credit demands)
                \item Banks began to issue bank notes but depended on reputation of bank -> known for insecurity, bank failure
            \end{itemize}
        \end{itemize}
    \end{itemize}
    \textbf{Between 1820 and 1840, businesses grew rapidly, particularly those based around corporations, which had recently been freed of numerous legal shackles, now able to be formed easily with limited stockholder liability. Credit grew in significance for funding larger businesses, but the demand could not be satisfied by the government's stock of gold and silver, leading to great bank instability.}}
    \cornell{How did the factory system develop?}{\begin{itemize}
        \item Pre-War of 1812, most manufacturing done w/in household: hand-operated; improved technology -> growth of factory beginning in New England textile industry w/ larger water-powered machines by 1820s
        \item Factories also pervaded shoe industry (in eastern MA); although still hand-made, based around division of various tasks to form identical, consistent shoes
        \item 1840-1860 saw even larger growth of factory system with rapid expansion in manufactured goods, matching value of total agricultural goods by 1860
        \item Majority and largest of plants in Northeast
    \end{itemize}
    \textbf{The factory system developed and began to replace homespun goods as technology expanded, beginning in the textile industry of New England in the 1820s and expanding to the shoe industry of MA. By 1860, factories had grown enormously (particularly in the Northeast), with manufactured goods finally equal in value to agricultural goods.}}
    \cornell{What were the main technological advances spurring the Industrial Revolution?}{\begin{itemize}
        \item Industry remained very rudimentary compared to modern times (\textit{ex}: cotton still produced coarsely), but machinery grew in U.S. more rapidly than anywhere else
        \begin{itemize}
            \item Reached point where GB and other European nations would travel to U.S. to study machines
        \end{itemize}
        \item Machine tools manufacturing (supported by govt. to support military) particularly large
        \begin{itemize}
            \item \textit{ex}: armory in MA formed turret lathe (cutting screws/metal) and milling machine (replaced hand chiseling), precision grinding machine for construction of other machines (like sewing machines) but more significantly for rifles
            \item Armories became known as hotbed for technological advancement, attracting craftsmen/factory owners to learn and share new ideas
        \end{itemize}
        \item Interchangeable parts of Eli Whitney/Simeon North expanded beyond gun industry, including to watch/clock making, trains, farm tools
        \item New energy sources benefitted industrialization greatly, w/ coal (mostly from around Pittsburgh) replacing wood/water -> factories could be placed further from streams
        \begin{itemize}
            \item Most factories remained dependent on water power, with largest factories close to natural waterfalls
            \item Dependence -> often closed during winter w/ frozen rivers 
        \end{itemize}
        \item American industry spurred by American inventors including Charles Goodyear (rubber industry), Elias Howe (sewing machine in conjunction w/ Isaac Singer)
    \end{itemize}
    \textbf{American industry grew at an unparalleled rate in the mid-19th century, rapidly surpassing even Britain in tool complexity, with military machine tools manufacturing plants (notably armories) representing the most significant hotbeds. Interchangeable parts and new energy sources were exploited by American inventors to develop advanced machinery and systematic methods of constructing parts.}}
    \cornell[Men and Women at Work]{How did the market revolution impact social conditions of men and women in the 19th century?}{\textbf{When recruitment for factories first began, it drew from the native population, with mid-Atlantic states taking in families and children and the Northeast taking in young women alone, who experienced comparatively comfortable conditions which slowly deteriorated. Immigrants then took their place, further diminishing wages and weakening working conditions due to the lack of a social pressure. A clash between skilled artisans and unskilled laborers quickly formed as cheaper goods threatened the livelihoods of these craftsmen, who were mostly displaced. Finally, workers began to push for freedom through political reform (reduced workday, child labor, and legal unions); although most workers felt they were free (in contrast to slavery in the South), the collective group of workers as a coalition paled in comparison to the British equivalent due to ethnic conflicts and a wide supply of immigrants to replace dissident workers.}}
    \cornell{What were the key early methods of native recruitment?}{\begin{itemize}
        \item Challenging: 90\% of Americans lived/worked on farms, urban residents were skilled artisans unwilling to take on unskilled jobs
        \item Industrial labor supply change caused by change in American agriculture
        \begin{itemize}
            \item Fertile NW farmlands, new machines, greater imports (particularly in New England) -> less labor required for cultivation -> more workers migrated to cities for factory work
        \end{itemize}
        \item Recruitment performed through multiple distinct systems
        \begin{itemize}
            \item Mid-Atlantic states often focused on transporting entire families, w/ parents working alongside children (some no more than 4 yrs. old)
            \item MA known for recruiting young women (mostly farmers' daughters) in late teens: Lowell and Waltham System; allowed women to save up wages and eventually marry
            \begin{itemize}
                \item Lowell workers \textit{far more comfortable} than women in England: received clean, maintained houses; well-fed (some New Englanders felt employing women immoral -> optimal conditions)
                \item Women received high wages but were expected to maintain moral behavior (church, curfew); had time to write \textit{Lowell Offering} magazine
                \item Transition from farm to factory often challenging: surrounded by strangers, forced to work \underline{far more repetitive} tasks in contrast to varied farm chore
                \item Few other options: most of society expected women not to expand to factory jobs or travel country looking for work 
            \end{itemize}
            \item Labor conditions particularly favorable (compared to England); although some young children, misery far less great due to parent supervision
        \end{itemize}
    \end{itemize}
        \textbf{As changing agricultural conditions created a surplus of workers migrating to cities, regions adopted varying recruitment policies. In the Mid-Atlantic states, whole family were generally recruited and children encouraged to work alongside them; contrastingly, particularly in MA (notably Lowell, MA), young women were hired, but they initially received very comfortable treatment but occasionally struggled with the transition from farm to factory.}}
    \cornell{How did Lowell's factory system take a turn for the worse?}{
        \textbf{Lowell's factory system was unable to survive at such degrees of prosperity for long: the textile boom (followed by bust) -> $\downarrow$ wages/working conditions, $\uparrow$ working hours, and overcrowding.}
        \begin{itemize}
            \item 1834: Lowell saw formation of union Factory Girls Association, holding strike against 25\% wage cut, later against rent increase; both failed and destroyed by 1837 recession
            \item 1842: Sarah Bagley formed Female Labor Reform Association, demanding ten-hour day rather than twelve-hour, conditions improvement 
            \begin{itemize}
                \item Appealed to state govt.
                \item Women had already begun -> teaching/domestic service
            \end{itemize}
        \end{itemize}
    \textbf{Increasingly poor conditions in the Lowell factory led the women to take action in multiple unions, appealing both to the factory and state government.}}
    \cornell{What characterized the growing immigrant workforce?}{\begin{itemize}
        \item Increasing \# of immigrant workers post-1840 greatly benefitted manufacturers: cheap/large source of labor unable to easily protest -> poorer working conditions
        \item Irish immigrants formed construction gangs to perform unskilled work on turnpikes/canals/railroads 
        \begin{itemize}
            \item Racism -> unable to prove skills -> very low wages often unable to support families (lived in terruble conditions)
        \end{itemize}
        \item Working conditions diminished in New England due to reduced pressure to provide well-kept environment; pushed for \textbf{piece wages} (based on amount produced)
        \begin{itemize}
            \item Lowell had become near-slum (still not at levels of Europe), known for danger/noise/unsanitariness, long workday over 14 hours, reduced wages even for skilled workers and even lesser for women/children
        \end{itemize}
    \end{itemize}
    \textbf{The immigrant workforce, particularly the Irish, led to a deterioration in working conditions, wages, and lifestyles in industrial cities: reduced social pressure due to racism meant that wages were reduced to optimize productivity.}}
    \cornell{How did the growth of factories clash with existing artisan workforces?}{\begin{itemize}
        \item Artisan based on older U.S. vision of yeoman, independent farmers; stuck to ideals of independence, equality
        \item Most artisans unable to compete w/ cheap goods of factories -> formed labor unions to protect identities (including printers, shoemakers, carpenters, masons, shipbuilders)
        \begin{itemize}
            \item Cities saw craft societies known as \textbf{trade unions} intended to join worker forces
            \item Extremely poor results: law favored independent workers; feared that coalition represented comspiracy; weakened by Panic of 1837
        \end{itemize}
    \end{itemize}
    \textbf{Artisans, aligning with an antiquated ideal of yeoman farmers, mainly refused to forsake such ideas. However, most were unable to compete with factory cheap goods, forming labor unions to protect themselves and trade unions to join forces; however, these performed very poorly due to legal restrictions.}}
    \cornell{How did industrial workers fight for greater control?}{\begin{itemize}
        \item Workers tried to convince state legislatures to set maximum workday, but only NH and PN passed 10-hour workdays (unless a contract was formed, which most employers ended up doing as hiring conditions) 
        \item Child labor restrictions were set by NH, PA, MA, but contracts were formed to counterract (once again!)
        \item Greatest legal victory: \textit{Commonwealth v. Hunt} -> unions/strikes were legal
        \begin{itemize}
            \item Unions remained ineffective as many workers feared permanent laboring force; unions rarely strong enough to stage impactful strikes
        \end{itemize}
        \item Skilled workers had greater success: unions more akin to guilds of preindustrial times, designed to protect positions of members by admitting small \#s to skilled trades
        \begin{itemize}
            \item Revived in 1850s w/ National Typographical Union in 1852, Stone Cutters in 1853,  Hat Finishers in 1854, Molders and Machinists in 1859
        \end{itemize}
        \item Early craft unions excluded women -> established protective unions, but still lacked power; represented support force for women
        \item Unions remained relatively weak compared to England (where workers banded together to transform political structure) due to immigrant workers w/ low wages unable to show anger without risk of firing, ethnic divisons -> internal fights, strong industrial capitalists
    \end{itemize}
    \textbf{Industrial workers pushed for control in state legislatures with failed attempts at a maximum workday and child labor restrictions; although unions and strikes were made legal and skilled workers were able to form relatively successful unions similar to preindustrial guilds, most unskilled workers struggled to form a powerful force due to internal divisions, a high labor supply leading to instant firing at any sign of discontent, and the power of the industrial capitalists.}}
    \cornell{How did workers fight for personal freedom?}{\begin{itemize}
        \item Most workers proud of hard work, believed themselves to be "sovereign individuals"
        \item Ideas of freedom far less organized than modern day: few men/no women could vote, workers were often bound to employers, and millions of slaves completely lacked freedom
        \item Some argued that a truly free individual should escape from capitalist society (\textit{ex}: Thoreau lived in isolation on Walden Pond), appreciate nature
        \item Liberty remained strong particularly in the North w/ lack of slavery seen as true freedom
        \begin{itemize}
            \item Able to change professions with ease, move throughout country
            \item Material conditions at times worse than Southern slaves
            \item Blacks in North still not considered true citizens; although they had arrived as skilled workers, often received worse conditions due to rivalry w/ white craftsmen -> domestic servants
        \end{itemize}
    \end{itemize}
    \textbf{Freedom remained a challenging subject in the United States due to the inability of the majority of the popiulation to vote. Although some believed that capitalist society was antithetical to freedom, most Northern workers felt that their lack of slavery and ability to move at will represented true freedom. However, blacks in the North often experienced worse conditions than slavery in the South: they were not considered citizens and often worked less advanced professions.}}
    \cornell[Patterns of Industrial Society]{What were the main social changes for industrial workers?}{\textbf{With wealth stratification further aggravated, many would expect significant conflict between the higher and lower classes; however, social mobility ultimately curbed that from being a significant factor. Middle classes saw greatly improved conditions with new diets and the freedom to own domestic servants and leave women at home to focus on domestic tasks (contrasting from economic ones within home of previous time), while working-classes continued to struggle. Leisure within cities, though limited due to strict schedules, began to take a turn for the bizarre, with great interest in circuses and freakish spectacles.}}
    \cornell{How was wealth stratification further aggravated?}{\begin{itemize}
        \item Commercial growth -> $\uparrow$ average income with more uneven distribution w/ large groups (like slaves, natives, landless farmers) without money and many others with very limited amounts
        \begin{itemize}
            \item Even during Revolution, 45\% of wealth in 10\% of citizens; 1845: 65\% of wealth in 4\% 
            \item Philadelphia known for particular disparity  
        \end{itemize}
        \item Merchants/industrialists began to gain large amounts of money -> cities known for wealth (contrasting w/ original plantation)
        \begin{itemize}
            \item Cities began to see large mansions w/ expensive goods, clubs/social rituals (\textit{ex}: NYC attempted to model Paris w/ Central Park in Manhattan developed by Olmsted/Vaux)
        \end{itemize}
        \item Growing population of urban poor
        \begin{itemize}
            \item Poor began to not only struggle to sustain but lack resources, homes, relying on charity/crime 
            \item Known as "paupers": often immigrants unable to adapt to life or struggling with nativism, widows/orphans, alcoholics/mental illness, but free blacks among the worst
            \begin{itemize}
                \item Major urban areas had significant black populations mostly descendants of escaped/freed slaves
                \item Most could access only simplest jobs w/ low pay; unable to vote or use public schools
            \end{itemize}
        \end{itemize}
    \end{itemize}
    \textbf{Greater commercial growth concentrated a greater amount of wealth in the hands of a smaller group despite higher average income. With a growing merchant and industrialist class, cities were particularly indicative of such changes: large mansions emerged but a larger urban poor developed particularly obvious in black communities.}
    \cornell{How did social mobility become a transformative factor in American society?}{\begin{itemize}
        \item Rarely significant class conflict because \underline{absolute position} of most laborers still increased: better than in Europe/on farms
        \item Working class allowed for some mobility w/ some workers moving from poverty to modest riches w/ hard work and some luck, but most moved up at least one notch (like unskilled -> skilled)
        \item Geographic mobility w/ migrations especially common due to West uncultivated lands 
        \begin{itemize}
            \item Many workers saved up to move to West, began farming; represented "safety valve"
            \item Few workers could afford such a move: most simply moved between towns due to layoffs; rarely signif. improvement but migration, lack of permanence -> workers struggled w/ organization
        \end{itemize}
        \item Many received expanded political opportunities, like opportunity to vote
    \end{itemize}
    \textbf{Discontent among lower classes remained low because most laborers still progressed from Europe and rural life, and some mobility existed within the working class, with many able to move from unskilled to skilled. Geographic mobility, with some migrating to the West but most simply changing between towns, created a rootlessness which hurt worker unity. Finally, few were discontent due to expanded political opportunities.}}
    \cornell{How did middle-class life change?}{\begin{itemize}
        \item Middle-class changed the most: econ. development allowed more to own business/trade/work in administration w/ land no longer indicative of true wealth 
        \item Women remained in the home; often hired servants for assistance -> many women could take breaks 
        \item Innovations like iron stove replaced fireplace; although someone dirty and difficult to use, seen as great luxury with more freedom in food preparation
        \item Diets changed significantly due to expansion of agriculture and ease of transport 
            \begin{itemize}
                \item Food more available due to rail transport: although fruits/veg. still difficult, meats/grains/dairy remained grew in popularity
                \item Some even purchased iceboxes (but most families used salt/sugar)
                \item Diet far more startchy -> middle-classes heavier than today
            \end{itemize}
        \item Purchase more elaborate goods, carpeting, wallpaper, curtains; simple styles eliminated in favor of elaborate designs of Victorian era 
        \begin{itemize}
            \item Homes grew in size (while artisans generally rented): children less frequently shared beds, family lived in multiple rooms w/ separate dining rooms from kitchen
            \item Many enjoyed indoor plumbing/toilets by 1850s
        \end{itemize}
    \end{itemize}
    \textbf{The middle class changed the most significantly during America's Industrial Revolution: women remained increasingly in the home but with servants to assist them, new innovations emerged like the iron stove, and diets changed signifciantly with food more accessible and some owning iceboxes. Many even began to enjoy larger houses and plumbing.}}}
    \cornell{How did family structure change during the Industrial Revolution?}{\begin{itemize}
        \item Movement of families from farm -> urban saw jobs > land and weakened patriarchal structure 
        \item Income-earning work shifted out of home into shop/mill/factory
        \begin{itemize}
            \item Family had once been place of great economic activity w/ men/women/children working together and sharing income 
            \item Farming -> Northwest w/ new machines for profitability -> farm owners relied less on families and instead on male workers -> farm women increasingly domestic w/o heavy labor
            \item With household no longer principal center of productions and income-earners departing to work, distinction betw. public world and workplace becoming sharper
            \begin{itemize}
                \item Family life dominated not by production but by domestic tasks 
            \end{itemize}
        \end{itemize}
        \item Birth rate changed significantly w/ avg. American woman in 1800 producing 7 children but in 1860 only 5
        \begin{itemize}
            \item Fell particularly for middle classes due to some birth control devices, \textbf{abortions} remaining legal in most states (until post-Civil War), and abstinence
        \end{itemize}
    \end{itemize}
    \textbf{Families changed significantly both from the change in social structure from farms to cities, but more significant was the shift of income-earning work out of the home, meaning most income-earning work was performed in shops, mills, and factories while domestic tasks were performed at home. Additionally, a slowing birth rate transformed family life.}}
    \cornell{What was the state of women's rights pre-Industrial revolution?}{\begin{itemize}
            \item Political rights lacking w/ little access to business
            \item Father ruled within family and implemented demands, woman unable to divorce or receive custody in male-initiated divorce; wife beating and marriage rape generally legal
            \item Lacked direct access to education, only encouraged to attend at elementary level but discouraged from higher education; very few colleges accepted women 
            \begin{itemize}
                \item \textit{ex}: Oberlin in OH first to accept female students w/ 4 enrolling in 1837; believed in importance of coeducational studies
                \item \textit{ex}: Mount Holyoke in MA founded by Mary Lyon
            \end{itemize}
        \end{itemize}
        \textbf{Most women lacked political rights, a dominant role within their families, and access to higher education (Oberlin and Mount Holyoke were among the few to accept women).}
    }
    \cornell{How did women experience significant changes in their roles?}{\begin{itemize}
        \item Distinction betw. public/private, workplace/home -> women and men took on more disparate roles
        \begin{itemize}
            \item Women no longer seen as income producers -> emphasis on domestic work w/ more nurturing role
            \item More important as consumers w/ value placed on elegant purchases/cleanliness of homes
        \end{itemize}
        \item Women occupied unique sphere w/ own culture
        \begin{itemize}
            \item Formed social networks which grew into female clubs notable for being breeding grounds of reforms
            \item Women's magazines, notably Sarah Hale's \textit{Godey's Lady's Book} focusing on fashion/shopping/homemaking and \underline{no politics}
            \item Most men viewed new sphere (despite significant oppression from a modern perspective) as empowering: allowed women to express their best qualities as benevolent custodians untainted by the workplace
            \begin{itemize}
                \item Justified rough/crude behavior from husbands counterracted by morally powerful influence of women
                \item Led women to become increasingly detached
            \end{itemize}
        \end{itemize}
        \item Social detachment had great ramifications for unmarried women
        \begin{itemize}
            \item Few employers would hire unmarried women due to new ideals of women's roles
            \item Some became teachers/nurses (seen as female qualities); but until after Civil War, most relied on relatives or became governesses (companions to widows/other women)
        \end{itemize}
        \item Middle classes had luxury to keep women at home; working-class women had little option
        \begin{itemize}
            \item Worked in poor conditions in factories, some employment in middle-class homes w/ domestic service
        \end{itemize}
    \end{itemize}
    \textbf{Middle-class women began to occupy a unique sphere, forming a unique culture with social networks, literature stressing "female" qualities; however, many men believed women were becoming more morally sound leading them to take on cruder roles in the home. Unmarried women struggled particularly from female social detachment: few would higher them, meaning most had to rely on relatives. Working classes however, had no choice but to send women to work: most worked in factories (poorer conditions than men) or in middle-class homes.}}
    \cornell{What were notable leisure activities?}{\begin{itemize}
        \item Leisure activities rare for all but wealthiest: most worked long days w/ only break on Sundays 
        \begin{itemize}
            \item Sundays generally devoted to religious activity -> many families frowned upon games/relaxation
            \item Holidays became times of great patriotism
        \end{itemize}
        \item Contrasting with rural inconsistent farming pattern -> frequent leisure, urban workers had tight schedules
        \begin{itemize}
            \item Men met in taverns after work to drink, play games, while women met in homes to talk, work on sewing, play card games
            \item Educated began to enjoy reading newspapers, magazines, books; women particularly known for reading w/ new genre of fiction ("sentimental novel")
            \item Public leisure developed
            \begin{itemize}
                \item Many theatres often attracted mu ltiple classes w/ Shakespeare, melodrama, minstrel shows (racist against blacks)
                \item Public sporting w/ baseball growing in popularity, cockfighting becoming controversial 
                \item Travelling circuses to entertain
            \end{itemize}
        \end{itemize}
        \item Popular taste notably bizarre: to escape familiar world, sought strange phenomena
        \begin{itemize}
            \item \textbf{P.T. Barnum} opened NYC's American Museum in 1842 to showcase freakish, w/ magicians, Siamese twins, ventriloquist
            \begin{itemize}
                \item Produced engaging lectures describing scientific advances, exotic travels, historical narrative
                \item Particularly attractive to women seeking guidance
            \end{itemize}
            \item Barnum launched circus in 1870s
        \end{itemize}
    \end{itemize}
    \textbf{Although few could enjoy frequent leisure activities in cities, with Sundays typically devoted to religion and holidays times for patriotism, men met in taverns and women in homes, reading became popular particularly for women, and public events like theatre, sporting, and circuses were growingly popular. A taste for the strange developed among the middle class, seeking to escape their familiar world; P.T. Barnum established the popular yet freakish American Museum.}}
    \cornell[The Agricultural North]{What aspects of the North remained agricultural?}{\textbf{The Northeast had a significant agricultural sector focusing on dairy, wheat, corn, grapes, and animals; however, a large portion of it shifted westward. In the Northwest, industry grew slightly but was mainly focused on agriculture, where productivity was essential due to the high eastern demand. The rural lifestyle of the Northwest was significantly vibrant and unique in the eastern parts, focusing on religion and communal gatherings; however, the west remaiend isolated.}}
    \cornell{How did agriculture remain powerful on the Northeast?}{\begin{itemize}
        \item Production -> shifted westward due to richer soil of NW; centers -> westward for significant farm goods (like wheat, corn, grapes, cattle, sheep, hogs)
        \item Many eastern farmers moved west, establishing new farms or becoming laborers themselves, focusing on supplying goods to eastern cities, selling vegetables in local towns
        \item Cities -> dairy farming grew significantly w/ half of products in east and rest from West (particularly OH); Northeast led hay and potato production 
        \item Agriculture remained important but became less important in industry of Northeast -> rural population $\downarrow$
    \end{itemize}
    \textbf{Agriculture in the Northeast saw production shift westward, causing significant rural migrations to the West, creating new farms or becoming laborers. The growth of cities spurred significant dairy farming, hay, and potatoes. Although agriculture remained significant, it was less important in the industry of the Northeast.}}
    \cornell{How did the Northwest remain agricultural?}{\begin{itemize}
        \item Northwest enjoyed some industry with steady growth along Lake Erie (Cleveland and Cincinnati); Chicago significant further west, producing machinery and packing meats
        \item Most Western industry centered around agriculture w/ farm machinery, agricultural products -> less important 
        \item Many parts of NW dominated by natives -> hunting/fishing w/ some agriculture remained principal for whites and natives 
        \item Further south, primarily agricultural due to rich land -> most citizens worked on family farms
        \item $\uparrow$ farm prices (due to $\downarrow$ Euro. agri. after Napoleonic wars -> high demand) -> western farmers partook in commercial agri., sending goods over MS River 
        \item Industrialization provided greatest boost w/ local factories/Northeast cities creating great demand for farm products
        \begin{itemize}
            \item Met demand with strenuous work, constant aim to increase capacity
            \begin{itemize}
                \item Moved into prairie regions and cleared forest lands
            \end{itemize}
            \item Wheat most significant; corn, potatoes, oats important 
        \end{itemize}
        \item New tech to reduce labor, including new seeds (Mediterranean wheat, new breeds of animal, machines, tools (cast-iron plow replaced w/ steel plow))
        \begin{itemize}
            \item Automatic reaper of \textbf{Cyrus H. McCormick} allowed 5x efficiency; McCormick established Chicago factory, allowing them to spread throughout region
            \item Thresher to separate grain from wheat stocks improved efficiency greatly by removing necessary handiwork
        \end{itemize}
        \item Northwest considered itself most democratic: based on economic freedom relying on rights of propertied
        \begin{itemize}
            \item Lincoln voiced political words by expressing importance of acquiring property
        \end{itemize}
    \end{itemize}
    \textbf{The Northwest enjoyed some industrial growth but all of it remained centered on agricultural development; high demand for goods leading to high farm prices meant that many farmers sought to expand productivity with new machinery, including the reaper, thresher, and steel plow. Politically, the Northwest was relatively democratic concerning the propertied.}}
    \cornell{What was the quality of life for rural inhabitants in the Northwest?}{\begin{itemize}
        \item Vibrant communities w/ churches, schools, stores, taverns in densely populated eastern portions of Northwest, but great isolation further west
        \item Religion very significant to bring together those of common heritage; met in churches, homes to read Bible, conduct prayer meetings
        \item Rural people often shared tasks like barn raisings (large festive suppers prepared by mothers while men worked); women often joined together to share tasks ("bees" where they worked on quilts, baking, etc.)
        \item World far less in contact w/ popular culture: rural people relished in occasional link to outside world, dreaming about distant places in cities which they had never seen
        \begin{itemize}
            \item Many still valued separation, autonomy; felt urban life represented lack of control
        \end{itemize}
    \end{itemize}
    \textbf{Rural Northwestern inhabitants closer to the East enjoyed far more vibrant communities brought together by religion, communal activities like barn raisings, weddings, baptisms, and funerals. Although their world remained distant from popular culture of cities, many relished the occasional connection to the outside world, but others feared the lack of autonomy brought by urban life.}}
    \end{document}