\documentclass[a4paper]{article}
    \usepackage[T1]{fontenc}
    \usepackage{tcolorbox}
    \usepackage{amsmath}
    \tcbuselibrary{skins}
    
    \title{
    \vspace{-3em}
    \begin{tcolorbox}
    \Huge\sffamily \begin{center} AP US History  \\
    \LARGE Chapter 10 - America's Economic Revolution \\
    \Large Finn Frankis \end{center} 
    \end{tcolorbox}
    \vspace{-3em}
    }
    \date{}
    \author{}
    
    \usepackage{background}
    \SetBgScale{1}
    \SetBgAngle{0}
    \SetBgColor{red}
    \SetBgContents{\rule[0em]{4pt}{\textheight}}
    \SetBgHshift{-2.3cm}
    \SetBgVshift{0cm}
    \usepackage[margin=2cm]{geometry} 
    
    \makeatletter
    \def\cornell{\@ifnextchar[{\@with}{\@without}}
    \def\@with[#1]#2#3{
    \begin{tcolorbox}[enhanced,colback=gray,colframe=black,fonttitle=\large\bfseries\sffamily,sidebyside=true, nobeforeafter,before=\vfil,after=\vfil,colupper=blue,sidebyside align=top, lefthand width=.3\textwidth,
    opacityframe=0,opacityback=.3,opacitybacktitle=1, opacitytext=1,
    segmentation style={black!55,solid,opacity=0,line width=3pt},
    title=#1
    ]
    \begin{tcolorbox}[colback=red!05,colframe=red!25,sidebyside align=top,
    width=\textwidth,nobeforeafter]#2\end{tcolorbox}%
    \tcblower
    \sffamily
    \begin{tcolorbox}[colback=blue!05,colframe=blue!10,width=\textwidth,nobeforeafter]
    #3
    \end{tcolorbox}
    \end{tcolorbox}
    }
    \def\@without#1#2{
    \begin{tcolorbox}[enhanced,colback=white!15,colframe=white,fonttitle=\bfseries,sidebyside=true, nobeforeafter,before=\vfil,after=\vfil,colupper=blue,sidebyside align=top, lefthand width=.3\textwidth,
    opacityframe=0,opacityback=0,opacitybacktitle=0, opacitytext=1,
    segmentation style={black!55,solid,opacity=0,line width=3pt}
    ]
    
    \begin{tcolorbox}[colback=red!05,colframe=red!25,sidebyside align=top,
    width=\textwidth,nobeforeafter]#1\end{tcolorbox}%
    \tcblower
    \sffamily
    \begin{tcolorbox}[colback=blue!05,colframe=blue!10,width=\textwidth,nobeforeafter]
    #2
    \end{tcolorbox}
    \end{tcolorbox}
    }
    \makeatother

    \parindent=0pt
    
    \begin{document}
    \maketitle
    \SetBgContents{\rule[0em]{4pt}{\textheight}}
    \cornell[Key Concepts]{What are this chapter's key concepts?}{\begin{itemize}
        \item \textbf{4.2.I.A} - Entrepreneuers stimulated production and commerce revolutions with organized relations between producers and consumers
        \item \textbf{4.2.I.B} - Textile machinery, steam engines, interchangeable parts, telegraph, agricultural innovations led to greater efficiency
        \item \textbf{4.2.I.C} - Judicial systems -> transportation networks linking North/Midwest but still limited connections between those regions/South
        \item \textbf{4.2.II.A} - Americans began to support themselves w/ production (frequently working in factories rather than w/ agriculture
        \item \textbf{4.2.II.B} - Manufacturing growth -> many more prosperous with larger middle class, but also larger poor class
        \item \textbf{4.2.II.C} - Market revolution changed gender/family roles w/ domestic ideals emphasizing domestic spheres (public v. private)
        \item \textbf{4.2.III.A} - International migrants -> industrial north while many Americans west of Appalachians -> OH/MS rivers
        \item \textbf{5.1.II.A} - International migrants from Europe/Asia (predom. Ireland/Germany) in etrhnic communities preserving languages/customs
        \item \textbf{5.1.II.B} - Anti-Catholic antivist movement developed to curb political power of new immigrants
    \end{itemize}}
    \cornell[The Changing American Population]{How did American demographic changes serve to stimulate economic and social changes in American society?}{\textbf{The American population became increasingly characterized by immigration, particularly Irish Catholics and Germans; furthermore, free states began to shift to an increasingly urban lifestyle. However, accompanying this rise in immigration was the rise of a more sinister movement: nativism: the belief in the superiority of native American people over foreign immigrants, often fueled by racism but also economic competition. Nativismn manifested itself in multiple covert societies.}}
    \cornell{What were the key characteristics of American population growth between 1820 and 1840?}{\begin{itemize}
        \item Population increased rapidly and many moved from countryside to cities, others westward
        \item Native population growth more rapid than in Europe due to public health 
        \begin{itemize}
            \item Fewer epidemics
            \item High birth rate from white women, higher likelihood of surviving -> adulthood
        \end{itemize}
        \item Immigration relatively insignificant through beginning of nineteenth century; returned in 1830s
        \begin{itemize}
            \item Stimulated by reduced transportation costs, economic opportunities paired with deterioration in rest of Europe
            \begin{itemize}
                \item Irish Catholics were particularly large new group of migrants
            \end{itemize}
            \item Most traveled to cities of Northeast
            \begin{itemize}
                \item Complemented urban travel of agricultural New England inhabitants (some to west, but many to cities) 
                \item NYC saw particular growth: largest US city by 1810 due to natural harbor, Erie Canal, liberal commercial state laws 
            \end{itemize}
        \end{itemize}
    \end{itemize}
    \textbf{The American population grew due to growing public health leading to higher birth rates and a greater child survival rate as well as significant immigration after the 1830s due to reduced transportation costs. The Northeast cities grew most significantly both from immigrants and arriving New England farmers, with New York City growing the most dramatically.}}
    \cornell{What were the key characteristics of American population growth between 1840 and 1860?}{\begin{itemize}
        \item Urban growth became even more rapid betw. 1840-1860 
        \begin{itemize}
            \item In east, particularly NYC, but also Philadelphia, Boston; 26\% of free state population lived in towns by 1860 
            \item Western regions saw urban growth, too, like St. Louis, Pittsburgh, Cincinnati, Louisville due to MS river; connected New Orleans with Midwestern farmers and Northeastern merchants
            \item Great Lakes to MS river saw growth of \textbf{Chicago}, Buffalo, Detroit, Milwaukee, Cleveland
        \end{itemize}
        \item Urban growth remained stimulated by partly migration of New England farmers but mostly immigration from abroad
        \begin{itemize}
            \item Over 1.5m Europeans betw. 1840-1850; rising numbers in 1850s w/ over half of NYC pop. immigrants; very few in South 
            \item Some newcomers from England/France/Italy/Scandinavia/Poland/Holland
            \item Most from Ireland (oppressive GB rule, potato famine -> widespread death) and Germany (industrial revolution -> great poverty)
            \begin{itemize}
                \item Irish -> eastern cities as unskilled laborers due to low initial money, mostly young, single women (able to find factory/domestic work)
                \item Germans -> NW as farmers due to higher initial wealth, mostly either \underline{single men or families} to whom farm life was more accessible
            \end{itemize}
        \end{itemize}
        \item National population grew rapidly, too, surpassing Britain by 1860 and nearing France/Germany
    \end{itemize}
    \textbf{The American population became even more urban between 1840 and 1860, with eastern (like NYC/Philadelphia/Boston), western (like Pittsburgh/St. Louis), and Great Lake cities (like Chicago) growing most significantly. Immigration remained significant, mostly stimulated by the German single men and families, who became northwestern farmers, and the Irish young women, who took on unskilled jobs in factories.}}
    \cornell{What factors catalyzed the rise of nativism?}{\begin{itemize}
        \item Many Americans welcomed immigration: cheap labor -> low wages, land speculators saw potential for westward immigration, political leaders hoped to grow population and thus state influence (\textit{ex}: WI allowed immediate voting to immigrants promising citizenship, inhabiting for 1 year)
        \item "Nativism," defense of native-born, grew significantly with hostility, desire to stop immigration
        \begin{itemize}
            \item Often out of racism: comparable to views of natives/African Americans
            \item Others felt immigrants were unfit for unique American society due to many originating from poor cities, assuming it was a choice
            \item Workers feared low-wage foreigners would steal jobs from natives
            \item Protestants feared greater Catholic influence, Whigs feared Democratic influence, many feared immigrants were bribed for votes, many older feared radical ideals
        \end{itemize}
        \item Secret societies formed to combat immigration
        \begin{itemize}
            \item Began in Northeast but soon spread to West/South
            \item Native American association began in 1837 (w/ 1845 convention in Philadelphia)
            \item Supreme Order of the Star-Spangled Banner in 1850 attempted to ban Catholics/foreigners from public office, require literacy tests to vote; code of secrecy w/ password "I know nothing" -> known as "Know-Nothings"
        \end{itemize}
        \item Know-Nothings in particular turned to partisan politics -> created American Party after election of 1852 w/ very successful vote in 1854, winning MA state government; eventually declined
    \end{itemize}
    \textbf{Although many Americans appreciated the cheap labor costs and population growth accompanied by immigration, many others feared the immigrants due to racism, a belief in civil superiority, low-wages potentially stealing jobs, as well as shifts in religious and partisan demographics. Nativism began to grow, manifested in parties like the Native American Association and the Know-Nothings, who even had a significant political influence.}}
    \end{document}