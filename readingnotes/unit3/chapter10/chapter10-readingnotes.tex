\documentclass[a4paper]{article}
    \usepackage[T1]{fontenc}
    \usepackage{tcolorbox}
    \usepackage{amsmath}
    \tcbuselibrary{skins}
    
    \title{
    \vspace{-3em}
    \begin{tcolorbox}
    \Huge\sffamily \begin{center} AP US History  \\
    \LARGE Chapter 10 - America's Economic Revolution \\
    \Large Finn Frankis \end{center} 
    \end{tcolorbox}
    \vspace{-3em}
    }
    \date{}
    \author{}
    
    \usepackage{background}
    \SetBgScale{1}
    \SetBgAngle{0}
    \SetBgColor{red}
    \SetBgContents{\rule[0em]{4pt}{\textheight}}
    \SetBgHshift{-2.3cm}
    \SetBgVshift{0cm}
    \usepackage[margin=2cm]{geometry} 
    
    \makeatletter
    \def\cornell{\@ifnextchar[{\@with}{\@without}}
    \def\@with[#1]#2#3{
    \begin{tcolorbox}[enhanced,colback=gray,colframe=black,fonttitle=\large\bfseries\sffamily,sidebyside=true, nobeforeafter,before=\vfil,after=\vfil,colupper=blue,sidebyside align=top, lefthand width=.3\textwidth,
    opacityframe=0,opacityback=.3,opacitybacktitle=1, opacitytext=1,
    segmentation style={black!55,solid,opacity=0,line width=3pt},
    title=#1
    ]
    \begin{tcolorbox}[colback=red!05,colframe=red!25,sidebyside align=top,
    width=\textwidth,nobeforeafter]#2\end{tcolorbox}%
    \tcblower
    \sffamily
    \begin{tcolorbox}[colback=blue!05,colframe=blue!10,width=\textwidth,nobeforeafter]
    #3
    \end{tcolorbox}
    \end{tcolorbox}
    }
    \def\@without#1#2{
    \begin{tcolorbox}[enhanced,colback=white!15,colframe=white,fonttitle=\bfseries,sidebyside=true, nobeforeafter,before=\vfil,after=\vfil,colupper=blue,sidebyside align=top, lefthand width=.3\textwidth,
    opacityframe=0,opacityback=0,opacitybacktitle=0, opacitytext=1,
    segmentation style={black!55,solid,opacity=0,line width=3pt}
    ]
    
    \begin{tcolorbox}[colback=red!05,colframe=red!25,sidebyside align=top,
    width=\textwidth,nobeforeafter]#1\end{tcolorbox}%
    \tcblower
    \sffamily
    \begin{tcolorbox}[colback=blue!05,colframe=blue!10,width=\textwidth,nobeforeafter]
    #2
    \end{tcolorbox}
    \end{tcolorbox}
    }
    \makeatother

    \parindent=0pt
    
    \begin{document}
    \maketitle
    \SetBgContents{\rule[0em]{4pt}{\textheight}}
    \cornell[Key Concepts]{What are this chapter's key concepts?}{\begin{itemize}
        \item \textbf{4.2.I.A} - Entrepreneuers stimulated production and commerce revolutions with organized relations between producers and consumers
        \item \textbf{4.2.I.B} - Textile machinery, steam engines, interchangeable parts, telegraph, agricultural innovations led to greater efficiency
        \item \textbf{4.2.I.C} - Judicial systems -> transportation networks linking North/Midwest but still limited connections between those regions/South
        \item \textbf{4.2.II.A} - Americans began to support themselves w/ production (frequently working in factories rather than w/ agriculture
        \item \textbf{4.2.II.B} - Manufacturing growth -> many more prosperous with larger middle class, but also larger poor class
        \item \textbf{4.2.II.C} - Market revolution changed gender/family roles w/ domestic ideals emphasizing domestic spheres (public v. private)
        \item \textbf{4.2.III.A} - International migrants -> industrial north while many Americans west of Appalachians -> OH/MS rivers
        \item \textbf{5.1.II.A} - International migrants from Europe/Asia (predom. Ireland/Germany) in etrhnic communities preserving languages/customs
        \item \textbf{5.1.II.B} - Anti-Catholic antivist movement developed to curb political power of new immigrants
    \end{itemize}}
    \cornell[The Changing American Population]{How did American demographic changes serve to stimulate economic and social changes in American society?}{\textbf{The American population became increasingly characterized by immigration, particularly Irish Catholics and Germans; furthermore, free states began to shift to an increasingly urban lifestyle. However, accompanying this rise in immigration was the rise of a more sinister movement: nativism: the belief in the superiority of native American people over foreign immigrants, often fueled by racism but also economic competition. Nativismn manifested itself in multiple covert societies.}}
    \cornell{What were the key characteristics of American population growth between 1820 and 1840?}{\begin{itemize}
        \item Population increased rapidly and many moved from countryside to cities, others westward
        \item Native population growth more rapid than in Europe due to public health 
        \begin{itemize}
            \item Fewer epidemics
            \item High birth rate from white women, higher likelihood of surviving -> adulthood
        \end{itemize}
        \item Immigration relatively insignificant through beginning of nineteenth century; returned in 1830s
        \begin{itemize}
            \item Stimulated by reduced transportation costs, economic opportunities paired with deterioration in rest of Europe
            \begin{itemize}
                \item Irish Catholics were particularly large new group of migrants
            \end{itemize}
            \item Most traveled to cities of Northeast
            \begin{itemize}
                \item Complemented urban travel of agricultural New England inhabitants (some to west, but many to cities) 
                \item NYC saw particular growth: largest US city by 1810 due to natural harbor, Erie Canal, liberal commercial state laws 
            \end{itemize}
        \end{itemize}
    \end{itemize}
    \textbf{The American population grew due to growing public health leading to higher birth rates and a greater child survival rate as well as significant immigration after the 1830s due to reduced transportation costs. The Northeast cities grew most significantly both from immigrants and arriving New England farmers, with New York City growing the most dramatically.}}
    \cornell{What were the key characteristics of American population growth between 1840 and 1860?}{\begin{itemize}
        \item Urban growth became even more rapid betw. 1840-1860 
        \begin{itemize}
            \item In east, particularly NYC, but also Philadelphia, Boston; 26\% of free state population lived in towns by 1860 
            \item Western regions saw urban growth, too, like St. Louis, Pittsburgh, Cincinnati, Louisville due to MS river; connected New Orleans with Midwestern farmers and Northeastern merchants
            \item Great Lakes to MS river saw growth of \textbf{Chicago}, Buffalo, Detroit, Milwaukee, Cleveland
        \end{itemize}
        \item Urban growth remained stimulated by partly migration of New England farmers but mostly immigration from abroad
        \begin{itemize}
            \item Over 1.5m Europeans betw. 1840-1850; rising numbers in 1850s w/ over half of NYC pop. immigrants; very few in South 
            \item Some newcomers from England/France/Italy/Scandinavia/Poland/Holland
            \item Most from Ireland (oppressive GB rule, potato famine -> widespread death) and Germany (industrial revolution -> great poverty)
            \begin{itemize}
                \item Irish -> eastern cities as unskilled laborers due to low initial money, mostly young, single women (able to find factory/domestic work)
                \item Germans -> NW as farmers due to higher initial wealth, mostly either \underline{single men or families} to whom farm life was more accessible
            \end{itemize}
        \end{itemize}
        \item National population grew rapidly, too, surpassing Britain by 1860 and nearing France/Germany
    \end{itemize}
    \textbf{The American population became even more urban between 1840 and 1860, with eastern (like NYC/Philadelphia/Boston), western (like Pittsburgh/St. Louis), and Great Lake cities (like Chicago) growing most significantly. Immigration remained significant, mostly stimulated by the German single men and families, who became northwestern farmers, and the Irish young women, who took on unskilled jobs in factories.}}
    \cornell{What factors catalyzed the rise of nativism?}{\begin{itemize}
        \item Many Americans welcomed immigration: cheap labor -> low wages, land speculators saw potential for westward immigration, political leaders hoped to grow population and thus state influence (\textit{ex}: WI allowed immediate voting to immigrants promising citizenship, inhabiting for 1 year)
        \item "Nativism," defense of native-born, grew significantly with hostility, desire to stop immigration
        \begin{itemize}
            \item Often out of racism: comparable to views of natives/African Americans
            \item Others felt immigrants were unfit for unique American society due to many originating from poor cities, assuming it was a choice
            \item Workers feared low-wage foreigners would steal jobs from natives
            \item Protestants feared greater Catholic influence, Whigs feared Democratic influence, many feared immigrants were bribed for votes, many older feared radical ideals
        \end{itemize}
        \item Secret societies formed to combat immigration
        \begin{itemize}
            \item Began in Northeast but soon spread to West/South
            \item Native American association began in 1837 (w/ 1845 convention in Philadelphia)
            \item Supreme Order of the Star-Spangled Banner in 1850 attempted to ban Catholics/foreigners from public office, require literacy tests to vote; code of secrecy w/ password "I know nothing" -> known as "Know-Nothings"
        \end{itemize}
        \item Know-Nothings in particular turned to partisan politics -> created American Party after election of 1852 w/ very successful vote in 1854, winning MA state government; eventually declined
    \end{itemize}
    \textbf{Although many Americans appreciated the cheap labor costs and population growth accompanied by immigration, many others feared the immigrants due to racism, a belief in civil superiority, low-wages potentially stealing jobs, as well as shifts in religious and partisan demographics. Nativism began to grow, manifested in parties like the Native American Association and the Know-Nothings, who even had a significant political influence.}}
    \cornell[The Changing American Population]{What were the most significant transportation, technological, and communication booms?}{\textbf{Transportation innovations including canals traversed by steamboats and railroads linking distant parts of the North through consolidation began to greatly overshadow highways, with railroads ultimately becoming the most significant form of transportation. The telegraph, too, initially significant for its linking of railroad stations to allow for scheduling, allowed for significant communication and paired with Hoe's steam cylinder, allowed journalism to take off with the Associated Press.}}
    \cornell{What were the early forms of transportation?}{\begin{itemize}
        \item 1790s - 1820s: turnpike era, relying on roads
        \item MS/OH had been significant for years, but most traffic was from flat barges (little more than rafts) with cargo, which were torn up at end of journey (could only travel downstream); upstream vessels far more time-consuming
        \item By 1820s, steamboat industry had expanded significantly
        \begin{itemize}
            \item Corn/wheat of NW farmers and cotton/tobacco of South carried to New Orleans far more rapidly 
            \item Passenger industry grew w/ countries building lavish ships
        \end{itemize}
        \item Highways developed across mountains; though costs for overland transportation lowered, remained high
        \end{itemize}
        \textbf{The predominant forms of travel were by steamboat over rivers, over roads (the turnpike era), and over the Mississippi and Ohio Rivers (these were expensive and circuitous).}}
        \cornell{How did canals become the dominant form of transportation for some time?}{
        \textbf{Western farmers and Eastern merchants were unsatisfied with the circuitous route based on existing rivers and roads due to its high price: they sought to send goods more directly.}
        \begin{itemize}
            \item Canals extremely advantageous economically -> interest to expand to West; canals too expensive for companies -> states took burden
            \begin{itemize}
                \item Began in NY due to existing land route making it easier to build; underestimated forestry between points -> many questioned viability, but governor De Witt Clinton advocated for it
                \begin{itemize}
                    \item Erie Canal largest construction project to date w/ forty foot-wide ditch four feet deep; required difficult cuts and aqueducts to carry canal across streams, heavy masonry w/ wooden gates
                    \item Instant financial success w/ extremely heavy traffic -> repaid entire cost w/in 7 years 
                    \item Route to Great Lakes provided direct access to Chicago from NYC -> NYC began to compete w/ New Orleans
                \end{itemize}
                \item Extended by OH and IN w/ connection betw. Lake Erie, Ohio River creating inland water route to NY 
                \begin{itemize}
                    \item Still required frequent transfer of goods betw. lake, river, and canal vessels
                    \item New canal -> increased white settlement in NW due to ease of travel for migrants
                \end{itemize}
                \item Rival cities unable to catch up to NY: Boston blocked by Berkshire mountains, Philadelphia/Baltimore made effort to cross larger Allegheny Mountains but too expensive for PN and unable to cross mountains for MD
            \end{itemize}
        \end{itemize}
    \textbf{Canals soon overshadowed turnpikes for their great efficiency of transferring goods using steamboats. Almost entirely state-funded, the Erie Canal from New York City to Lake Erie was by far the most successful, establishing NYC as a city formidable to New Orleans. Ohio and Indiana followed with a water route between Lake Erie and the Ohio River, but most large cities were unable to compete with NYC remaining the most dominant.}}
    \cornell{What marked the early development of railroads?}{\begin{itemize}
        \item Railroads were combination of tracks, steam power, railroad cars as public carriers
        \begin{itemize}
            \item Developed by both English and American inventors by 1804
            \item John Stevens formed first in 1820 around NJ estate 
            \item Short Stockton/Darlington Railroad created in England in 1825
        \end{itemize}
        \item American entrepreneurs intrigued by English experiment -> formed companies, first being Baltimore and Ohio creating thirteen-mile stretch in 1830; Mohawk and Hudson in NY in 1831; over 1k total miles by 1836
        \item Railroads remained small and insignificant
        \begin{itemize}
            \item Mostly designed to connect water routes
            \item Tracks often differed in width between connecting lines, preventing travel of one continuous train
            \item Schedules inconsistent, frequent wrecks
        \end{itemize}
        \item Saw advances of iron rails, redesigned passenger cars by 1830s and 1840s
        \item Competition between companies grew
        \begin{itemize}
            \item \textit{ex}: Chesapeake/Ohio Canal Company prevented Baltimore/Ohio Railroad  from travelling through Potomac
            \item NY prohibited from competing w/ Erie Canal
        \end{itemize}
    \end{itemize}
    \textbf{Railroads became truly significant in the 1830s, when mere experiments began to transform into genuinely lucrative industries; although they remained insignificant for some time, with frequent wrecks and inconsistent service, advances by the 1840s led to great competition between companies as well as from states.}}
    \cornell{How did railroads continue to expand?}{\begin{itemize}
        \item Northheast had most efficient system (more than NW and significantly more than South); began to reach west of MS River over iron bridges (St. Louis and Kansas City)
        \item Key trend: \textbf{consolidation} w/ shorter lines combining to form larger lines; 1853 saw connection of four tracks between Northeast and Northwest over Appalachians 
        \begin{itemize}
            \item NY Central / NY and Erie connected NYC w/ Lake Erie 
            \item PN railroad linked Philadelphia/Pittsburgh
            \item OH connected Baltimore w/ OH River at Wheeling
            \item Railroads into interior touched MS river at 8 points, predominantly Chicago as central western rail center
            \item Trunk lines diverted traffic from primary water routes (like Erie Canal / MS River), reducing NW connection to South 
        \end{itemize}
        \item Railroads funded in part by private investors, with railroad companies receiving loans from abroad; assostamce from local government and federal government as land grants
        \begin{itemize}
            \item 1850: Stephen A. Douglas convinced Congress to grant federal lands to Illinois Central; other states soon followed, demanding privileges
        \end{itemize}
    \end{itemize}
    \textbf{The Northeast enjoyed the most efficient rail system, with multiple routes beginning to cross the MS River. Consolidation, or the combination of multiple shorter lines to form larger lines, became critical to expanding the rail system: four critical lines began to reach westward. Funding for railroads came from both private investors (local and abroad) as well as from state/federal governments.}}
    \cornell{How did the telegraph revolutionze communication in the United States?}{\begin{itemize}
        \item \textbf{Magnetic telegraph} created by Samuel F.B. Morse, after sending news of Polk's nomination from Baltimore to DC; low cost made Morse telegraph system seem ideal
        \begin{itemize}
            \item Expanded rapidly: more than 50k miles by 1860; Pacific telegraph connected NYC and SFO
            \item Joined in Western Union Telegraph Company
        \end{itemize}
        \item Telegraphs had wide-reaching effects
        \begin{itemize}
            \item Extended along railroad tracks to connect stations and coordinate train scheduling
            \item Significant for communication between cities
            \item Further aggravated schism betw. South and North, connecting Northeast and Northwest because lines were far more extensive (mostly followed railroad tracks)
        \end{itemize}
    \end{itemize}
    \textbf{The magnetic telegraph expanded rapidly in the late 1850s, with important effects of coordinating railroad scheduling, encouraging communication between distant cities, and ultimately further separating the North and the South.}}
    \cornell{How did journalism change in the United States?}{\begin{itemize}
        \item 1846: \textbf{Richard Hoe} invented steam cylinder rotary press, allowing rapid/cheap printing of newspapers; paired with telegraph to share news far more easily betw. cities, revolutionized communication w/ formation of Associated Press
        \item Northeast created early metropolitan newspapers: NY saw Horace Greeley's \textit{Tribune}, James Gordon Bennett's \textit{Herald}, Henry J. Raymond's \textit{Times}; all detailed international events
        \item Journalism -> sectionalism in 1840s/1850s w/ most major magazines in North -> South felt subjugated due to smaller budgets for newspapers with little impact outside of communities 
    \end{itemize}
    \textbf{Journalism, revolutionized by the pairing of the telegraph and the steam cylinder rotary press, led to the formation of the Associated Press, dedicated to expanding communication throughout the nation. However, because the Northeast had the most significant and far-reaching newspapers, journalism further fueled sectionalism.}}
    \cornell[Commerce and Industry]{What were the primary developments in commerce and industry during the market revolution?}{\textbf{During the market revolution, business began to be increasingly dominated by larger, freer corporations due to reduced legal restrictions. The factory system, too, became particularly significant, particulary in the Northeast, beginning in the textile and shoe industries. Finally, technology advanced greatly to meet the growing industry, particularly in armories sponsored by the federal government, spurred on by interchangeable parts, new energy sources, and brilliant inventors.}}
    \cornell{How did business expand between 1820 and 1840?}{\begin{itemize}
        \item Grew rapidly due to population growth, transportation, new entrepreneur class w/ growing wealth
        \item Retail grew significantly: larger cities saw stores specializing in groceries, hardware, etc.
        \begin{itemize}
            \item Smaller towns still relied on "general stores"; some even performed business through barter
        \end{itemize}
        \item Business organization changed: although most remained founded by individuals/limited partnerships dominated by merchant capitalists, larger business saw \textbf{corporation}
        \begin{itemize}
            \item 1830s saw reduced legal restrictions for corporations (previously required special state charter to grow, replaced w/ general incorporation for small fee)
            \begin{itemize}
                \item Limited liability system meant that stockholders never liable for losses of corporation, only their own stock
            \end{itemize}
            \item Businesses could not survive on investment alone -> relied on credit
            \begin{itemize}
                \item Borrowing often -> instability due to weak credit system (govt. could only issue gold/silver, too little in quantity to meet credit demands)
                \item Banks began to issue bank notes but depended on reputation of bank -> known for insecurity, bank failure
            \end{itemize}
        \end{itemize}
    \end{itemize}
    \textbf{Between 1820 and 1840, businesses grew rapidly, particularly those based around corporations, which had recently been freed of numerous legal shackles, now able to be formed easily with limited stockholder liability. Credit grew in significance for funding larger businesses, but the demand could not be satisfied by the government's stock of gold and silver, leading to great bank instability.}}
    \cornell{How did the factory system develop?}{\begin{itemize}
        \item Pre-War of 1812, most manufacturing done w/in household: hand-operated; improved technology -> growth of factory beginning in New England textile industry w/ larger water-powered machines by 1820s
        \item Factories also pervaded shoe industry (in eastern MA); although still hand-made, based around division of various tasks to form identical, consistent shoes
        \item 1840-1860 saw even larger growth of factory system with rapid expansion in manufactured goods, matching value of total agricultural goods by 1860
        \item Majority and largest of plants in Northeast
    \end{itemize}
    \textbf{The factory system developed and began to replace homespun goods as technology expanded, beginning in the textile industry of New England in the 1820s and expanding to the shoe industry of MA. By 1860, factories had grown enormously (particularly in the Northeast), with manufactured goods finally equal in value to agricultural goods.}}
    \cornell{What were the main technological advances spurring the Industrial Revolution?}{\begin{itemize}
        \item Industry remained very rudimentary compared to modern times (\textit{ex}: cotton still produced coarsely), but machinery grew in U.S. more rapidly than anywhere else
        \begin{itemize}
            \item Reached point where GB and other European nations would travel to U.S. to study machines
        \end{itemize}
        \item Machine tools manufacturing (supported by govt. to support military) particularly large
        \begin{itemize}
            \item \textit{ex}: armory in MA formed turret lathe (cutting screws/metal) and milling machine (replaced hand chiseling), precision grinding machine for construction of other machines (like sewing machines) but more significantly for rifles
            \item Armories became known as hotbed for technological advancement, attracting craftsmen/factory owners to learn and share new ideas
        \end{itemize}
        \item Interchangeable parts of Eli Whitney/Simeon North expanded beyond gun industry, including to watch/clock making, trains, farm tools
        \item New energy sources benefitted industrialization greatly, w/ coal (mostly from around Pittsburgh) replacing wood/water -> factories could be placed further from streams
        \begin{itemize}
            \item Most factories remained dependent on water power, with largest factories close to natural waterfalls
            \item Dependence -> often closed during winter w/ frozen rivers 
        \end{itemize}
        \item American industry spurred by American inventors including Charles Goodyear (rubber industry), Elias Howe (sewing machine in conjunction w/ Isaac Singer)
    \end{itemize}
    \textbf{American industry grew at an unparalleled rate in the mid-19th century, rapidly surpassing even Britain in tool complexity, with military machine tools manufacturing plants (notably armories) representing the most significant hotbeds. Interchangeable parts and new energy sources were exploited by American inventors to develop advanced machinery and systematic methods of constructing parts.}}
    \end{document}