\documentclass[a4paper]{article}
    \usepackage[T1]{fontenc}
    \usepackage{tcolorbox}
    \usepackage{amsmath}
    \tcbuselibrary{skins}
    
    \title{
    \vspace{-3em}
    \begin{tcolorbox}
    \Huge\sffamily \begin{center} AP US History  \\
    \LARGE Chapter 11 - Cotton, Slavery, and The Old South \\
    \Large Finn Frankis \end{center} 
    \end{tcolorbox}
    \vspace{-3em}
    }
    \date{}
    \author{}
    
    \usepackage{background}
    \SetBgScale{1}
    \SetBgAngle{0}
    \SetBgColor{red}
    \SetBgContents{\rule[0em]{4pt}{\textheight}}
    \SetBgHshift{-2.3cm}
    \SetBgVshift{0cm}
    \usepackage[margin=2cm]{geometry} 
    
    \makeatletter
    \def\cornell{\@ifnextchar[{\@with}{\@without}}
    \def\@with[#1]#2#3{
    \begin{tcolorbox}[enhanced,colback=gray,colframe=black,fonttitle=\large\bfseries\sffamily,sidebyside=true, nobeforeafter,before=\vfil,after=\vfil,colupper=blue,sidebyside align=top, lefthand width=.3\textwidth,
    opacityframe=0,opacityback=.3,opacitybacktitle=1, opacitytext=1,
    segmentation style={black!55,solid,opacity=0,line width=3pt},
    title=#1
    ]
    \begin{tcolorbox}[colback=red!05,colframe=red!25,sidebyside align=top,
    width=\textwidth,nobeforeafter]#2\end{tcolorbox}%
    \tcblower
    \sffamily
    \begin{tcolorbox}[colback=blue!05,colframe=blue!10,width=\textwidth,nobeforeafter]
    #3
    \end{tcolorbox}
    \end{tcolorbox}
    }
    \def\@without#1#2{
    \begin{tcolorbox}[enhanced,colback=white!15,colframe=white,fonttitle=\bfseries,sidebyside=true, nobeforeafter,before=\vfil,after=\vfil,colupper=blue,sidebyside align=top, lefthand width=.3\textwidth,
    opacityframe=0,opacityback=0,opacitybacktitle=0, opacitytext=1,
    segmentation style={black!55,solid,opacity=0,line width=3pt}
    ]
    
    \begin{tcolorbox}[colback=red!05,colframe=red!25,sidebyside align=top,
    width=\textwidth,nobeforeafter]#1\end{tcolorbox}%
    \tcblower
    \sffamily
    \begin{tcolorbox}[colback=blue!05,colframe=blue!10,width=\textwidth,nobeforeafter]
    #2
    \end{tcolorbox}
    \end{tcolorbox}
    }
    \makeatother

    \parindent=0pt
    
    \begin{document}
    \maketitle
    \SetBgContents{\rule[0em]{4pt}{\textheight}}
    \cornell[Key Concepts]{What are this chapter's key concepts?}{\begin{itemize}
        \item \textbf{4.1.II.D} - Communities emerged allowing enslaved/free Afr. Americans to protect cultural and social heritage w/ political goals to improve status
        \item \textbf{4.1.III.B} - North saw successful abolitionist movements w/ growing free black populations, but govt. still restricted rights; only form of rebellion in South were failed uprisings
        \item \textbf{4.2.III.B} - Southern cotton production paired w/ Northern manufacturing -> growth of U.S. commercial ties, both domestic and w/ other nations
        \item \textbf{4.2.III.C} - Southerners relied on export of agriculture -> unique regional identity
        \item \textbf{4.3.II.A} - Overcultivation in Southeast -> west of Appalachians saw more successful slave plantations
        \item \textbf{4.3.II.B} - Majority of Southerners non-slave-owners, but most still argued that slavery was part of life
    \end{itemize}}
    \cornell[The Cotton Economy]{What were the essential characteristics of the southern cotton economy and what were its effects?}{\textbf{Cotton began to dominate existing sugar, rice, and tobacco cultivation in the South, becoming central by the 1860s, causing slave populations to skyrocket, particularly in the Deep South. However, industry remained subordinate to the plantation economy, with many fearing a growing industrial dependence on the North; little changed, however, because of the inherent profitability of land, long-term investment in lands limiting investment elsewhere, and a unique set of gracious, "cavalier" ideals.}}
    \cornell{What were the conditions of Southern economies before the advent of cotton?}{\begin{itemize}
        \item Upper South had continued to rely on unstable boom-bust, land-exhausting tobacco market; VA, NC, MD began to shift to wheat
        \item Southern regions of coastal South (SC, GA, FL) relied on stable rice cultivation requiring long growing season -> kept within small area
        \begin{itemize}
            \item Some long-staple cotton growth, but could only grow in limited coastal area
        \end{itemize}
        \item Gulf Coast growers enjoyed successful yet labor-intensive sugar industry
        \begin{itemize}
            \item Only accessible to wealthy planters, significant competition w/ Caribbean -> kept to southern LA, eastern TX
        \end{itemize}
    \end{itemize}
    \textbf{The Upper South continued to engage in the unstable tobacco industry with some shifts to wheat production, the coastal southern South focused on rice cultivation with a long growing season, and the Gulf Coast farmers focused on the sugar industry. All of these were restricted to a small growing area.}}
    \cornell{How did cotton transform Southern farming?}{\begin{itemize}
        \item Short-staple cotton, able to grow in mixed climates, initially unpopular due to difficulty of separating seeds; solved by cotton gin
        \item High demand from British and New England textile industry -> immigration to uncultivated lands for cotton regions
        \begin{itemize}
            \item Spread from western SC/GA to AL, MS, LA, TX, AR, \textit{central by 1850s}
            \item Overproduction to meet demand -> frequent busts, but growth continued w/ cotton becoming $\frac{2}{3}$ of U.S. exports and 100x more valuable than rice
        \end{itemize}
        \item Era known as "Cotton Kingdom" w/ large profits and white settlers moving in large quantities to the South, some wealthy planters simply transferring assets, but most were smaller farmers 
        \item Slave population in lower South grew significantly (over 10x in AL/MS) while upper South grew far less
        \begin{itemize}
            \item Many slaves accompanied upper South masters; others simply sold -> econ. of upper South grew
        \end{itemize}
    \end{itemize}
    \textbf{Short-staple cotton, due to the invention of the cotton gin, grew significantly, becoming the most significant American export crop and bringing large profits to the Deep South. It also caused a significant shift in slavery away from the upper South toward the Deep South.}}
    \cornell{What forms of industry emerged amidst the growth of cotton?}{\begin{itemize}
        \item Merchants/manufacturers significant w/ textile/iron/flour milling particularly in upper South (\textit{ex}: Tredegar Iron Works comparable to largest of Notheast)
        \item Industry grew, but still paled in comparison w/ cotton production
        \item Southern commercial sector generally assisted plantations, w/ \textbf{brokers, or "factors"} of southern towns to market crops
        \begin{itemize}
            \item Financial system weak w/ factors providing credit to plantation owners, debt growing during cotton decline periods -> merchant-bankers influential 
            \item Lawyers/editors/doctors dependent on plantation economy 
        \end{itemize}
        \item Transportation remained inadequate w/ no investment in canals, roads unable to carry heavy loads, railroads unable to truly tie
        \begin{itemize}
            \item Memphis only true hub connecting to Northeast, w/ Charleston/Atlanta/Savannah/Norfolk/Richmond having direct connections
            \item Majority remained disconnected from national system, w/ some lines between rivers but most relying on \textit{river and sea} transport 
        \end{itemize}
        \item Most southerners recognized economic inferiority to North
        \begin{itemize}
            \item AR journalist Albert Pike complained about dependence on North 
            \item James B.D. De Bow of New Orleans created \textit{De Bow's Review} from 1846-1880 emphasizing importance of agricultural/commercial growth 
            \begin{itemize}
                \item Lamented potential "colonial" relationship betw. South/North; believed significant money lost 
                \item Printed in NY w/ advertisements from northern manufacturers; sold signif. less in South than Northern magazines did in similar regions
            \end{itemize}
        \end{itemize}
    \end{itemize}
    \textbf{Although a small industrial sector developed in the South, it remained mostly dependent on and overshadowed by the plantation economy. Industry was further stagnated by weak transportation, with few railroads and canals meaning that rivers and the sea remained the priamry means of transportations. Many Southerners lamented their inferiority to and dependence on the North.}}
    \cornell{What were key causes for the South's continual dependence on agriculture?}{\begin{itemize}
        \item Profitability of lands -> few looked further; wealthy had majority of investments in land; climate potentially not conducive to industry; many in North felt South lacked work ethic
        \item Unique Southern values based on traditional chivalry/leisure/elegance far more gracious than northern "yankees": above industrial work
    \end{itemize}
    \textbf{Despite lamentations from many Southerners concerning their industrial dependence on the North, few were able to move beyond agriculture due to inherent land profitability, great investments made in existing lands, a hot climate, and, according to many Northerners, a weak work ethic. Personally, many Southerners felt their chivalric role as "cavaliers" meant they were above the toil of industry.}}
    \end{document}