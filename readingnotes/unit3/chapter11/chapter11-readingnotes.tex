\documentclass[a4paper]{article}
    \usepackage[T1]{fontenc}
    \usepackage{tcolorbox}
    \usepackage{amsmath}
    \tcbuselibrary{skins}
    
    \title{
    \vspace{-3em}
    \begin{tcolorbox}
    \Huge\sffamily \begin{center} AP US History  \\
    \LARGE Chapter 11 - Cotton, Slavery, and The Old South \\
    \Large Finn Frankis \end{center} 
    \end{tcolorbox}
    \vspace{-3em}
    }
    \date{}
    \author{}
    
    \usepackage{background}
    \SetBgScale{1}
    \SetBgAngle{0}
    \SetBgColor{red}
    \SetBgContents{\rule[0em]{4pt}{\textheight}}
    \SetBgHshift{-2.3cm}
    \SetBgVshift{0cm}
    \usepackage[margin=2cm]{geometry} 
    
    \makeatletter
    \def\cornell{\@ifnextchar[{\@with}{\@without}}
    \def\@with[#1]#2#3{
    \begin{tcolorbox}[enhanced,colback=gray,colframe=black,fonttitle=\large\bfseries\sffamily,sidebyside=true, nobeforeafter,before=\vfil,after=\vfil,colupper=blue,sidebyside align=top, lefthand width=.3\textwidth,
    opacityframe=0,opacityback=.3,opacitybacktitle=1, opacitytext=1,
    segmentation style={black!55,solid,opacity=0,line width=3pt},
    title=#1
    ]
    \begin{tcolorbox}[colback=red!05,colframe=red!25,sidebyside align=top,
    width=\textwidth,nobeforeafter]#2\end{tcolorbox}%
    \tcblower
    \sffamily
    \begin{tcolorbox}[colback=blue!05,colframe=blue!10,width=\textwidth,nobeforeafter]
    #3
    \end{tcolorbox}
    \end{tcolorbox}
    }
    \def\@without#1#2{
    \begin{tcolorbox}[enhanced,colback=white!15,colframe=white,fonttitle=\bfseries,sidebyside=true, nobeforeafter,before=\vfil,after=\vfil,colupper=blue,sidebyside align=top, lefthand width=.3\textwidth,
    opacityframe=0,opacityback=0,opacitybacktitle=0, opacitytext=1,
    segmentation style={black!55,solid,opacity=0,line width=3pt}
    ]
    
    \begin{tcolorbox}[colback=red!05,colframe=red!25,sidebyside align=top,
    width=\textwidth,nobeforeafter]#1\end{tcolorbox}%
    \tcblower
    \sffamily
    \begin{tcolorbox}[colback=blue!05,colframe=blue!10,width=\textwidth,nobeforeafter]
    #2
    \end{tcolorbox}
    \end{tcolorbox}
    }
    \makeatother

    \parindent=0pt
    
    \begin{document}
    \maketitle
    \SetBgContents{\rule[0em]{4pt}{\textheight}}
    \cornell[Key Concepts]{What are this chapter's key concepts?}{\begin{itemize}
        \item \textbf{4.1.II.D} - Communities emerged allowing enslaved/free Afr. Americans to protect cultural and social heritage w/ political goals to improve status
        \item \textbf{4.1.III.B} - North saw successful abolitionist movements w/ growing free black populations, but govt. still restricted rights; only form of rebellion in South were failed uprisings
        \item \textbf{4.2.III.B} - Southern cotton production paired w/ Northern manufacturing -> growth of U.S. commercial ties, both domestic and w/ other nations
        \item \textbf{4.2.III.C} - Southerners relied on export of agriculture -> unique regional identity
        \item \textbf{4.3.II.A} - Overcultivation in Southeast -> west of Appalachians saw more successful slave plantations
        \item \textbf{4.3.II.B} - Majority of Southerners non-slave-owners, but most still argued that slavery was part of life
    \end{itemize}}
    \cornell[The Cotton Economy]{What were the essential characteristics of the southern cotton economy and what were its effects?}{\textbf{Cotton began to dominate existing sugar, rice, and tobacco cultivation in the South, becoming central by the 1860s, causing slave populations to skyrocket, particularly in the Deep South. However, industry remained subordinate to the plantation economy, with many fearing a growing industrial dependence on the North; little changed, however, because of the inherent profitability of land, long-term investment in lands limiting investment elsewhere, and a unique set of gracious, "cavalier" ideals.}}
    \cornell{What were the conditions of Southern economies before the advent of cotton?}{\begin{itemize}
        \item Upper South had continued to rely on unstable boom-bust, land-exhausting tobacco market; VA, NC, MD began to shift to wheat
        \item Southern regions of coastal South (SC, GA, FL) relied on stable rice cultivation requiring long growing season -> kept within small area
        \begin{itemize}
            \item Some long-staple cotton growth, but could only grow in limited coastal area
        \end{itemize}
        \item Gulf Coast growers enjoyed successful yet labor-intensive sugar industry
        \begin{itemize}
            \item Only accessible to wealthy planters, significant competition w/ Caribbean -> kept to southern LA, eastern TX
        \end{itemize}
    \end{itemize}
    \textbf{The Upper South continued to engage in the unstable tobacco industry with some shifts to wheat production, the coastal southern South focused on rice cultivation with a long growing season, and the Gulf Coast farmers focused on the sugar industry. All of these were restricted to a small growing area.}}
    \cornell{How did cotton transform Southern farming?}{\begin{itemize}
        \item Short-staple cotton, able to grow in mixed climates, initially unpopular due to difficulty of separating seeds; solved by cotton gin
        \item High demand from British and New England textile industry -> immigration to uncultivated lands for cotton regions
        \begin{itemize}
            \item Spread from western SC/GA to AL, MS, LA, TX, AR, \textit{central by 1850s}
            \item Overproduction to meet demand -> frequent busts, but growth continued w/ cotton becoming $\frac{2}{3}$ of U.S. exports and 100x more valuable than rice
        \end{itemize}
        \item Era known as "Cotton Kingdom" w/ large profits and white settlers moving in large quantities to the South, some wealthy planters simply transferring assets, but most were smaller farmers 
        \item Slave population in lower South grew significantly (over 10x in AL/MS) while upper South grew far less
        \begin{itemize}
            \item Many slaves accompanied upper South masters; others simply sold -> econ. of upper South grew
        \end{itemize}
    \end{itemize}
    \textbf{Short-staple cotton, due to the invention of the cotton gin, grew significantly, becoming the most significant American export crop and bringing large profits to the Deep South. It also caused a significant shift in slavery away from the upper South toward the Deep South.}}
    \cornell{What forms of industry emerged amidst the growth of cotton?}{\begin{itemize}
        \item Merchants/manufacturers significant w/ textile/iron/flour milling particularly in upper South (\textit{ex}: Tredegar Iron Works comparable to largest of Notheast)
        \item Industry grew, but still paled in comparison w/ cotton production
        \item Southern commercial sector generally assisted plantations, w/ \textbf{brokers, or "factors"} of southern towns to market crops
        \begin{itemize}
            \item Financial system weak w/ factors providing credit to plantation owners, debt growing during cotton decline periods -> merchant-bankers influential 
            \item Lawyers/editors/doctors dependent on plantation economy 
        \end{itemize}
        \item Transportation remained inadequate w/ no investment in canals, roads unable to carry heavy loads, railroads unable to truly tie
        \begin{itemize}
            \item Memphis only true hub connecting to Northeast, w/ Charleston/Atlanta/Savannah/Norfolk/Richmond having direct connections
            \item Majority remained disconnected from national system, w/ some lines between rivers but most relying on \textit{river and sea} transport 
        \end{itemize}
        \item Most southerners recognized economic inferiority to North
        \begin{itemize}
            \item AR journalist Albert Pike complained about dependence on North 
            \item James B.D. De Bow of New Orleans created \textit{De Bow's Review} from 1846-1880 emphasizing importance of agricultural/commercial growth 
            \begin{itemize}
                \item Lamented potential "colonial" relationship betw. South/North; believed significant money lost 
                \item Printed in NY w/ advertisements from northern manufacturers; sold signif. less in South than Northern magazines did in similar regions
            \end{itemize}
        \end{itemize}
    \end{itemize}
    \textbf{Although a small industrial sector developed in the South, it remained mostly dependent on and overshadowed by the plantation economy. Industry was further stagnated by weak transportation, with few railroads and canals meaning that rivers and the sea remained the priamry means of transportations. Many Southerners lamented their inferiority to and dependence on the North.}}
    \cornell{What were key causes for the South's continual dependence on agriculture?}{\begin{itemize}
        \item Profitability of lands -> few looked further; wealthy had majority of investments in land; climate potentially not conducive to industry; many in North felt South lacked work ethic
        \item Unique Southern values based on traditional chivalry/leisure/elegance far more gracious than northern "yankees": above industrial work
    \end{itemize}
    \textbf{Despite lamentations from many Southerners concerning their industrial dependence on the North, few were able to move beyond agriculture due to inherent land profitability, great investments made in existing lands, a hot climate, and, according to many Northerners, a weak work ethic. Personally, many Southerners felt their chivalric role as "cavaliers" meant they were above the toil of industry.}}
    \cornell[White Society in the South]{How was Southern society structured for whites?}{\textbf{White society saw a disproportionately dominant planter class whose hard work was often masked behind a genteel image who focused on maintaining honor and dignity by protecting and subordinating women, who often had limited access to education and suffered from adulterous husbands (with slaves) and infant mortality. The lower classes generally owned few slaves but, apart from the highlanders (who actively rebelled against the plantation system) and the "white trash," (whose poverty gave them little means to rebel), most were deeply ingrained in and economically connected with plantation society. Class conflict remained limited, however, due to the continual white belief in racial superiority (a form of leadership) and close family connections between members of disparate classes.}}
    \cornell{How did slaveowning whites compare to non-slaveowning whites?}{\textbf{A relatively small minority of Southern whites owned slaves: slaveowning families (including all members of the family) amounted to at most one quarter of the population. Most owned very few.}}
    \cornell{How did the planter class dominate Southern society?}{\begin{itemize}
        \item Planter aristocracy had disproportionate influence in society
        \begin{itemize}
            \item Controlled all life of region 
            \item Received great incomes, owned large homes and pieces of land, often spent months in urban homes 
            \item Some travelled to Europe, hosted large social events
        \end{itemize}
        \item Most southerners compared planter class to old upper classes of England/Europe: long aristocracies, but in fact very disparate 
        \begin{itemize}
            \item In some parts of VA, a long aristocracy had developed
            \item Most planters relatively new to profession, many first-generation settlers who had underwent large struggle
            \item "Old South" (pre-Civil War) settled for less than two decades when war began
        \end{itemize}
        \item Few planters enjoyed true relaxation: great competitions with industry as risky as North
        \begin{itemize}
            \item Required skilled overseeing; many planters invested majority of wealth in money/slaves and lived in lesser conditions
            \item Planters often forced to move to lands with greater agriculture
            \item Many planters had worked hard $\to$ perhaps created "cavalier" image to defend their rights (particularly in lower, new South)
        \end{itemize}
    \end{itemize}
    \textbf{Although the planter class had a disproportionate social influence, often owning vast tracts of land, they were very different from Britain's aristocracy of the past: most were relatively new to their positions of wealth and still had to work hard to earn their money and prevent competition from other planters.}}
    \cornell{How did southern whites perpetuate an aristocratic image?}{\textbf{Most southern whites avoided trade and commerce, known as "coarse" occupations. Many wealthier members joined the military to fulfill the image of the medieval knight, often acting chivalric toward women.}}
    \cornell{How did white males develop a code of chivalry?}{\begin{itemize}
        \item White males felt required to defend honor -> frequent dueling (mostly vanished in the North); treated each other w/ great respect to distance from great cruelty of slavery 
        \item Beyond bravery, idea of honor significant for public image: maintaining pride/dignity by fighting all which challenged 
        \begin{itemize}
            \item \textit{ex}: SC congressman Preston Brooks beat Senator Charles Sumner of MA w/ cane for insult against relative -> South saw as hero, North as savage
            \item Greater focus on defending honor by avenging insults, particularly against women
        \end{itemize}
    \end{itemize}
    \textbf{White males focused greatly on the ideal of "honor," a virtue requiring defending at all costs. Any insult to one's dignity or to a woman required direct retaliation to save one's public image.}}
    \cornell{How were wealthy women of the South generally treated?}{\begin{itemize}
        \item Comparable to Northern middle-class w/ life in home, directly assisting husbands and raising children
        \begin{itemize}
            \item Rarely partook in public activities/employment for money (unlike North)
            \item Honor -> men should defend women -> white women even more subordinate, requiring to obey for protection 
        \end{itemize}
        \item Most women lived on farms without public access -> no ways of looking beyond roles in life
        \begin{itemize}
            \item Male dominance in South > than in North because more at-home activities like spinning/weaving, supervising slaves
            \item Wives of wealthier plantation owners rarely worked: generally mere showpieces
        \end{itemize}
        \item Less access to education w/ $\frac{1}{4}$ over 20 years old illiterate
        \begin{itemize}
            \item Wealthy planters rarely educated daughters
            \item Only female schools intended to train in being wife 
        \end{itemize}
        \item High birth rates -> $\uparrow$ infant mortality
        \item Slavery -> husbands often had affairs w/ female slaves, children becoming part of labor force 
        \item Some women -> abolitionists, joining northerners; others sought simple social reform; most lacked outlet, accepting position as distinctly Southern and placing them \underline{above poor}
    \end{itemize}
    \textbf{Wealthy women of the South rarely engaged in public activities and were even more subordinate to men due to the belief in honor requiring constant protection. Most were nearly isolated from society, lacking access to education, experiencing great infant mortality, and suffering the psychological effects of their husbands' affairs with female slaves. Although some women sought social reform through abolition, most accepted their position with little fight.}}
    \cornell{How did the lower southern classes grow in numbers and what were their social prospects?}{\begin{itemize}
        \item Most white southerners were modest farmers without slaves (those who did own some had them in $\downarrow$ numbers, worked closely together)
        \begin{itemize}
            \item 1850s saw $\uparrow$ nonslaveholding landowners faster than slaveowning ones 
        \end{itemize}
        \item Most aware of grim potential of social mobility
        \begin{itemize}
            \item Limited education for poor (universities important but only for planters' children) w/ non-universities generally poorly funded/maintained
            \item Isolation with little social influence -> why didn't they rebel against the aristocracy?
        \end{itemize}
    \end{itemize}
    \textbf{Most southerners worked without slaves, with the nonslaveholding class growing rapidly. They lacked potential for social mobility, with limited education and political influence.}}
    \cornell{How did some whites rebel against the planter aristocracy?}{\begin{itemize}
        \item Southern highlanders ("hill people") in Appalachians east of MS River, also in Ozarks to the west, and "backcountry" cut off from commerce 
        \begin{itemize}
            \item Most isolated w/o slaves, subsistence agri. only
            \item Proud of seclusion, but rarely connected with growing economy of South
            \item No surplus contribution to market, using all additional crops to barter for other items
        \end{itemize}
        \item Highlanders despised slavery: threatened own beliefs of independence, created unnecessary focus on personal property; older ideals of nationalism
        \item Highlanders disliked planter aristocracy, rarely partaking in sectionalism
        \begin{itemize}
            \item During Civil War, some refused to support Confed. and some fought for Union 
        \end{itemize}
    \end{itemize}
    \textbf{The Southern highlanders disliked the planter aristocracy most significantly: mostly isolated from southern commercial growth, their independence led them to despise slavery and their nationalism caused them to rarely take sides in the Civil War, occasionally fighting for the Union.}}
    \cornell{What whites were more deeply ingrained in the planter aristocracy?}{\textbf{Most southern whites depended on the plantation economy to access cotton gins, sell their goods, and receive credit; they were often linked through kinship (direct relations weakened class tension). Furthermore, because the South was particularly democratic for white men, with many feeling a sense of direct influence (though most officeholders were elites). Finally, the cotton boom almost always helped smaller farmers to buy slaves, land, or secure their long-held positions with a slight boost.}}
    \cornell{How did smaller farmers stress a patriarchal social order?}{\textbf{Because many economies took place in the household, men felt that their hard rule on children and women would ensure productivity; furthermore, the Northern assault on slavery led many to fear a future assault on the patriarchy.}}
    \cornell{How did a class of "white trash" emerge in the South?}{\textbf{The most degraded class was the "poor white trash," living on infertile lands in small cabins and occasionally working for neighbors. Most sustained themselves through hunting and foraging, but were often forced to eat diseased clay, leading to the spread of significant disease. Many slaves, even, pitied their conditions. However, because most of the impoverished both lacked the money to rebel and felt a racial leadership over blacks, they rarely rebelled against their conditions.}}
    \cornell[Slavery: The Peculiar Institution]{In what ways was American slavery unique?}{\textbf{American slaves, with conditions varying greatly by plantation size and crop, \underline{usually} received sufficient care to remain somewhat healthy, a direct contrast with the Caribbean. Furthermore, America saw many urban slaves, a rare sight which meant felt meshed poorly into society; free blacks, too, despite mostly living in poor, segregated conditions, were numerous, but generally preferred their conditions to slavery. Poor treatment was rampant, however, particularly in the domestic and foreign slave trades, dehumanizing behaviors; this stimulated rebellion, sometimes with direct violence but usually with sabotage or unexpected behavior.}}
    \cornell{What were the central forms of slavery?}{\begin{itemize}
        \item Slavery based around legal slave codes
        \begin{itemize}
            \item Forbade owning property, leaving without permission or after dark, meeting w/ other slaves (churches only), carrying firearms, attacking a white person (even for defense)
            \item Owners generally allowed to kill slaves w/o ramification; slaves generally killed for resisting a white person or causing revolt
            \item Anyone with even a trace of African American ancestry considered a slave (very hard to prove otherwise)
        \end{itemize}
        \item Laws often not rigid: some owned property, were educated; owners generally interpreted laws $\to$ great mix with some harsher and some more flexible
        \item Majority of small farm owners worked alongside small groups of slaves, generally intimate and often affectionate; sometimes tyrannical
        \begin{itemize}
            \item Always based on relationship of subordinacy
        \end{itemize}
        \item Majority of slaves on larger plantations w/ overseers, assistant overseers; leader slaves known as "head drivers" assisted by "subdrivers"
        \begin{itemize}
            \item Slave labor assigned either by \textbf{task system}: slaves assigned task in morning, free after work done (rice); or by \textbf{gang system}: divided into groups, worked all day long under rule of driver (cotton, sugar, tobacco)
        \end{itemize}
    \end{itemize}
    \textbf{Slavery was based fundamentally around strict laws limiting slaves' rights; however, owners rarely followed these directly, generally implementing their own policies, some stricter and some more flexible than the existing laws. Small farms, unlike the larger, more tightly run plantations with leadership hierarchies, generally saw slave owners working alongside their slaves with intimate relationships.}}
    \cornell{What were the living conditions of most slaves?}{\begin{itemize}
        \item Slaves generally received adequate food and housing to cultivate personal gardens, medical care from "plantation mistresses" but usually slave women focused did majority 
        \item Working conditions grueling: slowly built up from childhood, with harvesttime most challenging
        \begin{itemize}
            \item Hard work generally banned until adolescence to create more loyal, healthy slaves
        \end{itemize} 
        \item Women struggled most: often single parents due to separation of father and family, labored in fields \textit{and} worked in home
        \item Significantly less healthy: after slave trade banned in 1808, proportion of blacks to whites shrunk w/ few living to adulthood
        \begin{itemize}
            \item Material conditions often better than North factory workers and European peasants
            \item Less severe than Caribbean w/ less labor, slave trade legal -> less care for long-term health of slaves due to easily replenishment (in U.S., grew through \underline{natural reproduction})
        \end{itemize}
        \item Hired labor often used for most grueling tasks: paid Irish worker far cheaper than new slave
        \item On larger plantations, dedicated servants enjoyed easier life, w/ domestic staff often forming familial bonds with white families
        \begin{itemize}
            \item Servants often despised greater distance from other slaves; house servants often first to leave w/ emancipation 
            \item Female household servants often pressured into relationships w/ masters but often abused; white women often jealous $\to$ punished them
        \end{itemize}
    \end{itemize}
    \textbf{Most slaves were fed enough to remain relatively healthy and were given sufficient land to live in a small cabin and tend to a personal garden. Women struggled the most due to both domestic and field work; they were often pressured into sexual relations with their male masters and then punished by envious white women. However, slaves were generally devoted to creating a safe work environment for slaves, often using hired labor for the most dangerous tasks and prevent young children from working hard. Compared to those of the Caribbean, American slaves were far healthier and happier.}}
    \cornell{How did urban slavery differ from plantation slavery?}{\begin{itemize}
        \item Urban slaves often travelled freely throughout city for errands during the day, encountering free blacks; generally kept in guarded barracks at night -> less distinct line between free and enslaved
        \item Lack of Euro. immigrants -> South created market for laborers: even poorest whites generally above such labor forms
        \begin{itemize}
            \item Slaves often mined, constructed, worked on docks, at textile mills; some skilled blacksmiths/artisans
        \end{itemize}
        \item Many white southerners saw incompatibility in urban slavery -> growth of cities accompanied by decline in urban slavery w/ owners often selling to countryside due to fear of insurrection
        \begin{itemize}
            \item Led to more black women than men (selective selling); paired w/ more white men than women, many mixed-race mulattoes born 
            \item Direct \textbf{segregation} emerged within cities
        \end{itemize}
    \end{itemize}
    \textbf{Slaves working in cities were generally much freer, able to roam the city to perform errands during the day. The great demand for common labor was created by a lack of European immigrants and of white will to perform such basic tasks, leading to a significant urban slave population. However, many felt slavery was incompatible with urban life and slave numbers begna to decline, transferring to the countryside.}}
    \cornell{What were the conditions of free African Americans in the U.S.?}{\begin{itemize}
        \item Beginning of Civil War $\approx$ 250,000 free Afr. Americans w/ majority in VA/MD 
        \item Most were urban blacks who had worked at specialized tasks to earn money to buy freedom
        \begin{itemize}
            \item \textit{ex}: Elizabeth Keckley, slave woman who bought freedom for her/son by sewing; eventually became White House seamstress
            \item Masters generally unwilling to give up slaves
        \end{itemize}
        \item Some masters gave up slaves due to moral issues, either before or after death (in will)
        \item In response to revolts (notably Nat Turner's) of 1830s, state laws $\to$ more rigid w/ southerners fearing free blacks would spark future revolts
        \begin{itemize}
            \item Very difficult for owners to release/"manumit" slaves 
            \item Free Afr. Americans banned from entering southern states 
        \end{itemize}
        \item Some blacks (on northern fringes) became wealthy, sometimes owning slaves themselves (purchased relatives for emancipation)
        \begin{itemize}
            \item Some cities like Charleston, New Orleans, Natchez enjoyed economic stability for black communities
        \end{itemize}
        \item Most free blacks lived in complete poverty w/ conditions worse than in North; required white supervision for most congregations $\to$ not yet truly free 
        \begin{itemize}
            \item Still preferred to slavery
        \end{itemize}
    \end{itemize}
    \textbf{Free African Americans grew in number as slaves worked to buy their emancipation or masters chose to free them due to moral issues. However, in response to numerous revolts, 1830 laws tightened emancipation, making it challenging for slaves to be released. Although some northern African Americans became wealthy, most remained in segregated communities in poverty, but most agreed their conditions were superior to those of slavery.}}
    \cornell{How did slave trading continue both inside and outside the U.S.?}{\begin{itemize}
        \item Slaves migrated sometimes w/ owners moving to better lands; usually w/ professional traders travelling in large groups by foot to be sold in cities
        \item Domestic slave trade dehumanized slaves, separating families, particularly after the death of an owner
        \begin{itemize}
            \item Planters generally did not agree w/ it but took little action
        \end{itemize}
        \item Foreign slave trade disorganized due to illegal nature post-1808 w/ some running out of slaves by 1850s, seeking new purchases; govt. never approved
        \item Slaves resisted smuggling significantly 
        \begin{itemize}
            \item \textit{ex}: 1839, 53 slaves of Cuba took over ship \textit{Amistad} in hopes of returning to Afr., but reached U.S. coast
            \begin{itemize}
                \item Van Buren sought ship to return to Cuba (along w/ many Americans)
                \item JQA argued foreign slave trade illegal $\to$ had to be freed; approved by Supreme Court
            \end{itemize}
            \item \textit{ex}: 1841, American ship from Norfolk to New Orleans taken over, brought to British Bahamas w/ illegal slavery
            \begin{itemize}
                \item Given sanctuary, refuge
            \end{itemize}
        \end{itemize}
    \end{itemize}
    \textbf{The domestic slave trade, led by professional slave traders, transported slaves throughout the South to be sold in dehumanizing conditions, separated from their loved ones. The foreign slave trade, despite being illegal, continued in a disorganized manner. Slaves occasionally took over their ships and were often successful in earning their freedom.}}
    \cornell{How did slaves resist their conditions?}{\begin{itemize}
        \item \underline{Majority of slaves unhappy w/ conditions}, evidenced in joyful reaction to emancipation
        \item Slaves employed both adaptation and resistance
        \begin{itemize}
            \item Some took on stereotypical role, known as "Sambo," of unintellegent, head-scratching, grinning person to resist owners
            \item Others rebelled directly;
        \end{itemize}
        \item Although few physical rebellions emerged, many owners greatly fearful
        \begin{itemize}
            \item \textit{ex}: 1800 - Gabriel Prosser organized 1000 slaves but 2 revealed plot -> VA militia stifled
            \item \textit{ex}: 1822 - Charleston free black Denmark Vesey gathered a rumored 9000 followers but word leaked
            \item \textit{ex}: 1831 - Nat Turner, preacher, gathered band w/ guns and axes to attack white men/women/children in houses at night (60 killed)
            \begin{itemize}
                \item State/federal troops put down, executing 100 blacks 
                \item Only large-scale insurrection
            \end{itemize}
        \end{itemize}
        \item Other forms of rebellion far less confrontational
        \begin{itemize}
            \item Some ran away, w/ some travelling to North/Canada (through underground railroad arranged by whites)
            \begin{itemize}
                \item Probability unlikely from Deep South due to long distance, lack of geographical knowledge, "slave patrols" demanding permits from blacks and sending bloodhounds to find slaves
                \item Often successful in large numbers despite great risk of whipping
            \end{itemize}
            \item Change in everyday behavior like refusal to try hard 
            \item Stole from masters, sabotaged tools, incorrectly completed tasks
        \end{itemize}
    \end{itemize}
    \textbf{Almost all slaves were unhappy with their living conditions; some rebelled directly, like in Nat Turner's rebellion, led by a slave preacher and killing 60 whites; most, however, were less confrontational, including escape, robbery, or deliberately working poorly, taking on the "Sambo" stereotype of unintelligence.}}
    \end{document}