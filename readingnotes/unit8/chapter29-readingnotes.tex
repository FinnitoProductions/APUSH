\documentclass[a4paper]{article}
    \input{../notesheader.tex}
    \usepackage[normalem]{ulem}

    \newcommand{\chapternumber}{29}
    \newcommand{\chaptertitle}{Civil Rights, Vietnam, and the Ordeal of Liberalism}
    \title{\vspace{-3em}
    \begin{tcolorbox}
    \Huge\sffamily \begin{center} AP US History  \\
    \LARGE Chapter \chapternumber \, - \chaptertitle \\
    \Large Finn Frankis \end{center} 
    \end{tcolorbox}
    \vspace{-3em}
    }
    \date{}
    \author{}
    
    \begin{document}
        \maketitle
        \SetBgContents{\rule[0em]{4pt}{\textheight}}
        \cornell[Key Concepts]{What are this chapter's key concepts?}{\begin{itemize}
            \item \textbf{8.1.I.B} - U.S., fearing Communist/Soviet ideology, explored new means of containing communism, notably major military battles like Vietnam/Korea
            \item \textbf{8.1.I.C} - Cold War alternated betw. direct/indirect mil. confrontation and mutual coexistence (détente)
            \item \textbf{8.1.II.B} - Vietnam War $\to$ often violent antiwar protests as war escalated, marking first widespread resistance to anticommunist policies
            \item \textbf{8.2.I.B} - Three branches of fed. govt. partook in desegregation of armed services, \textit{Brown v. Board of Education}, Civil Rights Act of 1964 to promote racial equality
            \item \textbf{8.2.III.A} - Liberalism reached peak of influence by 1960s
            \item \textbf{8.2.III.B} - Lyndon Johnson's Great Society used fed. programs to end discrimination, poverty, address social issues; Supreme Court cases $\to$ $\uparrow$ civil rights, individual liberties
        \end{itemize}}
        \cornell[Expanding the Liberal State]{How did John F. Kennedy and Lyndon B. Johnson expand liberal desires?}{\textbf{President John F. Kennedy, seeking to pass his "New Frontier" plans, was soon replaced by VP Lyndon B. Johnson, with his "Great Society" plans after his assassination. Kennedy and Johnson supported medical care with Medicare for the elderly and Medicaid for those on welfare and the impoverished; they also supported urban housing and infrastructure development, giving funds to schools with particularly underresourced students, and limiting discrimination in immigration laws. The New Frontier and the Great Society, despite reducing hunger, widening medical aid, and bringing countless Americans above the poverty line, were denounced for their extremely high costs.}}
        \cornell{Who was John F.  Kennedy?}{\begin{itemize}
            \item 1960: Republican \textbf{Richard Nixon} faced off against Democrat \textbf{John Fitzgerald Kennedy} (slightly more uneasily)
            \item Kennedy was son of controversial Joseph Kennedy, U.S. ambassador to GB; despite privileged life, advocated for personal sacrifice, progress
            \item Despite Kennedy's young age, Catholicism, ultimately won against Nixon in tight race
            \item Kennedy had promised domestic reforms like New Deal, called \textbf{New Frontier}
            \begin{itemize}
                \item Very thin mandate, Congress dominated by Republicans and conservative Democrats $\to$ struggled to do more than tariff reduction
            \end{itemize}
            \item Made personality part of presidency; in Nov. 22, 1963, Kennedy shot in Dallas by Marxist \textbf{Lee Harvey Oswald}, who was then shot by Jack Ruby (some conspiraces that they worked together)
        \end{itemize}
        \textbf{John F. Kennedy won the 1960 election as a Catholic president with a tight mandate. He promised "New Frontier" reforms resembling those of the New Deal but faced a very hesitant Congress, only successfully a passing a tax cut. His assassination reverberated throughout the nation.}}
        \cornell{Who was Lyndon B. Johnson?}{\begin{itemize}
            \item \textbf{Lyndon Baines Johnson}, having grown up in poor region of Texas, known for ambition/dedication
            \item Counterbalance to Kennedy but believed in active use of power: from 1963-1966, relied on emotion over Kennedy's death to successfully pass several New Frontier reforms 
            \begin{itemize}
                \item Created \textbf{Great Society} reforms of his own; lobbied in Congress to get them passed
            \end{itemize}
            \item Saw himself as "coalition builder," focusing on the support of \underline{everyone}; successful for some time 
            \item Spent first year focusing on reelection: easily won against extremely conservative \textbf{Barry Goldwater} 
        \end{itemize}
        \textbf{Kennedy was succeeded by Lyndon Baines Johnson, a man known for ambition. He successfully passed several of Kennedy's New Frontier reforms and he passed many of his own Great Society Reforms. He hoped he could please everyone, but put his legislation on hold to win reelection, which he did with ease.}}
        \cornell{How did Lyndon B. Johnson implement poverty reforms?}{\begin{itemize}
            \item Created \textbf{Medicare} program for elderly health expenses; finally passed in 1965 after twenty-year debate 
            \begin{itemize}
                \item Not truly "welfare": open to all elderly Americans regardless of need 
                \item Doctors serving Medicare patients were free to operate privately, charge their own fees; government responsible for payment
                \item 1966: Johnson extended to \textbf{Medicaid} to support those on welfare as well as the poor
            \end{itemize}
            \item Represented one step in bigger war against poverty: centerpiece was \textbf{Office of Economic Opportunity}, creating educational/employment/housing/health-care programs; controversial for "\textbf{Community Action}"
            \begin{itemize}
                \item Community Action sought to give poor people agency in planning programs: gave them new jobs, taught them political work
                \item Included black/Hispanic/native populations, many of whom moved on to political careers
                \item Community Action impossible to sustain: administrative failures as well as poor decisions by a few making war on poverty as a whole seem futile
                \item OEO spent \$3b, eliminating poverty in some regions; weaknesses of programs and funding for them $\to$ unable to eliminate poverty entirely
            \end{itemize}
        \end{itemize}
        \textbf{Johnson passed the Medicare program, offering health insurance to all elderly members of American society; soon, he created Medicaid, extending medical service to the poor as well as those on welfare. In his goal to eliminate poverty, Johnson also created the OEO, based on Community Action, where the poor themselves would take jobs to create careful reform programs. Ultimately, however, the OEO failed, for its weak programs and limited funding prevented it from making a big splash.}}
        \cornell{How did the federal government support the development of cities, schools, and immigration?}{\begin{itemize}
            \item \textbf{Housing Act of 1961}: \$4.9b to cities to preserve open spaces, develop transit system, subsidize middle-class homes 
            \item 1966: Johnson created \textbf{Department of Housing and Urban Development} led by Afr. American Robert Weaver; Johnson also created \textbf{Model Cities} to provide federal subsidies to pilot city programs
            \item Kennedy wanted to give federal aid to public schools; failed bc. several feared federal control of schools, Catholics pushed the assistance had to include religious schools, too
            \begin{itemize}
                \item Kennedy circumvented w/ \textbf{Elementary and Secondary Education Act of 1965}, supporting all schoools based on the wealth of the students
            \end{itemize}
            \item Johnson supported \textbf{Immigration Act of 1965}, which placed limit on total newcomers to the nation but also got rid of "national origins" system of 1920s preferencing northern European over other immigrants 
            \begin{itemize}
                \item Still restricted some Latin American immigration; allowed all people from Asia, Europe, and Africa 
                \item Large groups, notably Asians, entered in the 1970s
            \end{itemize}
        \end{itemize}
        \textbf{The Housing Act of 1961 aimed to support cities with direct subasidies for open space, transit systems, and homes. Johnson's Department of Housing and Urban Development as well as Model Cities also poured significant funds into city infrastructure. Kennedy successfully passed the Elementary and Secondary Education Act, supporting all schools (both Catholic and Protestant) and providing aid based on student wealth. Finally, though Johnson's Immigration Act of 1965 set a total, it did not discriminate by region of origin (apart from Latin American nations).}}
        \cornell{What was the legacy of Johnson's "Great Society"?}{\begin{itemize}
            \item Represented $\uparrow$ fed. spending; initially supported by tax revenues until Kennedy's tax cut passed by Johnson $\to$ federal deficit
            \begin{itemize}
                \item Program expansion $\to$ competition with U.S. military ventures for funds; federal spending rapidly increased 
            \end{itemize}
            \item Cost of Great Society, inability of govt. to find money to spend $\to$ society disillusioned w/ federal attempts to mend social issues; most believed to be ineffective
            \begin{itemize}
                \item Did reduce American hunger, expand medical care to those previously unable to reach 
                \item Brought many above poverty line, w/ blacks and whites affected equally
            \end{itemize}
        \end{itemize}
        \textbf{The Great Society required large amounts of funds, which, after a tax cut was passed, caused a federal deficit which competed with military spending. Most deemed the Great Society a failure due to its costs; however, its ability to reduce hunger, expand medical care, and bring several American above the poverty line cannot be ignored.}}
    \end{document}