\documentclass[a4paper]{article}
    \usepackage[T1]{fontenc}
    \usepackage{tcolorbox}
    \usepackage{amsmath}
    \tcbuselibrary{skins}
    
    \usepackage{background}
    \SetBgScale{1}
    \SetBgAngle{0}
    \SetBgColor{red}
    \SetBgContents{\rule[0em]{4pt}{\textheight}}
    \SetBgHshift{-2.3cm}
    \SetBgVshift{0cm}
    \usepackage[margin=2cm]{geometry} 
    
    \makeatletter
    \def\cornell{\@ifnextchar[{\@with}{\@without}}
    \def\@with[#1]#2#3{
    \begin{tcolorbox}[enhanced,colback=gray,colframe=black,fonttitle=\large\bfseries\sffamily,sidebyside=true, nobeforeafter,before=\vfil,after=\vfil,colupper=blue,sidebyside align=top, lefthand width=.3\textwidth,
    opacityframe=0,opacityback=.3,opacitybacktitle=1, opacitytext=1,
    segmentation style={black!55,solid,opacity=0,line width=3pt},
    title=#1
    ]
    \begin{tcolorbox}[colback=red!05,colframe=red!25,sidebyside align=top,
    width=\textwidth,nobeforeafter]#2\end{tcolorbox}%
    \tcblower
    \sffamily
    \begin{tcolorbox}[colback=blue!05,colframe=blue!10,width=\textwidth,nobeforeafter]
    #3
    \end{tcolorbox}
    \end{tcolorbox}
    }
    \def\@without#1#2{
    \begin{tcolorbox}[enhanced,colback=white!15,colframe=white,fonttitle=\bfseries,sidebyside=true, nobeforeafter,before=\vfil,after=\vfil,colupper=blue,sidebyside align=top, lefthand width=.3\textwidth,
    opacityframe=0,opacityback=0,opacitybacktitle=0, opacitytext=1,
    segmentation style={black!55,solid,opacity=0,line width=3pt}
    ]
    
    \begin{tcolorbox}[colback=red!05,colframe=red!25,sidebyside align=top,
    width=\textwidth,nobeforeafter]#1\end{tcolorbox}%
    \tcblower
    \sffamily
    \begin{tcolorbox}[colback=blue!05,colframe=blue!10,width=\textwidth,nobeforeafter]
    #2
    \end{tcolorbox}
    \end{tcolorbox}
    }
    \makeatother

    \parindent=0pt
    \usepackage[normalem]{ulem}

    \newcommand{\chapternumber}{29}
    \newcommand{\chaptertitle}{Civil Rights, Vietnam, and the Ordeal of Liberalism}
    \title{\vspace{-3em}
    \begin{tcolorbox}
    \Huge\sffamily \begin{center} AP US History  \\
    \LARGE Chapter \chapternumber \, - \chaptertitle \\
    \Large Finn Frankis \end{center} 
    \end{tcolorbox}
    \vspace{-3em}
    }
    \date{}
    \author{}
    
    \begin{document}
        \maketitle
        \SetBgContents{\rule[0em]{4pt}{\textheight}}
        \cornell[Key Concepts]{What are this chapter's key concepts?}{\begin{itemize}
            \item \textbf{8.1.I.B} - U.S., fearing Communist/Soviet ideology, explored new means of containing communism, notably major military battles like Vietnam/Korea
            \item \textbf{8.1.I.C} - Cold War alternated betw. direct/indirect mil. confrontation and mutual coexistence (détente)
            \item \textbf{8.1.II.B} - Vietnam War $\to$ often violent antiwar protests as war escalated, marking first widespread resistance to anticommunist policies
            \item \textbf{8.2.I.B} - Three branches of fed. govt. partook in desegregation of armed services, \textit{Brown v. Board of Education}, Civil Rights Act of 1964 to promote racial equality
            \item \textbf{8.2.III.A} - Liberalism reached peak of influence by 1960s
            \item \textbf{8.2.III.B} - Lyndon Johnson's Great Society used fed. programs to end discrimination, poverty, address social issues; Supreme Court cases $\to$ $\uparrow$ civil rights, individual liberties
        \end{itemize}}
        \cornell[Expanding the Liberal State]{How did John F. Kennedy and Lyndon B. Johnson expand liberal desires?}{\textbf{President John F. Kennedy, seeking to pass his "New Frontier" plans, was soon replaced by VP Lyndon B. Johnson, with his "Great Society" plans after his assassination. Kennedy and Johnson supported medical care with Medicare for the elderly and Medicaid for those on welfare and the impoverished; they also supported urban housing and infrastructure development, giving funds to schools with particularly underresourced students, and limiting discrimination in immigration laws. The New Frontier and the Great Society, despite reducing hunger, widening medical aid, and bringing countless Americans above the poverty line, were denounced for their extremely high costs.}}
        \cornell{Who was John F.  Kennedy?}{\begin{itemize}
            \item 1960: Republican \textbf{Richard Nixon} faced off against Democrat \textbf{John Fitzgerald Kennedy} (slightly more uneasily)
            \item Kennedy was son of controversial Joseph Kennedy, U.S. ambassador to GB; despite privileged life, advocated for personal sacrifice, progress
            \item Despite Kennedy's young age, Catholicism, ultimately won against Nixon in tight race
            \item Kennedy had promised domestic reforms like New Deal, called \textbf{New Frontier}
            \begin{itemize}
                \item Very thin mandate, Congress dominated by Republicans and conservative Democrats $\to$ struggled to do more than tariff reduction
            \end{itemize}
            \item Made personality part of presidency; in Nov. 22, 1963, Kennedy shot in Dallas by Marxist \textbf{Lee Harvey Oswald}, who was then shot by Jack Ruby (some conspiraces that they worked together)
        \end{itemize}
        \textbf{John F. Kennedy won the 1960 election as a Catholic president with a tight mandate. He promised "New Frontier" reforms resembling those of the New Deal but faced a very hesitant Congress, only successfully a passing a tax cut. His assassination reverberated throughout the nation.}}
        \cornell{Who was Lyndon B. Johnson?}{\begin{itemize}
            \item \textbf{Lyndon Baines Johnson}, having grown up in poor region of Texas, known for ambition/dedication
            \item Counterbalance to Kennedy but believed in active use of power: from 1963-1966, relied on emotion over Kennedy's death to successfully pass several New Frontier reforms 
            \begin{itemize}
                \item Created \textbf{Great Society} reforms of his own; lobbied in Congress to get them passed
            \end{itemize}
            \item Saw himself as "coalition builder," focusing on the support of \underline{everyone}; successful for some time 
            \item Spent first year focusing on reelection: easily won against extremely conservative \textbf{Barry Goldwater} 
        \end{itemize}
        \textbf{Kennedy was succeeded by Lyndon Baines Johnson, a man known for ambition. He successfully passed several of Kennedy's New Frontier reforms and he passed many of his own Great Society Reforms. He hoped he could please everyone, but put his legislation on hold to win reelection, which he did with ease.}}
        \cornell{How did Lyndon B. Johnson implement poverty reforms?}{\begin{itemize}
            \item Created \textbf{Medicare} program for elderly health expenses; finally passed in 1965 after twenty-year debate 
            \begin{itemize}
                \item Not truly "welfare": open to all elderly Americans regardless of need 
                \item Doctors serving Medicare patients were free to operate privately, charge their own fees; government responsible for payment
                \item 1966: Johnson extended to \textbf{Medicaid} to support those on welfare as well as the poor
            \end{itemize}
            \item Represented one step in bigger war against poverty: centerpiece was \textbf{Office of Economic Opportunity}, creating educational/employment/housing/health-care programs; controversial for "\textbf{Community Action}"
            \begin{itemize}
                \item Community Action sought to give poor people agency in planning programs: gave them new jobs, taught them political work
                \item Included black/Hispanic/native populations, many of whom moved on to political careers
                \item Community Action impossible to sustain: administrative failures as well as poor decisions by a few making war on poverty as a whole seem futile
                \item OEO spent \$3b, eliminating poverty in some regions; weaknesses of programs and funding for them $\to$ unable to eliminate poverty entirely
            \end{itemize}
        \end{itemize}
        \textbf{Johnson passed the Medicare program, offering health insurance to all elderly members of American society; soon, he created Medicaid, extending medical service to the poor as well as those on welfare. In his goal to eliminate poverty, Johnson also created the OEO, based on Community Action, where the poor themselves would take jobs to create careful reform programs. Ultimately, however, the OEO failed, for its weak programs and limited funding prevented it from making a big splash.}}
        \cornell{How did the federal government support the development of cities, schools, and immigration?}{\begin{itemize}
            \item \textbf{Housing Act of 1961}: \$4.9b to cities to preserve open spaces, develop transit system, subsidize middle-class homes 
            \item 1966: Johnson created \textbf{Department of Housing and Urban Development} led by Afr. American Robert Weaver; Johnson also created \textbf{Model Cities} to provide federal subsidies to pilot city programs
            \item Kennedy wanted to give federal aid to public schools; failed bc. several feared federal control of schools, Catholics pushed the assistance had to include religious schools, too
            \begin{itemize}
                \item Kennedy circumvented w/ \textbf{Elementary and Secondary Education Act of 1965}, supporting all schoools based on the wealth of the students
            \end{itemize}
            \item Johnson supported \textbf{Immigration Act of 1965}, which placed limit on total newcomers to the nation but also got rid of "national origins" system of 1920s preferencing northern European over other immigrants 
            \begin{itemize}
                \item Still restricted some Latin American immigration; allowed all people from Asia, Europe, and Africa 
                \item Large groups, notably Asians, entered in the 1970s
            \end{itemize}
        \end{itemize}
        \textbf{The Housing Act of 1961 aimed to support cities with direct subasidies for open space, transit systems, and homes. Johnson's Department of Housing and Urban Development as well as Model Cities also poured significant funds into city infrastructure. Kennedy successfully passed the Elementary and Secondary Education Act, supporting all schools (both Catholic and Protestant) and providing aid based on student wealth. Finally, though Johnson's Immigration Act of 1965 set a total, it did not discriminate by region of origin (apart from Latin American nations).}}
        \cornell{What was the legacy of Johnson's "Great Society"?}{\begin{itemize}
            \item Represented $\uparrow$ fed. spending; initially supported by tax revenues until Kennedy's tax cut passed by Johnson $\to$ federal deficit
            \begin{itemize}
                \item Program expansion $\to$ competition with U.S. military ventures for funds; federal spending rapidly increased 
            \end{itemize}
            \item Cost of Great Society, inability of govt. to find money to spend $\to$ society disillusioned w/ federal attempts to mend social issues; most believed to be ineffective
            \begin{itemize}
                \item Did reduce American hunger, expand medical care to those previously unable to reach 
                \item Brought many above poverty line, w/ blacks and whites affected equally
            \end{itemize}
        \end{itemize}
        \textbf{The Great Society required large amounts of funds, which, after a tax cut was passed, caused a federal deficit which competed with military spending. Most deemed the Great Society a failure due to its costs; however, its ability to reduce hunger, expand medical care, and bring several American above the poverty line cannot be ignored.}}
        \cornell[The Battle for Racial Equality]{How did the civil rights movement progress in the 1960s?}{\textbf{Kennedy, though initially mildly commmitting himself to civil rights, soon committed on a national level after several successful acts of resistance by the civil rights movement, like lunch sit-ins, freedom rides, and nonviolent demonstrations in Birmingham. 1964 saw a great surge in the efforts of the civil right movement to expand the African American right to vote; Johnson passed the Voting Rights Act to provide federal protection to black voters. As African Americans migrated increasingly to cities, the civil rights movement changed in structure with time; urban violence rapidly expanded, too, typically against white oppression. The concept of black power began to emerge, urging not complete cooperation but instead appreciation of black individuality and heritage. Leaders like Martin Luther King, known for complete non-violence, were in stark contrast to those like Malcolm X, pushing for violence when necessary for defense.}}
        \cornell{How did protests for African American rights expand?}{\begin{itemize}
            \item Kennedy supported racial equality but never strongly committed due to fear of alienating southern Democrats 
            \begin{itemize}
                \item Sought mild reform by supporting existing laws, helping litigation aiming to overturn segregation statutes
            \end{itemize}
            \item Feb. 1960: black college students in Greensboro, NC staged sit-in at segregated lunch counter, later forming \textbf{Student Nonviolent Coordinating Committee} (SNCC)
            \item 1961: interracial students began \textbf{freedom rides} trying to desegregate bus stations by travelling throughout South on buses
            \begin{itemize}
                \item Whites often retaliated with extreme violence $\to$ Kennedy sent federal marshals to preserve peace while ordering integration of bus/train stations
                \item SNCC began pushing rural Americans to take action against Jim Crow laws
            \end{itemize}
            \item \textbf{Southern Christian Leadership Conference} (SCLC) created citizen education programs, grassroots movement to mobilize farmers, housewives, workers
            \item Continued push in courts to integrate public education w/ federal court ordering U of Mississippi to enroll black student, James Meredith, but governor refused, whites revolted $\to$ Kennedy sent troops to enforce court decision
            \item 1963: \textbf{Martin Luther King Jr.} began nonviolent demonstrations in Birmingham, AL,known for immense segregation
            \begin{itemize}
                \item Marches often broken up with arrests, attack dogs, tear gas, fire hoses w/ nation \underline{watching televised reports}
                \item Governor George Wallace, having won due to promise to resist integration, stood in doorway of University of Alabama building to prevent black students from entering; Robert Kennedy met w/ him, sent federal marshals to change mind
            \end{itemize}
        \end{itemize}
        \textbf{Kennedy initially sought mild reform to promote racial equality; however, he was unable to ignore the significant progress made by the civil rights movement. From staged sit-ins at lunch counters to "freedom rides" to integrate bus stations to judicial reform to allow blacks to attend universities in full to MLK's nonviolent Birmingham demonstrations, the civil rights movement was seeing gradual reform as a result of their efforts.}}
        \cornell{How did Kennedy commit to civil rights on a national level?}{\begin{itemize}
            \item Kennedy realized issue of race no longer avoidable, thus nationally committed to reforming civil rights
            \begin{itemize}
                \item Produced several proposals banning segregation in public areas, banning discrimination in employment, increasing govt. power for education reform
            \end{itemize}
            \item 200k demonstrators marched down Washington Mall in \textbf{March on Washington}, gathering by Lincoln Memorial for MLK to give powerful \textbf{"I have a dream" speech} w/ Kennedy's support
            \item Kennedy assassinated $\to$ new push for civil rights reform w/ Senate ultimately passing comprehensive civil rights reform in 1964
        \end{itemize}
        \textbf{Kennedy commmitted nationally to the issue of civil rights, creating several proposals against segregation and discrimination. Soon after, 200,000 demonstrators marched in Washington, led by MLK, to support Kennedy's legislation. Kennedy's assassination gave his civil rights reform the final push to ultimately be passed.}}
        \cornell{How did African Americans push to exercise their voting rights?}{\begin{itemize}
            \item Summer of 1964: \textbf{freedom summer} w/ civil rights workers travelling south (primarily Mississippi) to register blacks to vote
            \item Met w/ southern opposition: first two freedom workers murdered by KKK with support of local police
            \item Freedom summer resulted in \textbf{Mississippi Freedom Democratic Party} hoping to challenge existing party for seats at convention
            \begin{itemize}
                \item Johnson formed agreement w/ King to allow MFDP to exist as observers to reform later, but to ensure regular state party would remain official $\to$ anger on both sides
            \end{itemize}
            \item March 1965: King organized demonstrations in Selma, AL to push for right of blacks to vote 
            \begin{itemize}
                \item Selma sheriff attacked demonstrators w/ television coverage $\to$ brutality showcased to nation 
                \item Two northern whites supporting march were killed
                \item Outrage from Selma allowed LBJ to pass \textbf{Voting Rights Act}, providing active federal protection to blacks seeking the vote
            \end{itemize}
        \end{itemize}
        \textbf{In 1964 and early 1965, significant progress was made in black voting rights. In the summer of 1964, civil rights workers pushed blacks to register to vote despite great Southern opposition. The final impetus for Johnson's successful Voting Rights Act was King's Selma demonstration, where the sheriff's brutality provoked the nation and generated support for the right to vote.}}
        \cornell{How did the civil rights movement change in structure with time?}{\begin{itemize}
            \item Civil rights movement reflected demographic shift of Afr. Americans into cities, where poverty became even more aggravated with time
            \item School desegregation moved from attack against \textbf{de jure segregation} (based on legal practices) to one against \textbf{de facto segregation} (by practice, primarily due to residential patterns)
            \item Afr. Americans demanded not only no discrimination in jobs but also positive efforts to recruit minorities $\to$ Johnson's \textbf{affirmative action} extending to all businesses economically connected to fed. govt. 
            \item Chicago campaign in 1966 reflected new character of movement in attempt to highlight injustice of cities; unable to arouse national conscience and also evoked extremely violent responses
        \end{itemize}
        \textbf{As African Americans moved to cities, the civil rights movement began to focus more on economic factors, like preventing job discrimination by actively hiring African Americans and limiting de facto (by practice) school segregation. However, their campaign in Chicago in 1966 reflected that the former techniques were no longer effective.}}
        \cornell{What were the most significant incidences of urban race violence?}{\begin{itemize}
            \item Harlem saw some scattered discordance; most significant in \textbf{Watts} neighborhood of Los Angeles after black protestor struck by police club $\to$ 10k Afr. Americans partook in great violence
            \begin{itemize}
                \item Attacked white drivers, looted stores, attacked policemen
                \item 34 people died, 28 of whom were black; ultimately quelled by National Guard 
            \end{itemize}
            \item More outbreaks in Chicago, Cleveland, Detroit 
            \item Violence gave more urgency to civil rights movement
            \begin{itemize}
                \item Johnson's \textbf{Commision on Civil Disorders} issued report recommending spending to reduce ghetto conditions; believed in national committment
            \end{itemize}
        \end{itemize}
        \textbf{The Watts, Los Angeles riots reflected significant urban race violence; more in Chicago, Cleveland, and Detroit gave even further urgency to the civil rights movement and the pouring of money into ghettoes.}}
        \cornell{How did African Americans support the ideology of black power?}{\begin{itemize}
            \item Black power believed in moving away from cooperation betw. races, instead recognizing distinctiveness by encouraging racial pride in distinct way to whites (like in art, African names, hairstyles)
            \item Black power $\to$ division w/in civil rights movement: traditional organizations like the NAACP, Urban League, SCLC at odds with more radical SNCC and Congress of Racial Equality, w/ these groups beginning to advocate violence
            \item Some movements outside of mainstream movement, like Oakland, CA \textbf{Black Panther Party}, known for open violence and displays of weapons to show their fight for justice
        \end{itemize}
        \textbf{Black power, representing an acceptance of differences between blacks and whites as well as a great appreciation of black culture and African American pride, created a significant division between the traditional civil rights groups promoting cooperation and more radical groups like the SNCC and the Black Panther Party, pushing for direct violence.}}
        \cornell{Who was Malcolm X?}{\begin{itemize}
            \item Detroit's \textbf{Nation of Islam}, teaching strict, moral behavior for Afr. Americans, began to gain increased prominence 
            \item Most celebrated: Malcolm Little, calling himself \textbf{Malcolm X} to indicate lost African surname
            \begin{itemize}
                \item Malcolm known for oratory, harsh attacks against racism; despite not advocating violence, did push for black right to defend themselves
                \item Assassinated in 1965 by black gunmen likely part of rival faction of Nation of Islam 
            \end{itemize}
            \item Autobiography widely read after death; revered as symbol on par w/ MLK
        \end{itemize}
        \textbf{Malcolm X, from Detroit's strict Nation of Islam, pushed for black rights through his oratory and harsh attacks against racism. He did not directly advocate violence but instead pushed for the black right to defend. He was assassinated in 1965, but his legacy lived on through his autobiography.}}
        \cornell["Flexible Response" and the Cold War]{How was Kennedy's response to the Cold War "flexible"?}{\textbf{Kennedy made some expansions to guerrila warfare against communism, but he also created organizations for peaceful expansion. The Bay of Pigs disaster, a failed U.S.-driven revolt against Castro's Cuban government, Khrushchev's construction of the Berlin Wall between East and West Berlin, the Cuban Missile Crisis where the U.S. prepared to respond with force to the USSR's construction of missiles in Cuba, as well as Johnson's suppression of a communist revolt in the Dominican Republic, all further amplified Cold War tensions.}}
        \cornell{How did Kennedy's administration diversify foreign policy?}{\begin{itemize}
            \item Kennedy believed in ways of fighting communism beyodnd simple atomic bomb $\to$ expanded army of "Green Berets" for guerilla warfare
            \item Some belief in peaceful expansion: proposed \textbf{Alliance for Progress} to rekindle relationship w/ Latin America, \textbf{Agency of International Development} for foreign aid, \textbf{Peace Corps} for humanitarian work
            \item \textbf{Bay of Pigs} disaster one of Kennedy's greatest failures
            \begin{itemize}
                \item Project to attack Castro government began under Johnson but Kennedy finalized w/ CIA 
                \item Army of trained anti-Castro exiles arrived at Cuba's Bay of Pigs, expecting U.S. air assistance as well as uprising from within Cuba: received neither w/ Kennedy realizing things were going poorly $\to$ pulled air support 
            \end{itemize}
        \end{itemize}
        \textbf{Kennedy, though expanding the guerilla warfare program to fight communism, also developed several organizations for peaceful foreign reconciliation. Among his greatest failures was the Bay of Pigs, an attack on Cuba's government; after a group of anti-Castro exiles sent by the U.S. failed to rouse popular support, Kennedy pulled all air assistance and left the exiles to collapse.}}
        \cornell{How did Kennedy come head to head with the USSR?}{\begin{itemize}
            \item After Bay of Pigs in 1961, Kennedy met w/ Khrushchev in frosty meeting: Khrushchev still wanted U.S. to abandon West Berlin
            \item Khrushchev discontent w/ rapid migration of East Germans into West Germany through border in center of Berlin $\to$ ordered construction of \textbf{Berlin Wall}
            \begin{itemize}
                \item All who tried to cross were attacked
                \item Represented physical border betw. communist and noncommunist worlds
            \end{itemize}
            \item October 14th, 1961: \textbf{Cuban Missile Crisis} after U.S. found evidence that Soviets were creating nucl. weapons in Cuba 
            \begin{itemize}
                \item Soviets likely saw as retaliation against U.S. construction of missiles in Turkey 
                \item Kennedy saw as aggression $\to$ issued naval/air blockade around Cuba; began plans for U.S. air attack on missile sites
                \item Oct. 26th: situation diffused after Khrushchev promised to remove missile bases in exchange for U.S. pledge not to invade
            \end{itemize}
        \end{itemize}
        \textbf{Relations remained frosty with the Soviet Union: Khrushchev was angered by the Bay of Pigs and the failure of the U.S. to abandon West Berlin; the construction of the Berlin Wall to prevent migration from East Germany to West Germany served as a physical barrier between the two ideologies. The 1961 Cuban Missile Crisis, after the U.S. determined that the USSR was building missiles in Cuba, nearly led to an air attack but was eventually diffused.}}
        \cornell{What were Johnson's strides in foreign policy?}{\begin{itemize}
            \item Johnson had limited experience w/ international affairs $\to$ sought to prove force
            \item Internal rebellion in Dominican Republic after dictator assassinated, four years of competing factions $\to$ one dominant group of nationalists under \textbf{Juan Bosch}
            \begin{itemize}
                \item Johnson argued that Bosch would create communist regime $\to$ sent U.S. troops to suppress revolution until Bosch lost election
            \end{itemize}
        \end{itemize}
        \textbf{Johnson, with limited experience in international affairs, made an initial splash in the Dominican Republic, when he sent U.S. troops to quell a seemingly communist revolt in the region. Most influential, however, was his work in Vietnam.}}
        \cornell[The Agony of Vietnam]{What led up to and what were the effects of the war in Vietnam?}{\textbf{The conflict in Vietnam started as a colonial battle between Ho Chi Minh's Vietnamese independence movement and French colonists. However, the U.S. quickly became intertwined, providing massive aid to the pro-Western South Vietnam; after the southern government weakened and collapsed to revolt, Johnson began to intervene directly. With 500,000 troops in Vietnam by the end of 1967, an all-out war had begun. Johnson authorized bombings of key North Vietnam centers; he also destroyed villages with North Vietnam supporters, forcing them out as refugees. The U.S. goal to win the war purely by firepower proved ineffective due to North Vietnam's devotion: significant opposition was generated on the home front from students and politicians due to the long war with seemingly little progress as well as the economic damage created by Johnson's attempt to continue his Great Society reforms without abandoning the war.}}
        \cornell{What were the effects of the First Indochina war?}{\begin{itemize}
            \item Vietnam known both as major power as well as subjugated part of China; became French colony in mid-nineteenth century and controlled by Japan during WWII
            \item After WWII, question how to handle Vietnam saw French seeking to reassert colonial dominance pitted against strong independence movement from within nation
            \begin{itemize}
                \item Nationalists organized into political party called the \textbf{Vietminh} led by communist \textbf{Ho Chi Minh}, who created independent nation w/ Hanoi govt. in 1945
                \item Truman heavily pressured to support French side for sake of domestic economy; prioritized revival of Western Europe $\to$ U.S. did not stop French incursion
            \end{itemize}
            \item First Indochina War starting in 1946 saw Truman/Eisenhower supporting French mil. campaign, paying up to 80\% of war costs
            \begin{itemize}
                \item Went badly for French despite U.S. support 
                \item Late in 1953: French troops cornered at \textbf{Dien Bien Phu} w/ horrible siege; U.S. chose not to aid French $\to$ war ultimately ended
            \end{itemize}
        \end{itemize}
        \textbf{Vietnam's powerful independence movement led by Ho Chi Minh was directly at odds with France's colonial desires. Truman, hoping to revive the Western European economy, provided economic support to the French in the First Indochina War, which ultimately resulted in French troops getting cornered and surrendering as the U.S. pulled support.}}
        \cornell{How did a Geneva conference aim to settle the Vietnam conflict?}{\textbf{At a Geneva conference planned for the Korean War, it was agreed that the Vietnam conflict was to end in cease-fire without U.S. participation. It agreed to split Vietnam into North Vietnam, led by the Vietminh, and South Vietnam, a pro-Western regime; elections would proceed in 1956.}}
        \cornell{How did the U.S. provide significant support to South Vietnam?}{\begin{itemize}
            \item Southern govt. led by \textbf{Ngo Diem}, aristocratic Catholic from central Vietnam but also Vietnamese nationalist w/o alliances w/ French 
            \item U.S. CIA helped Diem to attack South Vietnamese mafia; U.S. govt. saw Diem as powerful alternative to Ho Chi Minh
            \item U.S. govt. supported Diem's choice to refuse the agreed-upon 1956 elections because Ho Chi Minh would inevitably win due to northern support; U.S. continued to pour in economic funds
            \item Diem had been successfully suppressing resistant sects $\to$ turned to North Vietnamese dissidents, w/ conflict resuming
            \begin{itemize}
                \item \textbf{National Liberation Front} (NLF) responded to Diem's successful suppression of dissidents $\to$ w/ permission from Vietminh and mil./material support, began attacking govt. officials of South
                \item Destabilized Diem regime by killing several government officials, convincing more groups to abandon ties to him
            \end{itemize}
            \item Diem attempted to suppress Buddhism to make Catholicism dominant $\to$ Buddhist anti-govt. demonstrations w/ monks performing dangerous feats
            \begin{itemize}
                \item Buddhist disaster $\to$ Kennedy reconsidered ties to Diem (still supported idea of South Vietnam)
                \item After Diem failed to begin any desired U.S. reforms, Kennedy supported revolt w/ Diem being assassinated and replaced w/ new, even less stable governments
            \end{itemize}
        \end{itemize}
        \textbf{Led by Ngo Diem, South Vietnam initially received large amounts of aid from the U.S., with the CIA helping to attack the mafia and other dissident sects as well as with the U.S. agreeing not to hold elections in 1956. However, Diem's choice to begin suppressing North Vietnamese dissidents saw the North-supported NLF take action, slowly destabilizing the regime. Furthermore, after Diem began suppressing Buddhists, leading to horrendous demonstrations, Kennedy reevaluated U.S. ties to Diem and ultimately supported a revolt overthrowing him.}}
        \cornell{How did the U.S. decide to intervene in Vietnam?}{\begin{itemize}
            \item Johnson faced conflicting pressures either to expand U.S. war involvement or limit it
            \begin{itemize}
                \item Sought to prove himself by continuing Kennedy's trajectory; surrounded himself w/ Kennedy's best advisers, all of whom believed in resisting Vietnamese communisim 
                \item Intervention represented continuity in foreign policy: anticommunist ally appealed to U.S. for assistance; containment doctrine required intervention
            \end{itemize}
            \item First months in office saw slight expansion in military involvement (+5000)
            \item August 1964: U.S. destroyers patrolling Gulf of Tonkin announced to have been attacked by North Vietnamese boats $\to$ Congress passed \textbf{Tonkin Resolution} authorizing Johnson to take any measures necessary to contain North Vietnam
            \begin{itemize}
                \item Tonkin story likely embellished or inaccurately portrayed
                \item South Vietnam govt. in disastrous shape $\to$ U.S. took ownership of resistance
            \end{itemize}
            \item Communist forces attacked U.S. mil. base $\to$ Johnson authorized bombings of transportation lines to limit flow of soldiers/supplies
            \item March 1965: two batallions of marines arrived, adding up to \underline{100k troops} in Vietnam
            \item Mid-1965: Johnson announced that war had changed to active combat $\to$ sent in more and more troops, reaching 500k by end of 1967
            \begin{itemize}
                \item Troops joined by civilians assisting cause
                \item Total weight of bombs dropped on Vietnam exceeded WWII; American casualties skyrocketed
            \end{itemize}
            \item U.S. had created stable govt. under \textbf{Nguyen Van Thieu} but still corrupt/brutal, unable to establish authority in rural regions
        \end{itemize}
        \textbf{After Johnson took office, he sought to continue Kennedy's containment policies of aggression: his announcement that U.S. destroyers had been attacked by North Vietnam resulted in the Tonkin Resolution giving him ultimate wartime power. In retaliation to a communist attack on a U.S. military base, Johnson authorized major, intermittent bombings on Vietnam; he poured in troops until they reached 500k by the end of 1967. In 1965, although the U.S. had developed a stable government in Vietnam, it was known for corruption and unable to truly expand its influence.}}
        \cornell{How did the U.S. involvement in Vietnam represent a sticky situation?}{\begin{itemize}
            \item U.S. followed policy of \textbf{attrition}, hoping to pour damage into enemy until they were forced to surrender
            \begin{itemize}
                \item North Vietnam willing to sacrifice more soldiers than expected
                \item U.S. bombing of major war production centers ineffective bc. Vietnam not industrialized $\to$ no dependence on one single flow of goods 
                \begin{itemize}
                    \item Created network of underground tunnels/shops/factories to resist bombing
                \end{itemize}
                \item Vietnam resisted by securing more aid from USSR/China; continued to infiltrate south
            \end{itemize}
            \item \textbf{Pacification} program sought to persuade several North supporters to change sides 
            \begin{itemize}
                \item Ultimately replaced by \textbf{relocation}, or moving villagers from homes into refugee camps/cities, destroying villages
                \item Viet Cong simply established more villages
            \end{itemize}
            \item U.S. officials pushed Johnson to expand mil. intervention but did not want to go too far for fear of ruining national reputation
        \end{itemize}
        \textbf{One of the U.S. policies in the war was attrition, or the hope that Vietnam would surrender due to the large amount of damage inflicted; however, Vietnam was able to pour in large amounts of resources and resist the attack on their factories. The U.S. relocation technique, where Notrhern villagers were moved from their homes and their villages destroyed, was met with the Viet Cong's creation of new sanctuaries. In this position, Johnson was pushed to add additional troops, but he feared that this would tarnish the national reputation as well as potentially stimulate war with the USSR and China.}}
        \cornell{How did the Vietnam War affect the U.S. on the home front?}{\begin{itemize}
            \item As war continued, widespread political support became limited 
            \item American students were first to resist in 1967: peace marches in NY, DC, other cities produced great opposition
            \begin{itemize}
                \item Opposition became part of popular culture in folk musicians as well as journalists
            \end{itemize}
            \item Government soon began to oppose war directly
            \begin{itemize}
                \item Senator Fulbright of AR, chairman of Senate Foreign Relations Committee, began to staunchly criticize war in Jan. 1966
                \item Testimonies against war by George Kennan, Robert F. Kennedy, Robert McNamara, former great supporter $\to$ opposition given legitimacy
            \end{itemize}
            \item Economy suffered w/ promise of simultaneous Great Society and war unable to be met: inflation rate began rising rapidly
            \begin{itemize}
                \item Asked Congress for tax increase $\to$ retaliated w/ \$6b reduction in Great Society programs
            \end{itemize}
        \end{itemize}
        \textbf{Socially and politically, great opposition began to emerge to the Vietnam War on the home front due to the continued battles and required power with seemingly no result; testimonies from powerful government officials further offered legitimacy. The U.S. economy suffered greatly: Johnson was unable to simultaneously implement Great Society programs and finance the Vietnam War, with his Great Society programs cut back significantly.}}
    \end{document}