\documentclass[a4paper]{article}
    \input{../notesheader.tex}
    \usepackage[normalem]{ulem}

    \newcommand{\chapternumber}{29}
    \newcommand{\chaptertitle}{Civil Rights, Vietnam, and the Ordeal of Liberalism}
    \title{\vspace{-3em}
    \begin{tcolorbox}
    \Huge\sffamily \begin{center} AP US History  \\
    \LARGE Chapter \chapternumber \, - \chaptertitle \\
    \Large Finn Frankis \end{center} 
    \end{tcolorbox}
    \vspace{-3em}
    }
    \date{}
    \author{}
    
    \begin{document}
        \maketitle
        \SetBgContents{\rule[0em]{4pt}{\textheight}}
        \cornell[Key Concepts]{What are this chapter's key concepts?}{\begin{itemize}
            \item \textbf{8.1.I.B} - U.S., fearing Communist/Soviet ideology, explored new means of containing communism, notably major military battles like Vietnam/Korea
            \item \textbf{8.1.I.C} - Cold War alternated betw. direct/indirect mil. confrontation and mutual coexistence (détente)
            \item \textbf{8.1.II.B} - Vietnam War $\to$ often violent antiwar protests as war escalated, marking first widespread resistance to anticommunist policies
            \item \textbf{8.2.I.B} - Three branches of fed. govt. partook in desegregation of armed services, \textit{Brown v. Board of Education}, Civil Rights Act of 1964 to promote racial equality
            \item \textbf{8.2.III.A} - Liberalism reached peak of influence by 1960s
            \item \textbf{8.2.III.B} - Lyndon Johnson's Great Society used fed. programs to end discrimination, poverty, address social issues; Supreme Court cases $\to$ $\uparrow$ civil rights, individual liberties
        \end{itemize}}
    \end{document}