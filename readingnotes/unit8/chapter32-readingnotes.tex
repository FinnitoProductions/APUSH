\documentclass[a4paper]{article}
    \usepackage[T1]{fontenc}
    \usepackage{tcolorbox}
    \usepackage{amsmath}
    \tcbuselibrary{skins}
    
    \usepackage{background}
    \SetBgScale{1}
    \SetBgAngle{0}
    \SetBgColor{red}
    \SetBgContents{\rule[0em]{4pt}{\textheight}}
    \SetBgHshift{-2.3cm}
    \SetBgVshift{0cm}
    \usepackage[margin=2cm]{geometry} 
    
    \makeatletter
    \def\cornell{\@ifnextchar[{\@with}{\@without}}
    \def\@with[#1]#2#3{
    \begin{tcolorbox}[enhanced,colback=gray,colframe=black,fonttitle=\large\bfseries\sffamily,sidebyside=true, nobeforeafter,before=\vfil,after=\vfil,colupper=blue,sidebyside align=top, lefthand width=.3\textwidth,
    opacityframe=0,opacityback=.3,opacitybacktitle=1, opacitytext=1,
    segmentation style={black!55,solid,opacity=0,line width=3pt},
    title=#1
    ]
    \begin{tcolorbox}[colback=red!05,colframe=red!25,sidebyside align=top,
    width=\textwidth,nobeforeafter]#2\end{tcolorbox}%
    \tcblower
    \sffamily
    \begin{tcolorbox}[colback=blue!05,colframe=blue!10,width=\textwidth,nobeforeafter]
    #3
    \end{tcolorbox}
    \end{tcolorbox}
    }
    \def\@without#1#2{
    \begin{tcolorbox}[enhanced,colback=white!15,colframe=white,fonttitle=\bfseries,sidebyside=true, nobeforeafter,before=\vfil,after=\vfil,colupper=blue,sidebyside align=top, lefthand width=.3\textwidth,
    opacityframe=0,opacityback=0,opacitybacktitle=0, opacitytext=1,
    segmentation style={black!55,solid,opacity=0,line width=3pt}
    ]
    
    \begin{tcolorbox}[colback=red!05,colframe=red!25,sidebyside align=top,
    width=\textwidth,nobeforeafter]#1\end{tcolorbox}%
    \tcblower
    \sffamily
    \begin{tcolorbox}[colback=blue!05,colframe=blue!10,width=\textwidth,nobeforeafter]
    #2
    \end{tcolorbox}
    \end{tcolorbox}
    }
    \makeatother

    \parindent=0pt
    \usepackage[normalem]{ulem}

    \newcommand{\chapternumber}{32}
    \newcommand{\chaptertitle}{The Age of Globalization}

    \title{\vspace{-3em}
\begin{tcolorbox}
\Huge\sffamily \begin{center} AP US History  \\
\LARGE Chapter \chapternumber \, - \chaptertitle \\
\Large Finn Frankis \end{center} 
\end{tcolorbox}
\vspace{-3em}
}
\date{}
\author{}
    \begin{document}
    \maketitle
    \SetBgContents{\rule[0em]{4pt}{\textheight}}
    \cornell[A Resurgence of Partisanship]{What defined the American resurgence of partisanship?}{\textbf{The Clinton administration struggled with the Republican dominance in Congress and was plagued by numerous scandals. Despite this, Clinton ushered in a time of budget surplus and resolved foreign disputes. George W. Bush succeeded Clinton, winning a tight election against Al Gore. Bush relied on Clinton's surplus to implement a tax reduction and implement staunchly Republican policies. In 2004, Bush won another election against John Kerry, once again by a small margin.}}
    \cornell{How did the Clinton presidency start off?}{
        \begin{itemize}
            \item Clinton administration initially defined by many misfortunes, requiring numerous withdrawals
            \item Defined by a few important achievements
            \begin{itemize}
                \item Included budget approval away from Reagan-Bush years, including tax increase on wealthy, reduction in government spending, expansion of tax credits to working people
                \item Advocated free trade, globalism
                \begin{itemize}
                    \item Seen in long battle for approval of NAFTA, eliminating trade barriers
                    \item Received approval of far-reaching trade agreement in GATT
                \end{itemize}
                \item Saw major reform of health care system, supervised by task force led by wife
                \begin{itemize}
                    \item Reform promised to guarantee coverage to all Americans
                \end{itemize}
                \item Some foreign successes, including negotation to end war between Muslims and Christians in Bosnia through partitioning
            \end{itemize}
        \end{itemize}
        \textbf{The Clinton presidency was initially defined by a few major setbacks requiring major changes in policy; however, Clinton later achieved free trade agreements, began major reform of health care system, and foreign successes.}
    }
    \cornell{What led to the Republican resurgence in 1994?}{
        \begin{itemize}
            \item Republicans gained both houses of Congress in 1994, taking advantage of this to construct ambitious legislative program
            \begin{itemize}
                \item Proposed measures to transfer power from federal government to states to consequently reduce federal spending
                \item Hoped to restructure Medicare program
            \end{itemize}
            \item Clinton responded to Republican majority by shifting agenda to center
            \begin{itemize}
                \item Proposed tax cuts and budget balances to align with Republicans
                \item Still challenging to find compromise, leading to federal shutdown for several days due to inability to agree on budget 
                \begin{itemize}
                    \item Discredited Republican leadership, improving Clinton's standings
                \end{itemize}
            \end{itemize}
        \end{itemize}
        \textbf{The Republican resurgence of 1994 was initially caused by a Republican majority in both houses of Congress. Although the Republicans took this opportunity to construct a new legislative program, their inability to agree with Clinton on key matters culminating in a federal government shutdown ultimately discredited their leadership and improved Clinton's standings.}}
    \cornell{What was the result of the election of 1996?}{
        \begin{itemize}
            \item Clinton reached commanding position for reelection by 1996, unopposed for nomination
            \begin{itemize}
                \item Faced \textbf{Robert Dole}, senator unable to inspire enthusiasm even within party
                \item Clinton reached position of high popularity due to centrist stance, undermining Republicans and championing ideals promoted by Reagan such as peace and prosperity
            \end{itemize}
            \item Congress passed many important bills as election neared
            \begin{itemize}
                \item Raised legal minimum wage
                \item Clinton reluctantly signed \textbf{welfare reform} bill 
                \begin{itemize}
                    \item Ended guaranteed federal assistance to families with dependent children
                    \item Transferred majority of power to state governments
                    \item Shifted welfare benefits to those with low-wage jobs rather than those without jobs
                \end{itemize}
            \end{itemize}
            \item Clinton won election despite slight campaign flagging by conclusion
            \begin{itemize}
                \item Failed to regain either house of Congress
                \item First Democratic president to win two terms since Franklin Roosevelt
            \end{itemize}
        \end{itemize}
        \textbf{Clinton won the election against Robert Dole by a significant amount due to his centrist stance and important bills passed as the election approached, including the raise of minimum wage.}
    }
    \cornell{What major events marked Clinton's second term as president?}{
        \begin{itemize}
            \item Clinton still faced hostile Republican Congress
            \begin{itemize}
                \item Forced to propose modest tax agenda with tax cuts, credits for middle-class Americans
                \item Negotiated balanced budget with Republicans, generating first surplus in 30 years by 1998
            \end{itemize}
            \item Despite having been faced with many scandals, most extreme was denied sexual relations with young intern \textbf{Lewinsky}
            \begin{itemize}
                \item Charged for having lied about events in deposition
                \item Continued to deny charges while heavily backed by public
                \begin{itemize}
                    \item Popularity soared to record levels
                \end{itemize}
                \item Scandal revived after Lewinsky testified about relationship with Clinton 
                \begin{itemize}
                    \item After \textbf{special counsel Starr} subpoenaed Clinton, president finally agreed to "improper relationship"
                    \item Recommended impeachment to Congress
                \end{itemize}
                \item Full House approved impeachment by 1998, finally moved to Senate which ended in acquittal
            \end{itemize}
            \item Serious foreign policy crisis emerged in 1999 in Balkans
            \begin{itemize}
                \item Serbian government and Kosovo separatists engaged in bitter civil war
                \item NATO forces dominated by U.S. began to bomb Serbians, leading to cease-fire in exchange for Serbian withdrawals
                \item Precarious peace followed
            \end{itemize}
            \item Despite numerous scandals, Clinton ended eight years with popularity higher than initially due to overall stability and prosperity
        \end{itemize}
        \textbf{Clinton's second term began with an important budget agreement leading to a major surplus, followed up by a major scandal concerning a sexual relation with an intern, Monica Lewinsky. Finally, he authorized NATO forces to bomb Serbia, marking an end to the Serbia-Kosovo separation crisis.}
    }
    \cornell{What was the result of the election of 2000?}{
        \begin{itemize}
            \item Republican \textbf{George W. Bush} and Democrat \textbf{Al Gore} both easily won party nominations
            \item Both ran centrist campaigns, with polls showing extremely tight race even up to end
            \item After the election, neither candidate immediately won due to inaccuracy in Florida
            \begin{itemize}
                \item Led to recount, resulting in Bush leading by no more than 300 votes
                \item When court deadline came, recount had not yet been complete; Republican Floridian secretary of state claimed that Bush had won
                \item Gore campaign contested, leading to 5-4 Supreme Court decision in favor of Bush
            \end{itemize}
        \end{itemize}
        \textbf{The election of 2000 was extremely controversial due to the approximately equal popularity of both Gore and Bush. Ultimately, however, after a recount in Florida and a Supreme Court decision, it was decided that Bush won an extremely tight race.}
    }
    \cornell{What defined Bush's first term in office?}
    {
        \begin{itemize}
            \item Principal campaign promise to use budget surplus to finance tax reduction; become narrowly possible
            \item Despite campaign as moderate centrist hoping to bridge gap between parties, governed as staunch conservative
            \begin{itemize}
                \item Refused to support renewal of Clinton's assault weapons ban
                \item Mobilized evangelical Christians as part of coalition
            \end{itemize}
            \item Entirety of presidency ultimately defined by September 11 attacks
        \end{itemize}
        \textbf{Bush's presidency was marked by a major tax reduction and staunchly Republican policies despite centrist campaign, including limited gun control. However, Bush's presidency was, in all, defined by the September 11 attacks.}
    }
    \cornell{What was the result of the election of 2004?}{
        \textbf{Bush won the election against uncontested John Kerry, once again by a very small amount with the votes approximately equal.}
    }
    \cornell[The Economic Boom]{What caused the dramatic transformation to the American economy?}{\textbf{The economic boom emerged due to reduced labor costs and the rapid growth of the technology sector; it led the wage gap to further increase, and also coincided with the globalization of the American economy.}}
    \cornell{What were the roots of the economic boom in America?}{
        \begin{itemize}
            \item Roots of economic growth of '80s onward lay in troubled years of '70s
            \begin{itemize}
                \item Stagnation encouraged American businesses to adopt new practices, most significantly investment in technology
                \item Sought to reduce labor costs, with many comparisons being drawn to increasingly prosperous nations with low-wage workers
                \begin{itemize}
                    \item Implemented by taking harsher actions against unions or moving where union activity was low
                    \item Often out-sourced production to China, Mexico
                \end{itemize}
            \end{itemize}
            \item Technology boom created many more jobs, but not as many as originally in the industrial sector
            \item Began to experience great prosperity at unprecedented levels, including booming stock prices, rapidly growing GDP, and significantly lowered rate of inflation 
            \item Lasted for long period of time
        \end{itemize}
        \textbf{The economic boom in the U.S. emerged from the great troubles in the '70s: it was primarily caused by reduced labor costs and the rapid growth of the technology sector.}
    }
    \cornell{What defines the American two-tiered economy?}{
        \textbf{The two-tiered economy emerged significantly with the economic boom as only those talented enough to profit from areas of growth were able to earn large incomes. For most Americans, income was unchanged or even reduced, with the poverty rate beginning to increase significantly.}
    }
    \cornell{What were the effects of the globalization of the American economy?}{
        \textbf{In the '50s/'60s, the economy prospered with little external influence; however, by the late '70s, the American economy had become heavily import-oriented, leading to a great trade imbalance with American products facing competition from within U.S.}
    }
    \cornell[Science and Technology in the New Economy]{How did the booming economy drive the furthering of science and technology?}{\textbf{The booming American economy allowed the digital revolution to unfold in America, tranforming the world by connecting people worldwide through the internet. Furthermore, the economy allowed the government to devote significant money to genetic engineering and, specifically, mapping the human genome.}}
    \cornell{What was the Digital Revolution?}{
        \begin{itemize}
            \item Development of microprocessor revolutionized American life, allowing small machines to perform large computations
            \item Microprocessor served as basis for personal computer, first by Apple and later IBM with PC (OS by Microsoft)
            \item Led to numerous major businesses, including computer manufacturers, silicon chip creators
        \end{itemize}
        \textbf{The Digital Revolution started with Intel's creation of the microprocessor, which ultimately gave way to the personal computer, a device which transformed the American lifestyle and economy.}
    }
    \cornell{What were the origins and initial impact of the Internet?}{
        \begin{itemize}
            \item Internet began in 1963 in U.S. government's Advanced Research Projects Agency, ARPA
            \begin{itemize}
                \item Created for defense-related purposes
            \end{itemize}
            \item By 1971, 23 computers had been linked; rapidly expanded afterward
            \begin{itemize}
                \item Widespread interest led to new technologies included e-mail, personal computers
                \item Users went from less than 1000 in 1984 to 2 billion in 2013
            \end{itemize}
            \item World Wide Web emerged in 1989, allowing for easy sharing of information
        \end{itemize}
        \textbf{The Internet began as a U.S. defense tool to link multiple computers for convenient communication; however, it quickly spread and allowed anyone with a computer to access a much larger network of other users.}
    }
    \cornell{What were the major early breakthroughs in genetics?}{
        \begin{itemize}
            \item Computer technology proved essential in growth of scientific research
            \begin{itemize}
                \item Predated by discoveries of DNA, double helix structure, genetic codes
                \item Science of genetic engineering emerged 
            \end{itemize}
            \item Scientists could slowly identify genes in humans, other creatures which determined key traits
            \begin{itemize}
                \item Process sped up gradually after government investment in Human Genome Project to continue to map the complete human Genome
                \item DNA attracted public attention for ability to uniquely identify a human
            \end{itemize}
        \end{itemize}
        \textbf{The early breakthroughs in genetics were generally assisted by the emerging computer technology,including the gradual identification of the complete human genome and the specific traits which DNA dictates uniquely.}
    }
    \cornell[A Changing Society]{What were the major social shifts in the United States in the '90s and '00s?}{\textbf{The most significant social changes in the United States during this period included the aging of the American population leading to dramatically increased immigration rates, the greater opportunity and success available to middle-class blacks in America paired with the grim state of many impoverished blacks living in the inner-city, and the drug use epidemic leading to AIDS which finally began to slow by the beginning of the 21st century.}}
    \cornell{What changes occurred in the American population?}{
        \begin{itemize}
            \item As birth rates decreased and life spans increased, population marked by increased agedness, specifically in "\textbf{baby boomers}"
            \begin{itemize}
                \item Aging population stressed Social Security, Medicare systems
                \item Important implications for workforce
                \begin{itemize}
                    \item Last 20 years of $20^\text{th}$ century saw increase in populations between 25-54 (prime workforce)
                    \item Beginning of $21^\text{st}$ century saw no changes
                \end{itemize}
            \end{itemize}
            \item Slowed growth of native population (particularly workforce) led to immigration boom
            \begin{itemize}
                \item Further helped by 1965 Immigration Reform act, preventing immigration discrimination based on national origins
                \item Largest groups were Latinos and Asians; others from Africa, Middle East, Russia
            \end{itemize}
        \end{itemize}
        \textbf{The American native-born population has aged dramatically over the previous 40 years, causing a decrease in growth of the prime workforce population. This opened the door to immigration from around the world, diversifying the American landscape.}
    }
    \cornell{How did the perception of America for African Americans compare to that for white Americans in the post-civil rights era?}{
        \begin{itemize}
            \item Civil rights movement had two distinct effects on African Americans
            \begin{itemize}
                \item Increasing opportunities for advancement progressively became available
                \item Reduced industrial sector and government services drew away many jobs
            \end{itemize}
            \item \textbf{Black middle-class} (over half of African American population) experienced most remarkable progress
            \begin{itemize}
                \item Moved into more affluent, suburban communities
                \item High school graduated blacks moved onto college at the same rate as whites
                \item 20\% of African Americans over 24 possessed bachelor's degrees compared to 30\% of whites
                \item Made large strides in previously segregated professions, like white-collar jobs
            \end{itemize}
            \item Many other groups, most prominently impoverished blacks ("\textbf{underclass}"), still left with few benefits from social prosperity
            \begin{itemize}
                \item Continued to live in inner-city, impoverished neighborhoods
                \item Less than half finished high school; more than 60\% unemployed
                \item In 1970, only 59\% of black minors lived with both parents, rest usually lived with single mother (34\% by 2010, however)
            \end{itemize}
        \end{itemize}
        \textbf{America became much more hopeful for middle-class blacks, who were able to move into affluent communities and experience great success in college and future professions. However, the improverished blacks remained underreached; most continued to live in decaying inner-cities.}
    }
    \cornell{What were the effects of the modern plagues of society?}{
        \begin{itemize}
            \item Dramatic increase in drug use in 1980s, creating multi-billion dollar industry, particularly in crack-cocaine
            \begin{itemize}
                \item Began to decline among middle-class people by 1980s
                \item Declined slowly in poorer urban neighborhoods
            \end{itemize}
            \item Drug use tied to epidemic spread of \textbf{AIDS}
            \begin{itemize}
                \item Began simply among gay men due to exchange of semen; declined as preventative measures began within gay community
                \item Later spread through intravenous drug users \textbf{sharing needles}
                \item Effective treatments emerged by mid-1990s
                \begin{itemize}
                    \item Required rigorous schedule of various cocktails of drugs
                    \item Prolonged life span of AIDS-carriers significantly
                    \item Could not reach poorer parts of America as well as poorer nations such as Africa due to high price tag
                    \begin{itemize}
                        \item UN contributed funds to fight AIDS crisis in Africa; progress remained slow
                    \end{itemize}
                \end{itemize}
            \end{itemize}
        \end{itemize}
        \textbf{The two modern plagues were drug use and AIDS. Drug use had severe effects especially on poorer communities; furthermore, it contributed significantly to the spread of AIDS through intravenous needles. By the beginning of the $21^\text{st}$ century, however, drug use had begun to decline worldwide and cheaper, effective AIDS treatments began to emerge.}
    }
    \cornell[A Contested Culture]{What were the most significant changes to American culture after World War II?}{\textbf{Post-WWII culture was marked by a split between parties concerning feminism, abortion, and the environment. Continued debates concerning the right-to-life movement vs the pro-choice movement, the Equal Rights Amendment, and the importance of global warming, meant that little substantial could be done as the presidency continually changed between parties.}
    }
    \cornell{What were the political undertones of the battles over feminism and abortion?}{
        \begin{itemize}
            \item "New Right" campaigned heavily against equal rights movement and abortion
            \item Result of \textit{Roe v. Wade} seemed to settle decision on pro-choice for abortions on national stage
            \begin{itemize}
                \item Created grassroots movement known as the "right-to-life" movement; championed by Catholic Church, Mormons, evangelical Christians
                \item Others oppposed due to its going against traditional family views
                \item Argued that fetuses were human beings with a "right to life"
            \end{itemize}
            \item Abortion was attacked in other ways 
            \begin{itemize}
                \item Congress barred use of public funds for abortion; Reagan and Bush administrations reduced right of federal doctors to openly suggest abortion
                \item Extremists often murdered doctors who performed abortions, terrorized/harrassed others
            \end{itemize}
            \item As Supreme Court membership began to shift to be more conservative, many renewed hope
            \begin{itemize}
                \item Led to strengthening of pro-choice movement; power shown by Clinton's reelection and Obama's election
            \end{itemize}
        \end{itemize}
        \textbf{The battles over feminism and abortion both were a matter of the New Right against liberals: many staunch Republicans strongly believed that all fetuses had the right to live and that the Equal Rights Amendment need not be passed. Both sides had immense political power.}
    }
    \cornell{What caused the environmental movement to grow in the 1970s?}{
        \begin{itemize}
            \item Made first headway with emergence of Earth Day in 1970, generating widespread concern about the environment
            \item Many began to study global warming and how it was aggravated by fossil fuels 
            \item Officials of major industrial nations met in Kyoto in 1997, agreed on treaty to reduce carbon emissions
            \begin{itemize}
                \item Republicans in U.S. Senate refused ratification; Bush feared economic impact; Obama has done little
                \item Without either China or U.S., Kyoto Protocol essentially dead
            \end{itemize}
        \end{itemize}
        \textbf{The environmental movement grew with the worldwide Earth Day held from 1970 onward, and the obvious impacts of global warming. However, the U.S. has seemingly made little obvious effort to cut carbon emissions.} 
    }
    \cornell[The Perils of Globalization]{What were some of the negative results of globalization in America?}{\textbf{The rapid globalization of the American economy and political landscape led to many protests within the nation focusing on economic dominance by other nations, human rights, and more. Worldwide, however, many of the more orthodox Muslims in the Middle East felt directly threatened by Western culture and imperialism, leading to numerous terrorist attacks ultimately culminating in 9/11. Bush's retaliation at first saw punishment of those who were responsible for the attack; however, it quickly unraveled to a show of American dominance in a war against Iraq for fabricated, proven-false reasoning.}}
    \cornell{What were the major types of opposition to the new, global world?}{
        \begin{itemize}
            \item Critics on the left argued that military action was being used to advance economic interests (like in Gulf War, Iraq War)
            \item Critics on the right argued that nation had began to be swayed by interests of other nations, ceding sovereignty to international organizations
            \item Largest criticism from those economically threatened 
            \begin{itemize}
                \item Labor unions feared major jobs being outsourced to other nations
                \item Humanitarians argued that labor in other nations was unethical, involving "slave laborers"
                \item Environmentalists argued that less developed nations without laws to control emissions were being exploited
                \item Human rights activists felt that multinational corporations were being needlessly empowered
            \end{itemize}
            \item All opponents agreed that discontent lay not in free-trade agreements, but instead multinational institutions advancing global economy
            \begin{itemize}
                \item Included World Trade Organization, International Monetary Fund, World Bank
            \end{itemize}
            \item 1999 meeting between leaders of 7 industrial nations met in Seattle
            \begin{itemize}
                \item Met with thousands of protestors, some peaceful but others violent, clashing with policies
                \item Other major distrupted events included IMF and World Bank meeting in D.C; leaders meeting in Genoa
            \end{itemize}
        \end{itemize}
        \textbf{The opposition to the global world including political opponents on both the left (economic influence)and right (political influence), as well as those directly economically threatened. Opponents staged large protests at major meetings between world leaders.}
    }
    \cornell{How did those who felt culturally threatened fight against globalization?}{
        \begin{itemize}
            \item Many citizens of nations where poverty had become aggravated due to globalization felt deep anger against U.S.
            \item Most significant anger came from Islamic Middle East nations, affected by both economy and culture
            \begin{itemize}
                \item Iranian Revolution emerged as orthodox Muslims feared growing influence of Western culture
                \item Spread to other Islamic nations, creating rapidly spreading fundamentalist phenomenon
            \end{itemize}
            \item Middle Eastern anger led to significant violence by some orthodox Muslims against the West to disrupt, create fear, classified as "terrorism"
            \begin{itemize}
                \item Terrorism has historically been used to classify many other non-Muslim events, including acts of Jacobins against French government, Irish revolutionaries, Israelis and Palestinians
                \item U.S. had also historically experienced terrorism
                \begin{itemize}
                    \item Many attacks occured away from American soil
                    \begin{itemize}
                        \item Marines faced attack in Beirut in 1983
                        \item Explosion in Scotland brought down small airliner
                        \item Embassies bombed in 1998
                        \item Naval vessel attacked
                    \end{itemize}
                    \item Others had began to occur on American soil even before 2001
                    \begin{itemize}
                        \item Bomb exploded in WTC parking lot in 1993
                        \item Van blew up in Oklahoma City, leading to conviction of militant antigovernment man Timothy McVeigh
                    \end{itemize}
                    \item Terrorism generally considered foreign problem by most Americans until 9/11
                \end{itemize}
                \item After tragedy of 2001, security measures tightened with far stricter security checks when departing airports
            \end{itemize}
        \end{itemize}
            \textbf{Although many smaller nations influenced by U.S. globalization (typically through an increased poverty rate) began to harbor hatred for the U.S., none had a higher percentage than many Middle Eastern nations, threatened not only by increased poverty but also the diffusion of American culture affecting their radically different fundamental traditions. Many resorted to terrorism to resist globalization, seeking to create fear among Americans.}
    }
    \cornell{What were the major parts of Bush's war on terrorism?}{
        \begin{itemize}
            \item Began with direct targeting of al-Qaeda and the Taliban, two groups believed to have caused 9/11 under Osama bin Laden
            \begin{itemize}
                \item Consistent bombing of Afghanistan caused Taliban regime to collapse, leaders fled capital
                \item American troops unable to capture bin Laden, but rounded up many suspects
                \begin{itemize}
                    \item Sent to Guantanamo Bay prison in Cuba
                    \item Patriot Act of 2001 allowed suspected terrorists to be held for months without access to lawyers, tortured and interrogated
                \end{itemize}
            \end{itemize}
            \item Afterward, Bush focused on "\textbf{axis of evil}": Iraq, Iran, NK
            \begin{itemize}
                \item Bush gradually built public case for invasion of Iraq, based around two central claims
                \begin{itemize}
                    \item Centered argument on potential creation of "weapons of mass destruction"
                    \item Supported with Iraq's supporting hostile terrorist groups
                    \item Final point asserted that Hussein, leader of Iraq, continually broke human rights laws (only \textit{true} of three points)
                \end{itemize}
                \item Began invasion in 2003 paired with Britain and partial UN authorization; quickly toppled Hussein regime
                \begin{itemize}
                    \item Popularity for war began to dwindle after reveal that "weapons of mass destruction" were nonexistent and that American soldiers had tortured Iraqi prisoners
                \end{itemize}
                \item Direct attack marked important change in U.S. foreign policy
                \begin{itemize}
                    \item From beginning of communist regimes, followed \textbf{containment} policy, maintaining stability without major violence; required immense constraint
                    \item Continual critics of constraints and supporters of direct intervention had influence during Bush's presidency
                    \begin{itemize}
                        \item Ended legacy of containment, created belief that U.S. had right/responsibility to spread freedom worldwide
                        \item Bush justified through example of ignorance toward Soviet domination at conclusion of WWII without violence leading to Cold War
                    \end{itemize}
                \end{itemize}
            \end{itemize}
        \end{itemize}
        \textbf{Bush's war on terrorism began with the short-term issue of punishing al Qaeada and the Taliban for their influence on 9/11. It quickly shifted to Iraq as Bush administration fabricated stories about Iraq's weapons of "mass destruction" and support of terrorist groups to justify their planned war against Iraq; this was most significant in that it signified an end to the historical U.S. policy of containment, or control with little violence.}
    }
    \cornell[Turbulent Politics]{What have been the major characteristics of the last 20 years of U.S. politics?}{\textbf{By the conclusion of the Bush presidency, he was viewed extremely negatively due to his irrational decision concerning the Iraq war. This strong anti-Bush sentiment disadvantaged the Republican frontrunner McCain, giving the presidency to Democrat Barack Obama. Obama entered during a significant recession which Bush had failed to limit; however, his economic stimulus policies were relatively successful. During his presidency, Obama worked to improve international relations and pass new bills allowing for universal health care (much to the chagrin of the Tea Party);however, aside from the economic recessions, he faced major challenges including an NSA privacy breach, protestors from "Occupy Wall Street" claiming he was only further aggravating economic disparity, and political gridlock preventing his bills for gun control, immigration benefits, and budget deals from being passed.}}
    \cornell{How did the Bush presidency unravel?}{
        \begin{itemize}
            \item Resolute stance against terrorism boosted popularity, even at first during the Iraq War
            \item Eventually declined in popularity after reveal that tax cuts disproportionately benefited wealthier Americans
            \begin{itemize}
                \item Reflected view that growth was best attained through benefitting those more likely to invest money
                \item Tax cuts led to increase of nearly \$10 trillion in national debt 
            \end{itemize}
            \item Education bill "No Child Left Behind" also very controversial
            \begin{itemize}
                \item Tied funding to success on standardized tests 
                \item Many felt that students should not be taught to a test 
            \end{itemize} 
            \item By the time of the election, extremely uncertain waters
            \begin{itemize}
                \item Veteran John Kerry's harsh assaults on Iraq War put in position of great popularity
                \item Mobilization of large numbers of conservatives allowed Bush to narrowly win election
            \end{itemize}
            \item Second term extremely difficult with Iraq War going poorly, government's slow response to Hurricane Katrina, major justice department scandals
            \begin{itemize}
                \item Caused Republicans to lose majority in both houses, leaving Bush with little support
            \end{itemize}
        \end{itemize}
        \textbf{Bush started off on a high note due to his resolute stance against terrorism; however, his approval ratings quickly plummeted after his tax cuts benefitted only the wealthy, his educational bill tied federal funding to standardized test scores, and his administration took far too long to react to Hurricane Katrina. Bush left the office somewhat disgraced, having lost his majority in both houses.}
    }
    \cornell{What was the result of the election of 2008 and how was it affected by the emerging financial crisis?}{
        \begin{itemize}
            \item Republican John McCain had assured nomination within party
            \item Democrats Obama and H. Clinton battled for months, Obama's organizational power finally earning him the nomination
            \item McCain and Obama entered race with extremely different programs
        \end{itemize}
        \begin{tabular}{c|c|c}
            & \textbf{McCain} & \textbf{Obama} \\ 
            \hline
            \textbf{Iraq War} & Good & Bad  \\ 
            \footnotesize \textbf{Health Insurance} & No & Yes \\ \textbf{Tax} & Cuts & \small Increase on Wealthy \\ 
            \textbf{VP} & Little-known (Palin) & Well-known (Biden)
        \end{tabular}
        \begin{itemize}
            \item Financial crisis had arisen in mid-2007, worst since Great Depression by 2008
            \begin{itemize}
                \item Caused by new credit instruments making borrowing money cheaper, encouraging large, risky mortgages with gradually increasing interest
                \item Rapidly increased sale of homes and decreased price
                \item Soon, as mortgage rates increased, people could no longer pay back, causing many homeowners to walk away and default loans
                \item Banks lost large sums of money
                \item Great Recession led to decrease in wages, frequent layoffs, making it eve harder to pay back loans, further aggravating situation
                \item Bush and eventually Obama won right to asset relief, using federal funds to bail out banks 
            \end{itemize}
            \item Obama won the election by a significant margin, as McCain's policies connected him to the failing Bush administration
        \end{itemize}
        \textbf{Obama won the election by a relatively significant margin due to McCain's similar ideology to the Bush administration. The election occurred amidst a major financial recession caused by banks' hefty, risky mortgages which many could not pay back.}
    }
    \cornell{What were Obama's domestic policies during his first term?}{
        \begin{itemize}
            \item Entered office with extremely high expectations, leading to some inevitable disappointment
            \begin{itemize}
                \item Faced with managing worst economy since 1930s, one war in Iraq and another in Afghanistan, polarized political climate
                \item Major companies faced bankrupcty, economy remained on the verge of collapse
            \end{itemize}
            \item First major policy was stimulus package for economy
            \begin{itemize}
                \item Largest economic stimulus in history
                \item Entailed tax cuts, unemployment benefits, increased spending on education, infrastructure, police, health care
                \item Passed by small margin; most argue that was successful in saving economy but others believed unnecessarily added to national debt
            \end{itemize}
            \item Pushed for Patient Protection and Affordable Act
            \begin{itemize}
                \item Urged all Americans to purchase health insurance
                \item Passed in Congress in 2010, but benefits only came into play by 2014
                \begin{itemize}
                    \item Website to process insurance applications failed to work
                \end{itemize}
                \item Major opponents to "Obamacare" were group of evangelical, conservative, libertarian Republicans known as "Tea Party"
                \begin{itemize}
                    \item Early leader was Ted Cruz
                    \item Connection to wealthy conservatives gave disproportionate political power
                    \item In mid-term elections, 130 Tea Party-endorsed candidates sworn in as members of Congress
                \end{itemize}
            \end{itemize}
        \end{itemize}
        \textbf{Obama's first term began with his economic stimulus plan, which successfully revive the failing economy. Furthermore, he immediately began work on the Affordable Care Act despite much opposition.}
    }
    \cornell{What were Obama's international policies in his first term?}{
        \begin{itemize}
            \item Obama hoped to reshape international American role 
            \begin{itemize}
                \item Completely ended combat role in 2010 in Iraq (while sending additional troops to Afghanistan)
                \item Worked with secretary of state Clinton to find peace between Israel and Palestine, rebuild relations damaged in Iraq War
                \begin{itemize}
                    \item Awarded Nobel Peace Prize in 2009
                    \item Visited India, China, South Korea for trade agreements (only successful in India)
                \end{itemize}
            \end{itemize}
        \end{itemize}
        \textbf{Obama's international policies were centered around limiting violence between nations and promoting America's economic interests.}
    }
    \cornell{What were Obama's major sources of opposition?}{
        \begin{itemize}
            \item Began to see some opposition from left
            \begin{itemize}
                \item Believed policies had compromised Democratic goals, failed on many promises (like Guantanamo closing)
                \item Failed to punish those deemed responsible for recession, policies continued to create 1\% of richest Americans
                \begin{itemize}
                    \item Economic inequality even more prominent in 2012 election
                \end{itemize}
            \end{itemize}
            \item Importance of money for politics quickly became mainstream news 
            \begin{itemize}
                \item \textit{Citizens United v. Federal Election Commission} saw conservative-leading court ruling that government could not limit campaign-related expenditures from unions/corporations
                \begin{itemize}
                    \item Many argued that policy favored Republicans due to wealthy donors
                \end{itemize}
            \end{itemize}
        \end{itemize}
        \textbf{Wealth inequality remained a prominent challenge for Obama's administration and it soon became the focal point of the 2012 election, earning the cricitism of Republicans and Democrats alike.} 
    }
    \cornell{What was the result of the 2012 election?}
    {
        \begin{itemize}
            \item Obama and Biden faced off against Romney and Ryan
            \item Romney's sudden change in policy to turn to a more conservative stance was preyed on by Democrats
            \item Focal points of 2012 election included health-care reform, immigration reform, budget deficit, taxation
            \begin{itemize}
                \item Strained economy brought issues onto domestic stage rather than international
                \item Despite continual struggle with unemployment, Obama focused on killing of bin Laden in 2011, success of Affordable Care Act in Supreme Court case
            \end{itemize}
            \item Race at time seemed somewhat close, but Romney's missteps and Obama's great success gave him significant advantage and win for second term
        \end{itemize}
        \textbf{The 2012 election ended with a second term for Obama after Romney made numerous political missteps and Obama had made clear his numerous successes while in office.}
    }
    \cornell{What were Obama's major challenges during his second term?}{
        \begin{itemize}
            \item Began with disclosure of classified documents by Snowden to world
            \begin{itemize}
                \item Revealed extensive government surveillance of Internet, telephone data including personal phone lines of Angela Merkel
                \item Many dismissed as traitors, others believed to have been heroic champion of freedom 
            \end{itemize}
            \item Civil war in Syria presented major challenge
            \begin{itemize}
                \item Assad killed around 120,000 of own people whom he classified as rebels
                \item U.S. and Europe worked with Russia to condemn Assad's use of chemical weapons
                \item Brokered deal to remove stockpile of weapons, but Obama was careful not to send troops after disaster of Iraq War
            \end{itemize}
            \item Began to focus on large Latino population, working with many Republicans to pass immigration reform bill
            \begin{itemize}
                \item Immigration Modernization Act allowed path to US citizenship for illegal immigrants
                \item Many Republicans continued to oppose, preventing bill from being passed - continual gridlock
            \end{itemize}
            \item Another major failure included failure to enact meaningful gun control laws despite numerous horrific shootings during office
            \begin{itemize}
                \item Congress resisted any legislation; NRA-sponsored senators warded off any change
            \end{itemize}
            \item House gridlock prevented budget deal with continual division based on taxation
            \begin{itemize}
                \item Barely avoided shutdown in early 2014; implied decreased Tea Party influence and restored power to mainstream leadership
            \end{itemize}
        \end{itemize}
        \textbf{Obama's second term began with a disastrous leak of confidential information revealing compromising behavior on behalf of the NSA. Although his choice to prevent direct violence during the war in Syria was praised, most other bills failed to be passed or enacted due to the continual gridlock in the Senate.}
    }
    \end{document} 