\documentclass[a4paper]{article}
    \usepackage[T1]{fontenc}
    \usepackage{tcolorbox}
    \usepackage{amsmath}
    \tcbuselibrary{skins}
    
    \usepackage{background}
    \SetBgScale{1}
    \SetBgAngle{0}
    \SetBgColor{red}
    \SetBgContents{\rule[0em]{4pt}{\textheight}}
    \SetBgHshift{-2.3cm}
    \SetBgVshift{0cm}
    \usepackage[margin=2cm]{geometry} 
    
    \makeatletter
    \def\cornell{\@ifnextchar[{\@with}{\@without}}
    \def\@with[#1]#2#3{
    \begin{tcolorbox}[enhanced,colback=gray,colframe=black,fonttitle=\large\bfseries\sffamily,sidebyside=true, nobeforeafter,before=\vfil,after=\vfil,colupper=blue,sidebyside align=top, lefthand width=.3\textwidth,
    opacityframe=0,opacityback=.3,opacitybacktitle=1, opacitytext=1,
    segmentation style={black!55,solid,opacity=0,line width=3pt},
    title=#1
    ]
    \begin{tcolorbox}[colback=red!05,colframe=red!25,sidebyside align=top,
    width=\textwidth,nobeforeafter]#2\end{tcolorbox}%
    \tcblower
    \sffamily
    \begin{tcolorbox}[colback=blue!05,colframe=blue!10,width=\textwidth,nobeforeafter]
    #3
    \end{tcolorbox}
    \end{tcolorbox}
    }
    \def\@without#1#2{
    \begin{tcolorbox}[enhanced,colback=white!15,colframe=white,fonttitle=\bfseries,sidebyside=true, nobeforeafter,before=\vfil,after=\vfil,colupper=blue,sidebyside align=top, lefthand width=.3\textwidth,
    opacityframe=0,opacityback=0,opacitybacktitle=0, opacitytext=1,
    segmentation style={black!55,solid,opacity=0,line width=3pt}
    ]
    
    \begin{tcolorbox}[colback=red!05,colframe=red!25,sidebyside align=top,
    width=\textwidth,nobeforeafter]#1\end{tcolorbox}%
    \tcblower
    \sffamily
    \begin{tcolorbox}[colback=blue!05,colframe=blue!10,width=\textwidth,nobeforeafter]
    #2
    \end{tcolorbox}
    \end{tcolorbox}
    }
    \makeatother

    \parindent=0pt
    \usepackage[normalem]{ulem}

    \newcommand{\unitnumber}{8}
    \newcommand{\chapternumber}{30}
    \newcommand{\chaptertitle}{The Crisis of Authority}

    \title{\vspace{-3em}
    \begin{tcolorbox}
    \Huge\sffamily \begin{center} AP US History  \\
    \LARGE Chapter \chapternumber \, - \chaptertitle \\
    \Large Finn Frankis \end{center} 
    \end{tcolorbox}
    \vspace{-3em}
    }
    \date{}
    \author{}
    
    \begin{document}
        \maketitle
        \SetBgContents{\rule[0em]{4pt}{\textheight}}
        \cornell[Key Concepts]{What are this chapter's key concepts?}{\begin{itemize}
            \item \textbf{8.1.II.D} - Mil./econ./ideological issues defined U.S. involvement in Middle East: oil crises $\to$ national energy policy in U.S.
            \item \textbf{8.2.II.A} - Feminist/gay/lesbian activists sought legal/econ./social equality
            \item \textbf{8.2.II.B} - Latino, natives, Asian Americans sought social/economic equality, apology for past injustices
            \item \textbf{8.2.II.D} - Environmental issues $\to$ env. movement using legislation/public efforts to combat pollution, protect resources
            \item \textbf{8.2.III.B} - Johnson's Great Society used legislation to end poverty/racism; Supreme Court further advanced civil rights
            \item \textbf{8.2.III.C} - 1960s saw conservatives challenge liberal laws, court decisions as well as moral/cultural decline; sought to limit fed. govt. power
            \item \textbf{8.2.III.D} - Left groups rejected liberal policies: felt leaders not active enough in changing racial/economic status at home 
            \item \textbf{8.2.III.F} - 1970s saw clashes betw. conservatives/liberals over social/cultural issues, fed. govt. power, race, individual rights
            \item \textbf{8.3.II.B} - Feminists, young ppl. of 1960s counterculture rejected values of parents' generation, seeking informality, changes in sexual norms w/in U.S. culture
            \item \textbf{9.2.I.C} - $\uparrow$ employment in service sectors; $\downarrow$ in manufacturing, unions
        \end{itemize}}
        \cornell[The Youth Culture]{What were the key elements of 1960s youth counterculture?}{\textbf{The 1960s was characterized by a "New Left" dominated by radicalized college students fighting for their right to protest at universities as well as directly opposing the war and military draft. Hippie culture, too, became prominent, representing a direct assault on middle-class values of materialism; hippies supported sexuality, drugs, long hair, and rock 'n' roll.}}
        \cornell{What were the broad visions of social and cultural protest?}{\textbf{One element was based around a community of "the people," or taking power from the elites to create a society without war or racial and economic inequality. The other element was based around "liberation," or giving oppressed groups a voice to define themselves as they wished and escape the "technocracy."}}
        \cornell{How did the left wing of American politics come together?}{\begin{itemize}
            \item Baby boomers beginning to grow up w/ over half of 1970 U.S. population under 30 yrs. old; more than 8 million Americans attending college
            \item College/university students particularly radicalized, forming the \textbf{New Left} 
            \begin{itemize}
                \item Primarily consisted of whites, but fought for cause of Afr. Americans, other minorities
                \item Blacks/minorities formed their own pol. movements
                \item Drew from social criticism of 1950s, including \textbf{C. Wright Mills} of Columbia
                \item Few were communists, but inspired by Marxist writings and leaders (Che Guevara of South America, Mao Zedong, Ho Chi Minh); most of inspiration from \underline{civil rights movement}
            \end{itemize}
            \item 1962: students from prestigious universities met in Michigan, forming \textbf{Students for a Democratic Society} (SDS), with beliefs and disillusionment expressed in \textbf{Port Huron Statement}
            \begin{itemize}
                \item Some members moved into inner cities to mobilize poor
                \item Most were students $\to$ fought for university reform
            \end{itemize}
        \end{itemize}
        \textbf{The "New Left," consisting mainly of white college and university students, came together in the Students for a Democratic Society, advocating a wide range of liberal reforms.}}
        \cornell{How did the SDS fight for university reform?}{\begin{itemize}
            \item \textbf{Free Speech Movement} fought for students to engage in political activities on campus
            \item Particularly strong at UC Berkeley, facing campus police, admin. offices w/ student strike representing critique of university's society
            \begin{itemize}
                \item Other strikes emerged, all focusing on impersonality of modern uni.; antiwar movement further aggravated w/ strikes at Columbia, Harvard known for great violence w/ police intervention
                \item 1969: Berkeley students wanted to build "\textbf{People's Park}" instead of parking garage $\to$ violent conflict w/ admin.
                \begin{itemize}
                    \item More and more students began to support by end of weeklong battle, seeing as liberation v. oppression $\to$ won in referendum
                \end{itemize}
                \item Rarely violent, but reputation typically as chaotic and violent due to \textbf{Weathermen} offshoot
                \begin{itemize}
                    \item Known for arson/bombings on campus, taking lives; few supported their radical pol. views
                \end{itemize}
            \end{itemize}
            \item United in opposition to military actions
            \begin{itemize}
                \item In response to continued Vietnam War, drove out college training programs for mil. officers, attacked laboratories making war weapons, conducted several marches including one on Pentagon 
                \item Detested mil. conscription: after conscription \underline{graduate students, teachers, husbands, fathers} was abolished, far more draft-age Americans faced with draft
                \begin{itemize}
                    \item Many simply fled country to Canada, Sweden; joined war deserters
                    \item Remained in exile until Carter's 1977 pardon
                \end{itemize}
            \end{itemize}
        \end{itemize}
        \textbf{The Free Speech Movement fought for students to take on a more vocal role in political and social reform at universities, notably through strikes; UC Berkeley saw several major student strikes. Another issue on which the SDS was unified was a resentment of military intervention: they opposed the Vietnam War as well as the forced draft.}}
        \cornell{What defined "hippie" culture?}{\begin{itemize}
            \item Openly attacked middle-class society with long hair, flamboyant clothes, disdain for traditional vals.; marijuana became nearly as popular as beer, as well as some LSD
            \item Relaxed approach to sexuality partly due to increased birth control, abortion; tied into belief to abandon inhibitions
            \item Attacked modern society for artificiality, materialism 
            \item Hippies (many in SF), social dropouts sought "natural" existence
            \item Values extended to affect young ppl.: long hair, crass language, marijuana defined entire generation
            \item Rock 'n' roll of 1950s spread further in 1960s, notably due to \textbf{The Beatles} w/ style transforming to reflect values of time, intrigued over drugs/Eastern religion; Rolling Stones used themes of anger/frustration
        \end{itemize}
        \textbf{Hippies attacked middle-class society for its materialism, known for long hair, drugs, open sexuality, and rock 'n' roll music with groups like the Beatles and the Rolling Stones; their influence affected an entire generation.}}
        \cornell[The Mobilization of Minorities]{How did minority groups come together to fight for their rights?}{\textbf{Native Americans, a small yet extremely oppressed group, became increasingly active in fighting for their civil rights after the government's failed attempt at the termination of tribal rights. Their civil rights movement fought first for social acceptance and then for political rights; due largely to the Supreme Court, natives received many more tribal rights. Latinos, too, primarily Mexican Americans, fought for increased wages and bilingual acceptance. Gay Americans, too, following the 1969 Stonewall Riots, gained far more public acceptance with time; despite significant Republican opposition, gay marriage became legalized nationally in 2015.}}
        \cornell{How did Native Americans begin to more actively oppose the oppression they faced?}{\begin{itemize}
            \item Natives were least prosperous group in nation as well as one of smallest
            \begin{itemize}
                \item Avg. family income \$1000 less than blacks w/ unemployment rate 10x national level and urban dwellers unable to get education/find jobs
                \item Life expectancy rate low; suicide rate high
                \item Far less attention from white Americans than black Americans received
            \end{itemize}
            \item After John Collier resigned as commissioner of Indian Affairs, fed. policy sought assimilation
            \begin{itemize}
                \item 1953 "\textbf{termination}" laws made subject to same jursidictions as white residents, encouraging adaptation to white world
                \item Termination successful in promoting adaptation to urban life; failure for tribes w/ bitter resistance
                \item Eisenhower ended termination unles tribes consented but damage already done to natives
            \end{itemize}
            \item Natives reinvigorated 1944 \textbf{National Congress of American Indians} to address rapidly growing native pop.
        \end{itemize}
        \textbf{Native Americans, having faced centuries of oppression at the hands of white Americans and typically living in extremely poor conditions in the 1950s as well as facing the government policy of "termination" which removed tribes as a special federal entity, began to reinvigorate their civil rights movement with the National Congress of American Indians.}}
        \cornell{How did natives take action for their civil rights?}{\begin{itemize}
            \item 1961: natives gathered in Chicago to collectively address social issues
            \begin{itemize}
                \item Movement $\to$ films rarely portrayed natives as savages attacking whites; some white institusions removed demeaning references (like Dartmouth renaming team from "Indians")
                \item Most militant created \textbf{American Indian Movement} (AIM), heavily supported by urban Native Americans
            \end{itemize}
            \item 1968: Congress passed \textbf{Indian Civil Rights Act} to recognize tribal laws within reservations 
            \begin{itemize}
                \item Natives unsatisfied: fishermen demanded exclusive right to fish in Washington rivers; occupied SF's Alcatraz ironically "by right of discovery" 
            \end{itemize}
            \item 1969: Nixon appointed Mohawk-Sioux tribe member as commissioner of Indian Affairs; promised increased tribal federal aid 
            \begin{itemize}
                \item Protests continued: Nov. 1972 saw Bureau of Indian Affairs forcibly occupied by Sioux natives
            \end{itemize}
            \item 1973: members of AIM seized/occupied Wounded Knee, demanding changes in administration as well as govt. agreement to honor treaties; led to deadly battle w/ fed. forces
            \item 1978: \textit{United States v. Wheeler} ruled that tribes were legally independent untouchable by Congress 
            \begin{itemize}
                \item In other rulings, allowed tribes to tax businesses within the reservations as well as act in many ways as sovereign nation
                \item \underline{County of Oneida v. Oneida Indian Nation} gave nations 100k acres of land guaranteed by treaty 
            \end{itemize}
            \item Although movement unable to gain equality like many desired in part due to disunity of the tribes as a whole, successful in earning far more legal rights for tribes
        \end{itemize}
        \textbf{The natives initially pushed for greater social acceptance, but soon fought for legal rights beyond those offered by the Indian Civil Rights Act; they frequently protested government decisions, at times occupying government offices. The Supreme Court provided natives with some of their greatest legal successes, ruling that tribes were legally independent entities.}}
        \cornell{How did Latinos fight for their rights?}{\begin{itemize}
            \item Latinos far greater in number than natives: fastest-growing minority group w/ many having migrated after WWII but others from early Spanish colonization
            \item Some Latinos included Puerto Ricans in NYC, Cubans in South Florida fleeing Castro, Guatemalans, Nicaraguans, Peruvians, Salvadorans; \underline{largest group was Mexican Americans}
            \begin{itemize}
                \item Mex. Americans entered during WWII; continued to migrate illegally after agreements allowing immigration expired
                \item 1953: govt. launched \textbf{Operation Wetback} for deportation of illegal immigrants; unable to prevent new arrivals
                \item Cities like El Paso, Detroit, \underline{Los Angeles} saw major Mex. American neighborhoods
            \end{itemize}
            \item By late 1960s, Mex. Americans outnumbered Afr. Americans; among most urbanized groups in pop. w/ 90\% in cities 
            \begin{itemize}
                \item Many well-established communities very successful, electing members of Congress
                \item Newly arrived typically less well-educated $\to$ unable to attain high-paying jobs (typically worked in service)
                \item Due to language barriers, family-centered culture, discrimination $\to$ political influence less than other minorities, but developed ethnic ties as "\textbf{chicanos}" all speaking Spanish (like black power)
                \item La Raza Unida pushed for autonomous Mexican state
            \end{itemize}
            \item CA saw major movement for Mex. American organization under \textbf{César Chávez}: created \textbf{United Farm Workers} against growers, demanding $\uparrow$ wages
            \begin{itemize}
                \item Assistance from college students, churches, civil rights grps., eventually signing many contracts w/ CA grape growers
                \item Supported Robert Kennedy in election
            \end{itemize}
            \item Other major issue: bilingualism w/ supporters arguing for schooling in native language (confirmed by Supreme Court in 1974) to ensure no disadvantage against native English speakers; opponents referenced cost, difficulty of assimilation into U.S. culture
        \end{itemize}
        \textbf{Latinos, predominantly Mexican Americans, began to outnumber African Americans particularly in the West. Although many Mexican Americans had become established in the region with time, many new arrivals, due to language barriers, limited education, and discrimination struggled to gain political influence. César Chávez of California pushed for greater influence by unionizing farm workers, demanding wages. Bilingualism, too, saw supporters win the right to receive schooling in their native language.}}
        \cornell{How did the gay liberation movement gain traction?}{\begin{itemize}
            \item 1960s saw effort by gay Americans to win pol./economic/social equality in world where homosexuality/lesbianism had been unacknowledged (Walt Whitman/Horatio Alger revealed to be gay long after deaths)
            \item 1969: police officers raided \textbf{Stonewall Inn}, gay bar in NYC, arresting patrons; police oppression of gays very common but Stonewall represented culmination of accumulated resentment
            \begin{itemize}
                \item Police taunted and attacked by gays surrounding bar; impassioned riots
            \end{itemize}
            \item Stonewall Riots gave greater public support to gay rights movement; greater public acceptance through discussions
            \item Gay men/lesbians felt far more comfortable w/ coming out despite disastrous AIDS epidemic
            \item 1990s saw great feats: openly gay politicians won election, universities created gay/lesbian studies programs; laws banning discrim. based on sexual preference making gradual local progress
            \item Backlash after Bill Clinton sought to lift ban for gays/lesbians to serve in military $\to$ forced to settle for "don't ask, don't tell" compromise where mil. would no longer ask abt. sexual preference
            \item George W. Bush's campaign based heavily around banning same-sex marriage in constitutional amendment; many states added to ballots
            \item 2013: \textit{United States v. Windsor} saw declaration that restricting marriage to heterosexual couples was unconstitutional; Congress repealed Don't Ask, Don't Tell allowing gay ppl. to serve openly; by 2014, gay marriage authorized in 17 states and D.C.; in 2015 legalized throughout nation
        \end{itemize}
        \textbf{The Stonewall Riots after the police raided a gay bar marked a powerful public beginning to the gay liberation movement. With time, society more openly accepted homoseexuality, but progress remained slow; Bill Clinton's attempt to allow gay people to serve in the military was met with great opposition and George W. Bush based his campaign around banning same-sex marriage. In recent years, however, the Supreme Court declared banning same-sex marriage unconstitutional and allowed gay people to serve in the military.}}
        \cornell[The New Feminism]{How did the feminist movement grow significantly alongside other minority reform movements?}{\textbf{Starting with Betty Friedan's \textit{The Feminine Mystique}, the American feminist movement began to gain significant traction. JFK provided early legislative support to the early feminist movement. In the early 1970s, the feminist movement took a new direction, representing the far more radical desire to crush the male power structure. This new movement saw many great achievements, including in affirmative action, admission to major universities, workforce success, politics, athletics, and academia. Abortion represented a women's right to control over her sexuality; despite great opposition, \textit{Roe v. Wade} legalized it in the first trimester.}}
        \cornell{How did feminism rapidly grow in American society?}{\begin{itemize}
            \item 1963: Betty Friedan published \textbf{\textit{The Feminine Mystique}}, having travelled around nation to interview women who had graduated w/ her from college in 1942
            \begin{itemize}
                \item Most women lived in suburbs w/ affluent life
                \item Felt suburbs were "comfortable concentration camp," discontent w/ inability to use education/intelligence/talent
            \end{itemize}
            \item JFK had already established \textbf{President's Commission on the Status of Women} to reduce sexual discrimination
            \begin{itemize}
                \item 1963 \textbf{Equal Pay Act} banned practice of paying women less for same work as men
                \item Civil Rights Act of 1964 gave women many of the same rights intended for Afr. Americans
            \end{itemize}
            \item Early 1960s exposed dramatic contradiction betw. image of happy domesticity and reality of women's roles as members of workplace facing great discrimination
            \item 1966: \textbf{National Organization for Women} founded by Friedan w/ other women, became nation's largest fem. organization 
            \begin{itemize}
                \item Responded to complaints of Friedan's book by demanding educational oppos.
                \item Primarily directed to women already in workplace, describing econ. discrimination and seeking equal treatment 
            \end{itemize}
        \end{itemize}
        \textbf{Betty Friedan's text describing widespread discontent among American women living in suburbs marked a significant milestone in the feminist movement. Although women's rights movements enjoyed some political successes like JFK's Equal Pay Act and the Civil Rights Act of 1964 extending more rights to women, women continued to fight for their rights, forming the 1966 National Organization for Women to address broad issues of equality as well as economic discrimination.}}
        \cornell{How did the women's movement take new directions?}{\begin{itemize}
            \item Late 1960s saw more radical baby-boom feminists inspired by New Left, civil rights movement, antiwar crusade
            \item Early 1970s saw far harsher critique of U.S. society than Friedan's text
            \begin{itemize}
                \item 1969: Kate Millett's \textit{Sexual Politics} argued all power in male hands
                \item Pushed less for women reaching personal goals but instead coming together to crush male power structure
            \end{itemize}
            \item Increasing numbers of U.S. women came together to form unified culture
        \end{itemize}
        \textbf{In the early 1970s, the women's movement transformed from one of personal fulfillment to one where women would come together to destroy the male power structure. It was heavily influenced by the New Left.}}
        \cornell{How did the women's movement see far greater achievements?}{\begin{itemize}
            \item 1971: govt. extended affirmative action to women 
            \item Women pushed for major universities to admit women (starting w/ Princeton and Yale in 1969); some women's colleges accepted male students
            \item Success in workforce: nearly half of married women had jobs, 90\% with college degrees worked; two-career family far more accepted w/ women pausing marriage/motherhood for sake of careers
            \item Symbolic progress w/ Ms. rather than Mrs. vs Miss to indicate irrelevance of marital status
            \item Politics saw women take far more active roles w/ Speaker of House Nancy Pelosi in modern world; Reagan appointed female justice
            \item Academia saw women entering scholarly fields 
            \item Female professional athletics became far more accepted w. fed, govt. pushing colleges/universities to provide women w/ equal athletic programs to men
            \item 1972: Congress approved Equal Rights Amendment but not ratified by enough states 
        \end{itemize}
        \textbf{The 1970s saw most of the major universities begin to admit women; far more women entered the workforce as well as politics, far more women become academic scholars as well as professional athletes. However, the Equal Rights Amendment never became ratified despite being approved by Congress.}}
        \cornell{What was the significant controversy over abortion?}{\begin{itemize}
            \item Abortion represented female control over sexual lives: started out w/ public address of problems like rape, sexual abuse, even birth control 
            \item Abortion once legal but banned in early 20th century through to 1960s 
            \item 1973: \textit{Roe v. Wade} allowed unrestricted abortion during first trimester
        \end{itemize}
        \textbf{Abortion represented the feminist goal for women to gain control over their sexual lives. Although it had been banned in most states in the early 20th century, \textit{Roe v. Wade} allowed for unrestricted abortion during the first trimester.}}
        \cornell[Environmentalism in a Turbulent Society]{How did environmentalism grow as a powerful movement?}{\textbf{The environmental movement gained significant traction with the science of ecology, describing the interrelatedness of the natural world to indicate that pollution cannot be treated as an insular problem. Environmental advocacy movements grew by working with existing non-profits; the obvious environmental degradation with water and air pollution as well as the destruction of major forests lended further credence to the environmental movement. In several acts signifying that environmentalism had entered the consciousness of the American people, the government approved Earth Day as well as passed the Environmental Protection Act: nearly all politicians supported environmentalism due to its presentation as a centrist movement without the radical issues of movements like the antiwar movement.}}
        \cornell{How did ecology develop as an influential science?}{\begin{itemize}
            \item \textbf{Ecology} describes interrelatedness of natural world, harmony betw. individual groups and its environment 
            \begin{itemize}
                \item Stresses that issues like air pollution, extinction, must be connected to the function of the greater environment
            \end{itemize}
            \item Early 20th century saw expansion w/ \textbf{Aldo Leopold} created env. literature stressing that humans should understand natural balance; pioneered idea of food chain/biodiversity
            \item 1962: \textbf{Rachel Carson} created \textit{Silent Spring} showing how DDT had killed critical animals
            \begin{itemize}
                \item Led to banning of DDT in 1972
                \item Enraged chemical industry
            \end{itemize}
            \item Number of ecologists grew postwar w/ direct govt. support for ecological science in universities; particularly focused on publishing work for recognition of issues
        \end{itemize}
        \textbf{Ecology, focusing on the relationships between species and the environment, strongly pushed the environmental movement in America. Aldo Leopold discovered ecological ideas like the food chain and biodiversity; Rachel Carson's \textit{Silent Spring} led to the banning of DDT by showcasing how DDT had caused several critical species to die out. With time, ecology became a far more respected science.}}
        \cornell{How did an environmental advocacy movement emerge?}{\begin{itemize}
            \item 20th century groups like the Sierra Club, the National Wildlife Federation, and the National Parks and Conservation Association became reenergized in 21st century
            \begin{itemize}
                \item Partnered w/ non-profits previously not devoted to environmentalism like American Civil Liberties Union, League of Women Voters, AFL-CIO 
            \end{itemize}
            \item Professional group of environmental activists emerged w/ legal knowledge to make political change: scientists, lawyers, and lobbyists came together to make change
        \end{itemize}
        \textbf{Older environmental groups joined with non-profit organizations like the AFL-CIO and the League of Women Voters to make a much bigger impact in American society. The movement became far more professional, with scientists, lawyers, and lobbyists coming together to fight environmental abuse.}}
        \cornell{How did environmental degradation further motivate the environmental movement?}{\begin{itemize}
            \item 1960s: water pollution began to become extremely dire, particularly in rivers/lakes: Cleveland, OH saw river occasionally burst into flame due to gas within it
            \item Awareness that air was becoming increasingly unhealthy due to factories, power plants, and \underline{automobiles}: smog forecasts emerged in cities like LA and Denver $\to$ respiratory challenges
            \item Depletion of oil/fossil fuels, destruction of lakes and forests due to acid rain leading to the depletion of the ozone layer, \underline{global warming} $\to$ people began to take action as scientific credence emerged 
        \end{itemize}
        \textbf{Starting in the 1960s, people recognized the disastrous effects of water and air pollution near cities as well as the depletion of fossil fuels and forests combined with global warming; this gave further credence to the environmental movement.}}
        \cornell{How did Earth Day lead to environmental change?}{\begin{itemize}
            \item April 22, 1970: first \textbf{Earth Day}, as proposed by senator \textbf{Gaylord Nelson}; deliberately made to seem far less threatening than antiwar demonstrations $\to$ widely accepted for centrist character
            \item Environmentalism entered consciousness of all Americans and became endorsed by nearly all politicians
            \item 1970: Nixon signed \textbf{Environmental Protection Act} creating Environmental Protection Agency to enforce antipollution standards; \textbf{Clean Air Act} and \textbf{Clean Water Act} furthered govt. power to protect environment
        \end{itemize}
        \textbf{Earth Day marked a major government endorsement of the environmental movement; it was particularly accepted due to its centrist nature. In 1970, Nixon signed the Environmental Protection Act, marking that environmentalism had become a significant force in American society.}}
        \cornell[Nixon; Kissinger; the War]{How did Nixon's administration handle the Vietnam War?}{\textbf{Nixon immediately realized the difficulty of exiting the war; with his astute security adviser Henry Kissinger, he began the process of Vietnamization, training South Vietnamese troops for battle to allow for American retreat. However, the antiwar movement, despite momentarily weakening, saw a sudden spike after Nixon covertly invaded Cambodia and after the Pentagon Papers were leaked, revealing government dishonesty throughout the war. With the 1972 presidential election approaching, Nixon began negotiations with North Vietnam; a ceasefire was ultimately agreed upon after Nixon stopped the brutal "Christmas bombings" on North Vietnamese targets. However, soon after American retreat, North Vietnamese reunified the nation by taking over Saigon without U.S. resistance; Cambodia fell to the repressive Khmer Rouge.}}
        \cornell{How did Nixon and Kissinger push for Vietnamization?}{\begin{itemize}
            \item Nixon, his sec. of state, and sec. of defense overshadowed by national security adviser \textbf{Henry Kissinger}; represented Nixon's goal for concentrated power
            \item Sought to limit internal war opposition: created "lottery" system to replace Selective Service System - specifically targeted limited group of nineteen year olds w/ low lottery numbers
            \begin{itemize}
                \item Nixon proposed all-volunteer army
            \end{itemize}
            \item \textbf{Vietnamization} meant training South Vietnamese troops to take over after U.S. retreated
            \begin{itemize}
                \item Allowed Nixon to begin program of reducing troop numbers 
                \item Stalemate in domestic policy remained strong
            \end{itemize}
        \end{itemize}
        \textbf{Nixon, working with his national security adviser Henry Kissinger, sought to limit war opposition with a lottery system to determine the draft, further limiting the size of the group eligible for the draft. His program of Vietnamization sought to train the South Vietnamese troops to reach American levels, allowing American troops to slowly reduce in number.}}
        \cornell{How did Nixon's actions further escalate the antiwar movement?}{\begin{itemize}
            \item Nixon believed that Vietnamese war bases in neutral Cambodia were greatly assisting Viet Cong $\to$ secretly ordered bombing of Cambodian bases, then secretly encouraging takeover of Cambodian govt. 
            \begin{itemize}
                \item Neutral Cambodian leader replaced w/ U.S. sympathizer \textbf{Lon Nol}, who approved American incursion in territory 
            \end{itemize}
            \item Nixon finally announced plan to invade Cambodia $\to$ antiwar movement reborn w/ hundreds of thousands protesting in Washington DC in spring 1970
            \begin{itemize}
                \item Kent State University saw extremely violent protests w/ police intervention
                \item University of MS saw two black students killed for protesting
            \end{itemize}
            \item Congress retaliated by repealing Tonkin Resolution but Nixon ignored
            \item \textit{New York Times} published excerpts known as \textbf{Pentagon Papers} revealing that govt. had been dishonest in reporting war progress, motives
            \item U.S. troops saw extremely $\downarrow$ morale w/ desertion, drug addiction, race issues, murder; Lieutenant William Calley oversaw \textbf{My Lai massacre} of 300 South Vietnamese civilians
            \item 1971: 2/3 of Americans sought withdraw, feeling Nixon's policies too timid
            \item March 1972: North Vietnamese began \textbf{Easter offensive} - barely stopped by U.S./South Vietnamese forces $\to$ clearly still needed U.S. help
            \begin{itemize}
                \item Nixon ramped up bombings of Hanoi, major harbors, mines
            \end{itemize}
        \end{itemize}
        \textbf{Nixon's covert overthrow of Cambodia to target North Vietnamese war bases greatly angered the American people, reaggravating the antiwar movement. Nixon ignored government retaliation as well as the Pentagon Papers, which exposed the government for dishonesty during the war. The long stalemate led to a reduction in morale for U.S. troops, culminating in the inexplicable My Lai massacre of South Vietnamese citizens. Battles continued, however, with the disastrous Easter offensive barely stopped by American troops.}}
        \cornell{How did Nixon strive to honorably reach an agreement of peace?}{\begin{itemize}
            \item 1972 pres. election neared $\to$ admin. sought to negotiate w/ North Vietnamese
            \begin{itemize}
                \item Nixon no longer insisted in removal of all North Vietnamese troops before U.S. withdrawal
                \item Kissinger worked with North Vietnamese foreign sec. \textbf{Le Duc Tho} to determine ceasefire 
            \end{itemize}
            \item Oct. 1972: plans for ceasefire ready to put into action, but South Vietnamese general Thieu still required removal of all N. Vietnamese troops from South
            \item U.S. ramped up bombings of major N. Vietnamese targets w/ high civilian casualties but 15 major U.S. bombers shot down $\to$ Nixon terminated bombings, resuming negotiations
            \item Jan. 1973: agreement signed to end the war, restore peace in Vietnam w/ Nixon attributing success to recent bombings 
            \begin{itemize}
                \item Paris accords resembled Kissinger-Tho agreement w/ ceasefire, release of U.S. POWs, allowance for Thieu to remain in power of South
            \end{itemize}
        \end{itemize}
        \textbf{Negotiations began as the 1972 presidential election neared: Kissinger and North Vietnamese foreign secretary Tho formed a ceasefire agreement, but General Thieu of the South refused to relent. The U.S. continued to bomb major North Vietnamese targets, but a formal agreement was formed due to major losses on both sides in January 1973: peace was restored with a ceasefire initiated and American prisoners of war released.}}
        \cornell{How did the South Vietnamese ultimately lose after the American retreat?}{\begin{itemize}
            \item After U.S. retreat, Paris accords immediately collapsed w/ battles continuing, seeing unprecedented losse 
            \item 1975: North Vietnamese launched major attack $\to$ Thieu appealed to President Ford for funding $\to$ Congress refused $\to$ communist forces reached Saigon, w/ Thieu fleeing
            \begin{itemize}
                \item Process of reunification began, renaming Saigon \textbf{Ho Chi Minh City}
                \item Lon Nol's Cambodian regime fell to \textbf{Khmer Rouge}, known for genocidal policies for over a decade 
            \end{itemize}
            \item Failure of war w/ extreme loss of lives and money $\to$ crushed U.S. confidence 
        \end{itemize}
        \textbf{After U.S. troops retreated, North Vietnamese troops rapidly invaded Saigon: Ford, unable to receive Congressional approval, was powerless to help. Vietnam was reunified and Cambodia was taken over by the oppressive Khmer Rouge. The failed war ultimately crushed American confidence.}}
        \cornell[Nixon; Kissinger; The World]{How did Nixon and Kissinger address foreign policy outside of Vietnam?}{\textbf{Nixon and Kissinger sought to address the changing, multipolar world by forging a stronger relatinoship with the previously isolated China as well as pursue a policy of denuclearization with the USSR. In the Third World, Nixon sought to pursue a mild policy of assisting in the defense of other nations without becoming too entangled in local affairs.}}
        \cornell{What defined Nixon's views on multipolarity?}{\textbf{Nixon believed the bipolar world with solely America and Russia as superpowers had ended; he sought to embrace the safer "multipolar" world including China, Japan, and Western Europe.}}
        \cornell{What defined the U.S. relationship with China and the USSR under Nixon?}{\begin{itemize}
            \item Since fall of Chiang Kai-shek, U.S. had refused to recognize communist govt.: recognized Taiwanese govt. as true Chinese leadership
            \begin{itemize}
                \item Nixon sought to create new relationship w/ Chinese communists as counterbalance to Soviet Union 
                \item China simultaneously hoped to end their isolation as well as prevent risky Soviet-American alliance
            \end{itemize}
            \item 1971: Kissinger went on secret mission to Beijing; president responded by announcing public visit to China; UN, w/ American agreement, formally recognized communist Chinese govt. instead of Taiwan 
            \item Simultaneous goal to work w/ Soviet Union: meeting in Helsinki to limit nuclear weapon, w/ \textbf{Strategic Arms Limitation Treaty} (SALT I) to freeze nuclear weapons at current levels
        \end{itemize}
        \textbf{Although the U.S. had refused to recognize the Chinese government for nearly 20 years, Nixon finally chose to forge a relationship with China, sending Kissinger in 1971 and soon going on a foreign tour of China. Furthermore, Nixon began the SALT I denuclearization agreement with the Soviet Union.}}
        \cornell{What defined the Nixon Doctrine?}{\begin{itemize}
            \item Nixon sought to maintain stable relationship w/ perceived unstable "Third World" without involving in local disputes
            \begin{itemize}
                \item \textbf{Nixon Doctrine} agreed that U.S. would partake in defense/development of allies while leaving responsibility for future development in hands of nations
            \end{itemize}
            \item Middle East: Israel took over city of Jerusalem against Egyptian/Syrian/Jordanian forces in \textbf{Six-Day War} 
            \begin{itemize}
                \item Several refugee Palestinians lost homes $\to$ created instability in neighboring countries 
                \item Oct. 1973: Israel attacked by surprise on Yom Kippur; Israel pushed Egyptian forces back $\to$ U.S. forces intervened, pushing Israel to cease further intervention
            \end{itemize}
            \item Settlement of Yom Kippur War revealed U.S. dependence on Arab oil: U.S. needed to stop Israel to ensure stable oil supply
        \end{itemize}
        \textbf{The Nixon Doctrine reflected Nixon's goal to assist Third World nations without taking direct action in the ruling of their government. Disaster struck in the Middle East when Israel took over Jerusalem from Arab forces, driving out Palestinian forces; the following Yom Kippur War saw Egypt push back into Israeli territory, but Israel soon mounted a counteroffense. The U.S. intervened to protect its oil interests, preventing Israel from pushing any further.}}
       \end{document}