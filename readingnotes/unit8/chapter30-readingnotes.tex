\documentclass[a4paper]{article}
    \input{../notesheader.tex}
    \usepackage[normalem]{ulem}

    \newcommand{\unitnumber}{8}
    \newcommand{\chapternumber}{30}
    \newcommand{\chaptertitle}{The Crisis of Authority}

    \title{\vspace{-3em}
    \begin{tcolorbox}
    \Huge\sffamily \begin{center} AP US History  \\
    \LARGE Chapter \chapternumber \, - \chaptertitle \\
    \Large Finn Frankis \end{center} 
    \end{tcolorbox}
    \vspace{-3em}
    }
    \date{}
    \author{}
    
    \begin{document}
        \maketitle
        \SetBgContents{\rule[0em]{4pt}{\textheight}}
        \cornell[Key Concepts]{What are this chapter's key concepts?}{\begin{itemize}
            \item \textbf{8.1.II.D} - Mil./econ./ideological issues defined U.S. involvement in Middle East: oil crises $\to$ national energy policy in U.S.
            \item \textbf{8.2.II.A} - Feminist/gay/lesbian activists sought legal/econ./social equality
            \item \textbf{8.2.II.B} - Latino, natives, Asian Americans sought social/economic equality, apology for past injustices
            \item \textbf{8.2.II.D} - Environmental issues $\to$ env. movement using legislation/public efforts to combat pollution, protect resources
            \item \textbf{8.2.III.B} - Johnson's Great Society used legislation to end poverty/racism; Supreme Court further advanced civil rights
            \item \textbf{8.2.III.C} - 1960s saw conservatives challenge liberal laws, court decisions as well as moral/cultural decline; sought to limit fed. govt. power
            \item \textbf{8.2.III.D} - Left groups rejected liberal policies: felt leaders not active enough in changing racial/economic status at home 
            \item \textbf{8.2.III.F} - 1970s saw clashes betw. conservatives/liberals over social/cultural issues, fed. govt. power, race, individual rights
            \item \textbf{8.3.II.B} - Feminists, young ppl. of 1960s counterculture rejected values of parents' generation, seeking informality, changes in sexual norms w/in U.S. culture
            \item \textbf{9.2.I.C} - $\uparrow$ employment in service sectors; $\downarrow$ in manufacturing, unions
        \end{itemize}}
        \cornell[The Youth Culture]{What were the key elements of 1960s youth counterculture?}{\textbf{The 1960s was characterized by a "New Left" dominated by radicalized college students fighting for their right to protest at universities as well as directly opposing the war and military draft. Hippie culture, too, became prominent, representing a direct assault on middle-class values of materialism; hippies supported sexuality, drugs, long hair, and rock 'n' roll.}}
        \cornell{What were the broad visions of social and cultural protest?}{\textbf{One element was based around a community of "the people," or taking power from the elites to create a society without war or racial and economic inequality. The other element was based around "liberation," or giving oppressed groups a voice to define themselves as they wished and escape the "technocracy."}}
        \cornell{How did the left wing of American politics come together?}{\begin{itemize}
            \item Baby boomers beginning to grow up w/ over half of 1970 U.S. population under 30 yrs. old; more than 8 million Americans attending college
            \item College/university students particularly radicalized, forming the \textbf{New Left} 
            \begin{itemize}
                \item Primarily consisted of whites, but fought for cause of Afr. Americans, other minorities
                \item Blacks/minorities formed their own pol. movements
                \item Drew from social criticism of 1950s, including \textbf{C. Wright Mills} of Columbia
                \item Few were communists, but inspired by Marxist writings and leaders (Che Guevara of South America, Mao Zedong, Ho Chi Minh); most of inspiration from \underline{civil rights movement}
            \end{itemize}
            \item 1962: students from prestigious universities met in Michigan, forming \textbf{Students for a Democratic Society} (SDS), with beliefs and disillusionment expressed in \textbf{Port Huron Statement}
            \begin{itemize}
                \item Some members moved into inner cities to mobilize poor
                \item Most were students $\to$ fought for university reform
            \end{itemize}
        \end{itemize}
        \textbf{The "New Left," consisting mainly of white college and university students, came together in the Students for a Democratic Society, advocating a wide range of liberal reforms.}}
        \cornell{How did the SDS fight for university reform?}{\begin{itemize}
            \item \textbf{Free Speech Movement} fought for students to engage in political activities on campus
            \item Particularly strong at UC Berkeley, facing campus police, admin. offices w/ student strike representing critique of university's society
            \begin{itemize}
                \item Other strikes emerged, all focusing on impersonality of modern uni.; antiwar movement further aggravated w/ strikes at Columbia, Harvard known for great violence w/ police intervention
                \item 1969: Berkeley students wanted to build "\textbf{People's Park}" instead of parking garage $\to$ violent conflict w/ admin.
                \begin{itemize}
                    \item More and more students began to support by end of weeklong battle, seeing as liberation v. oppression $\to$ won in referendum
                \end{itemize}
                \item Rarely violent, but reputation typically as chaotic and violent due to \textbf{Weathermen} offshoot
                \begin{itemize}
                    \item Known for arson/bombings on campus, taking lives; few supported their radical pol. views
                \end{itemize}
            \end{itemize}
            \item United in opposition to military actions
            \begin{itemize}
                \item In response to continued Vietnam War, drove out college training programs for mil. officers, attacked laboratories making war weapons, conducted several marches including one on Pentagon 
                \item Detested mil. conscription: after conscription \underline{graduate students, teachers, husbands, fathers} was abolished, far more draft-age Americans faced with draft
                \begin{itemize}
                    \item Many simply fled country to Canada, Sweden; joined war deserters
                    \item Remained in exile until Carter's 1977 pardon
                \end{itemize}
            \end{itemize}
        \end{itemize}
        \textbf{The Free Speech Movement fought for students to take on a more vocal role in political and social reform at universities, notably through strikes; UC Berkeley saw several major student strikes. Another issue on which the SDS was unified was a resentment of military intervention: they opposed the Vietnam War as well as the forced draft.}}
        \cornell{What defined "hippie" culture?}{\begin{itemize}
            \item Openly attacked middle-class society with long hair, flamboyant clothes, disdain for traditional vals.; marijuana became nearly as popular as beer, as well as some LSD
            \item Relaxed approach to sexuality partly due to increased birth control, abortion; tied into belief to abandon inhibitions
            \item Attacked modern society for artificiality, materialism 
            \item Hippies (many in SF), social dropouts sought "natural" existence
            \item Values extended to affect young ppl.: long hair, crass language, marijuana defined entire generation
            \item Rock 'n' roll of 1950s spread further in 1960s, notably due to \textbf{The Beatles} w/ style transforming to reflect values of time, intrigued over drugs/Eastern religion; Rolling Stones used themes of anger/frustration
        \end{itemize}
        \textbf{Hippies attacked middle-class society for its materialism, known for long hair, drugs, open sexuality, and rock 'n' roll music with groups like the Beatles and the Rolling Stones; their influence affected an entire generation.}}
    \end{document}