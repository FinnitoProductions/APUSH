\documentclass[a4paper]{article}
    \input{../notesheader.tex}
    \usepackage[normalem]{ulem}

    \newcommand{\unitnumber}{8}
    \newcommand{\chapternumber}{30}
    \newcommand{\chaptertitle}{The Crisis of Authority}

    \title{\vspace{-3em}
    \begin{tcolorbox}
    \Huge\sffamily \begin{center} AP US History  \\
    \LARGE Chapter \chapternumber \, - \chaptertitle \\
    \Large Finn Frankis \end{center} 
    \end{tcolorbox}
    \vspace{-3em}
    }
    \date{}
    \author{}
    
    \begin{document}
        \maketitle
        \SetBgContents{\rule[0em]{4pt}{\textheight}}
        \cornell[Key Concepts]{What are this chapter's key concepts?}{\begin{itemize}
            \item \textbf{8.1.II.D} - Mil./econ./ideological issues defined U.S. involvement in Middle East: oil crises $\to$ national energy policy in U.S.
            \item \textbf{8.2.II.A} - Feminist/gay/lesbian activists sought legal/econ./social equality
            \item \textbf{8.2.II.B} - Latino, natives, Asian Americans sought social/economic equality, apology for past injustices
            \item \textbf{8.2.II.D} - Environmental issues $\to$ env. movement using legislation/public efforts to combat pollution, protect resources
            \item \textbf{8.2.III.B} - Johnson's Great Society used legislation to end poverty/racism; Supreme Court further advanced civil rights
            \item \textbf{8.2.III.C} - 1960s saw conservatives challenge liberal laws, court decisions as well as moral/cultural decline; sought to limit fed. govt. power
            \item \textbf{8.2.III.D} - Left groups rejected liberal policies: felt leaders not active enough in changing racial/economic status at home 
            \item \textbf{8.2.III.F} - 1970s saw clashes betw. conservatives/liberals over social/cultural issues, fed. govt. power, race, individual rights
            \item \textbf{8.3.II.B} - Feminists, young ppl. of 1960s counterculture rejected values of parents' generation, seeking informality, changes in sexual norms w/in U.S. culture
            \item \textbf{9.2.I.C} - $\uparrow$ employment in service sectors; $\downarrow$ in manufacturing, unions
        \end{itemize}}
        \cornell[The Youth Culture]{What were the key elements of 1960s youth counterculture?}{\textbf{The 1960s was characterized by a "New Left" dominated by radicalized college students fighting for their right to protest at universities as well as directly opposing the war and military draft. Hippie culture, too, became prominent, representing a direct assault on middle-class values of materialism; hippies supported sexuality, drugs, long hair, and rock 'n' roll.}}
        \cornell{What were the broad visions of social and cultural protest?}{\textbf{One element was based around a community of "the people," or taking power from the elites to create a society without war or racial and economic inequality. The other element was based around "liberation," or giving oppressed groups a voice to define themselves as they wished and escape the "technocracy."}}
        \cornell{How did the left wing of American politics come together?}{\begin{itemize}
            \item Baby boomers beginning to grow up w/ over half of 1970 U.S. population under 30 yrs. old; more than 8 million Americans attending college
            \item College/university students particularly radicalized, forming the \textbf{New Left} 
            \begin{itemize}
                \item Primarily consisted of whites, but fought for cause of Afr. Americans, other minorities
                \item Blacks/minorities formed their own pol. movements
                \item Drew from social criticism of 1950s, including \textbf{C. Wright Mills} of Columbia
                \item Few were communists, but inspired by Marxist writings and leaders (Che Guevara of South America, Mao Zedong, Ho Chi Minh); most of inspiration from \underline{civil rights movement}
            \end{itemize}
            \item 1962: students from prestigious universities met in Michigan, forming \textbf{Students for a Democratic Society} (SDS), with beliefs and disillusionment expressed in \textbf{Port Huron Statement}
            \begin{itemize}
                \item Some members moved into inner cities to mobilize poor
                \item Most were students $\to$ fought for university reform
            \end{itemize}
        \end{itemize}
        \textbf{The "New Left," consisting mainly of white college and university students, came together in the Students for a Democratic Society, advocating a wide range of liberal reforms.}}
        \cornell{How did the SDS fight for university reform?}{\begin{itemize}
            \item \textbf{Free Speech Movement} fought for students to engage in political activities on campus
            \item Particularly strong at UC Berkeley, facing campus police, admin. offices w/ student strike representing critique of university's society
            \begin{itemize}
                \item Other strikes emerged, all focusing on impersonality of modern uni.; antiwar movement further aggravated w/ strikes at Columbia, Harvard known for great violence w/ police intervention
                \item 1969: Berkeley students wanted to build "\textbf{People's Park}" instead of parking garage $\to$ violent conflict w/ admin.
                \begin{itemize}
                    \item More and more students began to support by end of weeklong battle, seeing as liberation v. oppression $\to$ won in referendum
                \end{itemize}
                \item Rarely violent, but reputation typically as chaotic and violent due to \textbf{Weathermen} offshoot
                \begin{itemize}
                    \item Known for arson/bombings on campus, taking lives; few supported their radical pol. views
                \end{itemize}
            \end{itemize}
            \item United in opposition to military actions
            \begin{itemize}
                \item In response to continued Vietnam War, drove out college training programs for mil. officers, attacked laboratories making war weapons, conducted several marches including one on Pentagon 
                \item Detested mil. conscription: after conscription \underline{graduate students, teachers, husbands, fathers} was abolished, far more draft-age Americans faced with draft
                \begin{itemize}
                    \item Many simply fled country to Canada, Sweden; joined war deserters
                    \item Remained in exile until Carter's 1977 pardon
                \end{itemize}
            \end{itemize}
        \end{itemize}
        \textbf{The Free Speech Movement fought for students to take on a more vocal role in political and social reform at universities, notably through strikes; UC Berkeley saw several major student strikes. Another issue on which the SDS was unified was a resentment of military intervention: they opposed the Vietnam War as well as the forced draft.}}
        \cornell{What defined "hippie" culture?}{\begin{itemize}
            \item Openly attacked middle-class society with long hair, flamboyant clothes, disdain for traditional vals.; marijuana became nearly as popular as beer, as well as some LSD
            \item Relaxed approach to sexuality partly due to increased birth control, abortion; tied into belief to abandon inhibitions
            \item Attacked modern society for artificiality, materialism 
            \item Hippies (many in SF), social dropouts sought "natural" existence
            \item Values extended to affect young ppl.: long hair, crass language, marijuana defined entire generation
            \item Rock 'n' roll of 1950s spread further in 1960s, notably due to \textbf{The Beatles} w/ style transforming to reflect values of time, intrigued over drugs/Eastern religion; Rolling Stones used themes of anger/frustration
        \end{itemize}
        \textbf{Hippies attacked middle-class society for its materialism, known for long hair, drugs, open sexuality, and rock 'n' roll music with groups like the Beatles and the Rolling Stones; their influence affected an entire generation.}}
        \cornell[The Mobilization of Minorities]{How did minority groups come together to fight for their rights?}{\textbf{Native Americans, a small yet extremely oppressed group, became increasingly active in fighting for their civil rights after the government's failed attempt at the termination of tribal rights. Their civil rights movement fought first for social acceptance and then for political rights; due largely to the Supreme Court, natives received many more tribal rights. Latinos, too, primarily Mexican Americans, fought for increased wages and bilingual acceptance. Gay Americans, too, following the 1969 Stonewall Riots, gained far more public acceptance with time; despite significant Republican opposition, gay marriage became legalized nationally in 2015.}}
        \cornell{How did Native Americans begin to more actively oppose the oppression they faced?}{\begin{itemize}
            \item Natives were least prosperous group in nation as well as one of smallest
            \begin{itemize}
                \item Avg. family income \$1000 less than blacks w/ unemployment rate 10x national level and urban dwellers unable to get education/find jobs
                \item Life expectancy rate low; suicide rate high
                \item Far less attention from white Americans than black Americans received
            \end{itemize}
            \item After John Collier resigned as commissioner of Indian Affairs, fed. policy sought assimilation
            \begin{itemize}
                \item 1953 "\textbf{termination}" laws made subject to same jursidictions as white residents, encouraging adaptation to white world
                \item Termination successful in promoting adaptation to urban life; failure for tribes w/ bitter resistance
                \item Eisenhower ended termination unles tribes consented but damage already done to natives
            \end{itemize}
            \item Natives reinvigorated 1944 \textbf{National Congress of American Indians} to address rapidly growing native pop.
        \end{itemize}
        \textbf{Native Americans, having faced centuries of oppression at the hands of white Americans and typically living in extremely poor conditions in the 1950s as well as facing the government policy of "termination" which removed tribes as a special federal entity, began to reinvigorate their civil rights movement with the National Congress of American Indians.}}
        \cornell{How did natives take action for their civil rights?}{\begin{itemize}
            \item 1961: natives gathered in Chicago to collectively address social issues
            \begin{itemize}
                \item Movement $\to$ films rarely portrayed natives as savages attacking whites; some white institusions removed demeaning references (like Dartmouth renaming team from "Indians")
                \item Most militant created \textbf{American Indian Movement} (AIM), heavily supported by urban Native Americans
            \end{itemize}
            \item 1968: Congress passed \textbf{Indian Civil Rights Act} to recognize tribal laws within reservations 
            \begin{itemize}
                \item Natives unsatisfied: fishermen demanded exclusive right to fish in Washington rivers; occupied SF's Alcatraz ironically "by right of discovery" 
            \end{itemize}
            \item 1969: Nixon appointed Mohawk-Sioux tribe member as commissioner of Indian Affairs; promised increased tribal federal aid 
            \begin{itemize}
                \item Protests continued: Nov. 1972 saw Bureau of Indian Affairs forcibly occupied by Sioux natives
            \end{itemize}
            \item 1973: members of AIM seized/occupied Wounded Knee, demanding changes in administration as well as govt. agreement to honor treaties; led to deadly battle w/ fed. forces
            \item 1978: \textit{United States v. Wheeler} ruled that tribes were legally independent untouchable by Congress 
            \begin{itemize}
                \item In other rulings, allowed tribes to tax businesses within the reservations as well as act in many ways as sovereign nation
                \item \underline{County of Oneida v. Oneida Indian Nation} gave nations 100k acres of land guaranteed by treaty 
            \end{itemize}
            \item Although movement unable to gain equality like many desired in part due to disunity of the tribes as a whole, successful in earning far more legal rights for tribes
        \end{itemize}
        \textbf{The natives initially pushed for greater social acceptance, but soon fought for legal rights beyond those offered by the Indian Civil Rights Act; they frequently protested government decisions, at times occupying government offices. The Supreme Court provided natives with some of their greatest legal successes, ruling that tribes were legally independent entities.}}
        \cornell{How did Latinos fight for their rights?}{\begin{itemize}
            \item Latinos far greater in number than natives: fastest-growing minority group w/ many having migrated after WWII but others from early Spanish colonization
            \item Some Latinos included Puerto Ricans in NYC, Cubans in South Florida fleeing Castro, Guatemalans, Nicaraguans, Peruvians, Salvadorans; \underline{largest group was Mexican Americans}
            \begin{itemize}
                \item Mex. Americans entered during WWII; continued to migrate illegally after agreements allowing immigration expired
                \item 1953: govt. launched \textbf{Operation Wetback} for deportation of illegal immigrants; unable to prevent new arrivals
                \item Cities like El Paso, Detroit, \underline{Los Angeles} saw major Mex. American neighborhoods
            \end{itemize}
            \item By late 1960s, Mex. Americans outnumbered Afr. Americans; among most urbanized groups in pop. w/ 90\% in cities 
            \begin{itemize}
                \item Many well-established communities very successful, electing members of Congress
                \item Newly arrived typically less well-educated $\to$ unable to attain high-paying jobs (typically worked in service)
                \item Due to language barriers, family-centered culture, discrimination $\to$ political influence less than other minorities, but developed ethnic ties as "\textbf{chicanos}" all speaking Spanish (like black power)
                \item La Raza Unida pushed for autonomous Mexican state
            \end{itemize}
            \item CA saw major movement for Mex. American organization under \textbf{César Chávez}: created \textbf{United Farm Workers} against growers, demanding $\uparrow$ wages
            \begin{itemize}
                \item Assistance from college students, churches, civil rights grps., eventually signing many contracts w/ CA grape growers
                \item Supported Robert Kennedy in election
            \end{itemize}
            \item Other major issue: bilingualism w/ supporters arguing for schooling in native language (confirmed by Supreme Court in 1974) to ensure no disadvantage against native English speakers; opponents referenced cost, difficulty of assimilation into U.S. culture
        \end{itemize}
        \textbf{Latinos, predominantly Mexican Americans, began to outnumber African Americans particularly in the West. Although many Mexican Americans had become established in the region with time, many new arrivals, due to language barriers, limited education, and discrimination struggled to gain political influence. César Chávez of California pushed for greater influence by unionizing farm workers, demanding wages. Bilingualism, too, saw supporters win the right to receive schooling in their native language.}}
        \cornell{How did the gay liberation movement gain traction?}{\begin{itemize}
            \item 1960s saw effort by gay Americans to win pol./economic/social equality in world where homosexuality/lesbianism had been unacknowledged (Walt Whitman/Horatio Alger revealed to be gay long after deaths)
            \item 1969: police officers raided \textbf{Stonewall Inn}, gay bar in NYC, arresting patrons; police oppression of gays very common but Stonewall represented culmination of accumulated resentment
            \begin{itemize}
                \item Police taunted and attacked by gays surrounding bar; impassioned riots
            \end{itemize}
            \item Stonewall Riots gave greater public support to gay rights movement; greater public acceptance through discussions
            \item Gay men/lesbians felt far more comfortable w/ coming out despite disastrous AIDS epidemic
            \item 1990s saw great feats: openly gay politicians won election, universities created gay/lesbian studies programs; laws banning discrim. based on sexual preference making gradual local progress
            \item Backlash after Bill Clinton sought to lift ban for gays/lesbians to serve in military $\to$ forced to settle for "don't ask, don't tell" compromise where mil. would no longer ask abt. sexual preference
            \item George W. Bush's campaign based heavily around banning same-sex marriage in constitutional amendment; many states added to ballots
            \item 2013: \textit{United States v. Windsor} saw declaration that restricting marriage to heterosexual couples was unconstitutional; Congress repealed Don't Ask, Don't Tell allowing gay ppl. to serve openly; by 2014, gay marriage authorized in 17 states and D.C.; in 2015 legalized throughout nation
        \end{itemize}
        \textbf{The Stonewall Riots after the police raided a gay bar marked a powerful public beginning to the gay liberation movement. With time, society more openly accepted homoseexuality, but progress remained slow; Bill Clinton's attempt to allow gay people to serve in the military was met with great opposition and George W. Bush based his campaign around banning same-sex marriage. In recent years, however, the Supreme Court declared banning same-sex marriage unconstitutional and allowed gay people to serve in the military.}}
    \end{document}